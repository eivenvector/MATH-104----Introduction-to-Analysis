Finally we are ready to begin our study of differentiation.  We could have started this awhile ago, but we really needed certain facts about continuity, uniform convergence, and even integration, before we could address all the facts we care about related to differentiation.  This thus led us to a relatively broad study of the most general spaces\footnote{Meh, basically anyways} in which these notions makes sense.  For the time being, however, we return to $\R ^d$.

\section{Tensors and abstract index notation}

\begin{displayquote}
Throughout this chapter, all vector spaces will be finite-dimensional and \emph{real}.  For the remainder of this \emph{section}, $V$ and $W$ will always denote such vector spaces.  We omit proofs in this section under the assumption that you have either already seen the results or can prove them on your own if you so desired.
\end{displayquote}

We plan to do differentiation in $\R ^d$, and to do this (instead of just in $\R$), it will be useful to know some basic facts about linear algebra.  The real motivation for taking this small side route is that we want to use Penrose's \emph{abstract index notation}.\footnote{This is conceptually different, but mechanically very similar to Einstein's index notation.  You might say that abstract index notation is choice-free Einstein index notation (the choice of course being a choice of basis).}

\begin{displayquote}
If you don't understand the details of this section your first time through, that's fine.  Learn what you can, and then come back as you need to when the concepts come up in the actual differentiation part of the chapter.
\end{displayquote}

We first discuss the \emph{dual space} and \emph{tensor product}.
\begin{dfn}[Dual space]\label{DualSpace}
The \emph{dual space}\index{Dual space} of $V$, $V^{\dagger}$, is defined to be
\begin{equation}
V^{\dagger}\coloneqq \Mor _{\Vect _{\R}}(V,\R ).
\end{equation}
$V^{\dagger}$ has the structure of a vector space by defining addition and scalar multiplication pointwise.  The elements of $V^{\dagger}$ are \emph{linear functionals}\index{Linear functionals} or \emph{covectors}\index{Covector}.
\begin{rmk}
Though we have not precisely defined in yet, the category $\Vect _F$, for $F$ a field, is the category whose objects are vector fields over $F$ and whose morphisms are linear transformations.  Thus, $\Mor _{\Vect _{\R}}(V,\R )$ is our fancy-schmancy notation for the vector space of linear functions from $V$ into $\R$.
\end{rmk}
\begin{rmk}
In other words, the elements of $V^{\dagger}$ take in elements of $V$ and spit out numbers.  This is actually not that foreign of a concept---for example, the derivative takes in a vector (the direction in which to differentiate) and spits out a number (the directional derivative in that direction).
\end{rmk}
\begin{rmk}
We reserve the notation $V^*$ for the \emph{conjugate-dual} (something we won't see in these notes)---this is why we write $V^{\dagger}$ instead of $V^*$.
\end{rmk}
\begin{rmk}
If $V$ comes with a topology, you're only going to want to look at the \emph{continuous} linear functionals.  Of course, you can look at all of them (including the discontinuous ones), but in comparison this space will be an ugly beast of a motherfucker.\footnote{Can you tell I write not much differently from how I speak?  ;-)}
\end{rmk}
\end{dfn}
Of critical importance is that the dual of the dual is the original vector space.\footnote{Careful:  It will frequently be the case that $V^{\dagger}$ \emph{is} isomorphic to $V$, but in a noncanonical way.  On the other hand $(V^{\dagger})^{\dagger}$ and $V$ are \emph{canonically isomorphic}.  The way to make this intuition precise requires more category theory than is probably helpful.  Suffice it to say, the idea is to show that the `constructions' (read ``functors'') are isomorphic, not the objects themselves.}
\begin{prp}\label{prp5.1.4}
The map $V\ni v\mapsto \phi _v\in (V^\dagger )^\dagger$, where $\phi _v:V\rightarrow \R$ is defined by
\begin{equation}
\phi _v(\omega )\coloneqq \omega (v)
\end{equation}
is an isomorphism.
\begin{rmk}
Warning:  This is \emph{false} in infinite dimensions.  You will see that we show that this map is injective and linear, and so by the Rank-Nullity Theorem (something specific to finite-dimensions) is an isomorphism.
\end{rmk}
\end{prp}
\begin{dfn}[Tensor product]\label{TensorProduct}
Let $V$ and $W$ be finite-dimensional real vector spaces.  Then, the \emph{tensor product}\index{Tensor product (of vector spaces)} of $V$ and $W$, $V\otimes W$\index[notation]{$V\otimes W$} is defined to be the set of all functions $T:V^{\dagger}\times W^{\dagger}\rightarrow \R$ such that
\begin{enumerate}
\item for each fixed $\omega \in V^{\dagger}$, the map $\eta \mapsto T(\omega ,\eta )$ is linear; and
\item for each fixed $\eta \in W^{\dagger}$, the map $\omega \mapsto T(\omega ,\eta )$ is linear.
\end{enumerate}
Let $v\in V$ and $w\in W$.  Then, the \emph{tensor product}\index{Tensor product (of tensors)}, $v\otimes w\in V\otimes W$\index[notation]{$v\otimes w$}, is defined by
\begin{equation}
[v\otimes w](\omega ,\eta )\coloneqq \omega (v)\eta (w).
\end{equation}
\begin{rmk}
To clarify, there are tensor products of \emph{vector spaces}, and then there are tensor products of \emph{vectors themselves}.  The tensor product of two vectors `lives in' the tensor product of the corresponding vector spaces.  And in fact, \emph{everything} in $V\otimes W$, while \emph{not} of the form $v\otimes w$ itself necessarily, can be written as a finite sum of elements of this form---see \cref{prp5.1.8} below.  (Elements of the form $v\otimes w$ are sometimes called \emph{pure} or \emph{simple}, as opposed to, e.g.~, $v_1\otimes w_1+v_2\otimes w_2$).
\end{rmk}
\begin{rmk}
In other words, $V\otimes W$ is the set of all \emph{bilinear}\index{Bilinear} maps on $V^{\dagger}\times W^{\dagger}$, where bilinear means that, if you fix all arguments except one, you obtain a linear map.
\end{rmk}
\begin{rmk}
In practice, I find it easier to think of the tensor product as the vector space spanned by guys of the form $v\otimes w$ (as opposed to bilinear maps on the cartesian product of the duals---ick).
\end{rmk}
\begin{rmk}
This is neither the most general nor the most elegant definition of the tensor product.  The `right' way to define the tensor product is, as usual, by finding the properties which uniquely characterize it.  Maybe I will update the notes at a later date to include this, but my personal feeling right now is that this would take us a bit too far astray (after all, we're not studying linear algebra or tensors for their own sake---for us, they're just tools to do calculus).
\end{rmk}
\end{dfn}
\begin{prp}\label{prp5.1.8}
$V\otimes W$ is the span of $\{ v\otimes w:v\in V,\ w\in W\}$.
\end{prp}
\begin{dfn}[Tensor]\label{Tensor}
A \emph{tensor}\index{Tensor} of rank $\coord{k,l}$ over $V$ is an element of
\begin{equation}
\underbrace{V\otimes \cdots \otimes V}_k\otimes \underbrace{V^{\dagger}\otimes \cdots \otimes V^{\dagger}}_l
\end{equation}
$k$ is the \emph{contravariant rank}\index{Contravariant rank} and $l$ is the \emph{covariant rank}\index{Covariant rank}.  If $l=0$, then the tensor is \emph{contravariant}\index{Contravariant tensor}, and if $k=0$, then the tensor is \emph{covariant}\index{Covariant tensor}.  If $T$ is a tensor of rank $\coord{k,l}$, then we shall write
\begin{equation}
T\indices{^{a_1\cdots a_k}_{b_1\cdots b_l}}
\end{equation}
to help remind us what type of tensor this is.\footnote{Don't let the indices mislead you---there are no choices being made.  Everything is ``coordinate-free'', even later when we start manipulating the indices themselves---it's still all coordinate-free.}
\begin{rmk}
\emph{Do not be sloppy and not stagger your indices!}  If you do, you will eventually make a mistake.  For example, later we will be raising and lowering indices.  Suppose I start with $T^{ab}$, I lower to obtain $T_b^a$, and then I raise again to obtain $T^{ba}$---I should obtain the same thing, but in general $T^{ab}\neq T^{ba}$, and so I have an error.  It may seem obvious to the point of being silly when I point it out like this, but this is a mistake that is easy to make if there is a big long computation in between the raising and lowering (especially if it's more than just $a$ and $b$ floating around).  And of course, you will never have this problem if you stagger:  $T^{ab}$ goes to $T\indices{^a_b}$ goes back to $T^{ab}$.
\end{rmk}
\begin{rmk}
I claim that this is likewise not that foreign of a concept.  In fact, there are so many examples you are familiar with that I don't even want to put them in a remark, so see the next example.
\end{rmk}
\end{dfn}
\begin{exm}
\begin{enumerate}
Disclaimer:  While none of the examples themselves make use of things we haven't done yet, some of the notation does (e.g.~$v^a\omega _a$).  Read onwards and come back later if this really bothers you.
\item Vectors (written $v^a$) themselves are tensors of type $\coord{1,0}$.
\item Covectors (or linear functionals) (written $\omega _a$) are of type\footnote{By definition.  The term ``covector'' is our word for a tensor of type $\coord{0,1}$, or in other words, an element of $V^{\dagger}$ (i.e.~a linear functional).} $\coord{0,1}$.  For $\omega$ a linear functional and $v$ a vector, $\omega (v)$ is written as $v^a\omega _a$.
\item The dot product (written temporarily as $g_{ab}$) is an example of a tensor of type $\coord{0,2}$---it takes in two vectors and spits out a number, written $v\cdot w=v^aw^bg_{ab}$.
\item Linear transformations (written $T\indices{^a_b}$) are tensors of type $\coord{1,1}$---it takes in a single vector and spits out another vector (written $v^a\mapsto T\indices{^a_b}v^b$).  Note that it is $T\indices{^a_b}$ and not $T\indices{^b_a}$---your convention could go either way, but in the convention we choose the indices that are contracted during composition are closer together.
\end{enumerate}
\end{exm}

There are three key constructions involving tensors that we will need, the \emph{tensor product}, \emph{contraction}, and the \emph{dual vector}.  The tensor product we have already done in \cref{TensorProduct},\footnote{Well, I suppose we have to define the tensor products of \emph{arbitrary} tensors, as opposed to just vectors, but the definition in general is just an extension of the one we've already written down.} and so we simply explain the how to write the tensor product in index notation.
\begin{displayquote}
The tensor product of $[T_1]\indices{^{a_1\ldots a_{k_1}}_{b_1\ldots b_{l_1}}}$ and $[T_2]\indices{^{a_1\ldots a_{k_2}}_{b_1\ldots b_{l_2}}}$ is denoted
\begin{equation}
[T_1]\indices{^{a_1\ldots a_{k_1}}_{b_1\ldots b_{l_1}}}[T_2]\indices{^{a_1\ldots a_{k_2}}_{b_1\ldots b_{l_2}}}.
\end{equation}
That is, you literally just juxtapose them.
\end{displayquote}
We now turn to \emph{contraction}.
\begin{dfn}[Contraction]\label{Contraction}
Let $T\indices{^{a_1\ldots a_k}_{b_1\ldots b_l}}\coloneqq [v_1]^{a_1}\cdots [v_k]^{a_k}[\omega _1]_{b_1}\cdots [\omega _l]_{b_l}$ be a tensor of rank $\coord{k,l}$ (recall (\cref{prp5.1.8}) that every tensor can be written as a sum of tensors of this form).  Then, the \emph{contraction}\index{Contraction} of $T$ along the $a_i$ and $b_j$ index is defined to be
\begin{equation}
\begin{multlined}
T\indices{^{a_1\ldots a_{i-1}a_{i+1}\ldots a_k}_{b_1\ldots b_{j-1}b_{j+1}\ldots b_l}}\coloneqq \\ \omega _j(v_i)\cdot [v_1]^{a_1}\cdots [v_{i-1}]^{a_{i-1}}[v_{i+1}]^{a_{i+1}}\cdots [v_k]^{a_k}[\omega _1]_{b_1}\cdots [\omega _{j-1}]_{b_{j-1}}[\omega _{j+1}]_{b_{j+1}}\cdots [\omega _l]_{b_l}.
\end{multlined}
\end{equation}
The contraction of tensor that is a sum of simple tensors is defined to be the sum of the contraction of those simple tensors.
\begin{rmk}
In general, the way one can `decompose' a general tensor as a sum of simple ones is not unique, so we must technically check that this is well-defined.
\end{rmk}
\begin{rmk}
This might \emph{look} atrocious, but it's actually quite simple.  Covectors take in vectors and spit-out numbers, and so the contraction of a tensor product in its $a_i$ and $b_j$ index is formed by plugging in the $i^{\text{th}}$ vector into the $j^{\text{th}}$ covector.
\end{rmk}
\begin{rmk}
Keep in mind that you can \emph{only} contract upper-indices (contravariant) with lower (covariant) ones.
\end{rmk}
\begin{rmk}
A couple of examples:  The only possible contraction of the tensor $v^a\omega _b$ is written $v^a\omega _a$ and this of course is just $\omega (v)$.  If $T\indices{^a_b}$ is a linear transformation and $v^a$ is a vector, then the contraction $T\indices{^a_b}v^b$ is just $T(v)$, that is, the image of $v$ under the linear transformation $T$.
\end{rmk}
\begin{rmk}
The index $k$ in $[v_k]^{a_k}$ is part of the name of the vector---the entire name is $v_k$, and then the notation $[v_k]^{a_k}$ reminds us that $v_k$ is a $\coord{1,0}$-tensor, i.e.~just a vector.
\end{rmk}
\end{dfn}
We need one more ingredient before we can get to actual differentiation, namely that of a \emph{metric}.
\begin{dfn}[Metric (on a vector space)]\label{MetricVectorSpace}
A \emph{metric}\index{Metric (on a vector space)} $g$\footnote{The ``g'' is for gravity (in general relativity, gravity is modeled as a metric on space-time).} on $V$ is a covariant tensor of rank $2$ such that
\begin{enumerate}
\item \label{MetricVectorSpace.Symmetry}(Symmetry) $g(v_1,v_2)=g(v_2,v_1)$; and
\item \label{MetricVectorSpace.Nonsingularity}(Nonsingularity)\index{Nonsingular (metric)} the map from $V$ to $V^{\dagger}$ defined by $v\mapsto g(v,\blankdot )$, where $g(v,\blankdot )$ is the linear functional which sends $w$ to $g(v,w)$, is an isomorphism of vectors spaces.
\end{enumerate}
\begin{rmk}
If $v^a$ is a vector, then we write $v_a\coloneqq g_{ab}v^b$.  $v_a$ is the \emph{dual vector}\index{Dual vector} (which itself is not a vector---it's a covector) of $v^a$.  \emph{Nonsingualirty} is key because it allows us to reverse this process.  If $\omega _a$ is a covector, then because the map $v^a\mapsto v_a$ is an \emph{isomorphism}, there is a unique vector, written $\omega ^a$, that is equal to $\omega _a$ under this map.
\end{rmk}
\begin{rmk}
\ref{MetricVectorSpace.Symmetry} can be written $g_{ab}=g_{ba}$.  Also note that $g(v_1,v_2)=[v_1]^a[v_2]^bg_{ab}$.
\end{rmk}
\begin{rmk}
The idea of a notion of a metric on a vector space and a metric on a set (in the context of uniform space theory) have little to nothing to do with each other.  It is merely a coincidence of terminology that is so ingrained that even I dare not go against it.
\end{rmk}
\begin{rmk}
The term ``metric'' in this sense of the word should really not be thought of as a sort of distance, but rather as a sort of dot product.  Indeed, you can verify that the dot product is a metric, and furthermore, in a sense that we don't bother to make precise, every positive-definite metric (on a vector space) is equivalent to the usual euclidean dot product.  There is \emph{some} connection with the other notion of metric, however---positive-definite metrics give us norms (the square-root $g(v,v)$), which in turn gives us a metric (in the other sense).
\end{rmk}
\begin{rmk}
Nonsingularity is usually replaced with the requirement that $g(v,w)=0$ for all $w$ implies that $v=0$ (called \emph{nondegeneracy}\index{Nondegenrate (metric)}.  In finite dimensions, this is equivalent to nonsingularity (by the Rank-Nullity Theorem).  In infinite dimensions, however, they are not equivalent, and it is nonsingularity that we want (so that we can raise and lower indices).
\end{rmk}
\end{dfn}
It's worth noting that, everything \emph{except} raising and lowering indices we can do without a metric.  To raise and lower indices, we do need that \emph{extra} structure.  In particular, if you pick a different metric, then your meaning of $v_a$ will change even though the metric does not appear explicitly in this notation.

In summary:
\begin{enumerate}
\item The tensor product of two vectors $v^a$ and $w^a$, written $v^aw^b$, is defined to be the bilinear map that sends the pair of covectors $\coord{\omega _a,\eta _a}$ to $(\omega _av^a)(\eta _aw^a)$.  In practice, it's not particularly helpful to think of what this is\footnote{When you add $2$ to $3$ do you think about $2$ being an equivalence class of sets with respect to the equivalence relation of isomorphism in the category of sets?  Here, I'll help you out:  No, you do not.}---in practice what matters is can you manipulate them.
\item A general tensor of rank $\coord{k,l}$is an element in the tensor product of $k$ copies of $V$ with $l$ copies of $V^{\dagger}$.
\item The definition of the tensor product of vectors can be extended to the tensor product of any tensors.  In index notation, this is denoted simply by juxtaposition.
\item We can contract indices.
\item If we have a metric, we can also raise and lower indices.
\end{enumerate}

\section{The definition}

One thing that I personally found conceptually confusing with differentiation in $\R ^d$ itself that was elucidated for me when passing to the study of more general manifolds was the distinction between a \emph{vector} and a \emph{point}.  The problem in $\R ^d$ of course is that the space of points and the space of factors are effectively the same thing, they are both given by a $d$-tuple of real numbers, when in fact, they are really playing quite different roles.  The points tell you ``where'' we are and the vectors tell you ``what direction'' to go in.  In a general manifold, the points that tell you ``where'' form the points of the space itself and the vectors do \emph{not} live in the entire space itself, but rather the tangent spaces.

Thus, while we have no intention of doing manifold theory in general,\footnote{In contrast to topology, for example, you do have to prove essentially all of your results in $\R ^d$ first and \emph{then} extend them to arbitrary manifolds, whereas in principle you can prove all the results about topology you ever wanted without even mentioning $\R$.  This is one reason among others why we do general topology but not manifold theory.} we will make use of some suggestive notation that comes from the theory.
\begin{displayquote}
Throughout this chapter, the symbol $\R ^d$ will be used to denote $d$-dimensional euclidean space \emph{as a metric space}.  For each $x\in \R ^d$, we define $\tangent{\R ^d}{x}$, the \emph{tangent space at $x$ in $\R ^d$}\index{Tangent space} to be the \emph{metric vector space} $\R ^d$ (with metric being the dot product).\footnote{Careful:  the word ``metric'' here is being used in two totally different senses.  I know the terminology is perverse, but don't look at me!  I'm not the one that came-up with it.}  Furthermore, we declare that $\tangent{\R ^d}{x_1}\neq \tangent{\R ^d}{x_2}$ for $x_1\neq x_2$.\footnote{There are many ways to do this, but one way, for example, is to take $\tangent{\R ^d}{x}\coloneqq \R ^d\times \{ x\}$.}  We will often, but not always, use abstract index notation for vectors $v^a\in \tangent{\R ^d}{x}$ to help remind us that they are to be thought of as vectors instead of points.  It will sometimes be useful to assign other vectors spaces to each point (for example, for a function $f:\R ^d\rightarrow \R ^m\coloneqq V$).  In general, the assignment of a vector space $V_x$ to each point of $\R ^d$ will be called a \emph{vector bundle}\index{Vector bundle} on $\R ^d$.  The assignment to each point of the tangent space is a special vector bundle called the \emph{tangent bundle}\index{Tangent bundle}.
\end{displayquote}
In particular, as sets, we might have that $\R ^d=\tangent{\R ^d}{x}$, but that's it---the two objects don't even live in the same category, and so it doesn't even make sense to ask whether there is some isomorphism between them (unless you forget some of the structure, in which case you're actually changing the object).  For example, $\R ^d$ is just a metric space---you cannot add any two of its elements.  Tangent vectors, elements of $\tangent{\R ^d}{x}$, on the other hand, we can add just fine.

To clarify, this is not actually how the definition goes in general.  The general definition of the tangent space requires us to first be able to talk about manifolds, which in turn requires us to know how differentiation in $\R ^d$ works, which of course we have not done yet.  Thus, we are making use of this notation only to help clarify the study of differentiation in $\R ^d$.  In principle, once enough of this theory has been developed so that we can talk about tangent spaces in general, we would replace the above with the `actual' definition.  Likewise, this is not the actual definition of vector bundles---for us, the term ``vector bundle'' is just a phrase in the English language that we are going to use to communicate mathematical ideas.  It requires basic manifold theory to define vector bundles properly.

This speak of tangent spaces allows us to make an important definition.
\begin{dfn}[Tensor field]\label{TensorField}
A \emph{tensor field}\index{Tensor field} of rank $\coord{k,l}$ is a function on $\R ^d$ whose value at $x$ is a rank $\coord{k,l}$ tensor on $\tangent{\R ^d}{x}$.  A tensor field $T\indices{^{a_1\ldots a_k}_{b_1\ldots b_l}}$ is \emph{continuous}\index{Continuous (tensor)} iff $[v_1]^{b_1}\cdots [v_l]^{b_l}[\omega _1]_{a_1}\cdots [\omega _k]_{a_k}T\indices{^{a_1\ldots a_k}_{b_1\ldots b_l}}$ is continuous for all vectors $[v_1]^a,\ldots ,[v_l]^a$ and covectors $[\omega _1]_a,\ldots ,[\omega _k]_a$.
\begin{rmk}
It's just an assignment of a tensor to every point in $\R ^d$.  For example, a vector field is an assignment of a vector to every point.
\end{rmk}
\end{dfn}

Finally, with this (hopefully elucidating) notation in hand, we can define the derivative.
\begin{dfn}[Directional derivative]\label{DirectionalDerivative}
Let $S\subseteq \R ^d$, let $f:S\rightarrow \R$, let $x\in \Int (S)$, and let $v\in \tangent{\R ^d}{x}$.  Then, the \emph{directional derivative}\index{directional derivative} of $f$ at $x$ in the direction $v$, $\D _vf(x)$\index[notation]{$\D _vf(x)$}, is defined by
\begin{equation}\label{DifferenceQuotient}
\D _vf(x)\coloneqq \lim _{\epsilon \to 0}\frac{f(x+\epsilon v)-f(x)}{\epsilon}.
\end{equation}
If this limit exists, then $f$ is \emph{differentiable at $x$ in the direction $v^a$}.
\begin{rmk}
The expression inside the limit on the right-hand side of \eqref{DifferenceQuotient} is the \emph{difference quotient}\index{Difference quotient}.
\end{rmk}
\begin{rmk}
Note that we \emph{only} define the derivative at points on the interior of the domain.  The reason we do this is because otherwise we cannot guarantee that $x+\varepsilon v$ is in the domain of $f$ for even a single $\varepsilon >0$, so that $f(x+\epsilon v)$ does not make sense, in which case it is nonsensical to ask whether the limit of the difference quotient exists or not.
\end{rmk}
\begin{rmk}
For some reason, people tend to write $\lim _{h\to 0}$ in this definition instead of $\lim _{\epsilon \to 0}$.  Not sure why.  $\epsilon$ is used for `small' numbers everywhere else in analysis.  Moreover, what happens if your function is called $h$ (not that uncommon of a name for a function)?  What are you doing to do?  Write $h(x+hv)$?  Ew.  Maybe you object ``But $\epsilon$ can't be negative!  That's against the rules!''.  No.  $\varepsilon$ is typically taken to be positive, but I am not writing that\textellipsis I am writing ``$\epsilon$''.  The distinction is deliberate---$\varepsilon$ is used to indicate a `small' positive number and $\epsilon$ is used to indicate a `small' (not-necessarily-positive) number.\footnote{I do not guarantee that I will never break this convention.}  If you think the difference between these symbols is hard to notice, I recommend you increase your \TeX nichal skills\footnote{Or get glasses.} (one is ``\texttt{\textbackslash varepsilon}'' and the other is ``\texttt{\textbackslash epsilon}'')\footnote{Actually, if you're really \TeX nichal, you might have noticed that this itself is a slight lie---because of font issues, I had to `steal' \texttt{\textbackslash epsilonup} from the \texttt{newpx} package.}.
\end{rmk}
\end{dfn}
\begin{dfn}[Derivative (of a function)]\label{Derivative}
\begin{savenotes}
Let $S\subseteq \R ^d$, let $f:S\rightarrow \R$ and let $x\in \Int (S)$.  Then, $f$ is \emph{differentiable}\index{Differentiable} at $x$ iff
\begin{enumerate}
\item $\D _vf(x)$ exists for all $v\in \tangent{\R ^d}{x}$; and
\item the map $\tangent{\R ^d}{x}\ni v\mapsto \D _vf(x)$ is linear.
\end{enumerate}
In this case, we write
\begin{equation}
v^a\nabla _af(x)\coloneqq \D _vf(x)
\end{equation}
\index[notation]{$v^a\nabla _af(x)$} to indicate the linearity.  If $f$ is differentiable at $x$ for all $x\in \Int (S)$, then $f$ is differentiable (on $S$).
\begin{rmk}
Warning:  This is \emph{not} the universally accepted definition of differentiability.  This is (almost) what is referred to as \emph{g\^{a}teau differentiability}\index{G\^{a}teau differentiable} as opposed to a stronger definition called \emph{fr\'{e}chet differentiability}\index{Fr\'{e}chet differentiable}.  We say ``almost'' because we \emph{require that the map $v^a\mapsto \nabla _af(x)$ be linear}, whereas g\^{a}teaux differentiability merely requires the existence of $\D _vf(x)$ for all $v\in \tangent{\R ^d}{x}$.  (Thus, our definition is slightly stronger.)  We chose this definition because it is easier and, IMHO, more natural, than fr\'{e}chet differentiability.\footnote{We certainly need the map $v\mapsto \D _vf(x)$ to be linear if we are to do tensor calculus (otherwise the derivative itself won't be a tensor).}  That being said, it does have its pathologies.  Of particular note is that \emph{there are differentiable functions that are not continuous}---see \cref{exm6.2.15}.  Fortunately, this cannot happen in one-dimension---see \cref{prp6.5.1}.
\end{rmk}
\begin{rmk}
Of course, there exist examples in which the directional derivative is \emph{not} linear (otherwise we would not have made this an assumption)---see the following example (\cref{exm6.2.10}).  In particular, if we say that $f$ is differentiable at $x$, then \emph{we are assuming the directional derivative is linear}, not only that it exists.

Despite this pathology, fortunately, the derivative is homogeneous, a fact particularly relevant in one-dimension.
\end{rmk}
\begin{exr}
Let $S\subseteq \R ^d$, let $f:S\rightarrow \R$, let $x\in \R ^d$, let $v\in \tangent{\R ^d}{x}$, and let $\alpha \in \R$.  Show that, if $\D _vf(x)$ exists, then $\D _{\alpha v}f(x)$ exists and furthermore that
\begin{equation}
\D _{\alpha v}f(x)=\alpha \D _vf(x).
\end{equation}
\end{exr}
\begin{rmk}
A corollary of this is that, for a function $f:\R \rightarrow \R$ to be differentiable at a point, it need only have one directional derivative (because then it has \emph{all} directional derivatives by the previous exercise).  Furthermore, in one dimension, there is essentially only one choice of $v^a$---everything else is a scalar multiple.  Thus, in one dimension, we \emph{always} take $v^a=1\in \tangent{\R}{x}\cong _{\Vect _\R}\R$ and write
\begin{equation}
\frac{\dif}{\dif x}f(x)\coloneqq v^a\nabla _af(x).
\end{equation}
\end{rmk}
\begin{rmk}
The notation $v^a\nabla _af$ is obviously suggestive.  Hopefully it reminds you of the fact from multivariable calculus that the directional derivative of a function $f$ in the direction $\vec{v}$ is given by the dot product of $\vec{v}$ with the gradient of $f$:  $\vec{v}\cdot \vec{\nabla}f$.\footnote{Contraction of indices should always be thought of as a sort of dot product.}  For fixed $f$ and $x$, the map $v^a\mapsto [v^a\nabla _af](x)$ is linear in $v^a$, and so defines a \emph{linear functional}, that is to say, $\nabla _af(x)\in \tangent{\R ^d}{x}^{\dagger}$, or equivalently, that $\nabla _af$ is a covector field on $\R ^d$.  This covector field is the \emph{derivative}\footnote{That is to say, $\nabla _af$ is the derivative and $v^a\nabla _af$ is the derivative in the direction of $v^a$ (or the directional derivative in the direction of $v^a$).} or \emph{gradient}\index{Gradient} of $f$ at $x$.
\end{rmk}
\begin{rmk}
Perhaps more accurate notation would have been $[v^a\nabla _af](x)$, that is, $v^a\nabla _af$ is a function (in the case that $f$ is differentiable anyways), and so $v^a\nabla _af(x)$ is the value of $v^a\nabla _af$ at $x$, as opposed to, the derivative of the function $f(x)$, which is just $0$ of course.\footnote{I know to some this may seem pedantic, but $f$ is the function, $f(x)$ is the value at $x$ of the function, so that $f(x)$ is just a number.}  The point is:  you must compute the derivative, and \emph{then} plug-in $x$.  It might seem silly in such a simple context, but in much more complicated contexts I myself have made this very mistake.  For example, suppose you are computing a functional derivative (to find a noether charge, say) of some action functional in physics, and you want to see that this quantity is conserved `on-shell' (i.e.~when the equations of motion hold)---you cannot use the equations of motion before you finish computing the noether charge `off-shell':  that's cheating (and more importantly, possibly just plain wrong)!\footnote{Once again, if the physics analogy is meaningless to you, just ignore it.}
\end{rmk}
\begin{rmk}
Suppose that $f$ is differentiable.  Then, $v^a\nabla _af$ itself is a function on all of $\R ^d$, and so we may differentiate\footnote{If the limit exists, of course} this as well to obtain
\begin{equation}
w^b\nabla _b(v^a\nabla _af)=w^bw^a\nabla _b\nabla _af.
\end{equation}
(Warning:  This equality will not hold in general if $v^a$ is a nonconstant function of $x$.)  Just as $\nabla _af(x)$ defined a covector on $\tangent{\R ^d}{x}$, so to does $\nabla _b\nabla _af(x)$ define a covariant $2$-tensor on $\tangent{\R ^d}{x}$.\footnote{Note that a covariant $2$-tensor is an element of $\tangent{\R ^d}{x}^\dagger \otimes \tangent{\R ^d}{x}^\dagger$.  When we say it is a ``covariant $2$-tensor \emph{on} $\tangent{\R ^d}{x}$'', this is what we mean---it's neither a (real-valued) function nor an element of $\tangent{\R ^d}{x}$.}
\end{rmk}
\begin{rmk}
If all higher derivatives of $f$ exist, then $f$ is \emph{infinitely-differentiable}\index{Infinitely-differentiable}.  If all higher derivatives of $f$ exist \emph{and} they are continuous, then $f$ is \emph{smooth}\index{Smooth}.\footnote{Yes, there is a difference.  In fact, infinitely-differentiable functions need not even be continuous---see \cref{exm6.2.15}.}
\end{rmk}
\end{savenotes}
\end{dfn}
\begin{exm}[A function which is g\^{a}teaux differentiable but not differentiable]\footnote{This example comes from Ted Shifrin's \href{http://math.stackexchange.com/questions/694486/show-that-the-directional-derivative-is-linear-by-definition}{math.stackexchange answer}.}\label{exm6.2.10}
In other words, we seek a function all of whose directional derivatives exist at a point, but for which the map $v\mapsto \D _vf(x)$ is not linear.

Define $f:\R ^2\rightarrow \R$ by
\begin{equation}
f(\coord{x,y})\coloneqq \begin{cases}0 & \text{if }\coord{x,y}=\coord{0,0} \\ \frac{x^2y}{x^2+y^2} & \text{otherwise}\end{cases}.
\end{equation}
Let $v\coloneqq \coord{v_x,v_y}\in \tangent{\R ^d}{\coord{0,0}}$ be nonzero (the directional derivative in the direction of the zero vector is always---gasp---zero).  Then,
\begin{equation}
\frac{f(\coord{0,0}+\epsilon v)-f(\coord{0,0})}{\epsilon}=\frac{\frac{(\epsilon v_x)^2(\epsilon v_y)}{(\epsilon v_x)^2+(\epsilon v_y)^2}}{\epsilon}=\frac{v_x^2v_y}{v_x^2+v_y^2}.
\end{equation}
In particular, the limit of this as $\epsilon \to 0$ always exist.  On the other hand, the map
\begin{equation}
v\mapsto \frac{v_x^2v_y}{v_x^2+v_y^2}\eqqcolon \D _vf(\coord{0,0})
\end{equation}
is certainly not linear.  For example,
\begin{equation}
\D _{\coord{1,1}}f(\coord{0,0})=\frac{1}{2}\neq 0+0=\D _{\coord{1,0}}f(\coord{0,0})+\D _{\coord{0,1}}f(\coord{0,0}).
\end{equation}
\end{exm}
In fact, things that are arguably even worse than this can happen---it can happen that all directional derivatives exist \emph{and} furthermore the map that seconds a vector to the directional derivative in that direction is linear, but yet the function not even be continuous.  Unfortunately, we will have to postpone such a counter-example until after having defined the exponential function---see \cref{exm6.2.15}

We can use the definition of the derivative of a function to define the derivative for \emph{all} tensor fields.  The idea is that, by plugging-in enough vectors and covectors, all tensor fields reduce to just a function.
\begin{dfn}[Derivative (of a tensor)]\label{DerivativeTensor}
Let $T\indices{^{a_1\ldots a_k}_{b_1\ldots b_l}}$ be a tensor field of rank $\coord{k,l}$ on $\R ^d$.  Then, the \emph{derivative}\index{Derivative (of a tensor field)}, $\nabla _aT\indices{^{a_1\ldots a_k}_{b_1\ldots b_l}}$, is defined by
\begin{equation}
\begin{multlined}
v^a[\omega _1]_{a_1}\cdots [\omega _k]_{a_k}[v_1]^{b_1}\cdots [v_l]^{b_l}\nabla _aT\indices{^{a_1\ldots a_k}_{b_1\ldots b_l}}= \\ v^a\nabla _a\left( [\omega _1]_{a_1}\cdots [\omega _k]_{a_k}[v_1]^{b_1}\cdots [v_l]^{b_l}T\indices{^{a_1\ldots a_k}_{b_1\ldots b_l}}\right) 
\end{multlined}
\end{equation}
for all covectors $\omega _1,\ldots ,\omega _k$ and vectors $v,v_1,\ldots ,v_l$.
\begin{rmk}
This makes sense because the thing inside the derivative on the right-hand side is just a function.
\end{rmk}
\begin{rmk}
Among other things, we can how differentiate functions which take their values in $\R ^m$, as we just interpret such a function as a vector field on $\R ^d$.\footnote{We are cheating a bit.  $\R ^m$ is \emph{not} the tangent space of any point---you can tell because it might not even have the right dimension.  The appropriate way to deal with this is to attach a new vector space $V_x$ to each point, where $V_x\cong _{\Vect _{\R}}\R ^e$, and then interpret the value $f(x)$ as an element of $V_x$.  Fortunately, this has no effect on the above definition.  For such functions, I will write $f^\alpha$, to remind us that $f^\alpha$ lives in a different vector bundle than usual.}
\end{rmk}
\begin{rmk}
Thus, the  in particular shifts the covariant rank of a tensor up by $1$ (for example, as functions are just $\coord{0,0}$ tensors, it takes functions to covector fields).
\end{rmk}
\end{dfn}
\begin{exm}[Gradient, divergence, curl, and laplacian]
The gradient, divergence, curl, and laplacian from multivariable calculus are all just special cases of derivatives of tensors (along with other tensorial constructions such as contraction).

Throughout this example, let $f$ be a smooth function on $\R ^d$ and let $v^a$ be a vector field on $\R ^d$. 

The \emph{gradient} if $f$ is simply $\nabla _af$.  You've already seen this guy.

The \emph{divergence}\index{Divergence} of $v^a$ is $\nabla _av^a$.  Note how in multivariable calculus you might write this as $\vec{\nabla}\cdot \vec{v}$, the `dot product' of the gradient with the vector field $\vec{v}$.  The index contraction here hopefully makes obvious how this is the divergence that you know and (maybe) love.

The \emph{curl} is trickier, but this is not surprising, as it was trickier than the rest in multivariable calculus as well.  First of all, the curl of a vector field is in general \emph{not} a vector field---this is very specific to three-dimensions, essentially because the cross-product is specific to three-dimensions.  In general, it is also more convenient to define the curl of a \emph{covector} field instead of a vector field.\footnote{They are essentially equivalent, however, as you can translate back and forth by raising and lowering indices.}  For us, the \emph{curl}\index{Curl} of a covector field $\omega _a$ will be defined to be $\nabla _a\omega _b-\nabla _b\omega _a$.\footnote{To get back a vector field \emph{in three-dimensions} you have to in turn contract this with $\epsilon ^{abc}$, where $\epsilon ^{abc}$ is the unique completely antisymmetric contravariant tensor of rank $3$ of trace $1$.  ``Completely antisymmetric'' means if you flip any two indices you pick-up a minus sign (e.g.~$\epsilon ^{abc}=-\epsilon ^{bac}$; ``of trace $1$'' means $\epsilon _{abc}\epsilon ^{abc}=1$.  The reasons you don't get back a vector in higher dimensions is because in higher dimensions this becomes $\epsilon ^{a_1\ldots a_d}$, and then you will have instead a contravariant tensor of rank $d-2$.}  For what it's worth, in higher dimensions, mathematicians don't call this the curl---they call it the \emph{exterior derivative}\index{Exterior derivative}.\footnote{Not sure what's ``exterior'' about it though.}

The \emph{laplacian}\index{Laplacian} of a function is the divergence of the gradient---literally (modulo the raising of an index), that is, the laplacian of $f$ is $\nabla _a\nabla ^af$.
\end{exm}

\section{Tools for calculation}

And now we come to the results which you are probably most familiar with from calculus.
\begin{prp}[Algebraic Derivative Theorems]\index{Algebraic Derivative Theorems}\label{AlgebraicDerivativeTheorems}
Let $f,g:\R ^d\rightarrow \R$ be differentiable.\footnote{Everything works just as well (with the same proof) if you just assume it is differentiable in some direction at some point, but the notation is more tedious.} and let $\alpha \in \R$.  Then,
\begin{enumerate}
\item \label{AlgebraicDerivativeTheorems.Linearity}(Linearity) $\nabla _a(f+g)=\nabla _af+\nabla _ag$;
\item \label{AlgebraicDerivativeTheorems.Homogeneity}(Homogeneity) $\nabla _a(\alpha f)=\alpha \nabla _af$;
\item \label{AlgebraicDerivativeTheorems.ProductRule}(Product Rule)\index{Product Rule}$\nabla _a(fg)=(\nabla _af)g+f(\nabla _ag)$;\footnote{I would get in the habit of not mixing-up the order of $f$ and $g$ (e.g.~by writing $(\nabla _af)g+(\nabla _ag)f$ or something of the like).  It won't matter for us, but it can and will latter when you're working with things that are not commutative (the cross-product of vectors is probably the most elementary example).} and
\item \label{AlgebraicDerivativeTheorems.QuotientRule}(Quotient Rule)\index{Quotient Rule} $\nabla _a\left( \frac{f}{g}\right) =\frac{(\nabla _af)g-f(\nabla _ag)}{g^2}$ whenever $g\neq 0$.
\end{enumerate}
\begin{rmk}
The first three are true just as well for $f$ and $g$ arbitrary tensor fields, with essentially the same exact proofs (the juxtaposition denotes the tensor product of course).  The Quotient Rule does not make sense, however, as in general you cannot invert tensors.
\end{rmk}
\begin{proof}
\ref{AlgebraicDerivativeTheorems.Linearity} and \ref{AlgebraicDerivativeTheorems.Homogeneity} follows straight from the corresponding results above limits---see \cref{AlgebraicLimitTheorems}\ref{enmAlgebraicLimitTheorems.i} and \cref{AlgebraicLimitTheorems}\ref{enmAlgebraicLimitTheorems.ii}.

As for \ref{AlgebraicDerivativeTheorems.ProductRule}, we have
\begin{equation}
\begin{multlined}
\frac{f(x+hv)g(x+hv)-f(x)g(x)}{h}=\footnote{We added and subtracted $f(x)g(x+hv)$.} \\ \left( \frac{f(x+hv)-f(x)}{h}\right) g(x+hv)+f(x)\left( \frac{g(x+hv)-g(x)}{h}\right) ,
\end{multlined}
\end{equation}
and so taking limits gives us the Product Rule.

Similarly, the proof of the Quotient Rule amounts to just algebraic manipulation of the difference quotient:
\begin{equation}
\frac{\frac{f(x+hv)}{g(x+hv)}-\frac{f(x)}{g(x)}}{h}=\frac{\left( \frac{f(x+hv)-f(x)}{h}\right) g(x)-f(x)\left( \frac{g(x+hv)-g(x)}{h}\right)}{g(x+hv)g(x)}.
\end{equation}
\end{proof}
\end{prp}
\begin{prp}[Chain Rule]\index{Chain rule}\label{ChainRule}
Let $f^\alpha :\R ^d\rightarrow \R ^m$ and $g^\mu :\R ^m\rightarrow \R ^n$ be differentiable.\footnote{The $\mu$ index is used to remind us that $g^\mu$ lives in a $\R ^n$ (as opposed to $\R ^m$ or $\R ^d$).}  Then,
\begin{equation}
\nabla _a[g\circ f]^\mu (x)=\nabla _\alpha g^\mu (f(x))\nabla _af^\alpha (x).
\end{equation}
\begin{rmk}
One of the things I really love about index notation is that it almost dictates what the answer has to be---how many ways can you? construct a tensor with one $\R ^d$ covariant index and one $\R ^n$ contravariant index using only $f^\alpha$, $g^\mu$, and their derivatives?
\end{rmk}
\begin{proof}
To prove this, by the definition of the derivative of tensors, we need to show that
\begin{equation}
\omega _\mu \nabla _a[g\circ f]^\mu (x)=\omega _\mu \nabla _\alpha g^\mu (f(x))\nabla _af^\alpha (x)
\end{equation}
for all covectors (living in $\R ^n$) $\omega _\mu$.  In particular, it suffices to prove the result for $g:\R ^m\rightarrow \R$ (because now $\omega _\mu g^\mu :\R ^m\rightarrow \R$).

Let $v^a$ be a constant vector field on $\R ^d$.  Then, what we actually want to show is
\begin{equation}
v^a\nabla _a[g\circ f](x)=\left( \nabla _\alpha g(f(x))\right) \left( v^a\nabla _af^\alpha (x)\right) ,
\end{equation}
as $v^a$ is arbitrary.  On one hand
\begin{equation}\label{5.2.18}
v^a\nabla _a[g\circ f](x)=\lim _{h\to 0}\frac{g\left( f(x+hv)\right) -g(f(x))}{h}
\end{equation}
and on the other hand
\begin{equation}
\left( \nabla _\alpha g(f(x))\right) \left( v^a\nabla _af^\alpha (x)\right) =\lim _{h\to 0}\frac{1}{h}\left[ g\left( f(x)+hv^a\nabla _af(x)\right) -g(f(x))\right]
\end{equation}
Thus, we want to show that
\begin{equation}
f(x+hv)=f(x)+hv^a\nabla _af(x)\text{ as }h\to 0,
\end{equation}
but of course this is just the very definition of the derivative.
\end{proof}
\end{prp}

\section{The exponential function}

So, we've proven several properties about how to manipulate derivatives, but what is there to differentiate?  Polynomials?  That's no fun.  Let's find a function even easier to differentiate.  In fact, let's see if we can find a function that is equal to its own derivative
\begin{equation}
\tfrac{\dif}{\dif x}f(x)=f(x).
\end{equation}
Don't be a smart-ass---zero doesn't count.  Of course, you already know the answer---or so you think you do.  Can you define $\exp (x)$?  For what it's worth, you do have the tools to do so at this point, but it's quite likely that someone told you once upon a time that $\e \approx 2.718\ldots $ and then $\exp (x)\coloneqq \e ^x$.  If it's not clear to you at this point that this is just complete and utter nonsense, then apparently I'm not very good at writing mathematical exposition.\footnote{Or maybe you're just stupid.  (Disclaimer:  That was a joke.)}  What you could do, however, is define $\exp$ by its power-series, $\exp (x)\coloneqq \sum _{m\in M}\frac{x^m}{m!}$, but where did that formula come from?  Your ass?  No.  The proper way to define the exponential function is that it is the unique function from $\R$ to $\R$ that (i) is equal to its own derivative and (ii) is $1$ at $0$.\footnote{The unique function that that is equal to its own derivative and is equal to $0$ at $0$ is the function that is everywhere $0$.  $1$ is the next most obvious choice, as opposed to, say $\sqrt{\uppi}$.}  Of course, as always, we can't just go around asserting things like this exist willy-nilly.  Who can even comprehend the chaos that might ensue?  We must \emph{prove} that such a thing exists, and that only one such thing exists.  The theorem that does this for us (and a whole lot more) is Picard's Existence Theorem.
\begin{thm}[Picard's Existence Theorem]\index{Picard's Existence Theorem}\label{PicardsExistenceTheorem}
Let $F:\R \times \R \rightarrow \R$ be a function such that
\begin{enumerate}
\item for each fixed $y\in \R$, the map $x\mapsto F(x,y)$ is continuous; and
\item for each fixed $x\in \R$, the map $y\mapsto F(x,y)$ is lipschitz-continuous,\footnote{The definition is in \cref{BoundedMap} in case you've forgotten (or just plain missed it).}
\end{enumerate}
and let $\coord{x_0,y_0}\in \R \times \R$.  Then, for some $\varepsilon _0$, there is a unique function $f:(x_0-\varepsilon _0,x_0+\varepsilon _0)\rightarrow \R$ such that
\begin{equation}
\frac{\dif}{\dif x}f(x)=F\left( x,f(x)\right) \text{ and }f(x_0)=y_0.
\end{equation}
Furthermore, if for each fixed $x\in \R$, the map $y\mapsto F(x,y)$ is linear, then we may extend $f$ uniquely to a function on all of $\R$.
\begin{rmk}
Picard's theorem is about the existence and uniqueness of solutions to \emph{differential equations}.  The $F$ is supposed to be thought of as the differential equation itself, or at least everything in the differential equation that does not contain a derivative.  For example, in our case of primary interest ($\frac{\dif}{\dif x}f(x)=f(x)$), $F$ will be just $F(x,y)\coloneqq y$.
\end{rmk}
\begin{rmk}
The ``there exists some $\varepsilon _0$'' business is a result of the fact that we may not be able to find a solution to the differential equation on all of $\R$---in general, we can only do so on a neighborhood of where we started ($x=x_0$).  For example, consider the differential equation $\frac{\dif}{\dif x}f(x)=-f(x)^2$ (so for $F(x,y)\coloneqq -y^2$) with initial value $x_0=1=y_0$.  The unique solution will be $f(x)=\frac{1}{x}$, which you cannot extend past $0$ (so that $\varepsilon _0=1$ is the best we can do).
\end{rmk}
\end{thm}
The firepower of Picard allows us to define the exponential function.
\begin{dfn}[Exponential function]\label{ExponentialFunction}
The \emph{exponential function}\index{Exponential function}, $\exp$\index[notation]{$\exp$}, is the unique function $\exp :\R \rightarrow \R$ such that (i) $\frac{\dif}{\dif x}\exp (x)=\exp (x)$ and (ii) $\exp (0)=1$.
\end{dfn}
\begin{exr}
Show that $\lim _{x\to \infty}\exp (x)=\infty$ and $\lim _{x\to -\infty}x^m\exp (x)=0$ for all $m\in \N$.
\end{exr}

Cool, so now we have a function that is its own derivative.  What about a function that is \emph{minus} its own derivative.  Well, now that we have $\exp$ in hand, this is actually really easy.
\begin{exr}
Show that $x\mapsto \exp (-x)$ is the unique function from $\R$ to $\R$ such that (i) is equal to minus its derivative and (ii) is equal to $0$ at $1$.
\end{exr}

What about second derivatives?  Well, of course, we $\exp$ is equal to its own second derivative.  That's easy.  But what about a function that is equal to \emph{minus} its own second derivative?  If we wanted to play the same trick as before, we would need to consider the function $x\mapsto \exp (\alpha x)$ for some number $\alpha$ with $\alpha ^2=-1$, but that's just craziness.  Only looney-toons would ever think about such nonsense.\footnote{But seriously:  why is there no \emph{real} number whose square is $-1$?}  Looks like we're just going to have to come up with a new name for such a function.  ``Cosine'' sounds rad.  Let's call it that.
\begin{dfn}[Cosine and sine]
The \emph{cosine function}\index{Cosine function}, $\cos$\index[notation]{$\cos$}, is the unique function $\cos :\R \rightarrow \R$ such that (i) $\frac{\dif ^2}{\dif x^2}\cos (x)=-\cos (x)$, (ii) $\cos (0)=1$, and (iii) $\frac{\dif}{\dif x}\cos (0)=0$.

The \emph{sine function}\index{Sine function}, $\sin$\index[notation]{$\sin$}, is the unique function $\sin :\R \rightarrow \R$ such that (i) $\frac{\dif ^2}{\dif x^2}\sin (x)=-\sin (x)$, (ii) $\sin (0)=0$, and (iii) $\frac{\dif}{\dif x}\sin (0)=1$.
\begin{rmk}
In the case of $\exp$, because it was defined by a differential equation of \emph{first}-order, only one initial condition was required to uniquely specify it.  In contrast, however, there is \emph{more} than one function equal to minus its own second-derivative (and is $1$ at $0$).  To uniquely specify the function, you must also specify its first-derivative.  $\sin$ and $\cos$ are ``essentially everything'' in the sense that any other function that is equal to minus its own second derivative can be written as a linear combination of these (and furthermore, you can figure out what that linear combination should be by looking at initial conditions at $0$).
\end{rmk}
\end{dfn}

In terms of finding functions that are equal to their own derivatives or minus their own derivatives, we are essentially done.  $\exp$ always works for finding functions equal to its own $n^{\text{th}}$ derivative, and if you want a function equal to \emph{minus} its own $n^{\text{th}}$ derivative, $x\mapsto \exp (-x)$ will work for $n$ odd, and $\cos$ and $\sin$ will work for $n$ even.

\section{Differentiability and continuity}


Having defined the exponential function, we can now tie-up a loose-end from before.
\begin{exm}[A function which is infinitely-differentiable but not continuous]\footnote{This comes from \cite[pg.~116]{Gelbaum}.}\label{exm6.2.15}
Define $f:\R ^2\rightarrow \R$ by
\begin{equation}
f(\coord{x,y})\coloneqq \begin{cases}0 & \text{if }x=0 \\ \frac{\exp \left( -\tfrac{1}{x^2}\right) y}{\exp \left( -\tfrac{2}{x^2}\right) +y^2} & \text{otherwise}\end{cases}.
\end{equation}
Order $\R ^+$ by the usual order and consider the net $\R ^+\ni t\mapsto x_t\coloneqq \coord{t,\exp \left( -\frac{1}{t^2}\right) }\in \R ^2$.  Then, $\lim _{t\to 0^+}x_t=\coord{0,0}$, but
\begin{equation}
f(x_t)=f(\coord{t,\exp (-\tfrac{1}{t^2})})=\frac{\exp (-\tfrac{2}{t^2})}{\exp (-\tfrac{2}{t^2})+\exp (-\tfrac{2}{t^2})}=\tfrac{1}{2}\neq 0=f(\coord{0,0}),
\end{equation}
and so $f$ is not continuous at $\coord{0,0}$.

On the other hand, for $v\coloneqq \coord{v_x,v_y}\in \tangent{\R ^2}{\coord{0,0}}$, $v_x\neq 0$,\footnote{If $v_x=0$, then the $f(\coord{0,0}+\epsilon v)=0$, and so of course the directional derivative in this direction is $0$.} we have
\begin{equation}
\begin{split}
\frac{f(\coord{0,0}+\epsilon v)-f(\coord{0,0})}{\epsilon} & =\frac{\frac{\exp \left( -\tfrac{1}{(\epsilon v_x)^2}\right) (\epsilon v_y)}{\exp \left(-\tfrac{2}{(\epsilon v_x)^2}\right) +(\epsilon v_y)^2}}{\epsilon}=\frac{\exp \left( -\tfrac{1}{(\epsilon v_x)^2}\right) v_y}{\exp \left( -\tfrac{2}{(\epsilon v_x)^2}\right) +(\epsilon v_y)^2} \\
& =\frac{v_y}{\exp \left( -\tfrac{1}{(\epsilon v_x)^2}\right) +(\epsilon v_y)^2\exp \left( \tfrac{1}{(\epsilon v_x)^2}\right) }.
\end{split}
\end{equation}
Hence,
\begin{equation}
\D _vf(\coord{0,0})\coloneqq \lim _{\epsilon \to 0}=\frac{f(\coord{0,0}+\epsilon v)-f(\coord{0,0})}{\epsilon}=0.
\end{equation}
Of course, the map $v\mapsto 0$ is linear.

Thus, $\D _vf(\coord{0,0})$ exists for all $v\in \tangent{\R ^d}{\coord{0,0}}$ \emph{and} the map $v\mapsto D _vf(\coord{0,0})$ is linear, but nevertheless $f$ is not continuous at $\coord{0,0}$.

\begin{exr}
Show that in fact $f$ is infinitely-differentiable.
\begin{rmk}
Note that there is nothing to worry about ever away from $\coord{0,0}$.  As for $\coord{0,0}$, try using an induction argument to show that the power of $\exp \left( \tfrac{1}{(\epsilon v_x)^2}\right)$ will always be greater than the power of the same thing in the numerator of every term of every component of the higher derivatives.
\end{rmk}
\end{exr}
\end{exm}
\begin{exr}
Does there exist an infinitely-differentiable function which is nowhere continuous?
\begin{rmk}
Honestly, I have no fucking idea.  At this rate, probably.  lulz
\end{rmk}
\end{exr}
Fortunately, however, this cannot happen in one-dimension (thank god!).
\begin{prp}\label{prp6.5.1}
Let $f:\R \rightarrow \R$ be a function and let $x\in \R$.  Then, if $f$ is differentiable at $x_0$, then it is continuous at $x_0$.
\begin{proof}
Consider
\begin{equation}
f(x)-f(x_0)=\left( \frac{f(x)-f(x_0)}{x-x_0}\right) (x-x_0).
\end{equation}
Taking the limit of both sides as $x\to x_0$, because $\lim _{x\to x_0}\frac{f(x)-f(x_0)}{x-x_0}=\frac{\dif}{\dif x}'(x_0)$ exists (and is finite), we find that $\lim _{x\to x_0}(f(x)-f(x_0))=0$, so that $f$ is continuous at $x_0$.
\end{proof}
\end{prp}
Be careful though:  $f$ need not be continuous in a neighborhood of $x_0$.
\begin{exm}[A function on $\R$ differentiable at a point and not continuous in a neighborhood of that point]
Define $f:\R \rightarrow \R$ by
\begin{equation}
f(x)\coloneqq \begin{cases}x^2 & \text{if }x\in \Q \\ 0 & \text{if }x\in \Q ^{\comp}\end{cases}.
\end{equation}
\begin{exr}
Show that $f$ is continuous at $0$ and only at $0$.
\end{exr}
\begin{exr}
Show that $f$ is differentiable at $0$ with derivative $0$.
\end{exr}
\begin{rmk}
You might be tempted to ask ``Well, what if it is twice differentiable at a point?''.  The problem with this is, in order to even define the second derivative at the point, the derivative has to exist in a neighborhood of that point,\footnote{Otherwise, the limit as $h\to 0$ of the difference quotient of the derivative (the definition of the second derivative) does not make sense.} in which case it is then automatically continuous on that same neighborhood by the previous result.
\end{rmk}
\end{exm}
\begin{exr}[A function that is not differentiable]
Find an example of a function that is not differentiable.
\end{exr}
\begin{exr}[A function differentiable in one but not all directions]
Find a function $f:\R ^2\rightarrow \R$ that is differentiable along the $x$-axis at the origin, but not along the $y$-axis.
\end{exr}
\begin{exr}[A function differentiable on the standard basis but not differentiable]
Find a function $f:\R ^2\rightarrow \R$ that is differentiable along both the $x$-axis \emph{and} the $y$-axis at the origin, but not in the direction of the vector $\coord{1,1}$.
\begin{rmk}
In particular, it is quite possible for a the partial derivatives\footnote{The \emph{partial derivatives}\index{Partial derivatives} of a function are the directional derivatives in the direction of a standard orthonormal basis vector.} of a function to exist without that function being differentiable.
\end{rmk}
\end{exr}
\begin{exr}[A continuous function that is not differentiable]
Find an example a function that is continuous but not differentiable.
\end{exr}
As a matter of fact, you can even find continuous functions which \emph{aren't differentiable anywhere}!
\begin{exm}[A continuous function that is nowhere-differentiable]\footnote{Proof adapted from \cite{Coleman}.}\label{exm6.3.4}
\begin{savenotes}
Define $f_0:[0,1]\rightarrow \R$ by
\begin{equation}
f_0(x)\coloneqq \begin{cases}x & \text{if }x\in [0,\tfrac{1}{2}] \\ 1-x & \text{if }x\in [\tfrac{1}{2},1]\end{cases}
\end{equation}
and extend $f_0:\R \rightarrow \R$ periodically with period $1$.  For $k\in \N$, now define $f_k:\R \rightarrow \R$ by
\begin{equation}
f_k(x)\coloneqq \tfrac{1}{4^k}f_0(4^kx).
\end{equation}
Then define $s_m:\R \rightarrow \R$ by
\begin{equation}
s_m(x)\coloneqq \sum _{k=0}^mf_k(x).
\end{equation}
We wish to show that $m\mapsto s_m$ is cauchy in $\Mor _{\Top}(\R,\R )$.    Then, because $\Mor _{\Top}(\R ,\R )$ is complete (\cref{thm4.5.6}), we may define $f:\R \rightarrow \R$ by
\begin{equation}
f(x)\coloneqq \sum _{k=0}^\infty f_k(x)
\end{equation}
and will automatically have that $s\in \Mor _{\Top}(\R ,\R )$ (that is, that $f$ is continuous).  We will then show that $f$ is not differentiable anywhere.

We thus verify cauchyness.  For $n>m$, we have
\begin{equation}
\begin{split}
\abs{s_n(x)-s_m(x)} & =\sum _{k=m+1}^nf_k(x)<\sum _{k=m+1}^n\frac{1}{4^k}\leq \sum _{k=m+1}^\infty \frac{1}{4^k}=\frac{1}{1-\tfrac{1}{4}}-\sum _{k=0}^m\frac{1}{4^k} \\
& =\frac{1}{1-\tfrac{1}{4}}-\frac{1-(\tfrac{1}{4})^{m+1}}{1-\tfrac{1}{4}}=\tfrac{4}{3}\frac{1}{4^{m+1}}.
\end{split}
\end{equation}
In fact, for $K\subseteq \R$ quasicompact,
\begin{equation}
\norm[s_n-s_m]_K\leq \norm[s_n-s_m]_{\R}\leq \frac{4}{3}\frac{1}{4^{m+1}},
\end{equation}
and hence the sequence $m\mapsto s_m$ is cauchy in $\Mor _{\Top}(\R ,\R )$, and hence converges to $f$.

We now turn to the proof that $f$ is differentiable nowhere.  So, fix $x\in \R$.  We want to show that $f$ is not differentiable at $\R$.  To prove this, we first prove a small lemma:  we show that there exists a sequence $m\mapsto \epsilon _m$ such that $\abs{\epsilon _m}=\frac{1}{4^{m+1}}$ and
\begin{equation}
\abs{f_m(x+\epsilon _m')-f_m(x)}=\abs{\epsilon _m'}
\end{equation}
whenever $\abs{\epsilon _m'}\leq \abs{\epsilon _m}$.  Define $r_m\coloneqq 4^mx-\lfloor 4^mx\rfloor$, so that $r_m\in [0,1)$.  We define
\begin{equation}
\epsilon _m\coloneqq \tfrac{1}{4^{m+1}}\cdot \begin{cases}1 & \text{if }r\in [0,\tfrac{1}{4})\text{ or }r\in [\tfrac{1}{2},\tfrac{3}{4}) \\ -1 & \text{if }r\in [\tfrac{1}{4},\tfrac{1}{2})\text{ or }r\in [\tfrac{3}{4},1)\end{cases}.
\end{equation}
Obviously $\abs{\epsilon _m}=\frac{1}{4^{m+1}}$.  For the other fact, notice that
\begin{equation}
f_0(4^mx)=\begin{cases}r & \text{if }r_m\in [0,\tfrac{1}{2}) \\ 1-r & \text{if }x\in [\tfrac{1}{2},1)\end{cases},
\end{equation}
Thus,
\begin{equation}
f_0(4^m(x+\epsilon _m))-f_0(4^mx)=f_0(r_m+4^m\epsilon _m)-f_0(r_m)=f_0(r_m\pm \tfrac{1}{4})-f_0(r_m).
\end{equation}
The sign was purposefully chosen so that both $r_m$ and $r_m\pm \frac{1}{4}$ lie in the same `half' of the interval $[0,1]$, so that the same `rule' in the definition of $f_0$ applies.  In fact, the same is of course true if $\pm \tfrac{1}{4}$ replaced by any $4^m\epsilon _m'$ with $\abs{\epsilon _m'}\leq \epsilon _m$.  Thus, in either case, for such an $\epsilon _m'$, we have that
\begin{equation}
f_0(4^m(x+\epsilon _m'))-f_0(4^mx)=4^m\epsilon _m'.
\end{equation}
From this it follows that
\begin{equation}
\abs{f_m(x+\epsilon _m')-f_m(x)}=\tfrac{1}{4^m}\abs{f_0(4^m(x+\epsilon _m'))-f_0(4^mx)}=\tfrac{1}{4^m}\cdot 4^m\abs{\epsilon _m'}=\abs{\epsilon _m'}
\end{equation}

Take $n>m$.  Then,
\begin{equation}
f_n(x+\epsilon _m)=\tfrac{1}{4^n}f_0(4^n(x+\epsilon _m))=\tfrac{1}{4^n}f_0(4^nx\pm 4^{n-(m+1)})=\footnote{Because $f_0$ has period $1$ and $n\geq m+1$.}\tfrac{1}{4^n}f_0(4^nx)=f_n(x)
\end{equation}
Thus,
\begin{equation}
f(x+\epsilon _m)-f(x)=\sum _{k=0}^\infty [f_k(x+\epsilon _m)-f_k(x)]=\sum _{k=0}^m[f_k(x+\epsilon _m)-f_k(x)].
\end{equation}
On the other hand, for $n\leq m$,
\begin{equation}
f_n(x)=\tfrac{1}{4^n}f_0(f^nx)=\tfrac{4^{m-n}}{4^n}f_0\left( 4^m\frac{x}{4^{m-n}}\right) =4^{m-n}f_m(\tfrac{1}{4^{m-n}}x),
\end{equation}
and so
\begin{equation}\label{5.3.11}
\left| f_n(x+\epsilon _m)-f_n(x)\right| =4^{m-n}\left| f_n\left( \frac{x}{4^{m-n}}+\frac{h_m}{4^{m-n}}\right) -f_n\left( \tfrac{x}{4^{m-n}}\right) \right| =\footnote{Because, for $n\leq m$, $\abs{\frac{\epsilon _m}{4^{m-n}}}\leq \abs{\epsilon _m}$.}4^{m-n}\left| \frac{\epsilon _m}{4^{m-n}}\right|=\abs{\epsilon _m}
\end{equation}
Thus,
\begin{equation}
\Delta _m\coloneqq \frac{f(x+\epsilon _m)-f(x)}{\epsilon _m}=\sum _{k=0}^m\frac{f_k(x+\epsilon _m)-f_k(x)}{\epsilon _k}.
\end{equation}
By \eqref{5.3.11}, each term in this sum is $\pm 1$, so that the difference quotient $\frac{f(x+\epsilon _m)-f(x)}{\epsilon _m}$ is an integer.  Let $n_m^+$ denote the number of $+1$s that appear in this finite sum and similarly let $n_m^-$ denote the number of $-1$s that appear in this sum, so that $\Delta _m=n_m^+-n_m^-$ and $m=n_m^++n_m^-$.  Hence, $\Delta _m=m-2n_m^-$.  Thus, the parity of $\Delta _m$ is constant, and so cannot converge.  Thus, this limit, the derivative, does not exist at $x$.
\end{savenotes}
\end{exm}

It is often said that a function on a closed interval that has a continuous derivative is lipschitz-continuous.  One should be careful, however, because in our language, this is \emph{not} true.
\begin{exm}[A function on a closed interval with continuous derivative that is not lipschitz-continuous]
Define $f:[0,1]\rightarrow \R$ by
\begin{equation}
f(x)\coloneqq \begin{cases}x\sin (\tfrac{1}{x}) & \text{if }x\neq 0 \\ 0 & \text{if }x=0\end{cases}.
\end{equation}
Then, $f$ is differentiable on $[0,1]$ with derivative\footnote{Remember, we only define the derivative at \emph{interior} points of domain, and $f$ is in fact \emph{smooth} on $(0,1)$.}
\begin{equation}
\frac{\dif}{\dif x}f(x)=\sin (\tfrac{1}{x})-\tfrac{1}{x}\cos (\tfrac{1}{x}).
\end{equation}
\begin{exr}
From here, show that $f$ is not lipschitz-continuous on $[0,1]$.
\end{exr}
\end{exm}
On the other hand, if the derivative \emph{extends} to a continuous function on the entire domain, then we are good to go.
\begin{exr}
Let $K\subseteq \R ^d$ be quasicompact and let $f:K\rightarrow \R$.  Show that if $f$ has a derivative which extends to a continuous function on $K$, then $f$ is lipschitz-continuous.
\end{exr}

The derivative (as a function on the tangent space) is itself continuous (hence uniformly-continuous because continuous homomorphisms of topological groups are uniformly-continuous (\cref{prpB.10}).
\begin{exr}
Show that the map $\tangent{\R ^d}{x}\ni v^a\mapsto v^a\nabla _af(x)\in \R$ is continuous\footnote{Recall that $\tangent{\R ^d}{x}$ is a \emph{metric} vector space, with the metric being the dot product.  This metric induces a norm (the usual euclidean norm), which in turn induces a metric, which in turn induces a uniformity, which in turn induces a topology.} if $f$ is differentiable at $x$.
\end{exr}

\section{The Mean Value Theorem}

One of the most important theorems of traditional calculus is of course the \emph{Mean Value Theorem}.  We first prove a couple of results, important in their own right, leading up to it.
\begin{dfn}[Local extrema]\label{LocalExtrema}
Let $f:X\rightarrow \R$ be a function from a topological space into $\R$ and let $x_0\in X$.  Then, $x_0$ is a \emph{local maximum}\index{Local maximum} of $f$ iff there is some open neighborhood $U$ of $x_0$ such that (i) $f(x_0)=\sup _{x\in U}\{ f(x)\}$ and (ii) $x_0$ is the \emph{unique} such point.\footnote{So for example, constant functions have no local maxima.}  $x_0$ is a \emph{local minimum}\index{Local minimum} of $f$ iff there is some open neighborhood $U$ of $x_0$ such that (i) $f(x_0)=\inf _{x\in U}\{ f(x)\}$ and (ii) $x_0$ is the \emph{unique} such point.  $x_0$ is a \emph{local extremum}\index{Local extremum} of $f$ iff it is either a local maximum or a local minimum.
\end{dfn}
\begin{prp}[First Derivative Test]\index{First Derivative Test}\label{FirstDerivativeTest}
Let $S\subseteq \R ^d$, $f:S\rightarrow \R$ and let $x_0\in \R ^d$.  Then, if $f$ is differentiable at $x_0$ and $x_0$ is a local extremum of $f$, then $\nabla _af(x)=0$.
\begin{rmk}
Sometimes this is also referred to as \emph{Fermat's theorem}\index{Fermat's theorem}.  Evidently, it was not the last result of his life.
\end{rmk}
\begin{proof}
Suppose that $f$ is differentiable at $x_0$ and $x_0$ is a local extremum of $f$.  Let $U$ be an open neighborhood about $x_0$ so that $x_0$ is either a maximum or minimum in $U$.  We proceed by contradiction:  suppose that $\nabla _af(x_0)\neq 0$.  Then, there is some $v^a\in \tangent{\R ^d}{x_0}$ such that $v^a\nabla _af(x_0)\neq 0$.  Without loss of generality, suppose that $v^a\nabla _af(x_0)>0$.  Then, there is some $h>0$ such that $x_0+hv,x_0-hv\in U$ and
\begin{equation}
\frac{f(x_0+hv)-f(x_0)}{h}\eqqcolon K>0\text{ and }\frac{f(x_0-hv)-f(x_0)}{-h}\eqqcolon L>0,
\end{equation}
and so
\begin{equation}
f(x_0+hv)=f(x_0)+hK>f(x_0)\text{ and }f(x_0-hv)=f(x_0)-hL<f(x_0),
\end{equation}
so that $f(x_0)$ can be neither a maximum nor a minimum in $U$:  a contradiction.
\end{proof}
\end{prp}

\begin{thm}[Mean Value Theorem]\index{Mean Value Theorem}\label{MeanValueTheorem}
Let $f:[a,b]\rightarrow \R$ be continuous and differentiable.\footnote{Recall that (see the definition, \cref{Derivative}) a function being differentiable means that it is differentiable on its interior (we only defined the derivative for interior points).  Thus, while it may be the case that differentiability at a point implies continuity at a point, saying that $f$ is differentiable here only implies continuity on the interior of the domain, $(a,b)$.  But in fact, we \emph{do} need continuity on all of $[a,b]$---see the next example.}  Then, there is some $c\in (a,b)$ such that
\begin{equation}
\tfrac{\dif}{\dif x}f(x)=\frac{f(b)-f(a)}{b-a}.
\end{equation}
\begin{rmk}
The right-hand side of this equation is called the \emph{average slope}\index{Average slope} of $f$ on $[a,b]$.
\end{rmk}
\begin{rmk}
This essentially does not generalize at all to higher dimensions.  It just fails outright if you increase the dimension of the codomain---see \cref{exm6.4.8}.  On the other hand, there is the question of what should the statement of the Mean Value Theorem even be if you increase the dimension of the domain---what is the average slope over a square, for example?  What you can do is state the result in terms of curves between points in $\R ^d$, but this is really just the one-dimensional statement here for the composition of the function with the curve (which of course itself is a function from a closed interval in $\R$ to $\R ^d$).
\end{rmk}
\begin{proof}
Define $g:[a,b]\rightarrow \R$ by
\begin{equation}
g(x)\coloneqq f(x)-\left( \frac{f(b)-f(a)}{b-a}\right) (x-a).
\end{equation}
The first thing to notice is that we want to find some $c\in (a,b)$ such that $\frac{\dif}{\dif x}g(c)=0$.

$g$ is continuous on $[a,b]$, and therefore, by the \nameref{ExtremeValueTheorem}, attains a maximum and a minimum somewhere on $[a,b]$.  Suppose that both the maximum and minimum were achieved at the endpoints.  As $g(a)=f(a)$ and $g(b)=f(a)$, this would imply that $f(a)$ was both a minimum and a maximum for $g$ on $[a,b]$, which in turn would imply that $g$ is constant on $[a,b]$.  In particular, $\frac{\dif}{\dif x}g(c)=0$ for any $c\in (a,b)$.  On the other hand, if at least one of the maximum and the minimum where attained at $c\in (a,b)$, then it would be a local extremum, and so by the \nameref{FirstDerivativeTest}, we would have that $\frac{\dif}{\dif x}g(c)=0$.
\end{proof}
\end{thm}

\begin{exm}[A function which is differentiable on ${[a,b]}$ for which the \nameref{MeanValueTheorem} fails]
Define $f:[0,1]\rightarrow \R$ by
\begin{equation}
f(x)\coloneqq \begin{cases}0 & \text{if }x=0 \\ 1 & \text{otherwise}\end{cases}.
\end{equation}
$f$ is differentiable on $[0,1]$ because its derivative exists at every point in the interior of $[0,1]$ (it is always $0$ of course).  On the other hand, the average slope of this function is $\frac{1-0}{1-0}=1$.
\end{exm}
\begin{exm}[A function continuous and differentiable on ${[a,b]}$ into $\R ^2$ for which the \nameref{MeanValueTheorem} fails]\label{exm6.4.8}
Define $f_1,f_2:[-1,1]\rightarrow \R$ by
\begin{equation}
f_1(x)\coloneqq x^2\text{ and }f_2(x)\coloneqq x^3
\end{equation}
and $f:[-1,1]\rightarrow \R ^2$ by
\begin{equation}
f(x)\coloneqq \coord{f_1(x),f_2(x)}.
\end{equation}
We show there is no $c\in (-1,1)$ such that
\begin{equation}\label{6.4.10}
\frac{\dif}{\dif x}f(c)=\frac{f(1)-f(-1)}{1-(-1)}.
\end{equation}
$f_1(-1)=1$ and $f_1(1)=1$, and on the other hand, $f_2(-1)=-1$ and $f_2(1)=1$.  Therefore, the right-hand side of this equation becomes
\begin{equation}
\frac{f(1)-f(-1)}{1-(-1)}=\coord{0,1}.
\end{equation}
If \eqref{6.4.10} were to hold, then we must have that $f_1'(c)=0$, which forces $c=0$.  But then, $f_2'(c)=0\neq 1$.
\end{exm}

\begin{exr}
Let $f:[a,b]\rightarrow \R$ be continuous and differentiable.  Show that the following are true.
\begin{enumerate}
\item If $\frac{\dif}{\dif x}f(c)=0$ for all $c\in (a,b)$, then $f$ is constant on $[a,b]$.
\item if $\frac{\dif}{\dif x}f(c)>0$ for all $c\in (a,b)$, then $f$ is increasing on $[a,b]$.
\item If $\frac{\dif}{\dif x}f(c)<0$ for all $c\in (a,b)$, then $f$ is decreasing on $[a,b]$.
\end{enumerate}
\end{exr}
\begin{exr}
Let $f:[a,b]\rightarrow \R$ and let $c\in (a,b)$.  Then, if $f$ is differentiable at $c$ and $\frac{\dif}{\dif x}f(c)>0$, is $f$ necessarily increasing in a neighborhood of $c$?
\end{exr}
Be careful though:  we can deduce nothing if the derivative is positive at a point alone.
\begin{exm}[A point at which the derivative is positive but has no neighborhood in which the function is increasing]
Define $f:\R \rightarrow \R$ by
\begin{equation}
f(x)\coloneqq \begin{cases}x-\tfrac{1}{4} & \text{if }x\in \Q \\ x^2 & \text{if }x\in \Q ^{\comp}\end{cases}.
\end{equation}
We first show that the derivative at $\frac{1}{2}$ is $1$.
\begin{equation}
\frac{f(\tfrac{1}{2}+\epsilon )-f(\tfrac{1}{2})}{\epsilon}=\begin{cases}\frac{\left( (\tfrac{1}{2}+\epsilon )-\tfrac{1}{4}\right) -\tfrac{1}{4}}{\epsilon} & \text{if }\epsilon \in \Q \\ \frac{\left( \tfrac{1}{2}+\epsilon \right) ^2-\tfrac{1}{4}}{\epsilon} & \text{if }x\in \Q ^{\comp}\end{cases}=\begin{cases}1 & \text{if }\epsilon \in \Q \\ 1+\epsilon & \text{if }\epsilon \in \Q ^{\comp}\end{cases}.
\end{equation}
Hence, we do indeed have that
\begin{equation}
\lim _{\epsilon \to 0}\frac{f(\tfrac{1}{2}+\epsilon )-f(\tfrac{1}{2})}{\epsilon}=1>0.
\end{equation}
On the other hand, let $U$ be any open neighborhood of $\frac{1}{2}$ and let $\varepsilon >0$ be \emph{rational} and such that $\varepsilon +\frac{1}{2}\in U$.  Now choose $0<\delta <\varepsilon$ \emph{irrational} and such that $\delta +\delta ^2>\varepsilon$.  Then, despite the fact that $\frac{1}{2}+\delta <\frac{1}{2}+\varepsilon$, we nevertheless have that
\begin{equation}
f(\tfrac{1}{2}+\delta )=(\tfrac{1}{2}+\delta )^2=\tfrac{1}{4}+\delta +\delta ^2>\tfrac{1}{4}+\varepsilon =f(\tfrac{1}{4}+\varepsilon ).
\end{equation}
\end{exm}

\section{The Inverse Function Theorem}

We have talked about functions which take values in $\R ^m$ before, $f:\R ^d\rightarrow \R ^m$, and we have interpreted them as vector fields on $\R ^d$.  I don't want to do this now.  Now, the codomain is going to play what will be the role of a general manifold when you generalize, so for the time being I will write $M\coloneqq \R ^m$ to be suggestive of this.
\begin{displayquote}
When you generalize to manifolds and you have a smooth function $f:M\rightarrow N$ between manifolds, then for each $x\in X$ the derivative becomes a linear map at the tangent space $\tangent{M}{x}\ni v^a\mapsto v^b\nabla _bf(x)^\mu \in \tangent{N}{f(x)}$.  (Once again, the $\mu$ index indicates that this vector lives in a different vector space than $v^a$ does.)  Unfortunately, in the very special case of $f:\R ^d\rightarrow \R ^m$, because $\tangent{M}{x}\cong _{\Vect _{\R}}\R ^d$ is `the same as'\footnote{Or so the symbols would suggest.} $M$ and $\tangent{N}{f(x)}\cong _{\Vect _{\R}}\R ^m$ is `the same as' $N$, the `same' thing gets treated in very different ways in different, and this can make things very confusing.  For example, $f(x)\in \R ^m$ the metric space\footnote{If we could, we should regard it as a manifold, but we don't know what manifolds are, and a metric space is the next best thing (that we have at our disposal).}, but $v^a\nabla _af(x)^\mu \in \R ^m$ the vector space (namely, the tangent space of $\R ^m$ at $f(x)$).
\end{displayquote}
Thus, for a smooth map $f:\R ^d\rightarrow M$, at each point, $\nabla _af(x)^\mu \in \Mor _{\Vect _{\R}}(\tangent{\R ^d}{x},\tangent{M}{f(x)]}$, the linear map being defined as $v^a\mapsto v^b\nabla _bf(x)^\mu$.

For the special case $d=m$, something a little be special happens:  we can then ask whether the derivative is invertible at each point or not (regarding as a linear map between tangent spaces \emph{of the same dimension}).  An incredibly important fact, known as the \emph{Inverse Function Theorem}, is that, if a function is smooth and the derivative is invertible at a point, then the function itself is \emph{invertible with smooth inverse} in a neighborhood of that point.  You might say that infinitesimal invertibility implies local invertibility.
\begin{thm}[Inverse Function Theorem]\index{Inverse Function Theorem}\label{InverseFunctionTheorem}
Let $S\subseteq \R ^d$, let $f^a:S\rightarrow \R ^d$ be smooth, and let $x\in \Int (S)$.  Then, if $\nabla _af(x)^b:\tangent{\R ^d}{x}\rightarrow \tangent{\R ^d}{f(x)}$ is invertible, then there exists an open neighborhood $U$ of $x$ such that $\restr{f}{U}$ is invertible with smooth inverse.
\begin{rmk}
Bijective functions which are smooth and have smooth inverse will be called \emph{diffeomorphisms}\index{Diffeomorphism} when you pass to the category of manifolds (they wind-up being of course the isomorphisms in the category of manifolds).  Thus, we say that if a smooth map has a derivative that is invertible at a point, then it is a \emph{local diffeomoprhism}\index{Local diffeomorphism}.
\end{rmk}
\end{thm}

\section{A smooth version of $\Mor _{\Top}(\R ^d,\R )$}

Our goal in this section is to find a notion of convergence (i.e.~a topology) that has the property that a net of \emph{smooth} functions converges to another smooth function.  The statement that $\Mor _{\Top}(\R ^d,\R )$ is complete from the last chapter (\cref{thm4.5.6}) says that continuous functions which converge in $\Mor _{\Top}(\R ^d,\R )$ (i.e.~converges uniformly on quasicompact subsets) converge to a continuous function.\footnote{In fact, it says more than this---if your net is cauchy, it has a limit, and that limit is continuous.}  Unfortunately, this fails \emph{miserably} if we replace ``continuous'' with the word ``smooth''.

\begin{exm}[A sequence of smooth functions which converges in $\Mor _{\Top}(\R ,\R )$ to a function which is nowhere-differentiable]

\end{exm}

\begin{thm}[Clairut's Theorem]\index{Clairut's Theorem}\label{ClariutsTheorem}
Let $S\subseteq \R ^d$,let $x\in S$, and let $f:S\rightarrow \R$.  Then, if $f$ has a second derivative at $x$, then, $\nabla _a\nabla _b(x)f=\nabla _b\nabla _af(x)$.
\begin{rmk}
If $f$ is smooth, by applying this result inductively, it follows that all mixed partials agree.
\end{rmk}
\begin{rmk}
If you've seen this before, you might recall needing the hypothesis that the second derivative actually be continuous at $x$.  This is not correct.  The correct statement is that if all the second \emph{partial} derivatives exist and are continuous there, then they all agree.  The assumption that $\nabla _a\nabla _bf(x)$ exist is stronger than the assumption that only the second partial exists (which amounts to the assumption that $v^aw^b\nabla _a\nabla _bf(x)$ exists for only \emph{some} $v^a$ and $w^a$), and it turns out, as this theorem shows, that it is in fact strong enough to prove that the covariant $2$-tensor $\nabla _a\nabla _bf(x)$ is symmetric.
\end{rmk}
\begin{rmk}
This applies equally well for arbitrary tensors of course, with the usual proof or just reducing it to this case.
\end{rmk}
\begin{rmk}
Warning:  This fails in general, even for smooth functions, if there is curvature on a riemannian manifold!  As a matter of fact, the curvature tensor is \emph{defined} by this lack of commutativity.
\end{rmk}
\begin{proof}
The first thing to note is that, as derivatives are only defined at interior points, the mere \emph{existence} of the second derivative implies that the first derivative must exist in a neighborhood $U$ of $x$.  In turn, we then know that $f$ itself must be continuous on $U$.

Let $v^a,w^a\in \R ^d$.  Then,
\begin{equation}
\begin{split}
[v^a\nabla _a(w^b\nabla _bf)](x) & =\lim _{\epsilon _v\to 0}\frac{w^b\nabla _bf(x+\epsilon _vv)-w^b\nabla _bf(x)}{\epsilon _v} \\
& =\lim _{\epsilon _v\to 0}\lim _{\epsilon _w\to 0}\frac{f(x+\epsilon _v+\epsilon _ww)-f(x+\epsilon _vv)-f(x+\epsilon _ww)+f(x)}{\epsilon _v\epsilon _w}.
\end{split}
\end{equation}
For $\epsilon _w\in \R$, define $g_{\epsilon _w}:U_{h_w}\rightarrow \R$ by
\begin{equation}
g_{\epsilon _w}(\epsilon )\coloneqq f(x+\epsilon _ww+\epsilon v)-f(x+\epsilon v),
\end{equation}
where $U_{\epsilon _w}\subseteq \R $ is an open neighborhood of $0$ chosen so that $x+\epsilon _w+\epsilon v\in U$ is in the domain of $f$ for all $\epsilon \in \Cls (U_{\epsilon _w})$.  In fact, by making $U_{\epsilon _w}^1$ smaller if necessary, way may without loss of generality assume that it is an open interval containing $0$.  Using this definition, we have
\begin{equation}
[v^a\nabla _a(w^b\nabla _bf)](x)=\lim _{\epsilon _v\to 0}\lim _{\epsilon _w\to 0}\frac{1}{\epsilon _w}\frac{g_{\epsilon _w}(\epsilon _v)-g_{\epsilon _w}^1(0)}{\epsilon _v}.
\end{equation}
$g_{\epsilon _w}^1$ is continuous on the closed interval $\Cls (U_{\epsilon _w})$ and differentiable on its interior, and so by the \nameref{MeanValueTheorem} there is some $c_{\epsilon _v}\in (0,\epsilon _v)$ such that
\begin{equation}
\tfrac{\dif}{\dif \epsilon}g_{\epsilon _w}(c_{\epsilon _v})=\frac{g_{\epsilon _w}(\epsilon _v)-g_{\epsilon _w}(0)}{\epsilon _v},
\end{equation}
so that
\begin{equation}
\begin{split}
[v^a\nabla _a(w^b\nabla _bf)](x) & =\lim _{\epsilon _v\to 0}\lim _{\epsilon _w\to 0}\frac{1}{\epsilon _w}\tfrac{\dif}{\dif \epsilon}g_{\epsilon _w}(c_{\epsilon _v}) \\
& =\lim _{\epsilon _v\to 0}\lim _{\epsilon _w\to 0}\frac{v^a\nabla _af(x+\epsilon _ww+c_{\epsilon _v}v)-v^a\nabla _af(x+c_{\epsilon _v})}{\epsilon _w}.
\end{split}
\end{equation}
Similarly as before, for $\epsilon _v\in \R$, define\footnote{For what it's worth, this is what ultimately motivated me to change from $h$s to $\epsilon$s---at first I tried writing $g_{h_w}^1$ and $g_{h_v}^2$, which works, but\textellipsis ew.  Being able to use $h$ is so much better.} $h_{\epsilon _v}:V_{\epsilon _v}\rightarrow \R$ by
\begin{equation}
h_{\epsilon _v}(\epsilon )\coloneqq v^a\nabla _af(x+c_{\epsilon _v}v+\epsilon w).
\end{equation}

Continuity of the second derivative means in particular that
\begin{equation}
\lim _{\lambda \to 0}v^aw^b\nabla _a\nabla _bf(x+x_\lambda )=v^aw^b\nabla _a\nabla _bf(x)
\end{equation}
for any net $\Lambda \ni \lambda \mapsto x_\lambda \in \R ^d$ such that $\lim _\lambda x_\lambda =0$.  Hence,
\begin{equation}
\begin{split}
v^aw^b\nabla _a\nabla _bf(x) & =[v^a\nabla _a(w^b\nabla _bf)](x)=\lim _\lambda [v^a\nabla _a(w^b\nabla _bf)](x+x_\lambda ) \\
& =\lim _\lambda \lim _{h_v\to 0}\left[ \frac{1}{h_v}\left( [w^b\nabla _bf](x+x_\lambda +h_vv)\right. \right. \\
& \qquad \left. \left. -[w^b\nabla _bf](x+x_\lambda )\right) \right] \\
& =\lim _\lambda \lim _{h_v\to 0}\lim _{h_w\to 0}\left[ \frac{1}{h_vh_w}\left( f(x+x_\lambda +h_vv+h_ww)\right. \right. \\
& \qquad \left. \left. -f(x+x_\lambda +h_vv)-f(x+x_\lambda +h_ww)\right. \right. \\
& \qquad \left. \left. +f(x+x_\lambda )\right) \right] .
\end{split}
\end{equation}
Now take $\Lambda \coloneqq $
The inside of the limit is completely $h_v\leftrightarrow h_w$ symmetric, and so by simply renaming $h_v$ to be $h_w$ and $h_w$ to be $h_v$, the inside of the limit stays the same but the limits out front change from $\lim _{h_v\to 0}\lim _{h_w\to 0}$ to $\lim _{h_w\to 0}\lim _{h_v\to 0}$.  Then, by the exact same equations as above with $v\leftrightarrow w$, we obtain that this is equal to $v^aw^b\nabla _b\nabla _af(x)$.
\end{proof}
\end{thm}
So, first things first---fuck the riemann integral.  Seriously.  The only argument pro-riemann-integral is that it is easier.  What a ridiculous argument.  This is math, dude.  If you choose to do things because they're easy, you're in the wrong subject.  Moreover, I would argue that this is not even true---if you set things up right, you can literally \emph{define} the (lebesgue) integral to be the area (measure) under the curve.  Or, if you prefer, you can take a limit over the size of a partition of the sum of the areas of the rectangles corresponding to the subsets of the partition (the riemann integral).  Are you really going to sit here and try to argue that this is easier to teach?  I call bullshit.  And besides, if you're going to become a mathematician, you have to learn the lebesgue integral at some point anyways\textellipsis why learn something only to have to relearn it later?

Okay, so now that my rant is out of the way, let's actually do some mathematics.

\section{Measure theory}

All of integration theory ultimately boils down to measure theory.  The definition of the integral itself is relatively easy.  In fact, the definition of abstract measure spaces is even easier.  There's really no question that writing down the definition of the lebesgue integral is \emph{significantly} easier than that of the riemann integral.  What is a bit tricky, however, is constructing specific measures.  In our case, we will primarily be concerned with constructing lebesgue measure (on $\R ^d$), and this is really the only part that is a bit tricky.  Before we get there though, we will present the theory of measure spaces, and then get to lebesgue measure in the next section.

The intuition behind measure is actually quite easy---a measure is just an axiomatization of our intuition about notion of things like length, area, and volume.  Before we define a measure, it will be convenient to introduce a couple of terms.

\subsection{Basic terminology}

\begin{dfn}
Let $X$ be a set, let $\meas :2^X\rightarrow [0,\infty ]$, and let $\mathcal{M}\subseteq 2^X$.  
\begin{enumerate}
\item If $\mathcal{U}$ is a cover of $X$ on which $\meas$ is constant, then we say that $\mathcal{U}$ is \emph{uniformly-measurable}\index{Uniformly-measurable cover} with respect to $\meas$ and we denote by $\meas (\mathcal{U})$ its constant value on $\mathcal{U}$;
\item $\meas$ is \emph{subadditive}\index{Subadditive} on $\mathcal{M}$ iff for $\{ M_m:m\in \N \} \subseteq \mathcal{M}$ we have
\begin{equation}\label{5.1.2}
\meas \left( \bigcup _{m\in \N}M_m\right) \leq \sum _{m\in \N}\meas (M_m);
\end{equation}
\item $\meas$ is \emph{additive}\index{Additive (measure)} on $\mathcal{M}$ iff for $\{ M_m:m\in \N \} \subseteq \mathcal{M}$ a \emph{disjoint} collection we have
\begin{equation}\label{5.1.3}
\meas \left( \bigcup _{m\in \N}M_m\right) =\sum _{m\in \N}\meas (M_m).
\end{equation}
\end{enumerate}
$\meas$ is simply just subadditive (resp.~additive) if it is subadditive (resp.~additive) on all of $X$.
\end{dfn}
\begin{exr}\label{exr5.1.4}
Show that if $\meas$ is additive on $\mathcal{M}$ then it is subadditive on $\mathcal{M}$.
\begin{rmk}
There is something to show here.  While \eqref{5.1.3} itself is obviously a stronger condition than \eqref{5.1.2}, it is also only assumed for \emph{disjoint} collections.  The problem then is to show that, if \eqref{5.1.3} holds for disjoint collections, then \eqref{5.1.2} holds for \emph{all} collections.
\end{rmk}
\end{exr}
\begin{dfn}[Boolean algebra]
A \emph{boolean algebra}\index{Boolean algebra} is a set $X$ equipped with two binary operations, $\vee$ and $\wedge$, a unary operation $\neg$, and identities for $\vee$ and $\wedge$, $0$ and $1$ respectively, such that
\begin{enumerate}
\item $\coord{X,\vee ,0}$ and $\coord{X,\wedge ,1}$ are both monoids;
\item $\vee$ distributes over $\wedge$ and $\wedge$ distributes over $\vee$; and
\item $x\vee \not x=1$ and $x\wedge \not x=0$.
\end{enumerate}
\begin{rmk}
The example you should keep in mind, and indeed, the only example of relevance to us (together with its subalgebras) is $\coord{2^X,\cup ,\cap ,\blank ^{\comp},\emptyset ,X}$.
\end{rmk}
\end{dfn}
\begin{exr}
Let $\mathcal{M}\subseteq 2^X$.  Show that $\mathcal{M}$ is a subalgebra\footnote{That is, it itself is a boolean algebra with respect to union, intersection, and complementation.} of $2^X$ iff it is closed under finite union and complementation.
\end{exr}
A construction we'll be making use of a lot is the boolean algebra of sets \emph{generated} by a given collection.
\begin{prp}
Let $\mathcal{S}\subseteq 2^X$.  Then, there exists a unique subalgebra $\mathcal{M}$ of $2^X$, the boolean algebra \emph{generated by} $\mathcal{S}$, such that
\begin{enumerate}
\item $\mathcal{S}\subseteq \mathcal{M}$; and
\item if $\mathcal{M}'$ is any other subalgera containing $\mathcal{S}$, it follows that $\mathcal{M}\subseteq \mathcal{M}'$.
\end{enumerate}
Furthermore, $\mathcal{M}$ is the collection of all sets that can be written as a finite union of finite intersections of sets coming from $\mathcal{S}$ or complements of sets in $\mathcal{M}$.
\begin{proof}
We leave this as an exercise.
\begin{exr}
Complete the proof yourself.
\end{exr}
\end{proof}
\end{prp}

\subsection{Main definitions}

\begin{dfn}[Outer measure]\label{OuterMeasure}
Let $X$ be a set.  An \emph{outer measure}\index{Outer measure} on $X$ is a function $\meas :2^X\rightarrow [0,\infty ]$ such that
\begin{enumerate}
\item $\meas (\emptyset )=0$;
\item (Nondecreasing)\label{Measure.Monotonicity} $\meas :\coord{2^X,\subseteq}\rightarrow [0,\infty ]$ is nondecreasing;\footnote{Concretely, this means that $\meas (S)\leq \meas (T)$ if $S\subseteq T$.}; and
\item (Subadditivity) $\meas$ is subadditive.
\end{enumerate}
\begin{rmk}
Note that we allow the measure of sets to be infinite.  This is incredibly important---for example, we will want $\meas (\R )=\infty$ (for lebesgue measure anyways).
\end{rmk}
\begin{rmk}
As a consequence of this, we needn't worry about convergence in the third axiom (see \eqref{5.1.2}).  As a matter of fact, we definitely want to allow this sum to diverge---think about what the measure of $\bigcup _{m\in \Z}(m,m+1)$ should be.
\end{rmk}
\end{dfn}
\begin{dfn}[Uniform measure]\label{Measure}
Let $\coord{X,\widetilde{\mathcal{U}}}$ be a uniform space.  A \emph{uniform measure}\index{Uniform measure} on $X$ is an outer measure on $X$ for which there exists a uniform base $\widetilde{\mathcal{B}}$ of $X$ consisting of uniformly-measurable covers that is additive on the boolean algebra generated by $\bigcup _{\mathcal{B}\in \widetilde{\mathcal{B}}}\mathcal{B}$.  $\widetilde{\mathcal{B}}$ is a \emph{uniformly-measurable} base for $\coord{X,\meas}$.
\begin{rmk}
Think about what having a uniform base of uniformly-measurable covers means for a metric space---if we take as a uniform base the collection of all covers by $\varepsilon$-balls, then this is just the statement that every $\varepsilon$-ball has to have the same measure.
\end{rmk}
\begin{rmk}
Note that you definitely do not want to require \emph{every} uniform cover be uniformly-measurable.  For example, in a metric space, by upward-closedness the collection of all $\varepsilon$-balls together with a single $2\varepsilon$-ball will also be a uniform-cover---we definitely do not want to require that a $2\varepsilon$ ball has the same measure as an $\varepsilon$-ball.
\end{rmk}
\end{dfn}
\begin{displayquote}
At some point in the near future, we will be doing arithmetic with $\infty$---for example, what should the measure of $\R \times \{ 0\}$ in $\R ^2$ be?  Of course, from our definition of product measures, this will turn out to be $\infty \cdot 0$.  We hence declare that
\begin{equation}
\infty \cdot 0\coloneqq 0\eqqcolon 0\cdot \infty .
\end{equation}
There are other arithmetic notions we have to technically define (e.g.~$x+\infty=\infty$), but this is the only nonobvious one.
\end{displayquote}
\begin{exm}[The zero measure]
Let $\coord{X,\widetilde{\mathcal{U}}}$ be a uniform space and define $\meas :2^X\rightarrow [0,\infty ]$ by $\meas (S)\coloneqq 0$.  How terribly interesting.
\end{exm}
\begin{exm}[The infinite measure]
Let $\coord{X,\widetilde{\mathcal{U}}}$ be any uniform space, let $\widetilde{\mathcal{B}}$ be a uniform base no cover of which contains the empty-set, and define $\meas :2^X\rightarrow [0,\infty ]$ by
\begin{equation}
\meas (S)\coloneqq \begin{cases}0 & \text{if }S=\emptyset \\ \infty & \text{otherwise}\end{cases}.
\end{equation}
Dear god, this example is even more interesting than the last one.
\end{exm}
\begin{exm}[The counting measure]
Let $\coord{X,\widetilde{\mathcal{U}}}$ be a discrete uniform space (so that the set which a single cover, the cover by singletons, forms a uniform base), and for $S\subseteq X$ define $\meas (S)\coloneqq \abs{S}$, that is, the cardinality of $S$.
\begin{rmk}
This is actually incredibly important, as we shall see that sums are just integrals with respect to the counting measure.
\end{rmk}
\end{exm}
Before we get to any examples more interesting than this, we will first have to develop a bit of theory.
\begin{prp}\label{prp5.1.9}
Let $\meas$ be a measure on a uniform space $X$ with uniformly-measurable base $\widetilde{\mathcal{B}}$ and let $\mathcal{B},\mathcal{C}\in \widetilde{\mathcal{B}}$.  Then, if $\mathcal{B}\preceq \mathcal{C}$ (and in particular if $\mathcal{B}\llcurly \mathcal{C}$), then $\meas (\mathcal{B})\leq \meas (\mathcal{C})$.
\begin{proof}
Let $B\in \mathcal{B}$.  Then, there is some $C\in \mathcal{C}$ be such that $B\subseteq C$.  Thus, we have
\begin{equation}
\meas (\mathcal{B})\coloneqq \meas (B)\leq \meas (C)\eqqcolon \meas (\mathcal{C}).
\end{equation}
\end{proof}
\end{prp}
Our first relatively significant result, which we shall use to define lebesgue measure on $\R ^d$, is that, to define a measure, it suffices to define a measure on just the sets in a given uniform base.
\begin{thm}[Carath\'{e}odory's Extension Theorem]\label{CaratheodorysExtensionTheorem}
\begin{savenotes}
Let $X$ be a uniform space with uniform base $\widetilde{\mathcal{B}}$, let $\mathcal{M}$ be the collection of sets that are a finite union and intersection of elements of $\widetilde{\mathcal{B}}$ and their complements, and let $\meas :\mathcal{M}\rightarrow [0,\infty ]$ be such that
\begin{enumerate}
\item \label{CaratheodorysExtensionTheorem.i}$\meas (\emptyset )=0$;
\item \label{CaratheodorysExtensionTheorem.iii}every $\mathcal{B}\in \widetilde{\mathcal{B}}$ is uniformly-measurable with respect to $\meas$; and
\item \label{CaratheodorysExtensionTheorem.iv}$\meas$ is additive.
\end{enumerate}
Then,
\begin{equation}
\meas (S)\coloneqq \inf \left\{ \sum _{m\in \N}\meas (B_m):S\subseteq \bigcup _{m\in \N}M_m\text{ and  each }M_m\in \mathcal{M}\text{ (or }M_m=\emptyset \text{).}\right\} ,\footnote{The point of allowing each $M_m$ to be empty is to allow for finite unions as well.}
\end{equation}
is a measure on $X$ which agrees with $\meas$ on $\mathcal{M}$.
\begin{rmk}
These are just verbatim three out of four of the axioms of a measure.  The point is that, to define a measure, it suffices to define them on sets coming from a given uniform base.  Furthermore, you also do not have to check subadditivity of your original definition.
\end{rmk}
\begin{rmk}
Warning:  This measure need not be unique in general---see \cref{exm5.1.22}.
\end{rmk}
\begin{proof}
\Step{Introduce notation}
For the time being, let us write
\begin{equation}
\breve{\meas}(S)\coloneqq \inf (M(S))
\end{equation}
with the breve to distinguish it from what we started with, where
\begin{equation}
M(S) \coloneqq \left\{ \sum _{m\in \N}\meas (B_m):S\subseteq \bigcup _{m\in \N}M_m\text{ and  each }M_m\in \mathcal{M}\text{ (or }M_m=\emptyset \text{).}\right\} .
\end{equation}
(Once we show that this indeed agrees with what we started with, we shall drop the breve.)

\Step{Show that $\breve{\meas}(S)=\meas (S)$ for $S\in \mathcal{M}$}
First of all, as $\meas (S)\in M(S)$, we have that $\breve{\meas}(S)\leq \meas (S)$.  In particular, if $\breve{\meas}(S)=\infty$, then $\meas (S)=\infty$ as well, and so we may as well assume that $\breve{\meas}(S)$ is finite.  For the other inequality, let $\varepsilon >0$, and let $M_m\in \mathcal{M}$ (or empty) be such that (i) $S\subseteq \bigcup _{m\in \N}M_m$ and (ii)
\begin{equation}
\breve{\meas}(S)\leq \sum _{m\in \N}\meas (M_m)<\breve{\meas}(S)+\varepsilon .
\end{equation}
Define
\begin{equation}
M_m'\coloneqq S\cap \left( M_m\setminus \bigcup _{k=0}^{m-1}M_k\right) .
\end{equation}
Note that (i) each $M_m'\in \mathcal{M}$, (ii) $M_m'\subseteq M_m$, (iii) the $M_m'$s are disjoint, and (iv) $S=\bigcup _{m\in \N}M_m'$.  Thus,
\begin{equation}
\meas (S)=\sum _{m\in \N}\meas (M_m')\leq \sum _{m\in \N}\meas (M_m)<\breve{\meas}(S)+\varepsilon .
\end{equation}
Hence, $\meas (S)\leq \breve{\meas (S)}$, and so $\meas (S)=\breve{\meas}(S)$.  Thus, hereafter, we drop the breve.

\Step{Show that $\meas (S)\geq 0$}
As $M(S)\subseteq [0,\infty ]$, we always have that $\meas (S)\geq 0$.

\Step{Show that $\meas (\emptyset )=0$}
The empty-set itself is a countable cover of $\emptyset$, so that $0\in M(\emptyset )$, and hence $\meas (\emptyset )\leq 0$.  Of course, we already know that $\meas (\emptyset )\geq 0$, and so $\meas (\emptyset )=0$.

\Step{Show that $\meas$ is nondecreasing}
If $S\subseteq T$, then $M(S)\supseteq M(T)$, and so, taking the infimum, we have that $\meas (S)\leq \meas (T)$.

\Step{Show that $\meas$ is subadditive on $2^X$}
Let $\{ M_m:m\in \N \} \subseteq 2^X$.  If $\meas (M_m)=\infty$ for any $m\in \N$, then because $\meas$ is nondecreasing, we would likewise have that
\begin{equation}
\meas \left( \bigcup _{m\in \N}M_m\right) \geq \meas (M_m)=\infty ,
\end{equation}
and so in this case we are done.  Thus, we may as well assume without loss of generality that each $\meas (M_m)$ is finite.  Let $\varepsilon >0$.  Then,\footnote{This is why we needed finiteness.} there are $M_{m,n}\in \mathcal{M}\cup \{ \emptyset \}$, such that
\begin{equation}
\meas _(M_m)\leq \sum _{n\in \N}\meas (M_{m,n})<\meas (M_m)+\tfrac{\varepsilon}{2^m}.
\end{equation}
As $\bigcup _{m\in \N}M_m\subseteq \bigcup _{m,n\in \N}M_{m,n}$, we thus have that
\begin{equation}
\meas \left( \bigcup _{m\in \N}M_m\right) \leq \sum _{m,n\in \N}\meas (M_{m,n})<\sum _{m\in \N}\left[ \meas (M_m)+\tfrac{\varepsilon}{2^m}\right] =\sum _{m\in \N}\meas (M_m)+\varepsilon .
\end{equation}
Thus, indeed,
\begin{equation}
\meas _{\mathcal{B}}\left( \bigcup _{m\in \N}M_m\right) \leq \sum _{m\in \N}\meas (M_m).
\end{equation}

\Step{Conclude that $\meas$ is a measure on $X$}
The only remaining axiom that needs verifying is that it is additive on $\bigcup _{\mathcal{B}\in \widetilde{\mathcal{B}}}\mathcal{B}$, however, this was assumed, and so there is nothing to prove.\footnote{Well, I suppose we are technically using the fact that our new measure agrees with the old on $\mathcal{M}$, which contains $\bigcup _{\mathcal{B}\in \widetilde{\mathcal{B}}}\mathcal{B}$.}
\end{proof}
\end{savenotes}
\end{thm}
\begin{exm}[The extension need not be unique]\label{exm5.1.22}
Take $X\coloneqq \R$ and $\widetilde{\mathcal{B}}\coloneqq \{ \mathcal{B}_\varepsilon :\varepsilon >0\}$.  Then, the only noninfinite set which is contained in the collection $\mathcal{M}$ of sets that are a finite union and intersection of elements of $\widetilde{\mathcal{B}}$ and their complements is the empty-set.  Define $\meas :\mathcal{M}\rightarrow [0,\infty ]$ such that
\begin{equation}
\meas (S)\coloneqq \begin{cases}0 & \text{if }S=\emptyset \\ \infty & \text{otherwise}\end{cases}.
\end{equation}
Then, the counting measure and the infinite measure are two distinct extensions of $\meas$ to all of $2^X$.
\end{exm}

\begin{dfn}[Regular measure]\label{RegularMeasure}
Let $X$ be a uniform space and let $\meas :2^X\rightarrow [0,\infty ]$ be a uniform measure.  Then, $\meas$ is \emph{regular}\index{Regular measure}
\begin{enumerate}
\item $\meas$ is finite on quasicompact subsets;
\item (Outer-regular) for $S\subseteq X$,
\begin{equation}
\meas (S)=\inf \{ \meas (U):S\subseteq U,\ U\text{ open.}\} ;\text{ and }
\end{equation}
\item (Inner-regular on open sets) for $U\subseteq X$ open,
\begin{equation}
\meas (U)=\sup \{ \meas (K):K\subseteq U,\ K\text{ quasicompact.}\} .
\end{equation}
\end{enumerate}
\end{dfn}
\begin{dfn}[Positively-separated]\label{PositivelySeparated}
Let $X$ be a set, let $\mathcal{U}$ be a cover of $X$, and let $S,T\subseteq X$.  Then, $S$ and $T$ are \emph{positively-separated} with respect to $\mathcal{U}$ iff no element of $\mathcal{U}$ intersects both $S$ and $T$.
\end{dfn}

\begin{dfn}[Isogeneous space]\label{IsogeneousSpace}
An \emph{isogeneous space} is a uniform space $X$ equipped with a group of uniform-homeomorphisms $H$ such that
\begin{equation}
\widetilde{\mathcal{B}}_H\coloneqq \{ \mathcal{B}_U\} \text{ where }\mathcal{B}_U\coloneqq \{ h(U):h\in H\} \text{ and }U\subseteq X\text{ open.}\right\} .
\end{equation}
is a uniform base for $X$.
\begin{rmk}
$\widetilde{\mathcal{B}}_H$ is the \emph{isogeneous base}\index{Isogeneous base} and each $\mathcal{B}_U$ is an \emph{isogeneous cover}\index{Isogeneous cover}.
\end{rmk}
\begin{rmk}
The example you should have in mind here is that of a topological group $G$.  In this case, the uniform base is the canonical one, $\widetilde{\mathcal{B}}\coloneqq \{ \mathcal{B}_U\}$ with $\mathcal{B}_U\coloneqq \{ gU:g\in U\}$ for $U$ an open neighborhood of the identity, and the set of all uniform homeomorphisms is the set of all left translations, $H\coloneqq \{ h_g:g\in G\}$ where $h_g(x)\coloneqq gx$.  Of course, you can also choose right-translations over left-translations (in both $\widetilde{\mathcal{B}}$ and $H$!) if you so desire.
\end{rmk}
\begin{rmk}
What do you think the morphisms of isogeneous spaces should be?
\end{rmk}
\end{dfn}
\begin{exr}
Let $G$ be a topological group and let $H\coloneqq \{ h_g:G\in G\}$, where $h_g:G\rightarrow G$ is defined by $h_g(x)\coloneqq gx$.  Show that $\coord{G,H}$ is an isogeneous space.
\end{exr}
\begin{thm}[Howe's Theorem]\index{Howe's Theorem}\label{HowesTheorem}
\begin{savenotes}
Let $\coord{X,H}$ be a $T_0$ isogeneous space. Then, if $X$ has a quasicompact set $K$ with nonempty interior,\footnote{So that then there $\mathcal{B}_{\Int (K)}$, $K$ the quasicompact set with nonempty interior, is an isogeneous cover.} then there exists a unique regular $\meas$ measure on $X$ such that (i) each cover in $\widetilde{\mathcal{B}}_H$ is a uniformly-measurable base with respect to $\meas$ and (ii) $\meas (K)=1$.
\begin{rmk}
You should think of $K$ has a set with which we can `compare' all other sets to get a ``measure'' of `size'.  The condition that it be quasicompact you can think of the condition that the measure of $K$ be finite, and the condition that it have nonempty interior you can think of the condition that the measure of $K$ be positive.  Intuitively, you can imagine that if $\meas (K)=\infty$ or $\meas (K)=0$, then it will be essentially impossible to compare `compare' the `size' of other sets to $K$.
\end{rmk}
\begin{rmk}
If $G$ is a topological group and has a quasicompact set with nonempty interior, in this classical case, the resulting measure is called a (left) \emph{haar measure}\index{Haar measure} (for the symmetries being \emph{left}-translation).  In particular, \emph{lebesgue measure} will be the haar measure for the topological group $\coord{\R ,^d+,0,-}$.\footnote{If the group is commutative, the symmetries by left translation and right translations are the same, so left vs.~right does not matter.}
\end{rmk}
\begin{proof}
\Step{Make hypotheses and introduce notation}
Suppose that $X$ has a quasicompact set $K_0$ with nonempty interior.  Denote the uniform topology on $X$ by $\mathcal{U}$ and denote the collection of all quasicompact subsets of $X$ by $\mathcal{K}$

\Step{Define $(K:U)$ for $K\in \mathcal{K}$ and $U\in \mathcal{U}$}
The cover $\mathcal{B}_U\coloneqq \{ h(U):h\in H\}$ is an open cover of $K$, and so there are a finite subcover.  Let $(K:U)$ denote the cardinality of the smallest such subcover.

\Step{Define $\mathrm{H}_U\mathcal{K}\rightarrow \R _0^+$}
For $U\in \mathcal{U}$, define $\mathrm{H}_U:\mathcal{K}\rightarrow \R _0^+$ by
\begin{equation}
\mathrm{H}_U(K)\coloneqq \frac{(K:U)}{(K_0:U)}.\footnote{$K_0$ is nonempty, and so cannot be covered by anything empty.  Therefore, $(K_0:U)\geq 1$, and in particular, is not $0$.}
\end{equation}

\Step{Show that $\mathrm{H}_U(K)\leq (K:\Int (K_0))$}
We now check that $\mathrm{H}_U(K)\leq (K:\Int (K_0))$, that is, $(K:U)\leq (K:\Int (K_0))(K_0:U)$.  Let us temporarily write $m\coloneqq (K:\Int (K_0))$ and $n\coloneqq (K_0:U)$.  There are thus $h_1,\ldots ,h_m\in H$ such that $\{ h_1(\Int (K_0)),\ldots ,h_m(\Int (K_0))\}$ covers $K$.  There are also $h_1',\ldots ,h_n'\in H$ such that $\{ h_1(U),\ldots ,h_n(U)\}$ covers $K$.  Therefore,
\begin{equation}
K\subseteq \bigcup _{k=1}^mh_k(\Int (K_0))\subseteq \bigcup _{k=1}^mh_k(K_0)\subseteq \bigcup _{k=1}^mh_k\left( \bigcup _{l=1}^nh_l'(U)\right) =\bigcup _{k=1}^m\bigcup _{l=1}^n[h_k\circ h_l'](U)
\end{equation}
Hence, $K$ is covered by $mn$ elements of $\mathcal{B}_U$,\footnote{Here we are using the fact that $H$ is closed under composition, so that $h_k\circ h_l'\in h$.} and hence $(K:U)\leq mn\coloneqq (K:\Int (K_0))(K_0:U)$.

\Step{Define $\mathrm{H}$}
Define $T\coloneqq \prod _{K\in \mathcal{K}}[0,(K:K_0)]$.  Each $\mathrm{H}_U$ may be thought of as a point in $T$, whose component at $K\in \mathcal{K}$ is $\mathrm{H}_U(K)\in [0,(K:K_0)]$.\footnote{That was sort of the point of the previous step.}  Thus, for $U\in \mathcal{U}$, let us define
\begin{equation}
C_U\coloneqq \Cls \left( \left\{ \mathrm{H}_U:\mathcal{U}\ni V\subseteq U\right\} \right) 
\end{equation}
and
\begin{equation}
\mathcal{C}\coloneqq \{ C_U:U\in \mathcal{U}\} .
\end{equation}
We wish to show that the intersection of any finitely many elements of $\mathcal{C}$ is nonempty.  Then, because $T$ is quasicompact by \nameref{TychnoffsTheorem} (\cref{TychnoffsTheorem}), it will follow that the intersection over \emph{all} elements in $\mathcal{C}$ will be nonempty.

This is actually really easy, however, because for $U_1,\ldots ,U_m\in \mathcal{U}$, we have that
\begin{equation}
\mathrm{H}_{U_1\cap \cdots \cap U_m}\in \bigcap _{k=1}^mC_{U_k}.
\end{equation}
Therefore, by quasicompactness, there is some
\begin{equation}
\mathrm{H}\in \bigcap _{U\in \mathcal{U}}C_U.
\end{equation}

\Step{Show that $\mathrm{H}(K_1)\leq \mathrm{H}(K_2)$ if $K_1\subseteq K_2$}\label{Haar.4}
Let $K_1,K_2\in \mathcal{K}$ be such that $K_1\subseteq K_2$.  We first show that, for each $U\in \mathcal{U}$, $\mathrm{H}_U(K_1)\leq \mathrm{H}_U(K_2)$.  But this is trivial, because the covering of $K_2$ with $(K_2:U)$ elements of $\mathcal{B}_U$ is also a covering of $K_1$ with $(K_2:U)$ elements of $\mathcal{B}_U$, so that $(K_1:U)\leq (K_2:U)$, and hence $\mathrm{H}_U(K_1)\leq \mathrm{H}_U(K_2)$.

Thinking of elements $f$ of $T$ as functions from $\mathcal{K}$ to $\R$, consider the map\footnote{For each $K_1,K_2\in \mathcal{K}$ with $K_1\subseteq K_2$, we have such a map.} that sends $f\in T$ to $f(K_2)-f(K_1)$.  This is a composition of continuous functions, and hence continuous.\footnote{The first map from $T$ into $\R \times \R$ is the projection of $f\in T$ onto the $K_1^{\text{th}}$ coordinate in the first coordinate and the projection of $f\in T$ onto $K_2^{\text{th}}$ coordinate in the second coordinate.  This map is continuous because it is continuous in each coordinate.  Each coordinate is continuous because projections are continuous.  The first map is followed by the map from $\R \times \R$ into $\R$ given by subtraction, which is continuous because we know that $\coord{R,+,0,-}$ is a topological group.}  This map is also nonnegative on each $C_U$ because $\mathrm{H}_U(K_1)\leq \mathrm{H}_U(K_2)$ for each $U\in \mathcal{U}$ (we need continuity so that we know it is nonnegative on the \emph{closure} of $\{ \mathrm{H}_V:V\subseteq U\}$).  As $\meas$ is an element of each $C_U$, it follows that this map is also nonnegative at $\meas$, so that $\mathrm{H}(K_1)\leq \mathrm{H}(K_2)$.

\Step{Show that $\mathrm{H}(K_1\cup K_2)\leq \mathrm{H}(K_1)+\mathrm{H}(K_2)$}\label{Haar.5}
Let $K_1,K_2\in \mathcal{K}$.  We first show that $\mathrm{H}_U(K_1\cup K_2)\leq \mathrm{H}_U(K_1)+\mathrm{H}_U(K_2)$ for each $U\in \mathcal{U}$.  This is trivial, because a covering of $K_1$ with $(K_1:U)$ elements of $\mathcal{B}_U$ together with a covering of $K_2$ with $(K_2:U)$ elements of $\mathcal{B}_U$ is a cover of $K_1\cup K_2$ with $(K_1:U)+(K_2:U)$ elements of $\mathcal{B}_U$, so that $(K_1\cup K_2:U)\leq (K_1:U)+(K_2:U)$.  It follows that $\mathrm{H}_U(K_1\cup K_2)\leq \mathrm{H}_U(K_1)+\mathrm{H}_U(K_2)$.

Proceeding similarly as in \cref{Haar.4}, the map that sends $f\in T$ to $f(K_1)+f(K_2)-f(K_1\cup K_2)$ is continuous and nonnegative on each $C_U$, and hence is nonnegative for $\mathrm{H}\in T$.  Thus, $\mathrm{H}(K_1\cup K_2)\leq \mathrm{H}(K_1)+\mathrm{H}(K_2)$.

\Step{Show that $\mathrm{H}_U(K_1\cup K_2)=\mathrm{H}_U(K_1)+\mathrm{H}_U(K_2)$ if $K_1$ and $K_2$ are positively-separated with respect to $\mathcal{B}_U$}
Let $K_1,K_2\in \mathcal{K}$ be positively-separated with respect to $\mathcal{B}_U$.  We have already shown that $\mathrm{H}_U(K_1\cup K_2)\leq \mathrm{H}_U(K_1)+\mathrm{H}_U(K_2)$, so it suffices to show that $\mathrm{H}_U(K_1)+\mathrm{H}_U(K_2)\leq \mathrm{H}_U(K_1\cup K_2)$.  In other words, it suffices to show that $(K_1:U)+(K_2:U)\leq (K_1\cup K_2:U)\eqqcolon m$.  Let $h_1(U),\ldots ,h_m(U)\in \mathcal{B}_U$ be a cover of $K_1\cup K_2$.  By hypothesis,\footnote{This is the definition of positively-separated---see \cref{PositivelySeparated}.} every single one of these can only intersect $U_1$ or $U_2$, but not both.  Thus, after relabeling if necessary, the first $k$ of these guys will form a cover of $K_1$ and the latter $m-k$ will form a cover of $K_2$.  Thus, $(K_1:U)\leq k$ and $(K_2:U)\leq m-k$, and so $(K_1:U)+(K_2:U)\leq k+(m-k)=m\coloneqq (K_1\cup K_2:U)$, which completes this step.

\Step{Define the measure $\meas$ on all open subsets of $X$}
For $U\subseteq X$ open, define
\begin{equation}\label{5.1.46}
\meas (U)\coloneqq \sup \{ \mathrm{H}(K):K\subseteq U,\ K\in \mathcal{K}\} .
\end{equation}

\Step{Extend $\meas$ to all subsets of $X$}
Now, for an arbitrary subsets $S$ of $X$, define
\begin{equation}
\meas (S)\coloneqq \inf \{ \meas (U):S\subseteq U,\ U\in \mathcal{U}\} .
\end{equation}
\begin{exr}
Show that this agrees with \eqref{5.1.46} when $S$ is open, so that this is indeed an extension.
\end{exr}

\Step{Show that $\meas$ is an outer-measure}
\begin{exr}
Check that $\meas (\emptyset )=0$ and that $\meas$ is nondecreasing.
\end{exr}

We now check that it is subadditive.  To prove this, we will first need a lemma.
\begin{lma}
Let $X$ be $T_2$, let $K\subseteq X$ be quasicompact, and let $U_1,U_2\subseteq X$ be open and such that $K\subseteq U_1\cup U_2$>  Then, there are quasicompact subsets $K_1,K_2\subseteq X$ such that (i) $K_1\subseteq U_1$, (ii) $K_2\subseteq U_2$, and (iii) $K=K_1\cup K_2$.
\begin{proof}
We leave this as an exercise.
\begin{exr}
Complete the proof yourself.
\end{exr}
\end{proof}
\end{lma}
Having (hopefully) proved the lemma, we now show subadditivity for \emph{open} sets.  (We will then prove subadditivity in general.)  So, let $\{ U_m:m\in \N \}$ be a countable collection of open sets of X.  Let $K\subseteq \bigcup _{m\in \N}U_m$.  Then, there is some $m_K\in \N$ such that $K\subseteq \bigcup _{k=1}^{m_K}U_k$.  By applying this lemma inductively then, we may find quasicompact sets $K_1,\ldots ,K_m$ such that (i) $K_k\subseteq U_k$ for $1\leq k\leq m$ and $K=\bigcup _{k=1}^mK_k$.  Using the fact that we have already proved finite `subadditivity' (of $\mathrm{H}$) for quasicompact sets (\cref{Haar.5}), we find that
\begin{equation}
\mathrm{H}(K)\leq \sum _{k=1}^m\mathrm{H}(K_k)\leq \sum _{k=1}^m\meas (U_k)\leq \sum _{m\in \N}\meas (U_m).
\end{equation}
Taking the $\sup$ of $K$, we find that
\begin{equation}
\meas \left( \bigcup _{m\in \N}U_m\right) \coloneqq \sup \left\{ \mathrm{H}(K):K\subseteq \bigcup _{m\in \N}U_m,\ K\in \mathcal{K}\right\} \leq \sum _{m\in \N}\meas (U_m).
\end{equation}

Having proved subadditivity for open sets, we now prove it for arbitrary sets.  So, let $\{ S_m:m\in \N \}$ be an arbitrary countable collection of subsets of $X$.  If $\sum _{m\in \N}\meas (S_m)=\infty$, then there is nothing to show, and so we may as well suppose without loss of generality that $\sum _{m\in \N}\meas (S_m)<\infty$.  Let $\varepsilon >0$ and for each $m\in \N$ pick an open set $U_m$ such that (i) $S_m\subseteq U_m$ and (ii) $\meas (S_m)\leq \meas (U_m)<\meas (S_m)+\frac{\varepsilon}{2^m}$.  Then, using subadditivity for open sets, we find
\begin{equation}
\meas \left( \bigcup _{m\in \N}S_m\right) \leq \meas \left( \bigcup _{m\in \N}U_m\right) \leq \sum _{m\in \N}\meas (U_m)<\sum _{m\in \N}\left[ \meas (S_m)+\tfrac{\varepsilon}{2^m}\right] =\sum _{m\in \N}\meas (S_m)+2\varepsilon .
\end{equation}
Hence, as $\varepsilon >0$ was arbitrary, we have that
\begin{equation}
\meas \left( \bigcup _{m\in \N}S_m\right) \leq \sum _{M\in \N}\meas (S_m).
\end{equation}
Thus, $\meas$ is an outer-measure on $X$.

\Step{Show that each $\mathcal{B}_U$ is uniformly-measurable with respect to $\meas$}
Let $h\in H$.  We want to show that $\meas (h(U))=\meas (U)$.  Then, for any other $h'\in H$, we will have that $\meas (h(U))=\meas (U)=\meas (h'(U))$, so that indeed every element of $\mathcal{B}_U$ has the same measure.  However, $K$ is a quasicompact set contained in $U$ iff $h(K)$ is a quasicompact set contained in $h(U)$.\footnote{This implicitly uses the fact that $h^{-1}\in H$.}  Therefore, by the definition of $\meas (U)$ \eqref{5.1.46} it suffices to show that $\mathrm{H}(K)=\mathrm{H}(h(K))$.  To show this, we first show that $\mathrm{H}_U(K)=\mathrm{H}_U(h(K))$ for all $U\in \mathcal{U}$.  That is, we would like to show that $(K:U)=(h(K):U)$.  However, every cover of $K$ by elements of $\mathcal{B}_U$, $h_1(U),\ldots ,h_m(U)$, gives a cover of $h(K)$ by elements of $\mathcal{B}_U$ of the same cardinality, $h(h_1(U)),\ldots ,h(h_m(U))$.  It thus follows that $\mathrm{H}_U(K)=\mathrm{H}_U(h(K))$.

For $h\in H$ fixed, consider the map from $T$ to $\R$ that sends $f$ to $f(h(K))-f(K)$.  We just showed that this is $0$ on each $\mathrm{H}_U\in T$, and so it is $0$ on $C_U$, and so it is $0$ on $\mathrm{H}$, that is, $\mathrm{H}(K)=\mathrm{H}(h(K))$.

\end{proof}
\end{savenotes}
\end{thm}
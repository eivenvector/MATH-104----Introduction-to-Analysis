So, first things first---fuck the riemann integral.  Seriously.  The only argument pro-riemann-integral is that it is easier.  What a ridiculous argument.  This is math, dude.  If you choose to do things because they're easy, you're in the wrong subject.  Moreover, I would argue that this is not even true---if you set things up right, you can literally \emph{define} the (lebesgue) integral to be the area (measure) under the curve.  Or, if you prefer, you can take a limit over the size of a partition of the sum of the areas of the rectangles corresponding to the subsets of the partition (the riemann integral).  Are you really going to sit here and try to argue that this is easier to teach?  I call bullshit.  And besides, if you're going to become a mathematician, you have to learn the lebesgue integral at some point anyways\textellipsis why learn something only to have to relearn it later?

Okay, so now that my rant is out of the way, let's actually do some mathematics.

\section{Measure theory}

All of integration theory ultimately boils down to measure theory.  The definition of the integral itself is relatively easy.  In fact, the definition of abstract measure spaces is even easier.  There's really no question that writing down the definition of the lebesgue integral is \emph{significantly} easier than that of the riemann integral.  What is a bit tricky, however, is constructing specific measures.  In our case, we will primarily be concerned with constructing lebesgue measure (on $\R ^d$), and this is really the only part that is a bit tricky.  Fortunately, there is a \emph{huge} theorem that will just spit out lebesgue measure for us, \emph{the \nameref{HaarHowesTheorem}}.

\subsection{Outer-measures}

The intuition behind measure is actually quite easy---a measure is just an axiomatization of our intuition about notion of things like length, area, and volume.  Before we define a measure, it will be convenient to introduce a couple of terms.
\begin{dfn}[Subadditivity and additivity]
Let $X$ be a set, let $\meas :2^X\rightarrow [0,\infty ]$, and let $\mathcal{M}\subseteq 2^X$.  
\begin{enumerate}
\item $\meas$ is \emph{subadditive}\index{Subadditive} on $\mathcal{M}$ iff for $\{ M_m:m\in \N \} \subseteq \mathcal{M}$ we have
\begin{equation}\label{5.1.2}
\meas \left( \bigcup _{m\in \N}M_m\right) \leq \sum _{m\in \N}\meas (M_m);
\end{equation}
\item $\meas$ is \emph{additive}\index{Additive (measure)} on $\mathcal{M}$ iff for $\{ M_m:m\in \N \} \subseteq \mathcal{M}$ a \emph{disjoint} collection we have
\begin{equation}\label{5.1.3}
\meas \left( \bigcup _{m\in \N}M_m\right) =\sum _{m\in \N}\meas (M_m).
\end{equation}
\end{enumerate}
$\meas$ is simply just subadditive (resp.~additive) if it is subadditive (resp.~additive) on all of $X$.
\begin{rmk}
You might think that we should always have additivity, or at the very least, we should have finite additivity:  if $S$ and $T$ are disjoint, then $\meas (S\cup T)=\meas (S)+\meas (T)$.  Unfortunately, this is \emph{false}---see \cref{exm5.2.56}.  We will have additivity on a very large class of sets, however, the so-called \emph{measurable sets}---see \cref{MeasurableSet}.
\end{rmk}
\end{dfn}
\begin{exr}\label{exr5.1.4}
Let $\mathcal{M}$ be a collection of sets that is closed under union, intersection, and complementation.  Show that if $\meas$ is additive on $\mathcal{M}$ then it is subadditive on $\mathcal{M}$.
\begin{rmk}
There is something to show here.  While \eqref{5.1.3} itself is obviously a stronger condition than \eqref{5.1.2}, it is also only assumed for \emph{disjoint} collections.  The problem then is to show that, if \eqref{5.1.3} holds for disjoint collections, then \eqref{5.1.2} holds for \emph{all} collections.
\end{rmk}
\end{exr}

\begin{dfn}[Outer-measure]\label{OuterMeasure}
Let $X$ be a set.  An \emph{outer-measure}\index{Outer-measure} on $X$ is a function $\meas :2^X\rightarrow [0,\infty ]$ such that
\begin{enumerate}
\item $\meas (\emptyset )=0$;
\item (Nondecreasing)\label{Measure.Monotonicity} $\meas :\coord{2^X,\subseteq}\rightarrow [0,\infty ]$ is nondecreasing;\footnote{Concretely, this means that $\meas (S)\leq \meas (T)$ if $S\subseteq T$.} and
\item (Subadditivity) $\meas$ is subadditive.
\end{enumerate}
A set equipped with an outer-measure is an \emph{outer-measure space}\index{Outer-measure space}.
\begin{rmk}
Note that we allow the measure of sets to be infinite.  This is incredibly important---for example, we will want $\meas (\R )=\infty$ (for lebesgue measure anyways).
\end{rmk}
\begin{rmk}
As a consequence of this, we needn't worry about convergence in the third axiom (see \eqref{5.1.2}).  As a matter of fact, we definitely want to allow this sum to diverge---think about what the measure of $\bigcup _{m\in \Z}(m,m+1)$ should be.
\end{rmk}
\end{dfn}
In measure theory, things almost always matter only `up to' sets of measure $0$.  This concept is so important that there is a term for it.
\begin{mdf}[Almost-everywhere XYZ]\label{AlmostEverywhereXYZ}
Let $f:\coord{X,\meas}\rightarrow Y$ be a function on an outer-measure space.  Then, $f$ is \emph{almost-everywhere XYZ}\index{Almost-everywhere XYZ} iff
\begin{equation}
\meas \left( \left\{ x\in X:f(x)\text{ is not XYZ.}\right\} \right) =0.
\end{equation}
\end{mdf}
\begin{displayquote}
Whenever $X$ is equipped with a measure, \emph{all relations on $\Mor _{\Set}(X,Y)$ are defined only up to measure zero}.  For example, if we write $f=g$, what we really mean is that $\meas \left( \{ x\in X:f(x)\neq g(x)\} \right) =0$.  Similarly, if we write $f\leq g$, what we really mean is that $\meas \left( \{ x\in X:f(x)\not \leq g(x)\} \right) =0$.  Etc..  A fancier way of putting this, is that we work in the \emph{almost-everywhere category}.
\end{displayquote}
\begin{exm}[The almost-everywhere category]\index{Almost-everywhere category}
\begin{savenotes}
The almost-everywhere category is the category $\AlE$ whose collection of objects $\AlE _0$ is the collection of all outer-measure spaces, for every outer-meausre space $X$ and outer-meausre space $Y$ the collection of morphisms from $X$ to $Y$, $\Mor _{\AlE}(X,Y)$, is the quotient set $\Mor _{\Set}(X,Y)/\sim$, where $f\sim g$ iff $f$ and $g$ are equal almost everywhere, composition is given by ordinary function composition\footnote{One must check well-definedness---see below}, and the identities of the category are the (equivalence classes) of the identity functions.
\begin{exr}
Check that $\sim$ is an equivalence relation.
\end{exr}
\begin{prp}
Composition in $\AlE$ is well-defined.
\begin{proof}
Let $X,Y,Z$ be outer-measure spaces, let $f_1,g_1:X\rightarrow Y$ be equal almost-everywhere and let $f_2,g_2:Y\rightarrow Z$ be equal almost-everywhere.  We must show that $f_2\circ f_1$ is equal to $g_2\circ g_1$ almost everywhere.  However, of course, if $f_1(x)=g_1(x)$, then certainly $f_2(f_1(x))=g_2(g_1(x))$, and so
\begin{equation}
\left\{ x\in X:f_1(x)=g_1(x)\right\} \subseteq \left\{ x\in X:[f_2\circ f_1](x)=[g_2\circ g_1](x)\right\} ,
\end{equation}
and so
\begin{equation}
\left\{ x\in X:[f_2\circ f_1](x)\neq [g_2\circ g_1](x)\right\} \subseteq \left\{ x\in X:f_1(x)\neq g_1(x)\right\} .
\end{equation}
As the set on the right-hand side has measure zero, so too does the set on the left-hand side.
\begin{rmk}
Note that we didn't even need to make use of the fact that $f_2(x)=g_2(x)$ for almost-every $x$.
\end{rmk}
\end{proof}
\end{prp}
\end{savenotes}
\end{exm}
\begin{exm}[The zero measure]\label{ZeroMeasure}
Let $X$ be a set and define $\meas :2^X\rightarrow [0,\infty ]$ by $\meas (S)\coloneqq 0$.  How terribly interesting.
\end{exm}
\begin{exm}[The infinite measure]\label{InfiniteMeasure}
Let $X$ be a set and define $\meas :2^X\rightarrow [0,\infty ]$ by
\begin{equation}
\meas (S)\coloneqq \begin{cases}0 & \text{if }S=\emptyset \\ \infty & \text{otherwise}\end{cases}.
\end{equation}
Dear god, this example is even more interesting than the last one.
\end{exm}
\begin{exm}[The unit measure]\label{UnitMeasure}
Let $X$ be a set and define $\meas :2^X\rightarrow [0,\infty ]$ by
\begin{equation}
\meas (S)\coloneqq \begin{cases}0 & \text{if }S=\emptyset \\ 1 & \text{otherwise}\end{cases}.
\end{equation}
\begin{rmk}
Note that this is \emph{never} additive (unless of course $X$ is either empty or a single point).  This makes it useful for producing counter-examples (see \cref{exm5.1.42}), and not much else.
\end{rmk}
\end{exm}
\begin{exm}[The counting measure]
Let $X$ be a set and for $S\subseteq X$ define $\meas (S)\coloneqq \abs{S}$, that is, the cardinality of $S$.
\begin{rmk}
This is actually incredibly important, as we shall see that sums are just integrals with respect to the counting measure.
\end{rmk}
\end{exm}

While we will be assigning a measure to every set, not all of them will be considered to be \emph{measurable}.  In general, we will \emph{not} have additivity; however, when we restrict our (outer) measures to the collection of measurable sets, we \emph{will} have additivity.  This is more or less the point of talking about measurable sets---additivity is a nice thing to have.
\begin{dfn}[Measurable (set)]\label{MeasurableSet}
Let $\meas :2^X\rightarrow [0,\infty ]$ be an outer-measure on a set $X$ and let $M\subseteq X$.  Then, $M$ is \emph{measurable}\index{Measurable (set)} iff
\begin{equation}
\meas (S)=\meas (S\cap M)+\meas (S\cap M^{\comp})
\end{equation}
for all sets $S\subseteq X$.
\begin{rmk}
Think about what this means:  $M$ is chopping up $S$ into two pieces, the set of points in $M$ and the set of points not in $M$.  $M$ is measurable, then, if the measure of $S$ is the sum of the measure of these two pieces \emph{for all} $S$.  In particular, $S$ itself is definitely not required to be measurable.\footnote{For one thing, this would make the definition circular.}
\end{rmk}
\begin{rmk}
By subadditivity, we \emph{always} have that $\meas (S)\leq \meas (S\cap M)+\meas (S\cap M^{\comp})$.  Therefore, in fact, $M$ is measurable iff
\begin{equation}
\meas (S)\geq \meas (S\cap M)+\meas (S\cap M^{\comp})
\end{equation}
for all $S\subseteq X$.
\end{rmk}
\begin{rmk}
You might say that the motivation for the definition is that, for sets $S,T$ that satisfy this property, we should have at least finite additivity (i.e.~$S,T$ disjoint implies $\meas (S\cup T)=\meas (S)+\meas (T)$).  It turns-out that this is true, but in fact, perhaps surprisingly, we have much more than this---we actually have \emph{(countable) additivity}--see \cref{CaratheodorysTheorem}.  By asking for finite additivity, we get countable additivity for free!
\end{rmk}
\begin{rmk}
Warning:  There definitely exist sets that are not measurable in general!  In fact, nonmeasurable sets are easy to find in `artificial' spaces---see \cref{exm5.1.42}.  But such pathologies exist even for the nicest of measures.  Indeed, see \cref{exm5.2.47} for a set that is not measurable with respect to lebesgue measure.
\end{rmk}
\begin{rmk}
This is also what is sometimes referred to \emph{carath\'{e}odory measurable}.
\end{rmk}
\end{dfn}
\begin{exr}
Show that if $\meas (M)=0$, then all subsets of $M$ are measurable.
\end{exr}

\begin{thm}[Carath\'{e}odory's Theorem]\index{Carath\'{e}odory's Theorem}\label{CaratheodorysTheorem}
\begin{savenotes}
Let $\meas :2^X\rightarrow [0,\infty ]$ be an outer-measure.  Then,
\begin{enumerate}
\item \label{CaratheodorysTheorem.i}$\meas$ is additive on the collection of measurable sets;
\item \label{CaratheodorysTheorem.ii}the countable union of measurable sets is measurable;
\item \label{CaratheodorysTheorem.iii}the countable intersection of measurable sets is measurable;
\item \label{CaratheodorysTheorem.iv}the complement of a measurable set is measurable;
\item \label{CaratheodorysTheorem.v}$\emptyset$ and $X$ are measurable.
\end{enumerate}
\begin{rmk}
Note that \nameref{DeMorgansLaws} (\cref{DeMorgansLaws}) imply that \ref{CaratheodorysTheorem.ii} are \ref{CaratheodorysTheorem.iii} are equivalent if \ref{CaratheodorysTheorem.iv} is true, in which case together they imply \ref{CaratheodorysTheorem.v}.\footnote{Because $\emptyset =S\cap S^{\comp}$.}  A nonempty collection of sets which satisfies \ref{CaratheodorysTheorem.ii}--\ref{CaratheodorysTheorem.v} is called a \emph{$\sigma$-algebra}\index{$\sigma$-algebra}.  We do not use this language, but it is important to know for consulting other references.
\end{rmk}
\begin{proof}
\Step{Show \ref{CaratheodorysTheorem.iv}}
The definition of measurability is $S\leftrightarrow S^{\comp}$ symmetric, so \ref{CaratheodorysTheorem.iv} is automatically true.

\Step{Show \ref{CaratheodorysTheorem.v}}
The empty-set is measurable by the previous exercise, and hence by \ref{CaratheodorysTheorem.iv}, $X$ is measurable as well, which establishes \ref{CaratheodorysTheorem.v}.

\Step{Reduce the proof of \ref{CaratheodorysTheorem.iii} to the proof of \ref{CaratheodorysTheorem.ii}}
As was explained in a remark, we need not show \ref{CaratheodorysTheorem.iii} itself---it will now follow if we can show \ref{CaratheodorysTheorem.ii}.

\Step{Prove \ref{CaratheodorysTheorem.ii} for finite unions}
We first show that the union of finitely many measurable sets is measurable.  It suffices of course to then just show that the union of two measurable sets is measurable.  So, let $M_1,M_2\subseteq X$ be measurable and let $S\subseteq X$.  Then,
\begin{equation}
\begin{split}
\meas (S) & =\footnote{Because $M_2$ is measurable.}\meas (S\cap M_2)+\meas (S\cap M_2^{\comp}) \\
& =\footnote{Because $M_1$ is measurable (applied twice).}\meas (S\cap M_2\cap M_1)+\meas (S\cap M_2\cap M_1^{\comp})+\meas (S\cap M_2^{\comp}\cap M_1)+\meas (S\cap M_2^{\comp}\cap M_1^{\comp}) \\
& \geq \footnote{By subadditivity and the fact that $M_1\cup M_2=(M_1\cap M_2)\cup (M_1\cap M_2^{\comp})\cup (M_1^{\comp}\cap M_2)$.}\meas (S\cap (M_1\cup M_2))+\meas (S\cap (M_1\cup M_2)^{\comp})
\end{split}
\end{equation}
Thus, indeed, $M_1\cup M_2$ is measurable.

\Step{Complete the proof of \ref{CaratheodorysTheorem.ii}}
Let $\{ M_m:m\in \N \}$ be a countable collection of measurable sets.  We wish to show that
\begin{equation}
\bigcup _{m\in \N}M_m
\end{equation}
is measurable.  First of all, define
\begin{equation}\label{5.1.13}
M_m'\coloneqq M_m\setminus \bigcup _{k=0}^{m-1}M_k.
\end{equation}
Note that each $M_m'$ is measurable because we already know that finite unions, complements, and hence also finite intersections, of measurable sets are measurable.
\begin{exr}
Show that (i) $M_m'\subseteq M_m$, (ii) the collection $\{ M_m':m\in \N \}$ is disjoint, and (iii) $\bigcup _{m\in \N}M_m=\bigcup _{m\in \N}M_m'$.
\begin{rmk}
This trick (the one in \eqref{5.1.13}, that is) is important.  Don't forget it.
\end{rmk}
\end{exr}
Thus, as
\begin{equation}
\bigcup _{m\in \N}M_m=\bigcup _{m\in \N}M_m',
\end{equation}
it suffices to prove this step in the case where $\{ M_m:m\in \N \}$ is itself disjoint (just rename $M_m$ to now be $M_m'$).  Thus, we now without loss of generality assume that $\{ M_m:m\in \N \}$ is disjoint.

Now define
\begin{equation}
N_m\coloneqq \bigcup _{k=0}^mM_k\text{ and }N\coloneqq \bigcup _{k\in M}M_m.
\end{equation}
so that, by the previous step, we have that $N_m$ is measurable.  Thus, for $S\subseteq X$,
\begin{equation}
\begin{split}
\meas (S\cap N_m) & =\footnote{Because $E_m$ is measurable.}\meas (S\cap N_m\cap M_m)+\meas (S\cap N_m\cap M_m^{\comp})=\footnote{Because the collection $\{ M_m:m\in \N \}$ is disjoint.}\meas (S\cap M_m)+\meas (S\cap N_{m-1}) \\
& =\footnote{Apply this trick inductively}\sum _{k=0}^m\meas (S\cap M_k).
\end{split}
\end{equation}
Thus,
\begin{equation}
\begin{split}
\meas (S) & =\meas (S\cap N_m)+\meas (S\cap N_m^{\comp})=\sum _{k=0}^m\meas (S\cap M_k)+\meas (S\cap N_m^{\comp}) \\
& \geq \footnote{Because $N^{\comp}\subseteq N_m^{\comp}$.}\sum _{k=0}^m\meas (S\cap M_k)+\meas (S\cap N^{\comp}).
\end{split}
\end{equation}
Hence,
\begin{equation}\label{5.1.16}
\meas (S)\geq \sum _{m\in \N}\meas (S\cap M_m)+\meas (S\cap N^{\comp})\geq \footnote{By subadditivity.}\meas (S\cap N)+\meas (S\cap N^{\comp}),
\end{equation}
and so indeed $N$ is measurable.

\Step{Prove \ref{CaratheodorysTheorem.i}}
Let $N$ and $S$ be as in the previous step and take $S\coloneqq N$.  Then, by \eqref{5.1.16}, we have that
\begin{equation}
\meas \left( \bigcup _{m\in \N}M_m\right) \geq \sum _{m\in \N}\meas (M_m).
\end{equation}
The other inequality is automatic from subadditivity, and so indeed, we have equality.
\end{proof}
\end{savenotes}
\end{thm}
Just as we have a notion of measurable set, so too do we have a notion of measurable \emph{function}
\begin{dfn}[Measurable (function)]\label{MeasurableFunction}
Let $f:\coord{X_1,\meas _1}\rightarrow \coord{X_2,\meas _2}$ be a function between outer-measure spaces.  Then, $f$ is \emph{measurable}\index{Measurable (function)} iff
\begin{enumerate}
\item \label{MeasurableFunction.i}the preimage of every set is measurable; and
\item \label{MeasurableFunction.ii}the preimage of a set of measure $0$ has measure $0$.
\end{enumerate}
\begin{rmk}
This is a \emph{very strong} condition.  For example, there are \emph{uniform-homeomorphisms} on $\R$ that are not measurable---see \cref{CantorFunction} (the Cantor Function).
\end{rmk}
\begin{rmk}
Neither of these conditions imply one another---see the following examples.
\end{rmk}
\end{dfn}
\begin{exm}[A function which preserves measurability but not measure $0$]
Precisely, we give a function that has the property that the preimage of every set is measurable but the preimage of a set of measure $0$ does not have measure $0$.

Let $X_1\coloneqq \{ x_1\}$ be a one point set and let $\meas _1$ be the infinite measure on $X_1$.  Let $X_2\coloneqq \{ x_2\}$ be a one point set and let $\meas _2$ be the zero measure on $X_2$.  Let $f:X_1\rightarrow X_2$ be the only function that exists from $X_1$ to $X_2$.

Every subset of $X_1$ is measurable, and so trivially the preimage of every subset of $X_2$ is measurable.  On the other hand, $\meas _2(\{ x_2\} )=0$, but $\meas _1\left( f^{-1}(\{ x_2\} )\right) =\infty \neq 0$.
\end{exm}
\begin{exm}[A function which preserves measure $0$ but not measurability]
Precisely, we give a function that has the property that the preimage of every set of measure $0$ has measure $0$ but the preimage of some measurable set is not measurable.

Let $X_1\coloneqq \{ x_1,x_2\} \eqqcolon X_2$ be a two point set.  Equip $X_1$ with the unit measure and equip $X_2$ with the infinite measure.  Let $f\coloneqq \id _{\{ x_1,x_2\}}$.

There is only one subset of $X_2$ with measure $0$, namely $\emptyset$, and of course the preimage of the empty-set (the empty-set itself) also has measure $0$.  On the other hand, $\{ x_1\}$ is measurable with respect to the infinite measure, but not the unit measure, and so its preimage (namely itself) is not measurable.
\end{exm}

\begin{displayquote}
I should probably mention at this point that the way I am presenting measure theory goes against the orthodoxy.  (You can skip this comment if you don't plan to look in other sources.)  Every other author I am aware of \emph{only works with measurable sets}.  For them, a \emph{measure space} is a set, together with a $\sigma$-algebra, the ``measurable sets'', along with a set function that is additive on the $\sigma$-algebra (and sends the empty-set to $0$).  For them, outer-measures are merely tools for constructing measures (a l\`{a} Mr.~Carath\'{e}odory).  Of course, their ``measure''\footnote{The quotes are to indicate that this is what they call it---I myself have not defined this term.} is just the restriction of the outer-measure to the collection of all measurable sets.  You might say they simply `forget' that they had ever assigned a measure to the nonmeasurable sets.  I find this unnecessarily complicated and messy.  For one thing, if you do things the way I have presented them, you don't have to worry about $\sigma$-algebras (at least not explicitly).  The disadvantage is that now I have to add the hypothesis ``These sets are measurable'' to a lot of my theorems.  Meh.  It's a trade-off, but I quite like not having to every worry about $\sigma$-algebras explicitly.
\end{displayquote}

\begin{exr}\label{exr5.1.21}
Let $\meas :2^X\rightarrow [0,\infty ]$ be an outer-measure and let $S\subseteq X$ be measurable have finite measure.  Show that
\begin{equation}
\meas (S\setminus T)=\meas (S)-\meas (T).
\end{equation}
\begin{rmk}
You need $S$ to have finite measure otherwise the right-hand side of this equation will be undefined if $T$ also has infinite measure.
\end{rmk}
for any measurable set $T\subseteq S$.
\end{exr}
\begin{exr}\label{exr5.1.27}
Let $M_0\subseteq M_1\subseteq \cdots$ be an nondecreasing countable collection of measurable sets.  Show that
\begin{equation}
\meas \left( \bigcup _{k=0}^\infty M_k\right) =\lim _m\meas \left( \bigcup _{k=0}^mM_k\right) =\lim _m\meas (M_k).
\end{equation}
\end{exr}
\begin{exr}\label{exr5.1.29}
Let $M_0\supseteq M_1\supseteq \cdots$ be a nonincreasing countable collection of measurable sets.  Show that, \emph{if at least one $M_k$ has finite measure},
\begin{equation}
\meas \left( \bigcap _{k=0}^\infty M_k\right) =\lim _m\meas \left( \bigcap _{k=0}^mM_k\right) =\lim _m\meas (M_k).
\end{equation}
\begin{rmk}
Though we technically have not defined measure on $\R$ yet, it is easy to see intuitively why we would neither expect nor want this to hold if the measure of each $M_k$ were infinite.  For example, take $M_k\coloneqq (-\infty ,-kk)$.  Then, on one hand, $\meas (M_k)=\infty$ for all $k$, but yet $\meas (\bigcap _{k=0}^\infty M_k)=\meas (\emptyset )=0$.
\end{rmk}
\end{exr}

\begin{displayquote}
At some point in the near future, we will be doing arithmetic with $\infty$---for example, what should the measure of $\R \times \{ 0\}$ in $\R ^2$ be?  Of course, from our definition of product measures, this will turn out to be $\infty \cdot 0$.  We hence declare that
\begin{equation}
\infty \cdot 0\coloneqq 0\eqqcolon 0\cdot \infty .
\end{equation}
There are other arithmetic notions we have to technically define (e.g.~$x+\infty=\infty$), but this is the only nonobvious one.
\end{displayquote}

\subsection{Measures on topological and uniform spaces}

One can go ahead and develop the theory for outer-measures on arbitrary sets, but in practice, we will only be working with measures defined on spaces with \emph{a lot} of extra structure.  This motivates us to investigate outer-measures on topological and uniform spaces, in which case we are of course going to require our outer-measures to be compatible with this extra structure.

\subsubsection{Measures on topological spaces}

There are definitely outer-measures on topological spaces for which the open sets are not measurable---see \cref{exm5.1.42}---in fact, this can even be the case for regular measures (the example in \cref{exm5.1.42} is regular---see \cref{RegularMeasure} for the definition of a regular measure), but most outer-measures on topological spaces which are not cooked-up for the sole purpose of producing counter-examples have the property that open sets are measurable.  We have a name for such measures:  \emph{borel measures}.
\begin{dfn}[Borel measure]\label{BorelMeasure}
Let $X$ be a topological space and let $\meas :2^X\rightarrow [0,\infty ]$ be an outer-measure.  Then, $\meas$ is \emph{borel}\index{Borel measure} iff every open set is measurable.  A topological space equipped with a borel measure is a \emph{borel measure space}\index{Borel measure space}.
\begin{rmk}
By \nameref{CaratheodorysTheorem}, it follows that the ``$\sigma$-algebra'' generated by the open sets likewise consists of measurable sets.  The term for the sets in this $\sigma$-algebra is \emph{borel set}, hence the term, \emph{borel measure}---a borel measure is a measure in which the borel sets are measurable (which is of course equivalent to the open sets being measurable).
\end{rmk}
\end{dfn}

\begin{dfn}[Regular measure]\label{RegularMeasure}
Let $X$ be a topological space and let $\meas :2^X\rightarrow [0,\infty ]$ be an outer-measure.  Then, $\meas$ is \emph{regular}\index{Regular measure} iff
\begin{enumerate}
\item $\meas$ is finite on quasicompact subsets;
\item (Outer-regular) for $S\subseteq X$,
\begin{equation}
\meas (S)=\inf \{ \meas (U):S\subseteq U,\ U\text{ open.}\} ;\text{ and }
\end{equation}
\item (Inner-regular on open sets) for $U\subseteq X$ open,
\begin{equation}
\meas (U)=\sup \{ \meas (K):K\subseteq U,\ K\text{ quasicompact.}\} .
\end{equation}
\end{enumerate}
A topological space equipped with regular measure is a \emph{regular measure space}\index{Regular measure space}.
\begin{rmk}
Warning:  We will see below (\cref{prp5.1.25}) that open sets must be measurable in \emph{uniform} measure spaces; however, this need not be the case in \emph{regular} measure spaces---see the following example.
\end{rmk}
\end{dfn}
\begin{exm}[A regular measure that is not borel]\label{exm5.1.42}
Define $X\coloneqq \{ x_1,x_2\}$, and equip it with the discrete topology and the unit measure (see \cref{UnitMeasure}).  Of course the measure of every quasicompact subset is finite.  But furthermore, it is also inner-regular on open sets (and in fact on every set), because this is must the statement that $1=\sup \{ 1\}$ (the measure of a nonempty open set, $1$, is equal to the suprema of the set of all measures of quasicompact sets contained in it, namely, $1$).  Similarly, it is outer-regular, hence regular.  On the other hand,
\begin{equation}
\meas (\{ x_1,x_2\} )=1<1+1=\meas (\{ x_1\} )+\meas (\{ x_2\} ).
\end{equation}
Therefore, one of the open sets $\{ x_1\}$ or $\{ x_2\}$ must have been nonmeasurable.
\end{exm}
Of course, there is a counter-example to the other potential implication as well.
\begin{exm}[A borel measure that is not regular]
Consider the counting measure on $\R$.  The set $[0,1]$ is quasicompact, but has infinite measure.  Therefore, the counting measure on $\R$ is not regular.  On the other hand, the cardinality of the union of two disjoint sets is the sum of the cardinalities of those two sets,\footnote{This is how we defined addition of cardinals!} and so we always have $\meas (S)=\meas (S\cap M)+\meas (S\cap M^{\comp})$, that is to say, every set is measurable, and so certainly the open sets are measurable.
\end{exm}

In general topology, I really dislike imposing unnecessary countability assumptions.  On the other hand, countability is something fundamental in measure theory, simply because of the conditions of additivity and subadditivity---see \cref{OuterMeasure}.  Furthermore, as explained there, we \emph{don't want} any stronger additivity assumptions.  Thus, in the context of \emph{measures} on topological spaces, it makes sense to impose countability conditions on our spaces.  This leads us to the following definition.
\begin{dfn}[$\sigma$-quasicompact]\label{SigmaQuasicompact}
A topological space is \emph{$\sigma$-quasicompact} iff it is the countable union of quasicompact sets.
\end{dfn}
\begin{dfn}[Topological measure space]\label{TopologicalMeasureSpace}
A \emph{topological measure space}\index{Topological measure space} is a $\sigma$-quasicompact topological space equipped with a regular borel measure.
\end{dfn}
The following is nice characterization of measurability in topological measure spaces.  It is arguably the reason why we give the conditions ``$\sigma$-quasicompact, regular, borel'' a name in the first place.
\begin{prp}\label{prp5.1.39}
Let $\coord{X,\meas}$ be a topological measure space and $S\subseteq X$.  Then, $S$ is measurable iff for every $\varepsilon >0$, there is an open set $U_{\varepsilon}$ and a closed set $C_{\varepsilon}$ such that
\begin{equation}
C_{\varepsilon}\subseteq S\subseteq U_{\varepsilon}\text{ and }\meas (U_{\varepsilon}\setminus C_{\varepsilon})<\varepsilon .
\end{equation}
\begin{proof}
Write $X=\bigcup _{m\in \N}K_m$ for $K_m\subseteq X$ quasicompact.

\blankline
\noindent
$(\Rightarrow )$  Suppose that $S$ is measurable.  Define $S_m\coloneqq S\cap K_m$.  As $S_m\subseteq K_m$, $S_m$ has finite measure because $\meas$ is regular.  Let $\varepsilon >0$.  By outer-regularity, there is some open $U_m$ containing $S_m$ such that
\begin{equation}
\meas (S_m)\leq \meas (U_m)<\meas (S_m)+\frac{\varepsilon}{2^m}.
\end{equation}
Because $S$ is measurable, it in turn follows that\footnote{See \cref{exr5.1.21}.}
\begin{equation}
\meas (U_m-S_m)<\frac{\varepsilon}{2^m}.
\end{equation}
Define $U\coloneqq \bigcup _{m\in \N}U_m$.  Then,
\begin{equation}
\meas (U\setminus S)\leq \sum _{m\in \N}\meas (U_m-S_m)<2\varepsilon .
\end{equation}
Applying this same logic to $S^{\comp}$, we can find an open set $V$ containing $S^{\comp}$ such that
\begin{equation}
\meas (V\setminus S^{\comp })<2\varepsilon .
\end{equation}
Then, $V^{\comp}$ is of course a closed subset of $S$ and
\begin{equation}
\meas (U\setminus V^{\comp})=\meas (U\cap V)\leq \meas (U\cap V\cap S)+\meas (U\cap V\cap S^{\comp})\leq \meas (V\setminus S^{\comp})+\meas (U\setminus S)<4\varepsilon .
\end{equation}

\blankline
\noindent
$(\Leftarrow )$ Suppose that for every $\varepsilon >0$, there is an open set $U_{\varepsilon}$ and a closed set $C_{\varepsilon}$ such that
\begin{equation}
C_{\varepsilon}\subseteq S\subseteq U_{\varepsilon}\text{ and }\meas (U_{\varepsilon}\setminus C_{\varepsilon})<\varepsilon .
\end{equation}
For $\varepsilon \coloneqq \frac{1}{2^m}$, let $C_m\subseteq S$ be closed and such that $\meas (S\setminus C_m)<\frac{\varepsilon}{2^m}$.  Define
\begin{equation}
C\coloneqq \bigcup _{m\in \N}C_m.
\end{equation}
$C$ is measurable because it is the countable union of closed sets, which are measurable because, by hypothesis, the measure in borel.  On the other hand, $\meas (S\setminus C)<2\varepsilon$.  As $\varepsilon$ is arbitrary, it follows that $\meas (S\setminus C)=0$, and so $S\setminus C$ is measurable.  Therefore, $S=C\cup (S\setminus C)$ is measurable.
\end{proof}
\end{prp}
Before moving onto discussion of measures on uniform spaces, we have one more adjective that is used to describe outer-measures on topological spaces.
\begin{dfn}[Topological-additivity]\label{TopologicalAdditivity}
Let $X$ be a topological space and let $\meas$ be an outer-measure on $X$.  Then, $\meas$ is \emph{topologically-additive}\index{Topologically-additive} iff whenever $S,T\subseteq X$ are separated by neighborhoods, it follows that $\meas (S\cup T)=\meas (S)+\meas (T)$.
\begin{rmk}
Note that the unit measure on a two point space with the discrete topology (see \cref{exm5.1.42}) is an example of a regular measure that is not topologically-additive.  The point is, that we \emph{do} need to make this an additional assumption---we do not get it for free.  On the other hand, note that regular borel measures are automatically topologically-additive---see \cref{exr5.1.56}
\end{rmk}
\begin{rmk}
In uniform spaces, there are substantially less levels of separation.\footnote{Well, a posteriori anyways---many of them wind up being equivalent, though this is nontrivial---see \cref{crl4.4.16}.}  Thus, in uniform spaces, there are not many ``additivity'' conditions one can place on the measure (see the definition of \emph{uniform-additivity}, \cref{UniformAdditivity}).  On the other hand, the separation axioms in topological spaces do not collapse at all (see the diagram in \eqref{4.6.105}), and so there are many ``additivity'' conditions one might consider putting on measures on topological spaces.  We choose the one we do because it winds up being equivalent to uniform-additivity---see \cref{prp5.1.61}.
\end{rmk}
\end{dfn}
\begin{exr}\label{exr5.1.56}
Let $\meas$ be a regular borel measure.  Show that $\meas$ is topologically-additive.
\end{exr}
\begin{prp}\label{prp5.1.57}
Let $\meas$ a regular topologically-additive measure on a $T_2$ space.  Then, $\meas$ is borel.
\begin{rmk}
Thus, by the previous exercise, for $T_2$ spaces, borel is equivalent to topological additivity.
\end{rmk}
\begin{rmk}
Warning:  This will fail if the space is not $T_2$---see \cref{exm5.1.43}.
\end{rmk}
\begin{rmk}
By \nameref{CaratheodorysTheorem}, it thus follows that the ``$\sigma$-algebra generated by the topology'', the \emph{borel sets}\index{Borel sets}, are measurable.  In particular, $F_\sigma$ and $G_\delta$ sets are measurable.
\end{rmk}
\begin{proof}\footnote{Proof adapted from \cite[pg.~194]{Cohn}.}
Let $U\subseteq X$ be open and let $A\subseteq X$ be arbitrary.  We wish to show that
\begin{equation}
\meas (A)\geq \meas (A\cap U)+\meas (A\cap U^{\comp}).
\end{equation}
If $\meas (A)=\infty$, this is automatically satisfied, so we may as well assume that $\meas (A)<\infty$.

Let $\varepsilon >0$.  Then, by outer-regularity, there is some open set $U_\varepsilon$ that contains $U$ and
\begin{equation}
\meas (A)\leq \meas (U_{\varepsilon})<\meas (A)+\varepsilon .
\end{equation}
Then, by inner-regularity on opens, there is some quasicompact $K_{\varepsilon}\subseteq U_{\varepsilon}\cap U$ such that
\begin{equation}
\meas (U\cap U_{\varepsilon})-\varepsilon <\meas (K_{\varepsilon})\leq \meas (U\cap U_{\varepsilon}).
\end{equation}
As $X$ is $T_2$, so that $K_{\varepsilon}$ is closed, so that $U_{\varepsilon}\cap K_{\varepsilon}^{\comp}$ is open, and so there is some compact $L_{\varepsilon}\subseteq U_{\varepsilon}\cap K_{\varepsilon}^{\comp}$ such that
\begin{equation}
\meas (U_{\varepsilon}\cap K_{\varepsilon}^{\comp})-\varepsilon <\meas (L_{\varepsilon})\leq \meas (U_{\varepsilon}\cap K_{\varepsilon}^{\comp}).
\end{equation}
Hence,
\begin{equation}
\begin{split}
\meas (A) & >\meas (U_{\varepsilon})-\varepsilon \geq \footnote{Because $K_{\varepsilon}\cup L_{\varepsilon}\subseteq U_{\varepsilon}$.}\meas (K_{\varepsilon}\cup L_{\varepsilon})-\varepsilon =\footnote{You can separate by neighborhoods disjoint compact subsets of $T_2$ spaces.  Then we apply the fact that $\meas$ is topologically-additive.}\meas (K_\varepsilon )+\meas (L_{\varepsilon})-\varepsilon \\
& >\meas (U\cap U_{\varepsilon})+\meas (U_{\varepsilon}\cap K_{\varepsilon}^{\comp})-3\varepsilon \geq \footnote{Because $U_{\varepsilon}\cap U^{\comp}\subseteq U_{\varepsilon}\cap K_{\varepsilon}^{\comp}$.}\meas (U\cap U_{\varepsilon})+\meas (U_{\varepsilon}\cap U^{\comp})-3\varepsilon \\
& \geq \footnote{Because $A\subseteq U_{\varepsilon}$.}\meas (A\cap U)+\meas (A\cap U^{\comp})-3\varepsilon .
\end{split}
\end{equation}
As $\varepsilon$ is arbitrary, we have that $\meas (A)\geq \meas (A\cap U)+\meas (A\cap U^{\comp})$, and so $U$ is measurable.
\end{proof}
\end{prp}
By definition, regularity requires inner-regularity in open sets.  In certain nice cases, however, we also get inner-regularity on measurable sets of finite measure.
\begin{prp}\label{InnerRegularFinite}
Let $\coord{X,\meas}$ be a $T_2$ topological measure space.  Then, $\meas$ is inner-regular on measurable sets of finite measure.
\begin{proof}
Write $X=\bigcup _{m\in \N}K_m$ for $K_m$ quasicompact.  Let $S\subseteq X$ be measurable and of finite measure.  Define $S_m\coloneqq S\cap K_m$.  As $X$ is $T_2$, $K$ is closed.  Because $X$ is $T_2$, $\meas$ is borel, and so closed sets are measurable.  Thus, $K$ is measurable, and so $S_m$ is measurable.
    
Let $\varepsilon >0$.  Then, there are open sets $U_{m,\varepsilon}$ and closed sets $C_{m,\varepsilon}$ such that (i) $C_{m,\varepsilon}\subseteq S_m\subseteq U_{m,\varepsilon}$ and (ii) $\meas (U_{m,\varepsilon}\setminus C_{m,\varepsilon})<\frac{\varepsilon}{2^m}$.  As $C_{m,\varepsilon}\subseteq K_m$, $C_{m,\varepsilon}$ is quasicompact, and so $L_{m,\varepsilon}\coloneqq \bigcup _{k=0}^mC_{k,\varepsilon}$ is likewise quasicompact.  Furthermore,
\begin{equation}
\begin{split}
\meas (S)-\meas (L_{m,\varepsilon}) & =\sum _{k=0}^\infty \meas (S_k)-\sum _{k=0}^m\meas (C_{k,\varepsilon})=\sum _{k=m+1}^\infty \meas (S_k)+\sum _{k=0}^m\left( \meas (S_k)-\meas (C_{k,\varepsilon})\right) \\
& \leq \sum _{k=m+1}^\infty \meas (S_k)+\sum _{k=0}^m\left( \meas (U_{k,\varepsilon})-\meas (C_{k,\varepsilon})\right) <\sum _{k=m+1}^\infty \meas (S_k)+\sum _{k=0}^m\frac{\varepsilon}{2^k} \\
& \leq \sum _{k=m+1}^\infty \meas (S_k)+2\varepsilon
\end{split}
\end{equation}
As $\meas (S)=\sum _{k=0}^\infty \meas (S_k)$ is finite, we can make this arbitrarily small by taking $m$ sufficiently large.  As $L_{m,\varepsilon}\subseteq S$ is quasicompact, we have that
\begin{equation}
\meas (S)=\sup \{ \meas (K):K\subseteq S,\ K\text{ quasicompact}\} ,
\end{equation}
and so $\meas$ is inner-regular on $S$.
\end{proof}
\end{prp}
A related result is the following.
\begin{prp}\label{Semifinite}
Let $X$ be a topological space that can be written as the countable union of quasicompact sets, let $\meas$ be a regular measure, and let $S\subseteq X$.  Then, if $\meas (S)=\infty$, then there is a subset $T\subseteq S$ with $0<\meas (T)<\infty$.
\begin{rmk}
This condition, that every set of infinite measure has a subset of finite positive measure is sometimes called \emph{semifinite}\index{Semifinite}.
\end{rmk}
\begin{proof}
Suppose that $\meas (S)=\infty$.  Write $X=\bigcup _{m\in \N}K_m$ for $K_m$ quasicompact.  Define $S_m\coloneqq S\cap K_m$.  As $K_m$ is quasicompact, it has finite measure, and so $S_m$ has finite measure.  We also have that
\begin{equation}
\infty =\meas (S)=\meas \left( \bigcup _{m\in \N}S_m\right) \leq \sum _{m\in \N}\meas (S_m),
\end{equation}
which means that the measure of at least some $S_m$ is strictly positive.
\end{proof}
\end{prp}

One might expect that nonempty open sets have positive measure in topological measure spaces.  This is not true.
\begin{exm}[An outer-measure on $\R$ for which $\coord{\R ,\meas}$ is a topological measure space, but is not strictly-positive---the dirac measure]\footnote{\emph{Stricty-positive}\index{Strictly-positive (measure)} means just that---that every nonempty open set has positive measure.  The term here presumably comes from the dirac delta function.}\index{Dirac measure}
Define $\meas :2^\R \rightarrow [0,\infty ]$ by
\begin{equation}
\meas (S)\coloneqq \begin{cases}0 & \text{if }0\notin S \\ 1 & \text{if }0\in S\end{cases}.
\end{equation}

$\meas ((0,1))=0$ for example, so this is definitely not strictly-positive.  What we need to check is that it is a topological measure.

We first check that it is regular.  $\meas$ itself is finite, and so certainly finite on quasicompact sets.  Let $U\subseteq \R$ be open.  If $0\in U$, then as $\{ 0\}$ is quasicompact, we have that
\begin{equation}
\meas (U)=1=\sup \{ \meas (K):K\subseteq U,\ K\text{ quasicompact}\} .
\end{equation}
On the other hand, if $0\notin U$, then no subset of $U$ will contain $0$, and so once again we have
\begin{equation}
\meas (U)=0=\sup \{ \meas (K):K\subseteq U,\ K\text{ quasicompact}\} .
\end{equation}
Thus, $\meas$ is inner-regular on opens.  Now let $S\subseteq \R$ be arbitrary.  If $0\in S$, then every open set which contains $S$ will also contain $0$, and so we definitely have
\begin{equation}
\meas (S)=1=\inf \{ \meas (U):S\subseteq U\, U\text{ open}\} .
\end{equation}
On the other hand, if $S$ does not contain $0$, then $\{ 0\} ^{\comp}$ is an open set containing $S$, and so
\begin{equation}
\meas (S)=0=\inf \{ \meas (U):S\subseteq U\, U\text{ open}\} .
\end{equation}
Thus, $\meas$ is outer-regular, and hence regular.

We now check that $\meas$ is borel.  To show this, obviously it suffices to show that every subset of $\R$ is measurable with respect to $\meas$.

So, let $M,S\subseteq \R$.  We would like to show that $M$ is measurable, and so we need to show that
\begin{equation}
\meas (S)=\meas (S\cap M)+\meas (S\cap M^{\comp}).
\end{equation}
There are two cases:  either $0\in M$ or $0\notin M$.  By $M\leftrightarrow M^{\comp}$, we may as well assume that $0\in M$.  Thus, we need to show that
\begin{equation}
\meas (S)=\meas (S\cap M).
\end{equation}
There are two cases:  $0\in S$ or $0\notin S$.  In the former case, this equation reads $1=1$, an in the later case it reads $0=0$.  Either way, it is true, and so $M$ is measurable.
\end{exm}

\subsubsection{Measures on uniform spaces}

\begin{dfn}[Uniform-additivity]\label{UniformAdditivity}
Let $X$ be a uniform spac and let $\meas$ be an outer-measure on $X$.  Then, $\meas$ is \emph{uniformly-additive}\index{Uniformly-additive} iff whenever $S,T\subseteq X$ are uniformly-separated, it follows that $\meas (S\cup T)=\meas (S)+\meas (T)$.
\begin{rmk}
Warning:  The term ``uniform'' most of the time is something strictly-stronger than something only topological.  This is not the case here:  topological-additivity is superficially stronger than uniform-additivity because it is easier to be separated by neighborhoods than it is to be uniformly-separated.  In particular, topologically-additive measures on uniform spaces are automatically uniformly-additive.
\end{rmk}
\begin{rmk}
The second condition is a generalization of the defining condition of what is called a \emph{metric outer-measure}\index{Metric outer-measure}.  In particular, if $X$ is a metric space, then any uniform meausre is (by definition) a metric outer-measure.
\end{rmk}
\begin{rmk}
Note that the unit measure on a two point space with the discrete topology (see \cref{exm5.1.42}) is an example of a regular measure that is not topologically-additive.  The point is, that we \emph{do} need to make this an additional assumption---we do not get it for free.
\end{rmk}
\end{dfn}
\begin{exr}
Can you find an example of a regular measure $\meas$ on a uniform space $X$ with $S,T\subseteq X$ uniformly-separated, but $\meas (S\cup T)\neq \meas (S)+\meas (T)$.
\begin{rmk}
The point is:  do we really need to assume uniform-additivity, or can we get it for free?
\end{rmk}
\begin{rmk}
Hint:  We have already encountered a counter-example that will do the trick.
\end{rmk}
\end{exr}
\begin{prp}\label{prp5.1.61}
Let $\meas$ be a regular measure on a $T_0$ uniform space.  Then, the following are equivalent.
\begin{enumerate}
\item $\meas$ is topologically-additive.
\item $\meas$ is uniformly-additive.
\item $\meas$ is borel.
\end{enumerate}
\begin{proof}
Let $X$ be the $T_0$ uniform space that $\meas$ is a regular measure on.

\blankline
\noindent
$((i)\Leftrightarrow (iii))$ $T_0$ uniform spaces are $T_2$.  Therefore, by \cref{exr5.1.56,prp5.1.57}, topological-additivity is equivalent to being borel.

\blankline
\noindent
$((i)\Rightarrow (ii))$ Suppose that $\meas$ is topologically additive.  Let $S,T\subseteq X$ be uniformly-separated.  Then, $S$ and $T$ are separated by neighborhoods, and so $\meas (S\cup T)=\meas (S)+\meas (T)$.  Thus, $\meas$ is uniformly-additive.

\blankline
\noindent
$((ii)\Rightarrow (i))$ Suppose that $\meas$ is uniformly-additive.  Let $S,T\subseteq X$  be separated by neighborhoods.  By definition, we want to show that $\meas (S\cup T)=\meas (S)+\meas (T)$.  By subadditivity, it suffices to show that $\meas (S\cup T)\geq \meas (S)+\meas (T)$.  If either one of the sets has infinite measure, then this inequality reads $\infty \geq \infty$, and so is automatically satisfied.  Thus, we may as well without loss of generality assume that $\meas (S),\meas (T)$ is finite.  Then, by outer-regularity, for every $\varepsilon >0$, there is an open set $W_{\varepsilon}$ with $S\cup T\subseteq W_{\varepsilon}$ and
\begin{equation}
\meas (S\cup T)\leq \meas (W_{\varepsilon})<\meas (S\cup T)+\varepsilon .
\end{equation}

On the other hand, because $S$ and $T$ are separated by neighborhoods, we know that there are disjoint open sets $U$ and $V$ with $S\subseteq U$ and $T\subseteq V$.  Define $U_\varepsilon \coloneqq U\cap W_{\varepsilon}$ and $V_\varepsilon \coloneqq V\cap W_{\varepsilon}$, so that $U_{\varepsilon}$ and $V_{\varepsilon}$ are both disjoint and open.  By the previous result, $U_{\varepsilon}$ and $V_{\varepsilon}$ are measurable, and so we have
\begin{equation}
\meas (U_{\varepsilon}\cup V_{\varepsilon})=\meas (U_{\varepsilon})+\meas (V_{\varepsilon}).
\end{equation}
Thus,
\begin{equation}
\meas (S\cup T)>\meas (W_{\varepsilon})-\varepsilon \geq \footnote{Because $U_{\varepsilon}\cup V_{\varepsilon}\subseteq W_{\varepsilon}$.}\meas (U_{\varepsilon}\cup V_{\varepsilon})-\varepsilon =\meas (U_{\varepsilon})+\meas (V_{\varepsilon})-\varepsilon \geq \footnote{Because $S\subseteq U_{\varepsilon}$ and $T\subseteq V_{\varepsilon}$.}\meas (S)+\meas (T)-\varepsilon .
\end{equation}
It follows that $\meas (S\cup T)\geq \meas (S)+\meas (T)$, and we are done.
\end{proof}
\end{prp}
One important fact about uniformly-additive measures is that the open sets are automatically measurable.
\begin{prp}\label{prp5.1.25}
Let $\meas$ be a uniformly-additive on a $T_0$ uniform space.  Then, $\meas$ is a borel measure.
\begin{proof}
By the previous result, $\meas$ is a topologically-additive.  A $T_0$ uniform space is $T_2$.  Therefore, by \cref{prp5.1.57} (topologically-additive measures on $T_2$ spaces are borel), $\meas$ is borel.
\end{proof}
\end{prp}
\begin{exm}[A topologically-additive regular on a uniform space that is not borel]\footnote{Topologically-additive measures on uniform spaces are automatically uniformly-additive---see the remark in \cref{UniformAdditivity}.}\label{exm5.1.43}
Define $X\coloneqq \{ x_1,x_2,x_3\}$, and equip it uniform base with just one cover, $\mathcal{B}\coloneqq \{ \{ x_1,x_2\} ,\{ x_3\} \}$.  (This is a uniform base because this single cover star-refines itself.)  The neighborhood bases defined by this uniform base are
\begin{equation}
\begin{split}
\mathcal{B}_{x_1} & =\{ \Star _{\mathcal{B}}(x_1)\} =\{ \{ x_1,x_2\} \} \\
\mathcal{B}_{x_2} & =\{ \Star _{\mathcal{B}}(x_2)\} =\{ \{ x_1,x_2\} \} \\
\mathcal{B}_{x_3} & =\{ \Star _{\mathcal{B}}(x_3)\} =\{ \{ x_3\} \} .
\end{split}
\end{equation}
In particular, the open sets are\footnote{Recall that, for a topology defined by a neighborhood base (see \cref{NeighborhoodBase}, a set $U$ is open iff every point $x\in U$ has an element $B\in \mathcal{B}_x$ such that $x\in B\subseteq U$.  As our set only contains $3$ points, you can simply verify by hand that these are the only open sets.}
\begin{equation}
\emptyset ,X,\{ x_1,x_2\} ,\{ x_3\} .
\end{equation}

We define an outer-measure $\meas :2^X\rightarrow [0,\infty ]$ by
\begin{equation}
\meas (S)\coloneqq \begin{cases}0 & \text{if }S=\emptyset ,\{ x_3\} \\ 1 & \text{otherwise}\end{cases}.
\end{equation}
Every set has finite measure, and so certainly every quasicompact set does.  There are only four open sets to check, and you can simply check by hand that $\meas$ is inner-regular on each one of them.  Measures are outer-regular on open sets, and so we only need to check outer-regularity on the remaining $8-4=$ sets that are not open.  Once again, there are only four things to check, and so you can just do so by hand.

We now check that this measure is uniformly-additive.  If $S$ and $T$ are uniformly-separated, then, without loss of generality, we have that $S\subseteq \{ x_1,x_2\}$ and $T\subseteq \{ x_3\}$.  We automatically have that $\meas (S\cup T)=\meas (S)+\meas (T)$ if either $S$ or $T$ is empty, so we may without loss of generality assume that $T=\{ x_3\}$ and $S\subseteq \{ x_1,x_2\}$ is nonempty.  Then,
\begin{equation}
\meas (S\cup T)=1=1+0=\meas (S)+\meas (T),
\end{equation}
and so the measure is uniformly-additive.

On the other hand, it is not borel, because $U\coloneqq \{ x_1,x_2\}$ is not measurable, as we now check.  Take $S\coloneqq \{ x_1\}$.  Then,
\begin{equation}
\meas (S)=1\neq 1+1=\meas (S\cap U)+\meas (S\cap U^{\comp}).
\end{equation}
\end{exm}

\subsubsection{A summary}

Before finally moving on to the \nameref{HaarHowesTheorem}, let us briefly summarize.
\begin{enumerate}
\item Borel means that open sets are measurable---see \cref{BorelMeasure}.
\item Regular means quasicompact sets have finite measure, inner-regularity on opens, and outer-regularity on everything---see \cref{RegularMeasure}.
\item A topological space is $\sigma$-quasicompact iff it is the countable union of quasicompact sets---see \cref{SigmaQuasicompact}.
\item A topological measure space is a $\sigma$-quasicompact topological space equipped with a regular borel measure---see \cref{TopologicalMeasureSpace}.
\item Topological-additivity means that the measure is additive on two sets which are separated by neighborhoods---see \cref{TopologicalAdditivity}.
\item Uniform-additivity means that the measure is additive on two sets which are uniformly-separated---see \cref{UniformAdditivity}.
\item Neither regular topologically-additive nor regular uniformly-additive measures need be borel (\cref{exm5.1.43}), but in fact for $T_2$ spaces this is equivalent to being borel---see \cref{prp5.1.57,prp5.1.25}.
\end{enumerate}

\subsection{The Haar-Howes Theorem}

It turns out that there is a theorem, the \nameref{HaarHowesTheorem},\footnote{Warning:  This is the name I have chosen to call it, because I am unaware of another name for this result.  In particular, don't expect others to know what you're talking about if you reference this theorem by name.  (It is a generalization of the existence and essential uniqueness of haar measure, in case you are more familiar with that.)  While the proof of the result I cobbled together from other sources, the formulation of the result I found essentially in \cite{Howes}, in which he claims ``the author'' established essential uniqueness, as well as existence, which he and another mathematician (Izkowitz) established independently.}  be that is quite general and will just spit out regular uniformly-additive for us.  This is how we will construct lebesgue measure.  That being said, it doesn't just work for any old uniform space.  We're going to need extra structure:
\begin{dfn}[Isogeneous space]\label{IsogeneousSpace}
An \emph{isogeneous space} is a uniform space $X$ equipped with a group of uniform-homeomorphisms $\Phi$ such that
\begin{equation}\label{5.1.28}
\widetilde{\mathcal{B}}_\Phi \coloneqq \{ \mathcal{B}_U\} \text{ where }\mathcal{B}_U\coloneqq \{ \phi (U):\phi \in \Phi\} \text{ and }U\subseteq X\text{ open}.
\end{equation}
is a uniform base for $X$.
\begin{rmk}
$\Phi$ is called the \emph{group of symmetries} of $X$.  It is a subgroup of $\Aut _{\Uni}(X)$.  $\widetilde{\mathcal{B}}_{\Phi}$ is the \emph{isogeneous base}\index{Isogeneous base} and each $\mathcal{B}_U$ is an \emph{isogeneous cover}\index{Isogeneous cover}.
\end{rmk}
\begin{rmk}
The example you should have in mind here is that of a topological group $G$.  In this case, the uniform base is the canonical one, $\widetilde{\mathcal{B}}\coloneqq \{ \mathcal{B}_U\}$ with $\mathcal{B}_U\coloneqq \{ gU:g\in U\}$ for $U$ an open neighborhood of the identity, and the set of all uniform homeomorphisms is the set of all left translations, $\Phi \coloneqq \{ \phi _g:g\in G\}$ where $\phi _g(x)\coloneqq gx$.  Of course, you can also choose right-translations over left-translations (in both $\widetilde{\mathcal{B}}$ and $\Phi$!) if you so desire.
\end{rmk}
\begin{rmk}
What do you think the morphisms of isogeneous spaces should be?
\end{rmk}
\end{dfn}
\begin{exr}\label{exr5.1.48}
Let $X$ be a uniform space with uniform topology $\mathcal{U}$, let $\mathcal{B}$ be a base for the topology, and let $\Phi$ be a subgroup of $\Aut _{\Uni}(X)$.  Show that
\begin{equation}\label{5.1.49}
\left\{ \mathcal{B}_B:B\in \mathcal{B}\right\}
\end{equation}
is a uniform base iff $\widetilde{\mathcal{B}}_\Phi$ is.
\begin{rmk}
The point is that, in order to show that $\Phi$ makes $X$ into an isogeneous space, we only need to check that the smaller collection of covers in \eqref{5.1.49}, the covers coming from only elements of the base instead of \emph{all} open sets, form a uniform base.
\end{rmk}
\end{exr}
\begin{exr}
Let $G$ be a topological group and let $\Phi \coloneqq \{ \phi _g:G\in G\}$, where $\phi _g:G\rightarrow G$ is defined by $\phi _g(x)\coloneqq gx$.  Show that $\coord{G,\Phi}$ is an isogeneous space.
\end{exr}
\begin{dfn}[Uniformly-measurable]\label{UniformlyMeasurable}
Let $\meas :2^X\rightarrow [0,\infty ]$ be an outer-measure on a set $X$ and let $\mathcal{U}$ is a cover of $X$.  Then, $\mathcal{U}$ is \emph{uniformly-measurable}\index{Uniformly-measurable cover} iff $\meas$ is constant on $\mathcal{U}$.  A uniform base consisting of uniformly-measurable covers is a \emph{uniformly-measurable base}\index{Uniformly-measurable base}.
\begin{rmk}
Think about what having a uniform base of uniformly-measurable covers means for a metric space---if we take as a uniform base the collection of all covers by $\varepsilon$-balls, then this is just the statement that every $\varepsilon$-ball has to have the same measure.
\end{rmk}
\begin{rmk}
Note that you definitely do not want to require \emph{every} uniform cover be uniformly-measurable.  For example, in a metric space, by upward-closedness the collection of all $\varepsilon$-balls together with a single $2\varepsilon$-ball will also be a uniform-cover---we definitely do not want to require that a $2\varepsilon$ ball has the same measure as an $\varepsilon$-ball.
\end{rmk}
\end{dfn}
\begin{dfn}[Isogeneous measure]
Let $\coord{X,\Phi}$ be an isogeneous space.  A \emph{isogeneous measure}\index{Isogeneous measure} is a uniformly-additive regular measure for which $\widetilde{\mathcal{B}}_\Phi$ is a uniformly-measurable base.
\begin{rmk}
Explicitly, this means that
\begin{enumerate}
\item $\meas$ is regular;
\item $\meas$ is uniformly-additive; and
\item $\meas (h(U))=\meas (U)$ for $h\in U$ and $U\subseteq X$ open.
\end{enumerate}
\end{rmk}
\end{dfn}
\begin{exr}
Let $\meas$ be an isogeneous measure on an isogeneous space $\coord{X,\Phi}$.  Show that $\meas (\phi (S))=\meas (S)$ for all $\phi \in \Phi$ and $S\subseteq X$.
\begin{rmk}
In other words, this holds for \emph{all} $S$, not just open $S$.
\end{rmk}
\end{exr}
\begin{exr}
Let $\meas$ be an isogeneous measure on an isogeneous space $\coord{X,\Phi}$, let $M\subseteq X$ be measurable, and let $\phi \in \Phi$.  Show that $\phi (M)$ is measurable.
\begin{rmk}
In other words, the symmetries of the isogeneous space preserve measurability (in particular, for lebesgue measure, we will have that isometries of $\R ^d$ preserve measurability---see \cref{LebesgueMeasure} (the definition of lebesgue measure)).
\end{rmk}
\end{exr}

And finally now the key result that will allow us to define basically every measure we work with in these notes (and much more).
\begin{thm}[Haar-Howes Theorem]\index{The Haar-Howes Theorem}\label{HaarHowesTheorem}
\begin{savenotes}
Let $\coord{X,\Phi}$ be a $T_0$ locally quasicompact\footnote{Hence locally compact as $T_0$ implies uniformly-completely-$T_3$ implies $T_2$ and subspaces of $T_2$ spaces are $T_2$.} isogeneous space and let $K\subseteq X$ be quasicompact with nonempty interior.. Then, there exists a unique isogeneous $\meas$ on $X$ such that $\meas (K)=1$.
\begin{rmk}
You should think of $K$ has a set with which we can `compare' all other sets to get a ``measure'' of `size'.  The condition that it be quasicompact you can think of the condition that the measure of $K$ be finite, and the condition that it have nonempty interior you can think of the condition that the measure of $K$ be positive.  If $\meas (K)$ is neither infinite nor zero, then we can `normalize' to get $\meas (K)=1$.
\end{rmk}
\begin{rmk}
Note that in fact, by \cref{prp5.1.61}, the resulting measure will actually turn-out to be topologically-additive.
\end{rmk}
\begin{rmk}
Classically, the term ``haar measure'' is reserved for $G$ a $T_0$ locally quasicompact group with $\Phi$ the set of all left-translations.  In particular, \emph{lebesgue measure} will be the haar measure for the topological group $\coord{\R ,^d+,0,-}$.\footnote{If the group is commutative, the symmetries by left translation and right translations are the same, so left vs.~right does not matter.}
\end{rmk}
\begin{proof}
\Step{Make hypotheses and introduce notation}
Suppose that $X$ has a quasicompact set $K_0$ with nonempty interior.  Denote the uniform topology on $X$ by $\mathcal{U}$ and denote the collection of all quasicompact subsets of $X$ by $\mathcal{K}$

\Step{Define $(K:U)$ for $K\in \mathcal{K}$ and $U\in \mathcal{U}$}
The cover $\mathcal{B}_U\coloneqq \{ h(U):h\in H\}$ is an open cover of $K$, and so there are a finite subcover.  Let $(K:U)$ denote the cardinality of the smallest such subcover.

\Step{Define $\mathrm{H}_U:\mathcal{K}\rightarrow \R _0^+$}
For $U\in \mathcal{U}$, define $\mathrm{H}_U:\mathcal{K}\rightarrow \R _0^+$ by
\begin{equation}
\mathrm{H}_U(K)\coloneqq \frac{(K:U)}{(K_0:U)}.\footnote{$K_0$ is nonempty, and so cannot be covered by anything empty.  Therefore, $(K_0:U)\geq 1$, and in particular, is not $0$.}
\end{equation}

\Step{Show that $\mathrm{H}_U(K)\leq (K:\Int (K_0))$}
We now check that $\mathrm{H}_U(K)\leq (K:\Int (K_0))$, that is, $(K:U)\leq (K:\Int (K_0))(K_0:U)$.  Let us temporarily write $m\coloneqq (K:\Int (K_0))$ and $n\coloneqq (K_0:U)$.  There are thus $\phi _1,\ldots ,\phi _m\in \Phi$ such that $\{ \phi _1(\Int (K_0)),\ldots ,\phi _m(\Int (K_0))\}$ covers $K$.  There are also $\phi _1',\ldots ,\phi _n'\in \Phi$ such that $\{ \phi _1'(U),\ldots ,\phi _n'(U)\}$ covers $K$.  Therefore,
\begin{equation}
K\subseteq \bigcup _{k=1}^mh_k(\Int (K_0))\subseteq \bigcup _{k=1}^mh_k(K_0)\subseteq \bigcup _{k=1}^mh_k\left( \bigcup _{l=1}^nh_l'(U)\right) =\bigcup _{k=1}^m\bigcup _{l=1}^n[h_k\circ h_l'](U)
\end{equation}
Hence, $K$ is covered by $mn$ elements of $\mathcal{B}_U$,\footnote{Here we are using the fact that $\Phi$ is closed under composition, so that $\phi _k\circ \phi _l'\in \Phi$.} and hence $(K:U)\leq mn\coloneqq (K:\Int (K_0))(K_0:U)$.

\Step{Define $\mathrm{H}:\mathcal{K}\rightarrow \R _0^+$}\label{Haar.4x}
Define $\mathcal{H}\coloneqq \prod _{K\in \mathcal{K}}[0,(K:K_0)]$.  Each $\mathrm{H}_U$ may be thought of as a point in $\mathcal{H}$, whose component at $K\in \mathcal{K}$ is $\mathrm{H}_U(K)\in [0,(K:K_0)]$.\footnote{That was sort of the point of the previous step.}  Thus, for $U\in \mathcal{U}$, let us define
\begin{equation}
C_U\coloneqq \Cls \left( \left\{ \mathrm{H}_V:\mathcal{U}\ni V\subseteq U\right\} \right) 
\end{equation}
and
\begin{equation}
\mathcal{C}\coloneqq \{ C_U:U\in \mathcal{U}\} .
\end{equation}
We wish to show that the intersection of any finitely many elements of $\mathcal{C}$ is nonempty.  Then, because $\mathcal{H}$ is quasicompact by \nameref{TychnoffsTheorem} (\cref{TychnoffsTheorem}), it will follow that the intersection over \emph{all} elements in $\mathcal{C}$ will be nonempty.

This is actually really easy, however, because for $U_1,\ldots ,U_m\in \mathcal{U}$, we have that
\begin{equation}
\mathrm{H}_{U_1\cap \cdots \cap U_m}\in \bigcap _{k=1}^mC_{U_k}.
\end{equation}
Therefore, by quasicompactness, there is some
\begin{equation}
\mathrm{H}\in \bigcap _{U\in \mathcal{U}}C_U.
\end{equation}

\Step{Show that $\mathrm{H}(K_1)\leq \mathrm{H}(K_2)$ if $K_1\subseteq K_2$}\label{Haar.4}
Let $K_1,K_2\in \mathcal{K}$ be such that $K_1\subseteq K_2$.  We first show that, for each $U\in \mathcal{U}$, $\mathrm{H}_U(K_1)\leq \mathrm{H}_U(K_2)$.  But this is trivial, because the covering of $K_2$ with $(K_2:U)$ elements of $\mathcal{B}_U$ is also a covering of $K_1$ with $(K_2:U)$ elements of $\mathcal{B}_U$, so that $(K_1:U)\leq (K_2:U)$, and hence $\mathrm{H}_U(K_1)\leq \mathrm{H}_U(K_2)$.

Thinking of elements $f$ of $\mathcal{H}$ as functions from $\mathcal{K}$ to $\R$, consider the map\footnote{For each $K_1,K_2\in \mathcal{K}$ with $K_1\subseteq K_2$, we have such a map.} that sends $f\in T$ to $f(K_2)-f(K_1)$.  This is a composition of continuous functions, and hence continuous.\footnote{The first map from $\mathcal{H}$ into $\R \times \R$ is the projection of $f\in T$ onto the $K_1^{\text{th}}$ coordinate in the first coordinate and the projection of $f\in T$ onto $K_2^{\text{th}}$ coordinate in the second coordinate.  This map is continuous because it is continuous in each coordinate.  Each coordinate is continuous because projections are continuous.  The first map is followed by the map from $\R \times \R$ into $\R$ given by subtraction, which is continuous because we know that $\coord{R,+,0,-}$ is a topological group.}  This map is also nonnegative on each $C_U$ because $\mathrm{H}_U(K_1)\leq \mathrm{H}_U(K_2)$ for each $U\in \mathcal{U}$ (we need continuity so that we know it is nonnegative on the \emph{closure} of $\{ \mathrm{H}_V:V\subseteq U\}$).  As $\meas$ is an element of each $C_U$, it follows that this map is also nonnegative at $\meas$, so that $\mathrm{H}(K_1)\leq \mathrm{H}(K_2)$.

\Step{Show that $\mathrm{H}(K_1\cup K_2)\leq \mathrm{H}(K_1)+\mathrm{H}(K_2)$}\label{Haar.5}
Let $K_1,K_2\in \mathcal{K}$.  We first show that $\mathrm{H}_U(K_1\cup K_2)\leq \mathrm{H}_U(K_1)+\mathrm{H}_U(K_2)$ for each $U\in \mathcal{U}$.  This is trivial, because a covering of $K_1$ with $(K_1:U)$ elements of $\mathcal{B}_U$ together with a covering of $K_2$ with $(K_2:U)$ elements of $\mathcal{B}_U$ is a cover of $K_1\cup K_2$ with $(K_1:U)+(K_2:U)$ elements of $\mathcal{B}_U$, so that $(K_1\cup K_2:U)\leq (K_1:U)+(K_2:U)$.  It follows that $\mathrm{H}_U(K_1\cup K_2)\leq \mathrm{H}_U(K_1)+\mathrm{H}_U(K_2)$.

Proceeding similarly as in \cref{Haar.4}, the map that sends $f\in T$ to $f(K_1)+f(K_2)-f(K_1\cup K_2)$ is continuous and nonnegative on each $C_U$, and hence is nonnegative for $\mathrm{H}\in T$.  Thus, $\mathrm{H}(K_1\cup K_2)\leq \mathrm{H}(K_1)+\mathrm{H}(K_2)$.

\Step{Show that $\mathrm{H}_U(K_1\cup K_2)=\mathrm{H}_U(K_1)+\mathrm{H}_U(K_2)$ if $K_1$ and $K_2$ are uniformly-separated}\label{Haar.8}
Let $K_1,K_2\in \mathcal{K}$ be uniformly-separated with respect to $\mathcal{B}_U$.  We have already shown that $\mathrm{H}_U(K_1\cup K_2)\leq \mathrm{H}_U(K_1)+\mathrm{H}_U(K_2)$, so it suffices to show that $\mathrm{H}_U(K_1)+\mathrm{H}_U(K_2)\leq \mathrm{H}_U(K_1\cup K_2)$.  In other words, it suffices to show that $(K_1:U)+(K_2:U)\leq (K_1\cup K_2:U)\eqqcolon m$.  Let $\phi _1(U),\ldots ,\phi _m(U)\in \mathcal{B}_U$ be a cover of $K_1\cup K_2$.  By hypothesis,\footnote{This is the definition of uniformly-separated---see \cref{UniformlySeparated}.} every single one of these can only intersect $U_1$ or $U_2$, but not both.  Thus, after relabeling if necessary, the first $k$ of these guys will form a cover of $K_1$ and the latter $m-k$ will form a cover of $K_2$.  Thus, $(K_1:U)\leq k$ and $(K_2:U)\leq m-k$, and so $(K_1:U)+(K_2:U)\leq k+(m-k)=m\coloneqq (K_1\cup K_2:U)$, which completes this step.

\Step{Define the measure $\meas$ on all open subsets of $X$}
For $U\subseteq X$ open, define
\begin{equation}\label{5.1.46}
\meas (U)\coloneqq \sup \{ \mathrm{H}(K):K\subseteq U,\ K\in \mathcal{K}\} .
\end{equation}

\Step{Extend $\meas$ to all subsets of $X$}
Now, for an arbitrary subsets $S$ of $X$, define
\begin{equation}\label{5.1.38}
\meas (S)\coloneqq \inf \{ \meas (U):S\subseteq U,\ U\in \mathcal{U}\} .
\end{equation}
\begin{exr}
Show that this agrees with \eqref{5.1.46} when $S$ is open, so that this is indeed an extension.
\end{exr}

\Step{Show that $\meas$ is an outer-measure}
\begin{exr}
Check that $\meas (\emptyset )=0$ and that $\meas$ is nondecreasing.
\end{exr}

We now check that it is subadditive.  To prove this, we will first need a lemma.
\begin{lma}
Let $X$ be $T_2$, let $K\subseteq X$ be quasicompact, and let $U_1,U_2\subseteq X$ be open and such that $K\subseteq U_1\cup U_2$.  Then, there are quasicompact subsets $K_1,K_2\subseteq X$ such that (i) $K_1\subseteq U_1$, (ii) $K_2\subseteq U_2$, and (iii) $K=K_1\cup K_2$.
\begin{proof}
We leave this as an exercise.
\begin{exr}
Complete the proof yourself.
\end{exr}
\end{proof}
\end{lma}
Having (hopefully) proved the lemma, we now show subadditivity for \emph{open} sets.  (We will then prove subadditivity in general.)  So, let $\{ U_m:m\in \N \}$ be a countable collection of open sets of X.  Let $K\subseteq \bigcup _{m\in \N}U_m$.  Then, there is some $m_K\in \N$ such that $K\subseteq \bigcup _{k=1}^{m_K}U_k$.  By applying this lemma inductively then, we may find quasicompact sets $K_1,\ldots ,K_m$ such that (i) $K_k\subseteq U_k$ for $1\leq k\leq m$ and $K=\bigcup _{k=1}^mK_k$.  Using the fact that we have already proved finite `subadditivity' (of $\mathrm{H}$) for quasicompact sets (\cref{Haar.5}), we find that
\begin{equation}
\mathrm{H}(K)\leq \sum _{k=1}^m\mathrm{H}(K_k)\leq \sum _{k=1}^m\meas (U_k)\leq \sum _{m\in \N}\meas (U_m).
\end{equation}
Taking the $\sup$ of $K$, we find that
\begin{equation}
\meas \left( \bigcup _{m\in \N}U_m\right) \coloneqq \sup \left\{ \mathrm{H}(K):K\subseteq \bigcup _{m\in \N}U_m,\ K\in \mathcal{K}\right\} \leq \sum _{m\in \N}\meas (U_m).
\end{equation}

Having proved subadditivity for open sets, we now prove it for arbitrary sets.  So, let $\{ S_m:m\in \N \}$ be an arbitrary countable collection of subsets of $X$.  If $\sum _{m\in \N}\meas (S_m)=\infty$, then there is nothing to show, and so we may as well suppose without loss of generality that $\sum _{m\in \N}\meas (S_m)<\infty$.  Let $\varepsilon >0$ and for each $m\in \N$ pick an open set $U_m$ such that (i) $S_m\subseteq U_m$ and (ii) $\meas (S_m)\leq \meas (U_m)<\meas (S_m)+\frac{\varepsilon}{2^m}$.  Then, using subadditivity for open sets, we find
\begin{equation}
\meas \left( \bigcup _{m\in \N}S_m\right) \leq \meas \left( \bigcup _{m\in \N}U_m\right) \leq \sum _{m\in \N}\meas (U_m)<\sum _{m\in \N}\left[ \meas (S_m)+\tfrac{\varepsilon}{2^m}\right] =\sum _{m\in \N}\meas (S_m)+2\varepsilon .
\end{equation}
Hence, as $\varepsilon >0$ was arbitrary, we have that
\begin{equation}
\meas \left( \bigcup _{m\in \N}S_m\right) \leq \sum _{M\in \N}\meas (S_m).
\end{equation}
Thus, $\meas$ is an outer-measure on $X$.

\Step{Show that each $\mathcal{B}_U$ is uniformly-measurable with respect to $\meas$}
Let $\phi \in \Phi$.  We want to show that $\meas (\phi (U))=\meas (U)$.  Then, for any other $\phi '\in \Phi$, we will have that $\meas (\phi (U))=\meas (U)=\meas (\phi '(U))$, so that indeed every element of $\mathcal{B}_U$ has the same measure.  However, $K$ is a quasicompact set contained in $U$ iff $\phi (K)$ is a quasicompact set contained in $\phi (U)$.\footnote{This implicitly uses the fact that $\phi ^{-1}\in \Phi$.}  Therefore, by the definition of $\meas (U)$ \eqref{5.1.46} it suffices to show that $\mathrm{H}(K)=\mathrm{H}(h(K))$.  To show this, we first show that $\mathrm{H}_U(K)=\mathrm{H}_U(h(K))$ for all $U\in \mathcal{U}$.  That is, we would like to show that $(K:U)=(\phi (K):U)$.  However, every cover of $K$ by elements of $\mathcal{B}_U$, $\phi _1(U),\ldots ,\phi _m(U)$, gives a cover of $\phi (K)$ by elements of $\mathcal{B}_U$ of the same cardinality, $\phi (\phi _1(U)),\ldots ,\phi (\phi _m(U))$.  It thus follows that $\mathrm{H}_U(K)=\mathrm{H}_U(h(K))$.

For $\phi \in \Phi$ fixed, consider the map from $\mathcal{H}$ to $\R$ that sends $f$ to $f(\phi (K))-f(K)$.  We just showed that this is $0$ on each $\mathrm{H}_U\in T$, and so it is $0$ on $C_U$, and so it is $0$ on $\mathrm{H}$, that is, $\mathrm{H}(K)=\mathrm{H}(h(K))$.

\Step{Show that if $S$ and $T$ are uniformly-separated, then $\meas (S\cup T)=\meas (S)+\meas (T)$}
Note that we always have that $\meas (S\cup T)\leq \meas (S)+\meas (T)$, and so it suffices to show that $\meas (S\cup T)\geq \meas (S)+\meas (T)$.

We first prove this for open sets.  So, let $U,V\in \mathcal{U}$.  If either $U$ or $V$ s has infinite measure, then this inequality just reads $\infty \geq \infty$, and is so automatically satisfied.  Thus, without loss of generality, assume that $\meas (U),\meas (V)<\infty$.  Let $\varepsilon >0$.  Then, there is are some $K,L\in \mathcal{K}$ such that $K\subseteq U$, $L\subseteq V$, and
\begin{equation}
\meas (U)-\varepsilon <\mathrm{H}(K)\leq \meas (U)\text{ and }\meas (V)-\varepsilon <\mathrm{H}(L)\leq \meas (V).
\end{equation}
If $U$ and $V$ are uniformly-separated, then certainly $K$ and $L$ are uniformly-separated, and so by \cref{Haar.8}, we have that
\begin{equation}
\mathrm{H}(K\cup L)=\mathrm{H}(K)+\mathrm{H}(L),
\end{equation}
and so
\begin{equation}
\meas (U\cup V)\geq \mathrm{H}(K\cup L)=\mathrm{H}(K)+\mathrm{H}(L)>\meas (U)+\meas (V)-2\varepsilon .
\end{equation}
Hence, $\meas (U\cup V)\geq \meas (U)+\meas (V)$.

We now do the general case.  Once again, if either $S$ or $T$ has infinite measure, we are done, so we may as well suppose that $\meas (S),\meas (T)<\infty$.

Our first order of business it to show that there are \emph{some} open sets containing $S$ and $T$ respectively which are uniformly-separated.

Look at any open cover $\mathcal{B}$ which positively-separates $S$ and $T$, and take an open star-refinement $\mathcal{C}$ of this\footnote{Note that every $\mathcal{B}_U$ is an open cover, so there is no need to say ``open'' here---it is just to clarify.  We don't write $\mathcal{B}_U$ to simplify the notation (and also because we will want to write $U$ for something else).}.  Define $U\coloneqq \Star _{\mathcal{C}}(S)$ and $V\coloneqq \Star _{\mathcal{C}}(T)$.  We wish to show that $U$ and $V$ are uniformly-separated with respect to $\mathcal{C}$.  Because $\mathcal{B}$ positively-separates $S$ and $T$, by definition (see \cref{UniformlySeparated}), we have that $\Star _{\mathcal{B}}(S)$ and $\Star _{\mathcal{B}}(T)$ are disjoint.  Therefore, it suffices to show that $\Star _{\mathcal{C}}(U)\subseteq \Star _{\mathcal{B}}(S)$ (and similarly for $V$).  So, suppose that $C\in \mathcal{C}$ intersects $U$.  Then, by definition of $U$, it must intersect some element $C'\in \mathcal{C}$ which intersects $S$.  Let $B\in \mathcal{B}$ be such that $\Star _{\mathcal{C}}(C')\subseteq B$.  We then have that
\begin{equation}
C\subseteq \footnote{Because $C$ and $C'$ intersect.}\Star _{\mathcal{C}}(C')\subseteq B\subseteq \footnote{Because $B$ contains $C'$, which intersects $S$.}\Star _{\mathcal{B}}(S).
\end{equation}
Thus, indeed, $\Star _{\mathcal{C}}(U)\subseteq \Star _{\mathcal{B}}(S)$.

So, let $U,V\in \mathcal{U}$ be open sets containing $S$ and $T$ respectively which are uniformly-separated.  Let $\varepsilon >0$, and choose $W\in \mathcal{U}$ that contains $S\cup T$ and satisfies
\begin{equation}
\meas (S\cup T)\leq \meas (W)<\meas (S\cup T)+\varepsilon .
\end{equation}
Let us replace $U$ and $V$ by $U\cap W$ and $V\cap W$---upon doing so, it will still be the case that $U,V\in \mathcal{U}$, it will still be the case that $S\subseteq U$ and $T\subseteq V$, and it will still be the case that $U$ and $V$ are uniformly-separated, but now we will also have that $\meas (U\cup V)\leq \meas (W)$.  Now we have
\begin{equation}
\meas (W)<\meas (S\cup T)+\varepsilon \leq \meas (S)+\meas (T)+\varepsilon \leq \meas (U)+\meas (V)+\varepsilon =\meas (U\cup V)+\varepsilon \leq \meas (W)+\varepsilon .
\end{equation}
Thus, we do indeed have that
\begin{equation}
\meas (S\cup V)=\meas (S)+\meas (T).
\end{equation}

In particular, we have now shown that $\widetilde{\mathcal{B}}_\Phi$ is a uniformly-measurable base for $\meas$ and that $\meas$ is additive for uniformly-separated sets, so that indeed $\meas$ is a uniformly-additive on $X$.

It remains to show that $\meas$ is regular.

\Step{Show that $\mathrm{H}(K)\leq \meas (K)$}
Let $U\in \mathcal{U}$ contain $K$.  Then, by the definition of $\meas (U)$, \eqref{5.1.46}, we have that $\mathrm{H}(K)\leq \meas (U)$.  Taking the infimum over all such $U$, we obtain $\mathrm{H}(K)\leq \meas (K)$.

\Step{Show that $\meas$ is regular}
The first thing we check is that $\meas (K)<\infty$ for $K$ quasicompact.  By \cref{prp5.2.4}, there is some open $U\subseteq X$ containing $K$ with compact closure.  Hence, for $K\subseteq K'\subseteq U$, with $K'$ quasicompact, we have
\begin{equation}
\mathrm{H}(K')\leq \mathrm{H}(\Cls (U))<\infty .\footnote{Recall that $\mathrm{H}$ is always finite, by definition.}
\end{equation}
Taking the supremum over such $K'$, we have that $\meas (U)\leq \mathrm{H}(\Cls (U))$, and so, as $\meas (K)\leq \meas (U)$ (because $\meas$ is an outer-measure), $\meas (K)$ is finite.

$\meas$ is outer-regular by definition \eqref{5.1.38}.

We now turn to inner-regular on open subsets.  This is \emph{almost} true by the definition \eqref{5.1.46}, but we don't have $\mathrm{H}(K)=\meas (K)$.  However, we actually don't need this---we only ned $\mathrm{H}(K)\leq \meas (K)$.  To show this, let $U\in \mathcal{U}$ contain $K$.  Then, by the definition of $\meas (U)$, \eqref{5.1.46}, we have that $\mathrm{H}(K)\leq \meas (U)$.  Taking the infimum over all such $U$, we obtain $\mathrm{H}(K)\leq \meas (K)$.

Thus, $\meas$ is regular.

\Step{Show that $\meas (K_0)=1$}
First of all, from the definition, we have that $\mathrm{H}_U(K_0)=1$ for all $U\in \mathcal{U}$.  Perhaps we could try harder and show that we already do have that $\meas (K_0)=1$; however, this enough is to show simply that $\meas (K_0)>0$, and so by simply dividing by $\meas (K_0)$ if necessary, we obtain a regular uniformly-additive measure with uniformly-measurable base $\widetilde{\mathcal{B}}_\Phi$ and $\meas (K_0)=1$.

\Step{Show that $\meas$ is unique}
Let $\meas '$ be another regular uniformly-additive on $X$ with uniformly-measurable base $\widetilde{\mathcal{B}}_\Phi$ and $\meas (K_0)=1$.  As both $\meas$ and $\meas '$ are outer-regular, it suffices to show that they agree on open sets.  Then, because they are both inner-regular on open sets, it suffices to show that they both agree on quasicompact subsets.  However, on account of \cref{prp5.2.4} and outer-regularity, it in turn suffices to show that they agree on open sets with compact closure.  So, let $\mathcal{G}$ denote the collection of all open sets with compact closures.  Furthermore, let us define
\begin{equation}
\begin{multlined}
Z(\mathcal{G})\coloneqq \\ \left\{ S\in 2^X:\text{for every }\varepsilon >0\text{ there are }C_{\varepsilon}\text{ closed and }U_{\varepsilon}\in \mathcal{G}\text{ such that }\right. \\ \left. C_\varepsilon \subseteq S\subseteq U_{\varepsilon}\text{ and }\meas (U_{\varepsilon}\setminus C_{\varepsilon}),\meas '(U_{\varepsilon}\setminus C_{\varepsilon})<\varepsilon \right\} .
\end{multlined}
\end{equation}

For the rest of this proof, let us write $S\Subset T$ iff there is some uniform cover $\mathcal{B}\in \widetilde{\mathcal{B}}_\Phi$ for which $\Star _{\mathcal{B}}(S)\subseteq T$.  The first thing to note is that, if $S\Subset T$, then $\Cls (S)\subseteq \Int (T)$.  Suppose that $S\Subset T$ and let $x\in X$ be an accumulation point of $S$.  By definition, there is some uniform cover $\mathcal{B}\in \widetilde{\mathcal{B}}$ such that $\Star _{\mathcal{B}}(S)\subseteq T$.  Because $x$ is an accumulation point of $S$ and every element of $\mathcal{B}$ is open, every element of $\mathcal{B}$ that contains $x$ (of which there must be at least one, say $B\in \mathcal{B}$), must intersect $S$, and so we have that $x\in B\subseteq \Star _{\mathcal{B}}(S)\subseteq T$, which implies that $x\in \Int (T)$.

To prove that they agree on $\mathcal{G}$, we will first show that they agree on $Z(\mathcal{G})\cap \mathcal{G}$.  To show that this is in fact sufficient, we prove that
\begin{equation}\label{5.1.75}
\meas (U)=\sup \{ \meas (V):V\in \mathcal{G}\cap Z(\mathcal{G}),\ V\Subset U\} 
\end{equation}
for $U\in \mathcal{G}$ open (the exact same proof will work for $\meas '$).  As $\meas$ is inner-regular on open sets (and so in particular on elements of $\mathcal{G}$), it suffices to show that, for every $K\subseteq U$ quasicompact, there is some $V_K\in \mathcal{G}\cap Z(\mathcal{G})$ with $K\subseteq V_K\Subset U$.  We first show that we can find such a $V_K\in \mathcal{G}$.

As $X$ is $T_0$, it is uniformly-completely-$T_3$, and so we may uniformly-separated quasicompact sets from closed sets, and so there is some uniform cover $\mathcal{B}\in \widetilde{\mathcal{B}}_\Phi$ such that $\Star _{\mathcal{B}}(K)$ is disjoint from $\Star _{\mathcal{B}}(U^{\comp})$.  Because $X$ is a locally compact $T_2$ isogeneous space, by taking a star-refinement if necessary, we can without loss of generality assume that each element of $\mathcal{B}$ has compact closure (i.e.~is an element of $\mathcal{G}$).\footnote{Take any open set $U$ with compact closure (which exists by local compactness).  Then, $\mathcal{B}_U$ will be a cover whose elements are in $\mathcal{G}$.  Take a common star-refinement of this and $\mathcal{B}$.  The closures of elements of this new cover will be contained in the elements of $\mathcal{B}_U$, which themselves will be compact as closed sets of compact sets are compact.}  Take a star-refinement $\mathcal{C}\in \widetilde{\mathcal{B}}_{\Phi}$.  Once again, every element of $\mathcal{C}$ has compact closure.  By quasicompactness of $K$, there are finitely many $C_1,\ldots ,C_m\in \mathcal{C}$ that over $K$.  Define $C\coloneqq C_1\cup \cdots \cup C_m\in \mathcal{G}$.  Furthermore,
\begin{equation}
\Star _{\mathcal{C}}(C)\subseteq \Star _{\mathcal{B}}(K)\subseteq \Star _{\mathcal{B}}(U^{\comp})^{\comp}\subseteq U.
\end{equation}
Thus, we have shown that for $U\in \mathcal{G}$ and $K\subseteq U$ quasicompact, there is some $V_K\in \mathcal{G}$ with $K\subseteq V_K\Subset U$.  Thus, to finish the proof that it suffices to show that they agree on $Z(\mathcal{G})\cap \mathcal{G}$, it suffices to show that, for every $U,V\in \mathcal{G}$ with $U\Subset V$, there is some $W\in Z(\mathcal{G})\cap \mathcal{G}$ with $U\Subset W\Subset V$.

So, let $U,V\in \mathcal{G}$ with $U\Subset V$.  We showed before that this implies that $\Cls (U)\subseteq V$.  Then, because we may uniformly separate quasicompact sets from closed sets in uniformly-completely-$T_3$ spaces, there is some uniform cover $\mathcal{B}\in \widetilde{\mathcal{B}}_{\Phi}$ such that $\Star _{\mathcal{B}}(\Cls (U))$ is disjoint from $\Star _{\mathcal{B}}(V^{\comp})$.  Define
\begin{equation}
W\coloneqq \Star _{\mathcal{B}}(\Cls (U)).
\end{equation}
$W$ is open as every element of $\mathcal{B}$ is open.  It also has compact closure as its closure is contained in $\Cls (U)$, which is compact (closed subsets of compact sets are compact in $T_2$ spaces).  By definition, we have that $U\Subset W$.  We check that also $W\Subset V$.  To show this, we show that $\Star _{\mathcal{B}}(W)\subseteq V$.  So, let $B\in \mathcal{B}$ intersect $W$.  We proceed by contradiction:  suppose that $B$ intersects $V^{\comp}$.  Then, it is contained in $\Star _{\mathcal{B}}(V^{\comp})$, which is disjoint from $W$:  a contradiction.  Thus, $U\Subset W\Subset V$ and $W\in \mathcal{G}$.  It remains to show that $W\in Z(\mathcal{G})$.  This, however, follows from the fact that that $W$ is measurable with respect to both $\meas$ and $\meas '$ (because it is open---see \cref{prp5.1.25}), the fact that it has finite measure for both $\meas$ and $\meas '$ (because its closure is quasicompact and the measure is regular), and \cref{prp5.1.39} (we can force $U_{\varepsilon}$ there to have compact closure by intersecting it with $V$).  This finally establishes \eqref{5.1.75}, and so finishes our proof that it suffices to show that $\meas$ and $\meas '$ agree on $\mathcal{G}\cap Z(\mathcal{G})$.

For the rest of the proof, take note that everything we know about $\meas '$ is likewise true about $\meas$.  Therefore, everything we prove to be true about $\meas '$ will also be true about $\meas$.  Thus, hereafter, if we prove facts about either $\meas$ or $\meas '$, we shall prove them about $\meas '$---they are then automatically true about $\meas$.

For $U\in \mathcal{G}\cap Z(\mathcal{G})$, every cover $\mathcal{B}_V$ has a finite subcover of $U$, and so just as we did for $K$ compact, we may define $(U:V)$ to be the the cardinality of the smallest such subcover.  We shall use this notation in a moment.

\begin{exr}
Let $U\in \mathcal{G}\cap Z(\mathcal{G})$.  Show that, for every $\varepsilon >0$, there is some open $U_\varepsilon$, such that
\begin{equation}
\meas '\left( \Star _{\mathcal{B}_{U_{\varepsilon}}}(U)-U\right) <\varepsilon 
\end{equation}
\end{exr}
Now, fix $U_0\in Z(\mathcal{G})\cap \mathcal{G}$ and let $U$ open be arbitrary.
\begin{exr}
Show that there is some $U'\Subset U$ such that, for all $V$ sufficiently small (with respect to $\Subset$),
\begin{equation}
\frac{\meas '(U_0)}{\meas '(U)}\leq \frac{(U_0':V)}{(U':V)}
\end{equation}
whenever $\Star _{\mathcal{B}_{U'}}(U_0)\subseteq U_0'$.
\begin{rmk}
Hint:  See (10.2) in \cite{Howes}.
\end{rmk}
\end{exr}
\begin{exr}
Show that there is some $U''\Subset U'$ such that, for all $V$ sufficiently small (with respect to $\Subset$),
\begin{equation}
\frac{\meas '(U_0)}{\meas '(U'')}\geq \frac{(U_0':V)}{(U':V)}
\end{equation}
whenever $\Star _{\mathcal{B}_{U'}}(U_0')\subseteq U_0$.
\begin{rmk}
Hint:  See (10.3) in \cite{Howes}.
\end{rmk}
\end{exr}
\begin{exr}
Combine the last three exercises to show that $\meas '$ and $\meas$ agree on $\mathcal{G}\cap Z(\mathcal{G})$ up to a scalar multiple.
\begin{rmk}
Hint:  You should be able to combine these results to get an expression for $\meas '(U_0)$ that, besides factors of $\meas '(U'')$ and $\meas '(U')$, will be completely independent of $\meas '$.  As explained above, everything true of $\meas '$ must also be true of $\meas$, and so, we will have the same expression for $\meas$, with exception of the fact that the factors $\meas '(U')$ and $\meas '(U'')$ will be different.
\end{rmk}
\end{exr}
\end{proof}
\end{savenotes}
\end{thm}
Dayyyuuuummmm.  That was a hard theorem.  Probably the hardest in these notes.  But holy jesus was it worth it.  Check out this epic definition.
\begin{dfn}[Lebesgue measure]\label{LebesgueMeasure}
\emph{Lebesgue measure}\index{Lebesgue measure} $\meas$ on $\R ^d$ is the haar measure with respect to the symmetry group of all isometries\footnote{An isometry is an isomorphism in the category of metric spaces.  Concretely, this means that $\metric[f(x_1)][f(x_2)]=\metric[x_1][x_2]$.  In the case of $\R ^d$, all rotations, reflections, and translations are isometries.  (In fact, one can show, though we will not, that these generate all isometries.)} such that $\meas ([0,1]\times \cdots \times [0,1])=1$.
\begin{rmk}
You have to check that the group of isometries actually generate a uniform base for $\R ^d$ via \eqref{5.1.28}, but this is actually really trivial, because the isometric image of an $\varepsilon$-ball is---gasp---another $\varepsilon$-ball!
\end{rmk}
\begin{rmk}
This is actually much better than defining lebesgue measure to be `classical' haar measure (i.e.~haar measure with respect to the topological group structure).  With this definition we \emph{automatically} get that lebesgue measure is invariant under rotations for free, whereas with the ``classical'' definition, this requires some work.
\end{rmk}
\begin{rmk}
Besides using haar measure to define lebesgue measure, it is also common to use something called \emph{Carath\'{e}odory's Extension Theorem}, which, while not that bad, has the problem that it lacks a uniqueness result.\footnote{At least in general.  I think in certain nice cases it can be made to work.}  Moreover, carath\'{e}odory doesn't even give us a regular measure---it actually just gives us an outer-measure.  We would then have to go through by hand and check that the measure defined in this way is finite on quasicompact sets, inner-regular on open sets, outer-regular, has a uniformly-measurable base, is uniformly-additive, is invariant under translation, is invariant under rotation, is invariant under reflection, and that translations, rotations, and reflections actually give us all the isometries.  Ew.  The theory is just so much prettier when all of this hard work is done for us by \emph{one} result, instead of by fifty-bajillion separate ones.
\end{rmk}
\end{dfn}

We can even define counting measure using the \nameref{HaarHowesTheorem}.\footnote{Though this is a bit like nuking the planet to `swat' a fly.}
\begin{dfn}[Counting measure]
Let $X$ be a set equipped with the discrete uniformity and take $\Phi \coloneqq \Aut _{\Set}(X)$ to be the group of all bijections from $X$ to itself.  Then, 
\begin{equation}
\{ \mathcal{B}_{\{ x_0\}}:x_0\in X\}\coloneqq \left\{ \{ h(\{ x_0\} ):h\in H\} \right\} =\left\{ \left\{ \{ x\} :x\in X\right\} \right\} \footnote{This is the uniform base which has a single cover, namely, the cover by singletons.  I say this because what is going on here is actually very easy, even though the notation may be a bit hard to parse.}
\end{equation}
is a uniform base, and so, because $\mathcal{B}\coloneqq \{ \{ x_0\} :x_0\in X\}$ is a base for this topology, by \cref{exr5.1.48}, $\coord{X,H}$ is an isogeneous space.  Therefore, by the \nameref{HaarHowesTheorem}, there is a unique isogeneous measure $\meas$, the \emph{counting measure}, on $X$ such that $\meas (\{ x_0\} )=1$.
\end{dfn}

\section{The integral}

As you know, the integral is `supposed' to be the ``area under the curve''.  The functions we will be integrating will take values in the reals, and so the integral will spit out a real number.  This is probably nothing new to you.  What is almost certainly new to you, however, is that now the \emph{domains} of the functions we will be integrating will be regular measure spaces.\footnote{You can do things much more generally than this, but for us,there is no need.  As a rule, it is usually best to do things in the nicest theory that encompasses every example you're interested in, and for us, we will not be interested in measures that are not regular (except perhaps when it comes to counter-examples).}  That is, we will be integrating functions $f:X\rightarrow \R$, for $X$ a regular measure space.  We will then simply define the integral to be the measure of the set\footnote{At least when $f$ is nonnegative---we'll have to work just a teensy bit harder to take the `signed area' if the function is negative somewhere.}
\begin{equation}
\left\{ \coord{x,y}\in X\times \R :0<y<f(x)\right\} .
\end{equation}
To do this, of course, we must first define a measure on $X\times \R$.

\subsection{The product measure}

\begin{dfn}[Product measure]
Let $\coord{X_1,\meas _1}$ and $\coord{X_2,\meas _2}$ be regular measure spaces.  The \emph{product measure}, $\meas _1\times \meas _2:2^{X_1\times X_2}\rightarrow [0,\infty ]$, on $X_1\times X_2$ is defined by
\begin{equation}
\begin{split}
[\meas _1\times \meas _2](K_1\times K_2) & \coloneqq \meas _1(K_1)\meas _2(K_2)\text{ for }K_1\subseteq X_1,\ K_2\subseteq X_2\text{ quasicompact} \\
[\meas _1\times \meas _2](U) & \coloneqq \sup \left\{ \sum _{k=0}^m[\meas _1\times \meas _2](K_{1,k}\times K_{2,k}):\right. \\
& \qquad \left. K_{1,k}\subseteq X_1,\ K_{2,k}\subseteq X_2\text{ quasicompact},\right. \\
& \qquad \left. \left\{ K_{1,k}\times K_{2,k}:0\leq k\leq m\right\} \text{ is disjoint},\right. \\
& \qquad \left. \bigcup _{k=0}^mK_{1,k}\times K_{2,k}\subseteq U,\ m\in \N \right\} \text{ for }U\subseteq X_1\times X_2\text{ open} \\
[\meas _1\times \meas _2](S) & \coloneqq \inf \left\{ [\meas _1\times \meas _2](U):S\subseteq U,\ U\text{ open}\right\} .
\end{split}
\end{equation}
\begin{prp}
\begin{savenotes}
This is a well-defined regular measure on $X_1\times X_2$.
\begin{proof}
\Step{Introduce notation}
To help make notation more convenient, let us write
\begin{equation}
\begin{split}
\meas _{\mathrm{K}}(K_1\times K_2) & \coloneqq \meas _1(K_1)\meas _2(K_2)\text{ for }K_1\subseteq X_1,\ K_2\subseteq X_2\text{ quasicompact} \\
\meas _{\mathrm{U}}(U) & \coloneqq \sup \left\{ \sum _{k=0}^m\meas _K(K_{1,k}\times K_{2,k}):\right. \\
& \qquad \left. K_{1,k}\subseteq X_1,\ K_{2,k}\subseteq X_2\text{ quasicompact},\right. \\
& \qquad \left. \left\{ K_{1,k}\times K_{2,k}:0\leq k\leq m\right\} \text{ is disjoint},\right. \\
& \qquad \left. \bigcup _{k=0}^mK_{1,k}\times K_{2,k}\subseteq U,\ m\in \N \right\} \text{ for }U\subseteq X_1\times X_2\text{ open} \\
[\meas _1\times \meas _2](S) & \coloneqq \inf \{ \meas _{\mathrm{U}}(U):S\subseteq U,\ U\text{ open}\} .
\end{split}
\end{equation}
We introduce this separate notation because we do not know yet that they all agree.  Once we show that $\meas _1\times \meas _2$ is well-defined, we shall denote everything by $\meas _1\times \meas _2$.

\Step{Show that $\meas _1\times \meas _2$ is nondecreasing}
\begin{exr}
Check that $\meas _{\mathrm{K}}$ is nondecreasing on sets of the form $K_1\times K_2$ for $K_i\subseteq X_i$ quasicompact.
\end{exr}
\begin{exr}
Check that $\meas _{\mathrm{U}}$ is nondecreasing on open sets.
\end{exr}
\begin{exr}
Check that $\meas _1\times \meas _2$ itself is nondecreasing.
\end{exr}

\Step{Show that $\meas _{\mathrm{U}}(U_1\times U_2)=\meas _{\mathrm{U}}(U_1)\meas _{\mathrm{U}}(U_2)$ for $U_i\subseteq X_i$ open}
Let $U_i\subseteq X_i$ be open.  Let us first suppose that both $\meas _1(U_1)=\infty =\meas _2(U_2)$.  Then, for every $M>0$, there are quasicompact $K_i\subseteq U_i$ with $\meas (K_i)>M$.  Then,
\begin{equation}
\meas _{\mathrm{U}}(U_1\times U_2)\geq \meas _1(K_1)\meas _2(K_2)>M^2,
\end{equation}
and so $\meas _{\mathrm{U}}(U_1\times U_2)=\infty$.

Let us now do the case where one has measure $0$ and one has infinite measure.  Without loss of generality, take $\meas _1(U_1)=0$ and $\meas _2(U_2)=\infty$.  Then, whenever we have
\begin{equation}
\bigcup _{k=0}^mK_{1,k}\times K_{2,k}\subseteq U_1\times U_2,
\end{equation}
we must have that $K_{1,k}\subseteq U_1$ for all $1\leq k\leq m$,\footnote{Okay, you caught me.  This is not necessarily true if $K_{2,k}=\emptyset$.  Congratulations.  Would you like a cookie?} which forces $\meas _1(K_{1,k})=0$, and so in turn it forces
\begin{equation}
\sum _{k=0}^m\meas _{\mathrm{K}}(K_{1,k}\times K_{2,k})=0,
\end{equation}
and hence $\meas _{\mathrm{U}}(U_1\times U_2)=0=\meas _1(U_1)\meas _2(U_2)$.

\begin{exr}
Do the case where one has finite positive measure and one has infinite measure.
\end{exr}
\begin{exr}
Do the case where one has zero measure and the other has finite positive measure.
\end{exr}

Finally, consider the case where both $0<\meas (U_1),\meas (U_2)<\infty$.  Let $\min \{ \meas _1(U_1),\meas _2(U_2)\} >\varepsilon >0$ and choose $K_i\subseteq U_i$ quasicompact so that
\begin{equation}
\meas _i(K_i)\leq \meas _i(U_i)<\meas (K_i)+\varepsilon .
\end{equation}
Then,
\begin{equation}
\meas _1(U_1)\meas _2\meas _1(K_1)\meas _2(K_2)>(\meas _1(U_1)-\varepsilon )(\meas _2(U_2)-\varepsilon )
\end{equation}

We now check that $\meas _1\times \meas _2$ is well-defined.  Let $S\subseteq X_1\times X_2$ be open.  We must show that $[\meas _1\times \meas _2](S)=\meas _{\mathrm{U}}(S)$.  Of course, $\meas _{\mathrm{U}}(S)\in \{ \meas _{\mathrm{U}}(U):S\subseteq U,\ U\text{ open}\}$, which gives us that $[\meas _1\times \meas _2](S)\leq \meas _{\mathrm{U}}(S)$.  On the other hand, because $\meas _{\mathrm{U}}$ is nondecreasing on open sets, we have that $\meas _{\mathrm{U}}(S)$ is a lower-bound for $\{ \meas _{\mathrm{U}}(U):S\subseteq U,\ U\text{ open}\}$, which gives us the other inequality.

Now suppose that $S=K_1\times K_2$ for $K_i\subseteq X_i$ quasicompact.  We must show that $[\meas _1\times \meas _2](S)=\meas _{\mathrm{K}}(S)$.  Let $\varepsilon >0$.  Because $\meas _i$ is regular, there is some open $U_i\subseteq X_i$ with $K_i\subseteq U_i$ and
\begin{equation}
\meas _i(K_i)\leq \meas _i(U_i)<\meas _i(K_i)+\varepsilon .
\end{equation}
\end{proof}
\end{savenotes}
\end{prp}
\begin{exr}\label{exr5.2.4}
Show that this is a well-defined regular measure on $X_1\times X_2$.
\begin{rmk}
For example, if $S$ is open, you must check that both the second and third formulas agree.
\end{rmk}
\end{exr}
\begin{rmk}
So the formulas look ridiculous, but I promise, it's not that complicated.  The first thing says that the measure of a `rectangle' is the product of the measure of the sides---area is base times height.  The second thing says that we can approximate open sets from the inside with quasicompact sets of a special form (namely, finite disjoint unions of quasicompact rectangles)---essentially just inner regularity on opens.  And the third says that we can approximate arbitrary sets from the outside with opens---outer regularity.
\end{rmk}
\begin{rmk}
In particular, \emph{product measures are uniquely determined by the measures of quasicompact subsets of $X_1$ and $X_2$}.
\end{rmk}
\end{dfn}
\begin{prp}
Let $\coord{X_1,\meas _1}$ and $\coord{X_2,\meas _2}$ be regular measure spaces, and let $S_1\subseteq X_1$ and $S_2\subseteq X_2$.  Then,
\begin{equation}
[\meas _1\times \meas _2](S_1\times S_2)=\meas _1(S_1)\meas _2(S_2).
\end{equation}
\begin{rmk}
The point is that $S_1$ and $S_2$ do not have to be quasicompact (or open or whatever).
\end{rmk}
\begin{proof}
We leave this as an exercise.
\begin{exr}
Prove the result yourself.
\end{exr}
\end{proof}
\end{prp}
When we finally get to the integral (\cref{Integral}), we will take our domain to be a topological measure space.  The reason for doing so is that this will allow us to prove that the area under the curve satisfies certain desirable properties (e.g.~$\int _X\dif \meas (x)\, [f(x)+g(x)]=\int _X\dif \meas (x)\, f(x)+\int _X\dif \meas (x)\, g(x)$ for $f,g$ borel (see \cref{IntegrableFunction})).  I haven't had the time to look for counter-examples to show that it is strictly necessary to restrict our domain to be a topological measure space, but it certainly is a convenient assumption (and not unnatural) that doesn't rule out any examples of interest.\footnote{If you know of any examples of outer-measure spaces that are not topological measure spaces upon which one wants to do integration, please let me know.}  A crucial fact about topological measure spaces that we will make use of when the time comes is that the product of two topological measure spaces is a topological measures space.
\begin{prp}
Let $\coord{X_1,\meas _1}$ and $\coord{X_2,\meas _2}$ be topological measure spaces.  Then, $\coord{X_1\times X_2,\meas _1\times \meas _2}$ is a topological measure space.
\begin{proof}
We need to show that (i) $X_1\times X_2$ is $\sigma$-quasicompact, (ii) $\meas _1\times \meas _2$ is regular, and (iii) $\meas _1\times \meas _2$ is borel.

Write $X_1=\bigcup _{m\in \N}K_{1,m}$ and $X_2=\bigcup _{m\in \N}K_{2,m}$ for $K_{i,m}\subseteq X_i$ quasicompact.  Then,
\begin{equation}
X_1\times X_2=\bigcup _{m,n\in \N}K_{1,m}\times K_{2,n},
\end{equation}
and so $X_1\times X_2$ is $\sigma$-quasicompact.

That $\meas _1\times \meas _2$ is regular is part of the definition of the product measure---see \cref{exr5.2.4}.

We must check that $\meas _1\times \meas _2$ is borel.  So, let $U\subseteq X_1\times X_2$ be open.  Define $U_{m,n}\coloneqq U\cap (K_{1,m}\times K_{2,n})$.  As $U=\bigcup _{m,n}U_{m,n}$, by \nameref{CaratheodorysTheorem}, it suffices to show that each $U_{m,n}$ is measurable.
\end{proof}
\end{prp}

The product of two isogeneous spaces is canonically an isogeneous spaces, and if they are both $T_0$ locally quasicompact, so too will the product be.  Hence, the \nameref{HaarHowesTheorem} tells us that the product will obtain a measure.  The question, then, is whether this measure agrees with the product measure as defined above.  Fortunately, the answer is in the affirmative.
\begin{dfn}[Product isogeneous structure]
Let $\coord{X_1,H_1}$ and $\coord{X_2,H_2}$ be two isogeneous structures.  Then, the \emph{product isogeneous structure}\index{Product isogeneous structure} on $X_1\times X_2$ is given by
\begin{equation}
H_1\times H_2\coloneqq \{ h_1\times h_2:h_1\in H_1,\ h_2\in H_2\},
\end{equation}
where $h_1\times h_2:X_1\times X_2\rightarrow X_1\times X_2$ is defined by
\begin{equation}
[h_1\times h_2](\coord{x_1,x_2})\coloneqq \coord{h_1(x_1),h_2(x_2)}.
\end{equation}
\begin{exr}
Show that this in fact gives an isogeneous structure.
\begin{rmk}
In other words, you need to show that
\begin{equation}
\left\{ \mathcal{B}_{U_1\times U_2}\right\} 
\end{equation}
is a uniform base, where
\begin{equation}
\mathcal{B}_{U_1\times U_2}\coloneqq \{ h_1(U_1)\times h_2(U_2):h_1\in H_1,\ h_2\in H_2\} .
\end{equation}
\end{rmk}
\end{exr}
\end{dfn}
\begin{prp}
Let $\coord{X_1,H_1,\meas _1}$ and $\coord{X_2,H_2,\meas _2}$ be isogeneous measure with $\meas _1(K_1)=1=\meas _2(K_2)$.  Then, $\meas _1\times \meas _2$ is the haar measure on the product isogeneous space $X_1\times X_2$ that gives measure $1$ to $K_1\times K_2$.
\begin{rmk}
In particular, the product of lebesgue measure on $\R ^d$ with the product of lebesgue measure on $\R ^e$ is the lebesgue measure on $\R ^{d+e}$.
\end{rmk}
\begin{proof}
We leaves this as an exercise.
\begin{exr}
Prove the result yourself.
\end{exr}
\end{proof}
\end{prp}

\subsection{Lebesgue measure}

Before continuing on with integration, we prove some properties about lebesgue measure.  Knowing that lebesgue measure on $\R ^d$ is just the product of the lebesgue measure on $\R$ is actually incredibly useful, and so now that we know this, we take advantage of this fact.

\begin{prp}
The lebesgue measure of a point is $0$.
\begin{proof}
The argument is the exact same in $\R ^d$ as it is in $\R$, so we write down the argument in $\R$ and save us from some $\cdots$.

The first thing to notice is that, by translation invariance, the measure of every point is the same.  Let us denote this measure by $M$

Points are closed, and hence measurable, and so $\meas$ is additive on points.  Therefore, we have that
\begin{equation}
1=\meas ([0,1])\geq \meas \left( \bigcup _{x\in [0,1]\cap \Q}\{ x\} \right) =\sum _{x\in [0,1]\cap \Q}\meas (\{ x\} )=\sum _{m\in \N}M
\end{equation}
This equation forces $M=0$.
\end{proof}
\end{prp}
\begin{crl}
$\meas ([0,1)\times \cdots \times [0,1))=1$.
\begin{rmk}
Sets of the form
\begin{equation}\label{5.2.16}
[a_1,b_1)\times \cdots \times [a_d,b_d)
\end{equation}
are important in measure theory because, for example,
\begin{equation}
[0,2)=[0,1)\cup [1,2)
\end{equation}
is a \emph{disjoint} union.  If we tried replacing everything here with all open intervals we we would have that $(0,2)=(0,1)\cup (1,2)$, which is just plain false, and if we tried replacing everything here with all closed intervals the union would not be disjoint ($[0,1]$ and $[1,2]$ intersect at $1$).  The disjointness is important in measure theory of course because of additivity (on measurable sets).  Sets of the form \eqref{5.2.16} are \emph{half-open rectangles}\index{Half-open rectangles}.
\end{rmk}
\begin{proof}
Now that we know that
\begin{equation}
\meas ([0,1)\times \cdots \times [0,1))=\meas ([0,1))\cdots \meas ([0,1)),
\end{equation}
it suffices to prove this result in $\R$.

That it is true in $\R$ follows from the fact that
\begin{equation}
1=\meas ([0,1])=\meas ([0,1))+\meas (\{ 1\} )=\meas ([0,1)).
\end{equation}
\end{proof}
\end{crl}
\begin{prp}
$\meas ([a_1,b_1]\times \cdots [a_d,b_d])=(b_1-a_1)\cdots (b_d-a_d)$.
\begin{proof}
Now that we know that it is a product measure, it suffices to show the one-dimensional case, and so we prove that $\meas ([a,b])=b-a$.

By translation invariance, it suffices to show that $\meas ([0,b])=b$.

Because the measure of points is $0$, it suffices to show that $\meas ([0,b))=b$.

\begin{exr}
Show that $\meas ([0,m))=m$ for $m\in \N$.
\end{exr}
\begin{exr}
Show that $\meas ([0,\frac{p}{q}))=\frac{p}{q}$ for $p,q\in \Z ^+$.
\end{exr}
Then, we have that
\begin{equation}
q\meas ([0,q)) \leq \meas ([0,b))\leq \meas ([0,r))=r
\end{equation}
for all $p,q\in \Q ^+$ with $q\leq b\leq r$.  It follows that $\meas ([0,b))=b$.
\end{proof}
\end{prp}
\begin{exr}
Show that any open set in $\R ^d$ can be written at the countable disjoint union of half-open rectangles.
\end{exr}
\begin{exr}\label{exr5.2.28}
Let $X$ be a regular measure space and for $S\subseteq X\times \R ^d$ and $a\in \R$, define
\begin{equation}
aS\coloneqq \left\{ \coord{x,ay}:\coord{x,y}\in S\right\} ,
\end{equation}
where $x\in X$ and $y\in \R ^d$.  Show that
\begin{equation}
\meas (aS)=\abs{a}^d\meas (S).
\end{equation}
\begin{rmk}
Prove it for $\R ^d$ alone first (i.e.~if $X$ is just a point, so that $X\times \R ^d=\R ^d$).  For this case, prove it for $S$ open using the previous exercise and the fact that we already know it is true for half-open rectangles.  Then prove it in general for $\R ^d$ by outer-regularity.  Then generalize this to the case $X\times \R ^d$ for $X$ not-necessarily a point using the definition of product measures.
\end{rmk}
\end{exr}
\begin{exr}
Let $T:\R ^d\rightarrow \R ^d$ be a linear transformation and let $S\subseteq \R ^d$.  Show that
\begin{equation}
\meas (T(S))=\abs{\det (T)}\meas (S).
\end{equation}
\begin{rmk}
Hint:  Recall your matrix decompositions from linear algebra and use the fact that we already know (by Haar-Howe) that lebesgue measure is invariant under isometries together with the result of the previous exercise.
\end{rmk}
\end{exr}

\begin{exr}\label{exr5.2.33}
Show that countable sets have $0$ measure.
\begin{rmk}
In particular, $\Q$ has $0$ measure!
\end{rmk}
\end{exr}
\begin{exm}[An uncountable set of zero measure---the Cantor Set]\label{CantorSet}
\begin{savenotes}
Let $L\in \R ^+$ be less than $1$.

Define $C_0\coloneqq [0,1]$.  Define $C_1$ to be $C_0$ with the `middle' open interval of length $L$ removed, that is
\begin{equation}
C_1\coloneqq C_0\setminus (\tfrac{1}{2}-\tfrac{1}{2}L,\tfrac{1}{2}+\tfrac{1}{2}L)=[0,\tfrac{1}{2}-\tfrac{1}{2}L]\cup [\tfrac{1}{2}+\tfrac{1}{2}L,1].
\end{equation}
Thus, $C_1$ is the disjoint union of two-intervals of length $\frac{1}{2}(1-L)$.  Now from each of these intervals remove the middle open interval of length $L\cdot \left( \frac{1}{2}(1-L)\right) =\frac{1}{2}(L-L^2)$.\footnote{You should think of $L$ as the `fraction' of the interval that you remove at each step.}  This is $C_2$.  That is,
\begin{equation}
\begin{split}
C_2 & \coloneqq C_1\setminus \left( \left( \tfrac{1}{2}\cdot (\tfrac{1}{2}-\tfrac{1}{2}L)-\frac{1}{4}(L-L^2),\tfrac{1}{2}\cdot (\tfrac{1}{2}-\tfrac{1}{2}L)+\frac{1}{4}(L-L^2)\right) \right. \\
& \qquad \left. \cup \left( \tfrac{1}{2}\cdot (\tfrac{1}{2}+\tfrac{1}{2}L+1)-\frac{1}{4}(L-L^2),\tfrac{1}{2}\cdot (\tfrac{1}{2}+\tfrac{1}{2}L+1)+\frac{1}{4}(L-L^2)\right) \right) \\
& =\left[ 0,\tfrac{1}{2}\cdot (\tfrac{1}{2}-\tfrac{1}{2}L)-\frac{1}{4}(L-L^2)\right] \\
& \qquad \cup \left[ \tfrac{1}{2}\cdot (\tfrac{1}{2}-\tfrac{1}{2}L)+\frac{1}{4}(L-L^2),\tfrac{1}{2}-\tfrac{1}{2}L]\right] \\
& \qquad \cup \left[ \tfrac{1}{2}+\tfrac{1}{2}L,\tfrac{1}{2}\cdot (\tfrac{1}{2}+\tfrac{1}{2}L+1)-\frac{1}{4}(L-L^2)\right] \\
& \qquad \cup \left[ \tfrac{1}{2}\cdot (\tfrac{1}{2}+\tfrac{1}{2}L+1)+\frac{1}{4}(L-L^2),1\right] .
\end{split}
\end{equation}
Continue this process inductively, at each step removing the `middle' open intervals of length $\frac{1}{2^{k-1}}L$ to define $C_k$.  Finally, define
\begin{equation}
C\coloneqq \bigcap _{k=0}^\infty C_k.\footnote{At this point, as I haven't had a chance to add in my own picture of this, I highly suggest you google it to find a picture.}
\end{equation}
$C$ is the \emph{Cantor Set}\index{Cantor Set}.\footnote{Actually, the `classical' Cantor Set is the case $L=\frac{1}{3}$.}

$C_0$ is the disjoint union of $1$ closed interval of length $L_0\coloneqq 1$.

$C_1$ is the disjoint union of $2$ closed intervals of length 
\begin{equation}
L_1\coloneqq \frac{1}{2}(L_0-L\cdot L_0)=\frac{1}{2}(1-L)
\end{equation}

$C_2$ is the disjoint union of $4$ closed intervals of length
\begin{equation}
L_2\coloneqq \frac{1}{2}(L_1-L\cdot L_1)\footnote{The way to think about this is as follows:  You take the intervals of length $L_1$, you remove the fraction $L$ of that length from the middle to obtain a set made of two pieces of total length $L_1-L\cdot L_1$, so that each half itself is of length $\frac{1}{2}(L_1-L\cdot L_1)$.} =\frac{1}{4}(1-L)^2
\end{equation}

$C_3$ is the disjoint union of $8$ closed intervals of length
\begin{equation}
L_3\coloneqq \frac{1}{2}(L_2-L\cdot L_2)=\frac{1}{8}(1-L)^3.
\end{equation}
By now, hopefully you see the pattern, which you could prove by induction if you really wanted:  $C_k$ is the disjoint union of $2^k$ closed intervals of length $\left( \frac{1-L}{2}\right) ^k$.  Thus, in general, we have that
\begin{equation}
\meas (C_k)=(1-L)^k.
\end{equation}
By \cref{exr5.1.29} then, it follows that
\begin{equation}
\meas (C)=\lim _k(1-L)^k=0.
\end{equation}

It remains to show that $C$ is uncountable.  We do this in two steps.
\begin{exr}
Show that every point of $C$ is an accumulation point.
\end{exr}
\begin{exr}
Show that a nonempty closed subset of $\R$ that has the property that every point is an accumulation point is uncountable.
\end{exr}
\end{savenotes}
\end{exm}
\begin{exm}[A set with empty interior of positive measure---the Cantor-Smith-Volterra Set]\label{CantorSmithVolterraSet}
\begin{exr}
Modify construction of the Cantor Set starting with $C_0\coloneqq [0,1]$ again, but upon constructing $C_k$ from $C_{k-1}$, you remove the middle open interval $\frac{1}{4^k}$ from the middle of each closed interval of $C_{k-1}$.  The resulting set is the \emph{Cantor-Smith-Volterra Set}\index{Cantor-Smith-Volterra Set}.  Show that it has measure $\frac{1}{2}$, but has empty-interior.
\end{exr}
\begin{rmk}
The Cantor-Smith-Volterra Set is still uncountable of course, just as the Cantor Set likewise had empty-interior.  It's just that it's not surprising for a set with empty interior to have measure $0$---what is surprising however is a set with empty interior of \emph{positive measure} with empty-interior.  Similarly, it's not surprising for a uncountable set to have positive measure---what is surprising is an uncountable set of \emph{measure zero}0.
\end{rmk}
\end{exm}
\begin{exm}[A set that is not measurable]\begin{savenotes}\footnote{Construction adapted from \cite[pg.~407]{Pugh}.}\label{exm5.2.47}

Define $\q :\R \rightarrow [0,1)$ by $\q (x)\coloneqq x-\lfloor x\rfloor ]$.  In other words, this is just the ``fractional part'' of the real number (intuitively, you drop of the integer in front of its decimal expansion).  Note that $1$ gets sent to $0$, and so, if you like, you can think of the image as a circle, with $1$ `glued-to' $0$, if this helps your intuition.

Now fix $\theta \in \Q ^{\comp}$ and define $R:\R \rightarrow \R$ by $R(x)\coloneqq x+\theta$.\footnote{The ``R'' is for ``rotation'' because in the `circle' picture, this will correspond to a rotation by angle $t$.}  For $x_1,x_2\in \R$, define $x_1\sim x_2$ iff there is some $m\in \Z$ such that $x_1=R^m(x_2)\coloneqq x_2+m\theta$.
\begin{exr}
Show that $\sim$ is an equivalence relation on $\R$.
\end{exr}
Denote by $O_x$ the equivalence class of $x\in \R$ with respect to $\sim$.\footnote{The ``O'' is for ``orbit''---you can imagine the point $x$ ``orbiting'' around the circle as you apply $R$ to it over-and-over.}
\begin{exr}
Show that $\q (O_x)$ is dense in $[0,1)$.
\begin{rmk}
Hint:  This is why we needed $\theta$ to be \emph{irrational}.
\end{rmk}
\end{exr}

Now, let $N\subseteq \R$ be any set which contains exactly one point from each equivalence class with respect to $\sim$.  We claim that $\q (N)\subseteq [0,1)$ is not measurable.  We proceed by contradiction:  suppose that $\q (N)$ is measurable.
\begin{exr}
Use the fact that $\sim$ is an equivalence relation to show that $\q (R^m(N))$ and $\q (R^n(N))$ are disjoint iff $m\neq n$.
\begin{rmk}
Hint:  See \cref{prpA.1.12}---equivalence classes form a partition of the set.
\end{rmk}
\end{exr}
\begin{exr}
Show that
\begin{equation}
[0,1)=\bigcup _{m\in \Z}\q (R^m(N))
\end{equation}
\begin{rmk}
Hint:  Once again, uses the fact that equivalence classes form a partition.
\end{rmk}
\end{exr}
\begin{exr}
Let $S\subseteq \R$.  Show that $\q (S)$ is measurable iff $\q (R(S))$ is measurable.
\end{exr}
\begin{exr}
Let $S\subseteq \R$.  Show that $\meas (\q (S))=\meas (\q (R(S)))$.
\end{exr}
Thus, these exercises give us that
\begin{equation}
[0,1)=\bigcup _{m\in \Z}\q (R^m(N))
\end{equation}
is a disjoint union of measurable sets, all of which have the same measure $M\coloneqq \q (N)$.  If $M=0$, then, by additivity, we have $\meas ([0,1))=0$:  a contradiction.  On the other hand, if $M>0$, by additivity again, we have $\meas ([0,1))=\infty$:  a contradiction.  Therefore, it cannot be the case that $N$ is measurable.
\end{savenotes}
\end{exm}
\begin{exm}[Two disjoint sets $S,T$ with $\meas (S\cup T)\neq \meas (S)+\meas (T)$]\label{exm5.2.56}
Let $N$ be as in the previous example.  We showed there that it was not measurable.  Therefore, there must exist some $S\subseteq \R$ such that
\begin{equation}
\meas \left( (S\cap N)\cup (S\cap N^{\comp})\right) =\meas (S)\neq \meas (S\cap N)+\meas (S\cap N^{\comp}).
\end{equation}
\end{exm}
As a matter of fact, this trick can be adapted to prove that \emph{every} set of positive measure has a nonmeasurable subset.
\begin{prp}\label{prp5.2.58}
Let $S\subseteq \R$.  Then, if $\meas (S)>0$, then there is some subset $N\subseteq S$ that is not measurable.
\begin{proof}\footnote{Proof adapted from \cite[pg.~53]{BigRudin}.}
If $\meas (S)=\infty$, then because lebesgue measure is semifinite (\cref{Semifinite}), there is some subset $T\subseteq S$ with $0<\meas (T)<\infty$.  If $T$ contains a set that is not measurable, then of course so does $S$, so it suffices to prove the case where $S$ has \emph{finite} measure.  Thus, without loss of generality, suppose that $\meas (S)<\infty$.

Consider the collection of cosets\footnote{See \cref{Cosets} for the definition of cosets.} $\{ x+\Q :x\in \R \}$.  Let $N\subseteq \R$ be set that contains precisely one element of each coset.  It follows that\footnote{This follows from the fact that the cosets form a partition of $\R$, which in turn follows from the fact that equivalence classes form partitions---see \cref{crlA.1.13}.}
\begin{equation}
\R =\bigcup _{r\in \Q}(r+N)
\end{equation}
is a disjoint union, and so it in turn follows that
\begin{equation}
S=\bigcup _{r\in \Q}\left[ S\cap (r+N)\right]
\end{equation}
is a disjoint union.

We proceed by contradiction:  suppose that every subset of $S$ is measurable.  Then, the previous equality implies that
\begin{equation}
\meas (S)=\sum _{r\in \Q}\meas (S\cap (r+N))=
\end{equation}
We show that $\meas (S\cap (r+N))=0$, which will gives us that $\meas (S)=0$:  a contradiction.

As $\meas (S\cap (r+N))<\infty$ and $\R$ is inner-regular on measurable sets of finite measure (see \cref{InnerRegularFinite}), it suffices to show that every quasicompact subset of $S\cap (r+N)$ has measure $0$.  So, let $K\subseteq S\cap (r+N)$ be quasicompact.

Define
\begin{equation}
H\coloneqq \bigcup _{s\in \Q \cap [0,1]}(s+K).
\end{equation}
As $K\subseteq r+N$, this is a disjoint union, and so we have
\begin{equation}
\meas (H)=\sum _{s\in \Q \cap [0,1]}\meas (s+K)=\footnote{By translation invariance.}\sum _{s\in \Q \cap [0,1]}\meas (K).
\end{equation}
$H$ is bounded, and so have finite measure.  Thus, the above equality forces $\meas (K)=0$, and we are done.
\end{proof}.
\end{prp}

We mentioned awhile ago in the definition of measurable functions (\cref{MeasurableFunction}) the existence of a uniform-homeomorphism of $\R$ that preserves neither measurability not measure $0$.  It is time we return to this.
\begin{exm}[A uniform-homeomorphism of $\R$ that preserves neither measurability nor measure $0$---the Cantor Function]\index{Cantor Function}\label{CantorFunction}
\begin{savenotes}
We first define a uniformly-continuous function $f:[0,1]\rightarrow [0,1]$.  This function will be nondecreasing, and so $g\coloneqq f+\id _{[0,1]}:[0,1]\rightarrow [0,2]$ will be increasing, and hence injective.  It will turn out that $f(0)=0$ and $f(1)=1$, and so we will then extend $g$ to all of $\R$ `periodically', that is, defining it to be $g(x-1)+2$ for $x\in [1,2]$, etc..\footnote{Of course, we have to shift the graph up to maintain continuity.} 
\begin{rmk}
$f$ is the \emph{Devil's Staircase}\index{Devil's Staircase}, and $g$ is the \emph{Cantor Function}.\footnote{Some people use both these terms to refer to $f$.  I prefer this convention so that each has its own name.}
\end{rmk}

We define $f:[0,1]\rightarrow [0,1]$ as the uniform limit of a sequence of continuous functions.  It will then be continuous because $\Mor _{\Top}([0,1],\R )$ is complete, and hence uniformly-continuous because $[0,1]$ is quasicompact.  We then must check that $f$ is nondecreasing.  Once we do so, we will have that $g$ is increasing, and hence injective.  Then, because $[0,1]$ is quasicompact and $[0,1]$ is $T_2$, we will have that its inverse is continuous (\cref{exr3.6.46} does this for us), and hence uniformly-continuous, and hence a uniform-homeomorphism.  Extending $g$ `periodically' has no effect on this.

So, let us get started.\footnote{It will probably help to find a picture on this internet of what this is supposed to look like.  (Sorry.  Making diagrams takes awhile and I have not had the time.)  Let $C_m$ denote the `$m^{\text{th}}$ step' in the construction of the Cantor Set (see \cref{CantorSet}).  In words, $f_m$ is supposed to be the function that is constant $[0,1]\setminus C_m$ and increases linearly on the intervals that remain in $C_m$.}  Define $f_0:[0,1]\rightarrow [0,1]$ to be the identity function.  We define $f_m:[0,1]\rightarrow \R$ for $m>0$ recursively:
\begin{equation}\label{5.2.22}
f_m(x)\coloneqq \begin{cases}\tfrac{1}{2}f_{m-1}(3x) & \text{if }x\in [0,\tfrac{1}{3}] \\ \tfrac{1}{2} & \text{if }x\in [\tfrac{1}{3},\tfrac{2}{3}] \\ \tfrac{1}{2}+\tfrac{1}{2}f_{m-1}(3x-2) & \text{if }x\in [\tfrac{2}{3},1]\end{cases}.
\end{equation}
\begin{exr}
Show that $f_m$ is well-defined for all $m\in \N$.
\begin{rmk}
You must check that the two expressions agree for $x=\frac{1}{3}$ and $x=\frac{2}{3}$.
\end{rmk}
\end{exr}
\begin{exr}
Show that $f_m(0)=0$ and $f_m(1)=1$ for all $m\in \N$.
\end{exr}
\begin{exr}
Show that $f_m(x)\in [0,1]$, so that indeed $f_m$ defines a function into $[0,1]$.
\end{exr}
\begin{exr}
Show that $f_m$ is continuous.
\begin{rmk}
Hint:  Use induction and apply the \nameref{PastingLemma} (\cref{PastingLemma}).
\end{rmk}
\end{exr}
\begin{exr}
Show that $f_m$ is nondecreasing.
\end{exr}

We now check that $m\mapsto f_m\in \End _{\Top}([0,1])$ is cauchy.\footnote{Recall that $\End _{\Top}([0,1])\coloneqq \Mor _{\Top}([0,1],[0,1])$---see \cref{Endomorphism}.  To simplify notation, we shall denote the norm on $\End _{\Top}([0,1])$ simply as $\norm$.} To do this, we will show by induction that
\begin{equation}\label{5.2.28}
\norm[f_{m+1}-f_m]\leq \left( \frac{1}{2}\right) ^{m+1}.
\end{equation}
It will then follow from the Triangle Inequality that
\begin{equation}
\norm[f_n-f_m]\leq \sum _{k=m+1}^n\left( \frac{1}{2}\right) \leq \sum _{k=m+1}^\infty \left( \frac{1}{2}\right) ^k=\frac{1}{1-\tfrac{1}{2}}-\frac{1-(\tfrac{1}{2})^{m+1}}{1-\tfrac{1}{2}}=\left( \frac{1}{2}\right) ^m,
\end{equation}
from which cauchyness follows immediately.

Using the definition \eqref{5.2.22}, we have that
\begin{equation}
f_1(x)\coloneqq \begin{cases}\tfrac{1}{2}x & \text{if }x\in [0,\tfrac{1}{3}] \\ \tfrac{1}{2} & \text{if }x\in [\tfrac{1}{3},\tfrac{2}{3}] \\ \tfrac{1}{2}(3x-1) & \text{if }x\in [\tfrac{2}{3},1]\end{cases},
\end{equation}
and so
\begin{equation}
\left| f_1(x)-f_0(x)\right| =\begin{cases}\tfrac{1}{2}x & \text{if }x\in [0,\tfrac{1}{3}] \\ \abs{x-\tfrac{1}{2}} & \text{if }x\in [\tfrac{1}{3},\tfrac{2}{3}] \\ \tfrac{1}{2}(1-x) & \text{if }x\in [\tfrac{2}{3},1]\end{cases}.
\end{equation}
It follows that
\begin{equation}
\norm[f_1-f_0]=\tfrac{1}{6}\leq \left( \tfrac{1}{2}\right) ^{0+1}.
\end{equation}
This completes the base case.  As for the inductive case, fix $m\geq 1$ and assume that \eqref{5.2.28} holds for $0\leq n<m$.  We prove that it holds for $m$ as well.  From the definition \eqref{5.2.22} again, we have that
\begin{equation}
\abs{f_{m+1}(x)-f_m(x)}=\tfrac{1}{2}\cdot \begin{cases}\abs{f_m(3x)-f_{m-1}(3x)} & \text{if }x\in [0,\tfrac{1}{3}] \\ 0 & \text{if }x\in [\tfrac{1}{3},\tfrac{2}{3}] \\ \abs{f_m(3x-2)-f_{m-1}(3x-2)} & \text{if }x\in [\tfrac{2}{3},1]\end{cases}.
\end{equation}
Thus, by the induction hypothesis,
\begin{equation}
\norm[f_{m+1}-f_m]\leq \tfrac{1}{2}\norm[f_m-f_{m-1}]=\left( \frac{1}{2}\right) ^{m+1}.
\end{equation}

We may now define $f\coloneqq \lim _mf\in \End _{\Top}([0,1])$.  You showed that each $f_m$ is nondecreasing, and so, as limits preserve inequalities, $f$ itself is nondecreasing.  Now define $g\coloneqq f+\id _{[0,1]}$.  As described at the beginning of the example, $g:[0,1]\rightarrow [0,1]$ is a uniform-homeomorphism.

We now show that $g$ preserves neither measurability nor measure $0$.  Let $C\subset [0,1]$ denote the ($L=\tfrac{1}{3}$) \nameref{CantorSet} and let $C_m$ denote the set defined in the $m^{\text{th}}$ step of the construction of the \nameref{CantorSet}---see \cref{CantorSet} if you don't know what we're referring to.  For convenience of notation, define $D_m\coloneqq [0,1]\setminus C_m$ and $D\coloneqq [0,1]\setminus C$.
\begin{exr}
Show that $\restr{f_k}{D_m}=\restr{f_m}{D_m}$ is constant on each component of $D_m$ for all $k\geq m$.  In particular, the image of $f_k$ on $D_m$ is finite for all $k\geq m$.
\begin{rmk}
Though it's perhaps not clear from the formulas, the $f_m$s were defined so that precisely this is true---even though $f(0)=0$ and $f(1)=1$, $f$ does all of increasing on a set of measure $0$, namely the \nameref{CantorSet}.
\end{rmk}
\end{exr}
From this, it follows that $f$ itself is constant on each component of $D$.

Recall that each $D_m$ is a disjoint union of open intervals.  As $f$ is constant on each one of these intervals, the measure of the image of each one of these intervals under $g$ is just the length of that interval (the image is the interval itself plus whatever constant $f$ happened to be on that interval).  Furthermore, by injectivity, the images of each of these intervals must be disjoint, and hence, $\meas (g(D))$ is the sum of the measures of all these intervals, namely, $\meas (D)=1$.  As $g([0,1])=[0,2]$, we thus have that
\begin{equation}
\meas (g(C))=\meas (g([0,1]))-\meas (g([0,1]\setminus C)=\meas ([0,2])-\meas (g(D))=2-1=1.
\end{equation}
Thus, $\meas (g(C))=1$, despite the fact that $\meas (C)=0$.

Every subset of $\R$ of positive measure contains a nonmeasurable set (\cref{prp5.2.58}), so let $N\subseteq g(C)$ be some such set.  Then, $g^{-1}(N)\subseteq C$ is measurable because it has measure $0$.
\end{savenotes}
\end{exm}

\subsection{The integral}

An important type of function in measure theory are \emph{characteristic} functions.
\begin{dfn}[Characteristic function]
Let $X$ be a set and let $S\subseteq X$.  Then, the \emph{characteristic function}\index{Characteristic function} of $S$, $\chi _S:X\rightarrow \{ 0,1\}$\index[notation]{$\chi _S$}, is defined by
\begin{equation}
\chi _S(x)\coloneqq \begin{cases}1 & \text{if }x\in S \\ 0 & \text{if }x\notin S\end{cases}.
\end{equation}
\begin{rmk}
The reason that characteristic functions are important in measure theory is because the integral of $\chi _S$ will turn-out to simply be $\meas (S)$.  In particular, \emph{if you know the integral, you know the measure}.
\end{rmk}
\begin{rmk}
We have seen a characteristic function before---the \nameref{DirichletFunction} is the characteristic function of $\Q \subseteq \R$.
\end{rmk}
\begin{rmk}
You might say that the lebesgue integral is to characteristic functions as the riemann integral is to rectangles---one can define the lebesgue integral (though we will not) by approximating functions with characteristic functions,\footnote{Well, actually finite linear combinations of characteristic functions, the so-called \emph{simple functions}\index{Simple function}.} just as one defines the riemann integral by approximating functions with rectangles (aka (scalar multiples of) characteristic functions of intervals).\footnote{I actually find this approach a bit messy, but it's just yet more evidence that the lebesgue integral is hardly more difficult than the riemann integral---it's the same fucking thing with arbitrary (measurable) sets instead of intervals.}
\end{rmk}
\end{dfn}
In principle, we can integrate \emph{any} function.\footnote{Literally.  If you already know what it means for a function to be measurable, you don't even need this to be the case in order to integrate it.}  The `problem' with this, of course, is that the integral will not satisfy certain properties we would like it to (e.g.~$\int \dif x\, [f(x)+g(x)]=\int \dif x\, f(x)+\int \dif x\, g(x)$) if we don't restrict the functions we integrate.
\begin{dfn}[Borel and integrable]\label{IntegrableFunction}
Let $X$ be an outer-measure space and let $f:X\rightarrow [-\infty ,\infty ]$ be a function.  Then, $f$ is \emph{borel}\index{Borel (function)} iff
\begin{equation}\label{5.2.60}
\{ \coord{x,y}\in X\times [-\infty ,\infty ]:0\leq y<f(x)\} \text{ and }\{ \coord{x,y}\in X\times [-\infty ,\infty ]:y\leq f(x)<0\}
\end{equation}
are measurable.  If furthermore they have finite measure, then $f$ is \emph{integrable}\index{Integrable (function)}.
\begin{rmk}
The collection of all (equivalence classes of) borel functions on $X$ in $\Mor _{\AlE}(X,[-\infty ,\infty ])$ is denoted $\Bor (X)$\index[notation]{$\Bor (X)$}.  The collection of all nonnegative borel functions on $X$ is denoted $\Bor _0^+(X)$\index[notation]{$\Bor _0^+(X)$}.
\end{rmk}
\begin{rmk}
There is a symbol for the integrable functions on $X$ as well (namely ``$L^1$''), but we shall wait to discuss this further until we encounter the $L^p$ spaces.
\end{rmk}
\begin{rmk}
Of course, the sets in \eqref{5.2.60} are just `the area under the curve' (one for the nonnegative part of $f$ and one for the nonpositive part of $f$), and so the integral will wind up being the measure of the first minus the measure of the second.  Thus, a function is borel iff the two sets needed to define its integral are measurable, and is integrable if the measure of these sets are finite (so that one can subtract them to define the integral).
\end{rmk}
\end{dfn}

One last thing before we get to the integral---we decompose a function into its nonnegative and nonpositive parts.
\begin{dfn}
Let $X$ be a set and let $f:X\rightarrow \R$ be a function.  Then, we write
\begin{equation}
f_+\coloneqq \max \{ f,0\} \text{ and }f_-\coloneqq -\min \{ f,0\} .
\end{equation}\index[notation]{$f_+$}\index[notation]{$f_-$}
\begin{rmk}
That is, $f_+$ is equal to $f$ if $f$ is positive and $0$ otherwise; likewise, $f_-$ is equal to $f$ if $-f$ is negative and $0$ otherwise.
\end{rmk}
\begin{rmk}
Note that \emph{always} $f_+,f_-\geq 0$.
\end{rmk}
\begin{rmk}
Also note that $f=f_+-f_-$ and $\abs{f}=f_++f_-$.
\end{rmk}
\end{dfn}
\begin{thm}[Integral]\label{Integral}
\begin{savenotes}
Let $\coord{X,\meas}$ be a topological measure space.\footnote{Recall that this means that (i) $X$ is $\sigma$-quasicompact, (ii) $\meas$ is regular, and (iii) $\meas$ is borel.}  Then, there is a unique function $\mathrm{I}:\Bor _0^+\rightarrow [0,\infty ]$, the \emph{integral}, such that
\begin{enumerate}
\item (Normalization)\label{Integral.Normalization} $\mathrm{I}(\chi _S)=\meas (S)$;
\item (Additivity)\label{Integral.Additivity} $\mathrm{I}(f+g)=\mathrm{I}(f)+\mathrm{I}(g)$;
\item (Nonnegative-homogeneity)\label{Integral.NonnegativeHomogeneity} $\mathrm{I}(af)=a\mathrm{I}(f)$ for $a\geq 0$;\footnote{Of course, when we extend the definition of the integral to not-necessarily-nonnegative functions, this will be true for \emph{all} $a\in \R$, but for the time being, we require that $a\geq 0$ so that $af$ itself is nonnegative.} and
\item (Lebesgue's Monotone Convergence Theorem)\label{Integral.LebesguesMonotoneConvergenceTheorem}\index{Lebesgue's Monotone Convergence Theorem} whenever $m\mapsto f_m$ is a nondecreasing sequence functions then
\begin{equation}
\lim _m\mathrm{I}(f_m)=\mathrm{I}(\lim _mf).\footnote{Note that $\lim _mf(x)$ \emph{always} exists, by the usual \nameref{MonotoneConvergenceTheorem}---either it is bounded, in which case it converges in $\R$, or it is unbounded, in which case it converges to $\infty$.  (I lied a bit\textellipsis as it is only nondecreasing almost-everywhere, then the limit exists only almost-everywhere.}
\end{equation}
\end{enumerate}
Moreover,
\begin{equation}
\int _X\dif \meas (x)\, f(x)\coloneqq \mathrm{I}(f)=[\meas \times \meas _{\mathrm{L}}]\left( \left\{ \coord{x,y}\in X\times [0,\infty ]:f(x)<y\right\} \right) .\footnote{$\meas _{\mathrm{L}}$ is lebesgue measure, simply to distinguish it from the measure on $X$.  The strict $<$ as opposed to $\leq$ is crucial for making a couple of arguments easier, similar to how it is more convenient to work with $[a,b)$ than it is $[a,b]$.}
\end{equation}\index[notation]{$\int _X\dif \meas (x)\, f(x)$}
\begin{rmk}
For $f:X\rightarrow [-\infty ,\infty ]$ integrable, we define
\begin{equation}
\int _X\dif \meas (x)\, f(x)\coloneqq \int _X\dif \meas (x)\, f_+(x)-\int _X\dif \meas (x)\, f_-(x).
\end{equation}
\end{rmk}
\begin{rmk}
For $S\subseteq X$, we define
\begin{equation}
\int _S\dif \meas (x)\, f(x)\coloneqq \int _X\dif \meas (x)\, \chi _S(x)f(x).
\end{equation}\index[notation]{$\int _X\dif \meas (x)\, f(x)$}
If $X=\R$ and $S=[a,b]$, we define
\begin{equation}
\int _a^b\dif x\, f(x)\coloneqq \int _{[a,b]}\dif \meas _{\mathrm{L}}(x)\, f(x)
\end{equation}\index[notation]{$\int _{[a,b]}\dif x\, f(x)$}
\end{rmk}
\begin{rmk}
If the measure is clear from context, we may just simply write $\dif x$ instead of $\dif \meas (x)$.\index[notation]{$\int _X\dif x\, f(x)$}  We may even write $\int _X\dif \meas \, f$ or just $\int _Xf$ if `the variable of integration' is irrelevant.
\end{rmk}
\begin{rmk}
Damn this is sexy.  First of all, we are characterizing the integral uniquely via certain properties it satisfies, and then furthermore, it turns out that this unique such function is simply given by \emph{the `area' under the curve}.  What do you think of that, Riemann?\footnote{Seriously, people actually claim that the lebesgue integral is not ``geometric''.  Da fuq?  Can someone please explain to me how limits of partitions of sums of areas of rectangles is more geometric than the area under the curve?}
\end{rmk}
\begin{rmk}
Because $\mathrm{I}$ is defined on $\Bor _0^+(X)$, it is implicit that if $f(x)=g(x)$ almost-everywhere, then $\mathrm{I}(f)=\mathrm{I}(g)$.  So, for example, the \emph{integral of the \nameref{DirichletFunction} is $0$}!\footnote{Another way to see this is from \ref{Integral.Normalization} and the fact that $\meas _{\mathrm{L}}(\Q )=0$ because $\Q$ is countable---see \cref{exr5.2.33}.}
\end{rmk}
\begin{rmk}
I happen to prefer to put the $\dif x$ in front of the integral,\footnote{For two reasons:  (i) it tells you immediately what the variable you are integrating with respect to is (just a slight convenience, especially when the integrand is complicated) and (ii) it is more analogous to how we write the derivative---no one writes $\dif f(x)/\dif x$---people write $\frac{\dif}{\dif x}f(x)$.} but don't let that confuse you---the meaning is just the same as $\int f(x)\dif x$.
\end{rmk}
\begin{proof}\footnote{Proof adapted from \cite[pg.~377]{Pugh}.}
\Step{Introduce notation}
As we will be making use of the set quite a bit, it will be useful to introduce the notation
\begin{equation}
\Gamma _f\coloneqq \left\{ \coord{x,y}\in X\times [0,\infty ]:y\leq f(x)\right\} .
\end{equation}

\Step{Define $\mathrm{I}$}
Let $f:X\rightarrow \R _0^+$ and define
\begin{equation}
\mathrm{I}(f)\coloneqq [\meas \times \meas _{\mathrm{L}}](\Gamma _f) .
\end{equation}

\Step{Show that $\mathrm{I}$ descends to a well-defined function on $\Mor _{\AlE}(X,[0,\infty ])$}
We must check that if $f=g$ in $\Mor _{\AlE}(X,[0,\infty ])$, then $\mathrm{I}(f)=\mathrm{I}(g)$.  Define
\begin{equation}
S\coloneqq \{ x\in X:f(x)\neq g(x)\} .
\end{equation}
By definition of $\AlE$, we have that $\meas (S)=0$.  We then have that
\begin{equation}
\Gamma _f\cap \Gamma _g^{\comp})=\{ \coord{x,y}\in X\times [0,\infty ]:y\leq f(x)\text{ and }y>g(x)\} \subseteq S\times [0,\infty ],
\end{equation}
and so
\begin{equation}
[\meas \times \meas _{\mathrm{L}}](\Gamma _f\cap \Gamma _g^{\comp})\leq [\meas \times \meas _{\mathrm{L}}](S\times [0,\infty ])=0\cdot \infty \coloneqq 0.
\end{equation}
Hence,
\begin{equation}
[\meas \times \meas _{\mathrm{L}}](\Gamma _f)=[\meas \times \meas _{\mathrm{L}}](\Gamma _f\cap \Gamma _g)+[\meas \times \meas _{\mathrm{L}}(\Gamma _f\cap \Gamma _g^{\comp})=\meas \times \meas _{\mathrm{L}}[\Gamma _f\cap \Gamma _g).
\end{equation}
By $f\leftrightarrow g$ symmetry, we likewise have that $[\meas \times \meas _{\mathrm{L}}](\Gamma _g)=[\meas \times \meas _{\mathrm{L}}](\Gamma _f\cap \Gamma _g)$.  Combining this with the previous equality gives us that $\mathrm{I}(f)=\mathrm{I}(g)$.

\Step{Show that $\mathrm{I}(\chi _S)=\meas (S)$}
\begin{equation}
\Gamma _{\chi _S}\coloneqq \left\{ \coord{x,y}\in X\times \R :\chi _S(x)\leq y\right\} =S\times [0,1]
\end{equation}
and so
\begin{equation}
\mathrm{I}(\chi _S)\coloneqq [\meas \times \meas _{\mathrm{L}}](\Gamma _{\chi _S})=\meas (S)\cdot 1=\meas (S).
\end{equation}

\Step{Show that $\mathrm{I}$ is additive}
We wish to show that $\mathrm{I}(f+g)=\mathrm{I}(f)+\mathrm{I}(g)$.  If either $f$ or $g$ is not integrable, then this is trivially satisfied as this equation then reads $\infty =\infty$.  Thus, without loss of generality, we may assume that $\mathrm{I}(f),\mathrm{I}(g)<\infty$.

For $f:X\rightarrow [0,\infty ]$, define $\tau _f:X\times [0,\infty \rightarrow X\times [0,\infty ]$ by
\begin{equation}
\tau _f(\coord{x,y})\coloneqq \coord{x,f(x)+y}.
\end{equation}
This definition was made so that we have
\begin{equation}
\Gamma _{f+g}=\Gamma _f\cup \tau _f(\Gamma _g)=
\end{equation}
is a disjoint union.\footnote{This requires the use of the \emph{strict} inequality in the definition of $\Gamma _f$ and $\Gamma _g$.}  Thus, it suffices to show that (i) $\tau _f(M)$ is measurable if $M$ is and (ii) that $[\meas \times \meas _{\mathrm{L}}](\tau _f(M))=[\meas \times \meas _{\mathrm{L}}](M)$ for $M\subseteq X\times [0,\infty ]$ measurable.  To show this, we apply \cref{prp5.1.39} (sets in topological measure spaces are measurable iff you can approximate them with open and closed sets).

\Step{Show that $\mathrm{I}$ is nonnegative homogeneous}
Note that
\begin{equation}
\Gamma _{af}\coloneqq \left\{ \coord{x,y}\in X\times \R :af(x)<y\right\} =\left\{ \coord{x,ay}\in X\times \R :f(x)<y\right\} =a\Gamma _f.
\end{equation}
Then, it follows from \cref{exr5.2.28} that
\begin{equation}
\mathrm{I}(af)\coloneqq \meas (\Gamma _{af})=\meas (a\Gamma _f)=a\meas (\Gamma _f)=a\mathrm{I}(f).
\end{equation}

\Step{Prove Lebesgue's Monotone Convergence Theorem}
Let us define $f_\infty :X\rightarrow [0,\infty ]$ by $f_\infty (x)\coloneqq \lim _mf(x)$.  For almost every $x\in X$, the sequence $m\mapsto f_m(x)$ is nondecreasing, and so by the usual \nameref{MonotoneConvergenceTheorem}, this limit exists in $[0,\infty ]$\footnote{If the seqeucne is bounded, the \nameref{MonotoneConvergenceTheorem} tells us it converges in $\R$.  Otherwise, it converges to $\infty$.} for almost every $x$, and so defines a function $f_\infty \in \Mor _{\AlE}(X,[0,\infty ])$.  For all the points where we do not have monotonicity, we may without loss of generality take $f_\infty$ to be infinite there.  This gives us that
\begin{equation}
\Gamma _{f_m}\subseteq \Gamma _{f_\infty}
\end{equation}
for all $m$, and hence
\begin{equation}
\mathrm{I}(f_m)\coloneqq [\meas \times \meas_{\mathrm{L}}](\Gamma _{f_m})\leq [\meas \times \meas _{\mathrm{L}}(f_\infty)=\mathrm{I}(f_\infty )\coloneqq \mathrm{I}(\lim _mf_m).
\end{equation}
Taking the limit, we get
\begin{equation}
\lim _m\mathrm{I}(f_m)\leq \mathrm{I}(\lim _mf_m).
\end{equation}

\Step{Show uniqueness}

\Step{Show we have equality if functions are measurable}
\end{proof}
\end{savenotes}
\end{thm}
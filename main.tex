%%%%%%%%%%%%%%%%%%%%%%%%%%%%%%%%%%%%%%%%%
% The Legrand Orange Book
% LaTeX Template
% Version 2.0 (9/2/15)
%
% This template has been downloaded from:
% http://www.LaTeXTemplates.com
%
% Mathias Legrand (legrand.mathias@gmail.com) with modifications by:
% Vel (vel@latextemplates.com)
%
% License:
% CC BY-NC-SA 3.0 (http://creativecommons.org/licenses/by-nc-sa/3.0/)
%
% Compiling this template:
% This template uses biber for its bibliography and makeindex for its index.
% When you first open the template, compile it from the command line with the 
% commands below to make sure your LaTeX distribution is configured correctly:
%
% 1) pdflatex main
% 2) makeindex main.idx -s StyleInd.ist
% 3) biber main
% 4) pdflatex main x 2
%
% After this, when you wish to update the bibliography/index use the appropriate
% command above and make sure to compile with pdflatex several times 
% afterwards to propagate your changes to the document.
%
% This template also uses a number of packages which may need to be
% updated to the newest versions for the template to compile. It is strongly
% recommended you update your LaTeX distribution if you have any
% compilation errors.
%
% Important note:
% Chapter heading images should have a 2:1 width:height ratio,
% e.g. 920px width and 460px height.
%
%%%%%%%%%%%%%%%%%%%%%%%%%%%%%%%%%%%%%%%%%

%----------------------------------------------------------------------------------------
%	PACKAGES AND OTHER DOCUMENT CONFIGURATIONS
%----------------------------------------------------------------------------------------

\documentclass[11pt,fleqn]{book} % Default font size and left-justified equations

%----------------------------------------------------------------------------------------

\input{structure} % Insert the commands.tex file which contains the majority of the structure behind the template

\usepackage{xparse}

% 'Punctuation'
\DeclarePairedDelimiter{\abs}{\lvert}{\rvert}
\newcommand{\blank}{{-}}
\newcommand{\blankdot}{{\cdot}}
\newcommand{\comma}{{,}}
\DeclarePairedDelimiter{\coord}{\langle}{\rangle}

% Operations
\DeclareMathOperator{\Ad}{Ad}
\DeclareMathOperator{\ad}{ad}
\DeclareMathOperator{\Aut}{Aut}
\DeclareMathOperator{\CO}{CO}
\newcommand{\Cech}{\check{\mathrm{C}}}
\newcommand{\Cl}{\mathcal{C}\ell}
\DeclareMathOperator{\Cls}{Cls}
\DeclareMathOperator{\Cmp}{Cmp}
\DeclareMathOperator{\co}{co}
\renewcommand{\c}{\mathrm{c}}
\newcommand{\comp}{\mathsf{c}}
\DeclareMathOperator{\Char}{Char}
\DeclareMathOperator{\Coker}{Coker}
\DeclareMathOperator{\coker}{coker}
\DeclareMathOperator{\colim}{colim}
\DeclareMathOperator{\csch}{csch}
\newcommand{\Curl}{\vec{\nabla}\times}
\DeclareMathOperator{\D}{D}
\newcommand{\Div}{\vec{\nabla}\cdot}
\DeclareMathOperator{\diag}{diag}
\DeclareMathOperator{\diam}{diam}
\newcommand{\dif}{\mathop{}\!\mathrm{d}}
\DeclareMathOperator{\dist}{dist}
\DeclareMathOperator{\EV}{E}
\DeclareMathOperator{\End}{End}
\DeclareMathOperator{\Ext}{Ext}
\DeclareMathOperator{\Erf}{Erf}
\DeclareMathOperator{\Gal}{Gal}
\newcommand{\Grad}{\vec{\nabla}}
\DeclareMathOperator{\Hom}{Hom}
\DeclareMathOperator{\Ima}{Im}
\DeclareMathOperator{\Imag}{Im}
\DeclareMathOperator{\Index}{Index}
\DeclareMathOperator{\Inn}{Inn}
\DeclareMathOperator{\Iso}{Iso}
\DeclareMathOperator{\id}{id}
\DeclareMathOperator{\Int}{Int}
\DeclareMathOperator{\Ker}{Ker}
\DeclareMathOperator{\lcm}{lcm}
\DeclareMathOperator{\Lie}{Lie}
\DeclareMathOperator{\MaxSpec}{MaxSpec}
\DeclareMathOperator{\Mor}{Mor}
\newcommand{\m}{\mathrm{m}}
\newcommand{\meas}{\m}
\NewDocumentCommand{\metric}{o o}{\IfNoValueTF{#1}{\IfNoValueTF{#2}{\abs{\blankdot \comma \blankdot}}{\abs*{\blankdot \comma \, #2}}}{\IfNoValueTF{#2}{\abs*{#1\comma \blankdot}}{\abs*{#1\comma \, #2}}}}
\newcommand{\mmod}[1]{\left( \mathrm{mod}\, #1\right)}
\DeclareMathOperator{\Nil}{Nil}
\NewDocumentCommand{\norm}{o}{\IfNoValueTF{#1}{\abs{\blankdot}}{\abs*{#1}}}
\renewcommand{\o}{\mathrm{o}}
\DeclareMathOperator{\op}{op}
\DeclareMathOperator{\ord}{ord}
\newcommand{\q}{\mathrm{q}}
\DeclareMathOperator{\Real}{Re}
\DeclareMathOperator{\Res}{Res}
\newcommand\restr[2]{{% we make restriction an ordinary symbol
  \left.\kern-\nulldelimiterspace % automatically resize the bar with \right
  #1 % the function
%  \vphantom{\big|} % pretend it's a little taller at normal size
  \right|_{#2} % this is the delimiter
  }}
\DeclareMathOperator{\rk}{rk}
\DeclareMathOperator{\sgn}{sgn}
\DeclareMathOperator{\Spec}{Spec}
\DeclareMathOperator{\Star}{Star}
\DeclareMathOperator{\supp}{supp}
\DeclareMathOperator{\Syl}{Syl}
\DeclareMathOperator{\T}{T}
\DeclareMathOperator{\Tp}{Tp}
\DeclareMathOperator{\TZero}{T0}
\NewDocumentCommand{\tangent}{m m}{\T _{#2}(#1)}    % For tangent space
\DeclareMathOperator{\tp}{tp}
\DeclareMathOperator{\Tr}{Tr}
\DeclareMathOperator{\tr}{tr}

\makeatletter
\DeclareFontFamily{U}{mathx}{\hyphenchar\font45}
\DeclareFontShape{U}{mathx}{m}{n}{
      <5> <6> <7> <8> <9> <10>
      <10.95> <12> <14.4> <17.28> <20.74> <24.88>
      mathx10
      }{}
\DeclareSymbolFont{mathx}{U}{mathx}{m}{n}
\def\mathabx@undefine#1{\let#1=\undefined}%
\mathabx@undefine{\prod}
\DeclareMathSymbol{\prod}   {1}{mathx}{"B1}
\mathabx@undefine{\coprod}
\DeclareMathSymbol{\coprod} {1}{mathx}{"B2}
\makeatother % The preceding lines are to use the symbols \prod and \coprod from mathabx package (mathptmx disallows \coprod by default for some reason, also use \prod from mathabx for consistency)

\makeatletter
\def\re@DeclareMathSymbol#1#2#3#4{%
    \let#1=\undefined
    \DeclareMathSymbol{#1}{#2}{#3}{#4}}
% no OMX used 
\expandafter\ifx\csname npxmath@scaled\endcsname\relax
  \let\npxmath@@scaled\@empty%
\else
  \edef\npxmath@@scaled{s*[\csname npxmath@scaled\endcsname]}%
\fi
\DeclareFontEncoding{LMX}{}{}
\DeclareFontSubstitution{LMX}{npxexx}{m}{n}
\DeclareFontFamily{LMX}{npxexx}{}
\DeclareFontShape{LMX}{npxexx}{m}{n}{<-> \npxmath@@scaled zplexx}{}
\DeclareSymbolFont{lettersA}{U}{npxmia}{m}{it}
\re@DeclareMathSymbol{\epsilonup}{\mathord}{lettersA}{15}
\makeatother
\renewcommand{\epsilon}{\epsilonup} % The preceding lines are to used the symbol \epsilonup from the newpx package.  We then use this for \epsilon as the other fonts use the ssame symbol for \epsilon and \varepsilon

% Relations
\DeclareFontFamily{U}{mathb}{\hyphenchar\font45}
\DeclareFontShape{U}{mathb}{m}{n}{
      <5> <6> <7> <8> <9> <10> gen * mathb
      <10.95> mathb10 <12> <14.4> <17.28> <20.74> <24.88> mathb12
      }{}
\DeclareSymbolFont{mathb}{U}{mathb}{m}{n}
\DeclareMathSymbol{\llcurly}     {3}{mathb}{"CE}
\DeclareMathSymbol{\ggcurly}     {3}{mathb}{"CF} % The preceding lines are to use just the symbols \llcurly and \ggcurly from mathabx package (which itself conflicts with amsmath)

\newcommand*{\llcurlyeq}{\mathrel{\vbox{\offinterlineskip\hbox{\scalebox{.9}{$\llcurly$}}\vskip-.6ex\hbox{$-$}\vskip-.75ex}}}
\newcommand*{\ggcurlyeq}{\mathrel{\vbox{\offinterlineskip\hbox{\scalebox{.9}{$\ggcurly$}}\vskip-.6ex\hbox{$-$}\vskip-.75ex}}}

\newcommand{\st}{\ensuremath{\text{ s.t.~}}}

% Constants
\newcommand{\e}{\mathrm{e}}
\newcommand{\im}{\mathrm{i}\text{\hspace{1pt}}}

% 'Sets'
\newcommand{\A}{\mathbb{A}}
\renewcommand{\C}{\mathbb{C}}
\newcommand{\F}{\mathbb{F}}
\newcommand{\K}{\mathbb{K}}
\newcommand{\N}{\mathbb{N}}
\newcommand{\Q}{\mathbb{Q}}
\newcommand{\R}{\mathbb{R}}
\newcommand{\Z}{\mathbb{Z}}

% Categories
\newcommand{\Ab}{\mathsf{Ab}}
\newcommand{\Alg}{\mathsf{Alg}}
\newcommand{\Cat}{\mathsf{Cat}}
\newcommand{\Com}{\mathsf{Com}}
\newcommand{\Crng}{\mathsf{Crng}}
\newcommand{\Grp}{\mathsf{Grp}}
\newcommand{\Mag}{\mathsf{Mag}}
\newcommand{\Man}{\mathsf{Man}}
\newcommand{\Met}{\mathsf{Met}}
\newcommand{\Pre}{\mathsf{Pre}}
\newcommand{\Rg}{\mathsf{Rg}}
\newcommand{\Rig}{\mathsf{Rig}}
\newcommand{\Ring}{\mathsf{Ring}}
\newcommand{\Rng}{\mathsf{Rng}}
\newcommand{\Semi}{\mathsf{Semi}}
\newcommand{\Set}{\mathsf{Set}}
\newcommand{\Simp}{\mathsf{Simp}}
\newcommand{\SN}{\mathsf{SN}}
\newcommand{\Top}{\mathsf{Top}}
\newcommand{\Uni}{\mathsf{Uni}}
\newcommand{\Vect}{\mathsf{Vect}}

% Custom environments
\NewDocumentEnvironment{textequation}{o}{
    \IfNoValueTF{#1}{%
        \ntextequation%
    }%
    {%
        \ytextequation{#1}%
    }
}
{
    \IfNoValueTF{#1}{%
        \endntextequation%
    }%
    {%
        \endytextequation%
    }%
}
\NewEnviron{ytextequation}[1]{%
    \begin{equation}\label{#1}%
        \pbox{.875\textwidth}{%
            \BODY%
        }%
    \end{equation}
}
\NewEnviron{ntextequation}{%
    \begin{equation}%
        \pbox{.875\textwidth}{%
            \BODY%
        }%
    \end{equation}
}
\NewEnviron{textequation*}{%
    \begin{equation*}%
    \pbox{.9\textwidth}{%
    \BODY%
    }%
    \end{equation*}
}

\newenvironment{solution}
    {\proof[Solution]}
    {\endproof}
    
% Custom commands
\newcommand{\blankline}{\vspace{\baselineskip}}

\newcounter{step}
\crefname{step}{Step}{Steps}
\newcommand{\Step}[1]{%
    \refstepcounter{step}%
    %
    \ifthenelse{\value{step}=1}{}{%
        \par%
        \blankline
    }
    \noindent
    \textsc{Step  \thestep :  #1}
    
    \noindent
}

% Re-defines proof environment to reset step counter
\makeatletter
\let\oldproof\proof
\def\proof{\setcounter{step}{0}\@ifnextchar[\proof@i \proof@ii}
\def\proof@i[#1]{\proof[#1]}
\def\proof@ii{\oldproof}
\makeatother


\begin{document}

%----------------------------------------------------------------------------------------
%	TITLE PAGE
%----------------------------------------------------------------------------------------

\begingroup
\thispagestyle{empty}
\begin{tikzpicture}[remember picture,overlay]
\coordinate [below=12cm] (midpoint) at (current page.north);
\node at (current page.north west)
{\begin{tikzpicture}[remember picture,overlay]
\node[anchor=north west,inner sep=0pt] at (0,0) {\includegraphics[width=\paperwidth]{background}}; % Background image
\draw[anchor=north] (midpoint) node [fill=ocre!30!white,fill opacity=0.6,text opacity=1,inner sep=1cm]{\Huge\centering\bfseries\sffamily\parbox[c][][t]{\paperwidth}{\centering Introduction to Analysis\\[15pt] % Book title
{\huge Jonathan Gleason}}}; % Author name
\end{tikzpicture}};
\end{tikzpicture}
\vfill
\endgroup

%----------------------------------------------------------------------------------------
%	TABLE OF CONTENTS
%----------------------------------------------------------------------------------------

\chapterimage{norm.jpg} % Table of contents heading image

\pagestyle{empty} % No headers

\tableofcontents % Print the table of contents itself

\cleardoublepage % Forces the first chapter to start on an odd page so it's on the right

\pagestyle{fancy} % Print headers again

\textbf{TO DO}
\begin{itemize}
\item Alternating harmonic series converges to $\ln (2)$.

\item Define $\arctan$.

\item Counter-examples for Heine-Borel and Bolzano-Weierstrass (closed unit ball in $L^2$)

\item Fix spacing after exercises in definitions

\item Fix margins of textequations inside remarks

\item Stone-Weierstrass Theorem (requires integration---see Baby Rudin pg.~Theorem 7.32)

\item Fix doubling of footnotes in textequations

\item Switch to XeLaTeX

\item Add diagrams

\item Change Dirichlet

\item Integration to Picard's theorem to exponential (tangent to power rule) to sine and cosine to Weierstrass's function to topology on $\Mor _{\Man}(\R ^d,
R )$ to Taylor's Theorem to Clairut's Theorem.

\item Inverse Function Theorem

\item Nonmeasurable set

\item Fix step count

\item Change ``star-refinement'' to ``the star refines''. 
\end{itemize}



\chapter{What is a number?}\label{chp1}

What is a number?  Of course, the answer is that the word ``number'' means whatever we declare it to mean, and indeed, what a mathematician means when he says the word ``number'' will vary from context to context.  There is no single universal type of number that will work in all contexts, so instead of coming up with a single answer to this question, we will instead define and develop several different `number systems', those being the natural numbers, the integers, the rational numbers, and the real numbers.  Being the most fundamental, we start with the natural numbers.

\section{Cardinality of sets and the natural numbers}

The motivation for introducing the natural numbers is that these are the things that allow us to \emph{count} things.  But what does it mean to count?  We will make sense of this by making sense of the notion of the \emph{cardinality} of a set, the cardinality being a sort of measure of how many elements the set contains.

\subsection{The natural numbers as a set}

The first step in defining the cardinality of sets is being able to decide when two sets have the same number of elements.  So, suppose we are given two sets $X$ and $Y$ and that we would like to determine whether $X$ and $Y$ have the same number of elements.  How would you do this?

Intuitively, you could start by trying to label all the elements in $Y$ by elements of $X$, without repeating labels.  If either (i) you ran out of labels before you finished labeling all elements in $Y$ or (ii) you were forced to use a label more than once, then you could deduce that $X$ and $Y$ did \emph{not} have the same number of elements.  To make this precise, we think of this labeling as a function from $X$ to $Y$.  Then, the first case corresponds to this labeling function not being surjective and the second case corresponds to this labeling function not being injective.

The more precise intuition is then that the two sets $X$ and $Y$ have the same number of elements, that is, the same cardinality, iff there is a bijection $f:X\rightarrow Y$ between them:  that $f$ is an injection says that we don't use a label more than once (or equivalently that $Y$ has at least as many elements as $X$) and that $f$ is a surjection says that we label everything at least once (or equivalently that $X$ has at least as many elements as $Y$).  In other words, according to \cref{exr2.1.3}, two sets should have the same cardinality iff they are isomorphic in $\Set$.\footnote{$\Set$ is the category of sets.  If you are reading this linearly and you see something you don't recognize, chances are it's in the appendix.  Use the index and index of notation at the end of the notes to find exactly where.}
\begin{dfn}[Equinumerous]
Let $X$ and $Y$ be sets.  Then, $X$ and $Y$ are \emph{equinumerous}\index{Equinumerous} iff $X\cong _{\Set}Y$.
\end{dfn}

So we've determined what it means for two sets to have the same cardinality, but what actually \emph{is} a cardinality?  The trick is to identify a cardinal with the collection of all sets which have that cardinality.
\begin{dfn}[Cardinal number]\label{dfn1.1.2}
A \emph{cardinal number}\index{Cardinal number} is an element of
\begin{equation}
\K \coloneqq \Set _0/\cong _{\Set}\coloneqq \left\{ [X]_{\cong _{\Set}}:X\in \Set _0\right\} .
\end{equation}
\begin{rmk}
In other words, a cardinal is an equivalence class of sets, the equivalence relation being equinumerosity.\footnote{Recall that the relation of isomorphism is an equivalence relation in every category.  See \cref{exrA.2.11}.}  Furthermore, for $X$ a set, we write $\abs{X}\coloneqq [X]_{\cong _{\Set}}$\index[notation]{$\abs{X}$}.
\end{rmk}
\end{dfn}

The idea then is that the natural numbers are precisely those cardinals which are finite.  But what does it mean to be finite?  This is actually a bit tricky.

Obviously we don't have a precise definition yet, but everyone has an intuitive idea of what it means to be infinite.  Consider an infinite set $X$.  Now remove one element $x_0\in X$ to form the set $U\coloneqq X\setminus \{ x_0\}$.  For any reasonable definition of ``infinite'', removing a single element from an infinite set should not change the fact that it is infinite, and so $U$ should still be infinite.  In fact, more should be true.  Not only should $U$ still be infinite, but it should still have the same cardinality as $X$.\footnote{We will see in the next chapter that there are infinite sets which are not of the same cardinality.  That is, in this sense, there is more than one type of infinity.}  It is this idea that we take as the defining property of being infinite.
\begin{dfn}[Infinite]
Let $X$ be a set.  Then, $X$ is \emph{infinite}\index{Infinite} iff there is a bijection from $X$ to a \emph{proper} subset of $X$.
\end{dfn}
\begin{dfn}[Finite]
Let $X$ be a set.  Then, $X$ is \emph{finite}\index{Finite} iff it is not infinite.
\end{dfn}
And now finally:
\begin{dfn}[Natural numbers]
The \emph{natural numbers}\index{Natural numbers}, $\N$, are defined as
\begin{equation}
\N \index[notation]{$\N$}\coloneqq \left\{ \abs{X}:X\in \Set _0\text{ is finite.}\right\} .
\end{equation}
In words, the natural numbers are precisely the cardinals of finite sets.
\begin{rmk}
Some people take the natural numbers to not include $0$.  This is a bit silly for a couple of reasons.  First of all, if you think of the natural numbers as cardinals, as we are doing here, then $0$ has to be a natural number as it is the cardinality of the empty set.  Furthermore, as we shall see in the next section, it makes the algebraic structure of $\N$ slightly nicer because $0$ acts as an additive identity.
\end{rmk}
\end{dfn}

\subsection{The natural numbers as an integral crig}

Great!  We now know what the natural numbers are.  Our next objective then is to be able to add and multiply natural numbers.  In fact, we will define not only what it means to add and multiply natural numbers, but instead we will define what it means to add and multiply \emph{any} cardinal numbers.  Then, we will just need to check that the sum and product of two finite cardinals is again finite, so the the operations restricted to $\N$ are closed.

\begin{dfn}[Addition and multiplication]
Let $m,n\in \N$ and let $M$ and $N$ be sets such that $m=\abs{M}$ and $n=\abs{N}$.  Then, we define
\begin{equation}
m+n\coloneqq \abs{M\sqcup N}\text{ and }mn\coloneqq \abs{M\times N}.
\end{equation}
\begin{rmk}
Recall that $\abs{M}$ means the equivalence class of the set $M$ under the equivalence relation of equinumerosity (see the definition of a cardinal, \cref{dfn1.1.2}).  Whenever we define an operation on equivalence classes that makes reference to a specific representative of that equivalence class, we must check that our definition does not depend on this representative.  For example, perhaps if I take a different set $M'$ with $\abs{M'}=\abs{M}$, it will turn out that $\abs{M'\sqcup N}\neq \abs{M\sqcup N}$.  If this happens, then our definition doesn't make sense.  Of course, it doesn't happen, but we need to check that it doesn't happen.  This is what it means to be \emph{well-defined}\index{Well-defined}.
\end{rmk}
\begin{prp}
Addition and multiplication are well-defined.
\begin{proof}
Let $M_1,M_2,N_1,N_2$ be sets with $\abs{M_1}=\abs{M_2}$ and $\abs{N_1}=\abs{N_2}$.  By definition, this means that there are bijections $f:M_1\rightarrow M_2$ and $g:N_1\rightarrow N_2$.  We would like to show that $\abs{M_1\sqcup N_1}=\abs{M_2\sqcup N_2}$.  To show this, by definition, we need to construction a bijection from $M_1\sqcup N_1$ to $M_2\sqcup N_2$.  We do this as follows.  Define $h:M_1\sqcup N_1\rightarrow M_2\sqcup N_2$ by
\begin{equation}
h(x)\coloneqq \begin{cases}f(x) & \text{if }x\in M_1 \\ g(x) & \text{if }x\in N_1\end{cases}.
\end{equation}

We now check that $h$ is a bijection.  Suppose that $h(x_1)=h(x_2)$.  This single element must be contained in either $M_2$ or $N_2$, so without loss of generality suppose that $h(x_1)=h(x_2)\in M_2$.  Then, from the definition of $h$, we have that $f(x_1)=h(x_1)=h(x_2)=f(x_2)$, and so because $f$ is injective, we have that $x_1=x_2$.  Thus, $h$ is injective.  To show that $h$ is surjective, let $y\in M_2\sqcup N_2$.  Without loss of generality, suppose that $y\in M_2$.  Then, because $f$ is surjective, there is some $x\in M_1$ such that $f(x)=y$, so that $h(x)=f(x)=y$.  Thus, $h$ is surjective, and hence bijective.

Thus, we have shown that
\begin{equation}
\abs{M_1\sqcup N_1}=\abs{M_2\sqcup N_2},
\end{equation}
so that addition is well-defined.

\begin{exr}
Complete the proof by showing that multiplication is well-defined.
\end{exr}
\end{proof}
\end{prp}
\end{dfn}

\begin{dfn}[$0$ and $1$]
Define
\begin{equation}
0\coloneqq \abs{\emptyset},\qquad 1\coloneqq \abs{\{ \emptyset \}}\in \N .
\end{equation}
In words, $0$ is the cardinality of the empty set and $1$ is the cardinality of the set that contains the empty set and only the empty set.
\end{dfn}
\begin{prp}\label{prpA.1.16}
$(\K ,+,\cdot ,0,1)$ is an integral crig.\footnote{Of course, $\K$ is not actually a set, but we don't care about this.  See the remark in \cref{sbsA.1.1} \nameref{sbsA.1.1}.}
\begin{proof}
We simply need to verify the properties of the definition of a crig, \cref{dfnA.1.33} (and also check that it is integral, \cref{dfnA.1.69}).  We first check that $+$ is associative, so let $m,n,o\in \N$ and write $m=\abs{M},n=\abs{N},o=\abs{O}$ for sets $M$, $N$, and $O$.  Then,
\begin{equation}
\begin{split}
(m+n)+o & =\left( \abs{M}+\abs{N}\right) +\abs{O}=\abs{M\sqcup N}+\abs{O}=\abs{(M\sqcup N)\sqcup O}=\abs{M\sqcup (N\sqcup O)} \\
& =m+(n+o).
\end{split}
\end{equation}
A similar argument shows that additive is commutative.  As for the additive identity, we have
\begin{equation}
m+0=\abs{M\sqcup \emptyset}=\abs{M}=m=0+m.
\end{equation}
Thus, $(\K ,+)$ is a commutative monoid.

\begin{exr}
Check that $(\K ,\cdot )$ is a commutative semigroup.
\end{exr}
\begin{exr}
To show that $\K$ is a crig, there is one final property to check.  What is it?  Check it.
\end{exr}

To show that $\K$ is integral, suppose that $mn=0$.  Then, there must be a bijection between $M\times N$ and the empty-set, which implies that $M\times N$ is empty.  But if neither $M$ nor $N$ is empty, then $M\times N$ will be nonempty.  Thus, we must have that either $M$ or $N$ is empty, or equivalently, that either $m=0$ or $n=0$, so that $\K$ is integral.
\end{proof}
\end{prp}

Now we must check that addition and multiplication restrict to operations on the collection of finitely cardinalities, namely, $\N$.  This amounts to showing that the sum and product of finite cardinalities are both finite.
\begin{prp}
If $M,N$ are finite sets, then $M\sqcup N$ and $M\times N$ are finite sets.
\begin{proof}
We first show that $M\sqcup \{ \ast \}$ is finite if $M$ is.  We proceed by contradiction:  suppose there is a proper subset $S\subset M\sqcup \{ \ast \}$ and a bijection $f:M\sqcup \{ \ast \} \rightarrow S$.  If $M$ is empty, then $\{ \ast \}$ is finite because its only proper subset is the empty set to which there can be no bijection (in fact, there is \emph{no} function from a nonempty set to the empty set; see \cref{exrA.1.23}).  Thus, we may without loss of generality assume that $M$ is nonempty.  So, let $x_0\in M$.  We know that either $S$ is of the form $S=M'$ for $M'\subseteq M$ or $S=M'\sqcup \{ \ast \}$ for $M'\subset M$.  By relabeling $\ast$ as $x_0$ and $x_0$ as $\ast$, we may as well assume that we are in the latter case, so that we have a bijection $f:M\sqcup \{ \ast \} \rightarrow M'\sqcup \{ \ast \}$ for $M'\subset M$.  (Forget the label $x_0$; we will want that notation later to refer to something else.)  If $\ast$ maps to $\ast$ under $f$, then the restriction of $f$ to $M$ yields a bijection from $M$ to $M'$ showing that $M$ is infinite:  a contradiction.  Thus, we may as well assume that $f(x_0 )=\ast$ for $x_0\in M$.  Let $g:M\sqcup \{ \ast \} \rightarrow M\sqcup \{ \ast \}$ be any bijection which sends $\ast$ to $x_0$ (such a bijection exists by \cref{exrA.1.27}).  Then, $f\circ g:M\sqcup \{ \ast \} \rightarrow M'\sqcup \{ \ast \}$ is a bijection such that the image of $M$ is $M'$.  Thus, the restriction of $f\circ g$ to $M$ yields a bijection of $M$ onto a proper subset, showing that $M$ is infinite:  a contradiction.  Thus, $M\sqcup \{ \ast \}$ is finite.

Applying this result inductively shows that $M\sqcup N$ is finite if both $M$ and $N$ are.

To see that $M\times N$ is finite, think of the product as (\cref{exrA.1.28})
\begin{equation}
M\times N=\sqcup _{y\in N}M.
\end{equation}
That $M\times N$ is finite now follows inductively from the fact that the disjoint union of two finite sets is finite.
\end{proof}
\end{prp}
\begin{crl}
$(\N ,+,\cdot ,0,1)$ is an integral crig.
\end{crl}

\subsection{The natural numbers as a well-ordered set}

For the moment, we will set aside the algebraic structure we have just defined on $\N$ and equip $\N$ with a preorder (which turns out to be a well-order).  Then, in the next subsection, we will show that these two structures are compatible in a way that makes $\N$ into a well-ordered integral crig.  Once again, well will in fact define the preorder on all of $\K$ and show that it is a well-order on $\K$.  It will then follow automatically that it restricts to a well-order on $\N$.

\begin{dfn}\label{dfn1.1.23}
Let $m,n\in \K$ and let $M$ and $N$ be sets such that $m=\abs{M}$ and $n=\abs{N}$.  Then, we define $m\leq n$\index[notation]{$m\leq n$} iff there is an injective map from $M$ to $N$.
\begin{exr}
Check that $\leq$ is well-defined.
\end{exr}
\end{dfn}
\begin{prp}
$(\K ,\leq )$ is a preordered set.
\begin{proof}
Recall that being a preorder just means that $\leq$ is reflexive and transitive (see \cref{dfnA.1.19}).

Let $m,n,o\in \N$ and let $M,N,O$ be sets such that $m=\abs{M}$, $n=\abs{N}$, $o=\abs{O}$.  The identity map from $M$ to $M$ is an injection (and, in fact, a bijection), which shows that $m=\abs{M}\leq \abs{M}=m$, so that $\leq$ is reflexive.

To show transitivity, suppose that $m\leq n$ and $n\leq o$.  Then, there is an injection $f:M\rightarrow N$ and an injection from $g:N\rightarrow O$.  Then, $g\circ f:M\rightarrow O$ is an injection (this is part of \cref{exrA.1.10}), and so we have $m=\abs{M}\leq \abs{O}=o$, so that $\leq$ is transitive, and hence a preorder.
\end{proof}
\end{prp}

The next result is perhaps the first theorem we have come to that has a nontrivial amount of content to it.
\begin{thm}[Bernstein-Cantor-Schr\"{o}der Theorem]\index{Bernstein-Cantor-Schr\"{o}der Theorem}\label{thm1.1.26}
$(\K ,\leq )$ is a partially-ordered set.
\begin{rmk}
This theorem is usually stated as ``If there is an injection from $X$ to $Y$ and there is an injection from $Y$ to $X$, then there is a bijection from $X$ to $Y$.''.
\end{rmk}
\begin{proof}\footnote{Proof adapted from \cite[pg.~29]{Abbott}.}
\Step{Recall what it means to be a partial-order}
Recall that being a partial-order just means that $\leq$ is an antisymmetric preorder.  We have just shown that $\leq$ is a preorder (see \cref{dfnA.1.24}), so all that remains to be seen is that $\leq$ is antisymmetric.

\Step{Determine what explicitly we need to show}
Let $m,n\in \K$ and let $M,N$ be sets such that $m=\abs{M}$ and $n=\abs{N}$.  Suppose that $m\leq n$ and $n\leq m$.  By definition, this means that there is an injection $f:M\rightarrow N$ and an injection $g:N\rightarrow M$.  We would like to show that $m=n$.  By definition, this remains we must show that there is a bijection from $M$ to $N$.

\Step{Note the existence of left-inverse to both $f$ and $g$}
We use the result of \cref{exrA.1.9} which says that both $f$ and $g$ have left inverses.  Denote these inverses by $f^{-1}:N\rightarrow M$ and $g^{-1}:M\rightarrow N$ respectively, so that
\begin{equation}
f^{-1}\circ f=\id _M\text{ and }g^{-1}\circ g=\id _N.
\end{equation}
Note that it is \emph{not} necessarily the case that $f\circ f^{-1}=\id _N$ (and similarly for $g$).

\Step{Define $C_x$}
Fix an element $x\in M$ and define
\begin{equation}\label{1.1.26}
\begin{multlined}
C_x\coloneqq \left\{ \ldots ,g^{-1}\left( f^{-1}\left( g^{-1}(x)\right) \right) ,f^{-1}\left( g^{-1}(x)\right) ,g^{-1}(x),x, \right. \\ \left. f(x),g\left( f(x)\right) ,f\left( g\left( f(x)\right) \right) ,\ldots \right\} \subseteq M\sqcup N.
\end{multlined}
\end{equation}
(The ``$C$'' is for ``chain''.)

\Step{Show that $\{ C_x:x\in M\}$ forms a partition of $M\sqcup N$}
We now claim that the collection $\left\{ C_x:x\in M\right\}$ forms a partition of $M\sqcup N$ (recall that this means that any two given $C_x$s are either identical or disjoint; see \cref{dfnA.1.11}).  If $C_{x_1}$ is disjoint from $C_{x_2}$ we are done, so instead suppose that there is some element $x_0$ that is in both $C_{x_1}$ and $C_{x_2}$.  First, let us do the case in which $x_0\in M$.  From the definition of $C_x$ \eqref{1.1.26}, we then must have that
\begin{equation}
[g\circ f]^k(x_1)=x_0=[g\circ f]^l(x_2)
\end{equation}
for some $k,l\in \Z$.  Without loss of generality, suppose that $k\leq l$.  Then, applying $f^{-1}\circ g^{-1}$ to both sides of this equation $k$ times,\footnote{If $k$ happens to be negative, it is understood that we instead apply $g\circ f$ $-k$ times.} we find that
\begin{equation}
x_1=[g\circ f]^{l-k}(x_2).
\end{equation}
In other words, $x_1\in C_{x_2}$.  Not only this, but $f(x_1)\in C_{x_2}$ as well because $f(x_1)=f\left( [g\circ f]^{l-k}(x_2)\right)$.  Similarly, $g^{-1}(x_1)\in C_{x_2}$, and so on.  It follows that $C_{x_1}\subseteq C_{x_2}$.  Switching $1\leftrightarrow 2$ and applying the same arguments gives us $C_{x_2}\subseteq C_{x_1}$, and hence $C_{x_1}=C_{x_2}$.  Thus, indeed, $\left\{ C_x:x\in M\right\}$ forms a partition of $M\sqcup N$.  It follows that (because partitions determine equivalence relations; see \cref{exrA.1.41})
\begin{equation}\label{1.1.29}
C_x=C_{x'}\text{ for all }x'\in C_x.
\end{equation}

\Step{Define $X_1,X_2,Y_1,Y_2$}
Now define
\begin{equation}\label{1.1.30}
A\coloneqq \bigcup _{\substack{x\in M\st \\ C_x\cap N\subseteq f(M)}}C_x
\end{equation}
as well as
\begin{equation}
X_1\coloneqq M\cap A,\qquad Y_1\coloneqq N\cap A,\qquad X_2\coloneqq M\cap A^{\comp},\qquad Y_2\coloneqq N\cap A^{\comp}
\end{equation}

\Step{Show that $\restr{f}{X_1}:X_1\rightarrow Y_1$ is a bijection}
We claim that $\restr{f}{X_1}:X_1\rightarrow Y_1$ is a bijection.  Of course, it is injective because $f$ is.  To show surjectivity, let $y\in Y_1\coloneqq N\cap A$.  From the definition of $A$ \eqref{1.1.30}, we see that $y\in C_x\cap N$ for some $C_x$ with $C_x\cap N\subseteq f(M)$, so that $y=f(x')$ for some $x'\in M$.  We still need to show that $x'\in X_1$.  However, we have that $x'=f^{-1}(y)$, and so as $y\in C_x$, we have that $x'=f^{-1}(y)\in C_x$ as well.  We already had that $C_x\cap N\subseteq f(M)$, so that indeed $x'\in A$, and hence $x'\in X_1$.  Thus, $\restr{f}{X_1}:X_1\rightarrow Y_1$ is a bijection.

\Step{Show that $\restr{g}{Y_2}:Y_2\rightarrow X_2$ is a bijection}
We now show that $\restr{g}{Y_2}:Y_2\rightarrow X_2$ is a bijection.  Once again, all we must show is surjectivity, so let $x\in X_2=M\cap A^{\comp}$. It thus cannot be the case that $C_x\cap N$ is contained in $f(M)$, so that there is some $y\in C_x\cap N$ such that $y\notin f(M)$.  Then, by virtue of \eqref{1.1.29}, we have that $C_x=C_y$, and in particular $x\in C_y$.  From the definition of $C_y$ \eqref{1.1.26}, it follows that either (i) $x=y$, (ii) $x$ is in the image of $f^{-1}$, or (iii) $x$ is in the image of $g$.  Of course it cannot be the case that $x=y$ because $x\in M$ and $y\in N$.  Likewise, it cannot be the case that $x$ is in the image of $f^{-1}$ because $x\in A^{\comp}$.  Thus, we must have that $x=g(y')$ for some $y'\in N$.  Once again, we still must show that $y'\in Y_2$; however, we have that $y'=g^{-1}(x)$, so that $y'\in C_x\subseteq A^{\comp}$.  Thus, $y'\in A^{\comp}$, and so $y'\in Y_2$.  Thus, $\restr{g}{Y_2}:Y_2\rightarrow X_2$ is a bijection.

\Step{Construct the bijection from $M$ to $N$}
Finally, we can define the bijection from $M$ to $N$.  We define $h:M\rightarrow N$ by
\begin{equation}
h(x)\coloneqq \begin{cases}f(x) & \text{if }x\in X_1 \\ g^{-1}(x) & \text{if }x\in X_2\end{cases}.
\end{equation}
Note that $\{ X_1,X_2\}$ is a partition of $M$ and $\{ Y_1,Y_2\}$ is a partition of $N$.  To show injectivity, suppose that $h(x_1)=h(x_2)$.  If this element is in $Y_1$, then because $\restr{f}{X_1}:X_1\rightarrow Y_1$ is a bijection, it follows that both $x_1,x_2\in X_1$, so that $f(x_1)=h(x_1)=h(x_2)=f(x_2)$, and hence that $x_1=x_2$.  Similarly if this element is contained in $Y_2$.  To show surjectivity, let $y\in N$.  First assume that $y\in Y_1$.  Then, $f^{-1}(y)\in X_1$, so that $h\left( f^{-1}(y)\right) =y$.  Similarly, if $y\in Y_2$, then $h\left( g(y)\right) =y$.  Thus, $h$ is surjective, and hence bijective.
\end{proof}
\end{thm}

\begin{thm}\label{thm1.1.34}
$(\K ,\leq )$ is well-ordered.
\begin{proof}\footnote{Proof adapted from \cite{Honig}.}
\Step{Conclude that it suffices to show that every nonempty subset has a smallest element}
By \cref{prpA.1.51}, we do not need to check totality explicitly, and so it suffices to show that every nonempty subset of $\K$ has a smallest element.

\Step{Define $\mathcal{T}$ as a preordered set}
So, let $S\subseteq \K$ be a nonempty collection of cardinals and for each $m\in S$ write $m=\abs{M_m}$ for some set $M_m$.  Define
\begin{equation}
M\coloneqq \prod _{m\in S}M_m
\end{equation}
and
\begin{equation}\label{1.1.39}
\begin{split}
\mathcal{T} & \coloneqq \left\{ T\subseteq M:T\in \Set _0;\right. \\
& \qquad \left.  \text{ for all }x,y\in T\text{, if }x\neq y\text{ it follows that }x_m\neq y_m\text{ for all }m\in S\text{.}\right\} .
\end{split}
\end{equation}
Note that the statement that $T\in \Set _0$ is just the statement that $T$ is an \emph{actual} set, instead of just a set-like object (i.e.~a proper class).  Order $\mathcal{T}$ by inclusion.

\Step{Verify that $\mathcal{T}$ satisfies  the hypotheses of Zorn's Lemma}
We wish to apply Zorn's Lemma (\cref{ZornsLemma}) to $\mathcal{T}$.  To do that of course, we must first verify the hypotheses of Zorn's Lemma.  $\mathcal{T}$ is a partially-ordered set by \cref{exrA.1.26}.  Let $\mathcal{W}\subseteq \mathcal{T}$ be a well-ordered subset and define
\begin{equation}
W\coloneqq \bigcup _{T\in \mathcal{W}}T.
\end{equation}
It is certainly the case that $T\subseteq W$ for all $T\in \mathcal{W}$.  In order to verify that $W$ is indeed an upper-bound of $\mathcal{W}$ in $\mathcal{T}$, however, we need to check that $W$ is actually an element of $\mathcal{T}$.  So, let $x,y\in W$ be distinct.  Then, there are $T_1,T_2\in \mathcal{W}$ such that $x\in T_1$ and $y\in T_2$.  Because $\mathcal{W}$ is totally-ordered, we may without loss of generality assume that $T_1\subseteq T_2$.  In this case, both $x,y\in T_2$.  As $T_2\in \mathcal{T}$, it then follows that $x_m\neq x_m$ for all $m\in S$.  It then follows in turn that $W\in \mathcal{T}$.

\Step{Conclude the existence of a maximal element}
The hypotheses of Zorn's Lemma being verified, we deduce that there is a maximal element $T_0\in \mathcal{T}$.

\Step{Show that there is some projection whose restriction to the maximal element is surjective}
Let $\pi _m:M\rightarrow M_m$ be the canonical projection.  We claim that there is some $m_0\in S$ such that $\pi _{m_0}(T_0)=M_{m_0}$.  To show this, we proceed by contradiction:  suppose that for all $m\in M$ there is some element $x_m\in M_m\setminus \pi _m(T_0)$.  Then, $T_0\cup \{ x\}\in \mathcal{T}$ is strictly larger than $T_0$:  a contradiction of maximality.  Therefore, there is some $m_0\in S$ such that $\pi _{m_0}(T_0)=M_{m_0}$.

\Step{Construct an injection from $M_{m_0}$ to $M_m$ for all $m\in S$}
The defining condition of $\mathcal{T}$, \eqref{1.1.39}, is simply the statement that $\restr{\pi _m}{T}:T\rightarrow M_m$ is injective for all $T\in \mathcal{T}$.  In particular, by the previous step, $\restr{\pi _{m_0}}{T_0}:T_0\rightarrow M_{m_0}$ is a bijection.  And therefore, the composition $\pi _m\circ \restr{\pi _{m_0}}{T_0}:M_{m_0}\rightarrow M_m$ is an injection from $M_{m_0}$ to $M$.  Therefore,
\begin{equation}
m_0=\abs{M_{m_0}}\leq \abs{M_m}=m
\end{equation}
for all $m\in S$.  That is, $m_0$ is the smallest element of $S$, and so $\K$ is well-ordered.
\end{proof}
\end{thm}
\begin{crl}
$(\N ,\leq )$ is a well-ordered set.
\end{crl}

\subsection{The natural numbers as a well-ordered integral crig}

We have shown that $(\N ,+,\cdot )$ is a crig and that $(\N ,\leq )$ is a well-ordered.  We now finally show how these two different structures, the algebraic structure and the order structure, are compatible.  But before we do that, of course, we have to make precise what we mean by the word ``compatible''.
\begin{dfn}[Preordered rg]\label{dfn1.1.38}
A \emph{preordered rg}\index{Preordered rg} is a set $X$ equipped with two binary operations $+$ and $\cdot$, and a relation $\leq$, so that
\begin{enumerate}
\item \label{enm1.1.38.1}$(X,+,0,\cdot )$ is a rg,
\item \label{enm1.1.38.2}$(X,\leq )$ is a preordered set,
\item \label{enm1.1.38.3}$x\leq y$ implies that $x+z\leq y+z$ for all $x,y,z\in X$,
\item \label{enm1.1.38.4}$x\leq y$ and $0\leq z$ implies that $xz\leq yz$,
\end{enumerate}
and furthermore, in the case $(X,+,0,\cdot ,1)$ is a rig, that $0\leq 1$.
\begin{rmk}
If $X$ is a \emph{totally}-ordered ring, we automatically have $0\leq 1$.  By totality, we automatically have that either $0\leq 1$ or $1\leq 0$.  Do you see why the latter cannot happen (if $1\neq 0$)?

In general, however, we do need to make the requirement that $0\leq 1$.
\end{rmk}
For $X$ a preordered rg, we write
\begin{equation}
X^+\index[notation]{$X^+$}\coloneqq \left\{ x\in X:x>0\right\} \text{ and }X_0^+\index[notation]{$X_0^+$}\coloneqq \left\{ x\in X:x\geq 0\right\} .
\end{equation}
\begin{rmk}
A partially-ordered rg, totally-ordered rg, etc.~are just preordered rgs whose underlying preorder is respectively a partial-order, total-order, etc.
\end{rmk}
\begin{rmk}
In a totally-ordered ring, we define $\sgn :X\rightarrow \{ 0,1,-1\} \subseteq X$ by
\begin{equation}
\sgn (x):=\begin{cases}1 & \text{if }x>0 \\ 0 & \text{if }x=0 \\ -1 & \text{if }x<0\end{cases}.
\end{equation}\index[notation]{$\sgn$}
\end{rmk}
This is the \emph{signum function}\index{Signum function} and is meant merely to return the sign of an element.
\end{dfn}
One thing to note about preordered rngs is that, to define the order, it suffices only to be able to compare everything with $0$.
\begin{exr}\label{exr1.1.41}
Let $(X,+,0,-)$ be a rng and let $P$ be a subset of $X$ (thought of as the nonnegative elements) that is (i) closed under addition, (ii) closed under multiplication, and (iii) contains $0$.  Show that there is a unique preorder $\leq$ on $X$ such that (i) $(X,+,0,-,\leq )$ is a preordered rng and (ii) $P=\left\{ x\in X:x\geq 0\right\}$..  Furthermore, show that $\leq$ is a partial-order iff $x,-x\in P$ implies $x=0$, and furthermore show that $\leq$ is a total-order iff, in addition, either $x\in P$ or $-x\in P$ for all $x\in X$.
\end{exr}
\begin{dfn}[The category of preordered rgs]
The category of preordered rgs is the category $\Pre \Rg$\index[notation]{$\Pre \Rg$} whose collection of objects $\Pre \Rg _0$ is the collection of all preordered rgs, for every preordered rg $X$ and every preordered rg $Y$ the collection of morphisms from $X$ to $Y$, $\Mor _{\Rg}(X,Y)$, is precisely the set of all nondecreasing homomorphisms from $X$ to $Y$, composition is given by ordinary function composition, and the identities of the category are the identity functions.
\begin{rmk}
We similarly have categories of preordered rigs $\Pre \Rig$\index[notation]{$\Pre \Rig$}, preordered rngs $\Pre \Rng$\index[notation]{$\Pre \Rng$}, and preordered rings$\Pre \Ring$\index[notation]{$\Pre \Ring$}.
\end{rmk}
\end{dfn}
This should be pretty much what you expect:  a preordered rg has two different structures on it, namely the rg structure and the preorder structure, and so we require the morphisms in the category of preordered rgs to preserve \emph{both} of these structures.

\begin{prp}
$(\K ,+,\cdot ,0,1,\leq )$ is a well-ordered integral crig.
\begin{proof}
We just showed in the last two sections (\cref{prpA.1.16} and \cref{thm1.1.34}) that $(\K ,+,\cdot ,0,1)$ is an integral crig and that $(\K ,\leq )$ is well-ordered, so all that remains to be checked are properties \ref{enm1.1.38.3} and \ref{enm1.1.38.4} of \cref{dfn1.1.38}.

We first check \ref{enm1.1.38.3} of \cref{dfn1.1.38}, so let $m,n,o\in \K$ and let $M,N,O$ be sets such that $m=\abs{M}$, $n=\abs{N}$, and $o=\abs{O}$.  Suppose that $m\leq n$.  This means that there is an injection $f:M\rightarrow N$.  We would like to show that $m+o\leq n+o$.  In other words, we want to show that there is an injection from $M\sqcup O$ to $N\sqcup O$.  Of course, the function $g:M\sqcup O\rightarrow N\sqcup O$ defined by
\begin{equation}
g(x)\coloneqq \begin{cases}f(x) & \text{if }x\in M \\ x & \text{if }x\in O\end{cases}
\end{equation}
is an injection because $f$ is, and so $m+o\leq n+o$.

Now we show \ref{enm1.1.38.4}.  Suppose that $m\leq n$ and that $0\leq o$.\footnote{Of course we don't actually need to assume that $0\leq o$.  Every natural number is greater than or equal to $0$.}  Let $f:M\rightarrow N$ be the injection the same as before.  We wish to show that $mo\leq no$.  In other words, we wish to construct an injection from $M\times O$ into $N\times O$.  Of course, the function $g:M\times O\rightarrow N\times O$ defined by
\begin{equation}
g\left( \coord{x,y}\right) \coloneqq \coord{f(x),y}
\end{equation}
is an injection because $f$ is, and so $mo\leq no$.

Finally, $0\leq 1$ follows because $\emptyset \subseteq \{ \emptyset \}$ (the empty set is a subset of every set).

Thus, $(\K ,+,0,1,\leq )$ is a well-ordered integral crig.
\end{proof}
\end{prp}
\begin{crl}
$(\N ,+,\cdot ,0,1,\leq )$ is a well-ordered integral crig.
\end{crl}

Before moving on, we summarize all the properties of $\N$ that we have shown.  This is nothing more than explicitly spelling out what it means for $(\N ,+,\cdot ,0,1,\leq )$ to be a well-ordered integral crig.
\begin{enumerate}
\item $+$ is associative,
\item $+$ is commutative,
\item $0$ is an additive identity,
\item $\cdot$ is associative,
\item $\cdot$ is commutative,
\item $1$ is a multiplicative identity,
\item multiplication distributes over addition,
\item $mn=0$ implies either $m=0$ or $n=0$,
\item $\leq$ is reflexive,
\item $\leq$ is transitive,
\item $\leq$ is antisymmetric,
\item $\leq$ is total,
\item every nonempty subset has a smallest element,
\item $m\leq n$ implies $m+o\leq n+o$, and
\item $m\leq n$ and $0\leq o$ implies $mo\leq no$.
\end{enumerate}

Finally, we prove one more property of the natural numbers that we will need when discussing the integers.
\begin{prp}\label{prp1.1.50}
Let $m,n,o\in \N$.  Then, if $m+o\leq n+o$, then $m\leq n$.  In particular, because this is a partial order, if $m+o=n+o$, then $m=n$.
\begin{rmk}
Note that this is \emph{false} in $\K$.  For example, $0+\aleph _0=1+\aleph _0$, but $0\neq 1$.  ($\aleph _0\coloneqq \abs{\N}$.  See \cref{dfn2.2}.)
\end{rmk}
\begin{proof}
Suppose that $m+o\leq n+o$.  Let $M,N,O$ be finite sets such that $m=\abs{M}$, $n=\abs{N}$, and $o=\abs{O}$.  Because $m+o\leq n+o$, there is an injection $\phi :M\sqcup O\rightarrow N\sqcup O$.  Define
\begin{equation}
P\coloneqq \left\{ x\in M:\phi (x)\in O\right\} .
\end{equation}
(``P'' is for ``problematic''.)  We prove the result by induction on the cardinality of $P$.

If $P$ is empty, then we must have that $\phi (M)\subseteq N$, so that $\restr{\phi}{M}:M\rightarrow N$ is an injection, and hence $m\leq n$.
\begin{comment}
If $\phi (M)=N$, then $\restr{\phi}{M}:M\rightarrow N$ is a bijection, and we will be done.   Thus, we wish to show that $N\setminus \phi (M)$ is empty.  Suppose not.  Then, because $\phi$ is a bijection, it must the case that $\phi ^{-1}\left( N\setminus \phi (M)\right) \subseteq O$ is nonempty, and so
\begin{equation}\label{1.1.52}
O\setminus \left( \phi ^{-1}\left( N\setminus \phi (M)\right) \right) 
\end{equation}
is a \emph{proper} subset.  The image of $\phi$ is all of $N\sqcup O$, and so as the image of both $M$ and $\phi ^{-1}\left( N\setminus \phi (M)\right)$ lie in $N$, it must be the case that the image of \eqref{1.1.52} must be all of $O$.  But then, $\phi$ restricted to this proper subset of $O$ is a bijection onto $O$, and so $O$ is infinite:  a contradiction.  Therefore, it must have been the case from the beginning that $\phi (M)=N$.
\end{comment}

Now suppose the result is true if $\abs{P}=k$ for $k\geq 1$.  We show that it must also be true for $\abs{P}=k+1$.  We first show that $N\setminus \phi (M)$ is nonempty.  If it were empty, then we must have that $\phi (O)\subseteq O$, and hence, as $O$ is finite and $\phi$ is an injection, we must have that $\phi (O)=O$ (injections are bijections onto their image).  But then it cannot be the case that $P$ is nonempty:  a contradiction of the fact that $\abs{P}=k\geq 1$.  Thus, it must be the case that $N\setminus \phi (M)$ is nonempty.  Let $n_0\in N\setminus \phi (M)$.  Let $p_0\in P$, so that, by the definition of $P$, $\phi (p_0)\in O$.  Let $\psi :N\sqcup O\rightarrow N\sqcup O$ be the map that exchanges $\phi (p_0)$ and $n_0$ and leaves everything else fixed.  This is a bijection, and so $\psi \circ \phi :M\sqcup O\rightarrow N\sqcup O$ is an injection.  Furthermore, now the image of $p_0$ is contained in $N$ (and there is no new point in $M$ that gets mapped into $O$), so that now there are only $k$ elements of $M$ which map into $O$ via the injection $\psi \circ \phi$.  By the induction hypothesis, there is then a injection from $M$ to $N$, and we are done.
\end{proof}
\end{prp}

\subsubsection{The Peano axioms}

It is not uncommon for textbooks to introduce the natural numbers via the Peano axioms.  We included this material not because it is essential to the development of the real numbers, but rather for the sake of completeness:  even though it is not strictly necessary, people will expect every mathematician to know of the Peano axioms.

People who do use the Peano axioms to introduce the natural numbers, instead of constructing the natural numbers and proving they have the desired properties, will simply assume that a structure which satisfies the Peano axioms exists.  We, however, will instead \emph{prove} the Peano axioms are true; for us, they are theorems.  Before you try to go off and prove them, however, I had probably better tell you what they are.
\begin{thm}[The Peano axioms]\index{Peano axioms}
There exists a set $\N '$ which contains an element $0'\in \N'$ and a function $s:\N '\rightarrow \N '$ (called the \emph{successor function}\index{Successor function}), such that
\begin{enumerate}
\item \label{enm1.1.23.i}$m\in \N '$ is in the image of $S$ iff $m\neq 0'$;
\item \label{enm1.1.23.ii}$s$ is injective, and
\item \label{enm1.1.23.iii} a subset $S\subseteq \N '$ such that (iii.a) $0'\in S$ and (iii.b) $m\in S$ implies $s(m)\in S$ is equal to all of $\N '$.
\end{enumerate}
\begin{rmk}
The prime mark on $\N '$ simply to distinguish the set in this theorem from `our' natural numbers, namely the set that is the collection of cardinalities of finite sets  Similarly for $0'$.
\end{rmk}
\begin{rmk}
The successor of an element is of course `supposed' to be that element plus one.
\end{rmk}
\begin{rmk}
(iii) is of course thought of as \emph{induction}.
\end{rmk}
\begin{proof}
\Step{Define everything}
Define $\N '\coloneqq \N$, $s(m)\coloneqq m+1$, and $0'\coloneqq 0$.

\Step{Prove \ref{enm1.1.23.i}}
\begin{exr}
Prove that there is no $m\in \N$ such that $m+1=0$.
\end{exr}
\begin{exr}
Prove that if $m\neq 0$, then there is some $n\in \N$ such that $s(n)=m$.
\end{exr}

\Step{Prove \ref{enm1.1.23.ii}}
\begin{exr}
Prove that if $m+1=n+1$, then $m=n$.
\end{exr}

\Step{Prove \ref{enm1.1.23.iii}}
Let $S\subseteq \N$ have the properties that (iii.a) $0\in S$ and (iii.b) $m\in S$ implies $m+1\in S$.  We wish to show that $S=\N$.  We proceed by contradiction:  suppose that $S\neq \N$.  Then, $S^{\comp}$ is nonempty, and as $\N$ is well-ordered, $S^{\comp}$ has a least element $m_0\in S^{\comp}$.  As $0\in S$, it cannot be the case that $m_0=0$.  Then, by \ref{enm1.1.23.i}, there is some $n_0\in \N$ such that $s(n_0)=m_0$.  As we have defined $s(n_0)\coloneqq n_0+1$, we in particular have that $n_0<m_0$.  As $n_0$ is less than $m_0$ and $m_0$ is the \emph{least} element of $S^{\comp}$, we cannot have $n_0\in S^{\comp}$.  Therefore, $n_0\in S$.  But then, by hypothesis, $m_0=s(n_0)\in S$:  a contradiction (as $m_0\in S^{\comp}$).  Hence, we must have that $S=\N$.
\end{proof}
\end{thm}

\section{Additive inverses and the integers}

Suppose we have a simple algebraic equation involving three natural numbers, $m+n=o$, we are given $n$ and $o$, and we would like to find $m$.  Of course, we know what the answer \emph{should} be, namely $m=o-n$, but currently this is nonsensical as we have not defined what this crazy new symbol ``$-$'' means.  The job of the integers is to make sense out of this.

We thus would like to find a set with algebraic structure (which will turn out to be an integral cring whose elements are thought of as a new more general type of ``number'') that allows us to solve simple equations like $m+n=o$.  More precisely, we seek a cring (that is a crig with \emph{additive inverses}) which contains $\N$ and, in some sense, is the `simplest' cring that will do so.

To understand what it means to be a cring $Z$ that contains $\N$ is quite easy, but how does one make sense of the statement that $Z$ is the `simplest' cring with this property?  The precise sense in which $Z$ should be the simplest cring which contains $\N$ is, if $Z'$ is any other cring which contains $\N$, then $Z'$ contains $Z$ as well.  That is, $Z$ is contained in every cring which contains $\N$.

\begin{thm}[Integers]\label{Integers}
There exists a totally-ordered integral cring $\Z$\index[notation]{$\Z$}, the \emph{integers}\index{Integers}, such that
\begin{enumerate}
\item \label{enm1.2.1.1}$\Z$ contains $\N$ as a subpreordered rig;\footnote{When we say that $\Z$ contains $\N$ as a subpreordered rig, what we \emph{actually} mean is that there is an injective morphism of preordered-rigs $\N \hookrightarrow \Z$.  In particular, it need not literally be the case that $\N$ is a subset of $\Z$ (just an isomorphic copy of $\N$).} and
\item \label{enm1.2.1.2}if $Z'$ is any other totally-ordered integral cring which satisfies this property, then $Z'$ contains a unique copy of $\Z$ as a subpreordered ring.
\end{enumerate}
Furthermore, $\Z$ is unique up to isomorphism\index{Unique up to isomorphism} of preordered rings in the sense that, if $Z$ is any other totally-ordered integral cring which satisfies \emph{both} of these properties, then $\Z \cong _{\Pre \Ring}Z$.
\begin{rmk}
The order itself doesn't really play a role here.  The significance is in going from a crig to a cring; the order just `comes along for the ride', so to speak.
\end{rmk}
\begin{rmk}
This is actually a special case of a more general construction, the construction of the \emph{Grothendieck Group}\index{Grothendieck Group} of a commutative monoid.  We don't need this general construction and we are already pushing the envelope for level of abstraction in an introductory analysis course, so we don't present it in this level of generality.
\end{rmk}
\begin{rmk}
Integral crings are usually called \emph{integral domains}\index{Integral domain}.  In fact, this term is the origin of our use of the word ``integral''.
\end{rmk}
\begin{proof}
\Step{Define an equivalence relation on $\N \times \N$}
The idea being constructing the integers is to think of a pair of \emph{natural} numbers $\coord{m,n}$ as representing what should be $m-n$.  One issue with this, however, is that, if we do this, then it should be the case that $\coord{m+1,n+1}$ and $\coord{m,n}$ both represent the same number.  Thus, we put an equivalence relation on $\N \times \N$.

Define $\coord{m_1,n_1}\sim \coord{m_2,n_2}$ iff $m_1+n_2=m_2+n_1$.  We came up with this of course because the statement $\coord{m_1,n_1}\sim \coord{m_2,n_2}$ \emph{should} be $m_1-n_1=m_2-n_2$.  Of course, this itself doesn't make sense, and so we write this same thing as something that does make sense given what we have already defined, namely $m_1+n_2=m_2+n_1$.

\Step{Check that $\sim$ is an equivalence relation}
\begin{exr}
Show that $\sim$ is reflexive and symmetric.
\end{exr}
To show that it is transitive, suppose that $\coord{m_1,n_1}\sim \coord{m_2,n_2}$ and $\coord{m_2,n_2}\sim \coord{m_3,n_3}$.  Then, $m_1+n_2=m_2+n_1$ and $m_2+n_3=m_3+n_2$, and so
\begin{equation}
m_1+n_2+m_2+n_3=m_2+n_1+m_3+n_2.
\end{equation}
It follows from \cref{prp1.1.50} that $m_1+n_3=n_1+m_3$, and so $(m_1,n_1)\sim (m_3,n_3)$, so that $\sim$ is transitive.

\Step{Define $\Z$ as a set}
We define
\begin{equation}
\Z \coloneqq \N \times \N /\sim .
\end{equation}
Recall (see \cref{dfnA.1.42}) that this is the quotient set with respect to $\sim$, the set of equivalence classes.

\Step{Define addition and multiplication on $\Z$}
We define
\begin{equation}
[\coord{m_1,n_1}]+[\coord{m_2,n_2}]\coloneqq [\coord{m_1+m_2,n_1+n_2}]
\end{equation}
and
\begin{equation}
[\coord{m_1,n_1}]\cdot [\coord{m_2,n_2}]\coloneqq [\coord{m_1m_2+n_1n_2,m_1n_2+n_1m_2}].
\end{equation}
\begin{exr}
Show that $+$ and $\cdot$ are both well-defined.
\end{exr}

\Step{Define the identities}
We define
\begin{equation}
0\coloneqq [\coord{0,0}]\text{ and }1\coloneqq [\coord{1,0}].
\end{equation}

\Step{Show that $\Z$ is an integral cring}
\begin{exr}
Show that $\Z$ is an integral cring.
\end{exr}

\Step{Define a preorder on $\Z$}
We define
\begin{equation}
[\coord{m_1,n_1}]\leq [\coord{m_2,n_2}]\text{ iff }m_1+n_2\leq m_2+n_1.
\end{equation}
\begin{exr}
Show that $\leq$ is well-defined.
\end{exr}
\begin{exr}
Show that $\leq$ is a preorder.
\end{exr}

\Step{Show that $\leq$ is a total-order}
\begin{exr}
Show that $\leq$ is a total-order.
\end{exr}

\Step{Show that $(\Z ,+,\cdot ,0,1,-,\leq )$ is a totally-ordered integral cring}
\begin{exr}
Show that $(\Z ,+,\cdot ,0,1,-,\leq )$ is a totally-ordered integral cring.
\end{exr}

\Step{Show that $\Z$ contains $\N$ as a preordered-rig}
\begin{exr}
Define a function $\iota :\N \rightarrow \Z$.  Show that it is (i) injective and (ii) a morphism of preordered rigs.
\end{exr}

\Step{Show that every totally-ordered integral cring which contains $\N$ contains a unique copy of $\Z$}
Let $Z'$ be a totally-ordered integral cring which contains $\N$.  As $Z'$ contains additive inverses, it must contain the additive inverse of every element of $\N$.  However, $\N$ together with the additive inverses of every element of $\N$ in $Z'$ is just a copy of $\Z$.  There can only be one such isomorphic copy of $\Z$ because, if there were another copy, they would have to share the same multiplicative identity (by uniqueness of identities, \cref{exrA.1.77}).  Both copies being rings, it would then follow that $2\coloneqq 1+1$, $3\coloneqq 1+1+1$, etc.~would have to be the same element in each, and hence that by uniqueness of inverses (\cref{exrA.1.79}), $-1$, $-2$, $,-3$, etc.~would have to be the same element in each, and hence the two copies would be the same.

\Step{Show that $\Z$ is unique up to isomorphism}
Let $Z$ be some other totally-ordered cring that satisfies the two properties \ref{enm1.2.1.1} and \ref{enm1.2.1.2}.  As $Z$ contains $\N$ and $\Z$ possesses property \ref{enm1.2.1.2}, it follows that $Z$ contains $\Z$.  On the other hand, as $\Z$ contains $\N$ and $Z$ possesses property \ref{enm1.2.1.2}, it likewise follows that $\Z$ contains $Z$.  As $\Z$ contains $Z$ and $Z$ contains $\Z$, and furthermore, because these copies are \emph{unique}, it must be the case that $\Z =Z$.\footnote{We say that $\Z$ is unique \emph{up to isomorphism} because in fact all we can say is that $Z$ contains an \emph{isomorphic copy} of $\Z$.  We abused language and said just ``$\Z$'' instead of ``isomorphic copy of $\Z$''.  This abuse of language is very common in mathematics.  Indeed, you should usually be thinking of two things which are isomorphic as, for all intents and purpose, the same exact thing.}
\end{proof}
\end{thm}

You will notice a common theme through these notes, and indeed, throughout all of mathematics:  if ever we want something that isn't there, throw it in.  For example, we had the natural numbers, but wanted additive inverses, and so we came up with $\Z$ by just `adjoining' the additive inverses of all the elements of $\N$.  Similarly, we want multiplicative inverses, and so by throwing them in, we obtain $\Q$.  Doing the same with limits gives us $\R$, and in turn doing the same with roots of polynomials gives us $\C$.  If you've read the appendix, you may have seen already something vaguely similar (cf.~the paragraph containing \eqref{A.1.1} and what follows).  We wrote down the definition of a set-like thing, but then realized that it cannot be a set (this is Russel's Paradox).  To get around this, instead of working with things that are strictly sets, we work in a larger `universe' that contains proper classes as well:  you might say that we `throw in' the proper classes alongside sets.

We now know that $\Z$ is a totally-ordered integral cring.  Of course, $\Z$ satisfies other properties as well (a lot of which follow from the fact that $\Z$ is a totally-ordered integral cring).\begin{exr}
Let $X$ be a preordered ring and let $x_1,x_2\in X$.  Show the following statements.
\begin{enumerate}
\item $0\leq x_1$ implies $-x_1\leq 0$.
\item $x_1,x_2\leq 0$ implies $0\leq x_1x_2$.
\item If $X$ is totally-ordered, then $0\leq x_1^2$.
\item If $X$ is totally-ordered, then $0\leq 1$.
\item $x_1\leq 1$ and $0\leq x_2$ imply $x_1x_2\leq x_2$.
\end{enumerate}
\end{exr}
\begin{exr}\label{exr1.2.14}
Let $m\in \Z$ and suppose that $0\leq m\leq 1$.  Show that either $m=0$ or $m=1$.
\end{exr}

\section{Multiplicative inverses and the rational numbers}

The motivation of the introduction to the rationals is essentially the same as the motivation for the introduction of the integers, just with multiplication instead of division.  That is, we would like to be able to solve equations of the form $mn=o$ for $m,n,o\in \Z$.  There is one significant difference however.  Unlike with addition, we cannot invert everything:  in particular, we cannot invert $0$.
\begin{exr}
Let $R$ be a ring.  Show that if $0$ is invertible in $R$, then $R=0\coloneqq \{ 0\}$.
\end{exr}
\begin{exm}[Tropical integers]\label{exm1.3.2}
The tropical integers are an example of a nonzero crig in which $0$ is invertible.  Thus, you do indeed need additive inverses for the result of the previous exercise to hold.

Define $R\coloneqq \N$, $m+_Rn\coloneqq \max \{ m,n\}$, $m\cdot _Rn\coloneqq m+n$, and $0_R\coloneqq 0\eqqcolon 1_R$.  Then, $(R,+_R,\cdot _R,0_R,1_R)$ are the \emph{tropical integers}\index{Tropical integers}.  The subscript $R$ is meant to distinguish `tropical' version from the usual version (e.g.~$+_R\coloneqq \max$ vs $+$).
\begin{exr}
Show that indeed $(R,+_R,\cdot _R,0_R,1_R)$ is a crig.  Furthermore, show that $0_R\cdot _R0_R=1_R$, so that indeed $0_R$ is invertible with multiplicative inverse $0_R^{-1}=0_R$.
\end{exr}
\end{exm}
Besides $0$, however, we wind up being able to invert everything we would like.
\begin{thm}[Rational numbers]\label{RationalNumbers}
There exists a totally-ordered field $\Q$, the \emph{rational numbers}\index{Rational numbers}, such that
\begin{enumerate}
\item \label{enm1.3.2.i}$\Q$ contains $\Z$ as a subpreordered ring; and
\item \label{enm1.3.2.ii}if $Q'$ is any other totally-ordered field which satisfies this property, then $Q'$ contains a unique copy of $\Q$ as a subpreordered field.
\end{enumerate}
Furthermore, $\Q$ is unique up to isomorphism of preordered-fields.
\begin{rmk}
This is also a special case of a more general construction known as the construction of \emph{fraction fields}\index{Fraction field}.  For the same reason as with the Grothendieck group construction, we do not present this in its full generality.
\end{rmk}
\begin{proof}
We leave this as an exercise.
\begin{exr}
Try to do the entire proof yourself, using the proof of \cref{Integers} as guidance.
\end{exr}
\end{proof}
\end{thm}
\begin{prp}\label{prp1.3.4}
For all $x\in \Q$, there exist unique $m\in \Z$ and $n\in \Z^+$ such that (i) $\gcd (m,n)=1$ and (ii) $x=\frac{m}{n}$.
\begin{proof}
By \ref{enm1.3.2.i} of the previous theorem, $\Q$ contains $\Z$.  As $\Q$ is a field, it thus must contain multiplicative inverses of every nonzero integer.  For $m\in \Z$ not zero, denote its multiplicative inverse in $\Q$ by $\frac{1}{m}$.  For $n\in \Z$, denote $\frac{n}{m}\coloneqq n\cdot \frac{1}{m}$, and define
\begin{equation}
Q\coloneqq \left\{ \tfrac{n}{m}\in \Q :m,n\in \Z ,m\neq 0\right\} .
\end{equation}
Note that $Q$ is a field which contains $\Z$ has a subpreordered ring.  By \ref{enm1.3.2.ii} of the previous theorem, we thus have that $\Q \subseteq Q$.  Of course we already knew that $Q\subseteq \Q$, and so we have that $Q=\Q$.

Now for $x\in \Q$ arbitrary, we can write $x=\frac{n}{m}$ for some $m,n\in \Z$ with $m\neq 0$.  If $m<0$, then we can write $x=\frac{-n}{-m}$, so that now the denominator is positive.  Define $d\coloneqq \gcd (m,n)$ and write $m=m'd$ and $n=n'd$, so that $x=\frac{n'}{m'}$.
\begin{exr}
Show that $\gcd (m',n')=1$.
\end{exr}
This shows existence.

We now prove uniqueness.  Suppose that $\frac{n_1}{m_1}=\frac{n_2}{m_2}$ for $m_1,n_1,m_2,n_2\in \Z$ with $m_1,m_2>0$, $\gcd (m_1,n_1)=1=\gcd (m_2,n_2)$.  Re-arranging, we have $m_2n_1=m_1n_2$, so that $m_2\mid m_1n_2$.  As $\gcd (m_2,n_2)=1$, it follows that $m_2\mid m_1$.  On the other hand, this same equation implies that $m_1\mid m_2n_1$, and therefore $m_1\mid m_2$.  That $m_1\mid m_2$ an $m_2\mid m_1$ implies that $m_1=\pm m_2$.  However, as they are both positive, we have that $m_1=m_2$.  It then follows that $n_1=n_2$ from the equation $m_2n_1=m_1n_2$.
\end{proof}
\end{prp}
\begin{exr}
Let $X$ be a totally-ordered field and let $x_1,x_2\in X$ be nonzero.  Show that the following statements are true.
\begin{enumerate}
\item $x_1^{-1}$ has the same sign as $x_1$.
\item $0<x_1\leq x_2$ implies $0<x_2^{-1}\leq x_1^{-1}$.
\end{enumerate}
\end{exr}

As a matter of fact, $\Q$ is not just he smallest totally-ordered field that contains $\Z$, is is the smallest totally-ordered field \emph{period}.
\begin{exr}
Let $F$ be a totally-ordered field.  Show that $\Char (F)=0$.
\end{exr}
\begin{prp}\label{prp1.4.52}
Let $F$ be a totally-ordered field.  Then, there exists a subfield $Q\subseteq F$ and an isomorphism of totally-ordered fields from $\Q$ to $Q$.
\begin{rmk}
It is typical to identify $\Q$ with $Q$, so that $\Q$ can actually be regarded as a subfield of $F$.
\end{rmk}
\begin{rmk}
In fact, we could define the rationals as follows.
\begin{textequation}
There exists a totally-ordered field $\Q$, the \emph{rational numbers}, such that, if $Q$ is another totally-ordered field, then $\Q \subseteq Q$.  Furthermore, $\Q$ is unique up to isomorphism of preordered fields.
\end{textequation}
In fact, this is arguably preferable to \cref{RationalNumbers}, but we as we actually make use of that result in the following proof, to do this, we would either have to rephrase things entirely or `redefine' $\Q$.
\end{rmk}
\begin{proof}
As $\Char (F)=0$, $F$ contains a copy of $\N$, namely, $\left\{ 0,1,2,3,\ldots \right\}$.\footnote{We needed that $\Char (F)=0$ in order that this be a copy of $\N$.  For example, think of the integers modulo $m$ (\cref{exmA.1.117}; see also the sequence of examples starting with \cref{exmA.1.53}).  For example, in the case $m=3$, we have that $1+1+1=0$, and so this set $\left\{ 0,1,2\coloneqq 1+1,3\coloneqq 1+1+1,\ldots \right\}$ would just be $\{ 0,1,2\}$, certainly not a copy of the natural numbers.}  Recall that $\Z$ is the smallest totally-ordered integral cring which contains $\N$.  Thus, as $F$ is in particular now a totally-ordered integral cring which contains $\N$, we have that in fact $F$ contains $\Z$:  $\Z \subseteq \R$.  Similarly, as $\Q$ was the smallest totally-ordered field that contained $\Z$, and we have just showed that $F$ is a totally-ordered field that contains $\Z$, it follows that $\Q \subseteq F$.
\end{proof}
\end{prp}

\section{Least upper-bounds and the real numbers}

Finally we come to the first material that might be considered `the point' of this course.

Just as the integers corrected the `deficiency' of the natural numbers that was the lack of additive inverses, and the rationals corrected the `deficiency' of the integers that was the lack of multiplicative inverses, the real numbers will correct a `deficiency' of the rational numbers.  But what is this ``deficiency''?  The answer turns out to be that this `deficiency' is a lack of what are called least upper-bounds.

\subsection{Least upper-bounds and why you should care}

\begin{dfn}[Suprema]
Let $(X,\leq )$ be a preordered set, let $S\subseteq X$, and let $x\in X$.  Then, $x$ is a \emph{supremum}\index{Supremum} of $S$ iff
\begin{enumerate}
\item $x$ is an upper-bound of $S$, and
\item if $x'$ is any other upper-bound of $S$, then $x\leq x'$.
\end{enumerate}
\begin{rmk}
Supremum is synonymous with \emph{least upper-bounds} (because they are an upper-bound that is less than or equal to every other upper-bound).
\end{rmk}
\begin{rmk}
If $S$ has a supremum $x$, then we write $x\coloneqq \sup (S)$.  This is justified by the following exercise.
\end{rmk}
\begin{exr}\label{exr1.4.4}
Let $(X,\leq )$ be a partially-ordered set, let $S\subseteq X$, and let $x_1,x_2\in S$ be two suprema of $S$.  Show that $x_1=x_2$.
\end{exr}
We extend the definition of supremum as follows.
\begin{equation}
\sup (S)\coloneqq \begin{cases}\infty & \text{if }S\text{ is not bounded above} \\ -\infty & \text{if }S=\emptyset \end{cases}.
\end{equation}
\end{dfn}
\begin{exr}
Find an example of a preordered set $(X,\leq )$ an a subset $S\subseteq X$ with two distinct suprema.
\begin{rmk}
It is because of counter-examples like these that we shall primarily concern ourselves with partially-ordered sets in this section.
\end{rmk}
\end{exr}
We similarly have a notion of infimum, which is just the same concept with inequalities reversed.
\begin{dfn}[Infimum]
Let $(X,\leq )$ be a preordered set, let $S\subseteq X$, and let $s\in X$.  Then, $x$ is a \emph{infimum}\index{Infimum} of $S$ iff
\begin{enumerate}
\item $x$ is a lower-bound of $S$, and
\item if $x'$ is any other lower-bound of $S$, then $x'\leq x$.
\end{enumerate}
\begin{rmk}
A you might have guessed, infima are also called \emph{greatest lower-bounds}\index{Greatest lower-bounds}.
\end{rmk}
\begin{rmk}
Similarly, if $S$ has an infimum $x$, then we write $x\coloneqq \inf (S)$.
\end{rmk}
We extend the definition of infimum as follows.
\begin{equation}
\inf (S)\coloneqq \begin{cases}-\infty & \text{if }S\text{ is not bounded below} \\ \infty & \text{if }S=\emptyset \end{cases}.
\end{equation}
\end{dfn}

So that's what suprema (and infima) are, but why should you care?  At least one reason is the following.  Consider the set
\begin{equation}
S\coloneqq \left\{ \left( 1+\tfrac{1}{n}\right) ^n:n\in \Z ^+\right\} \subseteq \Q .
\end{equation}
This set has two key properties (i) it is bounded above, and (ii) for every $x\in S$, there is some $x'\in S$ with $x<x'$.  Draw yourself a picture of what this must look like.  Though we don't know what a limit is yet, from the picture we see that pretty much any reasonable definition of a limit should have the property that
\begin{equation}\label{1.4.7}
\lim _n\left[ 1+\tfrac{1}{n}\right] ^n=\sup (S).
\end{equation}
Though we don't even have the definition to make sense of it yet, you'll recall that the answer to the left-hand side \emph{should} be $\e \coloneqq \exp (1)$, which of course is not rational.\footnote{Of course, we haven't proven that $\e$ is not rational, but right now, as we are only concerned with motivation, simply knowing that it is not rational is enough to justify the desire to have suprema.}  Thus, despite the fact that $S\subseteq \Q$, $\sup (S)\notin \Q$, and in fact, for us, $\sup (S)$ just doesn't make sense (yet).  Ultimately, because we want to do calculus, we want to be able to take limits.  In particular, because of things like \eqref{1.4.7}, we had better be able to take suprema as well.  It turns out that throwing in all least upper-bounds gives us all the limits we were missing.  This is really the motivation for demanding the existence of least upper-bounds:  we want to be able to take limits.

Thus, the desired property which $\Q$ lacks is the following.
\begin{dfn}[Least upper-bound property]
A preordered set has the \emph{least upper-bound property}\index{Least upper-bound property} iff every nonempty subset that is bounded above has a least upper-bound.
\end{dfn}
Of course, we have the inequality-reversed notion as well.
\begin{dfn}[Greatest lower-bound property]
A preordered set has the \emph{greatest lower-bound property}\index{Greatest lower-bound property} iff every nonempty subset that is bounded below has a greatest lower-bound.
\end{dfn}
It turns out that it doesn't actually matter which property we require, but before we prove this, we require an important `lemma'.\footnote{``Lemma'' is in quotes because, while it is used to prove that a totally-ordered set has the least upper-bound property iff it has the greatest lower-bound property, it is useful in its own right.}
\begin{prp}\label{prp1.4.11}
Let $X$ be a totally-ordered set, let $S\subseteq X$ nonempty and bounded above, and let $x\in X$ be an upper-bound for $S$.  Then, $x=\sup (S)$ iff for every $x'\in X$ with $x'<x$, there is some $x''\in S$ with $x'<x''\leq x$.
\begin{rmk}
Note the hypothesis of \emph{totally}-ordered.
\end{rmk}
\begin{rmk}
You should think of this as saying that, in particular, $S$ contains elements `arbitrarily close' to its supremum.
\end{rmk}
\begin{proof}
$(\Rightarrow )$ Suppose that $x=\sup (S)$.  Let $x'\in X$ be such that $x'<x$.  We proceed by contradiction:  suppose that there is no $x''\in S$ such that $x'<x''\leq x$.  $x$ being an upper-bound of $S$, every $x''\in S$ is automatically less than or equal to $S$, so really this is just the same as saying that there is no $x''\in S$ with $x'<x''$.  \emph{By totality}, it thus follows that we must have $x''\leq x'$ for all $x''\in S$, in which case $x'$ is an upper-bound for $S$.  But $x'<x$, which contradicts the fact that $x$ is the \emph{least} upper-bound for $x$.  Thus, there must be some $x''\in S$ such that $x<x''\leq x$.

\blankline
\noindent
$(\Leftarrow )$ Suppose that for every $x'\in S$ with $x'<x$, there is some $x''\in S$ with $x<x''\leq x$.  Let $x'\in X$ be any other upper-bound for $S$.  We would like to show that $x\leq x'$:  we proceed by contradiction:  suppose that it is not the case that $x\leq x'$.  By totality, this is equivalent to $x'<x$.  Then, by hypothesis, there must be some $x''\in S$ such that $x'<x''\leq x$, which contradicts the fact that $x'$ is an upper-bound of $S$.  Thus, it must be the case that $x=\sup (S)$.
\end{proof}
\end{prp}
\begin{exr}
Write down and prove the analogous version of the previous proposition for the infimum.
\end{exr}
\begin{prp}\label{prp1.4.12}
Let $X$ be a totally-ordered set with the least upper-bound property and let $S\subseteq X$ be nonempty and bounded below.  Then, $T\coloneqq \left\{ x\in X:x\leq x'\text{ for all }x'\in S\right\}$ is nonempty and bounded above, and $\sup (T)=\inf (S)$.  In particular, $X$ has the greatest lower-bound property.
\begin{rmk}
Of course, switching inequalities around gives us the converse as well.
\end{rmk}
\begin{proof}
As $S$ is nonempty, there is some $x_0\in S$.  From the definition of $T$, it follows that $x_0$ is an upper-bound of $T$, so that $T$ is bounded above.  $T$ is the set of all lower-bounds of $S$, and so as $S$ is bounded below, $T$ is nonempty.  Thus, because $X$ has the least upper-bound property, $T$ has a supremum $t\coloneqq \sup (T)$.

We wish to show that $t$ is the infimum of $S$.  We must show two things:  (i) that it is a lower bound of $S$ and (ii) that it is at least as large as every other upper-bound.  To show the first, let $x\in S$.  We wish to show that $t\leq x$.  We proceed by contradiction:  suppose that $x<t$.  Then, by the previous proposition, because $t=\sup (T)$, there must be some $x'\in T$ such that $x<x'\leq t$.  But then it is not the case that $x'$ is less than or equal to every element of $S$:  a contradiction of the definition of $T$.  Thus, it must be the case that $t\leq x$, so that $t$ is a lower-bound of $S$.  Now let $t'\in X$ be some other lower-bound of $S$.  We would like to show that $t'\leq t$.  We proceed by contradiction:  suppose that $t<t'$.  Recall that $T$ itself is just the set of lower-bounds of $S$, so that both $t,t'\in T$.  But if this is true and also $t<t'$, then $t'$ is not a lower-bound of $T$:  a contradiction.  Thus, we must have that $t'\leq t$, so that $\sup (T)=t=\inf (S)$.
\end{proof}
\end{prp}
\begin{rmk}
Sometimes preordered sets which possess \emph{both} of these properties\footnote{Of course, in general the properties are not equivalent for partially-ordered sets which are not totally-ordered.} are called ``\emph{(dedekind) complete}''\index{Complete (preordered set)}\index{Dedekind complete}.  The term \emph{dedekind} complete is to contrast with the term ``\emph{cauchy} complete'', which we will meet in the \cref{chp5} \nameref{chp5}---see \cref{Completeness}.
\end{rmk}

\begin{rmk}
It is not uncommon to see others using facts like ``$\sqrt{2}$ is not rational.'' as motivation for the introduction of the real numbers.  This is stupid.  If all we really cared about were numbers like $\sqrt{2}$, then we shouldn't be going from $\Q$ to $\R$, but rather from $\Q$ to $\A$, the \emph{algebraic numbers} (see \cref{dfn2.13}).  The point of $\R$ is not to be able to take square-roots; the point is to be able to take limits.
\end{rmk}

\subsection{Dedekind cuts and the real numbers}\label{sbs1.4.2}

Everybody reading these notes probably already has some intuition about the real numbers, most likely gained from some sort of calculus course.  Let us suppose for a moment that we know what the real numbers are and that they make sense.  Given a real number $x_0\in \R$, how would you encode $x_0$ using only $\Q$?  The trick we use is to look at the set
\begin{equation}\label{1.4.14}
D_{x_0}\coloneqq \left\{ x\in \Q :x\leq x_0\right\} .
\end{equation}
This subset of $\Q$ uniquely determines $x_0$ because $\sup (D_{x_0})=x_0$.  The idea then is to use sets of the form \eqref{1.4.14} to define the real numbers.  The only thing we have to do for this to make sense in $\Q$ alone is to get rid of the reference to $x_0$.  We do that as follows.
\begin{dfn}[Dedekind cut]
Let $(X,\leq )$ be a preordered set and let $S\subseteq X$.  Then, $S$ is a \emph{Dedekind cut}\index{Dedekind cut} in $X$ iff
\begin{enumerate}
\item $S\neq \emptyset$, 
\item $S\neq X$, and
\item the set of all lower-bounds of the set of all upper-bounds of $S$ is equal to $S$ itself.
\end{enumerate}
\begin{rmk}
The word \emph{cut} is used because, for example, $D_{x_0}$ of \eqref{1.4.14} is sort of thought as `cutting' $\Q$ at the point $x_0$.
\end{rmk}
\end{dfn}
You'll note that $D_{x_0}$ of \eqref{1.4.14} is a Dedekind cut.  Indeed,
\begin{prp}
Let $(X,\leq )$ be a complete totally-ordered set and let $S\subseteq X$ be a Dedekind cut.  Then,
\begin{equation}
S=\left\{ x\in X:x\leq \sup (S)\right\} .
\end{equation}
\begin{proof}
Let us write $S'\coloneqq \left\{ x\in X:x\leq \sup (S)\right\}$.  As $\sup (S)$ is in particular an upper-bound of $S$, we immediately have the inclusion $S\subseteq S'$.  On the other hand, suppose that $x\leq \sup (S)$.  We wish to show that $x\in S$.  As $S$ is a Dedekind cut, this is the same as showing that $x$ is less than or equal to every upper-bound of $S$, so, let $u\in X$ be an upper-bound of $S$.  We now wish to show that $x\leq u$.  We proceed by contradiction:  suppose that $u<x$ (this uses totality).  However, this of course contradicts the fact that $u$ is an upper-bound for $S$.  Thus, we must have that $x\leq u$, so that $S'\subseteq S$, so that $S=S'$.
\end{proof}
\end{prp}

We will construct the real numbers as the set of all Dedekind cuts in $\Q$.\footnote{There is another construction of the reals that is commonly taught, namely, the ``cauchy sequence construction'' in which a real number is defined to be an equivalence class of cauchy sequences.  While this works, I find this more appropriate if one is thinking of the real numbers as a uniform space (in this case, a metric space), whereas we are currently thinking of everything as algebraic structures \emph{with order}.  Because of this, I find it more natural to complete the underlying partially-ordered set instead of the underlying uniform space (for one thing, we haven't actually put a uniform structure on $\Q$ yet).}
\begin{thm}[Real numbers]\label{RealNumbers}
There exists a nonzero complete totally-ordered field $\R$, the \emph{real numbers}\index{Real numbers}, which is unique up to isomorphism of preordered fields.
\begin{rmk}
Compare this with the analogous theorems for $\Z$ and $\Q$, \cref{Integers,RationalNumbers} respectively.  You'll note that, for uniqueness, we need not require that $\R$ be the `simplest' complete totally-ordered field which contains $\Q$, or even that $\R$ contain $\Q$ at all.  These will follow automatically from the fact that $\R$ is a complete totally-ordered field alone.  (In fact, we get that $\R$ obtains $\Q$ for free via \cref{prp1.4.52}.)
\end{rmk}
\begin{rmk}
As you might have guessed, this is also a special case of a more general construction, which takes partially-ordered sets to \emph{complete} partially-ordered sets, known as the \emph{Dedekind-MacNeille completion}\index{Dedekind-MacNeille completion}.  Be careful, however:  in general the arithmetic operations do not extend to the Dedekind-MacNeille completion; see \cref{exm3.2.13}.
\end{rmk}
\begin{proof}
\Step{Define $\R$ as a set}
Define
\begin{equation}
\R \coloneqq \left\{ D\in 2^{\Q}:D\text{ is a Dedekind cut}\right\} .
\end{equation}

\Step{Define a preorder on $\R$}
We define
\begin{equation}
D\leq E\text{ iff }D\subseteq E.
\end{equation}

\Step{Show that $\leq$ is a total-order}
$\leq$ is automatically a partial-order because set-inclusion is always a partial-order.  To show totality, let $D,E\in \R$.  If $D\leq E$, we are done, so suppose this is not the case.  We would like to show that $E\leq D$, i.e., that $E\subseteq D$, so let $e\in E$.  As $D$ is a Dedekind cut, it suffices to show that $e$ is a lower-bound of every upper-bound of $D$, so let $u\in \Q$ be an upper-bound of $D$.  We wish to show that $e\leq u$.  We proceed by contradiction:  suppose that $u<e$.  Now, $D$ is not a subset of $E$ (by hypothesis), there must be some $d\in D$ with $d\notin E$.  If we can show that $e\leq d$, then we will have $u<d$ (because $u<e$), a contradiction.  To show this itself (that $e\leq d$), we proceed by contradiction:  suppose that $d<e$.  Then, in particular, $d$ is less than every upper-bound of $E$, and so, as $E$ is a cut, we have $d\in E$:  a contradiction (recall that we have taken $d\notin E$).  Thus, we must that $e\leq d$, which completes the proof of totality.

\Step{Show that $\leq$ is complete}
As $\leq$ is a total-order, it suffices simply to show that $\leq$ has the least upper-bound property (by \cref{prp1.4.12}).  To show this, let $\mathcal{S}\subseteq \R$ be nonempty and bounded above.  Define
\begin{equation}
S\coloneqq \bigcup _{D\in \mathcal{S}}D.
\end{equation}
\begin{exr}
Check that $S$ is in fact a Dedekind cut.
\end{exr}
We wish to show that $S=\sup \left( \mathcal{S}\right)$.  As $S$ is a superset of every element of $\mathcal{S}$, we certainly have that $S$ is an upper-bound for $\mathcal{S}$.  To show that it is a least upper-bound, let $S'$ be some other upper-bound of $\mathcal{S}$.  We wish to show that $S\leq S'$.  We proceed by contradiction:  suppose that $S\not \leq S'$.  Then, there is some $x\in S$ with $x\notin S'$.  As $x\in S$, there must be some $D\in \mathcal{S}$ with $x\in D$ (by the definition of $S$).  However, as $S'$ is an upper-bound of $\mathcal{S}$, we have that $D\subseteq S'$, which implies that $x\in S'$:  a contradiction.  Thus, we must have that $S\leq S'$, so that $S=\sup \left( \mathcal{S}\right)$.

\Step{Define addition}
Addition of elements of $\R$ is just set addition:\footnote{If ever you're wondering how to come up with these definitions, just think of what the answer should be for sets of the form \eqref{1.4.14}.}
\begin{equation}
D+E\coloneqq \left\{ d+e:d\in D,e\in E\right\} .
\end{equation}
\begin{exr}
Check that $D+E$ is in fact a Dedekind cut.
\end{exr}

\Step{Show that $(\R ,+,0,-)$ is a commutative group}\label{stp1.4.18.6}
Associativity and commutativity follow from the fact that set addition is associative and commutative.  The cut
\begin{equation}
0\coloneqq \left\{ x\in \Q :x\leq 0\right\}
\end{equation}
functions as an additive identity.\footnote{Of course this is abuse of notation.  It should not cause any confusion as one is a subset of $\Q$ and the other is an element of $\Q$.}
\begin{exr}
Check that $0$ is an additive identity.
\end{exr}
We define the additive inverse
\begin{equation}
-D\coloneqq \left\{ x-y:x\leq 0\text{ and }y\notin D\right\} .
\end{equation}
Note that we have
\begin{equation}
D+(-D)=\left\{ d+(x-y):d\in D,x\leq 0,y\notin D\right\} .
\end{equation}
As $y\notin D$, we have that $d<y$, and so $d-y<0$, and so $d+(x-y)<0$.  In the other direction, let $x\leq 0$.  Then,
\begin{equation}
x=d+\left( \left( x+(y-d)\right) -y\right) ,
\end{equation}
and so it suffices to show that we can choose $d\in D$ and $y\in D^{\comp}$ so that $x+(y-d)<0$.  To do this, let $\varepsilon >0$.  We wish to show that there is some $d\in D$ so that $d+\varepsilon \notin D$.  We proceed by contradiction:  suppose that $d+\varepsilon \in D$ for all $d\in D$.  Then, for $d_0\in D$ fixed, we have $d_0+\varepsilon \in D$, and so $d_0+2\varepsilon \in D$, and so $d_0+3\varepsilon \in D$, etc..  As $D$ is bounded (by any element in $D^{\comp}$), this is a contradiction.  Thus, there is some $d\in D$ such that $d+\varepsilon \notin D$.  Then, taking $\varepsilon =-x$, we have that $d-x\notin D$, and so we may take $y\coloneqq d-x$, so that $x+(y-d)=0\leq 0$ as desired.

\Step{Show that $(\R ,+,0,\leq )$ is a totally-ordered commutative group}
\begin{exr}
Check that $D\leq E$ implies $D+F\leq E+F$.
\end{exr}

\Step{Define multiplication}
Multiplication is more complicated.  For example, the product $0\cdot 0$ \emph{should} be $0\cdot 0=0$; however, the set product, $00\coloneqq \left\{ de:d\in 0,e\in 0\right\}$, isn't even bounded above.  We have to break-down the definition into cases.  To simplify things, let us temporarily use the notation
\begin{equation}
D_0^+\coloneqq \left\{ d\in D:0\leq d\right\} .
\end{equation}
Here, $\cdot$ will denote multiplication in $\R$ and juxtaposition will denote set multiplication.  We define
\begin{equation}\label{1.4.31}
D\cdot E\coloneqq \begin{cases}D_0^+E_0^+\cup 0 & \text{if }0\leq D,E  \\ -\left( (-D)\cdot E\right) & \text{if }D\leq 0,0\leq E \\ -\left( D\cdot (-E)\right) & \text{if }0\leq D,E\leq 0 \\ (-D)\cdot (-E) & \text{if }D,E\leq 0\end{cases}
\end{equation}

From the definition \eqref{1.4.31}, it suffices to show associativity and commutativity for the case $0\leq D,E$.  Then,
\begin{equation}
D\cdot (E\cdot F)=D\cdot \left( E_0^+F_0^+\cup 0\right) =D_0^+E_0^+F_0^+\cup 0=(D\cdot E)\cdot F,
\end{equation}
and similarly for commutativity.  The cut
\begin{equation}
1\coloneqq \left\{ x\in \Q :x\leq 1\right\}
\end{equation}
functions as a multiplicative identity.
\begin{exr}
Check that $1$ is a multiplicative identity.
\end{exr}

\Step{Show that the additive inverse distributes}\label{stp1.4.18.9}
We wish to show that
\begin{equation}\label{1.4.37}
-(D+E)=(-D)+(-E).
\end{equation}
On one hand we have
\begin{equation}
-(D+E)=\left\{ x-y:x<0\text{ and }y\notin D+E\right\} .
\end{equation}
On the other hand,
\begin{equation}
\begin{split}
(-D)+(-E) & =\left\{ a+b:a\in -D,b\in -E\right\} \\
& =\left\{ (x_1-y_1)+(x_2-y_2):x_1,x_2<0;y_1\notin D;y_2\notin E\right\} \\
& =\left\{ (x_1+x_2)-(y_1+y_2):x_1,x_2<0;y_1\notin D;y_2\notin E\right\} \\
& =\left\{ x-(y_1+y_2):x<0,y_1\notin D,y_2\notin E\right\} .
\end{split}
\end{equation}
Comparing this with \eqref{1.4.37} above, we see that it suffices to show that $y\notin D+E$ iff $y=y_1+y_2$ for $y_1\notin D$ and $y_2\notin E$.  To show this, let $y_1\in D^{\comp},y_2\in E^{\comp}$ and suppose that $y_1+y_2\in D+E$, so that $y_1+y_2=d+e$ for $d\in D,e\in E$.  Then, $y_2=e+(d-y_1)<e$, which implies that $y_2\in E$:  a contradiction.  Conversely, let $y\notin D+E$.  Let $\varepsilon >0$ and choose $d\in D$ such that $d+\varepsilon \notin D$.  Let $M\geq 2\varepsilon$ be such that $y-M\in D+E$ but $y-(M-\varepsilon )\notin D+E$.  Write $y-M=d'+e'$ for $d'\in D,e'\in E$.  Without loss of generality, assume that $d'\leq d$.  Then,
\begin{equation}
y-M=d'+e'=d+\left( e'-(d-d')\right) .
\end{equation}
As $d-d'\geq 0$, $e\coloneqq e'-(d-d')\in E$.  Then,
\begin{equation}
y-(M-\varepsilon )=d+(e+\varepsilon ).
\end{equation}
Therefore, as $y-(M-\varepsilon )\notin D+E$, it must be the case that $e+\varepsilon \notin E$.  Then,
\begin{equation}
y=\left( d+(M-\varepsilon )\right) +(e+\varepsilon ).
\end{equation}
As $d+(M-\varepsilon )\geq d+\varepsilon \notin D$, it follows that $d+(M-\varepsilon )\notin D$, so that indeed $y=y_1+y_2$ for $y_1\notin D$ and $y_2\notin E$.

\Step{Show that $[E+F]_0^+=E_0^++F_0^+$ for $0\leq E,F$}
If $e\in E,e\geq 0$ and $f\in F,f\geq 0$, then of course $e+f\in E+F,e+f\geq 0$.  Conversely, let $x\in E+F,x\geq 0$ and write $x=e+f$ for $e\in E,f\in F$.  As $x\geq 0$, we must have that either $e\geq 0$ or $f\geq 0$  Without loss of generality, assume the former.  If $f\geq 0$, we are done, so instead suppose that $f<0$.  Then, $x=(e+f)+0$.  As $e+f<e$, $e+f\in E$, and of course, as $x\geq 0$, $e+f\geq 0$, so that indeed $x\in E_0^++\{ 0\} \subseteq E_0^++F_0^+$.

\Step{Show that $(\R ,+,0,-,\cdot ,1)$ is a cring}
All that remains to be shown is distributivity.  Let $D,E,F\in \R$ and consider
\begin{equation}
D\cdot (E+F).
\end{equation}
Let us first do the case with $0\leq D,E,F$.  Then, by the previous step,
\begin{equation}
\begin{split}
D\cdot (E+F) & =\left( D_0^+[E+F]_0^+\right) \cup 0=\left( D_0^+(E_0^++F_0^+)\right) \cup 0=\left( D_0^+E_0^++D_0^+F_0^+\right) \cup 0 \\
& =D_0^+E_0^+\cup 0+D_0^+F_0^+\cup 0=D\cdot E+D\cdot F.
\end{split}
\end{equation}
Because additive inverses distribute and the definition of multiplication, we may without loss of generality assume that $0\leq D,E+F$.  Thus, either $0\leq E$ or $0\leq F$.  Without loss of generality assume the former.  We have already done the case $0\leq F$, so let us instead assume that $F<0$.  Then,
\begin{equation}
\begin{split}
D\cdot E+D\cdot F & =D\cdot \left( (E+F)+(-F)\right) +D\cdot F=D\cdot (E+F)+D\cdot F+D\cdot (-F) \\
& =D\cdot (E+F)+D\cdot F+\left( -(D\cdot F)\right) =D\cdot (E+F),
\end{split}
\end{equation}
where we have used distributivity of additive inverse in $D\cdot (-F)=D\cdot (-F+0)=-D\cdot F$.

\Step{Show that $(\R ,+,0,-,\cdot ,1)$ is a field}
All that remains to be shown is the existence of multiplicative inverses.  For $D\in \R$ not $0$, we define
\begin{equation}
D^{-1}\coloneqq \begin{cases}\left\{ y^{-1}:y\in D^{\comp}\right\} & \text{if }D>0 \\ -\left( (-D)^{-1}\right) & \text{if }D<0\end{cases}.
\end{equation}
To finish this step, it suffices to prove that $D\cdot D^{-1}=1$ for $D>0$.
\begin{equation}
D\cdot D^{-1}=D_0^+[D^{-1}]_0^+\cup 0.
\end{equation}
We would like to show that this is equal to $1\coloneqq \{ x\leq 1\}$.  Let $\varepsilon >0$ and choose $d\in D,d>0$ such that $d+\varepsilon \notin D$.  Thus, $D$ is bounded above by $d+\varepsilon$ and $D^{\comp}$ is bounded below by $d$.  It follows that $D^{-1}$ is bounded above by $d^{-1}$, so that $D_0^+[D^{-1}]_0^+$ is bounded above by $(d+\varepsilon )d^{-1}=1+\varepsilon d^{-1}$.  As $\varepsilon$ is arbitrary (and $d$ gets smaller as $\varepsilon$ gets smaller), it follows that in fact $D_0^+[D^{-1}]_0^+$ is bounded above by $1$, which shows that $D\cdot D^{-1}\subseteq 1$.  For the other inclusion, let $x\leq 1$.  Let $\varepsilon >0$ and choose $y\notin D$ so that $y-\varepsilon \in D$.  As $x<1$, $d\coloneqq x(y-\varepsilon )\in D$.  Thus,
\begin{equation}
D\cdot D^{-1}\ni \left( x(y-\varepsilon )\right) y^{-1}=x-\varepsilon xy^{-1}.
\end{equation}
As this is true for all $\varepsilon >0$, we must have that $x\in D\cdot D^{-1}$, which completes the proof of this step.

\Step{Show that $(\R ,+,0,-,\cdot ,1)$ is a complete totally-ordered field}
All that remains to be shown is that $0\leq D,E$ implies $0\leq D\cdot E$.  Of course, if $d\in D,d\geq 0$ and $e\in E,e\geq 0$, then $de\in D\cdot E$, so that $0\leq D\cdot E$.

\Step{Show that $(\R ,+,0,-,\cdot ,1)$ is unique up to isomorphism of preordered fields}
Let $R$ be another nonzero complete totally-ordered field.  By \cref{prp1.4.52}, $R$ contains a copy of $\Q \subseteq R$ (as well as $\R$ itself, $\Q \subseteq \R$.).  Let $x\in X$.  Thus, because both $R$ and $\R$ contain $\Q$, abusing notation, we may consider
\begin{equation}
D_x:=\left\{ q\in \Q :q\leq x\right\}
\end{equation}
both as a subset of $R$ and $\R$.\footnote{If you want to be pedantic about things, there is a subfield $Q\subseteq R$ together with an isomorphism of totally-ordered fields $\psi :Q\rightarrow \Q$, where $\Q \subseteq \R$.}  With this abuse, we define $\phi :R\rightarrow \R$ by
\begin{equation}
\phi (x):=\sup (D_x),
\end{equation}
where on the right-hande side, $D_x$ is regarded as a subset of $\R$ and the supremum is taken in $\R$.
\begin{exr}
Show that $\phi$ is an isomorphism of preordered fields.
\end{exr}
\end{proof}
\end{thm}
\begin{dfn}[Irrational numbers]
An element $x\in \R$ is \emph{irrational}\index{Irrational numbers} iff $x\notin \Q$.
\begin{rmk}
In fact, it will be a little while before we can show that $\Q ^{\comp}$ is even nonempty.  For example, it is easy to show that there is no rational number whose square is $2$---but can you show that there \emph{is} a \emph{real} number whose square is $2$?  (See the subsubsection \textbf{Square-roots} in \cref{sbs3.3.3} \nameref{sbs3.3.3}.)
\end{rmk}
\end{dfn}

\begin{exr}
Let $A,B\subseteq \R$ be nonempty and bounded above.
\begin{enumerate}
\item Show that $\sup (A+B)=\sup (A)+\sup (B)$.
\item Is it necessarily the case that $\sup (AB)=\sup (A)\sup (B)$?
\end{enumerate}
\end{exr}

\section{Concluding remarks}

There are a couple of themes that we started to see in this chapter that you should pay particular attention to.

The first theme is that, if ever we want to have a certain object that just doesn't exist in the context in which we are working, enlarge the context in which you are working.  For example, when we wanted additive inverses but didn't have them, we went from $\N$ to $\Z$; when we wanted multiplicative inverses but didn't have them, we went from $\Z$ to $\Q$; and when we wanted limits but didn't have them, we went from $\Q$ to $\R$.

The second theme you should take note of is the fact that we almost always defined an object by the properties that uniquely specify it.  The point is that it doesn't really matter at the end of the day whether $\R$ is a set of Dedekind cuts or equivalence classes of Cauchy sequences; all that matters is that it is a nonzero complete totally-ordered field.

The third and final theme you should take note of (which is perhaps not quite as manifest as the other two) is that the morphisms matter just as much (if not more) than the objects themselves.  For example, in light of the second theme, we cannot even make sense of the idea of an object being unique without first talking about the morphisms.  More significantly is that if you keep an underlying set fixed, but change the relevant morphisms, you can completely different objects.  For example, if we consider $\Q$ and $\Z$ as crings, $(\Q ,+,0,\cdot ,1)$ and $(\Z ,+,0,\cdot ,1)$, $\Q$ and $\Z$ are totally different; on the other hand, if we forget the extra structure (or, to put it another way, change our morphisms from homomorphisms of rings to just ordinary functions) then $\Q$ and $\Z$ \emph{become the same object}, that is, $\Q \not \cong _{\Ring}\Z$ \emph{but} $\Q \cong \Z$.  The former is obvious (for example, $2$ has an inverse in $\Q$ but not in $\Z$).  As for the latter, we recommend to continue reading the following chapter\textellipsis


\chapter{Basics of the real numbers}

First of all, let us recall our definition of the real numbers.
\begin{textequation}
There exists a nonzero (dedekind) complete totally-ordered field which is unique up to isomorphism of totally-ordered fields.  This field is the field of real numbers.
\end{textequation}
Recall that we also showed (\cref{prp1.4.52}) that $\R$ must contain $\Q$, and hence in turn $\Z$ and $\N$.

\section{Cardinality and countability}\label{CardinalityAndCountability}

This section is a bit of an aside---while not on the real numbers per se, a knowledge of cardinality and countability is absolutely essential to an understanding of the real numbers.

In the first chapter, we briefly discussed the notion of cardinality and its relationship to the natural numbers.  In fact, the natural numbers themselves are, as a set, precisely the finite cardinals.  In this chapter, we investigate the cardinality of $\N$ \emph{itself}.

The first fact we point out is that the cardinality of the natural numbers is the smallest infinite cardinal.
\begin{prp}
Let $\kappa$ be an infinite cardinal.  Then, $|\N |\leq \kappa$.
\begin{proof}
Let $K$ be any set such that $|K|=\kappa$.  Recall that (\cref{dfn1.1.23}) to show that $|\N |\leq \kappa$ requires that we produce an injection from $\N$ into $K$.  We construct an injection $f:\N \rightarrow K$ inductively.  Let $k_0\in K$ be arbitrary and let us define $f(0)\coloneqq k_0$.  If $K\setminus \{ k_0\}$ were empty, then $K$ would not be infinite, therefore there must be some $k_1\in K$ distinct from $k_0$, so that we may define $f(1)\coloneqq k_1$.  Continuing this process, suppose we have defined $f$ on $0,\ldots ,n\in \N$, and wish to define $f(n+1)$.  If $K\setminus \left\{ f(0),\ldots ,f(n)\right\}$ were empty, then $K$ would be finite.  Thus, there must be some $k_{n+1}\in K$ distinct from $f(0),\ldots ,f(n)$.  We may then define $f(n+1)\coloneqq k_{n+1}$.  The function produced must be injective because, by construction, $f(m)$ is distinct from $f(n)$ for all $n<m$.  Hence, $|\N |\leq \kappa$.
\end{proof}
\end{prp}
Thus, the cardinality of the natural numbers is the smallest infinite cardinality.  We give a name to this cardinality.
\begin{dfn}[Countability]\label{dfn2.2}
Let $X$ be a set.  Then, $X$ is \emph{countably-infinite}\index{Countably-infinite} iff $|X|=|\N |$.  $X$ is \emph{countable} iff either $X$ is countably-infinite or $X$ is finite.  We write $\aleph _0\coloneqq |\N |$\index[notation]{$\aleph _0$}.
\begin{rmk}
It is not uncommon for people to use the term ``countable'' to mean what we call ``countably-infinite''.  They would would then just say ``countable or finite'' in cases that we would say ``countable''.
\end{rmk}
\end{dfn}

Our first order of business is to decide what other sets besides the natural numbers are countably-infinite.
\begin{prp}
The even natural numbers, $2\N$, are countably-infinite.
\begin{proof}
On one hand, $2\N \subseteq \N$, so that $|2\N |\leq \aleph _0$.  On the other hand, $2\N$ is infinite, and as we just showed that $\aleph _0$ is the smallest infinite cardinal, we must have that $\aleph _0\leq |2\N |$.  Therefore, by antisymmetry (Bernstein-Cantor-Schr\"{o}der Theorem, \cref{thm1.1.26}) of $\leq$, we have that $|2\N |=\aleph _0$.
\end{proof}
\end{prp}
\begin{exr}
Construct an explicit bijection from $\N$ to $2\N$.
\end{exr}
This is the first explicit example we have seen of some perhaps not-so-intuitive behavior of cardinality.  On one hand, our intuition might tell us that there are half as many even natural numbers as there are natural numbers, yet, on the other hand, we have just proven (in two different ways, if you did the exercise) that $2\N$ and $\N$ have the same number of elements!  This of course is not the only example of this sort of weird behavior.  The next exercise shows that this is actually quite general.
\begin{exr}
Let $X$ and $Y$ be countably-infinite sets.  Show that $X\sqcup Y$ is countably-infinite.
\end{exr}
Thus, it is literally the case that $2\aleph _0=\aleph _0$.  A simple corollary of this is that $\Z$ is countably-infinite.
\begin{exr}
Show that $|\Z |=\aleph _0$.
\end{exr}

You (hopefully) just showed that $2\aleph _0=\aleph _0$, but what about $\aleph _0^2$?
\begin{exr}
Let $X_0,X_1,\ldots $ be a countably-infinite collection of finite sets.  Show that
\begin{equation}
\bigsqcup _{m\in \N}X_m
\end{equation}
is countably-infinite.
\end{exr}
\begin{prp}
$\aleph _0^2=\aleph _0$.
\begin{proof}
For $m\in \N$, define
\begin{equation}
X_m\coloneqq \left\{ (i,j)\in \N \times \N:i+j=m\right\} 
\end{equation}
Note that each $X_m$ is finite and also that
\begin{equation}
\N \times \N =\bigsqcup _{m\in \N}X_m.
\end{equation}
Therefore, by the previous exercise, $|\N \times \N |\eqqcolon \aleph _0^2=\aleph _0$.
\end{proof}
\end{prp}
\begin{exr}
Use Berstein-Cantor-Schr\"{o}der and the previous proposition to show that $|\Q |=\aleph _0$.
\end{exr}
This result might seem a bit crazy at first.  I mean, just `look' at the number line, right?  There's like bajillions more rationals than naturals.  Surely it can't be the case there there are no more rationals than natural numbers, can it?  Well, yes, in fact that can be, and in fact is, precisely the case:  despite what your silly intuition might be telling you, there are no more rational numbers than there are natural numbers.

We mentioned briefly before the algebraic numbers in the remark right before \cref{sbs1.4.2} \nameref{sbs1.4.2}.  We go ahead and define them now because they provide a good exercise in cardinality.
\begin{dfn}[Real algebraic numbers]\label{dfn2.13}
A real number $\alpha \in \R$ is \emph{algebraic}\index{Algebraic (number)} iff there exists a nonzero polynomial with integer coefficients $p$ such that $p(\alpha )=0$.  We write $\A _\R$\index[notation]{$\A _\R$} for the set of real algebraic numbers.
\begin{rmk}
The algebraic numbers (denoted simply by $\A$\index[notation]{$\A$}) are those elements in $\C$ which are roots of a nonzero polynomial with integer coefficients.  The real algebraic numbers are then those elements of $\C$ which are both real numbers and algebraic numbers.  We haven't actually defined $\C$, which is why we only define the real algebraic numbers.
\end{rmk}
\end{dfn}
\begin{exr}
Show that $\A _\R$ is countable.
\end{exr}

So, we've now done both $\Z$ and $\Q$, but what about $\R$?  At first, you might have declared it obvious that there are more real numbers than natural numbers, but perhaps the result about $\Q$ has now given you some doubt.  In fact, it \emph{does} turn out that there are more real numbers than there are natural numbers.
\begin{thm}[Cantor's Cardinality Theorem]\index{Cantor's Cardinality Theorem}
Let $X$ be a set.  Then, $|X|<|2^X|$.
\begin{proof}
We must show two things:  (i) $|X|\leq |2^X|$ and (ii) $|X|\neq |2^X|$.

The first, by definition, requires that we construct an injection from $X$ to $2^X$.  This, however, is quite easy.  We may define a function $X\rightarrow 2^X$ by $x\mapsto \{ x\}$.  This is of course an injection.

The harder part is showing that $|X|\neq |2^X|$.  To show this, we must show that there is \emph{no} surjection from $X$ to $2^X$.  So, let $f:X\rightarrow 2^X$ be a function.  We show that $f$ cannot be surjective.  To do this, we construct a subset of $X$ that cannot be in the image of $f$.

We define
\begin{equation}
S\coloneqq \left\{ x\in X:x\notin f(x)\right\} .
\end{equation}
We would like to show that $S$ is not in the image of $f$.  We proceed by contradiction:  suppose that $S=f(x_0)$ for some $x_0\in X$.  Now, we must have that either $x_0\in S$ or $x_0\notin S$.  If the former were true, then we would have that $x_0\notin f(x_0)=S$:  a contradiction.  On the other hand, in the latter case, we would have $x\in f(x_0)=S$:  a contradiction.  Thus, as neither of these possibilities can be true, there cannot be any $x_0\in X$ such that $f(x_0)=S$.  Thus, $S$ is not in the image of $f$, and so $f$ is not surjective.
\end{proof}
\end{thm}
Now, to show that $|\R |>\aleph _0$, we show that $|\R |=2^{\aleph _0}$.  Before we do this, however, we must know a little more about the real numbers, and so we shall return to this in the next chapter (see the subsubsection \textbf{The uncountability of the real numbers} in \cref{sbs3.3.5} \nameref{sbs3.3.5}).

\section{The absolute value}

$\R$ already has a decent amount of structure:  both an order and a field structure.  We now equip it with yet another structure which will make arguments easier to work with and more intuitive (though as the entire definition involves only the order and algebraic structure (because of the additive inverse), in principle we could do without it).
\begin{dfn}[Absolute value]
The \emph{absolute value}\index{Absolute value} of $x\in \R$, denoted by $\abs{x}$\index[notation]{$\abs{x}$}, is defined by
\begin{equation}
\abs{x}\coloneqq \begin{cases}x & \text{if }x\geq 0 \\ -x & \text{if }x\leq 0\end{cases}.
\end{equation}
For $\varepsilon >0$ and $x_0\in \R$, we write
\begin{equation}
B_{\varepsilon}(x_0)\coloneqq \left\{ x\in \R :|x-x_0|<\varepsilon \right\} \text{ and }D_{\varepsilon}(x_0)\coloneqq \left\{ x\in \R :|x-x_0|\leq \varepsilon \right\} .
\end{equation}
\begin{rmk}
$B_{\varepsilon}(x_0)$\index[notation]{$B_{\varepsilon}(x_0)$} is the \emph{ball}\index{Ball} of radius $\varepsilon$ centered at $x_0$ and $D_{\varepsilon}(x_0)$\index[notation]{$D_{\varepsilon}(x_0)$} is the \emph{disk}\index{Disk} of radius $\varepsilon$ centered at $x_0$.
\end{rmk}
\end{dfn}
\begin{exr}\label{exr3.1.4}
Let $x_1,x_2\in \R$.  Show that the following statements are true.
\begin{enumerate}
\item \label{enm3.3.i}(Nonnegativity) $\abs{x_1}\geq 0$.
\item \label{enm3.3.ii}(Definiteness) $\abs{x_1}=0$ iff $x_1=0$.
\item \label{enm3.3.iii}(Homogeneity) $\abs{x_1x_2}=\abs{x_1}\abs{x_2}$.
\item \label{enm3.3.iv}(Triangle Inequality)\index{Triangle Inequality} $\abs{x_1+x_2}\leq \abs{x_1}+\abs{x_2}$.
\item \label{enm3.3.v}(Reverse Triangle Inequality)\index{Reverse Triangle Inequality} $\abs{\abs{x_1}-\abs{x_2}}\leq \abs{x_1-x_2}$.
\end{enumerate}
\begin{rmk}
The reason that \ref{enm3.3.iv} is called the \emph{triangle inequality} is the following.  First of all, by \ref{enm3.3.iii}, the triangle inequality can instead be written as $\abs{x_1-x_2}\leq \abs{x_1}+\abs{x_2}$.  Then, if you pretend that $x_1$ and $x_2$ are vectors representing the sides of the triangle, $x_1-x_2$ is a vector representing the third side of the triangle.  The triangle inequality then states that the length of a side of a triangle is at most the sum of the lengths of the other two sides (being equal iff the angle between those two sides is precisely $\uppi$).  Your solution should help explain why the reverse triangle inequality is called what it is.
\end{rmk}
\end{exr}
Intuitively, of course, $\abs{x}$ is supposed to be the distance $x$ is from $0$.  Then, $|x-y|$ is suppose to be the distance between $x$ and $y$.  A simple result we have is that, if distance between two integers is less than $1$, then they are the same integer.
\begin{prp}\label{prp3.2}
Let $0<\varepsilon <1$ and let $m,n\in \Z$.  Then, if $\abs{m-n}<\varepsilon$, then $m=n$.
\begin{proof}
Suppose that $\abs{m-n}<\varepsilon$.  Without loss of generality, suppose that $m\leq n$, so that we can write $n=m+k$ for $k\in \Z _0^+$.  Then, $\abs{m-n}=k<\varepsilon <1$.  It follows from \cref{exr1.2.14} that $k=0$, so that $m=n$.
\end{proof}
\end{prp}

\section{The Archimedean Property}\label{sct3.2}

The first property of the real numbers we come to is called the Archimedean Property.  The Archimedean Property essentially says that the natural numbers are unbounded in the reals.
\begin{dfn}[The Archimedean Property]
Let $F$ be a nonzero totally-ordered field so that $\Q \subseteq F$ (see \cref{prp1.4.52}).  Then, we say that $F$ is \emph{archimedean}\index{Archimedean field} iff for all $x\in F$ there is some $m\in \N \subseteq F$ such that $x<m$.
\end{dfn}
\begin{exr}\label{exr3.1.6}
Let $F$ be an archimedean ordered field and let $\varepsilon >0$.  Show that there is some $m\in \N \subseteq F$ such that $\frac{1}{m}<\varepsilon$.
\end{exr}
\begin{thm}[Archimedean property of the real numbers]\label{thm3.2.3}
$\R$ is archimedean.
\begin{proof}
If $x\leq 0$, we may take $m=0$.  Otherwise, $x>0$, and so the set
\begin{equation}
S\coloneqq \left\{ m\in \N :m<x\right\} 
\end{equation}
is nonempty (it contains $0$).  On the other hand, it is also bounded-above (by $x$), and so it has a supremum:  write $m_0\coloneqq \sup (S)$.  We first show that $m_0\in \N$, and then we will show that $x\leq m_0+1$.

By \cref{prp1.4.11}, there must be some $m_1\in S$ such that
\begin{equation}
m_0-\tfrac{1}{2}<m_1\leq m_0.
\end{equation}
If $m_1=m_0$, we are done (because $m_1\in \N$), so otherwise suppose that $m_1<m_0$.  Then, we may use this same proposition again to obtain an $m_2\in S$ such that
\begin{equation}
m_1<m_2\leq m_0.
\end{equation}
But then $|m_1-m_2|<\frac{1}{2}$ (because they are both between $m_0-\frac{1}{2}$ and $m_0$), and so $m_1=m_2$ (by \cref{prp3.2}):  a contradiction (of the fact that $m_1<m_2$).  hence, it must have been the case that $m_0=m_1\in \N$.

We cannot have that $m_0+1\in S$ because then otherwise $m_0$ would not be an upper-bound of $S$.  Therefore, $m_0+1\in \N \setminus S=\left\{ m\in \N :x\leq m\right\}$, and so $x\leq m_0+1$, and so $x<m_0+2$.
\end{proof}
\end{thm}
This might seem like it is obvious, but it is in fact not true of all totally-ordered fields.  To present such an example, we have to take a bit of an aside and discuss polynomial crings and fields of rational functions.  Such crings and fields are important for many other reasons than to present an example of a totally-ordered field that does not have the Archimedean Property, so it is worthwhile to understand the material anyways, though you might want to come back to it if you don't care about the counter-example at the time being.

\subsubsection{Polynomial crings and a totally-ordered field which is not archimedean}

\begin{dfn}[Polynomial cring]
\begin{savenotes}
Let $R$ be a totally-ordered cring, denote by $R[x]$ the set of all polynomials with coefficients in $R$, let $+$ and $\cdot$ on $R[x]$ be addition and multiplication of polynomials, and define $p>0$ iff the leading coefficient of $p$ is greater than $0$ in $R$.\footnote{Recall that this is enough to define a total-order by \cref{exr1.1.41}.} 
\begin{exr}
Show that $\left( R[x],+,0,-,\cdot ,1,\leq \right)$ is a totally-ordered cring.  What is $0\in R[x]$?  What is $1\in R[x]$?
\end{exr}
\noindent
$R[x]$ is the \emph{polynomial cring}\index{Polynomial cring} with coefficients in $R$.  The \emph{degree}\index{Degree (of a polynomial)} of a polynomial $p$, denoted by $\deg (p)$\index[notation]{$\deg (p)$}, is the highest power of $x$ that appears in $p$ with nonzero coefficient.
\end{savenotes}
\end{dfn}
\begin{exr}
Show that the following statements are true.
\begin{enumerate}
\item $\deg (p+q)\leq \max \{ \deg (p),\deg (q)\}$.
\item $\deg (pq)\leq \deg (p)+\deg (q)$.
\item If $R$ is integral, then $\deg (pq)=\deg (p)+\deg (q)$.
\end{enumerate}
\end{exr}
\begin{exr}
Show that $R$ is integral iff $R[x]$ is.
\end{exr}
\begin{thm}[Field of rational functions]
Let $F$ be a field, so that $F[x]$ is a totally-ordered integral cring.  Then, there exists a totally-ordered field $F(x)$, the field of \emph{rational functions with coefficients in $F$}\index{Rational functions}, such that
\begin{enumerate}
\item $F(x)$ contains $F[x]$ as a subpreordered ring, and
\item if $F'$ is any other totally-ordered field which satisfies this property, then $F'$ contains $F$ as a subpreordered field.
\end{enumerate}
Furthermore, $F(x)$ is unique up to isomorphism of preordered fields.
\begin{rmk}
Compare this to our definition of the rational numbers in \cref{RationalNumbers}.  $F(x)$ is to $F[x]$ as $\Q$ is to $\Z$.  Indeed, we mentioned there that the passage from $\Z$ to $\Q$ was an example of the more general construction of the \emph{fraction field} of an integral cring.  This is another example of this construction.  In fact, the proof of this theorem is exactly the same as the proof of \cref{RationalNumbers}, and so we refrain from presenting it again (it was an exercise anyways).
\end{rmk}
\end{thm}
Of course, we have a result from $F(x)$ which is completely analogous to the result \cref{prp1.3.4} for $\Q$ (which just says that we can write any rational function uniquely as the quotient of two relatively-prime polynomials with the denominator positive).
\begin{exm}[A totally-ordered field which is not archimedean]\label{exm2.3.12}
$\R (x)$ is a totally-ordered field, but on the other hand, $m<x$ for all $m\in \N$, and so $\R (x)$ is not archimedean.
\end{exm}
We mentioned in a remark when defining the real numbers (\cref{RealNumbers}) that, in general, the arithmetic operations on a partially-ordered field do not extend to its dedekind-MacNeille completion in a compatible way.  That $\R (x)$ is not archimedean immediately tells us that this field is also a counter-example to this statement.
\begin{exm}[A totally-ordered field whose dedekind-MacNeille completion cannot be given the structure of a totally-ordered field]\label{exm3.2.13}
If the arithmetic operations on $\R (x)$ extended to give the structure of a totally-ordered field on its dedekind-MacNeille completion, then, by uniqueness, its dedekind-MacNeille completion would have to be $\R$ itself.  In particular, $\R (x)$ would have to embed into $\R$.  But then, there would have to be some natural number $m\in \N$ with $\R \supseteq \R (x)\ni x\leq m$ (because $\R$ is archimedean:  a contradiction of the fact that $\R (x)$ is not archimedean.
\end{exm}

\horizontalrule

The second property of the real numbers we come to is what is sometimes called the density of the rationals in the reals.  This is not quite literally true, though we will see why people refer to this as density when we discuss basic topology in the next chapter---see \cref{exr4.2.38}.
\begin{thm}[`Density' of $\Q$ in $\R$]\label{thm3.2.14}
Let $a,b\in \R$.  Then, if $a<b$, then there exists $c\in \Q$ such that $c\in (a,b)$.
\begin{rmk}
It turns out that this is also the case for the irrational numbers, $\Q ^{\comp}$, but at the moment, we don't even know how to construct a single irrational number, so we will have to return to this at a later date (\cref{thm3.3.76}).\footnote{You might be able to show that there is no positive rational number whose square is $2$, but can you show that there \emph{is} some positive \emph{real} number whose square is $2$?}
\end{rmk}
\begin{proof}
The intuitive idea is to take a positive integer $m$ at least as large as the length of the interval $b-a$ and break-up the real line into intervals of length $\frac{1}{m}$.  Then, one of the end-points of these intervals must lie in the interval $(a,b)$.  This will be our desired point.

Define $\varepsilon \coloneqq b-a>0$ and choose $m_0\in \N$ with $\frac{1}{m_0}<\varepsilon$ (see \cref{exr3.1.6}).  Define
\begin{equation}
S\coloneqq \left\{ m\in \N :\tfrac{m}{m_0}\geq b\right\} .
\end{equation}
By the Archimedean Property, $S$ is nonempty.  Therefore, because $\N$ is well-ordered, it has a least element, say $\frac{n_0}{m_0}$.  We claim that $\frac{n_0-1}{m_0}\in (a,b)$.

First of all, we know that $n_0-1\notin S$, and so $\frac{n_0-1}{m_0}<b$.  On the other hand,
\begin{equation}
\tfrac{n_0-1}{m_0}=\tfrac{n_0}{m_0}-\tfrac{1}{m_0}\geq b-\tfrac{1}{m_0}>b-\varepsilon =b-(b-a)=a,
\end{equation}
and so $\frac{n_0-1}{m_0}\in (a,b)$.
\end{proof}
\end{thm}

\section{Nets, sequences, and limits}

\subsection{Nets and sequences}

You probably recall sequences from calculus.
\begin{textequation}
A (real-valued) \emph{sequence} is a function from $\N$ to $\R$.
\end{textequation}
(This is not our `official' definition of a sequence.  We will present another definition later.  This is why the above has no `definition bar'.)

A net is a generalization of a sequence.  We will actually not need to make use of nets much in this course; the point of introducing them despite this is to put emphasis on the \emph{structure} $\N$ is equipped with in this context.  In particular, the $\N$ in the definition of a sequence is \emph{not} to be thought of as a crig; for the purposes of sequences, the algebraic structure does not matter.  Instead, in the context of sequences, $\N$ should be thought of as a directed set.  Another motivation for working with nets is that, when you go to generalize to examples more exotic than the real numbers, some of the results that are true in the reals would fail to generalize if we restricted ourselves to only work with sequences.  See the remarks in \cref{prp3.4.21,prp3.4.56}.
\begin{dfn}[Directed set]\label{dfn3.3.2}
A \emph{directed set}\index{Directed set} is a nonempty partially-ordered set $(\Lambda ,\leq )$ such that, for every $x_1,x_2\in X$, there is some $x_3\in X$ with $x_1,x_2\leq x_3$.
\end{dfn}
\begin{exr}
Show that $(\N ,\leq )$ is a directed set.
\end{exr}
\begin{dfn}[Net]
A (real-valued) \emph{net}\index{Net} is a function from a directed set $(\Lambda ,\leq )$ into $\R$.
\begin{rmk}
Similarly as with sequences, if $a:\Lambda \rightarrow \R$ is a net, it is customary to write $a_\lambda \coloneqq a(\lambda )$\index[notation]{$a_\lambda$}.
\end{rmk}
\begin{rmk}
It is very common that the specific directed set that is the domain is not so important and that the result works for any directed set.  As a result, in attempt to simplify notation slightly, we frequently omit the actual domain of nets.  You should usually be able to infer the domain on the basis of the index we happen to be using.
\end{rmk}
\begin{rmk}
In general, nets will take their values in topological spaces.  Of course, we haven't defined what a topological space is yet (we could, but it would probably seem like I just pulled some definition out of my ass), and so for the time being nets will take their values in $\R$.
\end{rmk}
\end{dfn}
And now we give our `official' definition of a sequence.
\begin{dfn}[Sequence]\label{dfn3.3.4}
A (real-valued) \emph{sequence}\index{Sequence} is a net whose directed set is order-isomorphic (i.e.~isomorphic in $\Pre$) to $(\N ,\leq )$.
\begin{rmk}
Typically people take a sequence to be a function from $\N$ into $\R$.  This is essentially fine, but then, technically speaking, a function from $\Z ^+$ to $\R$ is not a sequence, and you have to reindex everything by $1$ to make it a sequence.  It is easier if we just ignore this by only requiring that the domain of the net be \emph{isomorphic to} (in $\Pre$) instead of \emph{equal to} $(\N ,\leq )$.
\end{rmk}
\begin{rmk}
If the name of the sequence is not important, we may simply denote it by a list of it's values, e.g.~$(0,1,2,3,\ldots )$.
\end{rmk}
\end{dfn}
\begin{exm}[A net which is not a sequence]
Recall that (\cref{exrA.1.26}) $2^{\R}$, the power-set of the reals, is a partially-ordered set with the order relation being inclusion.
\begin{exr}
Check that $(2^{\R},\leq )$ is directed.
\end{exr}
Now let $f:\R \rightarrow \R$ be \emph{bounded}.  That $f$ is bounded means that
\begin{equation}
a_S\coloneqq \sup \left\{ \abs{f(x)}:x\in S\right\} \coloneqq \sup _{x\in S}\left\{ \abs{f(x)}\right\}
\end{equation}\index[notation]{$\sup _{x\in S}$}
exists for all subsets $S\in 2^{\R}$.  Therefore, the map $S\mapsto a_S$ is a net.  Without knowing yet the precise definition of convergence, do you know what the limit should be?
\end{exm}

You'll recall that the third theme that we mentioned at the end of \cref{chp1} \nameref{chp1} was ``morphisms matter''.  The morphisms determine the relevant structure, and so, in particular, not just the set, but the structure with which it is equipped, matters.  The point of introducing the notion of a net at this stage is simply to point-out that, when working with sequences, $\N$ is to be thought of as a directed-set, as opposed to, for example, a crig.

\subsection{Limits}

We now define what it means to be the \emph{limit} of a net.  To make the definition as clean and concise as possible, we introduce the concept of nets \emph{eventually} doing somethign.

\subsubsection{Eventuality}

We often use the word \emph{eventually}\index{Eventually\textellipsis (nets)} in the context of nets and sequences.  For example, we might say ``The net $\lambda \mapsto a_\lambda$ is eventually XYZ.''.  This means that there is some $\lambda _0$ such that if $\lambda \geq \lambda _0$ it follows that $\lambda \mapsto a_\lambda$ is XYZ.  For example, a fact that will come in handy is that convergent nets (and in fact, cauchy nets, once we define ``cauchy'') are eventually bounded (\cref{prp3.3.28}).

If a net is eventually XYZ, it can be convenient to essentially just `throw away' the terms at the beginning which are not XYZ so as to obtain a net which is not just eventually XYZ, but is XYZ \emph{itself} (for example, you can `chop off' the beginning of a convergent net to obtain a bounded net).

For all intents and purposes, you should think of the `first elements' of a net as not mattering; only what \emph{eventually} happens is what matters.  For example, consider the sequence $m\mapsto a_m\coloneqq 1$, that is, the constant sequence $1\in \R$.  Obviously this converges to $1$.  The point to note here is that, you can do \emph{whatever you like} to any finite number of elements of this sequence, and you will have no effect upon the fact that it converges to $1$.  We can essentially `throw away' any finite amount of the sequence and nothing will change.  This idea can actually be quite important in proofs where we might know nothing about the first couple of elements.

\horizontalrule

Now with the concept of eventuality in hand, we present the definition of convergence.
\begin{dfn}[Limit (of a net)]\label{dfn3.3.8}
Let $\lambda \mapsto a_\lambda$ be a net and let $a_\infty\in \R$.  Then, $a_\infty$ is the \emph{limit}\index{Limit (of a net)} of $\lambda \mapsto a_\lambda$ iff for every $\varepsilon >0$, $\lambda \mapsto a_\lambda$ is eventually contained in $B_\varepsilon (a_\infty)$.   If a net has a limit, then we say that it \emph{converges}\index{Convergence}.\footnote{Divergence is not the same as nonconvergence.  We will define divergence momentarily.}
\begin{rmk}
Note the use of the term \emph{eventually}.  Explicitly, this is equivalent to
\begin{textequation}
$a_\infty$ is the limit of $\lambda \mapsto a_\lambda$ iff for every $\varepsilon >0$ there is some $\lambda _0$ such that, whenever $\lambda \geq \lambda _0$, it follows that $a_\lambda \in B_{\varepsilon}(a_\infty)$.
\end{textequation}
Of course, when actually doing proofs, you will often need to make use of the explicit definition, but I think it's fair to say that the definition that uses the term ``eventually'' is both more intuitive and concise.
\end{rmk}
\begin{exr}[Limits are unique (in $\R$)]
Let $a_\infty,b_\infty\in \R$ be limits of the net $\lambda \mapsto a_\lambda$.  Show that $a_\infty=b_\infty$.
\begin{rmk}
In general topological spaces (though not for most `reasonable' ones), limits need not be unique (hence the reason for adding ``in $\R$'').  In fact, limits are unique iff the space is $T_2$---see \cref{prp4.5.37}.
\end{rmk}
\end{exr}
\begin{rmk}
If $a_\infty$ is the limit of $\lambda \mapsto a_\lambda$, then we write $\lim _\lambda a_\lambda =a_\infty$\index[notation]{$\lim _\lambda x_\lambda$}.  Note that this is unambiguous by the previous exercise.
\end{rmk}
\end{dfn}

\begin{exr}
Let $x_0\in \R$ and define $\lambda \mapsto a_\lambda \coloneqq x_0$.  Show that $\lim _\lambda a_\lambda =x_0$.
\begin{rmk}
That is, constant nets converge to that constant.  While trivial, it is significant in that it becomes an axiom of the convergence definition of a topological space.
\end{rmk}
\end{exr}
\begin{exr}
Show that $\lim _m\frac{1}{m}=0$.
\end{exr}
\begin{prp}
Let $\abs{a}<1$.  Show that $\lim _ma^m=0$.
\begin{proof}
Define $b\coloneqq \frac{1}{a}$.  Let $M>0$.  We show that $m\mapsto b^m$ is eventually larger than $M$.  It will then follows that $m\mapsto a^m$ is eventually smaller than $\frac{1}{M}$.  As $\frac{1}{M}$ is just as arbitrary as $\varepsilon >0$, this will show that $\lim _ma^m=0$.

We have
\begin{equation}
b^m=(1+(b-1))^m=\sum _{k=0}^m\binom{m}{k}(b-1)^k\geq 1+m(b-1).
\end{equation}
Because $b-1$, it follows from the Archimedean Property (\cref{thm3.2.3}) that $m\mapsto b^m$ is eventually larger than $M$, which completes the proof.
\end{proof}
\end{prp}


\begin{dfn}[Divergence]
Let $\lambda \mapsto a_\lambda$ be a net.  Then, $\lambda \mapsto a_\lambda$ \emph{diverges to $+\infty$}\index{Divergence} iff for every $M>0$ there is some $\lambda _0$ such that whenever $\lambda \geq \lambda _0$, it follows that $a_\lambda \geq M$.  Similarly for $-\infty$ (you should consider writing this definition out explicitly yourself).  $\lambda \mapsto a_\lambda$ \emph{diverges} iff it either diverges to $+\infty$ or diverges to $-\infty$.
\begin{rmk}
Of course, the intuition is just that $a_\lambda$ grows arbitrarily large.
\end{rmk}
\end{dfn}
\begin{exr}
What is an example of a net which neither converges nor diverges?
\end{exr}

The biggest problem with proving that a net converges is that we first need to know what the limit is before hand.  For example, consider
\begin{equation}\label{3.2.16}
\lim _m\sum _{k=0}^m\tfrac{1}{k!}.
\end{equation}
You probably recall that this \emph{should} converge to $\e$, but what the hell is $\e$?  It has not yet been defined.  In fact, we will later define
\begin{equation}
\e \coloneqq \lim _m\sum _{k=0}^m\tfrac{1}{k!},
\end{equation}
but we cannot even make this definition if we don't know a priori that the limit of \eqref{3.2.16} exists!  Thus, we need to have a way of showing that \eqref{3.2.16} exists without making explicit reference to $\e$.  The concepts of \emph{cauchyness} and \emph{completeness} (in the sense of uniform spaces, as opposed to in the sense of partially-ordered sets) allow us to do this.

\subsection{Cauchyness and completeness}\label{sbs3.3.3}

Like with the concept of convergence and limits, we will first define what it means to be cauchy, and then explain the intuition behind the definition.
\begin{dfn}[Cauchyness]\label{dfn3.3.26}
Let $\lambda \mapsto a_\lambda$ be a net.  Then, $\lambda \mapsto a_\lambda$ is \emph{cauchy}\index{cauchy} iff for every $\varepsilon >0$, $\lambda \mapsto a_\lambda$ is eventually contained in some $\varepsilon$-ball.
\begin{rmk}
Note the use of the term \emph{eventually}.  Similarly as before, explicitly, this means that
\begin{textequation}
$\lambda \mapsto a_\lambda$ is cauchy iff there is some $\lambda _0$ and some $\varepsilon$-ball $B$ such that, whenever $\lambda \geq \lambda _0$, it follows that $a_\lambda \in B$.
\end{textequation}
\end{rmk}
\end{dfn}
You should compare this with the definition of a limit, \cref{dfn3.3.8}.  The only essential difference between the definition of convergence and the definition of cauchy is that, in the former case, there is a \emph{single} center of the $\varepsilon$-ball that works for all $\varepsilon$, whereas, in the case of cauchyness, the center of the ball can vary with the choice of $\varepsilon$.  Thus, we immediately have the implication:
\begin{exr}\label{exr3.3.27}
Let $\lambda \mapsto a_\lambda$ be a convergent net.  Show that $\lambda \mapsto a_\lambda$ is cauchy.
\end{exr}

As a short aside, we note that this is not how the definition of cauchyness is stated.  Instead, the following equivalent condition is given as the definition.
\begin{exr}\label{exr3.3.28}
Let $\lambda \mapsto a_\lambda$ be a net.  Show that $\lambda \mapsto a_\lambda$ is cauchy iff for all $\varepsilon >0$ there is some $\lambda _0$ such that whenever $\lambda _1,\lambda _2\geq \lambda _0$ it follows that $\abs{a_{\lambda _1}-a_{\lambda _2}}<\varepsilon$.
\end{exr}
We choose to use the \cref{dfn3.3.26} because (i) it more closely resembles the definition of convergence and (ii) it more closely resembles the definition in the more general case of uniform spaces (see \cref{Cauchyness}).  That being said, the equivalent condition in this exercise (\cref{exr3.3.28}) is frequently easier to check.  For example:
\begin{exm}\label{exm3.3.39}
In this example, we check that the sequence
\begin{equation}
m\mapsto s_m\coloneqq \sum _{k=0}^m\tfrac{1}{k!}
\end{equation}
is cauchy.  (Note that $s_m\in \Q$, so in fact, this sequence is cauchy in $\Q$ as well as in $\R$.)
\begin{equation}
\abs{s_m-s_n}=\left| \sum _{k=0}^m\tfrac{1}{k!}-\sum _{k=0}^n\tfrac{1}{k!}\right| =\sum _{k=m+1}^n\tfrac{1}{k!}\leq \sum _{k=m+1}^n\tfrac{1}{2^k}=\frac{2^n-2^m}{2^{m+n}}\leq 2^{-m}
\end{equation}
where we have without loss of generality assumed that $3\leq m\leq n$ (so that $2^k\leq k!$ for $k\geq m+1$).
\begin{exr}
Finish the proof that $m\mapsto s_m$ is cauchy.
\end{exr}
\end{exm}

One way to immediately tell if a net is not cauchy is if it is eventually unbounded. 
\begin{prp}\label{prp3.3.28}
Let $\lambda \mapsto a_\lambda$ be a cauchy net.  Then, $\lambda \mapsto a_\lambda$ is eventually bounded.  In particular, if this net is a sequence, the set $\left\{ a_m :m\in \N \right\}$ is bounded.
\begin{proof}
Applying the definition to $\varepsilon \coloneqq 1$, we deduce that there must be some $\lambda _0$ and some open ball $B$ of radius $1$ such that, whenever $\lambda \geq \lambda _0$, it follows that $a_\lambda \in B$.  Thus, $\lambda \mapsto a_\lambda$ is eventually in $B$, a bounded subset of $\R$, and hence itself is eventually bounded.

Now assume that $m\mapsto a_m$ is a sequence so that $m\in \N$.  Let us also write $m_0\coloneqq \lambda _0$.  The key to note here is that, unlike in the general case, the set $P_{m_0}^{\comp}=\{ m\in \N :m<m_0\}$ is \emph{finite}.  This enables us to make the definition
\begin{equation}
r_0\coloneqq \max \left\{ \abs{a_0}|,\ldots ,\abs{a_{m_0-1}},1+\abs{x_0}\right\} ,
\end{equation}
where $x_0$ is the center of the ball $B$.  The reason for the $1+|a_\infty|$ is as follows.  For $m\geq m_0$, we have that
\begin{equation}
\abs{a_m}=\abs{(a_m-x_0)+x_0}\leq \abs{a_m-a_\infty}+\abs{x_0}=1+\abs{x_0}.
\end{equation}
From this, it follows that $\left\{ a_m:m\in \N ]\right\} \subseteq B_{r_0}(a_\infty)$, so that $\left\{ a_m:m\in \N \right\}$ is bounded.
\end{proof}
\end{prp}

\subsubsection{Algebraic and Order Limit Theorems}

Ideally, the content in this subsubsection would have been discussed very soon after defining a limit itself, however, we will use the fact that convergent nets are eventually bounded.  We could have proved this, then done the Algebraic and Order Limit Theorems, but when we would have wound-up essentially proving the same theorem twice (because cauchy nets are eventually bounded as well).  We decided to just postpone these results to a slightly less-than-maximally-coherent location in the notes instead.
\begin{prp}[Algebraic Limit Theorems]\index{Algebraic Limit Theorems}\label{AlgebraicLimitTheorems}
\begin{savenotes}
Let $\lambda \mapsto a_\lambda$ and $\lambda \mapsto b_\lambda$ be two nets\footnote{The common index is supposed to indicate that the two nets have the same domain.} which converge to $a_\infty,b_\infty\in \R$ respectively, and let $\alpha \in \R$.  Then,
\begin{enumerate}
\item \label{enmAlgebraicLimitTheorems.i}$\lim (a_\lambda +b_\lambda )=a_\infty+b_\infty$,
\item \label{enmAlgebraicLimitTheorems.ii}$\lim (a_\lambda b_\lambda )=a_\infty b_\infty$,
\item \label{enmAlgebraicLimitTheorems.iii}$\lim (a_\lambda ^{-1})=a_\infty ^{-1}$ if $a_\infty \neq 0$, and
\item \label{enmAlgebraicLimitTheorems.iv}$\lim (\alpha a_\lambda )=\alpha a_\infty$.
\end{enumerate}
\begin{rmk}
Later, we will see how all of this follow automatically from continuity considerations (because $+$ is continuous, for example).
\end{rmk}
\begin{proof}
We leave \ref{enmAlgebraicLimitTheorems.i}, \ref{enmAlgebraicLimitTheorems.ii}, and \ref{enmAlgebraicLimitTheorems.iv} as exercises.  Should you need guidance, check out our proof of \ref{enmAlgebraicLimitTheorems.iii}.
\begin{exr}
Prove \ref{enmAlgebraicLimitTheorems.i}.
\end{exr}
\begin{exr}
Prove \ref{enmAlgebraicLimitTheorems.ii}.
\end{exr}
We prove \ref{enmAlgebraicLimitTheorems.iii}.  It is likely the most difficult, and if you can follow this, you should be able to do the others on your own.

As $a_\infty \neq 0$, this means that is is eventually bounded away from $0$:  Without loss of generality take $a_\infty >0$ (the other case is essentially identical).  Define $\varepsilon \coloneqq \frac{1}{2}\abs{a_\infty}=\frac{1}{2}a_\infty$.  Then, there is some $\lambda _0$ so that whenever $\lambda \geq \lambda _0$ it follows that $a_\infty -a_\lambda \leq \abs{a_\lambda -a_\infty}<\varepsilon \coloneqq \frac{1}{2}a_\infty$.  It follows that
\begin{equation}
a_\lambda >a_\infty -\tfrac{1}{2}a_\infty =\tfrac{1}{2}a_\infty.
\end{equation}
As $\lambda \mapsto a_\lambda$ is eventually (in this case) positive, we may without loss of generality assume that there there is some $M>0$ such that $a_\lambda >M$ \emph{for all} $\lambda$.

Let $\varepsilon >0$ and choose $\lambda _0$ such that, whenever $\lambda \geq \lambda _0$, it follows that $\abs{a_\lambda -a_\infty}<\varepsilon$.  Then, for $\lambda \geq \lambda _0$,
\begin{equation}
\abs{a_\lambda ^{-1}-a_\infty ^{-1}}=\frac{\abs{a_\infty -a_\lambda}}{\abs{a_\lambda a_\infty}}\leq \tfrac{1}{M^2}\abs{a_\lambda -a_\infty}<\tfrac{1}{M^2}\varepsilon .
\end{equation}
As $\frac{1}{M^2}\varepsilon$ is just as arbitrary as $\varepsilon$ (this uses the fact that $M$ does not depend on $\varepsilon$), this completes the proof.
\begin{exr}
Prove \ref{enmAlgebraicLimitTheorems.iv}.
\end{exr}
\end{proof}
\end{savenotes}
\end{prp}
\begin{exr}[Order Limit Theorem]\index{Order Limit Theorem}\label{exr3.3.30}
Let $\lambda \mapsto a_\lambda$ and $\lambda \mapsto b_\lambda$ be two convergent nets and suppose that it is eventually the case that $a_\lambda \leq b_\lambda$.  Show that $\lim _\lambda a_\lambda \leq \lim _\lambda b_\lambda$.
\begin{rmk}
Sometimes this (or really, I suppose, an immediate corollary of this) is called the \emph{Squeeze Theorem}\index{Squeeze Theorem}.
\end{rmk}
\end{exr}

\horizontalrule

As you (hopefully) just showed in \cref{exr3.3.27}, convergent sequences are always cauchy.  However, it is in general not the case that every cauchy sequence converges.  For example, once we show that the definition of $\e \coloneqq \lim _m\sum _{k=0}^m\frac{1}{k!}$ makes sense (we're about to) and that $\e$ is irrational, then this will serve as an example of a sequence that is cauchy in $\Q$ (as we just showed) but does not converge in $\Q$ (because $\e$ is irrational).  However, it \emph{is} true that every cauchy sequence does converge in $\R$, and in fact, you might say this is the real reason we care about $\R$ at all and that the motivation for requiring the existence of least upper-bounds was so that we could prove this.  Before we actually prove this however, we first need to discuss limit superiors and limit inferiors.

\subsubsection{Limit superiors and limit inferiors}

The Monotone Convergence Theorem is the tool that will allow us to define $\lim \sup$ and $\lim \inf$.
\begin{prp}[Monotone Convergence Theorem]\index{Monotone Convergence Theorem}\label{MonotoneConvergenceTheorem}
Let $\lambda \mapsto a_\lambda$ be a nondecreasing net that is bounded above, or alternatively, a nonincreasing net that is bounded below.  Then, $\lambda \mapsto a_\lambda$ converges.
\begin{rmk}
Thus, if $\lambda \mapsto a_\lambda$, it \emph{always} makes sense to write down $\lim _\lambda a_\lambda$.  In the nondecreasing case, either the net is bounded above, and the Monotone Convergence Theorem tells us that $\lim _\lambda a_\lambda$ exists, or it is not bounded above, in which case $\lim _\lambda a_\lambda \coloneqq +\infty$.
\end{rmk}
\begin{proof}
We just do the case where $\lambda \mapsto a_\lambda$ is nondecreasing and bounded above.  The other case is essentially identical.  Define
\begin{equation}
a_\infty\coloneqq \sup _\lambda \{ a_\lambda \} .
\end{equation}
As the net is bounded above, this definition makes sense.  Now we need only prove that $\lim _\lambda a_\lambda =a_\infty$.

So, let $\varepsilon >0$.  By \cref{prp1.4.11}, there is some $a_{\lambda _0}$ such that
\begin{equation}
a_\infty-\varepsilon <a_{\lambda _0}\leq a_\infty.
\end{equation}
Now suppose that $\lambda \geq \lambda _0$.  Then, by monotonicity, we have that
\begin{equation}
a_\infty-\varepsilon <a_{\lambda _0}\leq a_\lambda \leq a_\infty.
\end{equation}
This, however, implies that $a_\lambda \in B_{\varepsilon}(a_\infty)$, so that, by definition, $\lim _\lambda a_\lambda =a_\infty$.
\end{proof}
\end{prp}
\begin{dfn}[Limit superior and limit inferior]
Let $a:\Lambda \rightarrow \R$ be a net and for each $\lambda _0\in \Lambda$ define
\begin{equation}\label{3.3.48}
u_{\lambda _0}\coloneqq \sup _{\lambda \geq \lambda _0}\{ a_\lambda \} \text{ and }l_{\lambda _0}\coloneqq \inf _{\lambda \geq \lambda _0}\{ a_\lambda \} .
\end{equation}
\begin{exr}
Check that $\lambda \mapsto u_\lambda$ is nonincreasing and that $\lambda \mapsto l_\lambda$ is nondecreasing.
\end{exr}
Then, we define the \emph{limit superior}\index{Limit superior} and the \emph{limit inferior}\index{Limit inferior} respectively
\begin{equation}\label{3.3.50}
\begin{split}
\lim \sup a_\lambda & \coloneqq \lim _\lambda u_\lambda \\
\lim \inf a_\lambda &\coloneqq \lim _\lambda l_\lambda .
\end{split}
\end{equation}\index[notation]{$\lim \sup _\lambda x_\lambda$}\index[notation]{$\lim \inf _\lambda x_\lambda$}
\begin{rmk}
Note that it is the Monotone Convergence Theorem which guarantees that this definition makes sense.  In particular, unlike limits themselves, $\lim \sup$ and $\lim \inf$ \emph{always} make sense.
\end{rmk}
\begin{rmk}
The intuition is that $\lambda \mapsto u_\lambda$ is the net of `eventual upper bounds'.  $\lim \sup _\lambda a_\lambda$ is then the limit of this net (similarly for $\lim \inf$).
\end{rmk}
\end{dfn}
\begin{dfn}
Let $\lambda \mapsto a_\lambda$ be a net.  Show one of the following two statements.
\begin{enumerate}
\item $\lim \sup _\lambda a_\lambda$ is finite iff $\lambda \mapsto a_\lambda$ is eventually bounded above.
\item $\lim \inf _\lambda a_\lambda$ is finite iff $\lambda \mapsto a_\lambda$ is eventually bounded below.
\end{enumerate}
\begin{rmk}
Both statements are true of course, but the proofs will be so similar that there is not really much point in asking you to write both down.
\end{rmk}
\end{dfn}
\begin{exr}\label{exr3.3.50}
Let $\lambda \mapsto a_\lambda$ be a net.  Show that $\lim \inf _\lambda a_\lambda \leq \lim \sup _\lambda a_\lambda$.
\begin{rmk}
Note that if we have equality at a finite value, then the net converges to the common value.  See \eqref{3.3.52}.
\end{rmk}
\end{exr}
A useful result is the following, though we won't have the tools to prove it until the end of the chapter (see \cref{prp3.3.52}).
\begin{textequation}[3.3.52]
Let $\lambda \mapsto a_\lambda$ be a net.  Show that $\lambda \mapsto a_\lambda$ converges iff $\lim \sup _\lambda a_\lambda =\lim \inf _\lambda a_\lambda$ is finite, and in this case, $\lim _\lambda a_\lambda$ is equal to this common value.
\end{textequation}
\begin{exm}
Note that we can have equality at an \emph{infinite} value.  For example, consider the sequence $m\mapsto a_m\coloneqq m$.  This is obviously bounded above, and so by definition $\lim \sup _ma_m=\infty$.  On the other hand, $\inf _{m\geq m_0}\{ a_m\} =m_0$, and so $\lim \inf _ma_m=\infty$ as well.
\end{exm}

\horizontalrule

Now we can finally return to our current goal of proving that cauchy nets converge in $\R$.
\begin{thm}[Completeness of $\R$]
Let $\lambda \mapsto a_\lambda$ be a cauchy net.  Then, $\lambda \mapsto a_\lambda$ converges.
\begin{rmk}
This property of having all cauchy nets converge is known as \emph{(cauchy) completeness}.  Being complete is a property that a uniform space may or may not have,\footnote{No, you are not expected to know what a uniform space is yet (though see \cref{UniformSpace}) if you can't wait to find out.} and we will see that, when we equip $\R$ with a uniform structure, this result is equivalent to completeness in the sense of uniform spaces.  You should be careful not to confuse cauchy-completeness with dedekind-completeness.  For example, while $\R$ is the unique (up to isomorphism) dedekind-complete totally-ordered field, there are cauchy complete totally-ordered field distinct from $\R$.  In brief, the example will turn out to be the cauchy completion of $\R (x)$ (which cannot be dedekind-complete because otherwise $\R (x)$ would embed into $\R$; see \cref{exm3.2.13}), but we will have to wait until the next chapter to be more precise about this.
\end{rmk}
\begin{proof}
To show that $\lambda \mapsto a_\lambda$ converges, we first have to find some number $a_\infty\in \R$ which we think is going to be the limit of the net.  Our guess, of course, is going to be $a_\infty\coloneqq \lim \sup _\lambda a_\lambda$.  We know that the net $\lambda \mapsto a_\lambda$ is bounded (\cref{prp3.3.28}), so that $a_\infty \in \R$ is finite.

We wish to show that $\lim _\lambda a_\lambda =a_\infty$.  So, let $\varepsilon >0$.  First of all, choose $\lambda _0$ so that there is some $\varepsilon$-ball $B_{\varepsilon}$ such that
\begin{equation}\label{3.3.45}
\left\{ a_\lambda :\lambda \geq \lambda _0\right\} \subseteq B_{\varepsilon}.\footnote{This is the definition of cauchy.}
\end{equation}
We will use this later.

Now recall the definition of $\lim \sup$:
\begin{equation}
a_\infty \coloneqq \lim _\lambda u_\lambda \coloneqq \lim _{\lambda _0}\left( \sup _{\lambda \geq \lambda _0}\{ a_\lambda \}\right) ,
\end{equation}
so there is some $\lambda _0'$ such that, whenever $\lambda \geq \lambda _0'$,
\begin{equation}\label{3.3.47}
u_\lambda -a_\infty <\varepsilon .
\end{equation}
Redefine $\lambda _0$ to be the maximum of $\lambda _0$ and $\lambda _0'$.  This way, both \eqref{3.3.45} and \eqref{3.3.47} will hold for $\lambda \geq \lambda _0$.

On the other hand, by \cref{prp1.4.11}, there is some $\lambda '\geq \lambda \geq \lambda _0$ such that
\begin{equation}
u_\lambda -\varepsilon <a_{\lambda '}\leq u_\lambda .
\end{equation}
In particular,
\begin{equation}
u_\lambda -a_{\lambda '}<\varepsilon .
\end{equation}
Hence,
\begin{equation}
\abs{a_{\lambda '}-a_\infty}=\abs{(a_{\lambda '}-u_\lambda )+(u_\lambda -a_\infty )}\leq \abs{a_{\lambda '}-u_\lambda}+\abs{u_\lambda -a_\infty}<\varepsilon +\varepsilon =2\varepsilon .
\end{equation}
We're almost done.  The only problem with this is that $\lambda '$ is a specific index.  We need this inequality to hold for all $\lambda$ sufficiently large, not just a single $\lambda '$.  Fortunately, the cauchyness can do this for us:  it follows from \eqref{3.3.45} that, whenever $\lambda \geq \lambda _0$, $\abs{a_\lambda -a_{\lambda '}}<2\varepsilon$.  Hence, for $\lambda \geq \lambda _0$,
\begin{equation}
\abs{a_\lambda -a_\infty}=\abs{(a_\lambda -a_{\lambda '})+(a_{\lambda '}-a_\infty)}\leq \abs{a_\lambda -a_{\lambda '}}+\abs{a_{\lambda '}-a_\infty}<2\varepsilon +2\varepsilon =4\varepsilon .
\end{equation}
\end{proof}
\begin{rmk}
You'll note that the final inequality the proof ended with was $\abs{a_\lambda -a_\infty}<4\varepsilon$, in contrast to the inequality (implicitly) in the definition of convergence (\cref{dfn3.3.8}) $\abs{a_\lambda -a_\infty}<\varepsilon$.  This of course makes no difference because $\frac{\varepsilon}{4}>0$ is just as arbitrary as $\varepsilon>0$.  Some people actually go to the trouble of doing the proof and then going back and changing all the $\varepsilon$s to $\frac{\varepsilon}{n}$s just so that the final inequality has an $\varepsilon$ in it.  This is completely unnecessary.  Don't waste your time.
\end{rmk}
\end{thm}

We showed above in \cref{exm3.3.39} that the sequence $m\mapsto s_m\coloneqq \sum _{k=0}^m\frac{1}{k!}$ is cauchy.  The completeness of $\R$ that we just established then tells us that this sequence converges, and so now, finally, we can define
\begin{equation}
\e \coloneqq \lim _m\sum _{k=0}^m\tfrac{1}{k!}.
\end{equation}
But let's not.  Given what we currently know, it would seem like we would just be pulling this definition out of our ass.  This series is not why $\e$ matters.  In fact, I would argue that $\e$ itself doesn't matter---it is the function $\exp$ that arises naturally in calculus (being the unique (up to scalar multiples) nonzero function equal to its own derivative).  Thus, we refrain from `officially' defining $\e$ until we have defined the exponential, which of course we won't be able to do until we know about differentiation.

Nevertheless, it would be nice to `officially' know that there is at least some real number that is not rational.

\subsubsection{Square-roots}

I mentioned way back right before \cref{sbs1.4.2} \nameref{sbs1.4.2} that often people introduce the real numbers so that we can take square-roots of numbers, but that this logic is a bit silly, because if that was the objective, then we should really be extending our number system from $\Q$ to $\A$, not from $\Q$ to $\R$.  That being said, even though our justification for the introduction of the reals was really so that we can do calculus (take limits), it does turn out that we do get square-roots of all nonnegative reals.  This should be viewed more as ``icing on the cake'' instead of the raison d'\^{e}tre.

\begin{prp}\label{prp3.3.59}
Let $x\geq 0$, define $a_0\coloneqq 1$ and
\begin{equation}\label{3.3.64}
a_{m+1}\coloneqq \frac{1}{2}\left( a_m+\frac{x}{a_m}\right)
\end{equation}
for $m\geq 0$.  Then, $m\mapsto a_m$ converges and $(\lim _ma_m)^2=x$.
\begin{rmk}
Not only does this proposition show the existence of a number whose square is $x$, but it tells you how to compute it as well.
\end{rmk}
\begin{proof}
Let us first assume that $m\mapsto a_m$ \emph{does} converge, say to $a_\infty \coloneqq \lim _ma_m$.  Then, taking the limit of both sides of the equation \eqref{3.3.64}, we find that it must be the case that
\begin{equation}
a_\infty =\frac{1}{2}\left( a_\infty +\frac{x}{a_\infty}\right) ,
\end{equation}
and hence that
\begin{equation}
a_\infty ^2=x.
\end{equation}
Thus, if the limit exists, it converges to a nonnegative number whose square is $x$ (nonnegative because each $a_m$ is nonnegative).  Thus, we now check that in fact it converges.

We apply the Monotone Convergence Theorem.  The sequence is bounded below by $0$, so it suffices to show that it is nonincreasing.  We thus look at
\begin{equation}
a_{m+1}-a_m=\frac{1}{2}\left( a_m+\frac{x}{a_m}\right) -a_m=\frac{x-a_m^2}{2a_m}.
\end{equation}
We want this to be $\leq 0$, so it suffices to show that eventually $a_m^2\geq x$.  However, for $m\geq 1$ (so that $a_{m-1}$ makes sense),
\begin{equation}
\begin{split}
a_m^2 & =\frac{1}{4}\left( a_{m-1}+\frac{x}{a_{m-1}}\right) ^2=x+\left[ \frac{1}{4}\left( a_{m-1}^2+2x+\frac{x^2}{a_{m-1}^2}\right) -x\right] \\
& =x+\frac{1}{4}\left( a_{m-1}-\frac{x}{a_{m-1}}\right) ^2\geq x.
\end{split}
\end{equation}
\end{proof}
\end{prp}
\begin{exr}
Let $m\in \Z ^+$.  Develop an algorithm which computes $m^{\text{th}}$ roots.
\end{exr}
\begin{prp}\label{prp3.3.66}
Let $m\in \Z ^+$ and let $x\geq 0$.  Then, there is a unique nonnegative real number, $\sqrt[m]{x}$, whose $m^{\text{th}}$ power is $x$.
\begin{proof}
We have just established existence.
\begin{exr}
Show that the map $x\mapsto x^m$ is strictly increasing on $\R _0^+$.
\end{exr}
As strictly increasing (and decreasing) functions are injective, it follows that there is at most one real number whose $m^{\text{th}}$ power is $x$.
\end{proof}
\end{prp}

We would have been able to show almost from the very beginning that, if $m\in \Z ^+$ is not a perfect square, then there is no element $x\in \Q$ such that $x^2=m$.  What we haven't been able to show until just now, however, is that there \emph{is} a real number $x\in \R$ such that $x^2=m$.  We now finally check that there, indeed, there is no rational number whose square is $m$.  Thus, we will have finally established the existence of real numbers which are not rational.
\begin{prp}\label{prp3.3.68}
Let $m\in \Z ^+$.  Then, if $m$ is not a perfect square, then there is no $x\in \Q$ such that $x^2=m$.
\begin{rmk}
Thus, in particular, \cref{prp3.3.68} gives an example of a sequence that is cauchy in $\Q$ but does \emph{not} converge in $\Q$.  In other words, $\Q$ is \emph{not} cauchy complete.
\end{rmk}
\begin{proof}
Suppose that $m$ is not a perfect square.  Then, we can write $m=k^2n$ where $n$ is square-free.  It thus suffices to show that there is no rational number whose square is $n$.  We proceed by contradiction:  suppose there is some $x\in \Q$ such that $x^2=n$.  Write $x=\frac{a}{b}$ with $b>0$ and $\gcd (a,b)=1$.  Then,
\begin{equation}
a^2=nb^2,
\end{equation}
and so every prime factor of $n$ divides $a^2$, and hence $a$ (because the factor is prime).  Let $p$ be some prime factor of $n$ and write $n=pn'$ and $a=pa'$.  This gives us
\begin{equation}
p^2(a')^2=pn'b^2,
\end{equation}
and hence
\begin{equation}
p(a')^2=n'b^2
\end{equation}
As $n$ is square-free, $\gcd (p,n')=1$, and hence this equation implies that $p$ divides $b^2$, and hence divides $b$.  But then $\gcd (a,b)\geq p>1$:  a contradiction.
\end{proof}
\end{prp}

We now show that all intervals in $\R$ are exactly what you think they are (confer \cref{Interval}).  We could have done this awhile ago now, but we wanted to wait until we could definitely provide an example of an interval in $\Q$ that is \emph{not} of the form you would expect.\footnote{By ``what you would expect'', we mean something like $[a,b]$, or $[a,b)$, etc.}
\begin{prp}\label{prp3.3.70}
Let $I\subseteq \R$.  Then, $I$ is an interval iff either
\begin{enumerate}
\item \label{enm3.3.70.i}$I=[a,b]$ for $-\infty <a\leq b<\infty $,
\item \label{enm3.3.70.ii}$I=(a,b)$ for $-\infty \leq a\leq b\leq \infty$,
\item \label{enm3.3.70.iii}$I=[a,b)$ for $-\infty <a\leq b\leq \infty$, or
\item \label{enm3.3.70.iv}$I=(a,b]$ for $-\infty \leq a\leq b<\infty$,
\end{enumerate}
where $a=\inf (I)$ and $b=\sup (I)$.
\begin{rmk}
Note that when the side of the interval is open, we allow $\pm \infty$ (depending on which side it is).
\end{rmk}
\begin{rmk}
In the future, we shall simply write $I=[(a,b)]$\index[notation]{$[(a,b)]$} to indicate that each end may be either open or closed---it is a pain to keep writing out all the four cases separately.
\end{rmk}
\begin{proof}
$(\Rightarrow )$ Suppose that $I$ is an interval.  If $I$ is empty, then we have that $I=(0,0)$, and so is of the form \ref{enm3.3.70.ii}.  Otherwise, we may define $b\coloneqq \sup (I)$ and $a\coloneqq \inf (I)$.  For each $a$ and $b$ there are two possibilities: either $I$ contains $a$ or it does not (and similarly for $b$).  We do one of the four cases:  suppose that $a,b\in I$.  Then, because $I$ is an interval, we must have immediately that $[a,b]\subseteq I$.  To show the other inclusion, let $x\in I$.  We proceed by contradiction:  suppose that either $x<a$ or $x>b$.  Both cases are similar, so let us just assume that $x>b$.  Then, $b$ is no longer an upper-bound for $I$:  a contradiction.  Therefore, $I\subseteq [a,b]$, and so $I=[a,b]$.

\blankline
\noindent
$(\Leftarrow )$ Suppose that $I$ is of the form \ref{enm3.3.70.i}--\ref{enm3.3.70.iv}.  Recall that the definition of an interval (\cref{Interval}) is that, for any $x_1,x_2\in I$ with $x_1\leq x_2$, then $x_1\leq x\leq x_2$ implies that $x\in I$.  As each of these forms satisfies this property (for trivial reasons), $I$ is an interval.
\end{proof}
\end{prp}
Of course, we mentioned above, this is \emph{not} true in $\Q$.
\begin{exm}[An interval in $\Q$ not of the form \cref{prp3.3.70}(\ref{enm3.3.70.i}--\ref{enm3.3.70.iv})]\label{exm3.3.71}
Define $I\coloneqq [0,\sqrt{2}]\cap \Q \subseteq \Q$.  It follows from the fact  that $[0,\sqrt{2}]$ is an interval in $\R$ that $I$ is an interval in $\Q$ (because it satisfies the defining condition in \cref{Interval}).  If it were of the form $[a,b]$ (or open-end variations of this), then we would have to have $I=[0,b]$ or $I=[0,b)$.  From the definition of $I$, however, it follows that we must have $b^2=2$, and so there is no such $b$ (in $\Q$).
\end{exm}

\subsubsection{`Density' of $\Q ^{\comp}$ in $\R$}

We mentioned back when we showed the `density' of $\Q$ in $\R$ (\cref{thm3.2.14}) that it was also true that $\Q ^{\comp}$ was `dense' in $\R$.  At the time, however, we could not even construct a real number that we could prove was irrational.  Now, however, we have done so, and so we can return to the issue of the `density' of $\Q ^{\comp}$ in $\R$.
\begin{thm}[`Density' of $\Q ^{\comp}$ in $\R$]\label{thm3.3.76}
Let $a,b\in \R$.  Then, if $a<b$, then there exists $c\in \Q ^{\comp}$ such that $c\in (a,b)$.
\begin{proof}
We have just shown that $\Q ^{\comp}$ is nonempty, so let $x\in \Q ^{\comp}$.  By `density' of $\Q$, we may take $a'\in \Q \cap (a-x,b-x)$.  Then, $a'+x\in (a,b)$, and furthermore, $a'+x\eqqcolon q$ must be irrational, because if it were rational, then $x=q-a'$ would be rational:  a contradiction.
\end{proof}
\end{thm}

\subsubsection{Counter-examples}

In this small subsubsection, we present a couple of surprising counter-examples.
\begin{exm}\label{exm3.3.73}
For $m,n\in \Z ^+$, define\footnote{This is an example of a case where we would have to reindex by $1$, so that we don't divide by $0$, were we forced to take $m,n\in \N$.  See the first remark in our definition of a sequence, \cref{dfn3.3.4}.}
\begin{equation}
a_{m,n}\coloneqq \frac{\tfrac{1}{m}\tfrac{1}{n}}{\tfrac{1}{m^2}+\tfrac{1}{n^2}}.
\end{equation}
Then, for \emph{fixed} $n\in \Z ^+$,
\begin{equation}
\lim _ma_{m,n}=\frac{0}{0+\tfrac{1}{n^2}}=0.
\end{equation}
Similarly, for fixed $m\in \Z ^+$,
\begin{equation}
\lim _na_{m,n}=0.
\end{equation}
On the other hand, if we set $m=n$ and \emph{then} take a limit, we get
\begin{equation}
\lim _ma_{m,m}=\lim _m\left( \frac{\tfrac{1}{m^2}}{\tfrac{1}{m^2}+\tfrac{1}{m^2}}\right) =\lim _m(\tfrac{1}{2})=\tfrac{1}{2}.
\end{equation}
Thus:
\begin{textequation}
Even if there is some number $a_\infty$ such that (i) \emph{for all} $\lambda$, $\lim _\mu a_{\lambda ,\mu}=a_\infty$; and (ii) \emph{for all} $\mu$, $\lim _\lambda a_{\lambda ,\mu}=a_\infty$, it need \emph{not} be the case that $\lim _\lambda a_{\lambda ,\lambda}=a_\infty$.
\end{textequation}
This can sort of be fixed, however---see \cref{prp3.3.154}.
\end{exm}
\begin{exm}[Iterated limits need not agree]
For $m,n\in \Z ^+$, define
\begin{equation}
a_{m,n}\coloneqq \frac{\tfrac{1}{m}}{\tfrac{1}{m}+\tfrac{1}{n}}.
\end{equation}
Then,
\begin{equation}
\lim _ma_{m,n}=\frac{0}{0+\tfrac{1}{n}}=0,
\end{equation}
and hence
\begin{equation}
\lim _n\left( \lim _ma_{m,n}\right) =0.
\end{equation}
On the other hand,
\begin{equation}
\lim _na_{m,n}=\frac{\tfrac{1}{m}}{\tfrac{1}{m}+0}=1,
\end{equation}
and hence
\begin{equation}
\lim _m\left( \lim _na_{m,n}\right) =1.
\end{equation}
Thus:
\begin{textequation}
If it possible for both iterated limits, $\lim _\lambda \left( \lim _\mu a_{\lambda ,\mu}\right)$ and $\lim _\mu \left( \lim _\lambda a_{\mu ,\lambda}\right)$, to exist and \emph{not} agree.
\end{textequation}
\begin{rmk}
This is actually incredibly important because it is often really tempting to interchange limits, especially if they might be hidden implicitly in something like a derivative, but this is in general \emph{just plain wrong}.
\end{rmk}
\end{exm}

\subsection{Series}\label{sbs3.3.5}

As you probably know from calculus, a series is an `infinite sum'.  In other words, it is a limit of a finite sum:
\begin{dfn}[Series]
Let $m\mapsto a_m$ be a sequence and define $s_m\coloneqq \sum _{k=0}^ma_k$.  Then,
\begin{equation}
\sum _{k\in \N}a_k\coloneqq \sum _{k=0}^\infty a_k\coloneqq \lim _m\sum _{k=0}^ma_k
\end{equation}
is a \emph{series}\index{Series} and the $s_m$s are the \emph{partial sums}\index{Partial sums} of this series.\footnote{Of course this limit need not exist; the use of $\sum _{k\in \N}a_k$ here is just an abuse of notation.}  The series is said to \emph{converge} iff the sequence $m\mapsto s_m$ does and similarly for \emph{diverge}.  The series \emph{converges absolutely}\index{Absolute convergence} iff $\sum _{k=0}^m\abs{a_k}$ converges.  The series \emph{converges conditionally}\index{Conditional convergence} iff it converges but not absolutely.
\end{dfn}

\begin{exr}[Geometric series]\index{Geometric series}
Let $\abs{a}<1$.  Show that $\sum _{m\in \N}a^m=\frac{1}{1-a}$.
\end{exr}

You'll probably recall several tests from calculus that allow us to determine whether or not a given series converges.  One of the primary goals of this section is to prove several of these tests.

Perhaps the first thing you should check is whether or not the terms of the series go to $0$, if only because it (often) takes no time to check at all.
\begin{prp}
Let $m\mapsto a_m$ be a sequence.  Then, if $\sum _{k\in \N}a_k$ converges, then $\lim _ma_m=0$.
\begin{proof}
Suppose that $\sum _{k\in \N}a_k$ converges.  Let $\varepsilon >0$.  Then, the sequence of partial sums is cauchy, there is some $m_0\in \N$ such that, whenever $m_0\leq m\leq n$, it 
\begin{equation}
\left| \sum _{k=m+1}^na_k\right| =\left| \sum _{k=0}^na_k-\sum _{k=0}^ma_k\right| <\varepsilon .
\end{equation}
As this hold \emph{for all} $m\leq n$, we may take $n\coloneqq m+1$, in which case this inequality reduces to
\begin{equation}
\abs{a_{m+1}-0}=\abs{a_{m+1}}<\varepsilon .
\end{equation}
Hence, by definition, $\lim _ma_m=0$.
\end{proof}
\end{prp}
Thus, if the terms do not go to $0$, you can immediately conclude that the series does not converge.
\begin{prp}[Absolute convergence implies convergence]
Let $m\mapsto a_m$ be a sequence and suppose that $\sum _{k\in \N}\abs{a_k}$ exists.  Then, $\sum _{k\in \N}a_k$ exists.
\begin{proof}
To show that $\sum _{k\in \N}a_k$ exists, we show that the sequences of partials sums $m\mapsto s_m\coloneqq \sum _{k=0}^ma_k$ is cauchy.  Take $m\leq n$.  Then,
\begin{equation}\label{3.3.61}
\left| \sum _{k=0}^na_k-\sum _{k=0}^ma_k\right| =\left| \sum _{k=m+1}^na_k\right| \leq \sum _{k=m+1}^n\abs{a_k}=\left| \sum _{k=0}^n\abs{a_k}-\sum _{k=0}^m\abs{a_k}\right| .
\end{equation}
Now we are essentially done.  Do you see why?

Let $\varepsilon >0$ and choose $m_0$ so that whenever $m_0\leq m\leq n$ it follows that
\begin{equation}
\left| \sum _{k=0}^n\abs{a_k}-\sum _{k=0}^m\abs{a_k}\right| <\varepsilon .
\end{equation}
Then, it follows from \eqref{3.3.61} that whenever $m_0\leq m\leq n$ that
\begin{equation}
\left| \sum _{k=0}^na_k-\sum _{k=0}^ma_k\right| <\varepsilon ,
\end{equation}
which shows that the sequence of partial sums is cauchy, and hence the series converges.
\end{proof}
\end{prp}

The next test we present is the \emph{Alternating Series Test}.
\begin{dfn}[Alternating series]
A series $\sum _{k\in \N}a_k$ is \emph{alternating}\index{Alternating series} iff $\sgn (a_{k+1})=-\sgn (a_k)$ for all $k\in \N$.
\begin{rmk}
In this case, we may write $a_k=\pm (-1)^kb_k$ where $b_k\geq 0$ and the $\pm$ is determined depending on the sign of the first term.
\end{rmk}
\end{dfn}
\begin{prp}[Alternating Series Test]
Let $m\mapsto a_m$ be a nonincreasing sequence that converges to $0$.  Then, $\sum _{m\in \N}(-1)^ma_m$ converges.
\begin{proof}
The first thing to notice is that
\begin{equation}
\sum _{k=m}^n(-1)^{k-m}a_k=a_m-(a_{m+1}-a_{m+2})-(a_{m+3}-a_{m+4})-\cdots -(a_{n-1}-a_n)\leq a_m
\end{equation}
because each $a_k-a_{k+1}\geq 0$.\footnote{Depending on whether $m-n$ is even or odd, the last term might have instead just be $-a_n$ instead of $-(a_{n-1}-a_n)$.  Either way, the same inequality holds.}

Using this, we see that
\begin{equation}
\left| \sum _{k=0}^n(-1)^ka_k-\sum _{k=0}^m(-1)^ka_k\right| =\left| \sum _{k=m+1}^n(-1)a_k\right| \leq a_{m+1}.
\end{equation}
Now that the partial sums are cauchy follows from the fact that $\lim _ma_m=0$.
\end{proof}
\end{prp}
\begin{prp}[Comparison Test]\index{Comparison Test}
Let $m\mapsto a_m$ and $m\mapsto b_m$ be eventually nonnegative sequences such that $a_m\leq b_m$.  Then,
\begin{enumerate}
\item \label{enm3.3.87.i}if $\sum _ma_m$ diverges, then $\sum _mb_m$ diverges, and
\item \label{enm3.3.87.ii}if $\sum _mb_m$ converges, then $\sum _ma_m$ converges, and furthermore, $\sum _{m\in \N}a_m\leq \sum _{m\in \N}b_m$.
\end{enumerate}
\begin{proof}
\ref{enm3.3.87.i} follows from the fact that the operation of taking limits (in this case, applied to the partial sums) is nondecreasing (see \cref{exr3.3.30}).
\begin{exr}
Prove \ref{enm3.3.87.ii}.
\end{exr}
\end{proof}
\end{prp}
\begin{prp}[Limit Comparison Test]\label{prpLimitComparisonTest}\index{Limit Comparison Test}
Let $m\mapsto a_m$ and $m\mapsto b_m$ be eventually nonnegative sequences such that $\lim _m\left| \frac{a_m}{b_m}\right|$ converges to a nonzero number.  Then, $\sum _{m\in \N}a_m$ converges iff $\sum _{m\in \N}b_m$ converges.
\begin{proof}
Define $L\coloneqq \lim _m\left| \frac{a_m}{b_m}\right| >0$.  Let $\varepsilon >0$ be less than $L$, and choose $m_0\in \N$ such that, whenever $m\geq m_0$, it follows that
\begin{equation}
\left| \tfrac{a_m}{b_m}-L\right| <\varepsilon ,
\end{equation}
that is,
\begin{equation}
-\varepsilon <\tfrac{a_m}{b_m}-L<\varepsilon ,
\end{equation}
or rather
\begin{equation}
b_m(L-\varepsilon )<a_m<b_m(L+\varepsilon ).
\end{equation}
Note that $L-\varepsilon >0$.  It follows from the Comparison Test applied to the first inequality that, if $\sum _{m\in \N}a_m$ converges, then $\sum _{m\in \N}b_m$ converges.  Likewise, it follows from the second inequality that, if $\sum _{m\in \N}b_m$ converges, then $\sum _{m\in \N}a_m$ converges.
\end{proof}
\end{prp}
\begin{prp}[Root test]\index{Root Test}
Let $m\mapsto a_m$ be a sequence.  Then, if $\lim \sup _m\abs{a_m}^{\frac{1}{m}}<1$, then $\sum _{m\in \N}a_m$ converges absolutely; if $\lim \sup _m\abs{a_m}^{\frac{1}{m}}>1$, then $\sum _{m\in \N}a_m$ diverges.
\begin{proof}
Suppose that $u\coloneqq \lim \sup _m\abs{a_m}^{\frac{1}{m}}<1$.  Let $\varepsilon >0$ be less than $1-u>0$.  Then, there is some $m_0$ such that, whenever $m\geq m_0$, it follows that
\begin{equation}
\sup _{n\geq m}\{ \abs{a_n}^{\frac{1}{n}}\} -u<\varepsilon ,
\end{equation}
so that $\sup _{n\geq m}\{ \abs{a_n}^{\frac{1}{n}}\} <\varepsilon +u<1$, so that $\abs{a_n}^{\frac{1}{n}}<r\coloneqq \varepsilon +u<1$, so that $\abs{a_n}<r^n$ for all $n\geq m_0$.  It follows that $\sum _{m\in \N}a_m$ converges absolutely by the comparison test.

\begin{exr}
Prove the case where $\lim \sup _m\abs{a_m}^{\frac{1}{m}}>1$.
\end{exr}
\end{proof}
\end{prp}
\begin{exm}[Harmonic series]
Consider the series $\sum _{m\in \Z ^+}(-1)^{m+1}\frac{1}{m}$ with terms $a_m\coloneqq (-1)^{m+1}\frac{1}{m}$.  This is called the \emph{alternating harmonic series}\index{Alternating harmonic series}, whereas the series of absolute values $\sum _{m\in \Z ^+}\frac{1}{m}$ is the \emph{harmonic series}\index{Harmonic series}.  We seem immediately from the Alternating Series Test that the alternating harmonic series converges.  In fact, it converges to $\ln (2)$, though we don't know that yet (we don't even know what $\ln (2)$ is!).  The harmonic series itself though diverges (so that the alternating harmonic series is conditionally convergent).  To see this, we apply the Comparison Test:
\begin{equation}
\begin{split}
\sum _{m\in \Z ^+}\tfrac{1}{m} & =1+\tfrac{1}{2}+\tfrac{1}{3}+\tfrac{1}{4}+\tfrac{1}{5}+\tfrac{1}{6}+\tfrac{1}{7}+\tfrac{1}{8}+\cdots \\
& \geq 1+\tfrac{1}{2}+\left( \tfrac{1}{4}+\tfrac{1}{4}\right) +\left( \tfrac{1}{8}+\tfrac{1}{8}+\tfrac{1}{8}+\tfrac{1}{8}+\tfrac{1}{8}\right) =1+\tfrac{1}{2}+\tfrac{1}{2}+\tfrac{1}{2}+\cdots =\infty .
\end{split}
\end{equation}
\begin{rmk}
If you take a string and fix its endpoints, then there are only countably many `fundamental' frequencies that the string can vibrate at (fundamental in the sense that any way in which the string might vibrate can be written as a sum of the fundamental frequency solutions).  These are called the \emph{harmonics} and the wavelength of every harmonic is of the form $\frac{1}{m}2L$, where $L$ is the length of the string.  This is where the term ``harmonic'' comes from (or so it seems).
\end{rmk}
\end{exm}

\subsubsection{The uncountability of the real numbers}

We mentioned at the end of \cref{CardinalityAndCountability} \nameref{CardinalityAndCountability} that $\abs{\R}=2^{\aleph _0}$ but at the time we did not know enough to prove it.  We now return to this.
\begin{thm}
$\abs{\R}=2^{\aleph _0}$.
\begin{proof}
To prove this, we apply the Bernstein-Cantor-Schr\"{o}der Theorem (\cref{thm1.1.26}).  Thus, we wish to construct in injection from $2^{\N}$ to $\R$ and an injection from $\R$ to $2^{\N}$.  By \cref{exrA.1.26x}, we may replace $2^{\N}$ with $\{ 0,1\} ^{\N}$ in this statement, and hence in return with $\{ 0,2\} ^{\N}$ (you will see why we do this momentarily).

The set $\{ 0,2\} ^{\N}$ is just the collection of sequences $m\mapsto a_m$ with $a_m \in \{ 0,2\}$.  We thus define $\phi :\{ 0,2\} ^{\N}\rightarrow \R$ by
\begin{equation}
\phi \left( m\mapsto a_m\right) \coloneqq \sum _{m\in \N}\tfrac{a_m}{3^m}.
\end{equation}
Intuitively, the sequence $m\mapsto a_m$ is thought of as the ternary expansion of a real number.   Switching from $\{ 0,1\}$ to $\{ 0,2\}$ was tantamount to changing from binary to ternary and not including any numbers with $1$ in their ternary expansion.  The reason for this is because, in binary, we have
\begin{equation}
1=.\bar{1}\coloneqq .111\cdots ,
\end{equation}
and so the resulting function would not be injective.  Of course, in ternary, we still have things like
\begin{equation}
1=.\bar{2}=.222\cdots ,
\end{equation}
but $1$ corresponds to the sequence $(1,0,0,0,\ldots )$, which is not an element of $\{ 0,2\} ^{\N}$, and so injectivity works out.
\begin{exr}
Show that $\phi$ is injective.
\end{exr}
It follows that $2^{\aleph _0}\leq \abs{\R}$.

As $\abs{\Q}=\aleph _0$, it suffices to replace $2^{\N}$ by $2^{\Q}$ above, and so it suffices to produce an injection from $\R$ to $2^{\Q}$, the power-set of $\Q$.  Define $\psi :\R \rightarrow 2^{\Q}$ by
\begin{equation}
\psi (x)\coloneqq \left\{ q\in \Q :q\leq x\right\} .
\end{equation}
\begin{exr}
Show that $\psi$ is injective.
\end{exr}
It follows that $\abs{\R}\leq 2^{\aleph _0}$, and hence that $\abs{\R}=2^{\aleph _0}$.
\end{proof}
\end{thm}

\subsubsection{Addition of infinitely many numbers is `noncommutative'}

We will make the title of this subsubsection precise in a moment, but for the moment we will settle for something imprecise:  there exists two convergent series $\sum _{m\in \N}a_m$ and $\sum _{m\in \N}b_m$ which converge to different values, but yet $\{ a_m:m\in \N \} =\{ b_m:m\in \N \}$.  That is, the terms themselves are the same (though in different order of course), but yet the series converge to different values!  If this is your first time studying `rigorous' mathematics, this is probably one of these ``WTF!? moments'' when you realize that sometimes mathematics can be so counter-intuitive so as to demand rigor---if we didn't require a proof, it would be very easy to dismiss `commutativity' of infinite series as ``obvious''.  In fact, it's not usually a good idea to use the word ``obvious'' in proofs at all---at best it's lazy, and at worst, it's just plain wrong.
\begin{thm}
Let $m\mapsto a_m$ be a sequence such that $\sum _{m\in \N}a_m$ converges absolutely and let $\phi :\N \rightarrow \N$ be a bijection.  Then, $\sum _{m\in \N}a_{\phi (m)}$ converges absolutely and $\sum _{m\in \N}a_m=\sum _{m\in \N}a_{\phi (m)}$.
\begin{rmk}
That $\phi$ is a bijection is the way we make precise the idea that $b_m\coloneqq a_{\phi (m)}$ is just a `rearrangement' of the original terms.  Thus, this theorem says that the rearrangement of any absolutely convergent series converges to the same value.
\end{rmk}
\begin{proof}
Define $S\coloneqq \sum _{m\in \N}a_m$.  Let $\varepsilon >0$ and choose $m_0$ such that, whenever $m\geq m_0$, it follows that $\left| \sum _{k=0}^ma_k-S\right| <\varepsilon$.  Choose $m_1$ such that, whenever $m\geq m_1$, it follows that $\sum _{k=m+1}^\infty \abs{a_k}<\varepsilon$ (we may do this because of the absolute convergence).  Replace $m_0$ by $\max \{ m_0,m_1\}$, so that, whenever $m\geq m_0$, it follows that both of these inequalities hold.  Define
\begin{equation}
n_0\coloneqq \max \left( \phi ^{-1}\left( \{ m\in \N :m\leq m_0\} \right) \right) .
\end{equation}
The set $\{ m\in \N :m\leq n_0\}$ is finite, and so $\phi$ of that set is finite, and so the definition of $n_0$ makes sense.  This definition guarantees that
\begin{equation}\label{3.3.109}
\{ 0,1,\ldots ,m_0-1,m_0\} \subseteq \phi \left( \{ 0,1,\ldots ,n_0-1,n_0\} \right) .
\end{equation}
Suppose that $n\geq n_0$.  Then,
\begin{equation}
\left| \sum _{k=0}^na_{\phi (k)}-S\right| \leq \left| \sum _{k=0}^na_{\phi (k)}-\sum _{k=0}^{m_0}a_k\right| +\left| \sum _{k=0}^{m_0}a_k-S\right| <\footnote{In the first term, because of our choice of $n_0$, every $a_k$ for $0\leq k\leq a_{m_0}$ appears somewhere in $a_{\phi (k)}$ for $0\leq k\leq n$.  Therefore, the difference s a \emph{finite} sum of terms all whose indices are at least $m_0+1$.  (The finite is crucial---we know we can rearrange finitely many terms, which is why we are able to get rid of the $\phi$ in $\sum _{k=m_0+1}^\infty \abs{a_k}$.)}\sum _{k=m_0+1}^\infty \abs{a_k}+\varepsilon <2\varepsilon ,
\end{equation}
where the inequality of the first term follows from the fact that $\sum _{k=0}^na_{\phi (k)}-\sum _{k=0}^{m_0}a_k$ contains only terms $a_k$ for $k\geq m_0+1$ (by \eqref{3.3.109}).
\end{proof}
\end{thm}
\begin{thm}
Let $m\mapsto a_m$ be a sequence such that $\sum _{m\in \N}a_m$ converges conditionally and let $-\infty \leq x\leq y\leq \infty$.  Then, there exists a bijection $\phi :\N \rightarrow \N$ such that $\lim \inf _m\sum _{k=0}^ma_{\phi (k)}=x$ and $\lim \sup _m\sum _{k=0}^ma_{\phi (k)}=y$.  In particular, taking $x=y$, there is a bijection $\phi :\N \rightarrow \N$ such that $\sum _{m\in \N}a_{\phi (m)}=x$.
\begin{rmk}
This theorem says something really quite surprising:  Given a series that converges conditionally, you can rearrange the terms to obtain a series which converges to \emph{any real number you choose whatsoever}.
\end{rmk}
\begin{proof}\footnote{Adapted from \cite{Rudin}.}
Define
\begin{equation}
a_m^+\coloneqq \tfrac{1}{2}(\abs{a_m}+a_m)\text{ and }a_m^-\coloneqq \tfrac{1}{2}(\abs{a_m}-a_m),
\end{equation}
so that
\begin{equation}
a_m=a_m^+-a_m^-\text{ and }\abs{a_m}=a_m^++a_m^-
\end{equation}
with $a_m^+,a_m^-\geq 0$.  In particular,
\begin{equation}
\infty =\sum _{m\in \N}\abs{a_m}=\sum _{m\in \N}[a_m^+-a_m^-].
\end{equation}
Thus, at least one of $\sum _{m\in \N}a_m^+$ and $\sum _{m\in \N}a_m^-$ must diverge.  On the other hand,
\begin{equation}
\sum _{m\in \N}a_m=\sum _{m\in \N}[a_m^+-a_m^-]
\end{equation}
converges, and so we cannot just have one of $\sum _{m\in \N}a_m^+$ and $\sum _{m\in \N}a_m^-$ diverge.  Thus, we must have that
\begin{equation}\label{3.3.116}
\sum _{m\in \N}a_m^+=\infty =\sum _{m\in \N}a_m^-.
\end{equation}

Note that $a_m^+=a_m$ and $a_m^-=0$ if $a_m\geq 0$, and $a_m^+=0$ and $a_m^-=-a_m$ if $a_m\leq 0$.  Thus, modulo the existence of zero terms which make no difference, every term in both of the series of \eqref{3.3.116} is a term in the original series $\sum _{m\in \N}a_m$ (up to a minus sign in the latter case).  We will build up a rearrangement of the original series $\sum _{m\in \N}a_m$ using $a_m^+$ and $-a_m^-$.

The intuition is as follows:  Because both of the series of \eqref{3.3.116} diverge, I can move as `far to the right' as I like by choosing terms of the form $a_m^+$, and likewise, I can move as `far to the left' as I like by choosing terms of the form $-a_m^-$.  We make this precise as follows.

Let $m_0$ be the smallest natural number such that
\begin{equation}
\sum _{k=0}^{m_0}a_k^+\geq y.
\end{equation}
Such an $m_0$ exists because the first series in \eqref{3.3.116} diverges.  Note that, because $m_0$ is the \emph{smallest} such number, we must have that 
\begin{equation}
\sum _{k=0}^{m_0-1}a_k^+<y,
\end{equation}
so that
\begin{equation}\label{3.3.119}
0\leq \sum _{k=0}^{m_0}a_k^+-y<a_{m_0}^+.
\end{equation}
Then, let $n_0$ be the smallest natural number such that
\begin{equation}
\sum _{k=0}^{m_0}a_k^+-\sum _{k=0}^{n_0}a_k^-\leq x.
\end{equation}
Similarly as before, we now have that
\begin{equation}\label{3.3.121}
0\leq x-\left( \sum _{k=0}^{m_0}a_k^+-\sum _{k=0}^{n_0}a_k^-\right) <a_{n_0}^-.
\end{equation}
Do this again:  let $m_1$ and $n_1$ be the smallest natural numbers such that
\begin{equation}\label{3.3.130}
\sum _{k=0}^{m_0}a_k^+-\sum _{k=0}^{n_0}a_k^-+\sum _{k=m_0+1}^{m_1}a_k^+\geq y
\end{equation}
and
\begin{equation}\label{3.3.131}
\sum _{k=0}^{m_0}a_k^+-\sum _{k=0}^{n_0}a_k^-+\sum _{k=m_0+1}^{m_1}a_k^+-\sum _{k=n_0+1}^{n_1}a_k^-\leq x
\end{equation}
respectively.  (Of course, inequalities analogous to \eqref{3.3.119} and \eqref{3.3.121} hold here as well.)  Continue this process inductively.  The series
\begin{equation}
\sum _{k=0}^{m_0}a_k^+-\sum _{k=0}^{n_0}a_k^-+\sum _{k=m_0+1}^{m_1}a_k^+-\sum _{k=n_0+1}^{n_1}a_k^-+\sum _{k=m_1+1}^{m_2}a_k^+-\sum _{k=n_1+1}^{n_2}a_k^-+\cdots
\end{equation}
is a rearrangement of the original series.  Denote the partial sums of this series by $S_m$.  Thus, the inequalities \eqref{3.3.119} and \eqref{3.3.131}, in terms of $S_m$, look like
\begin{equation}
0\leq S_m-y<a_{i_m}^+\text{ and }0\leq x-S_m<a_{j_m}^-,
\end{equation}
where $i,j:\N \rightarrow \N$ are strictly increasing functions of $m$ (these are the $m_k$ and $n_k$s).  Hence,
\begin{equation}\label{3.3.134}
0\leq \sup _{m\geq m_0}\{ S_m\}-y<\sup _{m\geq m_0}\{ a_{i_m}^+\} \text{ and }0\leq x-\inf _{m\geq m_0}\{ S_m\} <\inf _{m\geq m_0}\{ a_{j_m}^-\}
\end{equation}
Recall however that $\sum _{m\in \N}a_m$ converges.  It follows that $\lim _ma_m=0$, and so in turn $\lim _ma_m^+=0=\lim _ma_m^-$.that $\lim \inf _mS_m=x$ and $\lim \sup _mS_m=y$.  Thus, taking the limit of \eqref{3.3.134} with respect to $m_0$, we obtain $\lim \inf _mS_m=x$ and $\lim \sup _mS_m=y$.
\end{proof}
\end{thm}
\begin{exr}
Let $\sum _{k\in \N}a_k$ be a conditionally convergent series.  Does there exist a rearrangement $\phi :\N \rightarrow \N$ such that (i) $\lim \sup _m\sum _{k=0}^ma_{\phi (k)}=+\infty$ and (ii) $\lim \inf _m\sum _{k=0}^ma_{\phi (k)}=-\infty$?  Provide a proof or counter-example.
\begin{rmk}
If this statement is true, holy bitch-tits batman that is insane!  Disclaimer:  This is not meant to be a hint, just a comment (no, seriously).
\end{rmk}
\end{exr}

\subsection{Subnets and subsequences}

The concept of a subnet is \emph{almost} what you think it should be.  To help understand the concept before we go to the precise definition, let's think of what subnets of \emph{sequences} should be.

Let $a:\N \rightarrow \R$ be a sequence and let $S\subseteq \N$.  When should $\restr{a}{S}$ be a \emph{subsequence} of the original $m\mapsto a_m$?  Well certainly if $S$ is finite, we should not consider $\restr{a}{S}$ to be a subsequence---we need the indices to get arbitrarily large.  Moreover, whatever our definition of subsequence is, it should have the property that, if $m\mapsto a_m$ converges to $a_\infty$, then every subsequence of $m\mapsto a_m$ should converge to $a_\infty$ as well.  If $S$ is allowed to be finite, then of course this will not be the case.  For example, if we allowed this, $(0)$ would be a subsequence of $(0,1,1,1,\ldots )$.  This is just silly.  Thus, a key requirement is that elements of $S$ have to be able to become arbitrarily large.

Now let $\Lambda$ be a general directed set and let $\Lambda '\subseteq \Lambda$ be a subset whose elements are arbitrarily large.  (Precisely, this means that, for all $\lambda \in \Lambda$, there is some $\mu \in \Lambda '$ such that $\mu \geq \lambda$.)  One way to see we need to make this requirement is because, without this requirement, $\Lambda '$ would not itself be a directed set in general.  However, if we do require the elements of $S$ to be arbitrarily large, then $\Lambda '$ will be a directed set, and so $\restr{a}{S}$ will indeed be a net, and certainly it will turn-out that $\restr{a}{\Lambda '}$ is a subnet of $a$.  However, it is \emph{not} the case that every subnet of $\lambda \mapsto a_\lambda$ is of this form.  The reason for this is ultimately because, if we don't allow for more general subnets, then theorems we want to be true will fail to be true (see, for example, the proofs of \cref{prp3.4.56,KelleysConvergenceTheorem}).
\begin{dfn}[Subnet]\label{dfnSubnet}
Let $a:\Lambda \rightarrow \R$ be a be a net.  Then, a \emph{subnet}\index{Subnet} of $a$ is a net $b:\Lambda '\rightarrow \R$ such that
\begin{enumerate}
\item \label{enmSubnet.i}for all $\mu \in \Lambda '$, $b_\mu =a_{\lambda _\mu}$ for some $\lambda _\mu \in \Lambda$; and
\item \label{enmSubnet.ii}whenever $U\subseteq \R$ eventually contains $a$, it eventually contains $b$.
\end{enumerate}
A \emph{subsequence}\index{Subsequence} is a subnet that is a sequence.
\begin{rmk}
In other words, a subnet of a net is a net whose terms are all terms from the original net and is eventually contained in every set that originally contains the original net.
\end{rmk}
\begin{rmk}
There are at least two definitions in the literature that are distinct from this one.  Our definition is strictly weaker than both of them (see \cref{prp3.3.92,exm3.3.93,prp3.3.93,exr3.3.94}).  These definitions are not so good because they do not correspond precisely to the notion of filterings and filters (see \cref{sct4.4} \nameref{sct4.4}).\footnote{You are neither supposed to know what these are yet nor why this is significant.}  This definition also makes a few proofs easier (see, for example, the proofs of \cref{prp3.4.56,KelleysConvergenceTheorem}).
\end{rmk}
\begin{rmk}
If $b:\Lambda '\rightarrow \R$ is a subnet of a net $a:\Lambda \rightarrow \R$, then, by \cref{enmSubnet.i}, it follows that there is some function $\iota :\Lambda '\rightarrow \Lambda$ such that $b=a\circ \iota$.  However, \emph{in general there will not be a unique such function}.  This almost never matters, and it is customary to write $b_{\mu}=a_{\iota (\mu)}\coloneqq a_{\lambda _\mu}$ for some noncanonically chosen $\iota$.
\end{rmk}
\begin{rmk}
Our definition of \emph{subsequence} is slightly different than most authors (though our definition of subnet is the same).  The primary reason for this is because typically people do not introduce nets in a first analysis course, in which case the `naive' definition of subsequence, i.e., a subnet of the form $\restr{a}{S}$, works.  For them, subsequences are what we would call \emph{strict subnets} of a sequence (see \cref{dfnStrictSubnet}).  In particular, we allow for `repeats' whereas most authors will not.
\end{rmk}
\end{dfn}
As subnets of the more `naive type' are still quite important, we do given them a special name.
\begin{dfn}[Cofinal subset]
Let $S$ be a subset of a preordered set $X$.  Then, $S$ is \emph{cofinal}\index{Cofinal} iff for every $x\in X$ there is some $s\in S$ such that $s\geq x$.
\begin{rmk}
Of course, saying that a subset is cofinal is just our fancy-schmancy way of saying that the elements are arbitrarily large.
\end{rmk}
\end{dfn}
One way to see that we need elements to grow arbitrarily large is because, in a directed set, the subset will itself be directed.
\begin{dfn}[Strict subnet]\label{dfnStrictSubnet}
A \emph{strict subnet}\index{Strict subnet} of a net $a:\Lambda \rightarrow \R$ is a subnet of the form $\restr{a}{\Lambda '}:\Lambda '\rightarrow \R$ for $\Lambda '\subseteq \Lambda$ cofinal.
\begin{exr}
Let $a:\Lambda \rightarrow \R$ be a net and let $\Lambda '\subseteq \R$.  Then, if $\Lambda '$ is cofinal, show that $\restr{a}{\Lambda '}:\Lambda '\rightarrow \R$ is a subnet of $x:\Lambda \rightarrow \R$.
\begin{rmk}
Thus, our definition does in fact make sense.
\end{rmk}
\end{exr}
\begin{rmk}
One key difference between the definition of a subnet and the more `naive' definition (that is, if one were only to allow \emph{strict} subnets) is that you are allowed to repeat elements in a subnet, for example, $(0,0,1,2,3,\ldots )$ is a subnet of $(0,1,2,3,\ldots )$, but \emph{not} a strict subnet.
\end{rmk}
\end{dfn}

We mentioned in the definition of a subnet, \cref{dfnSubnet}, that there are at least two definitions of subnets in the literature that are distinct from ours.  Our definition is strictly weaker than both of these, as we now show.
\begin{prp}\label{prp3.3.92}
Let $a:\Lambda \rightarrow \R$ and $b:\Lambda '\rightarrow \R$ be nets.  Then, if there is a function $\iota :\Lambda '\rightarrow \Lambda$ such that (i) $b=a\circ \iota$ and (ii) for all $\lambda \in \Lambda$ there is some $\mu _0\in \Lambda '$ such that, whenever $\mu \geq \mu _0$, it follows that $\iota (\mu )\geq \lambda$, then $b$ is a subnet of $a$.
\begin{rmk}
Thus, in different notation, a subnet of $\lambda \mapsto a_\lambda$ is a net of the form $\mu \mapsto a_{\lambda _\mu}$, where the function $\mu \mapsto \lambda _\mu$ has the property that, for all $\lambda _0$, there is some $\mu _0$ such that, whenever $\mu \geq \mu _0$, it follows that $\lambda _\mu \geq \lambda _0$.  Note that, of course, in this case, there \emph{is} a canonically chosen $\iota :\Lambda '\rightarrow \Lambda $ (confer the remarks in the definition of a subnet, \cref{dfnSubnet}).
\end{rmk}
\begin{rmk}
This is sometimes taken as the definition of a subnet (for example, see \cite[pg.~70]{Kelley}).  Our definition is strictly weaker than this one as the following example shows.  This definition is more or less perfectly okay for almost all purposes.  The reason we have chosen the definition we have over this one (aside from the fact that it makes some proofs slightly easier), is that it is more natural in the sense that our definition is the one that corresponds to the analogous notion with filters (see \cref{sct4.4} \nameref{sct4.4}).
\end{rmk}
\begin{proof}
Suppose that there is a function $\iota :\Lambda '\rightarrow \Lambda$ such that (i) $b=a\circ \iota$ and (ii) for all $\lambda \in \Lambda$ there is some $\mu _0\in \Lambda '$ such that, whenever $\mu \geq \mu _0$, it follows that $\iota (\mu )\geq \lambda$. Let $U\subseteq \R$ eventually contain $a$.  Then, there is some $\lambda _0\in \Lambda$ such that, whenever $\lambda \geq \lambda _0$, it follows that $a_\lambda \in U$.  Let $\mu _0\in \Lambda '$ be such that, whenever $\mu \geq \mu _0$, it follows that $\iota (\mu )\eqqcolon \lambda _\mu \geq \lambda _0$.  Thus, whenever $\mu \geq \mu _0$, it follows that $a_{\lambda _\mu}\in U$, so that $\mu \mapsto a_{\lambda _\mu}$ is eventually contained in $U$, and hence is a subnet of $\lambda \mapsto a_\lambda$.
\end{proof}
\end{prp}
\begin{exm}[A subnet that would not be a subnet in the sense of \cref{prp3.3.92}]\label{exm3.3.93}
Consider the constant sequence $m\mapsto a_m\coloneqq 0$ and define $\iota :\N \rightarrow \N$ by $\iota (n)\coloneqq 0$.  Then, $n\mapsto a_{\iota (n)}=0$, and so is certainly eventually contained in every set which eventually contains $a$, and so is a subnet.  On the other hand, $\iota$ does definitely not satisfy (ii) of \cref{prp3.3.92}.  Thus, this is an example of a subnet which would not be considered a subnet if we had taken the conditions in \cref{prp3.3.92} as our definition of a subnet.
\end{exm}
And now we come to the second definition that is sometimes in the literature.
\begin{prp}\label{prp3.3.93}
Let $a:\Lambda \rightarrow \R$ and $b:\Lambda '\rightarrow \R$ be nets.  Then, if there is a function $\iota :\Lambda '\Rightarrow \Lambda$ such that (i) $b=a\circ \iota$, (ii) is nondecreasing and (ii) has cofinal image., $b:\Lambda '\rightarrow \R$ is a subnet of $a$.
\begin{rmk}
This is sometimes taken as the definition of a subnet (for example, this is currently\footnote{6 July 2015} the definition given on Wikipedia).  The definition that is sometimes given as describe in \cref{prp3.3.92} is strictly weaker than this definition (see the next exercise), and so in turn our definition is strictly weaker than this definition.  This definition also fails to exactly correspond to the analogous notion with filters, but, unlike the `definition' of \cref{prp3.3.92}, using this definition can actually make things quite a bit more difficult.\footnote{Or maybe even impossible?  I don't know because I don't use this definition.}  The problem is that we often construct a subnet of the form in \cref{prp3.3.92}, and then showing that $\iota$ is nondecreasing is an extra step at best and requires modification of the subnet at worst.  As far as I am aware, there is just no good reason to use this definition.
\end{rmk}
\begin{proof}
We apply the previous proposition.  Let $\lambda \in \Lambda$ be arbitrary.  Then, because $\iota$ has cofinal image, there is some $\mu _0\in \Lambda '$ such that $\iota (\mu _0)\geq \lambda$.  Because $\iota$ is nondecreasing, it follows that, if $\mu \geq \mu _0$, then $\iota (\mu )\geq \iota (\mu _0)\geq \lambda$.
\end{proof}
\end{prp}
\begin{exr}[A subnet that would not be a subnet in the sense of \cref{prp3.3.93}]\label{exr3.3.94}
Find a net $a:\Lambda \rightarrow \R$, a directed set $\Lambda '$ with function $\iota :\Lambda '\rightarrow \Lambda$ that has the property that, for all $\lambda \in \Lambda$, there is some $\mu _0\in \Lambda '$ such that, whenever $\mu \geq \mu_0$, it follows that $\iota (\mu )\geq \lambda$, but yet either (i) is not nondecreasing or (ii) does not have cofinal image.
\begin{rmk}
That is to say, the conditions of \cref{prp3.3.92} are strictly weaker than the conditions of \cref{prp3.3.93}.
\end{rmk}
\end{exr}

\begin{exm}
\begin{enumerate}
\item $(0,0,0,\ldots )$ is a subsequence of $(0,1,0,1,0,1,\ldots )$.
\item $(0,2,4,,\ldots )$ is a subsequence of $(0,1,2,\ldots )$.
\item $(1,1,1,\ldots )$ is \emph{not} a subsequence of $(0,1,2,\ldots )$.  (The indices that you hit in the original sequence must get arbitrarily large; here, the only index hit is $1$.)
\item $(0,1,4,9,16,\ldots )$ is a subsequence of $(0,1,2,\ldots )$.
\end{enumerate}
\end{exm}
\begin{prp}\label{prp3.3.61}
Let $\mu \mapsto a_{\lambda _\mu}$ be a subnet of a net $\lambda \mapsto a_\lambda$ which converges to $a_\infty \in \R$.  Then, $\mu \mapsto a_{\lambda _\mu}$ converges to $a_\infty$.
\begin{proof}
Let $\varepsilon >0$.  Choose $\lambda _0$ such that, whenever $\lambda \geq \lambda _0$, it follows that $\abs{a_\lambda -a_\infty}<\varepsilon$.  By the definition of a subnet, there is some $\mu _0$ such that, whenever $\mu \geq \mu _0$, it follows that $\lambda _\mu \geq \lambda _0$.  Hence, for $\mu \geq \mu _0$, it follows that $\abs{a_{\lambda _\mu}-a_\infty}<\varepsilon$.
\end{proof}
\end{prp}

The following result is important because it becomes an axiom of the convergence definition of topological spaces.
\begin{prp}\label{prp3.3.95}
Let $\lambda \mapsto a_\lambda$ be a net.  Then, $\lim _\lambda a_\lambda =a_\infty$ iff every subnet $\mu \mapsto a_{\lambda _\mu}$ has in turn a subnet itself $\nu \mapsto a_{\lambda _{\mu _\nu}}$ such that $\lim _\nu a_{\lambda _{\mu _\nu}}=a_\infty$.
\begin{proof}
$(\Rightarrow )$ This follows from two applications of the preceding proposition.

\blankline
\noindent
$(\Leftarrow )$ Suppose that every subnet $\mu \mapsto a_{\lambda _\mu}$ has in turn a subnet itself $\nu \mapsto a_{\lambda _{\mu _\nu}}$ such that $\lim _\nu a_{\lambda _{\mu _\nu}}=a_\infty$.  We proceed by contradiction:  suppose that it is not the case that $\lim _\lambda a_\lambda =a_\infty$.  Then,
\begin{textequation}[3.3.63]
there is some $\varepsilon _0>0$ such that for all $\lambda$ there is some $\mu _\lambda \geq \lambda$ such that $\abs{a_{\mu _\lambda}-a_\infty}\geq \varepsilon _0$.
\end{textequation}
Define $S\coloneqq \left\{ \mu _\lambda :\lambda \right\}$ and denote by $\lambda \mapsto a_{\mu _\lambda}$ the corresponding strict subnet, so that $\abs{a_{\mu _\lambda}-a_\infty}\geq \varepsilon _0$ for all $\lambda$.

We wish to show that $\lambda \mapsto a_{\mu _\lambda}$ has no subnet which converges to $a_\infty$.  This will be a contradiction, thereby proving the result.  To show this itself, we again proceed by contradiction:  suppose there is some subnet $\nu \mapsto a_{\mu _{\lambda _\nu}}$ of $\lambda \mapsto a_{\mu _\lambda}$ that converges to $a_\infty$.  Then, there must be some $\nu _0$ such that, whenever $\nu \geq \nu _0$, it follows that $\abs{a_{\mu _{\lambda _\nu}}-a_\infty}<\varepsilon _0$.  From this, we wish to derive a contradiction with \eqref{3.3.63} for $\lambda \coloneqq \lambda _{\nu _0}$:  a contradiction.
\end{proof}
\end{prp}
The following result, together with the above (and the fact that constant nets converge to that constant) turn out to be sufficient to characterize the topology on a topological space.  Before we present it, however, we must first define an order on a product of preordered sets.
\begin{dfn}[Product order]
Let $\mathcal{P}$ be a collection of preordered sets and let $x,y\in \prod _{P\in \mathcal{P}}P$.  Then, we define $x\leq _{\mathcal{P}}y$ iff $x_P\leq _Py_P$ for all $P\in \mathcal{P}$.
\begin{exr}
Show that $\leq _{\mathcal{P}}$ is a preorder.
\end{exr}
\end{dfn}
\begin{exr}
If $\mathcal{P}$ is a collection of directed sets, show that $(\prod _{P\in \mathcal{P}}P,\leq _{\mathcal{P}})$ is a directed set.
\end{exr}
\begin{prp}\label{prp3.3.154}
Let $I$ be a directed set\footnote{``I'' is for ``index''.} and for each $i\in I$ let $a^i:\Lambda ^i\rightarrow \R$ be a convergent net.  Then, if $(a^\infty )_\infty \coloneqq \lim _i\lim _\lambda (a^i)_\lambda$ exists, then $I\times \prod _{i\in I}\Lambda ^i\ni (i,\lambda )\mapsto (a^i)_{\lambda ^i}$ converges to $(a^\infty )_\infty$.
\begin{rmk}
In other words, if you have a `net's worth of nets' such that the iterated limit converges, then you can write this limit as a limit of a single net (which itself is formed from the `net's worth of nets').
\end{rmk}
\begin{rmk}
Note that if you insist upon working with only sequences, you haven't a chance in hell to make something like this work.  In fact, recall our counter-example (\cref{exm3.3.73}), in which we had $\lim _m (a^n)_m=0$ for all $n\in \N$, and so of course we had that $\lim _n\left( \lim _m(a^n)_m\right)$ existed (and was equal to $0$).  In fact, the same was true with $m$ and $n$ reversed.  We then hoped that $\lim _{m,m}(a^m)_m=0$, but found that this was not in fact the case.  This result tells us that you can indeed form a net from a nets worth of convergent nets that converges to the thing you would like to, the catch being that the answer is not as nice as one might have liked.
\end{rmk}
\begin{rmk}
As messy as this answer might seem, in some sense, it's the best we could do.  Can you write down any other net at all formed from just the $(a^\lambda )_{\lambda ^\mu}$s?  As all the directed sets $\Lambda ^\mu$ are in general distinct, this is essentially the simplest thing we can write down.
\end{rmk}
\begin{proof}
Suppose that $(a^\infty )_\infty \coloneqq \lim _i\lim _\lambda (a^i)_\lambda$ exists.  Let $\varepsilon >0$.  Let $i_0\in I$ be such that, whenever $i\geq i_0$, it follows that
\begin{equation}
\left| \lim _\lambda (a^i)_\lambda -(a^\infty )_\infty \right| <\varepsilon .
\end{equation}
Define
\begin{equation}
(a^i)_\infty \coloneqq \lim _\lambda (a^i)_\lambda .
\end{equation}
Let $\lambda ^i_0\in \Lambda ^i$ be such that whenever $\lambda ^i\geq \lambda ^i_0$ it follows that
\begin{equation}
\left| (a^i)_{\lambda ^i}-(a^i)_\infty \right| <\varepsilon .
\end{equation}
Then, whenever $(i,\lambda )\geq (i_0,\lambda _0)$, by definition of the product order, $i\geq i_0$ and $\lambda ^i\geq \lambda ^i_0$ for all $i\in I$, and so
\begin{equation}
\left| (a^i)_{\lambda ^i}-(a^\infty )_\infty \right| \leq \left| (a^i)_{\lambda ^i}-(a^i)_\infty \right| +\left| (a^i)_\infty -(a^\infty )_\infty \right| <\varepsilon +\varepsilon =2\varepsilon .
\end{equation}
\end{proof}
\end{prp}

\section{Basic topology of the real numbers}

Though we have not defined it yet (and will not do so until the next chapter), a \emph{topological space} is the most general context in which one can make precise the notion of \emph{continuity}.\footnote{This is arguably a slight lie, but in any case, I think it's fair to say that continuity is really the point of working with topological spaces.}  The point is, is that if continuity is something you care about, then topology in turn is something you should also care about.

\subsection{Continuity}

\begin{dfn}[Limit (of a function)]\label{dfn3.4.1}
Let $f:\R \rightarrow \R$ be a function and let $a,L\in \R$.  Then, $L$ is the \emph{limit}\index{Limit (of a function)} of $f$ at $a$ iff for every net $\lambda \mapsto x_\lambda$ such that (i) $x_\lambda \neq a$ and (ii) $\lim _\lambda x_\lambda =a$ we have $\lim _\lambda f(x_\lambda )=L$.
\begin{rmk}
If $L$ is the limit of $f$ at $a$, then we write $\lim _{x\to a}f(x)=L$\index[notation]{$\lim _{x\to a}f(x)=L$}.
\end{rmk}
\begin{rmk}
Consider the function $f:\R \rightarrow \R$ defined by
\begin{equation}
f(x)\coloneqq \begin{cases}0 & \text{if }x\neq 0 \\ 1 & \text{if }x=0\end{cases}.
\end{equation}
Then we \emph{would like} to say that $\lim _{x\to 0}f(x)=0$.  This is the motivation for imposing the constraint $x_\lambda \neq a$.  Because, for example, the constant net $\lambda \mapsto x_\lambda \coloneqq a$ does \emph{not} satisfy $\lim _\lambda f(x_\lambda )=0$.
\end{rmk}
\end{dfn}
\begin{exr}
Let $f:\R \rightarrow \R$ and let $a,L\in \R$.  Show that $\lim _{x\to a}f(x)=L$ iff for every $\varepsilon >0$ there is some $\delta >0$ such that, whenever $0<\abs{x-a}<\delta$, it follows that $\abs{f(x)-L}<\varepsilon$.
\begin{rmk}
The motivation for the condition $0<\abs{x-a}$ is the same as the motivation for the condition $x_\lambda \neq a$ in the previous definition.
\end{rmk}
\end{exr}
\begin{dfn}[Continuous (real) function]
Let $f:\R \rightarrow \R$ be a function and let $a\in \R$.  Then, $f$ is \emph{continuous}\index{Continuous (at a point)} at $a$ iff $\lim _{x\to a}f(x)=f(a)$.  $f$ is \emph{continuous}\index{Continuous} iff it is continuous at $a$ for all $a\in \R$.
\end{dfn}
\begin{exm}[Dirichlet Function]\label{exmDirichletFunction}
Define $f:\R \rightarrow \R$ by
\begin{equation}
f(x)\coloneqq \begin{cases}1 & \text{if }x\in \Q \\ -1 & \text{if }x\in \Q ^{\comp}\end{cases}.
\end{equation}
This is the \emph{Dirichlet Function}\index{Dirichlet Function}.
\begin{exr}
Where is the Dirichlet Function continuous?  Where is $\abs{f}$ continuous?
\end{exr}
\begin{rmk}
Sometimes people take $f(x)\coloneqq 0$ for $x\in \Q ^{\comp}$.
\end{rmk}
\end{exm}
\begin{exm}[Thomae Function]
Define $f:\R \rightarrow \R$ by
\begin{equation}
f(x)\coloneqq \begin{cases}\tfrac{1}{n} & \text{if }x=\tfrac{m}{n}\in \Q \text{ with }\gcd (m,n)=1,n>0 \\ 0 & \text{if }x\in \Q ^{\comp}\end{cases}.
\end{equation}
This is the \emph{Thomae Function}\index{Thomae Function}.
\begin{exr}
Show that the Thomae Function is continuous at $x\in \R$ iff $x\in \Q ^{\comp}$.  Hint:  For a fixed $n\in \Z ^+$, how many rational numbers are there in the interval $[0,1]$ with denominator smaller than $n$?
\end{exr}
\end{exm}
\begin{exr}\label{exr3.4.5}
Let $f:\R \rightarrow \R$ be a function and let $a\in \R$.  Show that the following are equivalent.
\begin{enumerate}
\item \label{enm3.4.5.i}$f$ is continuous at $a$.
\item \label{enm3.4.5.ii}For every net $\lambda \mapsto x_\lambda$ such that (i) $x_\lambda \neq a$ and (ii) $\lim _\lambda x_\lambda =a$ we have $\lim _\lambda f(x_\lambda )=f(a)$.
\item \label{enm3.4.5.iii}For every $\varepsilon >0$ there is some $\delta >0$ such that, whenever $0<\abs{x-a}<\delta$, it follows that $\abs{f(x)-f(a)}<\varepsilon$.\footnote{Note that as $\abs{f(a)-f(a)}<\varepsilon$ for all $\varepsilon >0$, the ``$<$'' in ``$0<\abs{x-a}<\delta$'' is not strictly necessary anymore.}
\item \label{enm3.4.5.iv}For every $\varepsilon >0$ there is some $\delta >0$ such that $f(B_\delta (a))\subseteq B_\varepsilon (f(a))$.
\item \label{enm3.4.5.v}For every $\varepsilon >0$ there is some $\delta >0$ such that $B_\delta (a)\subseteq f^{-1}(B_\varepsilon (f(a)))$.
\end{enumerate}
\end{exr}
\begin{exr}\label{exr3.4.12}
Show that the sum and product of continuous functions are continuous.  Show that the quotient of a continuous function by a function that never vanishes is continuous.
\end{exr}
These equivalences are nice, but they all somehow have the `disadvantage' that they make reference to the points of $\R$.\footnote{Admittedly, at this point, it should probably not be clear as to why this would be a disadvantage at all.}  There is, however, a way to characterize continuity without making reference to points at all.  This is done by the introduction of \emph{open sets}.

\subsection{Open and closed sets}

\begin{dfn}[Open set]
Let $U\subseteq \R$.  Then, $U$ is \emph{open}\index{Open (in $\R$)} iff for every $x\in U$ there is some $\varepsilon >0$ such that $B_\varepsilon (x)\subseteq U$.
\begin{rmk}
The intuition is that there is `wiggle room' around every point.
\end{rmk}
\end{dfn}
\begin{exr}\label{exr3.4.13}
Explain why $\emptyset$ is open.
\begin{rmk}
When we generalize to topological spaces, we will require that the empty-set is open.
\end{rmk}
\end{exr}
\begin{exr}\label{exr3.4.14}
Show that $\R$ is open.
\begin{rmk}
Likewise, when we generalize to topological spaces, we will also require that the entire set be open.
\end{rmk}
\end{exr}
\begin{exr}\label{exr3.4.7}
Let $x\in \R$ and $\varepsilon >0$.  Show that $B_\varepsilon (x)$ is open.
\end{exr}
The following result is incredibly important; it is neither particularly deep nor particularly difficult, but it is what is taken as the \emph{definition} of continuity when we generalize to topological spaces, even though it might not be particularly intuitive at first.
\begin{thm}\label{thm3.4.16}
Let $f:\R \rightarrow \R$.  Then, $f$ is continuous iff the preimage of every open set is open (i.e.~iff $U\subseteq \R$ open implies $f^{-1}(U)\subseteq \R$ is open).
\begin{proof}
$(\Rightarrow )$ Suppose that $f$ is continuous.  Let $U\subseteq \R$ be open and let $x\in f^{-1}(U)$.  Then, $f(x)\in U$, so because $U$ is open, there is some $\varepsilon >0$ such that $B_\varepsilon (f(x))\subseteq U$.  Then, by \cref{exr3.4.5}\ref{enm3.4.5.iv}, there is some $\delta >0$ such that $f(B_\delta (x))\subseteq B_\varepsilon (f(x))\subseteq U$.  It follows that $B_\delta (x)\subseteq f^{-1}(U)$, and so $f^{-1}(U)$ is open.

\blankline
\noindent
$(\Leftarrow )$ Suppose that the preimage of every open set is open.  Let $x\in \R$ and $\varepsilon >0$.  $B_\varepsilon (f(x))$ is open by \cref{exr3.4.7}, and so $f^{-1}(B_\varepsilon (f(x)))$ is open.  As this set contains $x$, there is some $\delta >0$ such that $B_\delta (x)\subseteq f^{-1}(B_\varepsilon (f(x)))$, and so $f(B_\delta (x))\subseteq B_\varepsilon (f(x))$.  Thus, $f$ is continuous by \cref{exr3.4.5}\ref{enm3.4.5.iv}.
\end{proof}
\end{thm}

`Dual' (but not \emph{opposite}!)\footnote{Don't make the mistake Hitler did; see \url{https://www.youtube.com/watch?v=SyD4p8_y8Kw}.} to the notion of an open set is that of a \emph{closed} set.
\begin{dfn}[Closed set]\label{dfn3.4.17}
Let $C\subseteq \R$.  Then, $C$ is \emph{closed}\index{Closed (in $\R$)} iff $C^{\comp}$ is open.
\end{dfn}
\begin{exm}[$\R$ and the empty-set]
Hopefully you saw in \cref{exr3.4.13,exr3.4.14} that both $\emptyset$ and $\R$ are open.  As $\emptyset ^{\comp}=\R$ and $\R ^{\comp}=\emptyset$, it follows that both $\emptyset$ and $\R$ are closed \emph{as well}.  In particular, it is possible for sets to be \emph{both open and closed}.  This is what we meant when we said that openness and closedness are ``dual'' but not ``opposite''.
\end{exm}
\begin{exr}
Let $f:\R \rightarrow \R$.  Then, $f$ is continuous iff the preimage of every closed set is closed.
\end{exr}
Showing that $C^{\comp}$ is open to show that $C$ is closed can in fact be a very efficient way of doing so.  Nevertheless, it would be nice to have a direct way of checking that $C$ is closed, which is why we introduce the notion of \emph{accumulation point}.
\begin{dfn}[Accumulation point]\label{dfn3.4.20}
Let $S\subseteq \R$ and let $x_0\in \R$.  Then, $x_0$ is an \emph{accumulation point}\index{Accumulation point} of $S$ iff for every $\varepsilon >0$, $B_\varepsilon (x_0)$ intersects $S$ at a point \emph{distinct from} $x_0$.
\begin{rmk}
You might think of the accumulation points of $S$ as being `infinitely close' to $S$ in some sense.
\end{rmk}
\begin{rmk}
The reason we require that it intersect at a point \emph{distinct} from $x_0$ is because some results would just fail to be true without it (for example, \cref{prp3.4.27}).  A similar problem which would arise is that the \nameref{BolzanoWeierstrassTheorem} (\cref{BolzanoWeierstrassTheorem}) would be trivial without this extra condition.
\end{rmk}
\end{dfn}
\begin{exr}\label{exr3.4.22}
Let $S\subseteq \R$ and let $x_0\in \R$.  Show that $x_0$ is an accumulation point of $S$ iff for every $\varepsilon >0$, $B_\varepsilon (x_0)$ intersects $S$ at \emph{infinitely many points}.
\begin{rmk}
Of course, there is a similar equivalence which `combines' this exercise with the previous.
\end{rmk}
\begin{rmk}
Warning:  Many of the results we prove in this section generalize perfectly to the case of a general topological space.  This is not one of them!  For example, obviously this cannot be true in a topological space which only has finitely many points.
\end{rmk}
\end{exr}
\begin{dfn}[Limit point]
Let $S\subseteq \R$ and let $x_0\in \R$.  Then, $x_0$ is a \emph{limit point}\index{Limit point} of $S$ iff there exists a net $\lambda \mapsto x_\lambda \in S$ with $x_\lambda \neq x_0$ such that $\lim _\lambda x_\lambda =x_0$.
\begin{rmk}
We choose our definition of limit point so that it agrees with the notion of accumulation point (see the next result, \cref{prp3.4.21}).  This is the reason for the requirement $x_\lambda \neq x_0$.
\end{rmk}
\end{dfn}
\begin{prp}\label{prp3.4.21}
Let $S\subseteq \R$ and let $x_0\in \R$.  Then, $x_0$ is an accumulation point of $S$ iff it is a limit point of $S$.
\begin{rmk}
If you replace ``net'' with ``sequence'' in the definition of a limit point, then this result will be \emph{false} in general!  Sequences are just fine if we restrict ourselves to $\R$, but when we generalize, this result would fail to hold if we constrained ourselves to only work with sequences.
\end{rmk}
\begin{proof}
$(\Rightarrow )$ Suppose that $x_0$ is an accumulation point of $S$.  Let $\varepsilon >0$.  Then, $B_\varepsilon (x_0)\cap S$ contains some element $x_\varepsilon$ distinct from $x_0$.  Note that the positive-real numbers $(\R ^+,\preceq)$ equipped with the \emph{reverse} of the usual ordering $\preceq$ (i.e.,$x\preceq y$ is defined to be true iff $y\leq x$) is a directed set, so that $\varepsilon \mapsto x_\varepsilon$ is a net with $x_\varepsilon \neq x_0$.  We claim that $\lim _\varepsilon x_\varepsilon =x_0$, so that $x_0$ will then be a limit point of $S$.  Let $\varepsilon >0$.  Suppose that $\delta \succeq \varepsilon$ (we are taking our `$\lambda _0$' in the definition of the limit of a net, \cref{dfn3.3.8}, to be $\varepsilon$ itself).  Then, $x_\delta \in B_\delta (x_0)$, and so $\abs{x_\delta -x_0}<\delta$.  As $\delta \succeq \varepsilon$, we have $\delta \leq \varepsilon$, and so $\abs{x_\delta -x_0}<\varepsilon$, which shows that $\lim _\varepsilon x_\varepsilon =x_0$, and so $x_0$ is a limit point of $S$.

\blankline
\noindent
$(\Leftarrow )$ Suppose that $x_0$ is a limit point of $S$.  Then, there is some net $\lambda \mapsto x_\lambda \in S$ with $x_\lambda \neq x_0$ such that $\lim _\lambda x_\lambda =x_0$.  Let $\varepsilon >0$.  Then, there is some $x_{\lambda _0}$ such that $\abs{x_{\lambda _0}-x_0}<\varepsilon$.  In other words, $x_{\lambda _0}\in B_\varepsilon (x_0)\cap S$, so that $x_0$ is an accumulation point of $S$.
\end{proof}
\end{prp}

The following result is an equivalent characterization of being a closed set and was our motivation for introducing accumulation points and limit points at all.
\begin{prp}\label{prp3.4.23}
Let $C\subseteq \R$.  Then, $C$ is closed iff it contains all its accumulation points.
\begin{proof}
$(\Rightarrow )$ Suppose that $C$ is closed.  Let $x\in \R$ be an accumulation point of $C$.  We proceed by contradiction:  suppose that $x\in C^{\comp}$.  Then, because $C^{\comp}$ is open, there is some $\varepsilon _0>0$ such that $B_{\varepsilon _0}(x)\subseteq C^{\comp}$.  But then, $B_{\varepsilon _0}(x)\cap C$ is empty:  a contradiction of the fact that $x$ is an accumulation point of $C$.  Thus, we must have had that $x\in C$.

\blankline
\noindent
$(\Leftarrow )$ Suppose that $C$ contains all its accumulation points.  Let $x\in C^{\comp}$.  Then, $x$ is not an accumulation point of $C$, and so there must be some $\varepsilon _0$ such that $B_{\varepsilon}(x)\cap C$ is empty (it cannot even contain $x$ because $x\notin C$).  In other words, it must be the case that $B_{\varepsilon}(x)\subseteq C^{\comp}$, so that $C^{\comp}$ is open.
\end{proof}
\end{prp}
\begin{crl}
Let $C\subseteq \R$.  Then, $C$ is closed iff it contains all its limit points.
\begin{rmk}
I am quite confident this is not in fact the correct etymology of the term, but you might think of closed sets as being `closed under the operation of taking limits'.
\end{rmk}
\end{crl}
\begin{exr}\label{exr3.4.27}
Let $S\subseteq \R$ be closed and bounded above.  Show that $\sup (S)\in S$.
\begin{rmk}
Similarly, of course, if $S$ is closed and bounded below, then $\inf (S)\in S$.
\end{rmk}
\end{exr}
\begin{prp}\label{prp3.4.27}
Let $m\mapsto x_m$ be a sequence that is not eventually constant and let $x\in \R$.  Then, $x$ is an accumulation point of $\{ x_m :m\in \Z ^+\}$ iff there is a subsequence of $m\mapsto x_m$ which converges to $x$.
\begin{rmk}
Note that this result would be \emph{false} if we did not require that $B_\varepsilon (x_0)$ intersect the set at a point \emph{distinct} from $x_0$ in the definition of an accumulation point (\cref{dfn3.4.20}).  For example, if we did not require this, $1$ would be an accumulation point of $(1,\frac{1}{2},\frac{1}{3},\frac{1}{4},\ldots )$, but of course no subsequence converges to $1$.
\end{rmk}
\begin{rmk}
Warning:  This is \emph{not} true if you replace ``sequence'' with ``net''.  See the following example.
\end{rmk}
\begin{rmk}
Warning:  This is \emph{not} true in general topological spaces---see \cref{exm4.2.15}.
\end{rmk}
\begin{proof}
$(\Rightarrow )$ Suppose that $x$ is an accumulation point of $\{ x_m :m\in \Z ^+\}$.  We construct a subsequence $n\mapsto x_{m_n}$ inductively that has the property that $x_{m_n}\in B_{\frac{1}{n}}(x)$ with $x_{m_n}\neq x$.  If we can do so, then $n\mapsto x_{m_n}$ will converge to $x$ by the Archimedean Property (that is, because numbers of the form $\frac{1}{n}$ can be chosen to be arbitrarily small).  Because $x$ is an accumulation point, we have that $B_1(x)\cap \{ x_m:m\in \Z ^+\}$ is contains some point distinct from $x$, and so we can take $x_{m_0}$ to be any such element.  Suppose now that we have constructed $x_{m_0},\ldots ,x_{m_n}$ and we wish to construct $x_{m_{n+1}}$.  By \cref{exr3.4.22}, not only is $B_{\frac{1}{n+1}}\cap \{ x_m:m\in \Z \}$ nonempty, but it is infinite.  Therefore, $B_{\frac{1}{n+1}}\cap \{ x_m:m>m_n\}$ is nonempty, and so we can choose $x_{m_{n+1}}$ to be any element in this set distinct from $x$.
\begin{rmk}
Note that this direction of the proof fails for nets in general.  We are implicitly using the fact that infinite subsets of $\N$ are cofinal in $\N$ (and hence give us a (strict) subsequence), and this is not true for general directed sets.
\end{rmk}

\blankline
\noindent
$(\Leftarrow )$ Suppose that there is a subsequence $m\mapsto x_m$ which converges to $x$.  Let $\varepsilon >0$.  Then, there is some $n_0$ such that, whenever $n\geq n_0$, it follows that $x_{m_n}\in B_\varepsilon (x)$.  In particular, $B_\varepsilon (x)\cap \{ x_m:m\in \Z ^+\}$ constant an element distinct from $x$ (because the sequence $m\mapsto x_m$ is not eventually constant), and so $x$ is an accumulation point of $\{ x_m:m\in \Z ^+\}$.
\end{proof}
\end{prp}
\begin{exm}[An accumulation point of a net to which no subnet converges]\label{exm3.4.29}\footnote{This example was inspired by a similar example showed to me by a student.}
Define $\R ^+\ni \lambda \mapsto x_\lambda \coloneqq \frac{1}{\lambda}$.  Then,
\begin{equation}
\{ x_\lambda :\lambda \in \R ^+\} =(0,\infty ).
\end{equation}
Thus, for example, $1\in (0,\infty )$ is an accumulation point of the net.  However, as $\lim _\lambda x_\lambda =0$, it follows that no subnet can converge to $1$.\footnote{Of course, there is nothing special about $1$---any element of $(0,\infty )$ would work just as well.}
\end{exm}

There is a `dual' (well, sort of) notion of accumulation point, though it is perhaps not quite as useful.
\begin{dfn}[Interior point]
Let $S\subseteq \R$ and let $x_0\in \R$.  Then, $x_0$ is an \emph{interior point}\index{Interior point} of $S$ iff there is some $\varepsilon _0$ such that $B_{\varepsilon _0}(x_0)\subseteq S$.
\end{dfn}
The result dual to \cref{prp3.4.23} is in the following easy exercise.
\begin{exr}\label{exr3.4.26}
Let $U\subseteq \R$.  Show that $U$ is open iff every point in $U$ is an interior point.
\end{exr}

The next couple of results are incredibly important for the same reason that \cref{thm3.4.16} (the characterization of continuity in terms of open sets) was:  they are neither deep nor difficult (in fact, they're quite trivial), but they will be defining requirements of open sets in the more general setting of topological spaces.
\begin{thm}\label{thm3.4.34}
Let $\mathcal{U}$ be a collection of open subsets of $\R$.  Then,
\begin{equation}
\bigcup _{U\in \mathcal{U}}U
\end{equation}
is open.
\begin{rmk}
In other words, an \emph{arbitrary} union of open sets is open.
\end{rmk}
\begin{proof}
Let $x\in \bigcup _{U\in \mathcal{U}}U$.  Then, $x\in U$ for some $U\in \mathcal{U}$.  Because $U$ is open, there is some $\varepsilon _0>0$ such that $B_{\varepsilon _0}(x)\subseteq U\subseteq \bigcup _{U\in \mathcal{U}}U$, and so $\bigcup _{U\in \mathcal{U}}U$ is open.
\end{proof}
\end{thm}
\begin{thm}\label{thm3.4.36}
Let $U_1,\ldots ,U_m\subseteq \R$ be open.  Then,
\begin{equation}
\bigcap _{k=1}^mU_k
\end{equation}
is open.
\begin{rmk}
In other words, the \emph{finite} intersection of open sets is open.
\end{rmk}
\begin{proof}
Let $x\in \bigcap _{k=1}^mU_k$, so that $x\in U_k$ for $1\leq k\leq m$.  Let $\varepsilon _k>0$ be such that $B_{\varepsilon _k}(x)\subseteq U_k$.  Define $\varepsilon _0\coloneqq \min \{ \varepsilon _1,\ldots ,\varepsilon _m\} >0$.  Then, $B_{\varepsilon _0}(x)\subseteq B_{\varepsilon _k}(x)\subseteq U_k$ for all $k$, and so $B_{\varepsilon _0}(x)\subseteq \bigcap _{k=1}^mU_k$, so that $\bigcap _{k=1}^mU_k$ is open.
\end{proof}
\end{thm}
\begin{exr}
Find an infinite collection of open sets whose intersection is \emph{not} open.
\end{exr}
\begin{exr}\label{exr3.4.38x}
Let $\mathcal{C}$ be a collection of closed subsets of $\R$.  Show that
\begin{equation}
\bigcap _{C\in \mathcal{C}}C
\end{equation}
is closed.
\end{exr}
\begin{exr}\label{exr3.4.40}
Let $C_1,\ldots ,C_m\subseteq \R$ be closed.  Show that
\begin{equation}
\bigcup _{k=1}^mC_k
\end{equation}
is closed.
\end{exr}

The closure of a set is the `smallest' closed set which contains it.  Likewise, the interior of a set is the `largest' open set which it contains.  The sense in which these are respectively ``smallest'' and ``largest'' is made precise by the following results.
\begin{prp}[Closure]\label{prp3.4.34}
Let $S\subseteq \R$.  Then, there exists a unique set $\Cls (S)\subseteq \R$\index[notation]{$\Cls (S)$}, the \emph{closure}\index{Closure} of $S$, that satisfies
\begin{enumerate}
\item \label{enm3.4.38.i}$\Cls (S)$ is closed;
\item \label{enm3.4.38.ii}$S\subseteq \Cls (S)$; and
\item \label{enm3.4.38.iii}if $C$ is any other closed set which contains $S$, then $\Cls (S)\subseteq C$.
\end{enumerate}
\begin{rmk}
Compare this with the definition of the integers and rationals (\cref{Integers,RationalNumbers}).
\end{rmk}
\begin{rmk}
Sometimes people denote the closure by $\bar{S}$.  We prefer the notation $\Cls (S)$ because (i) it is less ambiguous (the over-bar is used to denote many things in mathematics) and (ii) the notation $\Cls (S)$ is just slightly more descriptive.
\end{rmk}
\begin{proof}
Define
\begin{equation}\label{3.4.39}
\Cls (S)\coloneqq \bigcap _{\substack{C\subseteq \R \text{ closed } \\ S\subseteq C}}C.
\end{equation}
$\Cls (S)$ is closed because the intersection of an arbitrary collection of closed sets is closed.  $S\subseteq \Cls (S)$ because, by definition, $S$ is contained in every subset in the intersection \eqref{3.4.39}.  Let $C$ be some other closed set which contains $S$.  Then, $C$ itself appears in the intersection of \eqref{3.4.39}, and so $\Cls (S)\subseteq C$.

If $C$ is some other subset of $\R$ which satisfies \ref{enm3.4.38.i}--\ref{enm3.4.38.iii}, then, by \ref{enm3.4.38.iii} applied to $\Cls (S)$, we have that $\Cls (S)\subseteq C$.  On the other hand, by \ref{enm3.4.38.iii} applied to $C$, we have that $C\subseteq \Cls (S)$.  Thus, $\Cls (S)=C$.
\end{proof}
\end{prp}
We have a dual result for the interior.
\begin{prp}[Interior]
Let $S\subseteq \R$.  Then, there exists a unique set $\Int (S)\subseteq \R$\index[notation]{$\Int (S)$}, the \emph{interior}\index{Interior} of $S$, that satisfies
\begin{enumerate}
\item $\Int (S)$ is open;
\item $\Int (S)\subseteq S$; and
\item if $U$ is any other open set which is contained in $S$, then $U\subseteq \Int (S)$.
\end{enumerate}
\begin{rmk}
Sometimes people denote the interior by $S\degree$.  We prefer the notation $\Int (S)$ for essentially the same reasons as we prefer the notation $\Cls (S)$.
\end{rmk}
\begin{proof}
\begin{exr}
Complete this proof by yourself, using the dual proof for the closure as guidance.
\end{exr}
\end{proof}
\end{prp}
\begin{exr}\label{exr3.4.38}
Let $S\subseteq \R$.  Show that
\begin{enumerate}
\item if $S$ is closed, then $\Cls (S)=S$; and
\item if $S$ is open, then $\Int (S)=S$.
\end{enumerate}
\end{exr}
There is a relatively concrete description of the closure.
\begin{prp}
Let $S\subseteq \R$.  Then, $\Cls (S)$ is the union of $S$ and its set of accumulation points.
\begin{proof}
Let $C$ be the union of $S$ and its accumulation points.  We simply have to verify that it satisfies the axioms of the definition of the closure in \cref{prp3.4.34}.

To show that it is closed, we must show that it contains all its accumulation points.  So, let $x$ be an accumulation point of $C$.  If $x\in S$, we are done, so we may as well assume that $x\notin S$.  We show that $x$ is an accumulation point of $S$, so that $x\in C$.  Let $\varepsilon >0$.  We wish to show that $B_\varepsilon (x)$ intersects $S$ (as $x\notin S$, we know the point of intersection must be distinct from $x$).  As $x$ is an accumulation point of $C$, we know, however, that $B_\varepsilon (x)$ contains either a point of $S$ or an accumulation point of $S$ distinct from $x$.  In the former case, we are done, so let $a_\varepsilon \in B_\varepsilon (x)$ be an accumulation point of $S$ distinct from $x$.  Because $a_\varepsilon$ is an accumulation point of $S$, it must be the case that $B_\varepsilon (a_\varepsilon )$ intersects $S$ at a point $x_\varepsilon \in S$ distinct from $a_\varepsilon$.  But then, by the triangle inequality, $x_\varepsilon \in B_{2\varepsilon}(x)$.  Thus, $x$ is an accumulation point of $S$ ($x_\varepsilon$ and $x$ must be distinct because one is in $S$ and the other is not), and hence an element of $C$.  Thus, $C$ is closed.

Because any closed set must contain all its accumulation points (\cref{prp3.4.23}), it follows that $C$ must be contained in any closed set which contains $S$, and so $C=\Cls (S)$.
\end{proof}
\end{prp}
There is a dual concrete description of the interior.
\begin{prp}
Let $S\subseteq \R$.  Then, $\Int (S)$ is the set of interior points of $S$.
\begin{proof}
Because of the dual result to \cref{prp3.4.23}, namely \cref{exr3.4.26} (a set is open iff all of its points are interior points), just as in the dual proof above, it suffices to show that the set of interior points of $S$ is open.

So, let $x\in \R$ be an interior point of $S$.  Then, there is some $\varepsilon _0>0$ such that $B_{\varepsilon _0}(x)\subseteq S$.  To show that the set of interior points is open, we need to show that in fact every point of $B_{\varepsilon _0}(S)$ is an interior point of $S$.  This however follows from the fact that balls are open (\cref{exr3.4.7}).
\end{proof}
\end{prp}
\begin{exr}
Let $S\subseteq \R$.
\begin{enumerate}
\item Show that $S$ is closed iff $S=\Cls (S)$.
\item Show that $S$ is open iff $S=\Int (S)$.
\end{enumerate}
\end{exr}

In the next chapter, we will define a topological space as a set equipped with a choice of open sets.  The choice of open sets will be called a \emph{topology}.  Of course, it turns out that there are many equivalent ways to specify a topology, and one way to do this is by defining what the closure (or interior) of each set is.  The following result is important because, when we go to generalize, it will play the role of axioms which a closure (or interior) `operator' must satisfy.
\begin{thm}[Kuratowski Closure Axioms]\index{Kuratowski Closure Axioms}
Let $S,T\subseteq \R$.  Then,
\begin{enumerate}
\item \label{enm3.4.39.i}$\Cls (\emptyset) =\emptyset$;
\item \label{enm3.4.39.ii}$S\subseteq \Cls (S)$;
\item \label{enm3.4.39.iii}$\Cls (S)=\Cls \left( \Cls (S)\right)$; and
\item \label{enm3.4.39.iv}$\Cls (S\cup T)=\Cls (S)\cup \Cls (T)$.
\end{enumerate}
\begin{rmk}
Careful:  The closure of a \emph{finite} union is the union of the closures, but this does not hold in general---see the \cref{exr3.4.53} below.
\end{rmk}
\begin{proof}
The empty-set is closed, contains the empty-set, and is contained in every closed set which contains the empty-set, and hence, by definition, $\Cls (\emptyset )=\emptyset$.

\ref{enm3.4.39.ii} follows from the definition.

\ref{enm3.4.39.iii} follows from the fact that the closure of a set is closed (by definition) and the fact that the closure of a closed set is itself (\cref{exr3.4.38}).

We now prove \ref{enm3.4.39.iv}.  We show that $\Cls (S)\cup \Cls (T)$ satisfies the axioms of the closure of $S\cup T$.  $\Cls (S)\cup \Cls (T)$ is a closed set which contains $S\cup T$, and so it therefore contains $\Cls (S\cup T)$.  Let $C$ be some other closed set which contains $S\cup T$.  $C$ therefore contains $S$, and so it must contain $\Cls (S)$.  Likewise, it must contain $\Cls (T)$, and so $C$ must contain $\Cls (S)\cup \Cls (T)$.  Therefore, by definition, $\Cls (S)\cup \Cls (T)=\Cls (S\cup T)$.
\end{proof}
\end{thm}
Of course, we have the dual result for interior.
\begin{thm}[Kuratowski Interior Axioms]\index{Kuratowski Interior Axioms}
Let $S,T\subseteq \R$.  Then,
\begin{enumerate}
\item $\Int (\R )=\R$;
\item $\Int (S)\subseteq S$;
\item $\Int (S)=\Int \left( \Int (S)\right)$; and
\item $\Int (S\cap T)=\Int (S)\cap \Int (T)$.
\end{enumerate}
\begin{proof}
\begin{exr}
Complete this proof yourself, using the dual proof for the closure as guidance.
\end{exr}
\end{proof}
\end{thm}
\begin{exr}\label{exr3.4.53}
Let $\mathcal{S}\subseteq 2^{\R}$ be a collection of subsets of $\R$.  Show that the following are true.
\begin{enumerate}
\item $\bigcup _{S\in \mathcal{S}}\Cls (S)\subseteq \Cls \left( \bigcup _{S\in \mathcal{S}}S\right)$.
\item $\bigcap _{S\in \mathcal{S}}\Int (S)\supseteq \Int \left( \bigcup _{S\in \mathcal{S}}S\right)$.
\end{enumerate}
Find examples to show that we need not have equality in general.
\end{exr}

\subsection{Quasicompactness}

You will find with experience that closed intervals on the real line are particularly nice to work with.  For example, you'll probably recall from calculus that, on a closed interval, every continuous function \emph{attains} a maximum and minimum (the Extreme Value Theorem).  In particular, continuous functions are bounded on closed intervals.  This is not just true of all closed intervals, however, but is in fact true about any closed bounded subset of $\R$.

The objective then is to characterize closed bounded sets in such a way that will (i) generalize to arbitrary topological spaces and (ii) retain most of the nice properties that closed bounded sets have in $\R$.  As the notion of bounded itself does not generalize, we will need an equivalent characterization.  This characterization is what is called \emph{quasicompactness}.
\begin{dfn}[Quasicompact]
Let $S\subseteq \R$.  Then, $S$ is \emph{quasicompact}\index{Quasicompact} iff every open cover of $S$ has a finite subcover.
\begin{rmk}
Of course, we have not defined what open covers are yet, so see the next definition in case it is not obvious what this means.
\end{rmk}
\begin{rmk}
For most authors, and in fact, for probably all authors of introductory analysis books, my definition of quasicompact for them will be called just \emph{compact}.  Instead, I reserve the term \emph{compact} for spaces which are both quasicompact \emph{and} $T_2$---see \cref{Compact}.\footnote{You are not expected to know what $T_2$ means.  See the next chapter for details (specifically, \cref{T2}), though keep in mind, the details don't matter at the moment.}  As $\R$ is $T_2$, the notions of compact and quasicompact are the same for the real numbers, but in general they will differ.  To the best of my knowledge, the term quasicompact originated in algebraic geometry because it was felt that things that should \emph{not} intuitively be thought of as compact nevertheless satisfied the defining condition above.  Thus, it was decided that such spaces should be referred to as quasicompact and that instead the term compact should be reserved for spaces which are both quasicompact and $T_2$.  I prefer this terminology for two reasons:  (i) the terminology is more precise, that is, I have two terms to work with instead of just one; and (ii) I strongly feel that it is a bad idea for the terminology we use to be dependent on the mathematics we happen to be doing at the moment---``compact'' should not mean one thing today and something else tomorrow just because I decided to work on something different.  Terminology should be as consistent as possible across all of mathematics.
\end{rmk}
\end{dfn}
\begin{dfn}[Cover]
Let $S\subseteq \R$ and let $\mathcal{U}\subseteq 2^{\R}$.  Then, $\mathcal{U}$ is a \emph{cover}\index{Cover} of $S$ iff $S\subseteq \bigcup _{U\in \mathcal{U}}U$.  $\mathcal{U}$ is an \emph{open cover}\index{Open cover} iff every $U\in \mathcal{U}$ is open.  A \emph{subcover} of $\mathcal{U}$ is a subset $\mathcal{V}\subseteq \mathcal{U}$ that is still a cover of $S$.
\end{dfn}

And now of course we had better check that this condition properly characterizes ``closed and bounded'' in the real numbers, as desired.
\begin{thm}[Heine-Borel Theorem]\index{Heine-Borel Theorem}\label{HeineBorelTheorem}
Let $S\subseteq \R$.  Then, $S$ is closed and bounded iff it is quasicompact.
\begin{savenotes}
\begin{rmk}
This is equally true in $\R ^d$, but fails to hold in general topological spaces (for one thing, ``boundedness'' does not make sense in arbitrary topological spaces).  It will not even hold in $T_2$ topological spaces in which the notion of boundedness makes sense.
\end{rmk}
\begin{proof}\footnote{Proof adapted from \cite[pg.~87--89]{Abbott}.}
$(\Rightarrow )$
\Step{Assume hypotheses}
Suppose that $S$ is closed and bounded.  Let $\mathcal{U}$ be an open cover of $S$.  We proceed by contradiction:  suppose that $\mathcal{U}$ has no finite subcover.

\Step{Construct a sequence of nonincreasing closed intervals whose lengths go to $0$ and whose intersection with $S$ has no finite subcover}
Let $M>0$ be such that $S\subseteq [-M,M]$.  Then, at least one of $[-M,0]\cap S$ and $[0,M]\cap S$ must not have a finite subcover of $\mathcal{U}$, because if they both did, then the cover of $[-M,0]\cap S$ together with the cover of $[0,M]\cap S$ would comprise a finite subcover of $S$.  Without loss of generality, assume that $[0,M]\cap S$ has no finite subcover.  Then, by exactly the same logic, either $[0,\frac{M}{2}]\cap S$ or $[\frac{M}{2},M]\cap S$ has no finite subcover.  Proceeding inductively, we obtain a nonincreasing sequence of closed intervals $I_0\supseteq I_1\supseteq I_2\supseteq \cdots$ such that (i) $I_k\cap S$ has no finite subcover and (ii) the lengths of the intervals converges to $0$.

\Step{Construct an element in $S\cap \bigcap _{k\in \N}I_k$}
Note that, as $I_k\cap S$ has no finite subcover, in particular, it cannot be empty (otherwise any subset of $\mathcal{U}$ would cover it).  So, let $x_k\in I_k\cap S$.  Because the lengths of the intervals go to $0$, $m\mapsto x_m$ is a cauchy sequence, and hence converges to some $x_\infty \in \R$.  As $S$ is closed, we have in addition that $x_\infty \in S$.  We wish to show that in addition $x_\infty \in I_k$ for all $k$.  Write $I_k=[a_k,b_k]$ for $a_k\leq b_k$.  We wish to show that $a_k\leq x_\infty \leq b_k$.  We show just $x_\infty \leq b_k$ (the other inequality is similar).  We proceed by contradiction:  suppose that there is some $b_{k_0}$ such that $x_\infty >b_{k_0}$.  Then, there is some $m_0$ such that, whenever $m\geq m_0$, it follows that $x_m>b_{k_0}$.  However, for $m$ at least as large as $k_0$, we need $x_m\in I_m\subseteq I_{k_0}$, so that $x_m\leq b_{k_0}$:  a contradiction.  Therefore, $x_\infty \leq b_k$ for all $k$, and so $x_\infty \in I_k$ for all $k$.

\Step{Deduce the contradiction}
As $x_\infty \in S$, there is some $U\in \mathcal{U}$ such that $x_\infty \in U$.  Then, because the lengths of the intervals converge to $0$ and $U$ is open, there must be some $I_{m_0}$ such that $x_\infty \in I_{m_0}\subseteq U$.  But then, $\{ U\}$ is a finite open cover of $I_{m_0}\cap S$:  a contradiction.  Therefore, $S$ is quasicompact.

\blankline
\noindent
$(\Leftarrow )$
\Step{Assume hypotheses}
Suppose that $S$ is quasicompact.

\Step{Show that $S$ is bounded}
The cover $\left\{ B_M(0):M>0\right\}$ covers all of $\R$, and so certainly covers $S$.  Therefore, this is a finite subcover $\left\{ B_{M_1}(0),\ldots ,B_{M_m}(0):M_1,\ldots ,M_m>0\right\}$.  Define $M\coloneqq \max \{ M_1,\ldots ,M_m\}$.  Then, $B_M(0)$ contains each $B_{M_k}(0)$, and so contains $S$.  Therefore, $S$ is bounded.

\Step{Show that $S$ is closed}
Let $\lambda \mapsto x_\lambda \in S$ be a net converging to $x_\infty \in \R$.  We must show that $x_\infty \in S$.  We proceed by contradiction:  suppose that $x_\infty \notin S$.  Then, for each $s\in S$, because $s\neq x_\infty$, there is some $\varepsilon _s>0$ such that $x_\infty \notin B_{\varepsilon _s}(s)$.\footnote{For what it's worth, it is this step that does not work in general.}  The collection $\left\{ B_{\varepsilon _s}(s):s\in S\right\}$ is certainly an open cover of $S$, and so there is some finite subcover $\left\{ B_{\varepsilon _{s_1}}(s_1),\ldots ,B_{\varepsilon _{s_m}}(s_m)\right\}$.  Define $\varepsilon _0\coloneqq \min _{1\leq k\leq m}\left\{ \abs{s_k-x_\infty}\right\}$.  Then, in particular, there is some $x_{\lambda _0}\in B_{\varepsilon _0}(x_\infty)$.  However, by the Reverse Triangle Inequality (\cref{exr3.1.4}\ref{enm3.3.v}),
\begin{equation}
\begin{split}
\abs{x_{\lambda _0}-s_k} & =\left| (x_{\lambda _0}-x_\infty )+(x_\infty -s_k)\right| \geq \left| \left| x_{\lambda _0}-x_\infty \right| -\left| x_\infty -s_k\right| \right| \\
& \geq \left| x_\infty -s_k\right| -\left| x_{\lambda _0}-x_\infty \right| >\left| x_\infty -s_k\right| -\min _{k\leq 1\leq m}\left\{ \abs{s_k-x_\infty}\right\} \geq 0,
\end{split}
\end{equation}
and so in particular, $x_{\lambda _0}\notin B_{\varepsilon _{s_k}}(s_k)$, a contradiction of the fact that $\left\{ B_{\varepsilon _{s_1}}(s_1),\ldots ,B_{\varepsilon _{s_m}}(s_m)\right\}$ is a cover of $S$.  Therefore, $\lim _\lambda x_\lambda =x_\infty \in S$, and we are done.
\end{proof}
\end{savenotes}
\end{thm}

\subsubsection{Equivalent formulations of quasicompactness}

If for some reason you find the definition of quasicompactness in terms of open covers off-putting, there are a couple of other equivalent formulations of the concept that we present in this section.  In contrast to the \nameref{HeineBorelTheorem}, these characterizations of quasicompactness do generalize to arbitrary topological spaces.
\begin{prp}\label{prp3.4.58}
\begin{savenotes}
Let $K\subseteq \R$ and let $\mathcal{C}$ be a collection of closed subsets of $\R$.  Then, $K$ is quasicompact iff whenever every finite intersection of elements of $\mathcal{C}$ intersects $K$, the entire intersection $\bigcap _{C\in \mathcal{C}}C$ also intersects $K$.
\begin{proof}
$(\Rightarrow )$ Suppose that $K$ is quasicompact.  Let $\mathcal{C}$ be a collection of closed subsets of $\R$ that has the property that the intersection of any finite number of elements of $\mathcal{C}$ intersects $K$.  We proceed by contradiction:  suppose that $\bigcap _{C\in \mathcal{C}}C$ does not intersect $K$.  Then, $\left( \bigcap _{C\in \mathcal{C}}C\right) ^{\comp}=\bigcup _{C\in \mathcal{C}}C^{\comp}$ contains $K$,\footnote{Recall De Morgan's Laws (\cref{DeMorgansLaws}).} and therefore the collection $\mathcal{U}:=\left\{ C^{\comp}:C\in \mathcal{C}\right\}$ constitutes an open cover of $K$.  Because $K$ is quasicompact, it follows that there is some finite subcover $C_1^{\comp}\cup \cdots \cup C_m^{\comp}\supseteq K$.  But then $C_1\cap \cdots \cap C_m$ does not intersect $K$:  a contradiction.  Therefore, $\bigcap _{C\in \mathcal{C}}C$ intersects $K$.

\blankline
\noindent
$(\Leftarrow )$
\begin{exr}
Prove the converse.
\end{exr}
\end{proof}
\end{savenotes}
\end{prp}
\begin{thm}\label{prp3.4.56}
Let $K\subseteq \R$.  Then, $K$ is quasicompact iff every net $\lambda \mapsto a_\lambda \in K$ has a subnet that converges to a limit in $K$.
\begin{rmk}
This is yet another result that will not hold in general if you replace the word ``net'' with ``sequence'' (though it will hold in $\R$).
\end{rmk}
\begin{proof}
$(\Rightarrow )$ Suppose that $K$ is quasicompact.  Let $\Lambda \ni \lambda \mapsto a_\lambda \in K$ be a net.  Define
\begin{equation}
C_\lambda :=\Cls \left( \left\{ a_\mu :\mu \geq \lambda \right\} \right) 
\end{equation}
and $\mathcal{C}:=\left\{ C_\lambda :\lambda \in \Lambda \right\}$.
\begin{exr}
Show that the intersection of finitely many $C_\lambda$s is nonempty.
\end{exr}
By the previous characterization of quasicompactness, it follows that $\bigcap _{\lambda \in \Lambda}C_\lambda$ intersects $K$, so let $x\in K$ be in this intersection.

Then,
\begin{textequation}[3.4.62]
for every $\varepsilon >0$ and for every $\mu$, there is some $a_{\lambda _{\varepsilon ,\mu}}\in B_\varepsilon (x)$ with $\lambda _{\varepsilon ,\mu}\geq \mu$.
\end{textequation}
Define
\begin{equation}
\Lambda '\coloneqq \left\{ (\varepsilon ,\lambda ):\varepsilon \in \R ^+,\lambda :a_\lambda \in B_{\varepsilon}(x)\right\} .
\end{equation}
We order $\R ^+$ with the reverse of the usual ordering.  Then, $\R ^+\times \Lambda$ is directed.  We verify that $\Lambda '$ is likewise directed.  Let $\varepsilon _1,\varepsilon _2>0$ and let $\lambda _1,\lambda _2$ be such that $a_{\lambda _k}\in B_{\varepsilon _k}(x)$.  Take $\varepsilon \coloneqq \min \{ \varepsilon _1,\varepsilon _2\}$ and let $\lambda _3$ be at least as large as $\lambda _1$ and $\lambda _2$.  By \eqref{3.4.62}, there is some $\lambda \geq \lambda _3$ such that $a_\lambda \in B_\varepsilon (x)$.  Then, $(\varepsilon ,\lambda )\in \Lambda '$ and $(\varepsilon ,\lambda )\geq (\varepsilon _k,\lambda _k)$, so that $\Lambda '$ is directed.

Now, for $(\varepsilon ,\mu )\in \Lambda '$, pick some $\lambda _{\varepsilon ,\mu}$ such that (i) $a_{\lambda _{\varepsilon ,\mu}}\in B_\varepsilon (x)$ and (ii) $\lambda _{\varepsilon ,\mu}\geq \mu$.  Of course $(\varepsilon ,\mu )\mapsto a_{\varepsilon ,\mu}$ converges to $x$, but we still need to check that it is a subnet.

Let $\lambda _0$ be an arbitrary index.  Then, if $(\varepsilon ,\mu )\geq (1,\lambda _0)$, it follows that $\lambda _{\varepsilon ,\mu}\geq \mu \geq \lambda _0$, and so, by definition, this is indeed a subnet.

\blankline
\noindent
$(\Leftarrow )$ Suppose that every net $\lambda \mapsto a_\lambda$ has a subnet that converges to a limit in $K$.  Let $\mathcal{C}$ be a collection of closed sets such that the intersection of finitely many elements of $\mathcal{C}$ intersects $K$.  Let $\tilde{\mathcal{C}}$ be the collection of all finite subsets of $\mathcal{C}$ ordered by inclusion, so that it is indeed a directed set.  For each element $\tilde{C}\in \tilde{\mathcal{C}}$, let $a_{\tilde{C}}\in \bigcap _{C\in \tilde{C}}C$.  By hypothesis, this has a subnet $\mu \mapsto a_{\tilde{C}_\mu}$ that converges to $x\in K$.  We wish to show that $x\in \bigcap _{C\in \mathcal{C}}C$.  To show this, it suffices to show that $\mu \mapsto a_{\tilde{C}_\mu}$ is eventually in each $C\in \mathcal{C}$.  However, for $C_0\in \mathcal{C}$, $\{ C_0\} \in \tilde{\mathcal{C}}$, and so there there is some $\mu _0$ such that that for all $\mu \geq \mu _0$ it follows that $\tilde{C}_\mu \geq \{ C_0\}$.  In other words, for all $\mu \geq \mu _0$, it follows that $C_0\in \tilde{C}_\mu$, and as $a_{\tilde{C}_\mu}\in \bigcap _{C\in \tilde{C}_\mu}C$, it in particular follows that $a_{\tilde{C}_\mu}\in C_0$ for all $\mu \geq \mu _0$.  That is, $\mu \mapsto a_{\tilde{C}_\mu}$ is eventually in $C_0$.
\end{proof}
\end{thm}
\begin{crl}[Bolzano-Weierstrass Theorem]\index{Bolzano-Weierstrass Theorem}\label{BolzanoWeierstrassTheorem}
Every eventually bounded net has a convergent subnet.
\begin{proof}
Every eventually bounded net is eventually contained in some closed interval $[-M,M]$, and so, every eventually bounded net has a subnet which is contained (not \emph{eventually} contained) in $[-M,M]$.  $[-M,M]$ is quasicompact by the \nameref{HeineBorelTheorem} (\cref{HeineBorelTheorem}).  By the subnet characterization of quasicompactness (the previous theorem), it follows that this net has a convergent subnet.
\end{proof}
\end{crl}
\begin{crl}
Bounded infinite sets have accumulation points.
\begin{rmk}
This is sometimes also called the Bolzano-Weierstrass Theorem.
\end{rmk}
\begin{proof}
Any infinite set will have a sequence of distinct points contained in it.  If the set is bounded, this sequence will be bounded, and so by the Bolzano-Weierstrass Theorem has a convergent subnet.  The limit of this subnet is an accumulation point of the set (\cref{prp3.4.27}).
\end{proof}
\end{crl}

And now we finally return to the result we mentioned way back in \cref{3.3.52} which we now have the tools to prove.
\begin{prp}
Let $\mu \mapsto a_{\lambda _\mu}$ be a subnet of a net $\lambda \mapsto a_\lambda$.  Then,
\begin{equation}
\lim \inf _\lambda a_\lambda \leq \lim \inf _\mu a_{\lambda _\mu}\leq \lim \sup _\mu a_{\lambda _\mu}\leq \lim \sup _\lambda a_\lambda .
\end{equation}
\begin{proof}
We already know the middle inequality holds from \cref{exr3.3.50}.  We prove the $\lim \sup$ inequality holds; the other is similar.

Let $\lambda _0$ be an arbitrary index, and let $\mu _0$ be such that, whenever $\mu \geq \mu _0$, it follows that $\lambda _\mu \geq \lambda _0$.  Then,
\begin{equation}
\left\{ a_{\lambda _\mu}:\mu \geq \mu _0\right\} \subseteq \left\{ a_\lambda :\lambda \geq \lambda _0\right\} ,
\end{equation}
and hence
\begin{equation}
\sup _{\mu \geq \mu _0}\{ a_{\lambda _\mu}\} \leq \sup _{\lambda \geq \lambda _0}\left\{ a_\lambda \right\} .
\end{equation}
It follows from the definition of the limit superior \eqref{3.3.50} and the Order Limit Theorem (\cref{exr3.3.30}) that $\lim \sup _\mu a_{\lambda _\mu}\leq \lim _\lambda a_\lambda$.
\end{proof}
\end{prp}
\begin{prp}\label{prp3.3.52}
Let $\lambda \mapsto a_\lambda$ be a net.  Show that $\lambda \mapsto a_\lambda$ converges iff $\lim \sup _\lambda a_\lambda =\lim \inf _\lambda a_\lambda$  is finite, and in this case, $\lim _\lambda a_\lambda$ is equal to this common value.
\begin{proof}
$(\Rightarrow )$ Suppose that $\lambda \mapsto a_\lambda$ converges.  Then, it is eventually bounded, and so both $\lim \sup _\lambda a_\lambda$ and $\lim \inf _\lambda a_\lambda$ are finite.  Define $u\coloneqq \lim \sup _\lambda a_\lambda$ and $l\coloneqq \lim \inf _\lambda a_\lambda$.  Let $\varepsilon >0$ and choose $\lambda _0$ such that, whenever $\lambda \geq \lambda _0$, it follows that
\begin{equation}
\sup _{\mu \geq \lambda}\{ a_\mu \} -u<\varepsilon \text{ and }l-\inf _{\mu \geq \lambda}\{ a_\mu \} <\varepsilon .
\end{equation}
Adding these two inequality, we find that, for $\lambda \geq \lambda _0$,
\begin{equation}
l-u<2\varepsilon +\left( \inf _{\mu\geq \lambda}\{ a_\mu \} -\sup _{\mu \geq \lambda}\{ a_\mu \}\right) .
\end{equation}
Now define $a_\infty \coloneqq \lim _\lambda a_\lambda$ and choose $\lambda _0'$ such that, whenever $\lambda \geq \lambda _0'$, we have that $\abs{a_\lambda -a_\infty}<\varepsilon$.  In other words,
\begin{equation}
a_\lambda -a_\infty <\varepsilon \text{ and }a_\infty -a_\lambda <\varepsilon .
\end{equation}
Taking the $\sup$ of this inequality (and using the fact that $\sup (-S)=-\inf (S)$, we find
\begin{equation}
\sup _{\mu \geq \lambda}\{ a_\mu \} <\varepsilon +a_\infty \text{ and }-\inf _{\mu \geq \lambda}\{ a_\mu \} <\varepsilon -a_\infty .
\end{equation}
Pick something larger than both $\lambda _0$ and $\lambda _0'$ and change notation so that this new larger thing is called $\lambda _0$.  Thus, now, for $\lambda \geq \lambda _0$, both inequalities hold, and so
\begin{equation}
l-u<2\varepsilon +\left( (\varepsilon -a_\infty) +(\varepsilon -a_\infty )\right) =4\varepsilon .
\end{equation}
As $\varepsilon$ was arbitrary, we have $l=u$.

\blankline
\noindent
$(\Leftarrow )$ Suppose that $\lim \sup _\lambda a_\lambda =\lim \inf _\lambda a_\lambda$  is finite.  Define $L\coloneqq \lim \sup _\lambda a_\lambda =\lim \inf _\lambda a_\lambda$.  We show that every subnet of $\lambda \mapsto a_\lambda$ has in turn a subnet which converges to $L$ (see \cref{prp3.3.95}.  So, let $\mu \mapsto a_{\lambda _\mu}$ be a subnet.  As $L$ is finite, the original net is bounded, and so certainly $\mu \mapsto a_{\lambda _\mu}$ is bounded, and hence, by the Bolzano-Weierstrass Theorem, it has in turn a convergent subnet $\nu \mapsto a_{\lambda _{\mu _\nu}}$.  But then,
\begin{equation}
L\coloneqq \lim \inf _\lambda a_\lambda \leq \lim _\nu a_{\lambda _{\mu _\nu}}\leq \lim \sup _\lambda a_\lambda \eqqcolon L
\end{equation}
and so $\lim _na_{\lambda _{\mu _\nu}}=L$, where we have applied both the $(\Rightarrow )$ direction of this result as well as the previous proposition
\end{proof}.
\end{prp}

\section{Summary}

This has been a rather long chapter and we have covered many different properties of the real numbers.  For convenience, we summarize here some of the main points we have covered.
\begin{enumerate}
\item The real numbers are the unique (up to isomorphism of totally-ordered fields) nonzero dedekind complete totally-ordered field.
\item The real numbers are cauchy complete.
\item The real numbers have cardinality $2^{\aleph _0}>
\aleph _0$.
\item Nondecreasing/nonincreasing nets bounded above/bounded below converge (Monotone Convergence Theorem).
\item The real numbers are archimedean (the natural numbers are unbounded).
\item Every bounded net in $\R$ has a convergent subnet (Bolzano-Weierstrass Theorem).
\item A subset of $\R$ is closed and bounded iff it is quasicompact (\nameref{HeineBorelTheorem}).
\end{enumerate}


\chapter{Basics of general topology}

\section{The definition of a topological space}

We have mentioned topological spaces several times throughout the notes already, and finally we turn to studying general topological spaces themselves.  I said before that I think it is fair to say that the purpose of topological spaces is to introduce the most general context as possible in which one can talk about continuity.  The motivating result in this regard is \cref{thm3.4.16}, which says that a function is continuous iff the preimage of every open set is continuous.  The idea is then to axiomatize the notion of open set:  a topological space will be a set $X$ equipped with a collection of subsets $\mathcal{U}\subseteq 2^X$, called the \emph{open sets}.  Of course, if we want to be able to say anything of interest, we can't just take any old collection of subsets---we must require the collection to satisfy some conditions.  The conditions we require are those that come from \cref{thm3.4.34,thm3.4.36}, namely, that an arbitrary union and finite intersection of open sets is open.  So, without further ado, I present to you, the definition of a topological space.
\begin{dfn}[Topological space]\label{TopologicalSpace}
A \emph{topological space}\index{Topological space} is a set $X$ equipped with a collection of subsets $\mathcal{U}\subseteq 2^X$, the \emph{topology}\index{Topology} on $X$, such that
\begin{enumerate}
\item \label{enmTopologicalSpace.i}$\emptyset ,X\in \mathcal{U}$;
\item \label{enmTopologicalSpace.ii}$\bigcup _{U\in \mathcal{V}}U\in \mathcal{U}$ if $\mathcal{V}\subseteq \mathcal{U}$; and
\item \label{enmTopologicalSpace.iii}$\bigcap _{k=1}^mU_k\in \mathcal{U}$ if $U_k\in \mathcal{U}$ for $1\leq k\leq m$.
\end{enumerate}
\begin{rmk}
The elements of $\mathcal{U}$ are \emph{open sets}\index{Open (in a topological space)}.
\end{rmk}
\begin{rmk}
A subset $C\subseteq X$ is \emph{closed}\index{Closed (in a topological space)} iff $C^{\comp}$ is open.  (Recall that this is precisely the definition we gave in $\R$---see \cref{dfn3.4.17}.)
\end{rmk}
\begin{rmk}
In order to exclude stupid things like the empty topology, we require that the empty-set and the entire set are open.  (Recall that these were open in $\R$---see \cref{exr3.4.13,exr3.4.14}.)
\end{rmk}
\end{dfn}
Of course, you can specify a topology just as well by saying what the closed sets are (the open sets are then just the complements of these sets).
\begin{exr}\label{exr4.1.2}
Let $X$ be a set and let $\mathcal{C}\subseteq 2^X$ be a collection of subsets of $X$ such that
\begin{enumerate}
\item $\emptyset ,X\in \mathcal{C}$;
\item $\bigcap _{C\in \mathcal{D}}C$ if $\mathcal{D}\subseteq \mathcal{C}$; and
\item $\bigcup _{k=1}^mC_k\in \mathcal{C}$ if $C_k\in \mathcal{C}$ for $1\leq k\leq m$.
\end{enumerate}
Show that there is a unique topology on $X$ whose collection of closed sets is precisely $\mathcal{C}$.
\begin{rmk}
In other words, your collection of closed sets must be nontrivial, closed under arbitrary intersection, and closed under finite union, just as was the case in $\R$---see \cref{exr3.4.38x,exr3.4.40}.
\end{rmk}
\end{exr}
\begin{dfn}[$G_\delta$ and $F_\sigma$ sets]\label{GDeltaFSigma}
Let $S\subseteq X$ be a subset of a topological space.  Then, $S$ is a \emph{$G_\delta$ set}\index{$G_\delta$ set} iff $S$ is the countable intersection of open sets.  $S$ is an \emph{$F_\sigma$ set}\index{$F_\sigma$ set} iff $S$ i the countable union of closed sets.
\begin{rmk}
For us, these concepts will not come-up very often, so it is not imperative that you remember these terms.  They do come-up, however, and so it would be incomplete to not include them.  And of course, others use this terminology so you should at least know where to refer back to if you ever need to know.
\end{rmk}
\end{dfn}

\subsection{Bases, neighborhood bases, and generating collections}

It is often not necessary to define every single open set explicitly, but rather, only a special class of open sets that determine all the others.  For example, in the real numbers, to determine whether an arbitrary set was open, we made use of the `special' open sets $B_\varepsilon (x)$.  The idea that generalizes this notion is that of a \emph{base} for a topology.
\begin{dfn}[Base]\label{Base}
Let $X$ be a topological space and let $\mathcal{B}$ be a collection of open sets of $X$.  Then, $\mathcal{B}$ is a \emph{base}\index{Base} for the topology of $X$ iff the statement that a subset $U$ of $X$ is open is equivalent to the statement that, for every $x\in U$, there is some $B\in \mathcal{B}$ such that $x\in B\subseteq U$.
\end{dfn}
\begin{exm}
The collection $\left\{ B_\varepsilon (x):x\in \R ,\ \varepsilon >0\right\}$ is a base for the topology of $\R$ (by definition).
\end{exm}
The real reason bases are important is because they allow us to \emph{define} topologies, and so it is important to know when a collection of subsets of a set form a base for some topology.
\begin{prp}\label{prp4.1.5}
Let $X$ be a set and let $\mathcal{B}$ be a collection of subsets of $X$.  Then, there exists a unique topology for which $\mathcal{B}$ is a base iff
\begin{enumerate}
\item \label{enm4.1.4.i}$\mathcal{B}$ covers $X$; and
\item \label{enm4.1.4.ii}for every $x\in X$ and $B_1,B_2\in \mathcal{B}$ with $x\in B_1,B_2$, there is some $B_3\in \mathcal{B}$ such that $x\in B_3\subseteq B_1\cap B_2$.
\end{enumerate}
\begin{rmk}
If $X$ does not a priori come with a topology, we will still refer to any collection of sets that satisfy \ref{enm4.1.4.i}--\ref{enm4.1.4.ii} as a \emph{base}.
\end{rmk}
\begin{proof}
$(\Rightarrow )$ Suppose that there is a unique topology $\mathcal{U}$ for which $\mathcal{B}$ is a base.  We first show that $\mathcal{B}$ covers $X$.  We proceed by contradiction:  suppose there is some $x\in X$ which is not contained in any $B\in \mathcal{B}$.  Then, as $\mathcal{B}$ is a base for the topology, $X$ would not be open:  a contradiction.  Therefore, $\mathcal{B}$ covers $X$.

Now for the second property.  By the definition of a base, we have that $\mathcal{B}\subseteq \mathcal{U}$.  In particular, $B_1\cap B_2\in \mathcal{U}$, and so because $\mathcal{B}$ is a base for $\mathcal{U}$, there must be some $B_3\in \mathcal{B}$ such that $B_3\subseteq B_1\cap B_2$.

\blankline
\noindent
$(\Leftarrow )$ Suppose that (i) $\mathcal{B}$ covers $X$, and (ii) for every for every $B_1,B_2\in \mathcal{B}$ intersecting, there is some $B_3\in \mathcal{B}$ such that $B_3\subseteq B_1\cap B_2$.  We declare $U\subseteq X$ to be open iff for every $x\in U$ there is some $B\in \mathcal{B}$ with $x\in B\subseteq U$.  By the definition of bases, this was the only possibility.  We need only check that this is in fact a topology.  The empty-set is vacuously open.  $X$ is open because $\mathcal{B}$ covers $X$.  Let $\mathcal{V}$ be a collection of open sets and let $x\in \bigcup _{U\in \mathcal{V}}U$.  Then, $x\in U$ for some $U\in \mathcal{V}$, and so there is some $B\in \mathcal{B}$ such that $x\in B\subseteq U\subseteq \bigcup _{U\in \mathcal{V}}U$.  Thus, $\bigcup _{U\in \mathcal{V}}U$ is open.  Let $U_1,\ldots ,U_m$ be open and let $x\in \bigcap _{k=1}^mU_k$.  Then, there is some $B_k\in \mathcal{B}$ such that $x\in B_k\subseteq U_k$.  By \ref{enm4.1.4.ii}, there is some $B\in \mathcal{B}$ with $x\in B\subseteq B_1\cap \cdots \cap B_m\subseteq \bigcap _{k=1}^mU_k$, and so $\bigcap _{k=1}^mU_k$ is open.
\end{proof}
\end{prp}
\begin{exr}\label{exr4.1.7}
Let $X$ be a topological space and let $\mathcal{B}$ be a base for the topology on $X$.  Show that every open set is a union of elements of $\mathcal{B}$.
\end{exr}
There is a similar way of defining a topology.  Instead of specifying a base for all open sets, you specify a base for all the open sets at a point (see the following definitions) for every point.
\begin{dfn}[Neighborhood]\label{Neighborhood}
Let $X$ be a topological space and let $S\subseteq N\subseteq X$.  Then, $N$ is a \emph{neighborhood}\index{Neighborhood} of $S$ iff there is some open set $U\subseteq X$ such that $S\subseteq U\subseteq N$.  An \emph{open neighborhood}\index{Open neighborhood} of $S$ is just an open set which contains $S$.  A(n open) neighborhood of a point $x$ is a(n open) neighborhood of $\{ x\}$.
\end{dfn}
\begin{dfn}[Neighborhood base]\label{NeighborhoodBase}
Let $X$ be a topological space and for each $x\in X$ let $\mathcal{B}_x$ be a collection of neighborhoods\footnote{Not necessarily open!} of $x$.  Then, $\{ \mathcal{B}_x:x\in X\}$ is a \emph{neighborhood base}\index{Neighborhood base} for the topology of $X$ (and $\mathcal{B}_x$ is a \emph{neighborhood base} at $x$) iff the statement that a subset $U$ of $X$ is open is equivalent to the statement that, for every $x\in U$, there is some $B_x\in \mathcal{B}_x$ such that $B_x\subseteq U$.  In the case that every element of each $\mathcal{B}_x$ is open, we say that the neighborhood base is a \emph{local base}\index{Local base}.
\end{dfn}
\begin{exm}[A neighborhood base with no open sets]
Take $X\coloneqq \R$, and for $x\in X$ define $\mathcal{B}_x\coloneqq \{ D_{\varepsilon}(x):\varepsilon >0\}$.\footnote{Recall that $D_{\varepsilon}(x)\coloneqq \{ y\in \R :\abs{y-x}\leq \varepsilon \}$.}  While we have not actually defined a topology on $\R$ yet, once we do very shortly (\cref{exm3.1.21}), you can verify that this is in fact a neighborhood base, but no $D_{\varepsilon}(x)$ is open.
\end{exm}
Once again, neighborhood bases allow us to \emph{define} topologies.
\begin{prp}\label{prp4.1.8}
Let $X$ be a set and for each $x\in X$ let $\mathcal{B}_x$ be a nonempty collection of subsets of $X$ which contain $x$.  Then, there exists a unique topology for which $\mathcal{B}_x$ is a neighborhood base of $x$ iff for every $x\in X$ and $B_1,B_2\in \mathcal{B}_x$, there is some $B_3\in \mathcal{B}_x$ such that $B_3\subseteq B_1\cap B_2$.
\begin{proof}
$(\Rightarrow )$ Suppose that there exists a unique topology for which $\mathcal{B}_x$ is a neighborhood base at $x$.  Let $B_1,B_2\in \mathcal{B}_x$.  Then, $B_1$ and $B_2$ are neighborhoods of $x$, and so there are open sets $U_1\subseteq B_1$ and $U_2\subseteq B_2$ containing $x$.  Then, $U_1\cap U_2$ is open and contains $x$, and therefore, because $\mathcal{B}_x$ is a neighborhood base at $x$, there is some $B_3\in \mathcal{B}_x$ with $B_3\subseteq U_1\cap U_2\subseteq B_1\cap B_2$.

\blankline
\noindent
$(\Leftarrow )$ Suppose that for every $x\in X$ and $B_1,B_2\in \mathcal{B}_x$, there is some $B_3\in \mathcal{B}_x$ such that $B_3\subseteq B_1\cap B_2$.  We declare $U\subseteq X$ to be open iff for every $x\in U$ there is some $B\in \mathcal{B}_x$ with $x\in B\subseteq U$.  By definition of neighborhood bases, this was the only possibility.  We need only check that this is in fact a topology.  The empty-set is vacuously open.  $X$ is open because each $\mathcal{B}_x$ is nonempty.  Let $\mathcal{V}$ be a collection of open sets and let $x\in \bigcup _{U\in \mathcal{V}}U$.  Then, $x\in U$ for some $U\in \mathcal{V}$, and so there is some $B\in \mathcal{B}_x$ such that $x\in B\subseteq U\subseteq \bigcup _{U\in \mathcal{V}}U$.  Thus, $\bigcup _{U\in \mathcal{V}}U$ is open.  Let $U_1,\ldots ,U_m$ be open and let $x\in \bigcap _{k=1}^mU_k$.  Then, $x\in U_k$ for each $k$, and so there is some $B_k\in \mathcal{B}_x$ such that $x\in B_k\subseteq U_k$.  By hypothesis, there is then some $B\in \mathcal{B}_x$ with $x\in B\subseteq B_1\cap \cdots \cap B_m\subseteq \bigcap _{k=1}^mU_k$, and so $\bigcap _{k=1}^mU_k$ is open.
\end{proof}
\end{prp}
Sometimes we have a collection of sets that we would like to be open, but they do not necessarily form a base (or a neighborhood base), and so in this case we cannot just invoke \cref{prp4.1.5}.  However, what we can do is take the `smallest' topology which contains these sets.
\begin{prp}[Generating collection (of a topology)]\label{GeneratingCollection}
Let $X$ be a set and let $\mathcal{S}\subseteq 2^X$.  Then, there exists a unique topology $\mathcal{U}$ on $X$, the topology \emph{generated}\index{Generate (a topology)} by $\mathcal{S}$, such that
\begin{enumerate}
\item $\mathcal{S}\subseteq \mathcal{U}$; and
\item if $\mathcal{U}'$ is any other topology on $X$ containing $\mathcal{S}$, it follows that $\mathcal{U}\subseteq \mathcal{U}'$.
\end{enumerate}
Furthermore, the collection of all finite intersections of elements of $\mathcal{S}\cup \{ X\}$ is a base for this topology.\footnote{We throw $X$ in in case $\mathcal{S}$ did not cover $X$ (recall that bases (\cref{Base}) need to cover the space).}.  $\mathcal{S}$ is a \emph{generating collection}\index{Generating collection (of a topology)}.
\begin{rmk}
You should compare this with the definitions of the integers (\cref{Integers}), rationals (\cref{RationalNumbers}), closure (\cref{Closure}), and interior (\cref{Interior}).
\end{rmk}
\begin{proof}
Let $\mathcal{B}$ be the collection of all finite intersections of elements of $\mathcal{S}\cup \{ X\}$.  This definitely covers $X$ as $X\in \mathcal{B}$.  Furthermore, the intersection of any two elements of $\mathcal{B}$ is also an element of $\mathcal{B}$, by definition.  Therefore, there is a unique topology $\mathcal{U}$ on $X$ for which $\mathcal{B}$ is a base (\cref{prp4.1.5}).

By construction, $\mathcal{S}\subseteq \mathcal{B}\subseteq \mathcal{U}$.  On the other hand, if $\mathcal{U}'$ is any other topology for which every element of $\mathcal{S}$ is open, then, because topologies are closed under finite intersection, $\mathcal{U}'$ must contain $\mathcal{B}$, and hence it must contain $\mathcal{U}$ (because $\mathcal{U}'$ is closed under arbitrary union and every element of $\mathcal{U}$ is a union of elements of $\mathcal{B}$ (\cref{exr4.1.7})).
\end{proof}
\end{prp}
Generating collections are actually quite nice because several things can be checked just by looking at generating collections, which are generally significantly `smaller'\footnote{Though not literally in the sense of cardinality.} than the entire topology---see \cref{exr4.1.27,exr4.2.41}, and the \nameref{AlexanderSubbaseTheorem} (\cref{AlexanderSubbaseTheorem}).

\subsection{Some basic examples}

We can use the notion of a base to define the \emph{order topology}.
\begin{dfn}[Order topology]\label{OrderTopology}
Let $X$ be a totally-ordered set, and let $\mathcal{B}$ be the collection of sets of the form $(a,b)\coloneqq \left\{ x\in X:a<x<b\right\}$ for $a,b\in X$ with $a<b$.  (We also allow $a=-\infty$ and $b=+\infty$, in which case $(a,+\infty )\coloneqq \left\{ x\in X:x>a\right\}$ and similarly for $(-\infty ,b)$ and $(-\infty, +\infty)$.)
\begin{exr}
Use \cref{Base} to show that $\mathcal{B}$ is a base for a topology.
\end{exr}
The topology defined by $\mathcal{B}$ is the \emph{order topology}\index{Order topology} on $X$.
\end{dfn}
\begin{exm}[A partially-ordered set whose open intervals do not form a base]
Define
\begin{equation}
X\coloneqq \{ x,y_1,y_2,y_3,z_1,z_2\} ,
\end{equation}
and declare that
\begin{equation}
x\leq y_k\text{ and }z_1\geq y_1,y_2\text{ and }z_2\geq y_2,y_3
\end{equation}
for all $k$ ((and then take the `reflexive and transitive closure', so that, for example, we also have $x\leq z_l$, $y_k\leq y_k$, etc.)  Then,
\begin{equation}
(x,z_1)=\{ y_1,y_2\} \text{ and }(x,z_2)=\{ y_2,y_3\} ,
\end{equation}
so that
\begin{equation}
(x,z_1)\cap (x,z_2)=\{ y_2\} .
\end{equation}
However, $\{ y_2\}$ is not an interval because if we had $\{ y_2\} =(a,b)$ for $a<y_2$ and $b>y_2$, we would necessarily have $a=x$ or $a=-\infty$, and $b=z_1,z_2,+\infty$.  In all of these $6$ cases, $(a,b)$ must contain at least either $y_1$ or $y_3$ as well, and so in particular, it would not be the case that $\{ y_2\} =(a,b0$.
\begin{rmk}
This is why we only define the order topology for \emph{totally}-ordered sets.
\end{rmk}
\end{exm}
\begin{exm}[$\N$, $\Z$, $\Q$, and $\R$]\label{exm3.1.21}
$\N$, $\Z$, $\Q$, and $\R$ are all totally-ordered and so we may (and do) equip them with the order topology.
\end{exm}
The order topology on $\N$ and $\Z$ are examples of a very special topology, the `finest' one possible, namely the \emph{discrete topology}.
\begin{dfn}[Discrete topology]
Let $X$ be any set.  The \emph{discrete topology}\index{Discrete topology} is the topology in which \emph{every} subset is open.
\begin{rmk}
In other words, the topology is just the power-set of $X$.  This is a topology for tautological reasons.
\end{rmk}
\end{dfn}
\begin{prp}\label{prp4.1.11}
The order topologies on $\N$ and $\Z$ are both discrete.
\begin{proof}
We prove that the order topology on $\Z$ is discrete and leave the case of $\N$ as an exercise (a very similar argument will work).  We wish to show that every subset of $\Z$ is open.  To show this, it suffices to show that each singleton set is open because an arbitrary set is going to be a union of singletons.  So, let $m\in \Z$.  To show that $\{ m\}$ is open, it suffices to find $a,b\in \Z$ such that $(a,b)=\{ m\}$ (because every element in the base of a topology is automatically open).  Take $a\coloneqq m-1$ and $b\coloneqq m+1$.  Recall (\cref{exr1.2.14}) that there is no integer between $0$ and $1$.  It follows that there is no integer between $k$ and $k+1$ for all $k\in \Z$, and so indeed, $(m-1,m+1)=\{ m\}$.
\begin{exr}
Show that the order topology on $\N$ is discrete.
\end{exr}
\end{proof}
\end{prp}
\begin{exr}
Why is the topology on $\Q$ not discrete?
\end{exr}
The discrete topology is the `finest' topology you can have (finer means more open sets).  The `coarsest' topology you can have is the \emph{indiscrete topology}.
\begin{dfn}[Indiscrete topology]\label{dfnIndiscreteTopology}
Let $X$ be any set.  The \emph{indiscrete topology}\index{Indiscrete topology} is just $\{ \emptyset ,X\}$.
\begin{rmk}
The definition of a topology requires that at least the empty-set and the entire set are open.  The indiscrete topology is when nothing else is open.  It is not particularly useful, but it can be an easy source of some counter-examples (cf.~\cref{exm4.1.20}).
\end{rmk}
\end{dfn}

\subsection{Continuity}

We now finally vastly generalize our definition of continuity.
\begin{dfn}[Continuous function]
Let $f:X\rightarrow Y$ be a function between topological spaces and let $x\in X$.  Then, $f$ is \emph{continuous}\index{Continuous (at a point)} iff the preimage of every open neighborhood of $f(x)$ is an open neighborhood of $x$.  $f$ is \emph{continuous}\index{Continuous} iff $f$ is continuous at $x$ for all $x\in X$.
\end{dfn}
\begin{exr}
Show that $f:X\rightarrow Y$ is continuous iff the preimage of every open set is open.
\end{exr}
In fact, we can do a bit better than this.
\begin{exr}\label{exr4.1.27}
Let $X$ and $Y$ be topological spaces and let $\mathcal{S}$ be generate the topology of $Y$.  Show that $f$ is continuous iff $f^{-1}(S)$ is open for all $S\in \mathcal{S}$.
\begin{rmk}
In particular, this is true for $\mathcal{S}$ a \emph{base} for the topology of $Y$
\end{rmk}
\end{exr}
\begin{exr}
Let $f:X\rightarrow Y$ be any function between two topological spaces.  Show that, if $X$ has the discrete topology, then $f$ is continuous.
\begin{rmk}
In other words, every function on a discrete space is continuous.
\end{rmk}
\end{exr}

\begin{exm}[The category of topological spaces]
The category of topological spaces is the category $\Top$\index[notation]{$\Top$} whose collection of objects $\Top _0$ is the collection of all topological spaces, for every topological space $X$ and topological space $Y$ the collection of morphisms from $X$ to $Y$, $\Mor _{\Top}(X,Y)$, is precisely the set of all continuous functions from $X$ to $Y$, composition is given by ordinary function composition, and the identities of the category are the identity functions.
\begin{exr}
Show that the composition of two continuous functions is continuous.
\begin{rmk}
Note that this is something you need to check in order for $\Top$ to actually form a category ($\Mor _{\Top}(X,Y)$ needs to be closed under composition).  You also need to verify that the identity function is continuous, but this is trivial (the preimage of a set is itself, so\textellipsis ).
\end{rmk}
\end{exr}
\end{exm}
\begin{dfn}[Homeomorphism]\label{Homeomorphism}
Let $f:X\rightarrow Y$ be a function between topological spaces.  Then, $f$ is a \emph{homeomorphism}\index{Homeomorphism} iff $f$ is an isomorphism in $\Top$.
\end{dfn}
\begin{exr}
Show that a function is a homeomorphism iff (i) it is bijective, (ii) it is continuous, and (iii) its inverse is continuous.
\end{exr}
\begin{exr}\label{exr3.1.34}
Find an example of a function that is bijective and continuous, but not a homeomorphism.
\begin{rmk}
Contrast this with, for example, isomorphisms in $\Grp$.  It follows immediately from the definition that a function between groups is an isomorphism in $\Grp$ iff (i) it is bijective, (ii) it is a homomorphism, and (iii) its inverse is a homomorphism.  However, by \cref{exrA.2.11x}, if the original function is a homomorphism, then we get that its inverse is a homomorphism for free, so we only need to actually check (i) and (ii).
\end{rmk}
\end{exr}
\begin{dfn}[Embedding]\label{Embedding}
An \emph{(topological) embedding}\index{Embedding} is a function that is a homeomorphism onto its image.
\begin{rmk}
In other words, an embedding is like a homeomorphism with the exception that it might not be surjective.
\end{rmk}
\begin{rmk}
The word ``topological'' is in parentheses because the word ``embedding'' is used in other contexts (for analogous, but different, things---see, for example, the remark in \cref{prpB.10}).  This is not unlike how we have homomorphisms of rings and homomorphisms of groups.
\end{rmk}
\end{dfn}
\begin{exr}
Show that $f:X\rightarrow Y$ is a homeomorphism iff it is (i) bijective, (ii) it is continuous, and (iii) its inverse is continuous.
\end{exr}
\begin{exr}
What is an example of a function that is (i) bijective, (ii) continuous, but (iii) does not have continuous inverse.
\begin{rmk}
Note the contrast with isomorphisms in the category $\Ring$, for example.  If a function between rings is a bijective homomorphism, then its inverse is automatically a homomorphism.
\end{rmk}
\end{exr}

There is a useful result that is applicable in general whenever you want to check that a ``piece-wise'' function is continuous.
\begin{prp}[Pasting Lemma]\index{Pasting Lemma}\label{PastingLemma}
Let $X$ and $Y$ be topological spaces, let $C_1,C_2\subseteq X$ be closed, and let $f_1:C_1\rightarrow Y$ and $f_2:C_2\rightarrow Y$ be continuous.  Then, if $\restr{f_1}{C_1\cap C_2}=\restr{f_2}{C_1\cap C_2}$, then the function $f:C_1\cup C_2\rightarrow Y$ defined by
\begin{equation}
f(x)\coloneqq \begin{cases}f_1(x) & \text{if }x\in C_1 \\ f_2(x) & \text{if }x\in C_2\end{cases}
\end{equation}
is well-defined and continuous.
\begin{proof}
We leave this as an exercise.
\begin{exr}
Prove this yourself.
\end{exr}
\end{proof}
\end{prp}

\subsection{Some motivation}

At this point, it is reasonable for one to ask ``Why do we care about such generality in an introductory \emph{real} analysis course?  Shouldn't we only be concerned with the real numbers for now?''.  I would argue that, even in the case where you really are only interested in the real numbers, abstraction can shed light onto such a specific example.  The real numbers are many things:  a field, a totally-ordered set, a totally-ordered field, a metric space, a uniform space, a topological space, a manifold, etc..  In mathematics, we are not just concerned about \emph{what} is true, but \emph{why} things are true.  Because the real numbers are so special, having so much structure, there are many things true about them.  But by the same token, because there is so much structure, unless you go through the proofs yourself in detail, it can be difficult to recall \emph{why} things are true---does property XYZ hold for the real numbers because they are a totally-ordered field or because they are a metric space or because they are a topological vector space or because\textellipsis ?  Instead, however, if we step back and only study $\R$ as a topological space, it becomes clearer why certain things are true.  This allows us to tell easily what is true about the real numbers because they are a topological space, as opposed to what is true about the real numbers because they are a field.    For example, consider the following.
\begin{exm}
Though we have not technically defined it yet, hopefully you are familiar with the function $\arctan$ from calculus.  $\arctan :\R \rightarrow (-\frac{\uppi}{2},\frac{\uppi}{2})$ is a homeomorphism.
\end{exm}
Thus, from the point of view of general topology, there is no distinction between $\R$ and $(0,1)$ (see the exercise below).  This is completely analogous to the sense in which there is no difference between $\R$ and $2^{\N}$ at the level of sets.  ($\R$ and $(0,1)$ are isomorphic in $\Top$, and $\R$ and $2^{\N}$ are isomorphic in $\Set$.)  Recall from the very end of \cref{chp1} \nameref{chp1}---morphisms matter.
\begin{exr}
Find a homeomorphism from $(-\frac{\uppi}{2},\frac{\uppi}{2})$ to $(0,1)$.
\end{exr}

Perhaps a good example of the more general context making it clearer \emph{why} things are true is the Intermediate Value Theorem.  Here is the `classical' statement you are probably familiar with from calculus.\footnote{See \cref{ClassicalIntermediateValueTheorem} for the proof.}
\begin{textequation}
Let $f:[a,b]\rightarrow \R$ be continuous.  Then, for all $y$ between $f(a)$ and $f(b)$ (inclusive) there is some $x\in [a,b]$ such that $f(x)=y$.
\end{textequation}
Compare that with the more general statement.\footnote{See \cref{IntermediateValueTheorem} for the proof.}
\begin{textequation}
The continuous image of a connected set is connected.
\end{textequation}
I think it's fair to say that the latter is much more elegant, and perhaps even easier to understand.\footnote{Admittedly, the proof to go from the general statement to the `classical' statement is not completely trivial (a subset of $\R$ is connected iff it is an interval---see \cref{thm4.5.14}), but the two results are still `morally' the same.}

\section{A review}

Quite a many things that we did with the real numbers in the last chapter carry over to general topological spaces no problem.  For convenience, we present here all definitions and theorems which carry over nearly verbatim to topological spaces.  We will try to point out when things do \emph{not} carry over identically (for example, limits no longer need be unique (\cref{exm4.1.20})).

One important thing to keep in mind is:  \emph{Open neighborhoods of a point in a general topological space play the same role that $\varepsilon$-balls did in $\R$}.  Not only do we use this to determine appropriate generalizations, but if you are feeling a bit uncomfortable with topological spaces, this should help your intuition.

For the entirety of this section, $X$ and $Y$ will be general topological spaces.  We recommend that this section be used mainly as a reference---the pedagogy of the concepts is contained in the last chapter.
\begin{dfn}[Net]
A \emph{net}\index{Net} in $X$ is a function from a nonempty directed set $(\Lambda ,\leq )$ into $X$.
\end{dfn}
\begin{dfn}[Sequence]
A \emph{sequence}\index{Sequence} is a net whose directed set is order-isomorphic (i.e.~isomorphic in $\Pre$) to $(\N ,\leq )$.
\end{dfn}
\begin{dfn}[Limit (of a net)]
Let $\lambda \mapsto x_\lambda$ be a net and let $x_\infty \in X$.  Then, $x_\infty$ is a \emph{limit}\index{Limit (of a net)} of $\lambda \mapsto x_\lambda$ iff for every open neighborhood $U$ of $x_\infty$ there is some $\lambda _0$ such that, whenever $\lambda \geq \lambda _0$, it follows that $x_\lambda \in U$.  If a net has a limit, then we say that it \emph{converges}\index{Convergence}.
\end{dfn}
\begin{exm}[Limits need not be unique]\label{exm4.1.20}
Define $X\coloneqq \{ 0,1\}$ and equip $X$ with the indiscrete topology (\cref{dfnIndiscreteTopology}).  Then, the constant net $\lambda \mapsto x_\lambda \coloneqq 0$ converges to both $0$ and $1$ (the only open neighborhood of both of these points is $X$ itself, and of course the net is eventually contained in $X$).
\end{exm}
\begin{dfn}[Subnet]\label{Subnet}
Let $x:\Lambda \rightarrow X$ be a be a net.  Then, a \emph{subnet}\index{Subnet} of $x$ is a net $y:\Lambda '\rightarrow X$ such that
\begin{enumerate}
\item for all $\mu \in \Lambda '$, $y_\mu =x_{\lambda _\mu}$ for some $\lambda _\mu \in \Lambda$; and
\item whenever $U\subseteq X$ eventually contains $x$, it eventually contains $y$.
\end{enumerate}
A \emph{strict subnet}\index{Strict subnet} of $x$ is a net of the form $\restr{x}{\Lambda'}$ for $\Lambda '\subseteq \Lambda$ cofinal.  A \emph{subsequence}\index{Subsequence} is a subnet that is a sequence.
\end{dfn}
\begin{thm}[Kelley's Convergence Axioms]\index{Kelley's Convergence Axioms}\label{KelleysConvergenceAxioms}
\begin{enumerate}
\item \label{enmKelleysConvergenceAxioms.i}Constant nets converge to that constant.
\item \label{enmKelleysConvergenceAxioms.ii}A net converges to $x_\infty \in X$ iff every subnet has in turn a subnet which converges to $x_\infty \in X$.
\item \label{enmKelleysConvergenceAxioms.iii}Let $I$ be a directed set and for each $i\in I$ let $x^i:\Lambda ^i\rightarrow X$ be a convergent net.  Then, if $(x^\infty )_\infty \coloneqq \lim _i\lim _\lambda (x^i)_\lambda$ exists, then $I\times \prod _{i\in I}\Lambda ^i\ni (i,\lambda )\mapsto (x^i)_{\lambda ^i}$ converges to $(x^\infty )_\infty$.
\end{enumerate}
\begin{rmk}
We will see later (\cref{KelleysConvergenceTheorem}) that these three properties can be used to define a topology (hence, the reason we refer to them as \emph{axioms}).
\end{rmk}
\end{thm}
\begin{dfn}[Limit (of a function)]\label{dfnLimitOfAFunction}
Let $f:X \rightarrow Y$ be a function between topological spaces, and let $x\in X$ and $y\in Y$.  Then, $y$ is the \emph{limit}\index{Limit (of a function)} of $f$ at $x$ iff for every net $\lambda \mapsto x_\lambda$ such that (i) $x_\lambda \neq x$ and (ii) $\lim _\lambda x_\lambda =x$ we have $\lim _\lambda f(x_\lambda )=y$.
\begin{rmk}
In the real numbers, this is true if you replace the word ``net'' with the word ``sequence'', but this fails in general topological spaces---see \cref{exm4.2.8}.
\end{rmk}
\end{dfn}
\begin{prp}
Let $f:X\rightarrow Y$ be a function and let $x_0\in X$.  Then, $f$ is continuous at $x_0$ iff $\lim _{x\to x_0}f(x)=f(x_0)$.  $f$ is continuous iff it is continuous at $x_0$ for all $x_0\in X$.
\end{prp}
\begin{prp}
Let $f:X\rightarrow Y$ be a function.  Then, $f$ is continuous iff the preimage of every closed set is closed.
\end{prp}
\begin{dfn}[Accumulation point]
Let $S\subseteq X$ and let $x\in X$.  Then, $x$ is an \emph{accumulation point}\index{Accumulation point} of $S$ iff every open neighborhood of $x_0$ intersects $S$ at a point distinct from $x_0$.
\begin{rmk}
If you remove the requirement that the point be distinct from $x_0$, we obtain what is usually called an \emph{adherent point}\index{Adherent point}.
\end{rmk}
\end{dfn}
\begin{dfn}[Limit point]
Let $S\subseteq X$ and let $x\in X$.  Then, $x$ is a \emph{limit point}\index{Limit point} of $S$ iff there exists a net $\lambda \mapsto x_\lambda \in S$ with $x_\lambda \neq x$ such that $\lim _\lambda x_\lambda =x$.
\end{dfn}
\begin{prp}
Let $S\subseteq X$ and let $x\in X$.  Then, $x$ is an accumulation point of $S$ iff it is a limit point of $S$.
\begin{rmk}
If you replace ``net'' with ``sequence'' in the definition of a limit point, then this result will be \emph{false} in general; see \cref{exm4.2.8x}
\end{rmk}
\end{prp}
\begin{prp}
Let $C\subseteq X$.  Then, $C$ is closed iff it contains all its accumulation points.
\end{prp}
\begin{crl}
Let $C\subseteq X$.  Then, $C$ is closed iff it contains all its limit points.
\end{crl}
We proved that (\cref{prp3.4.27}) in $\R$ that $x\in \R$ is an accumulation point of a sequence iff there was a subsequence that converged to $x$.  We also gave an example (\cref{exm3.4.29}) of how this fails (even in $\R$) for general nets.  Not only this, but this also fails to hold (for sequences) in general topological spaces.
\begin{exm}[An accumulation point of a sequence to which no subsequence converges]\label{exm4.2.15}
Define $X\coloneqq \{ x_1,x_2,x_3\}$ and
\begin{equation}
\mathcal{U}\coloneqq \left\{ \emptyset ,X,\{ x_1,x_2\} \right\} .
\end{equation}
Consider the sequence $m\mapsto a_m$ defined by
\begin{equation}
a_m\coloneqq \begin{cases}x_2 & \text{if }m=0 \\ x_3 & \text{otherwise}\end{cases}.
\end{equation}
Then, it is true that every open neighborhood of $x_1\in X$ contains an element of the set $\{ a_m:m\in \N \}$, namely $a_0\coloneqq x_2$ distinct from $x_1$.  On the other hand, no subsequence of $m\mapsto a_m$ converges to $x_1$ as it is eventually outside an open neighborhood of $x_1$ (namely the open neighborhood $\{ x_1,x_2\}$).
\end{exm}
\begin{dfn}[Interior point]
Let $S\subseteq X$ and let $x_0\in X$.  Then, $x_0$ is an \emph{interior point}\index{Interior point} of $S$ iff there is some open neighborhood $U$ of $x_0$ such that $U\subseteq S$.
\end{dfn}
\begin{prp}[Closure]\label{Closure}
Let $S\subseteq X$.  Then, there exists a unique set $\Cls (S)\subseteq X$\index[notation]{$\Cls (S)$}, the \emph{closure}\index{Closure} of $S$, that satisfies
\begin{enumerate}
\item $\Cls (S)$ is closed;
\item $S\subseteq \Cls (S)$; and
\item if $C$ is any other closed set which contains $S$, then $\Cls (S)\subseteq C$.
\end{enumerate}
\end{prp}
\begin{prp}
Let $S\subseteq X$.  Then, $\Cls (S)$ is the union of $S$ and its set of accumulation points.
\end{prp}
\begin{thm}[Kuratowski Closure Axioms]\index{Kuratowski Closure Axioms}
Let $S,T\subseteq X$.  Then,
\begin{enumerate}
\item $\Cls (\emptyset) =\emptyset$;
\item $S\subseteq \Cls (S)$;
\item $\Cls (S)=\Cls \left( \Cls (S)\right)$; and
\item $\Cls (S\cup T)=\Cls (S)\cup \Cls (T)$.
\end{enumerate}
\end{thm}
\begin{prp}[Interior]\label{Interior}
Let $S\subseteq X$.  Then, there exists a unique set $\Int (S)\subseteq \R$\index[notation]{$\Int (S)$}, the \emph{interior}\index{Interior} of $S$, that satisfies
\begin{enumerate}
\item $\Int (S)$ is open;
\item $\Int (S)\subseteq S$; and
\item if $U$ is any other open set which is contained in $S$, then $U\subseteq \Int (U)$.
\end{enumerate}
\end{prp}
\begin{prp}
Let $S\subseteq X$.  Then, $\Int (S)$ is the set of interior points of $S$.
\end{prp}
\begin{thm}[Kuratowski Interior Axioms]\index{Kuratowski Interior Axioms}
Let $S,T\subseteq X$.  Then,
\begin{enumerate}
\item $\Int (X)=X$;
\item $\Int (S)\subseteq S$;
\item $\Int (S)=\Int \left( \Int (S)\right)$; and
\item $\Int (S\cap T)=\Int (S)\cap \Int (T)$.
\end{enumerate}
\end{thm}
\begin{prp}\label{prp4.2.21}
Let $S\subseteq X$.  Then, $S$ is closed iff $S=\Cls (S)$.
\end{prp}
\begin{prp}
Let $S\subseteq X$.  Then, $S$ is open iff $S=\Int (S)$.
\end{prp}

We postponed the following counter-examples because we technically had not introduced the notion of closure in a general topological space.
\begin{exm}[A limit point that is not a sequential limit point]\label{exm4.2.8x}
Define $X\coloneqq \R$.  We equip $\R$ with a nonstandard topology, the so-called \emph{cocountable topology}\index{Cocountable topology}.\footnote{Just as cofinite means that the complement is finite, \emph{Cocountable}\index{Cocountable} means that the complement is countable.  The name of the topology here derives from the fact that the \emph{open} sets have countable complement (except the empty-set of course).}  Let $C\subseteq X$ and declare that
\begin{textequation}
$C$ is closed iff either (i) $C=X$ or (ii) $C$ is countable.
\end{textequation}
Because the finite union of countable sets is countable and an arbitrary intersection of countable sets is countable (obviously), it follows that this defines a topology on $X$.

We first show that every convergent sequence is eventually constant.  So, let $m\mapsto x_m\in X$ be sequence that converges to $x_\infty \in X$.  We proceed by contradiction:  suppose that it is not eventually constant.  Then, the set
\begin{equation}
\{ m\in \N :x_m\neq x_\infty \}
\end{equation}
is cofinal, and hence defines a subsequence $n\mapsto x_{m_n}$ that (i) converges to $x_\infty$ but (ii) is never equal to $x_\infty$.  Hence, the set $C\coloneq \{ x_{m_n}:n\in \N \}$ does not contain $x_\infty$, and so $C^{\comp}$ is an open neighborhood of $x_\infty$.  But of course $n\mapsto x_{m_n}$ is not eventually contained in $C^{\comp}$---no term of this subsequence is contained in $C^{\comp}$.  Therefore, $n\mapsto x_{m_n}$ cannot converge to $x_\infty$:  a contradiction.  Therefore, $m\mapsto x_m$ must be eventually constant.

Define $U\coloneqq \Q ^{\comp}$.  As $U^{\comp}=\Q$ is countable, $U$ is open.  We note that the closure of $U$ is all of $\R$:  no countable set can contain $U$, and so the only closed set which contains $U$ is $\R$ itself.

It thus follows that, in particular, $0$ is a limit point of $U$.  On the other hand, as every convergent sequence is eventually constant, no sequence in $U\coloneqq \Q ^{\comp}$ can converge to $0$.
\end{exm}
\begin{exm}[Two distinct topologies with the same notion of sequential convergence]\label{exm4.2.25}
Let $X$ be as in \cref{exm4.2.8x} and denote the topology on $X$ given there (the cocountable topology) by $\mathcal{U}$.  Denote by $\mathcal{D}$ the discrete topology on $X$.  We saw in \cref{exm4.2.8x} that sequences converge iff they are eventually constant.  However, we also have the following result.
\begin{exr}
Let $X$ be a discrete space and let $\lambda \mapsto x_\lambda \in X$ be net.  Show that $\lambda \mapsto x_\lambda$ iff it is eventually constant.
\end{exr}
Thus, a given sequence $m\mapsto x_m\in X$ converges with respect to $\mathcal{U}$ iff it converges with respect to $\mathcal{D}$, and in this case, they converge to the same limit.  On the other hand, the set $\{ 0\}$ is \emph{not} open with respect to $\mathcal{U}$\footnote{This uses the fact that the real numbers are uncountable!} but it is open with respect to $\mathcal{D}$.

Even though the topologies are distinct, perhaps it is the case that they are \emph{homeomorphic}?  We show that this cannot happen.  In fact, we show that no bijective function from $\phi :(X,\mathcal{U})\rightarrow (X,\mathcal{D})$ is continuous.  If $\phi$ were such a function, then $\phi ^{-1}(\Q ^{\comp})$ would be uncountable and proper, and hence would not be closed, a contradiction of the fact that $\phi$ is continuous and $\Q ^{\comp}$ is closed with respect to $\mathcal{D}$.
\end{exm}
\begin{exm}[A sequentially-continuous function that is not continuous]\label{exm4.2.8}
Let $X$ be as in \cref{exm4.2.8x} ($\R$ with the cocountable topology) and define $f:X\rightarrow X$ by
\begin{equation}
f(x)\coloneqq \begin{cases}1 & \text{if }x\in \Q \\ -1 & \text{if }x\in \Q ^{\comp}\end{cases}.
\end{equation}
(Note that this is just the Dirichlet Function (\cref{DirichletFunction})!).

This function is certainly not continuous because, for example, the preimage of $\{ -1\}$ is not closed.  On the other hand, because every convergent sequence is eventually constant, the condition that $x_\lambda \neq x$ in the definition of a limit (\cref{dfnLimitOfAFunction}), $f$ vacuously satisfies the continuity condition for sequences.
\end{exm}
\begin{dfn}[Cover]
Let $S\subseteq X$ and let $\mathcal{U}\subseteq 2^{X}$.  Then, $\mathcal{U}$ is a \emph{cover}\index{Cover} of $S$ iff $S\subseteq \bigcup _{U\in \mathcal{U}}U$.  $\mathcal{U}$ is an \emph{open cover}\index{Open cover} iff every $U\in \mathcal{U}$ is open.  A \emph{subcover} of $\mathcal{U}$ is a subset $\mathcal{V}\subseteq \mathcal{U}$ that is still a cover of $S$.
\end{dfn}
\begin{dfn}[Quasicompact]\label{Quasicompact}
Let $S\subseteq X$.  Then, $S$ is \emph{quasicompact}\index{Quasicompact} iff every open cover of $S$ has a finite subcover.
\end{dfn}
\begin{exr}\label{exr4.2.33x}
Show that finite spaces are quaiscompact.
\end{exr}
\begin{exr}\label{exr4.2.33}
Show that closed subsets of quasicompact spaces are quasicompact.
\end{exr}
\begin{prp}\label{prp4.2.32}
Let $K\subseteq X$ and let $\mathcal{C}$ be a collection of closed subsets of $X$.  Then, $K$ is quasicompact iff whenever every finite intersection of elements of $\mathcal{C}$ intersects $K$, the entire intersection $\bigcap _{C\in \mathcal{C}}C$ also intersects $K$.
\end{prp}
\begin{prp}\label{prp4.2.31}
Let $K\subseteq X$.  Then, $K$ is quasicompact iff every net $\lambda \mapsto a_\lambda \in K$ has a subnet that converges to a limit in $K$.
\end{prp}
Note that the Heine-Borel and Bolzano-Weierstrass Theorems (\cref{HeineBorelTheorem,BolzanoWeierstrassTheorem}) do not hold in general, but we will wait until after having studied integration before writing down counter-examples.

\subsection{A couple new things}

We present in this subsection a couple of facts that, while not technically review per se, make more sense to place here once we begin study of new topological concepts in earnest.

\begin{dfn}[Dense]\label{Dense}
Let $X$ be a topological space and let $S\subseteq X$.  Then, $S$ is \emph{dense}\index{Dense} in $X$ iff $\Cls (S)=X$.
\end{dfn}
\begin{exr}\label{exr4.2.38}
Show that $\Q$ and $\Q ^{\comp}$ are both dense in $\R$.
\begin{rmk}
We mentioned way back when we discussed `density' of $\Q$ and $\Q ^{\comp}$ in $\R$ (\cref{thm3.2.14,thm3.3.76}) that these results aren't literally the statements that $\Q$ and $\Q ^{\comp}$ are dense in $\R$.  When we say that $\Q$ is dense in $\R$, what we mean of course is that $\Cls (\Q )=\R$ (and similarly for $\Q ^{\comp}$).  People `abuse' language and refer to the properties of \cref{thm3.2.14,thm3.3.76} as ``density'' because density in this sense (that is, in the sense of the definition above) follows as an easy corollary of these theorems.
\end{rmk}
\end{exr}

The next couple of facts have to do with generating collections and their relations to concepts reviewed in the previous subsection.
\begin{exr}\label{exr4.2.41}
Let $X$ be a topological space, let $\mathcal{S}$ generate the topology of $X$, let $\lambda \mapsto x_\lambda \in S$ be a net, and let $x_\infty \in X$.  Show that $\lambda \mapsto x_\lambda$ converges to $x_\infty$ iff $\lambda \mapsto x_\lambda$ is eventually contained in every element of $\mathcal{S}$ which contains $x_\infty$.
\begin{rmk}
In other words, for the purposes of convergence, it suffices to look only at a generating collection of the topology.
\end{rmk}
\end{exr}

There is another characterization of quasicompactness that we did not introduce in the previous chapter because it involves the concept of generating collections, something that we hadn't defined in that context.
\begin{thm}[Alexander Subbase Theorem]\label{AlexanderSubbaseTheorem}
Let $X$ be a topological space and let $\mathcal{S}$ be a generating collection for the topology on $X$.  Then, $X$ is quasicompact iff every cover by elements of $\mathcal{S}$ has a finite subcover.
\begin{rmk}
This is of course just the defining property of quasicompactness, the only change being that we need only check covers whose elements come from the generating collection $\mathcal{S}$.
\end{rmk}
\begin{rmk}
In particular, if $\mathcal{S}$ does not cover $X$, then $X$ is quasicompact.
\end{rmk}
\begin{rmk}
The term \emph{subbase}\index{Subbase} is sometimes used for generating collections which cover the space.  In fact, people usually make the requirement that the generating collection cover the space, but there is no need.
\end{rmk}
\begin{proof}\footnote{Proof adapted from \cite[pg.~139]{Kelley}.}
$(\Rightarrow )$ Of course, if $X$ is quasicompact, then \emph{every} open cover has a finite subcover, and so certainly open covers that come from $\mathcal{S}$ will have finite subcovers.

\blankline
\noindent
$(\Leftarrow )$ 
\Step{Make hypotheses}
Suppose that every cover by elements of $\mathcal{S}$ has a finite subcover.  To show that $X$ is quasicompact, we prove the contrapositive of the defining condition of quasicompactness.  That is, we show that every collection of open sets that has the property that no finite subset covers $X$, also does not cover $X$.  So, let $\mathcal{U}$ be a collection of open sets that has the property that no finite subset covers $X$.  We show that $\mathcal{U}$ itself does not cover $X$.

\Step{Enlarge $\mathcal{U}$ to a maximal collection}
Let $\tilde{\mathcal{U}}$ be the collection of all collections of open sets which (i) contain $\mathcal{U}$ (ii) also have the property that no finite subset covers $X$.  This is a set that is partially-ordered by inclusion, and so we intend to apply Zorn's Lemma (\cref{ZornsLemma}) to extract a maximal such element.  So, let $\tilde{\mathcal{W}}$ be a well-ordered subset of $\tilde{\mathcal{U}}$ and define
\begin{equation}
\mathcal{W}_0\coloneqq \bigcup _{\mathcal{W}\in \tilde{\mathcal{W}}}\mathcal{W}.
\end{equation}
Certainly $\mathcal{W}_0$ is a collection of open sets, a collection which contains $\mathcal{U}$.  In order to be an upper-bound for $\tilde{\mathcal{W}}$, however, we need to check that no finite subset covers $X$.  So, let $W_1,\ldots ,W_m\in \mathcal{W}_0$.  Then, each $W_k\in \mathcal{W}_k$ for some $\mathcal{W}_k\in \tilde{\mathcal{W}}$.  Because $\tilde{\mathcal{W}}$ is in particular totally-ordered, one of $\mathcal{W}_1,\ldots ,\mathcal{W}_m$ must contain all the others, and in particular, all the $W_k$s are contained in a single $\mathcal{W}_k$.  It follows that $\{ W_1,\ldots ,W_m\}$ cannot cover $X$.

Therefore, by Zorn's Lemma, there is a maximal collection of open sets $\mathcal{U}_0$ that (i) contains $\mathcal{U}$ and (ii) has the property that no finite subset covers $X$.  To show that $\mathcal{U}$ does not
cover $X$, it suffices to show that $\mathcal{U}_0$ does not cover $X$.

\Step{Show that if an element of $\mathcal{U}_0$ contains an intersection of open sets, it must contain one of those sets}
Let $U\in \mathcal{U}_0$ and suppose that $U_1\cap \cdots \cap U_m\subseteq U$ for $U_1,\ldots ,U_m$ open.  We proceed by contradiction:  suppose that $U_k\notin \mathcal{U}_0$ for all $k$.  This means that, by maximality, for each $U_k$, there are finitely many $U_k^1,\ldots ,U_k^{n_k}$ such that $X=U_k\cup U_k^1\cup \cdots \cup U_k^{n_k}$.  But then
\begin{equation}
U\cup \bigcup _{k=1}^m\bigcup _{l=1}^{n_k}U_k^l=\left( \bigcap _{k=1}^mU_k\right) \cup \left( \bigcup _{k=1}^m\bigcup _{l=1}^{n_k}U_k^l\right) \supseteq \bigcap _{k=1}^m\left( U_k\cup U_k^1\cup \cdots \cup U_k^{n_k}\right) =X,
\end{equation}
so that
\begin{equation}
U\cup \bigcup _{k=1}^m\bigcup _{l=1}^{n_k}U_k^l,
\end{equation}
a contradiction.  Therefore, some $U_k\in X$.

\Step{Deduce that $\mathcal{U}_0$ does not cover $X$}
Define
\begin{equation}
\mathcal{V}_0\coloneqq \mathcal{U}_0\cap \mathcal{S}.
\end{equation}
First of all, as no finite subset of $\mathcal{U}_0$ covers $X$, in particular, $X\notin \mathcal{U}_0$, so that
\begin{equation}
\mathcal{V}_0=\mathcal{U}_0\cap (\mathcal{S}\cup \{ X\} ).
\end{equation}
As every element of $\mathcal{V}_0$ comes from $\mathcal{U}_0$, it follows that no finite subset of $\mathcal{V}_0$ covers $X$.  On the other hand, every element of $\mathcal{V}_0$ comes from $\mathcal{S}$, so that, by hypothesis, it in turn follows that $\mathcal{V}_0$ does not cover $X$.  Thus, we will be done if we can show that
\begin{equation}
\bigcup _{V\in \mathcal{V}_0}V=\bigcup _{U\in \mathcal{U}_0}U.
\end{equation}
The $\subseteq$ inclusion is obvious because $\mathcal{V}_0\subseteq \mathcal{U}_0$.  For the other inclusion, let $x\in \bigcup _{U\in \mathcal{U}_0}U$.  Then, $x\in U$ for some $U\in \mathcal{U}_0$.  Because $\mathcal{S}$ is a generating collection, the collection of all finite intersections of elements of $\mathcal{S}\cup \{ X\}$ is a base (see \cref{GeneratingCollection}), and so there are $U_1,\ldots ,U_m\in \mathcal{S}$ such that $x\in U_1\cap \cdots \cap U_m\subseteq U$.  But then, by the previous step, $U_k\in \mathcal{U}_0$ for some $U_k$.  Of course, $U_k$ came from $\mathcal{S}\cup \{ X\}$, and so in fact $U_k\in \mathcal{U}_0\cap (\mathcal{S}\cup \{ X\} )=\mathcal{V}_0$, so that indeed $x\in \bigcup _{V\in \mathcal{V}_0}V$.
\end{proof}
\end{thm}

\section{Filter bases}\label{sct4.4}

This section is a bit of an aside and can probably be skipped without too much trouble.  Our motivation for covering filter bases is (i) it will help us demonstrate by our definition of subnet is the `correct' one, as opposed to, for example, the one currently given\footnote{5 July 2015} on Wikipedia, and (ii) it is something that is important enough that you should probably at least be aware of and will almost certainly eventually encounter if you decide to become a mathematician.

Filter bases are actually an alternative to nets.  In principle, one could do the entirety of topology never speaking of nets and instead using only filter bases.  This would be one motivation for introducing them (though we have decided to primarily stick to nets).
\begin{dfn}[Filter base]\label{FilterBase}
Let $(X,\leq )$ be a partially-ordered set and let $F\subset X$ be nonempty and not containing the empty-set.  Then, $F$ is a \emph{filter base}\index{Filter base} of $X$ iff for $x_1,x_2\in F$, there is some $x_3\in F$ such that $x_3\leq x_1,x_2$.\footnote{This property is called being \emph{downward-directed}.}
\begin{rmk}
This is the definition of an abstract filter base in any partially-ordered set.  For us, we will essentially only be interested in filter bases of the partially-ordered set $(2^X,\subseteq )$ for $X$ a topological space:  if $F$ is a filter base of $(2^X,\subseteq )$, $X$ a topological space, then we shall say that $F$ is a \emph{filter base in $X$}.  (Note that the elements of filter base do not have to be open sets.)
\end{rmk}
\begin{rmk}
This condition is exactly analogous to the condition for directed sets $\Lambda$, that for $x_1,x_2\in \Lambda$, there is some $x_3\in \Lambda$ with $x_3\geq x_1,x_2$.  In fact, you might even say that the definition is what it is because \emph{a filter base ordered by reverse-inclusion is a directed set}.
\end{rmk}
\end{dfn}
The relation between nets and filter bases is given by the following.
\begin{dfn}[Derived filter base]\label{DerivedFilterBase}
Let $X$ be a topological space, let $\lambda \mapsto x_\lambda \in X$ be a net, and define
\begin{equation}
\mathcal{F}_{\lambda \mapsto x_\lambda}\coloneqq \left\{ F\subseteq X:F\text{ eventually contains }\lambda \mapsto x_\lambda \right\} .
\end{equation}\index[notation]{$\mathcal{F}_{\lambda \mapsto x_\lambda}$}
\begin{exr}
Show that $\mathcal{F}$ is a filter base.
\end{exr}
$\mathcal{F}_{\lambda \mapsto x_\lambda}$ is the \emph{derived filter}\index{Derived filter} of $\lambda \mapsto x_\lambda$.
\begin{rmk}
Given a net, we just defined a canonically associated filter.  Of course, there will be many nets which give us this filter.  For example, I can change a single term of a net without affecting its derived filter.\footnote{In general at least.  In stupid cases, of course, e.g.~if the domain of the net is a single point, then this will change the derived filter.}
\end{rmk}
\begin{rmk}
Because of this, one might argue that filter bases are more fundamental than nets.  Nets somehow contain extra information that is completely irrelevant to topology.  An example of this is how one can always throw away finitely many terms of a sequence without affecting anything of importance.  Because the derived filter base of a net only contains information about what \emph{eventually} happens with the net, this extraneous information is lost when passing from the net to its derived filter.
\end{rmk}
\begin{rmk}
I personally find nets much more intuitive than filter bases (probably because they are a much more straightforward generalization of sequences than filter bases are), and thinking of how filter bases come from nets helps me understand some of the intuition of filter bases themselves
\end{rmk}
\end{dfn}

At the bare minimum, in order for filter bases and nets to be effectively equivalent for the purposes of topology, we must at least (i) define convergence of filter bases and (ii) show that convergence of a net agrees with convergence of its derived filter.
\begin{dfn}[Limit (of a filter base)]
Let $X$ be a topological space, let $\mathcal{F}$ be a filter base in $X$, and let $x_\infty \in X$.  Then, $x_\infty$ is a \emph{limit}\index{Limit (of a filter)} of $\mathcal{F}$ iff for every open neighborhood $U$ of $x_\infty$ there is some $F\in \mathcal{F}$ such that $F\subseteq U$.  If a filter base has a limit, then we say that it \emph{converges}\index{Convergence}.
\end{dfn}
\begin{prp}\label{prp4.3.6}
Let $X$ be a topological space, let $\lambda \mapsto x_\lambda \in X$ be a net, and let $x_\infty \in X$.  Then, $\lambda \mapsto x_\lambda$ converges to $x_\infty$ iff $\mathcal{F}_{\lambda \mapsto x_\lambda}$ converges to $x_\infty$.
\begin{proof}
$(\Rightarrow )$ Suppose that $\lambda \mapsto x_\lambda$ converges to $x_\infty$.  Let $U$ be an open neighborhood of $x_\infty$.  Then, $\lambda \mapsto x_\lambda$ is eventually contained in $U$.  Therefore, $U\in \mathcal{F}_{\lambda \mapsto x_\lambda}$, and so of course $\mathcal{F}_{\lambda \mapsto x_\lambda}$ converges to $x_\infty$.

\blankline
\noindent
$(\Leftarrow )$ Suppose that $\mathcal{F}_{\lambda \mapsto x_\lambda}$ converges to $x_\infty$.  Let $U$ be an open neighborhood of $x_\infty$.  Then, there is some $F\in \mathcal{F}_{\lambda \mapsto x_\lambda}$ such that $F\subseteq U$.  By definition of derived filter bases, it follows that $\lambda \mapsto x_\lambda$ is eventually contained in $F$, and hence eventually contained in $U$.  Therefore, $\lambda \mapsto x_\lambda$ converges to $x_\infty$.
\end{proof}
\end{prp}

Now we turn to filterings and subnets.  A filtering is to a filter base as a subnet is to a net.  Recall that one of our motivations for talking about filter bases at all was to argue that our definition of subnet was the `correct' one.  One nice thing about filter bases is that there is no confusion about what the definition of a filtering should be.\footnote{Though evidently there is some confusion as to what they should be called---see the remark in the definition below.}  We will then show that our definition of subnet coincides with the notion of a filtering.
\begin{dfn}[Filtering]\label{Filtering}
Let $\mathcal{F}$ be a filter base on $X$.  Then, a \emph{filtering}\index{Filtering} of $\mathcal{F}$ is a filter base $\mathcal{G}$ that has the property that, for every $F\in \mathcal{F}$, there is some $G\in \mathcal{G}$ such that $G\subseteq F$.
\begin{rmk}
Note that in some place\footnote{\emph{*cough*}---Wikipedia---\emph{*cough*}} this is called a \emph{refinement}.  This is poor terminology because it disagrees with the usual definition of refinements of covers---see \cref{dfnC.1}.  For comparison, we reproduce that definition here.
\begin{textequation}
$\mathcal{G}$ is a \emph{refinement} of $\mathcal{F}$ iff for every $G\in \mathcal{G}$ there is some $F\in \mathcal{F}$ such that $G\subseteq F$.
\end{textequation}
\end{rmk}
\end{dfn}
\begin{exr}\label{exr4.4.8}
Let $\mathcal{F}$ and $\mathcal{G}$ be filter bases.  Show that if $\mathcal{F}\subseteq \mathcal{G}$, then $\mathcal{G}$ is a filtering of $\mathcal{F}$.
\end{exr}
In fact, for derived filter bases, every filtering is of this form.
\begin{prp}\label{prp4.4.8}
Let $X$ be a topological space, let $\Lambda \ni \lambda \mapsto x_\lambda \in X$ be a net, and let $\Lambda '\ni \mu \mapsto \lambda _\mu \in \Lambda$ be a function such that $\mu \mapsto x_{\lambda _\mu}$ is a net.  Then the following are equivalent.
\begin{enumerate}
\item \label{enm4.4.8.i}$\mu \mapsto x_{\lambda _\mu}$ is a subnet of $\lambda \mapsto x_\lambda$.
\item \label{enm4.4.8.ii}$\mathcal{F}_{\lambda \mapsto x_\lambda}\subseteq \mathcal{F}_{\mu \mapsto x_{\lambda _\mu}}$.
\item \label{enm4.4.8.iii}$\mathcal{F}_{\mu \mapsto x_{\lambda _\mu}}$ is a filtering of $\mathcal{F}_{\lambda \mapsto x_\lambda}$.
\end{enumerate}
\begin{rmk}
In particular, as the definitions of subnet given in \cite{Kelley} (\cref{prp3.3.92}) and on Wikipedia\footnote{As of 16 July 2015} (\cref{prp3.3.93}) are \emph{not} equivalent (\cref{exm3.3.93,exr3.3.94}) to our definition of subnet (\cref{Subnet}), they are in turn not equivalent to the filtering of filter bases!
\end{rmk}
\begin{rmk}
Of course, in general there are filterings not of this form, but for \emph{derived} filter bases, filtering is equivalent to containing (as sets).
\end{rmk}
\begin{proof}
$(\ref{enm4.4.8.i}\Rightarrow \ref{enm4.4.8.ii})$ Suppose that $\mu \mapsto x_{\lambda _\mu}$ is a subnet of $\lambda \mapsto x_\lambda$.  Let $F\in \mathcal{F}_{\lambda \mapsto x_\lambda}$.  Then, $\lambda \mapsto x_\lambda$ is eventually in $F$ (see the definition of subnet, \cref{Subnet}), and so $\mu \mapsto x_{\lambda _\mu}$ is eventually in $F$, and so $F\in \mathcal{F}_{\mu \mapsto x_{\lambda _\mu}}$.

$(\ref{enm4.4.8.ii}\Rightarrow \ref{enm4.4.8.iii})$ \cref{exr4.4.8}

$(\ref{enm4.4.8.iii}\Rightarrow \ref{enm4.4.8.i})$
Suppose that $\mathcal{F}_{\mu \mapsto x_{\lambda _\mu}}$ is a filtering of $\mathcal{F}_{\lambda \mapsto x_\lambda}$.  Let $F\subseteq X$ be such that $F$ eventually contains $\lambda \mapsto x_\lambda$.  Then, $F\in \mathcal{F}_{\lambda \mapsto x_\lambda}$, and so there is some $F'\in \mathcal{F}_{\mu \mapsto x_{\lambda _\mu}}$ such that $F'\subseteq F$.  As $\mu \mapsto x_{\lambda _\mu}$ is eventually contained in $F'$, it is eventually contained in $F$, and so $\mu \mapsto x_{\lambda _\mu}$ is a subnet of $\lambda \mapsto x_\lambda$ (once again, by the definition of subnets, \cref{Subnet}).
\end{proof}
\end{prp}

In the next section, we present several new ways of defining topologies.  One of these ways will be by defining what it means for filters to converge, and so present here as theorems the results that will be used as axioms.
\begin{prp}[Kelley's Filter Convergence Axioms]\index{Kelley's Filter Convergence Axioms}\label{KelleysFilterConvergenceAxioms}
Let $X$ be a topological space.  Then,
\begin{enumerate}
\item \label{enmKelleysFilterConvergenceAxioms.i}$\mathcal{P}_x$ converges to $x$, where $\mathcal{P}_x\coloneqq \left\{ U\subseteq X:x\in U\right\}$;\footnote{``P'' is for \emph{principal}, the etymology being from the use of the word ``principal'' in the context of ideals in ring theory (to the best of my knowledge anyways).}
\item \label{enmKelleysFilterConvergenceAxioms.ii}$\mathcal{F}$ converges to $x$ iff for every filtering $\mathcal{G}\supseteq \mathcal{F}$, there is some filtering $\mathcal{H}\supseteq \mathcal{G}$ such that $\mathcal{H}$ converges to $x$; and
\item \label{enmKelleysFilterConvergenceAxioms.iii}for all directed sets $I$ and filters $\mathcal{F}^i$ converging to $x^i\in X$, for $i\in I$, if $\mathcal{F}_{i\mapsto x^i}$ converges to $x^\infty$, then
\begin{equation}
\begin{split}
\mathcal{F}_\infty & \coloneqq \left\{ U\subseteq X:\text{there exists }i_U\in I\text{ such that,}\right. \\
& \qquad \left. \text{whenever }i\geq i_U\text{, }U\supseteq F^i\text{ for some }F^i\in \mathcal{F}^i.\right\}
\end{split}
\end{equation}
also converges to $x^\infty$.
\end{enumerate}
\begin{rmk}
In other words, $\mathcal{F}_{\infty}$ consists of those sets that eventually contain some element of $\mathcal{F}^i$.
\end{rmk}
\begin{rmk}
Note that these are completely analogous to Kelley's (Net) Convergence Axioms (\cref{KelleysConvergenceAxioms}).
\end{rmk}
\begin{rmk}
Perhaps this could be made into an argument that somehow nets are more fundamental---to define a topology by convergence of filters, you have to make use of nets (in this case, $i\mapsto x^i$).\footnote{Or rather, I am not aware of a way to state the axioms without at least implicitly using nets.}
\end{rmk}
\begin{proof}
We ask you to prove \ref{enmKelleysFilterConvergenceAxioms.i} and \ref{enmKelleysFilterConvergenceAxioms.ii}.
\begin{exr}
Show \ref{enmKelleysFilterConvergenceAxioms.i}.
\end{exr}
\begin{exr}
Show \ref{enmKelleysFilterConvergenceAxioms.ii}.
\begin{rmk}
Hint:  Check out \cref{prp3.3.95}.
\end{rmk}
\end{exr}
Let $I$ be a directed set, for each $i\in I$ let $\mathcal{F}^i$ be a filter converging to $x^i\in X$, and suppose that $\mathcal{F}_{i\mapsto x^i}$ converges to $x^\infty$.  By \cref{prp4.3.6}, $i\mapsto x^i$ converges to $x^\infty$.  To show that $\mathcal{F}_\infty$ converges to $x^\infty$, it suffices to show that every open neighborhood of $x^\infty$ is an element of $\mathcal{F}_\infty$.  So, let $U$ be an open neighborhood of $x^\infty$.  Then, as $\mathcal{F}_{i\mapsto x^i}$ converges to $x^\infty$, it follows that there is some $V\in \mathcal{F}_{i\mapsto x^i}$ such that $V\subseteq U$.  As $i\mapsto x^i$ is eventually contained in $V$, there is some $i_0\in I$ such that, whenever $i\geq i_0$, it follows that $x^i\in V\subseteq U$.  Thus, for $i$ sufficiently large, $U$ is an open neighborhood of $x^i$.  But then, because $\mathcal{F}^i$ converges to $x^i$, it follows that there is some $F^i\in \mathcal{F}^i$ such that $F^i\subseteq U$, and so $U\in \mathcal{F}_{\infty}$ as desired.
\end{proof}
\end{prp}

\section{Equivalent definitions of topological spaces}

There is more than one way to define a topology.  By definition, the specification of a topology is just the specification of what sets are open.  Sometimes, however, what the open sets should be is not nearly as obvious as, for example, what convergence of nets should mean.  In this section, we present a couple of other ways you may define a topological space.  Which one is most useful, of course, will depend on the particular problem at hand.

To summarize, we have already shown that we may define a topology in the following ways.  We can define a topology
\begin{enumerate}
\item by specifying the open sets (\cref{TopologicalSpace});
\item by specifying the closed sets (\cref{exr4.1.2});
\item by specifying a base for the topology (\cref{prp4.1.5});
\item by specifying a neighborhood base for the topology (\cref{prp4.1.8}); or
\item by specifying a generating collection for the topology (\cref{GeneratingCollection}).
\end{enumerate}
Of course, there are many other ways to specify a topology as well, and it is the purpose of this section to list several others ways.

\subsection{Definition by specification of closures or interiors}

We have mentioned the Kuratowski Closure (Interior) Axioms as well as Kelley's (Filter) Convergence Axioms.  These play a role analogous to \ref{enmTopologicalSpace.i}--\ref{enmTopologicalSpace.iii} in the definition of a topological space, \cref{TopologicalSpace}.  For example, we have the following.
\begin{thm}[Kuratowski's Closure Theorem]\label{KuratowskisClosureTheorem}\index{Kuratowski's Closure Theorem}
Let $X$ be a set and let $\mathrm{C}:2^X\rightarrow 2^X$ be a function on the power-set of $X$.  Then, if
\begin{enumerate}
\item \label{enmKuratowskiClosureTheorem.i}$\mathrm{C}(\emptyset )=\emptyset$;
\item \label{enmKuratowskiClosureTheorem.ii}$S\subseteq \mathrm{C}(S)$;
\item \label{enmKuratowskiClosureTheorem.iii}$\mathrm{C}(S)=\mathrm{C}\left( \mathrm{C}(S)\right)$; and
\item \label{enmKuratowskiClosureTheorem.iv}$\mathrm{C}(S\cup T)=\mathrm{C}(S)\cup \mathrm{C}(T)$,
\end{enumerate}
then there exists a unique topology on $X$ such that $\Cls (S)=\mathrm{C}(S)$.
\begin{proof}
\Step{Make hypotheses}
Suppose that (i) $\mathrm{C}(\emptyset )=\emptyset$, (ii) $S\subseteq \mathrm{C}(S)$, (iii) $\mathrm{C}(S)=\mathrm{C}\left( \mathrm{C}(S)\right)$, and (iv) $\mathrm{C}(S\cup T)=\mathrm{C}(S)\cup \mathrm{C}(T)$.

\Step{Show that $S\subseteq T$ implies $\mathrm{C}(S)\subseteq \mathrm{C}(T)$}\label{stpKuratowskiClosureTheorem.2}
Suppose that $S\subseteq T$, so that $T=S\cup (T\setminus S)$, and hence $\mathrm{C}(T)=\mathrm{C}(S)\cup \mathrm{C}(T\setminus S)$, and  so $\mathrm{C}(S)\subseteq \mathrm{C}(T)$.

\Step{Define what should be the closed sets}
The idea of the proof is that the closed sets should be precisely the sets that are equal to their closure (\cref{prp4.2.21}).  We thus make the definition
\begin{equation}
\mathcal{C}\coloneqq \left\{ C\in 2^X:C=\mathrm{C}(C)\right\} .
\end{equation}

\Step{Verify that this defines a topology}
By \ref{enmKuratowskiClosureTheorem.i}, the empty-set is closed (by which we mean it is an element of $\mathcal{C}$).  By \ref{enmKuratowskiClosureTheorem.ii}, we have
\begin{equation}
X\subseteq \mathrm{C}(X)\subseteq X,
\end{equation}
so that $X$ is closed as well.  Let $\mathcal{D}\subseteq \mathcal{C}$ and define
\begin{equation}
B\coloneqq \bigcap _{C\in \mathcal{D}}C
\end{equation}
Then, of course, $B\subseteq C$ for all $C\in \mathcal{D}$, and so by \cref{stpKuratowskiClosureTheorem.2}, we have that $\mathrm{C}(B)\subseteq \mathrm{C}(C)$ for all $C\in \mathcal{D}$, and so
\begin{equation}
\mathrm{C}(B)\subseteq \bigcap _{C\in \mathcal{D}}\mathrm{C}(C)=\bigcap _{C\in \mathcal{D}}C\eqqcolon B.
\end{equation}
As $B\subseteq \mathrm{C}(B)$ by \ref{enmKuratowskiClosureTheorem.ii}, we have that $B=\mathrm{C}(B)$, so that $\mathcal{C}$ is closed under arbitrary intersections.  We now check that it is closed under finite unions, so let $C,D\in \mathcal{C}$.  Then,
\begin{equation}
\mathrm{C}(C\cup D)=\mathrm{C}(C)\cup \mathrm{C}(D)=C\cup  D,
\end{equation}
and so $C\cup D\in \mathcal{C}$.  Thus, $\mathcal{C}$ is closed under finite unions, and hence defines a topology.

\Step{Show that $\Cls (S)=\mathrm{C}(S)$}
Let $S\subseteq  X$.  By \ref{enmKuratowskiClosureTheorem.iii}, $\mathrm{C}(S)$ is closed, and by \ref{enmKuratowskiClosureTheorem.ii}, it contains $S$.  Therefore, $\Cls (S)\subseteq \mathrm{C}(S)$.  It follows that $\mathrm{C}\left( \Cls (S)\right) \subseteq \mathrm{C}(S)$.  On the other hand, because $S\subseteq \Cls (S)$, it follows that $\mathrm{C}(S)\subseteq \mathrm{C}\left( \Cls (S)\right) =\Cls (S)$, and so indeed $\Cls (S)=\mathrm{C}(S)$.
\end{proof}

\Step{Demonstrate uniqueness}
If another topology $\mathcal{V}$ satisfies this property, that is, $\Cls _{\mathcal{V}}(S)=C(S)$, then we have that $\Cls _{\mathcal{V}}(S)=\Cls (S)$ (no subscript indicates the closure in the topology defi8ned by $\mathcal{C}$).  It follows that a set $S$ is closed with respect to $\mathcal{V}$ iff it is equal to $\Cls _{\mathcal{V}}(S)$ iff it is equal to $\Cls (S)$ iff it is closed in the topology defined by $\mathcal{C}$.  Thus, the two topologies have the same closed sets, and hence are the same.
\end{thm}
Similarly, we have a `dual' interior theorem.
\begin{thm}[Kuratowski's Interior Theorem]\label{KuratowskisInteriorTheorem}\index{Kuratowski's Interior Theorem}
Let $X$ be a set and let $\mathrm{I}:2^X\rightarrow 2^X$ be a function on the power-set of $X$.  Then, if
\begin{enumerate}
\item $\mathrm{I}(X)=X$;
\item $\mathrm{I}(S)\subseteq S$;
\item $\mathrm{I}(S)=\mathrm{I}\left( \mathrm{I}(S)\right)$; and
\item $\mathrm{I}(S\cap T)=\mathrm{I}(S)\cap \mathrm{I}(T)$,
\end{enumerate}
then there exists a unique topology on $X$ such that $\Int (S)=\mathrm{I}(S)$.
\begin{rmk}
We omit the proof as it is completely `dual' to the corresponding closure proof.
\end{rmk}
\end{thm}

\subsection{Definition by specification of convergence}

The previous two results showed that we can define a topology by defining what the closure or interior of every set should be.  The next result says that we can define a topology by defining what it means for nets to converge.
\begin{thm}[Kelley's Convergence Theorem]\label{KelleysConvergenceTheorem}\index{Kelley's Convergence Theorem}
Let $X$ be a set, denote by $\mathcal{N}$ the collection of all nets in $X$, and let $\to$ be a relation on $\mathcal{N}\times X$.  Then, if
\begin{enumerate}
\item \label{enmKelleysConvergenceTheorem.i}$(\lambda \mapsto x_\infty)\to x_\infty$;
\item \label{enmKelleysConvergenceTheorem.ii}$(\lambda \mapsto x_\lambda )\to x_\infty$ iff every subnet $\mu \mapsto x_{\lambda _\mu}$ has in turn a subnet $\nu \mapsto x_{\lambda _{\mu _\nu}}$ such that $(\nu \mapsto x_{\lambda _{\mu _\nu}})\to x_\infty$; and
\item \label{enmKelleysConvergenceTheorem.iii} for all directed sets $I$ and nets $x^i:\Lambda ^i\rightarrow X$, if $x^i\to (x^i)_\infty$ and $(i\mapsto (x^i)_\infty )\to (x^\infty )_\infty$, then $\left( I\times \prod _{i\in I}\Lambda ^i\ni (i,\lambda )\mapsto (x^i)_{\lambda ^i}\right) \to (x^\infty )_\infty$,
\end{enumerate}
then there is a unique topology on $X$ such that $\lambda \mapsto x_\lambda$ converges to $x_\infty$ iff $(\lambda \mapsto x_\lambda )\to x_\infty$.
\begin{rmk}
People sometimes attempt to define a topology by defining what it means to \emph{sequences} to converge.  This is nonsensical.  For example, \ref{enmKelleysConvergenceTheorem.iii} doesn't even make sense in this context.  Moreover, because of examples like \cref{exm4.2.25} (the cocountable topology and discrete topology on $\R$ have the same notion of sequential convergence), trying to rectify this in general is hopeless.  My best guess is that this is because people tend to shy away from the usage of nets for some reason, and so they tend to define topologies using convergence of sequences.  In any case, I don't recall ever seeing a case where such a definition was given that \emph{didn't} make sense after you replaced the word ``sequence'' with the word ``net'.  That is to say, even though they are technically wrong, there is usually a very easy fix.
\end{rmk}
\begin{proof}
\Step{Make hypotheses}
Suppose that (i) $(\lambda \mapsto x_\infty)\to x_\infty$; (ii) $(\lambda \mapsto x_\lambda )\to x_\infty$ iff every subnet $\mu \mapsto x_{\lambda _\mu}$ has in turn a subnet $\nu \mapsto x_{\lambda _{\mu _\nu}}$ such that $(\nu \mapsto x_{\lambda _{\mu _\nu}})\to x_\infty$; and (iii) for all directed sets $I$ and nets $x^i:\Lambda ^i\rightarrow X$, if $x^i\to (x^i)_\infty$ and $(i\mapsto (x^i)_\infty )\to (x^\infty )_\infty$, then $\left( I\times \prod _{i\in I}\Lambda ^i\ni (i,\lambda )\mapsto (x^i)_{\lambda ^i}\right) \to (x^\infty )_\infty$.

\Step{Define the notion of a \emph{adherent point}}
For a subset $S\subseteq X$ and $x\in X$, we say that $x$ is an \emph{adherent} of $S$ iff there is some net $\lambda \mapsto x_\lambda \in S$ such that $(\lambda \mapsto x_\lambda )\to x$.

\Step{Define what should be the closure}
For a subset $S\subseteq X$, we define $\mathrm{C}(S)$ to be the set of adherent points of $S$.

\Step{Show that $\mathrm{C}$ satisfies Kuratowski's Closure Axioms}
The empty-set has no adherent points points, and so trivially $\mathrm{C}(\emptyset )=\emptyset$.

It follows from \ref{enmKelleysConvergenceTheorem.i} that $S\subseteq \mathrm{C}(S)$.

We wish to show that $\mathrm{C}\left( \mathrm{C}(S)\right) \subseteq \mathrm{C}(S)$, so let $(x^\infty )_\infty \in \mathrm{C}\left( \mathrm{C}(S)\right)$.  Then, there is a net $I\ni i\mapsto x^i\in \mathrm{C}(S)$ such that $(i\mapsto x^i )\to (x^\infty )_\infty$.  As each $x^i\in \mathrm{C}(S)$, there is a net $\Lambda ^i\ni \lambda ^i\mapsto (x^i)_{\lambda ^i}\in S$ such that $\left( \lambda ^i\mapsto (x^i)_{\lambda ^i}\right) \to (x^i)_\infty$.  By \ref{enmKelleysConvergenceTheorem.iii}, it then follows that $\left( I\times \prod _{i\in I}\Lambda ^i\ni (i,\lambda )\mapsto (x^i)_{\lambda ^i}\right) \to (x^\infty )_\infty$.  As each $(x^i)_{\lambda ^i}\in S$, it follows that $(x^\infty )_\infty$ is an adherent point of $S$, which completes the proof that $\mathrm{C}\left( \mathrm{C}(S)\right) =\mathrm{C}(S)$ (the $\mathrm{C}(S)\subseteq \mathrm{C}\left( \mathrm{C}(S)\right)$ follows from the fact that $S\subseteq \mathrm{C}(S)$).

We now show that $\mathrm{C}(S\cup T)=\mathrm{C}(S)\cup \mathrm{C}(T)$.  We have that $S\subseteq \mathrm{C}(S\cup T)$, and so $\mathrm{C}(S)\subseteq \mathrm{C}(S\cup T)$.  Similarly for $T$, and so we have $\mathrm{C}(S)\cup \mathrm{C}(T)\subseteq \mathrm{C}(S\cup T)$.

We now show that $\mathrm{C}(S\cup T)\subseteq \mathrm{C}(S)\cup \mathrm{C}(T)$.  So, let $x_\infty \in \mathrm{C}(S\cup T)$, so that there is a net $\Lambda \ni \lambda \mapsto x_\lambda \in S\cup T$ such that $(\lambda \mapsto x_\lambda )\to x_\infty$.  Define
\begin{equation}
\Lambda _S\coloneqq \left\{ \lambda \in \Lambda :x_\lambda \in S\right\} \text{ and }\Lambda _T\coloneqq \left\{ \lambda \in \Lambda :x_\lambda \in T\right\} .
\end{equation}
As $\Lambda =\Lambda _S\cup \Lambda _T$, either $\Lambda _S$ or $\Lambda _T$ is cofinal in $\Lambda$.  Without loss of generality, suppose that $\Lambda _S$ is cofinal in $\Lambda$, so that $\restr{x}{\Lambda _S}$ is a strict subnet of $\lambda \mapsto x_\lambda$.  By hypothesis, this in turn must have a subnet $\mu \mapsto x_{\lambda _\mu}$, $\lambda _\mu \in \Lambda _S$, such that $(\mu \mapsto x_{\lambda _\mu})\to x_\infty$.  Thus, $x_\infty \in \mathrm{C}(S)$, and so $\mathrm{C}(S\cup T)\subseteq \mathrm{C}(S)\cup \mathrm{C}(T)$.

It follows by Kuratowski's Closure Theorem (\cref{KuratowskisClosureTheorem}) that there is a unique topology on $X$ such that $\Cls (S)=\mathrm{C}(S)$.

\Step{Show that $\lambda \mapsto x_\lambda$ converges to $x_\infty$ iff $(\lambda \mapsto x_\lambda)\to x_\infty$}
Suppose that $\Lambda \ni \lambda \mapsto x_\lambda$ converges to $x_\infty$.  We proceed by contradiction:  suppose that it is not the case that $(\lambda \mapsto x_\lambda )\to x_\infty$.  Then, there must be some subnet $I\ni \mu \mapsto x_{\lambda _\mu}$ that has no subnet $\nu \mapsto x_{\lambda _{\mu _\nu}}$ such that $(\nu \mapsto x_{\lambda _{\mu _\nu}})\to x_\infty$.  To obtain a contradiction, we construct such a subnet.  For each $\mu _0$, define $S_{\mu _0}\coloneqq \{ x_{\lambda _\mu}:\mu \geq \mu _0\}$.  As $\lambda \mapsto x_\lambda$ converges to $x_\infty$, so does $\mu \mapsto x_{\lambda _\mu}$, and so $x_\infty \in \Cls (S_{\mu _0})$ for all $\mu _0$.  Hence, by the definition of our closure, for each $\mu _0\in I$ there is a net $\Lambda ^{\mu _0}\ni \nu ^{\mu _0}\mapsto x_{\lambda _{\mu _{\nu ^{\mu _0}}}}\in S_{\mu _0}$, so that $\mu _{\nu ^{\mu _0}}\geq \mu _0$, with $\left( \nu ^{\mu _0}\mapsto x_{\lambda _{\mu _{\nu ^{\mu _0}}}}\right) \to x_\infty$.\footnote{The superscript $\mu _0$ in $\nu ^{\mu _0}$ is just to help us keep track of which directed set $\nu ^{\mu _0}$ is contained in.}  From \ref{enmKelleysConvergenceTheorem.iii}, it follows that $\left( I\times \prod _{\mu _0\in I}\Lambda ^{\mu _0}\ni (\mu _0,\nu )\mapsto x_{\lambda _{\mu _{\nu ^{\mu _0}}}})\right) \to x_\infty$.  Thus, if this is in fact a subnet of $\mu \mapsto x_{\lambda _\mu}$, we will have our contradiction.  To show that this is indeed a subnet, let $\mu _0$ be arbitrary.  To show this, we apply \cref{prp3.3.92}.  Let $\nu _0\in \prod _{\mu _0\in I}\Lambda ^{\mu _0}$ be arbitrary, and suppose that $(\mu ,\nu )\geq (\mu _0,\nu _0)$.  Recall that we have that $\mu _{\nu ^{\mu _0}}\geq \mu _0$ always, and so certainly we will have that $\mu _{\nu ^{\mu _0}}\geq \mu _0$. 

Now suppose that $(\Lambda \ni \lambda \mapsto x_\lambda )\to x_\infty$.  We proceed by contradiction:  suppose that $\lambda \mapsto x_\lambda$ does not converge to $x_\infty$.  Then, there is a subnet $\mu \mapsto x_{\lambda _\mu}$ which has no subnet converging to $x_\infty$.  In particular, $\mu \mapsto x_{\lambda _\mu}$ itself does not converge to $x_\infty$, and so there is an open neighborhood $U$ of $x_\infty$ which does not eventually contain $\mu \mapsto x_{\lambda _\mu}$.  It follows that $I\coloneqq \{ \lambda _\mu :x_{\lambda _\mu}\notin U\}$ is cofinal, so that $I\ni \mu \mapsto x_{\lambda _\mu}$ is a subnet contained in $U^{\comp}$.  On the other hand, we know that $(I\ni \mu \mapsto x_{\lambda _\mu})\to x_\infty$, so that $x_\infty \in \Cls (\{ x_{\lambda _\mu}:\mu \in I\})$, but this is a contradiction of the fact that it has an open neighborhood which contains no point of this set.

\Step{Demonstrate uniqueness}
Recall that sets are closed iff they contain all their limit points.  If we have two topologies with the same notion of convergence, then the set of limit points of a given set in each topology are the same, and consequently, a set is closed in one iff it is closed in the other.
\end{proof}
\end{thm}
One thing to note is that, in practice, it is often easier to check two conditions which are equivalent to the second axiom:   (i) subnets of convergent nets converge to the same thing and (ii) that $(\Leftarrow )$ direction of the second axiom.
\begin{prp}\label{prp3.4.22}
Let $X$ be a set, denote by $\mathcal{N}$ the collection of all nets in $X$, and let $\to$ be a relation on $\mathcal{N}\times X$.  Then, the following are equivalent.
\begin{enumerate}
\item \label{enm3.4.22.i}$(\lambda \mapsto x_\lambda )\to x_\infty$ iff every subnet $\mu \mapsto x_{\lambda _\mu}$ has in turn a subnet $\nu \mapsto x_{\lambda _{\mu _\nu}}$ such that $(\nu \mapsto x_{\lambda _{\mu _\nu}})\to x_\infty$.
\item \label{enm3.4.22.ii}\begin{enumerate}
\item \label{enm3.4.22.ii.a}If $(\lambda \mapsto x_\lambda )\to x_\infty$ and $\mu \mapsto x_{\lambda _\mu}$ is a subnet of $\lambda \mapsto x_\lambda$, then $(\mu \mapsto x_{\lambda _\mu})\to x_\infty$; and
\item \label{enm3.4.22.ii.b}If every subnet $\mu \mapsto x_{\lambda _\mu}$ has in turn a subnet $\nu \mapsto x_{\lambda _{\mu _\nu}}$ such that $(\nu \mapsto x_{\lambda _{\mu _\nu}})\to x_\infty$, then $(\lambda \mapsto x_\lambda )\to x_\infty$.
\end{enumerate}
\end{enumerate}
\begin{proof}
$(\Rightarrow )$ Suppose that $(\lambda \mapsto x_\lambda )\to x_\infty$ iff every subnet $\mu \mapsto x_{\lambda _\mu}$ has in turn a subnet $\nu \mapsto x_{\lambda _{\mu _\nu}}$ such that $(\nu \mapsto x_{\lambda _{\mu _\nu}})\to x_\infty$.  \ref{enm3.4.22.ii.b} holds by hypothesis.  We thus check \ref{enm3.4.22.ii.a}.  So, let $\lambda \mapsto x_\lambda$ be a net such that $(\lambda \mapsto x_\lambda )\to x_\infty$ and let $\mu \mapsto x_{\lambda _\mu}$ be a subnet.  To show that $(\mu \mapsto x_{\lambda _\mu})\to x_\infty$, we show that every subnet $\nu \mapsto x_{\lambda _{\mu _\nu}}$ has in turn a subnet $\xi \mapsto x_{\lambda _{\mu _{\nu _\xi}}}$ such that $(\xi \mapsto x_{\lambda _{\mu _{\nu _\xi}}})\to x_\infty$.  So, let $\nu \mapsto x_{\lambda _{\mu _\nu}}$ be a subnet of $\mu \mapsto x_{\lambda _\mu}$.  This itself is also a subnet of $\lambda \mapsto x_\lambda$, and so by hypothesis, it has a subnet $\xi \mapsto x_{\lambda _{\mu _{\nu _\xi}}}$ such that $(\xi \mapsto x_{\lambda _{\mu _{\nu _\xi}}})\to x_\infty$.

$(\Leftarrow )$ Suppose that (a) if $(\lambda \mapsto x_\lambda )\to x_\infty$ and $\mu \mapsto x_{\lambda _\mu}$ is a subnet of $\lambda \mapsto x_\lambda$, then $(\mu \mapsto x_{\lambda _\mu})\to x_\infty$; and (b) if every subnet $\mu \mapsto x_{\lambda _\mu}$ has in turn a subnet $\nu \mapsto x_{\lambda _{\mu _\nu}}$ such that $(\nu \mapsto x_{\lambda _{\mu _\nu}})\to x_\infty$, then $(\lambda \mapsto x_\lambda )\to x_\infty$.  The $(\Leftarrow )$ direction of \ref{enm3.4.22.i} is true by hypothesis, so it suffices to show the $(\Rightarrow )$ direction of \ref{enm3.4.22.i}.  So, suppose that $(\lambda \mapsto x_\lambda )\to x_\infty$ and let $\mu \mapsto x_{\lambda _\mu}$ be a subnet.  Then, by hypothesis, we also have that $(\mu \mapsto x_{\lambda _\mu})\to x_\infty$, and so this subnet $\mu \mapsto x_\mu$ has in turn a subnet (namely itself) that is related to $x_\infty$ by $\to$.
\end{proof}
\end{prp}

And of course, there is an analogous result for filters.
\begin{thm}[Kelley's Filter Convergence Theorem]\index{Kelley's Filter Convergence Theorem}\label{KelleysFilterConvergenceTheorem}
Let $X$ be a set, denoted by $\tilde{\mathcal{F}}$ the collection of all filter bases in $X$, and let $\to$ be a relation on $\tilde{\mathcal{F}}\times X$.  Then, if
\begin{enumerate}
\item \label{enmKelleysFilterConvergenceTheorem.i}$\mathcal{P}_x\to x$, where $\mathcal{P}_x\coloneqq \left\{ U\subseteq X:x\in U\right\}$;\footnote{``P'' is for \emph{principal}, the etymology being from the use of the word ``principal'' in the context of ideals in ring theory (to the best of my knowledge anyways).}
\item \label{enmKelleysFilterConvergenceTheorem.ii}$\mathcal{F}\to x$ iff for every filtering $\tilde{\mathcal{F}}\ni \mathcal{G}\supseteq \mathcal{F}$, there is some filtering $\tilde{\mathcal{F}}\ni \mathcal{H}\supseteq \mathcal{G}$ such that $\mathcal{H}\to x$; and
\item \label{enmKelleysFilterConvergenceTheorem.iii}for all directed sets $I$ and convergent filters $\mathcal{F}^i\to x^i\in X$, for $i\in I$, if $\mathcal{F}_{i\mapsto x^i}\to x^\infty$, then
\begin{equation}
\begin{split}
\mathcal{F}_\infty & \coloneqq \left\{ U\subseteq X:\text{there exists }i_U\in I\text{ such that,}\right. \\
& \qquad \left. \text{whenever }i\geq i_U\text{, }U\supseteq F^i\text{ for some }F^i\in \mathcal{F}^i\text{.}\right\} \to x^\infty ,
\end{split}
\end{equation}
\end{enumerate}
then there is a unique topology on $X$ such that $\mathcal{F}$ converges to $x_\infty \in X$ iff $\mathcal{F}\to x_\infty$.
\begin{proof}
\Step{Make hypotheses}
Suppose that \ref{enmKelleysFilterConvergenceTheorem.i}$\mathcal{P}_x\to x$, where $\mathcal{P}_x\coloneqq \left\{ U\subseteq X:x\in U\right\}$; \ref{enmKelleysFilterConvergenceTheorem.ii} $\mathcal{F}\to x$ iff for every filtering $\tilde{\mathcal{F}}\ni \mathcal{G}\supseteq \mathcal{F}$, there is some filtering $\tilde{\mathcal{F}}\ni \mathcal{H}\supseteq \mathcal{G}$ such that $\mathcal{H}\to x$; and \ref{enmKelleysFilterConvergenceTheorem.iii} for all directed sets $I$ and convergent filters $\mathcal{F}^i\to x^i\in X$, for $i\in I$, if $\mathcal{F}_{i\mapsto x^i}\to x^\infty$, then $\mathcal{F}_\infty \to x^\infty$.

\Step{Define a relation on $\mathcal{N}\times X$}
Let $\mathcal{N}$ denote the collection of all nets in $X$, and for $\lambda \mapsto x_\lambda \in \mathcal{N}$ and $x_\infty \in X$, define
\begin{equation}
(\lambda \mapsto x_\lambda )\to x_\infty \text{ iff }\mathcal{F}_{\lambda \mapsto x_\lambda}\to x_\infty .\footnote{Even though there is no topology around (yet), the notion of ``eventually containing'' still makes sense, and so of course $\mathcal{F}_{\lambda \mapsto x_\lambda}$ still makes sense.}
\end{equation}

\Step{Verify \ref{enmKelleysConvergenceTheorem.i} of Kelley's Convergence Theorem}
As the derived filter base of the constant net $\lambda \mapsto x_\infty$ is just the principal filter base $\mathcal{P}_{x_\infty}$, it follows that \ref{enmKelleysConvergenceTheorem.i} of Kelley's Convergence Theorem, \cref{KelleysConvergenceTheorem} holds.

\Step{Verify $(\Rightarrow)$ of \ref{enmKelleysConvergenceTheorem.ii} of Kelley's Convergence Theorem}
Let $\lambda \mapsto x_\lambda$ be a net such that $(\lambda \mapsto x_\lambda )\to x_\infty$, and let $\mu \mapsto x_{\lambda _\mu}$ be a subnet of $\lambda \mapsto x_\lambda$.  Then, $\mathcal{F}_{\lambda \mapsto x_{\lambda}}\to x_\infty$.  As $\mathcal{F}_{\mu \mapsto x_{\lambda _\mu}}\supseteq \mathcal{F}_{\lambda \mapsto x_\lambda}$, by \ref{enmKelleysFilterConvergenceTheorem.ii}, there is a filter base $\mathcal{G}\supseteq \mathcal{F}_{\mu \mapsto x_{\lambda _\mu}}$ with $\mathcal{G}\to x_\infty$.  Without loss of generality, we may assume that each $G\in \mathcal{G}$ contains some $x_{\lambda _\mu}$ (we may throw away all such sets which do not contain some $x_{\lambda _\mu}$, and as $(\lambda \mapsto x_\lambda )\to x_\infty$< this will not affect the fact that $\mathcal{G}\to x_\infty$).  For $G_1,G_2\in \mathcal{G}$, define $G_1\leq G_2$ iff $G_1\supseteq G_2$.  By the definition of a filter base (\cref{FilterBase}), it follows that $(\mathcal{G},\leq )$ is a directed set.  For each $G\in \mathcal{G}$, let $x_{\lambda _{\mu _G}}$ denote any element contained of the net $\mu \mapsto x_{\lambda _\mu}$ contained in $G$.

We first check that $G\mapsto x_{\lambda _{\mu _G}}$ is a subnet of $\mu \mapsto x_{\lambda _\mu}$.  So, let $U\subseteq X$ eventually contain $\mu \mapsto x_{\lambda _\mu}$.  Then, $U\in \mathcal{F}_{\mu \mapsto x_{\lambda _\mu}}$, and so $U\in \mathcal{G}$ because $\mathcal{G}\supseteq \mathcal{F}_{\mu \mapsto x_{\lambda _\mu}}$.  Now suppose that $G\geq U$, so that, by definition of $\leq$ on $\mathcal{G}$ as a directed set, $G\subseteq U$.  Hence, for $G\geq U$, $x_{\lambda _{\mu _G}}\in G\subseteq U$, so that $G\mapsto x_{\lambda _{\mu _G}}$ is eventually contained in $U$, and hence is a subnet of $\mu \mapsto x_{\lambda _\mu}$.

Thus, as $G\mapsto x_{\lambda _{\mu _G}}$ is a subnet of $\mu \mapsto x_{\lambda _\mu}$ and $(\mu \mapsto x_{\lambda _\mu})\to x_\infty$, it follows that $(G\mapsto x_{\lambda _{\mu _G}})\to x_\infty$.  Thus, the arbitrary subnet of $\mu \mapsto x_{\lambda _\mu}$ of $\lambda \mapsto x_\lambda$ does indeed have a subnet that `converges to' $x_\infty$,\footnote{``Converges to' is in quotes because what we really mean is that it is related to $x_\infty$ by the relation $\to$.} and so $(\Rightarrow)$ of \ref{enmKelleysConvergenceTheorem.ii} of Kelley's Convergence Theorem is indeed satisfied.

\Step{Verify $(\Leftarrow )$ of \ref{enmKelleysConvergenceTheorem.ii} of Kelley's Convergence Theorem}
Let $\lambda \mapsto x_\lambda$ be a net such that for every subnet $\mu \mapsto x_{\lambda _\mu}$, there is some subnet of that subnet $\nu \mapsto x_{\mu _\nu}$ such that $(\nu \mapsto x_{\mu _\nu})\to x_\infty$.  We wish to show that $(\lambda \mapsto x_\lambda )\to x_\infty$.  In other words, we want to show that $\mathcal{F}_{\lambda \mapsto x_\lambda}\to x_\infty$.  To show this, we apply \ref{enmKelleysFilterConvergenceTheorem.ii}.  So, let $\mathcal{G}\supseteq \mathcal{F}_{\lambda \mapsto x_\lambda}$ be a filter base.  Then, using the same technique as in the previous step,\footnote{Turn $\mathcal{G}$ into a directed set by reverse-inclusion, show that each $G\in \mathcal{G}$ must without loss of generality contain some $x_\lambda$, and pick any one such lambda to form a subnet $G\mapsto x_{\lambda _G}$.} there is a subnet $\mathcal{G}\ni G\mapsto x_{\lambda _G}$ with $\mathcal{G}$ ordered by reverse-inclusion.  By hypothesis, there is then in turn a subnet of this $\mu \mapsto x_{\lambda _{G_\mu}}$ such that $(\mu \mapsto x_{\lambda _{G_\mu}})\to x_\infty$, and hence $\mathcal{F}_{\mu \mapsto x_{\lambda _{G_\mu}}}\to x_\infty$.  If $\mathcal{F}_{\mu \mapsto x_{\lambda _{G_\mu}}}\supseteq \mathcal{G}$, this completes this step by \ref{enmKelleysFilterConvergenceTheorem.ii}.  To show this, let $G_0\in \mathcal{G}$ and suppose that $G\geq G_0$.  Then, $G\subseteq G_0$, and so $x_{\lambda _G}\in G\subseteq G_0$, so that $G\mapsto x_{\lambda _G}$ is eventually contained in $G_0$.  Therefore, by the definition of a subnet (\cref{Subnet}), $\mu \mapsto x_{\lambda _{G_\mu}}$ is eventually contained in $G$, and so $G\in \mathcal{F}_{\mu \mapsto x_{\lambda _{G_\mu}}}$, and so $\mathcal{F}_{\mu \mapsto x_{\lambda _{G_\mu}}}\supseteq \mathcal{G}$.

\Step{Show \ref{enmKelleysConvergenceTheorem.iii} of Kelley's Convergence Theorem}
Let $I$ be a directed set, for each $i\in I$ let $x^i:\Lambda ^i\rightarrow X$ be a net such that $x^i\to (x^i)_\infty$, and suppose that $(i\mapsto (x^i)_\infty )\to (x^\infty )_\infty \in X$.  Note that, for $\mathcal{F}^i\coloneqq \mathcal{F}_{\lambda ^i\mapsto (x^i)_{\lambda ^i}}$ as in \ref{enmKelleysFilterConvergenceTheorem.iii},
\begin{equation}
\begin{split}
\mathcal{F}_\infty & \coloneqq \left\{ U\subseteq X:\text{there exists }i_U\in I\text{ such that, whenever }i\geq i_U\text{,}\right. \\
& \qquad \left. \text{it follows that }U\text{ eventually contains }\lambda ^i\mapsto (x^i)_{\lambda ^i}\right\} .
\end{split}
\end{equation}
By \ref{enmKelleysFilterConvergenceTheorem.iii}, $\mathcal{F}^\infty \to x^\infty$.

We wish to show that
\begin{equation}
\left( I\times \prod _{i\in I}\Lambda ^i\ni (i,\lambda )\mapsto (x^i)_{\lambda ^i}\right) \to (x^\infty )_\infty .
\end{equation}
In other words, we would like to show that
\begin{equation}
\mathcal{F}_{(i,\lambda )\mapsto (x^i)_{\lambda ^i}}\to (x^\infty )_\infty .
\end{equation}
To show this, by \ref{enmKelleysFilterConvergenceTheorem.ii}, it suffices to show that
\begin{equation}
\mathcal{F}_\infty \subseteq \mathcal{F}_{(i,\lambda )\mapsto (x^i)_{\lambda ^i}}.
\end{equation}
So, let $U\in \mathcal{F}_\infty$.  Then, there is some $i_U\in I$ such that, whenever $i\geq i_U$, it follows that $U$ eventually contains $\lambda ^i\mapsto (x^i)_{\lambda ^i}$.  It follows from this that, for all such $i$, there is a $(\lambda ^i)_U$ such that, whenever $\lambda ^i\geq (\lambda ^i)_U$, it follows that $(x^i)_{\lambda ^i}\in U$.  Define $\lambda _U\in \prod _{i\in I}\Lambda ^i$ such that $(\lambda _U)^i\coloneqq (\lambda ^i)_U$ for $i\geq i_U$ and anything for $i\not \geq i_u$.  Then, whenever $(i,\lambda )\geq (i_U,\lambda _U)$, it follows that $i\geq i_U$ and each $\lambda ^i\geq (\lambda _U)^i\coloneqq (\lambda ^i)_U$, so that
\begin{equation}
(x^i)_{\lambda ^i}\in U.
\end{equation}
Thus, $U$ eventually contains $(i,\lambda )\mapsto (x^i)_{\lambda ^i}$, and we are done with this step.

\Step{Show that the topology satisfies $\mathcal{F}\to x_\infty$ iff $\mathcal{F}$ converges to $x_\infty$}
After verifying \ref{enmKelleysConvergenceTheorem.i}--\ref{enmKelleysConvergenceTheorem.iii} of Kelley's Convergence Theorem, it follows that there is a unique topology on $X$ that has the property that $(\lambda \mapsto x )\to x_\infty$ iff $\lambda \mapsto x_\lambda$ converges to $x_\infty$.  We now check that the analogous property in terms of filters holds for this topology.

For the remainder of this step, we will make use of Kelley's Filter Convergence Axioms (\cref{KelleysFilterConvergenceAxioms}), which we already know to be true about actual convergence of filter bases.

Suppose that $\mathcal{F}\to x_\infty$.  To show that $\mathcal{F}$ converges to $x_\infty$, we prove that any filtering $\mathcal{G}\supseteq \mathcal{F}$ has in turn a filtering which converges to $x_\infty$.  Order $\mathcal{G}$ by reverse-inclusion and pick $x_G\in G\in \mathcal{G}$ so that $G\mapsto x_G$ is a net.  It follows that $\mathcal{F}_{G\mapsto x_G}\supseteq \mathcal{G}$, and so by \ref{enmKelleysFilterConvergenceTheorem.ii}, $\mathcal{F}_{G\mapsto x_G}\to x_\infty$, and so $(G\mapsto x_G)\to x_\infty$, and so $G\mapsto x_G$ converges to $x_\infty$.  To show that $\mathcal{F}_{G\mapsto x_G}$ converges to $x_\infty$, let $U\subseteq X$ be an open neighborhood of $x_\infty$.  Then, $G\mapsto x_G$ is eventually in $U$, and so $U\subseteq U\in \mathcal{F}_{G\mapsto x_G}$.  Thus, every filtering of $\mathcal{F}$ has in turn a filtering which converges to $x_\infty$, and so $\mathcal{F}$ converges to $x_\infty$.

The converse follows because this argument is ``$\to$''$\leftrightarrow$``converges to'' symmetric.
\end{proof}
\end{thm}
And of course, there is an analogous result to \cref{prp3.4.22} which gives an equivalent characterization of the second axiom which can sometimes be easier to check.
\begin{prp}
Let $X$ be a set, denoted by $\tilde{\mathcal{F}}$ the collection of all filter bases in $X$, and let $\to$ be a relation on $\tilde{\mathcal{F}}\times X$.  Then, the following are equivalent.
\begin{enumerate}
\item $\mathcal{F}\to x$ iff for every filtering $\tilde{\mathcal{F}}\ni \mathcal{G}\supseteq \mathcal{F}$, there is some filtering $\tilde{\mathcal{F}}\ni \mathcal{H}\supseteq \mathcal{G}$ such that $\mathcal{H}\to x$.
\item \begin{enumerate}
\item If $\mathcal{F}\to x_\infty$ and $\mathcal{G}\supseteq \mathcal{F}$, then $\mathcal{G}\to x_\infty$; and
\item If every filtering $\tilde{\mathcal{F}}\ni \mathcal{G}\supseteq \mathcal{F}$, there is some filtering $\tilde{\mathcal{F}}\ni \mathcal{H}\supseteq \mathcal{G}$ such that $\mathcal{H}\to x$.
\end{enumerate}
\end{enumerate}
\begin{proof}
We leave this as an exercise.
\begin{exr}
Complete the proof yourself, using the proof of \cref{prp3.4.22} as a guide.
\end{exr}
\end{proof}
\end{prp}

\subsection{Definition by specification of continuous functions}

The following two results define a topology on a set by simply `declaring' that a collection of functions be continuous.  This is similar in nature to how we defined a topology with a generating collection (\cref{GeneratingCollection})---in this case, we started with a collection of subsets, and simply `declared' them to be open.
\begin{prp}[Initial topology]\label{InitialTopology}
Let $X$ be a set, let $\mathcal{Y}$ be an indexed collection of topological spaces, and for each $Y\in \mathcal{Y}$ let $f_Y:X\rightarrow Y$ be a function.  Then, there exists a unique topology  $\mathcal{U}$ on $X$, the \emph{initial topology}\index{Initial topology} with respect to $\{ f_Y:Y\in \mathcal{Y}\}$, such that
\begin{enumerate}
\item $f_Y:X\rightarrow Y$ is continuous with respect to $\mathcal{U}$ for all $Y\in \mathcal{Y}$; and
\item if $\mathcal{U}'$ is another topology for which each $f_Y:X\rightarrow Y$ is continuous, then $\mathcal{U}\subseteq \mathcal{U}'$.
\end{enumerate}
Furthermore, if $\mathcal{S}_Y$ generates the topology on $Y$, then the collection
\begin{equation}
\{ f_Y^{-1}(U):Y\in \mathcal{Y},\ U\in \mathcal{S}_Y\}
\end{equation}
generates $\mathcal{U}$.
\begin{rmk}
In other words, the initial topology is the smallest topology for which each $f_Y$ is continuous.
\end{rmk}
\begin{rmk}
But what about the largest such topology?  Well, the largest such topology is always going to be the discrete topology, which is not very interesting.  This is how you remember whether the initial topology is the smallest or largest---it can't be the largest because the discrete topology always works.
\end{rmk}
\begin{rmk}
In particular,
\begin{equation}
\{ f_Y^{-1}(U):Y\in \mathcal{Y},\ U\subseteq Y\text{ open}\}
\end{equation}
generates the initial topology.
\end{rmk}
\begin{rmk}
Compare this with the definition of the integers, rationals, closure, interior, and generating collections (\cref{Integers,RationalNumbers,Closure,Interior,GeneratingCollection}).
\end{rmk}
\begin{proof}
For $Y\in \mathcal{Y}$, let $\mathcal{S}_Y$ be any generating collection, define
\begin{equation}
\mathcal{S}\coloneqq \{ f_Y^{-1}(U):Y\in \mathcal{Y},\ U\in \mathcal{S}_Y\} ,
\end{equation}
and let $\mathcal{U}$ be the topology generated by $\mathcal{S}$ (\cref{GeneratingCollection}).

As every element of $\mathcal{S}$ is open, it is certainly the case that each $f_Y$ is continuous by \cref{exr4.1.27} (it suffices to check continuity on a generating collection).  On the other hand, if $\mathcal{U}'$ is another topology for each each $f_Y$ is continuous, then it must certainly contain $\mathcal{S}$, in which case $\mathcal{U}'$ contains $\mathcal{U}$ by the definition of a generating collection.

\begin{exr}
Show that the initial topology is unique.
\end{exr}
\end{proof}
\end{prp}
There is a nice characterization of convergence in initial topologies.
\begin{prp}\label{prp4.4.5}
Let $X$ have the initial topology with respect to the collection $\{ f_Y:Y\in \mathcal{Y}\}$, let $\lambda \mapsto x_\lambda \in X$ be a net, and let $x_\infty \in X$.  Then, $\lambda \mapsto x_\lambda$ converges to $x_\infty$ in $X$ iff $\lambda \mapsto f_Y(x_\lambda )$ converges to $f_Y(x_\infty )$ in $Y$ for all $Y\in \mathcal{Y}$.  Furthermore, the initial topology is the unique topology that has this property.
\begin{proof}
$(\Rightarrow )$ Suppose that $\lambda \mapsto x_\lambda$ converges to $x_\infty$ in $X$.  Then, because each $f_Y$ is continuous, it follows that $\lambda \mapsto f_Y(x_\lambda )$ converges to $f_Y(x_\infty )$ in $Y$ for all $Y\in \mathcal{Y}$.

\blankline
\noindent
$(\Leftarrow )$ Suppose that $\lambda \mapsto f_Y(x_\lambda )$ converges to $f_Y(x_\infty )$ in $Y$ for all $Y\in \mathcal{Y}$.  To show that $\lambda \mapsto x_\lambda$ converges to $x_\infty$ in $X$, we apply \cref{exr4.2.41} (it suffices to check convergence on a generating collections).  We know that
\begin{equation}
\{ f_Y^{-1}(U):Y\in \mathcal{Y},\ U\subseteq Y\text{ open}\}
\end{equation}
generates the initial topology, and so we apply \cref{exr4.2.41} to this.  So, let $f_Y^{-1}(U)$ be an open neighborhood of $x_\infty$.  We must show that $\lambda \mapsto x_\lambda$ is eventually contained in $f_Y^{-1}(U)$.  However, if $f_Y^{-1}(U)$ is an open neighborhood of $x_\infty$, then $U\supseteq f_Y(f_Y^{-1}(U))$ is an open neighborhood of $f_Y(x_\infty )$, and so $\lambda \mapsto f_Y(x_\lambda )$ is eventually contained in $U$ because $\lambda \mapsto f_Y(x_\lambda )$ converges to $f_Y(x_\infty )$.  But then $\lambda \mapsto x_\lambda$ is eventually contained in $U$, and we are done.

\blankline
\noindent
Uniqueness follows from \nameref{KelleysConvergenceTheorem}.
\end{proof}
\end{prp}
Initial topologies are great in that you can determine whether functions into $X$ are continuous or not by looking at their composition with each $f_Y$.
\begin{prp}\label{prp3.4.6}
Let $X$ have the initial topology with respect to the collection $\{ f_Y:Y\in \mathcal{Y}\}$, let $Z$ be a topological space, and let $f:Z\rightarrow X$ be a function.  Then, $f$ is continuous iff $f_Y\circ f$ is continuous for all $Y\in \mathcal{Y}$.  Furthermore, the initial topology is the unique topology with this property.
\begin{proof}
$(\Rightarrow )$ Suppose that $f$ is continuous.  Then, because each $f_Y$ is itself continuous and compositions of continuous functions are continuous, it follows that $f_Y\circ f$ is continuous for all $Y\in \mathcal{Y}$.

\blankline
\noindent
$(\Leftarrow )$ Suppose that $f_Y\circ f$ is continuous for all $Y\in \mathcal{Y}$.  Let $\lambda \mapsto x_\lambda$ converge to $x_\infty \in X$.  To show that $f$ is continuous, it suffices to show that $\lambda \mapsto f(x_\lambda)$ converges to $f(x_\infty )$.  However, because each $f_Y\circ f$ is continuous, $\lambda \mapsto f_Y(f(x_\lambda ))$ converges to $f_Y(f(x_\infty ) )$.  Therefore, by the previous result, we do indeed have that $\lambda \mapsto f(x_\lambda)$ converges to $f(x_\infty )$.

\blankline
\noindent
Uniqueness follows from the uniqueness in the previous proposition.
\end{proof}
\end{prp}

There is a `dual' version of the initial topology, in which the functions map \emph{into} the set on which we would like to define a topology.
\begin{prp}[Final topology]\label{FinalTopology}
Let $X$ be a set, let $\mathcal{Y}$ be an indexed collection of topological spaces, and for each $Y\in \mathcal{Y}$ let $f_Y:Y\rightarrow X$ be a function.  Then, there exists a unique topology $\mathcal{U}$ on $X$, the \emph{final topology}\index{Final topology} with respect to $\{ f_Y:Y\in \mathcal{Y}\}$, such that
\begin{enumerate}
\item $f_Y:Y\rightarrow X$ is continuous with respect to $\mathcal{U}$ for all $Y\in \mathcal{Y}$; and
\item if $\mathcal{U}'$ is another topology for which each $f_Y$ is continuous, then $\mathcal{U}\supseteq \mathcal{U}'$.
\end{enumerate}
Furthermore,
\begin{equation}
\mathcal{U}=\{ U\in 2^X:f_Y^{-1}(U)\text{ is open for all }Y\in \mathcal{Y}\text{.}\} .
\end{equation}
\begin{rmk}
In other words, the final topology is the largest topology for which each $f_Y$ is continuous.
\end{rmk}
\begin{rmk}
But what about the smallest such topology?  Well, the smallest such topology is always going to be the indiscrete topology, which is not very interesting.  This is how you remember whether the final topology is the smallest or largest---it can't be the smallest because the indiscrete topology always works.
\end{rmk}
\begin{proof}
Define
\begin{equation}\label{3.4.9}
\mathcal{U}\coloneqq \{ U\in 2^X:f_Y^{-1}(U)\text{ is open for all }Y\in \mathcal{Y}\text{.}\} .
\end{equation}
\begin{exr}
Check that $\mathcal{U}$ is actually a topology.
\begin{rmk}
We didn't need to do any such checking in the construction of the initial topology because there we just took the topology \emph{generated} by the collection.
\end{rmk}
\begin{rmk}
I am not aware of `dual' to \cref{prp4.4.5} that characterizes convergence in final topologies.
\end{rmk}
\end{exr}
Of course every $f_Y$ is continuous with respect to $\mathcal{U}$ as, by definition, the preimage of every element of $\mathcal{U}$ is open.  Furthermore, anything larger than $\mathcal{U}$ would necessarily have to contain some set for which the preimage under some $f_Y$ would not be open, and so that $f_Y$ would not be continuous.
\begin{exr}
Show that the final topology is unique.
\end{exr}
\end{proof}
\end{prp}
There is likewise a `dual' result to \cref{prp3.4.6} which tells us when continuous functions \emph{on} a space equipped with a final topology are continuous.
\begin{prp}\label{prp3.4.34x}
Let $X$ have the final topology with respect to the collection $\{ f_Y:Y\in \mathcal{Y}\}$, let $Z$ be a topological space, and let $f:X\rightarrow Z$ be a function.  Then, $f$ is continuous iff $f\circ f_Y$ is continuous for all $Y\in \mathcal{Y}$.  Furthermore, the final topology is the unique topology with this property.
\begin{proof}
$(\Rightarrow )$ Suppose that $f$ is continuous.  Then, because each $f_Y$ is itself continuous and compositions of continuous functions are continuous, it follows that $f\circ f_Y$ is continuous for all $Y\in \mathcal{Y}$.

\blankline
\noindent
$(\Leftarrow )$ Suppose that $f\circ f_Y$ is continuous for all $Y\in \mathcal{Y}$.  Let $U\subseteq Z$ be open.  We must show that $f^{-1}(U)$ is open.  However, from \eqref{3.4.9}, we know that $f^{-1}(U)$ will be open iff $f_Y^{-1}\left( f^{-1}(U)\right)$ will be open for all $Y\in \mathcal{Y}$.  However, $f_Y^{-1}\left( f^{-1}(U)\right) =[f\circ f_Y]^{-1}(U)$ is open because $f\circ f_Y$ is continuous.

\blankline
\noindent
Let $\mathcal{U}$ be another topology that has the property that $f:X\rightarrow Z$ is continuous iff $f\circ f_Y$ is continuous for all $Y\in \mathcal{Y}$.  To show that $\mathcal{U}$ is the final topology, by the definition (\cref{FinalTopology}), it suffices to show that each $f_Y:Y\rightarrow X$ is continuous with respect to $\mathcal{U}$ and that any other such topology is contained in $\mathcal{U}$.  As $\id _X:\coord{X,\mathcal{U}}\rightarrow \coord{X,\mathcal{U}}$ is continuous, by hypothesis, it follows that $\id _X\circ f_Y=f_Y$ is continuous with respect to $\mathcal{U}$.  Let $\mathcal{U}'$ be another topology such that $f_Y$ is continuous with respect to $\mathcal{U}'$ for all $Y\in \mathcal{Y}$.  We must show that $\mathcal{U}'\subseteq \mathcal{U}$.  To show this, it suffices to show that $\id _X:\coord{X,\mathcal{U}}\rightarrow \coord{X,\mathcal{U}'}$ is continuous.  By the defining property of $\mathcal{U}$, to show this, it suffices to show that the composition of this with each $f_Y$ is continuous.  In other words, it suffices to show that each $f_Y$ is continuous with respect to $\mathcal{U}'$, but this is true by hypothesis.
\end{proof}
\end{prp}

\subsection{Summary}

We now quickly recap all the ways in which we know how to specify a topology on a set.
\begin{enumerate}
\item We can specify the open sets (\cref{TopologicalSpace})..
\item We can specify the closed sets (\cref{exr4.1.2}).
\item We can specify a base (\cref{Base}).
\item We can specify a neighborhood base (\cref{NeighborhoodBase}).
\item We can generate a topology (\cref{GeneratingCollection}).
\item We can define the closure of each set (\cref{KuratowskisClosureTheorem}).
\item We can define the interior of each set (\cref{KuratowskisInteriorTheorem}).
\item We can define convergence of nets (\cref{KelleysConvergenceTheorem}).
\item We can define convergence of filters (\cref{KelleysFilterConvergenceTheorem}).
\item We can declare that functions on the space are continuous (the initial topology---see \cref{InitialTopology}).
\item We can declare that functions into the space are continuous (the final topology---see \cref{FinalTopology}.
\end{enumerate}

\section{The subspace, quotient, product, and disjoint-union topologies}

The purpose of this section is to present several ways of constructing new topologies spaces from old.  In brief, the subspace topology will be the topology we put on subsets of topological spaces, the quotient topology will be the topology we put on quotients of topological spaces,\footnote{Quotients in the sense of \cref{dfnA.1.42}.}, the product topology is the topology we put on cartesian-products, the disjoint-union topology (surprise, surprise) is the topology we put on disjoint-unions of topological spaces.  They key to defining all of these topologies are the initial (for the subspace and product topologies) and final topologies (for the quotient and disjoint-union topologies) (\cref{InitialTopology,FinalTopology}).

\subsection{The subspace topology}

All \emph{subsets} of topological spaces have a canonically associated topology, called the \emph{subspace topology}.  Note that this is not completely immediate---for example, it is not the case that every subset of a ring is a ring.  There is definitely something to define and something to check (that the subspace topology is in fact a topology).
\begin{prp}[Subspace topology]\label{SubspaceTopology}
Let $X$ be a topological space and let $S\subseteq X$.  Then, there exists a unique topology $\mathcal{U}$ on $S$, the \emph{subspace topology}\index{Subspace topology}, that has the property that a function into $S$ is continuous iff it is continuous regarded as a function into $X$.  Furthermore, the subspace topology is the initial topology with respect to the inclusion $\iota :S\hookrightarrow X$.  In particular,
\begin{equation}
\mathcal{U}=\{ U\cap S:U\subseteq X\text{ is open.}\} .
\end{equation}
\begin{rmk}
Unless otherwise stated, subsets of topological spaces are always equipped with the subspace topology.
\end{rmk}
\begin{proof}
All of this follows from the definition of the initial topology and its characterization of continuity of continuity of functions into initial topologies (\cref{InitialTopology,prp3.4.6}), with exception of the fact that $\mathcal{U}=\{ U\cap S:U\subseteq X\text{ is open.}\}$.  \cref{InitialTopology} tells us that this generates the initial topology, but it does not tell us that it is a topology itself.  Of course, however, if a generating collection is itself a topology, then the topology it generates is just itself.  Therefore, it suffices just to check that $\{ U\cap S:U\subseteq X\text{ is open.}\}$ is in fact a topology.
\begin{exr}
Check that $\{ U\cap S:U\subseteq X\text{ is open.}\}$ is in fact a topology.
\end{exr}
\end{proof}
\end{prp}
\begin{exm}\label{exm4.1.14}
Consider the subspace topology on $[0,1]\subseteq \R$.  Note that $(\frac{1}{2},1]=(\frac{1}{2},\infty )\cap [0,1]$ is \emph{open in $[0,1]$}, despite the fact that it obviously not open in $\R$.
\end{exm}
\begin{exm}
Note that the order topology on $\Q$ is the space as the subspace topology inherited from $\R$.  This is of course because they are both equipped with the order topology of the same order.  Likewise, for $\N \subseteq \Z$ and $\Z \subseteq \Q$.
\end{exm}

\subsection{The quotient topology}

Whenever we have a surjective function from a topological space $X$ onto a set $Y$, we can use this function and the topology on $X$ to place a topology on $Y$.  Recall (\cref{exrA.1.81}) that every surjective function can be viewed as a quotient function---this of course is the etymology of the term ``quotient topology''.
\begin{prp}
Let $X$ be a topological space, let $Y$ be a set, and let $\q :X\rightarrow Y$ be surjective.  Then, there exists a unique topology $\mathcal{U}$ on $Y$, the \emph{quotient topology}\index{Quotient topology}, that has the property that a function on $Y$ is continuous iff its composition with $\q$ is continuous.  Furthermore, the subspace topology is the final topology with respect to $\q :X\rightarrow Y$.  In particular,
\begin{equation}
\mathcal{U}=\{ U\subseteq Y:\q ^{-1}(U)\text{ is open.}\} .
\end{equation}
\begin{rmk}
Unless otherwise stated, quotients of topological spaces are always equipped with the quotient topology.
\end{rmk}
\begin{proof}
All of this follows from the definition of the final topology and its characterization in terms of continuity of functions defined on final topologies (\cref{FinalTopology,prp3.4.34x}).
\end{proof}
\end{prp}

\subsection{The product topology}

The product topology is the canonical topology we put on a cartesian-product of topological spaces $X\times Y$.  While we do technically use this in places, we use it in ways where we could have gotten-away with not speaking of the product topology per se (for example, the product topology on $\R \times \R$ is the same as the topology defined by $\varepsilon$-balls).  The real reason we go to the trouble of talking about the product topology explicitly is for the proof of producing a \emph{counter-example} to the statement
\begin{textequation}
A space is quasicompact iff every net has a convergent \emph{strict} subnet.\footnote{In case you're skimming and didn't read the context, this statement is \emph{false}.}
\end{textequation}
We mentioned when we defined subnets (\cref{Subnet}) that the notion of a strict subnet was the more obvious ``naive'' notion (that is, you just take terms from the original net, subject to the only condition that your indices get arbitrarily large), but that this ``naive'' notion was insufficient because it didn't allow us to prove certain theorems.  They key result, that spaces are quasicompact iff every net has a convergent subnet (\cref{prp4.2.31}), was precisely the example of a theorem we had in mind.

In brief, the counter-example will be
\begin{equation}
X\coloneqq \prod _{2^{\N}}\{ 0,1\} ,
\end{equation}
that is, an uncountable product (precisely, a product over the power set of $\N$) of the two-element set $\{ 0,1\}$.  Obviously $\{ 0,1\}$ is quasicompact (all finite spaces are)---Tychnoff's Theorem is the statement that \emph{arbitrary} products of quasicompact spaces are quasicompact, which tells us that $X$ is quasicompact.

So before we do anything else then, we must first define the product topology.
\begin{prp}[Product topology]\label{ProductTopology}
Let $\mathcal{X}$ be an indexed collection of topological spaces.  Then, there exists a unique topology, the \emph{product topology}\index{Product topology}, on $\prod _{X\in \mathcal{X}}X$, that has the property that a function into $\prod _{X\in \mathcal{X}}X$ is continuous iff each component of the function is continuous.  Furthermore, the product topology is the initial topology with respect to $\{ \pi _X:X\in \mathcal{X}\}$.\footnote{Recall (\cref{CartesianProductCollection}) that $\pi _X:\prod _{X\in \mathcal{X}}X\rightarrow X$ is just the projection.}  In particular, if $\mathcal{S}_X$ is a generating collection for the topology on $X\in \mathcal{X}$, then the collection
\begin{equation}\label{4.5.4x}
\{ \pi _X^{-1}(U):U\in \mathcal{S}_X\}
\end{equation}
generates the product topology, so that
\begin{equation}\label{4.5.4}
\mathcal{B}\coloneqq \left\{ \prod _{X\in \mathcal{X}}S_X:S_X\in \mathcal{S}_X\text{ and all but finitely many }S_X=X\right\} \footnote{We can always without loss of generality assume that $X\in \mathcal{B}_X$ because, if it wasn't there before, throw it in.}
\end{equation}
is a base for the product topology.
\begin{rmk}
The ``all but finitely many'' phrase is \emph{crucial}.  For example, in the space
\begin{equation}
\prod _\N \R ,
\end{equation}
that is, a countably-infinite product of $\R$, the set
\begin{equation}
(0,1)\times (0,1)\times (0,1)\times \cdots 
\end{equation}
is \emph{not} even open in the product topology on $\prod _\N \R$ (much less an element of any base).  This topology (called the \emph{box topology}\index{Box topology}) might be your more naive guess, but it is `wrong' in the sense that, if we allow things like this, then we lose the property that the continuity of a function is determined by the continuity of the components of the function---see \cref{exm4.5.20} below.
\end{rmk}
\begin{rmk}
Unless otherwise stated, products of topological spaces are always equipped with the product topology.
\end{rmk}
\begin{proof}
All of this follows from the definition of the initial topology, its characterization in terms of continuity of functions into initial topologies, and the defining result of generating collections (\cref{InitialTopology,prp3.4.6,GeneratingCollection}).
\end{proof}
\end{prp}
We mentioned in the remarks of this theorem that if you take as a base sets of the form $\prod _{X\in \mathcal{X}}S_X$ (with $S_X\neq X$ \emph{for all $X\in \mathcal{X}$} permissible), then you will lose the property that a function into the product is continuous iff each of its components is.  We now present a counter-example.
\begin{exm}[A discontinuous function into the box topology with each component continuous]\label{exm4.5.20}
Define \emph{as a set}
\begin{equation}
X\coloneqq \prod _{2^\N}\R 
\end{equation}
and consider the function
\begin{equation}
\id _X:\coord{X,\text{product topology}}\rightarrow \coord{X,\text{box topology}}.
\end{equation}
Then, each component of this function is continuous because the component of the identity is just the projection from $X$ onto the corresponding copy of $\R$ (the preimage of $U\subseteq \R$ under this projection is open in the product topology because in fact, according to the theorem, such an element is in a generating collection of the product topology).  On the other hand,
\begin{equation}
\prod _{2^\N}(0,1)
\end{equation}
is open in the box topology (by definition), but not open in the product topology by the theorem above.
\end{exm}
There is a relatively useful corollary of the definition of the product topology that characterizes convergence.
\begin{crl}\label{crl4.5.15}
Let $\mathcal{X}$ be an indexed collection of topological spaces, let $\lambda \mapsto x_\lambda \in \prod _{X\in \mathcal{X}}X$ be a net, and let $x_\infty \in \prod _{X\in \mathcal{X}}X$.  Then, $\lambda \mapsto x_\lambda$ converges to $x_\infty$ iff $\lambda \mapsto (x_\lambda )_X$ converges to $(x_\infty )_X$ in $X$ for all $X\in \mathcal{X}$.
\begin{rmk}
In other words, nets converge to an element in a product iff every component converges to the corresponding component of that element.
\end{rmk}
\begin{proof}
As the product topology on $\prod _{X\in \mathcal{X}}X$ is the initial topology with respect to the projections, $\{ \pi _X:X\in \mathcal{X}\}$, this result follows from \cref{prp4.4.5}.
\end{proof}
\end{crl}
\begin{exr}[Projections are open]\label{ProjectionsAreOpen}
Let $\mathcal{X}$ be an indexed collection of topological spaces, let $X\in \mathcal{X}$, and let $U\subseteq \prod _{X\in \mathcal{X}}X$ be open.  Show that $\pi _X(U)\subseteq X$ is open.
\begin{rmk}
Functions that have the property that the image of open sets are open are \emph{open functions}\index{Open function}.
\end{rmk}
\end{exr}

Now that we have defined the product topology, we prove a relatively deep result concerning quasicompactness of products which will allow us to produce the desired counter-example.
\begin{thm}[Tychnoff's Theorem]\index{Tychnoff's Theorem}\label{TychnoffsTheorem}
Let $\mathcal{X}$ be a collection of quasicompact spaces.  Then, $\prod _{X\in \mathcal{X}}X$ is quasicompact.
\begin{proof}\footnote{Proof adapted from \cite[pg.~143]{Kelley}.}
Recall that the product topology on $\prod _{X\in \mathcal{X}}X$ has a generating collection of the form $\pi _X^{-1}(U_X)$ for $U_X\subseteq X$ open, where $\pi _X:\prod _{X\in \mathcal{X}}X\rightarrow X$ is the projection.  We apply Alexander's Subbase Theorem (\cref{AlexanderSubbaseTheorem}) to this generating collection.

So, let $\mathcal{U}$ be an open cover of $\prod _{X\in \mathcal{X}}X$ of subsets of the form $\pi _X^{-1}(U_X)$.  It suffices to show that if no finite subset of $\mathcal{U}$ covers $\prod _{X\in \mathcal{X}}X$, then $\mathcal{U}$ itself does not cover $X$.  Define
\begin{equation}
\mathcal{U}_X\coloneqq \left\{ U\subseteq X\text{ open}:\pi _X^{-1}(U)\in \mathcal{U}\right\} .
\end{equation}
If a finite subset of $\mathcal{U}_X$ covered $X$, then its preimage would cover $\prod _{X\in \mathcal{X}}X$, and so by quasicompactness of $X$, it follows that $\mathcal{U}_X$ does not cover $X$, so choose $x_X\in X$ not contained in $\bigcup _{U\in \mathcal{U}_X}U$.  Then, the element $x\in \prod _{X\in \mathcal{X}}X$ whose coordinate at $X\in \mathcal{X}$ is $x_X$ is not contained in $\bigcup _{U\in \mathcal{U}}U$.
\end{proof}
\end{thm}

And finally we are able to present our counter-example.
\begin{exm}[A quasicompact space with a net that has no convergent \emph{strict} subnet]\footnote{This example was shown to me by Eric Wofsey on \href{http://mathoverflow.net/questions/210947/a-quasicompact-space-with-a-net-that-contains-no-convergent-strict-subnet}{mathoverflow.net}.}
Define
\begin{equation}
X\coloneqq \prod _{S\subseteq \N}\{ 0,1\} =\{ 0,1\} ^{2^\N}
\end{equation}
that is, a product of $2^{\aleph _0}$ copies of the two element set $\{ 0,1\}$.  In other words, it is the set of all functions from $2^{\N}$ into $\{ 0,1\}$---in fact, for the most of this example, we shall think of this space as the collection of functions.   This is quasicompact by Tychnoff's Theorem (and because finite sets are quasicompact---see \cref{exr4.2.33x}).  On the other hand, we may define a sequence $m\mapsto x_m\in X$ as follows.
\begin{equation}
x_m(S)\coloneqq \begin{cases}1 & \text{if }m\in S \\ 0 & \text{if }m\notin S\end{cases},
\end{equation}
where $S\subseteq \N$ ($x_m$ is a function from $2^\N$ into $\{ 0,1\}$, and so $x_m(S)$ is the value of this function at the element $S\in 2^\N$).

We now show that this sequence has no convergent strict subnet (necessarily also a sequence).  We proceed by contradiction:  suppose that there were a cofinal subset $\Lambda '\subseteq \N$, $\Lambda '=\{ m_0,m_1,m_2,\ldots \}$ (with $m_n\leq m_{n+1}$), such that $n\mapsto x_{m_n}$ converges.  Then, by \cref{crl4.5.15} (nets in products converge iff each component does), for each $S\subseteq \N$, the sequence $n\mapsto x_{m_n}(S)\in \{ 0,1\}$ would have to converge.  Thus, for each $S\subseteq \N$, the sequence $n\mapsto x_{m_n}(S)$ would have to be either eventually $0$ or eventually $1$.  In other words, either (i) for all but finitely many $m_n\in \Lambda '$, $m_n\notin S$;  or (ii) for all but finitely many $m_n\in \Lambda '$, $m_n\in S$.  In other words, for all $S\subseteq \N$, either (i) there is a cofinite\footnote{\emph{Cofinite}\index{Cofinite} means that the complement is finite.} subset of $\Lambda '$ that is contained in $S^{\comp}$ or (ii) there is a cofinite subset of $\Lambda '$ that is contained in $S$.\footnote{Cofinite in $\Lambda '$, that is.}

So, take $S\coloneqq \{ m_0,m_2,m_4,\ldots \} \subset \Lambda '$.  Then, there is some cofinite subset $\Lambda ''\subseteq \Lambda '$ such that either $\Lambda ''\subseteq S$ or $\Lambda ''\subseteq S^{\comp}$.  In the former case, we have that
\begin{equation}
\text{finite set }=\Lambda '\setminus \Lambda '' \supseteq \Lambda '\setminus S=\{ m_1,m_3,m_5,\ldots \} :
\end{equation}
a contradiction.  Thus, we must have that $\Lambda ''\subseteq S^{\comp}$, and so
\begin{equation}
\text{finite set }=\Lambda '\setminus \Lambda ''\supseteq \Lambda '\setminus S^{\comp}=\{ m_0,m_2,m_4,\ldots \} :
\end{equation}
a contradiction.  As both possibilities resulted in a contradiction, this itself is a contradiction, and so our assumption that there was a convergent strict subnet must have been incorrect.  Therefore, $m\mapsto x_m$ contains no convergent strict subnet, despite the fact that $X$ is quasicompact.
\end{exm}

\subsection{The disjoint-union topology}

Our inclusion of the disjoint-union topology is mostly because of its obvious duality to the product topology---it feels incomplete not to include it.  On the other hand, while in principle, being completely dual to the product topology, it should be no more or less difficult work with, in practice it seems that it is \emph{much} easier to get a handle on, and so probably doesn't deserve as in-depth a treatment.
\begin{prp}[Disjoint-union topology]\label{DisjointUnionTopology}
Let $\mathcal{X}$ be an indexed collection of topological spaces.  Then, there exists a unique topology, the \emph{disjoint-union topology}\index{disjoint-union topology}, on $\coprod _{X\in \mathcal{X}}X$, that has the property that a function defined on $\coprod _{X\in \mathcal{X}}X$ is continuous iff its restriction to each component is continuous.  Furthermore, the disjoint-union topology is the final topology with respect to $\{ \iota _X:X\in \mathcal{X}\}$.\footnote{Recall (\cref{DisjointUnionCollection}) that $\iota _X:X\rightarrow \coprod _{X\in \mathcal{X}}X$ is just the inclusion.}  In particular, a set $U\subseteq \coprod _{X\in \mathcal{X}}X$ is open iff $U\cap \iota _X(X)$ is open for all $X\in \mathcal{X}$.
is a base for the product topology.
\begin{rmk}
Unless otherwise stated, disjoint-unions of topological spaces are always equipped with the disjoint-union topology.
\end{rmk}
\begin{proof}
All of this follows from the definition of the final topology and its characterization in terms of continuity of functions defined on final topologies (\cref{FinalTopology,prp3.4.34x}).
\end{proof}
\end{prp}

\section{Separation Axioms}\label{sct4.5}

We have mentioned the term ``$T_2$'' a couple of times now (see, for example, the definition of quasicompactness (\cref{Quasicompact}) and the \nameref{HeineBorelTheorem} (\cref{HeineBorelTheorem})).  One of the purposes of this section is to explain what we meant by this.  The term ``$T_2$'' is a separation axiom, and you should know the other separation axioms as well if you plan to become a mathematician, but this is admittedly not a priority for this course (a study of them in detail is better suited for a course on general topology itself).

\subsection{Separation of subsets}

In this subsection, we will define various levels of ``sepration'' of subsets of a topological space.  In the next section, we will then define corresponding levels of separation for spaces, which is roughly the condition that any two points have the corresponding level of separation.\footnote{If that doesn't make sense now, don't worry---I can't be completely clear just yet.  Worry if it doesn't make sense after the next subsection ;-).}

Throughout this subsection, let $S_1,S_2\subseteq X$ be \emph{disjoint} subsets of a topological space $X$.  In the various definitions will follow, we will say things like ``$S_1$ and $S_2$ are XYZ.''.  If $S_1=\{ x_1\}$ and $S_2=\{ x_2\}$ are singletons, then instead we will say that ``$x_1$ and $x_2$ are XYZ.''.  In fact, this is probably the case of most interest (though certainly not the only case).
\begin{dfn}[Topologically-distinguishable]\label{TopologicallyDistinguishable}
$S_1$ and $S_2$ are \emph{topologically-distinguish\-able}\index{Topologically-distinguishable} iff there is an open set containing $S_1$ but not $S_2$ or vice-versa.
\begin{rmk}
In other words, they do not have precisely the same open neighborhoods.
\end{rmk}
\begin{rmk}
For example, in an indiscrete space, no two points are topologically-distinguishable.  In $\R$ on the other hand (and almost every space you work with that is not expressly cooked-up for the sole purpose of being a counter-example to something), any two points are topologically-distinguishable.
\end{rmk}
\end{dfn}
\begin{exm}[Two distinct points which are not topologically-distinguishable]\label{exm4.5.2}
We just mentioned this in the remark above, but decided to place it in an example of its own to make it easier to spot if skimming.  Take $X\coloneqq \{ x_1,x_2\}$ and equip $X$ with the indiscrete topology, that is,
\begin{equation}
\mathcal{U}\coloneqq \{ \emptyset ,X\} .
\end{equation}
Then, $x_1\neq x_2$ but $x_1$ and $x_2$ are contained in precisely the same open sets.
\end{exm}
\begin{dfn}[Separated]\label{Separated}
$S_1$ and $S_2$ are \emph{separated}\index{Separated} iff there is an open neighborhood $U_1$ of $S_1$ not intersecting $S_2$ \emph{and} an open neighborhood $U_2$ of $S_2$ not intersecting $S_1$.
\begin{rmk}
The difference between this and topological-distinguishability is that this has to happen to \emph{both} $S_1$ and $S_2$, whereas, in the case of topological-distinguishability, we only require that at least one of them has an open neighborhood that does not intersect the other.
\end{rmk}
\end{dfn}
\begin{exm}[Two points which are topologically-distinguishable but not separated]\label{exm4.5.3}
Define $X\coloneqq \{ x_1,x_2\}$ and
\begin{equation}
\mathcal{U}\coloneqq \left\{ \emptyset ,X,\{ x_1\} \right\} .
\end{equation}
Then, $x_1$ and $x_2$ are topologically-distinguishable as $\{ x_1\}$ is an open neighborhood of $x_1$ that does not contain $x_2$.  On the other hand, every neighborhood of $x_2$ contains $x_1$.
\end{exm}
\begin{dfn}[Separated by neighborhoods]\label{SeparatedByNeighborhoods}\index{Separated by neighborhoods}
$S_1$ and $S_2$ are \emph{separated by neighborhoods}\index{Separated by neighborhoods} iff there is a neighborhood $U_1$ of $S_1$ and a neighborhood $U_2$ of $S_2$ with $U_1$ and $U_2$ disjoint.
\begin{rmk}
Equivalently, we may replace $U_1$ and $U_2$ with open neighborhoods.
\end{rmk}
\begin{rmk}
This is just like being separated, except that we may put $U_1$ around $S_1$ and $U_2$ around $S_2$ \emph{`simultaneously'} and have no intersection, whereas in the separated case, the $U_1$ and $U_2$ that `work' will in general intersect.
\end{rmk}
\end{dfn}
\begin{exm}[Two points which are separated but not separated by neighborhoods]\label{exm4.5.8}
Define $X\coloneqq \{ x_1,x_2,x_3\}$ and
\begin{equation}
\mathcal{U}\coloneqq \left\{ \emptyset ,X,\{ x_1,x_3\} ,\{ x_2,x_3\} ,\{ x_3\} \right\} .
\end{equation}
Then, $\{ x_1,x_3\}$ is an open neighborhood of $x_1$ which does not contain $x_2$ and $\{ x_2,x_3\}$ is an open neighborhood of $x_2$ which does not contain $x_1$.  On the other hand, every open neighborhood of $x_1$ intersects every open neighborhood of $x_2$ at $x_3$.
\end{exm}
\begin{dfn}[Separated by closed neighborhoods]\label{SeparatedByClosedNeighborhoods}
$S_1$ and $S_2$ are \emph{separated by closed neighborhoods}\index{Separated by closed neighborhoods} iff there is a closed neighborhood $C_1$ of $S_1$ and a closed neighborhood $C_2$ of $S_2$ with $C_1$ and $C_2$ disjoint.
\begin{rmk}
This is the same as being separated by neighborhoods, except that we can further require that the neighborhoods are closed.\footnote{Recall that neighborhoods do not have to be open---see \cref{Neighborhood}.}
\end{rmk}
\end{dfn}
\begin{exm}[Two points which are separated by neighborhoods but not by closed neighborhoods]\label{exm4.5.11}
Define $X\coloneqq \{ x_1,x_2,x_3\}$ and
\begin{equation}
\mathcal{U}\coloneqq \left\{ \emptyset ,X,\{ x_1\} ,\{ x_3\} ,\{ x_1,x_3\} \right\} .
\end{equation}
Then, $\{ x_1\}$ is a neighborhood of $x_1$, $\{ x_3\}$ is a neighborhood of $x_3$, and these two neighborhoods are disjoint, so that $x_1$ and $x_3$ are separated by neighborhoods.  On the other hand, $x_1$ only has three neighborhoods:  $\{ x_1\}$, $\{ x_1,x_3\}$, and all of $X$.  The latter must intersect every neighborhood of $x_3$ and the former is not closed because $\{ x_1\} ^{\comp}=\{ x_2,x_3\}$ is not open.  Therefore, we cannot separate $x_1$ and $x_3$ with \emph{closed} neighborhoods.
\end{exm}
There is an equivalent, alternative way to think about being separated by closed neighborhoods that you may find useful.
\begin{prp}\label{prp4.5.13}
$S_1$ and $S_2$ are separated by closed neighborhoods iff they are separated by open neighoborhoods with disjoint closure.
\begin{proof}
$(\Rightarrow )$ Suppose that $S_1$ and $S_2$ are separated by closed neighborhoods $C_1$ and $C_2$ respectively.  By the definition of a neighborhood (\cref{Neighborhood}), there are then open neighborhoods $U_1$ and $U_2$ with $S_1\subseteq U_1\subseteq C_1$ and $S_2\subseteq U_2\subseteq C_2$.  Then, $\Cls (U_1)\subseteq C_1$ and $\Cls (U_2)\subseteq C_2$, and so $U_1$ and $U_2$ constitute open neighborhoods of $S_1$ and $S_2$ with disjoint closures.

\blankline
\noindent
$(\Leftarrow )$ Suppose that $S_1$ and $S_2$ are separated by open neighborhoods with disjoint closure.  Then, these closure constitute closed neighborhoods which separate $S_1$ and $S_2$.
\end{proof}
\end{prp}
\begin{dfn}[Completely-separated]\label{CompletelySeparated}
$S_1$ and $S_2$ are \emph{completely-separated}\index{Completely-separated} or \emph{separated by (continuous) functions}\index{Separated by continuous functions} iff there is a continuous function $f:X\rightarrow [0,1]$ such that $\restr{f}{S_1}=0$ and $\restr{f}{S_2}=1$.
\begin{rmk}
Why does being completely-separated imply being separated by closed neighborhoods?\footnote{All the other implications are true too (i.e.~separated implies topologically-distinguishable, separated by neighborhoods implies separated, etc.), this is just the first one that is not completely obvious, which is why it is the only one we have asked about.}
\end{rmk}
\end{dfn}
\begin{exm}[Two points which are separated by closed neighborhoods but not completely-separated]\label{ArensSquare}\footnote{This is significantly more nontrivial than the preceding counter-examples and comes from \cite[pg.~98]{Steen}.}
Define
\begin{equation}
S\coloneqq \left( (0,1)\times (0,1)\right) \cap (\Q \times \Q )=\left\{ \coord{x,y}\in (0,1)\times (0,1):x,y\in \Q \right\} ,
\end{equation}
\begin{equation}
T\coloneqq \{ \tfrac{1}{2}\} \times \{ r\sqrt{2}:r\in \Q \} =\{ \coord{\tfrac{1}{2},r\sqrt{2}}:r\in \Q \} 
\end{equation}
and
\begin{equation}
X\coloneqq S\cup T\cup \{ \coord{0,0})\} \cup \{ \coord{1,0})\}.
\end{equation}
We define a topology on $X$ by defining a neighborhood base at each point (see \cref{prp4.1.8}).  For $\coord{x,y}\in X$, there are four cases:  (i) $\coord{x,y}=\coord{0,0}$, (ii) $\coord{x,y}=\coord{1,0}$, (iii) $\coord{x,y}\in T$, and (iv) $\coord{x,y}\in S$.  We define
\begin{equation}
\mathcal{B}_{\coord{x,y}}\coloneqq \begin{cases}\left\{ U\subseteq S:U\text{ is open in }S\text{.}\footnote{Open in the usual topology (the subspace topology inherited from $(0,1)\times (0,1)$).}\right\} & \text{if }\coord{x,y}\in S \\ \left\{ U_{\coord{x,y}}^m:m\in \Z ^+\right\} & \text{if }\coord{x,y}\in T\cup \{ \coord{0,0}\} \cup \{ \coord{1,0}\} \end{cases},
\end{equation}
where
\begin{equation}\label{4.5.20}
\begin{split}
U_{\coord{0,0})}^m & \coloneqq \{ \coord{0,0}\} \cup \left\{ \coord{x,y}\in (0,\tfrac{1}{4})\times (0,\tfrac{1}{m}):x,y\in \Q \right\} \\
U_{\coord{1,0}}^m & \coloneqq \{ \coord{1,0}\} \cup \left\{ (x,y)\in (\tfrac{3}{4},1)\times (0,\tfrac{1}{m}):x,y\in \Q \right\} \\
U_{\coord{\tfrac{1}{2},r\sqrt{2}}}^m & \coloneqq \left\{ \coord{x,y}\in (\tfrac{1}{4},\tfrac{3}{4})\times (r\sqrt{2}-\tfrac{1}{m},r\sqrt{2}+\tfrac{1}{m}):x,y\in \Q \right\} .
\end{split}
\end{equation}
By \cref{prp4.1.8}, there is a unique topology for which $\mathcal{B}_{\coord{x,y}}$ is a neighborhood base of $\coord{x,y}\in X$.

The closures of $\{ \coord{0,0}\} \cup (0,\frac{1}{4})\times (0,\frac{1}{n})$ and $\{ \coord{1,0}\} \cup (\frac{3}{4},1)\times (0,\frac{1}{n})$ in $X$ must be disjoint as, in particular, any point in the former cannot have $x$-coordinate exceeding $\frac{1}{4}$ and any point in the latter cannot have $x$-coordinate strictly less than $\frac{3}{4}$.

On the other hand, $\coord{0,0}$ and $\coord{1,0}$ cannot be separated by a function.  To see this, suppose that $f:X\rightarrow \R$ were a continuous function such that $f(\coord{0,0})=0$ and $f(\coord{1,0})=1$.  Then, $f^{-1}([0,\frac{1}{4}))$ would be an open neighborhood of $\coord{0,0}$, and so must contain $U_{\coord{0,0}}^m$ for some $m\in \Z ^+$.  Similarly, $f^{-1}((\frac{3}{4},1])$ must contain $U_{\coord{1,0}}^n$ for some $n\in \Z ^+$.  Let $r\in \Q$ be such that $r\sqrt{2}<\min \{ \frac{1}{m},\frac{1}{n}\}$.  Obviously, $f(\coord{\frac{1}{2},r\sqrt{2}})$ cannot be in both $[0,\frac{1}{4})$ and $(\frac{3}{4},1]$ as these sets are disjoint, so without loss of generality assume that it is not contained in $[0,\frac{1}{4})$, so let $U \subseteq [0,1]$ be an open neighborhood of $f(\coord{\frac{1}{2},r\sqrt{2}})$ with $\Cls (S)$ disjoint from $\Cls \left( [0,\frac{1}{4})\right) $, so that the preimages of $\Cls (U)$ and $\Cls \left( [0,\frac{1}{4}\right)$ are disjoint closed neighborhoods of $\coord{\frac{1}{2},r\sqrt{2}}$ and $\coord{0,0}$ respectively.  On the other hand, a disjoint closed neighborhood of $\coord{\frac{1}{2},r\sqrt{2}}$ must contain $U_{\coord{\frac{1}{2},r\sqrt{2}}}^o$ for $o\in \Z ^+$ with $r\sqrt{2}-\frac{1}{o}>0$.  As $r<\frac{1}{m}$, we have that $\frac{1}{o}<\frac{\sqrt{2}}{m}$.  But then, $\Cls \left( U_{\coord{0,0}}^m\right)$ and $\Cls (U_{\coord{\frac{1}{2},r\sqrt{2}}}^o)$ must intersect at $\coord{\frac{1}{4},r\sqrt{2}-\frac{1}{o}}$ because $r\sqrt{2}-\frac{1}{o}<\frac{1}{m}-\frac{1}{o}<\frac{1}{m}$.
\begin{rmk}
This is the \emph{Arens Square}\index{Arens Square}.
\end{rmk}
\end{exm}
\begin{dfn}[Perfectly-separated]
$S_1$ and $S_2$ are \emph{perfectly-separated}\index{Perfectly-separated} or \emph{precisely-separated}\index{Precisely-separated} iff there is a continuous function $f:X\rightarrow [0,1]$ such that $S_1=f^{-1}(0)$ and $S_2=f^{-1}(1)$.
\begin{rmk}
Being completely-separated means that $f$ is $0$ on $S_1$ and $1$ on $S_2$.  Being perfectly-separated means that, furthermore, $f$ is $0$ \emph{nowhere else} except on $S_1$ and $f$ is $1$ \emph{nowhere else} except on $S_2$.
\end{rmk}
\end{dfn}
\begin{exm}[Two points which are completely-separated but not perfectly-separated]\footnote{This comes from \cite[pg.~52]{Steen}.}\label{UncountableFortSpace}
This example is fairly similar to the cocountable topology example---see \cref{exm4.2.8x}.

Define $X\coloneqq \R$.  Let $C\subseteq X$ and declare that
\begin{textequation}
$X$ is closed iff either (i) $C$ contains $0$ or (ii) $C$ is finite.
\end{textequation}
You can check for yourself that this satisfies the defining conditions for a topology in terms of closed sets (\cref{exr4.1.2}).

We wish to show that $0,1\in X$ are completely-separated, but not perfectly-separated.  We first check that they are completely-separated by producing a continuous function $f:X\rightarrow [0,1]$ such that $f(0)=0$ and $f(1)=1$.  Define
\begin{equation}
f(x)\coloneqq \begin{cases}1 & \text{if }x=1 \\ 0 & \text{otherwise}\end{cases}.
\end{equation}
We first check that $f$ is continuous.  Let $C\subseteq [0,1]$ be closed.  If $C$ contains $0\in [0,1]$, then $f^{-1}(C)$ contains $0\in X$, and so is closed.  If it does not contain $0$, then $f^{-1}(C)$ is finite---either $C$ contained $1$ in which case $f^{-1}(C)=\{ 1\}$ or it did not in which case $f^{-1}(C)=\emptyset$.

Now we show that $0,1\in X$ are not \emph{perfectly} separated.  Suppose that there exists a continuous function $f:X\rightarrow [0,1]$ such that $\{ 0\} =f^{-1}(0)$ and $\{ 1\} =f^{-1}(1)$.  Then,
\begin{equation}\label{4.5.23}
\{ 0\} =f^{-1}(0)=f^{-1}\left( \bigcap _{m\in \Z ^+}[0,\tfrac{1}{m})\right) =\footnote{\cref{exrA.1.30}\ref{enmA.1.30.ii}}\bigcap _{m\in \Z ^+}f^{-1}\left( [0,\tfrac{1}{m})\right) .
\end{equation}
That is, $\{ 0\}$ is a $G_\delta$ set.\footnote{Recall that this is just a fancy-shmancy term for a set which is a countable intersection of open sets---see \cref{GDeltaFSigma}.}  However, by definition, a set is open iff it does not contain $0\in X$ or its complement is finite.  Of course, all the sets appearing in \eqref{4.5.23} must be of the latter kind.  However, taking the complement of this equation, we find
\begin{equation}
\R \setminus \{ 0\} ^{-1}\footnote{De Morgan's Laws---see \cref{DeMorgansLaws}}=\bigcup _{m\in \Z ^+}f^{-1}(0,\tfrac{1}{m})^\comp ,
\end{equation}
so that $\R \setminus \{ 0\}$ is a countably-infinite union of finite sets---a contradiction.
\begin{rmk}
This is the \emph{Uncountable Fort Space}\index{Uncountable Fort Space}.
\end{rmk}
\end{exm}

Note that we obviously have the implications
\begin{textequation}
perfectly-separated $\Rightarrow$ completely-separated $\Rightarrow $ separated by closed neighborhoods $\Rightarrow $ separated by neighborhoods $\Rightarrow$ separated $\Rightarrow$ topologically-distinguishable $\Rightarrow$ distinct.
\end{textequation}
Here, ``distinct'' literally means that $S_1$ and $S_2$ are not the same thing, i.e. $S_1\neq S_2$.  Furthermore, we have presented counter-examples after each definition to show that each implication is strict.

\subsection{Separation axioms of spaces}

In the previous subsection, we defined several levels of ``separation'' between two disjoints subsets of a topological space.  We now use these definitions to put conditions on topological spaces themselves.

Throughout this section, let $X$ be a topological space.
\begin{dfn}[$T_0$]\label{T0}
$X$ is \emph{$T_0$}\index{$T_0$} iff any two distinct points are topologically-distinguishable.
\begin{rmk}
Sometimes this condition is called \emph{kolmogorov}\index{Kolmogorov (topological space)}.  You will find that a lot (if not all) of the separation axioms of spaces have other names.  We have chosen the names we have because (i) other terminology is less consistent and (ii) it carries less information (of course that the subscript $0$ in $T_0$ has some significance).
\end{rmk}
\begin{rmk}
This is an insanely reasonable condition.  I might even argue that if you have a space you're trying to study that is not $T_0$, you're doing something wrong.  If the topology cannot distinguish between two points, either (i) you may as well identify those two points (see \cref{T0Quotient}) or (ii) you should probably consider adding more structure to your space that does distinguish between them.
\end{rmk}
\end{dfn}
\begin{exm}[A space which is not $T_0$]
Any indiscrete space with at least two points.  The example above in \cref{exm4.5.2} worked.
\end{exm}
\begin{prp}[$T_0$ quotient]\label{T0Quotient}
Let $X$ be a topological space.  Then, there exists a unique topological space $\TZero (X)$, the \emph{$T_0$ quotient}\index{$T_0$ quotient} of $X$, and a surjective map $\q :X\rightarrow \TZero (X)$ such that
\begin{enumerate}
\item \label{T0Quotient.i}$\TZero (X)$ is $T_0$; and
\item \label{T0Quotient.ii}if $Y$ is another $T_0$ space with a continuous map $\phi :X\rightarrow Y$, then there is a unique continuous map $\phi ':\TZero (X)\rightarrow Y$ such that $\phi =\phi '\circ \q$.
\end{enumerate}
\begin{rmk}
As $T_0$ is sometimes called kolmogorov, so to this is sometimes called the \emph{kolmogorov quotient}\index{Kolmogorov quotient}.
\end{rmk}
\begin{rmk}
Compare this with the definitions of the integers, rationals, closure, interior, generating collections, initial topology, and final topology (\cref{Integers,RationalNumbers,Closure,Interior,InitialTopology,FinalTopology}).  Note how this is a bit different---the key difference here is that the map from $X$ to $\TZero (X)$ is now \emph{surjective} (i.e.~a quotient map) instead of in all the previous cases where it was \emph{injective} (i.e.~an inclusion).
\end{rmk}
\begin{proof}
Define $x_1\sim x_2$ iff the open sets which contain $x_1$ are precisely the same as the open sets which contain $x_2$.
\begin{exr}
Show that $\sim$ is an equivalence relation.
\end{exr}
Define $\TZero (X)\coloneqq X/\sim$ and let $\q :X\rightarrow \TZero (X)$.
\begin{exr}
Show that $\TZero (X)$ satisfies \ref{T0Quotient.i} and \ref{T0Quotient.ii}.
\end{exr}
\begin{exr}
Show that $\TZero (X)$ is unique.
\end{exr}
\end{proof}
\end{prp}
\begin{dfn}[$T_1$]\label{T1}
$X$ is \emph{$T_1$}\index{$T_1$} iff any two distinct points are separated.
\begin{rmk}
Sometimes this condition is called \emph{accessible}\index{Accessible (topological space)} of \emph{fr\'{e}chet}\index{Fr\'{e}chet (topological space)}.  I also prefer $T_1$ over ``accessible'' because, not only does it carry slightly more information, but it's also a lot more common.  I would \emph{definitely} recommend not to use the term ``Fr\'{e}chet'' to describe this, as the term ``fr\'{e}ceht space'' is usually meant to describe something else entirely.
\end{rmk}
\begin{rmk}
Warning:  This is not the same as ``any two topologically-distinguishable points are separated''.  That condition is called $R_0$ and is rarely, if ever, used.\footnote{There is no difference between $R_0$ and $T_1$ for spaces which are $T_0$.}
\end{rmk}
\begin{rmk}
In contrast to the $T_0$ condition, there are \emph{incredibly} important examples of spaces that are not $T_1$.
\end{rmk}
\end{dfn}
\begin{exm}[A space that is $T_0$ but not $T_1$]
The example above in \cref{exm4.5.3} worked.
\end{exm}
While the above is probably the best way to state $T_1$ as a definition because it makes its similarity with other separation conditions more apparent, it is often best to think of $T_1$ spaces are spaces in which points are closed.
\begin{prp}\label{prp4.5.32}
Let $X$ be a topological space.  Then, $X$ is $T_1$ iff $\{ x\}$ is closed for all $x\in X$.
\begin{proof}
$(\Rightarrow )$ Suppose that $X$ is $T_1$.  Let $x\in X$.  For all $y\in X$ distinct from $x$, let $U_y$ be an open neighborhood of $y$ that does not contain $x$.  Then,
\begin{equation}
\bigcup _{y\neq x}U_y=X\setminus \{ x\}
\end{equation}
is open, and so
\begin{equation}
\bigcap _{y\neq x}U_y^{\comp}=\{ x\}
\end{equation}
is closed.

\blankline
\noindent
$(\Leftarrow )$ Suppose that $\{ x\}$ is closed for all $x\in X$.  Let $x_1,x_2\in X$.  Then, $\{ x_1\} ^{\comp}$ is an open neighborhood of $x_2$ that does not contain $x_1$ and $\{ x_2\} ^{\comp}$ is an open neighborhood of $x_1$ that does not contain $x_2$.
\end{proof}
\end{prp}
\begin{crl}
Every finite $T_1$ topological space is discrete.
\begin{rmk}
In particular, it is a waste of time to look for counter-examples in finite spaces if your space is $T_1$.
\end{rmk}
\begin{proof}
If the space is $T_1$, then every point is closed.  If the space is finite, then every subset is a union of finitely many points, that is, a finite union of closed sets, and hence closed.
\end{proof}
\end{crl}
\begin{dfn}[$T_2$]\label{T2}
$X$ is \emph{$T_2$}\index{$T_2$} iff any two distinct points are separated by neighborhoods.
\begin{rmk}
This is quite often referred to as \emph{hausdorff}\index{Hausdorff (topological space)}.  In constrast to most of the alternative terminologies, this actually might be more common than the term ``$T_2$''.
\end{rmk}
\end{dfn}
\begin{dfn}[Compact]\label{Compact}
$X$ is \emph{compact}\index{Compact} iff it is quasicompact and $T_2$.
\end{dfn}
\begin{prp}\label{prp4.5.37}
Let $X$ be a topological space.  Then, $X$ is $T_2$ iff limits are unique.
\begin{rmk}
Because of lack of uniqueness in general, we have been hesitant to write $\lim _\lambda x_\lambda$---if limits are not unique, then what limit does this symbol refer to?  Hereafter, however, in $T_2$ spaces, we will not hesitate to use this notation.
\end{rmk}
\begin{proof}
$(\Rightarrow )$ Suppose that $X$ is $T_2$.  Let $\lambda \mapsto x_\lambda \in X$ be a net and let $x_\infty ,x_\infty '\in X$ be limits of $X$.  We proceed by contradiction:  suppose that $x_\infty \neq x_\infty '$.  Then, there exist disjoint open neighborhoods $U$ of $x_\infty$ and $U'$ of $x_\infty '$.  As $\lambda \mapsto x_\lambda$ converges to $x_\infty$, it is eventually contained in $U$.  But then, as $U$ and $U'$ are disjoint, $\lambda \mapsto x_\lambda$ is not eventually contained in $U'$---a contradiction.

\blankline
\noindent
$(\Leftarrow )$ Suppose that limits are unique.  Let $x_\infty ,x_\infty '\in X$ be distinct.  We wish to show that there exist disjoint open neighborhoods of $x_\infty$ and $x_\infty '$.  We proceed by contradiction:  suppose that every open neighborhood of $x_\infty$ intersects every open neighborhood of $x_\infty '$.  Let $\Lambda$ be the collection of these intersections, that is, the collection of sets of the form $U\cap U'$ for $U$ and $U'$ open neighborhoods of $x_\infty$ and $x_\infty '$ respectively.  Order $\Lambda$ by reverse-inclusion so as to form a directed set, and for each $U\cap U'\in \Lambda$, choose $x_{U\cap U'}\in U\cap U'$.  Then, $\Lambda \ni U\cap U'\mapsto x_{U\cap U'}$ converges to both $x_\infty$ and $x_\infty'$---a contradiction.
\end{proof}
\end{prp}
\begin{exr}\label{exr4.6.37}
Show that a subspace of a $T_2$ space is $T_2$.
\end{exr}
\begin{exr}\label{exr4.6.38}
Show that an arbitrary product of $T_2$ spaces is $T_2$.
\begin{rmk}
Thus, by \nameref{TychnoffsTheorem} (\cref{TychnoffsTheorem}), an arbitrary product of compact spaces is compact.
\end{rmk}
\end{exr}
\begin{exr}\label{exr4.6.39}
Show that quasicompact subsets (which are in fact compact by \cref{exr4.6.37}) of a $T_2$ space can be separated by neighborhoods.
\end{exr}
Hopefully you found an example in \cref{exr3.1.34} of a continuous bijective function that was not a homeomorphism.  In certain special cases, however, you can immediately make this deduction.
\begin{exr}\label{exr3.6.46}
Show that continuous injective function from a quasicompact space into a $T_2$ space is a homeomorphism onto its image.
\end{exr}
\begin{exm}[A space that is $T_1$ but not $T_2$]
Let $X$ be as in \cref{exm4.2.8x}, that is the real numbers with the cocountable topology.  By definition, countable sets are closed (along with all of $X$ of course), and so certainly points are closed, and so $X$ is $T_1$.  On the other hand, any open neighborhood of $0$, must intersect any open neighborhood of $1$,  because both of these neighborhoods have countable complements and so, by the uncountability of $\R$, must intersect.
\end{exm}

Here is where the consistency of the terminology breaks-down.  If you guessed that the term $T_3$ refers to spaces in which any two distinct points are separated by closed neighborhoods, you'd be wrong.  Unfortunately, it seems that the term $T_3$ was already taken (we'll see what it means in a bit) when someone went to write this definition down, and so it is called $T_{2\frac{1}{2}}$.
\begin{dfn}[$T_{2\frac{1}{2}}$]\label{T212}
$X$ is \emph{$T_{2\frac{1}{2}}$}\index{$T_{2\frac{1}{2}}$} iff any two distinct points are separated by closed neighborhoods.
\begin{rmk}
The alternate terminology for this seems to be \emph{urysohn}\index{Urysohn}.
\end{rmk}
\end{dfn}
\begin{exm}[A space that is $T_2$ but not $T_{2\frac{1}{2}}$]\footnote{This comes from \cite[pg.~100]{Steen}.}\label{SimplifiedArensSquare}
This example is very similar to the Arens Square---see \cref{ArensSquare}.

Define
\begin{equation}
S\coloneqq ((0,1)\times (0,1))\cap (\Q \times \Q )=\left\{ \coord{x,y}\in (0,1)\times (0,1):x,y\in \Q \right\} 
\end{equation}
and
\begin{equation}
X\coloneqq S\cup \{ \coord{0,0}\} \cup \{ \coord{1,0}\} .
\end{equation}
(This is the rationals in the open unit square together with the bottom-left and bottom-right corners.)  We define a topology on $X$ by defining a neighborhood base at each point (see \cref{prp4.1.8}).  For $(x,y)\in X$, there are three cases:  (i) $\coord{x,y}=\coord{0,0}$, (ii) $\coord{x,y}=\coord{1,0}$, and (iii) $\coord{x,y}\in S$.  We define
\begin{equation}
\mathcal{B}_{\coord{x,y}}\coloneqq \begin{cases}\left\{ U\subseteq S:U\text{ is open in }S\text{.}\footnote{Open in the usual topology (the subspace topology inherited from $(0,1)\times (0,1)$).}\right\} & \text{if }\coord{x,y}\in S \\ \left\{ U_{\coord{x,y}}^m:m\in \Z ^+\right\} & \text{otherwise}\end{cases},
\end{equation}
where
\begin{equation}
\begin{split}
U_{\coord{0,0}}^m & \coloneqq \{ \coord{0,0}\} \cup \left\{ \coord{x,y}\in (0,\tfrac{1}{2})\times (0,\tfrac{1}{m}):x,y\in \Q \right\} \\
U_{\coord{1,0}}^m & \coloneqq \{ \coord{1,0}\} \cup \left\{ \coord{x,y}\in (\tfrac{1}{2},1)\times (0,\tfrac{1}{m}):x,y \in \Q \right\} .
\end{split}
\end{equation}
By \cref{prp4.1.8}, there is a unique topology for which $\mathcal{B}_{\coord{x,y}}$ is a neighborhood base of $\coord{x,y}\in X$.

The closure of any open neighborhood of $\coord{0,0}$ contains points with $x$-coordinate $\frac{1}{2}$ and $y$-coordinate arbitrarily small, and the same goes from any open neighborhood of $(1,0)$.  Thus, these two points are not separated by closed neighborhoods.\footnote{Recall that two points are separated by closed neighborhoods iff they are separated by open neighborhoods with disjoint closure---see \cref{prp4.5.13}.}

On the other hand, the $x$-coordinate of every point in every open neighborhood of $\coord{0,0}$ is strictly less than $\frac{1}{2}$, and similarly the $x$-coordinate of every point in every open neighborhood of $\coord{1,0}$ is strictly greater than $\frac{1}{2}$.  In particular, these two points are separated by neighborhoods.

For any point $\coord{x,y}$ distinct from $\coord{0,0}$ and $\coord{1,0}$, we can separate $\coord{x,y}$ from $\coord{0,0}$ with neighborhoods by taking $U_{\coord{0,0}}^m\ni \coord{0,0}$ for $m\in \Z ^+$ with $\frac{1}{m}<y$ and any $\varepsilon$-ball about $\coord{x,y}$ of with radius less than $y-\frac{1}{m}$.  Similarly for $\coord{x,y}$ and $\coord{1,0}$.  Of course, any two points distinct from both $\coord{0,0}$ and $\coord{1,0}$ are separated by neighborhoods because they can be separated by neighborhoods in $S$.  Thus, $X$ is indeed $T_2$.
\begin{rmk}
This is the \emph{Simplified Arens Square}\index{Simplified Arens Square}.
\end{rmk}
\end{exm}
You might be thinking to yourself ``Well, if $T_3$ wasn't separated by closed neighborhoods, then it must be completely-separated, right?''.  Sorry.  Wrong again.
\begin{dfn}[Completely-$T_2$]\label{CompletelyT2}
$X$ is \emph{completely-$T_2$}\index{Completely-$T_2$} iff any two distinct points are completely-separated.
\begin{rmk}
Naturally, this is sometimes called \emph{completely-hausdorff}\index{Completely-hausdorff}.
\end{rmk}
\begin{rmk}
Sometimes people will also say that \emph{continuous functions separate points}.
\end{rmk}
\end{dfn}
\begin{exm}[A space that is $T_{2\frac{1}{2}}$ but not completely-$T_2$]
The Arens Square from \cref{ArensSquare} will do the trick.  There, we provided an example of two points which are separated by closed neighborhoods, but not completely-separated.  This is of course already enough to show that the Arens Square is not completely-$T_2$, but we still need to check that \emph{any} two points can be separated by closed neighborhoods.

Denote the Arens Square by $X$ and let $\coord{\frac{1}{2},r\sqrt{2}}\in X$ for $r\in \Q$.  Take $m\in \Z ^+$ such that $r\sqrt{2}-\frac{1}{m}>0$ and take $n\in \Z ^+$ such that $\frac{1}{n}<r\sqrt{2}-\frac{1}{m}$.  Then, using the definition \eqref{4.5.20}, we see that the closures of $U_{(1,0)}^m$ and $U_{\coord{\frac{1}{2},r\sqrt{2}}}$ are disjoint.  Similarly, a point of this form can be separated from $(1,0)$ by closed neighborhoods.

For $\coord{x,y}\in S$, just take $m\in \Z ^+$ with $\frac{1}{m}<y$.  Then, the closure of $U_{\coord{0,0}}^m$ and any $\varepsilon$-ball around $\coord{x,y}$ with radius less than $y-\frac{1}{m}$ will be disjoint.  A similar trick works for $\coord{1,0}$ of course.

For $r,q\in \Q$ with $r<q$, choose $m,n\in \Z ^+$ so that $r\sqrt{2}+\frac{1}{m}<q\sqrt{2}-\frac{1}{n}$ (and so that $r\sqrt{2}-\frac{1}{m}>0$ and $q\sqrt{2}+\frac{1}{n}<1$ of course so that the open neighborhoods are actually contained in $X$).  Then, the closures of $U_{\coord{\frac{1}{2},r\sqrt{2}}}^m$ and $U_{\coord{\frac{1}{2},q\sqrt{2}}}^n$ are disjoint.

For $\coord{x,y}\in S$ and $\coord{\frac{1}{2},r\sqrt{2}}\in T$, we must have that $y\neq r\sqrt{2}$, and so, without loss of generality, that $y<r\sqrt{2}$.  Then, choose $m\in \Z ^+$ large enough so that $r\sqrt{2}-\frac{1}{m}>y$ (and so that $U_{\coord{\frac{1}{2},\sqrt{2}}}^m$ is contained in $X$), and take a $\varepsilon$-ball about $\coord{x,y}$ with radius less than $(r\sqrt{2}-\frac{1}{m})-y$.  Then, the closure of these two neighborhoods will be disjoint.

Finally, can separate two points in $S$ by closed neighborhoods because we could do so in $(0,1)\times (0,1)$ (recall that $S$ just as the subspace topology).
\end{exm}
And of course, as now you probably saw coming, $T_3$ does not mean distinct points can be perfectly-separated.
\begin{dfn}[Perfectly-$T_2$]\label{PerfectlyT2}
$X$ is \emph{perfectly-$T_2$}\index{Perfectly-$T_2$} iff any two distinct points can be perfectly-separated.
\begin{rmk}
Disclaimer:  I have never seen this term before.  Then again, I've never seen \emph{any} term to describe such spaces.  But honestly, if completely-$T_2$ means you can completely-separate points, then a space in which you can perfectly-separate points is going to be called\textellipsis
\end{rmk}
\end{dfn}
\begin{exm}[A space that is completely-$T_2$ but not perfectly-$T_2$]\label{exm4.5.48}
The Uncountable Fort Space from \cref{UncountableFortSpace} will do the trick.  There, we provided an example of two points which are completely-separated, but not perfectly-separated.  This is of course already enough to show that the Uncountable Fort Space is not perfectly-$T_2$, but we still need to check that \emph{any} two points can be completely-separated.

Recall that the Uncountable Fort Space was defined to be $X\coloneqq \R$ with the closed sets being precisely the finite sets and also the sets which contained $0$.  We showed above in \cref{UncountableFortSpace} that $1\in X$ can be completely-separated from $0\in X$.  Of course, there was nothing special about $1$, and so all we need to do is to show that we can completely-separate any two nonzero points in $X$.

So, let $x_1,x_2\in X$ be nonzero and distinct, and define $f:X\rightarrow [0,1]$ by
\begin{equation}
f(x)\coloneqq \begin{cases}0 & \text{if }x=x_1 \\ 1 & \text{if }x=x_2 \\ \tfrac{1}{2} & \text{otherwise}\end{cases}.
\end{equation}
Then, if $C\subseteq [0,1]$ closed contains $\frac{1}{2}$, $f^{-1}(C)$ contains $0\in X$, and is hence closed.  Otherwise, it is finite, as $f^{-1}\left( [0,1]\setminus \{ \frac{1}{2}\} \right) =\{ x_1,x_2\}$, and hence closed.  Thus, $f$ is continuous, and so completely-separates $x_1$ and $x_2$.
\end{exm}

But now we've gone through all the separation properties, right?  How could $T_3$ be a thing?  Well, now we enter a collection of separation axioms of a different nature---now we will focus on separating \emph{closed sets}.  One unfortunate fact, however, is that, if points are not closed, then these new separation axioms are \emph{not} strictly stronger than the separation axioms we just presented.  There are thus two versions of the following separation axioms:  ones in which the points are closed and ones in which they aren't.
\begin{dfn}[Regular]\label{Regular}
$X$ is \emph{regular}\index{Regular (topological space)} iff any closed set and a point not contained in it can be separated by neighborhoods.
\begin{rmk}
You'll note that we skipped right over being topologically-distinguishable and just being separated.  This is because a point is automatically separated from a closed set---the complement of the closed set is an open neighborhood of the point which does not intersect the closed set.
\end{rmk}
\begin{rmk}
Unfortunately, the terminology of separation axioms is so fucked that there's really no way of being consistent with all of the literature.  There are sources which reverse my conventions of regular and $T_3$.  We explain the motivation of our choice of convention in the definition of $T_3$ (\cref{T3}).
\end{rmk}
\end{dfn}
\begin{exm}[A space that is $T_0$ regular but not $T_1$]
The indiscrete topology on any set with at least two points is vacuously regular but not $T_0$.
\end{exm}
Stupid examples like this is why we almost always care about the case when we in addition impose the condition of being $T_1$.  This finally brings us to the definition of $T_3$.
\begin{dfn}[$T_3$]\label{T3}
$X$ is $T_3$ iff it is $T_1$ and regular.
\begin{rmk}
The $T_1$ condition (points are closed) is added so that this is a strict specifization of being $T_2$.  In fact, by the following proposition, we would have equally well said $T_0$.
\end{rmk}
\begin{rmk}
We choose to call this $T_3$ instead of regular because then we have $T_3\rightarrow T_2$.  If the terms were reversed, then there would be $T_3$ spaces (like indiscrete spaces) that were not $T_2$---ew.
\end{rmk}
\begin{rmk}
Warning:  Up until now, all of the $T_k$ axioms (including things like $T_{2\frac{1}{2}}$, completely-$T_2$, and perfectly-$T_2$) have been strictly comparable, that is, $T_1$ strictly implies $T_0$, $T_2$ strictly implies $T_1$, etc..  With the introduction of $T_3$, this is no longer the case.  It turns out that $T_3$ implies $T_{2\frac{1}{2}}$, but that $T_3$ is not comparable with either completely-$T_2$ or perfectly-$T_2$.
\end{rmk}
\end{dfn}
\begin{prp}\label{prp4.6.53}
If $X$ is $T_0$ and regular, then it is $T_2$.
\begin{proof}
Suppose that $X$ is $T_0$ and regular.  Let $x_1,x_2\in X$.  Because $X$ is $T_0$, without loss of generality there is some open neighborhood $U$ of $x_1$ which does not contain $x_2$.  Then, $U^{\comp}$ is closed and does not contain $x_1$, so because $X$ is regular, there are disjoint open neighborhoods $V_1$ and $V_2$ of $x_1$ and $U^{\comp}$ respectively.  Then, as $x_2\in U^{\comp}$, this implies that $x_1$ and $x_2$ are separated by neighborhoods.
\end{proof}
\end{prp}
\begin{prp}
If $X$ is $T_3$, then $X$ is $T_{2\frac{1}{2}}$.
\begin{proof}
Suppose that $X$ is $T_3$.  Let $x_1,x_2\in X$ be distinct.  As $X$ is $T_3$, it is definitely $T_2$, and so we can separate $x_1$ and $x_2$ with open neighborhoods $U_1$ and $U_2$ containing $x_1$ and $x_2$ respectively.  Then, $U_1^{\comp}$ is a closed set not containing $x_1$, and so because $X$ is $T_3$, there is an open neighborhood $V_1$ of $x_1$ and an open neighborhood $V_2$ of $U_1^{\comp}$ which are disjoint.  Then,
\begin{equation}
\Cls (U_2)\subseteq U_1^{\comp}\subseteq V_2\text{ and }\Cls (V_1)\subseteq V_2^{\comp},
\end{equation}
so that $\Cls (U_2)$ and $\Cls (V_1)$ are disjoint, so that $X$ is $T_{2\frac{1}{2}}$ by \cref{prp4.5.13}.
\end{proof}
\end{prp}
\begin{exm}[A space that is perfectly-$T_2$ but not $T_3$]\label{CocountableExtensionTopology}
Define $X\coloneqq \R$.  We equip $\R$ with the so-called \emph{cocountable extension topology}\index{Cocountable extension topology}.  Let $C\subseteq X$ and declare that
\begin{textequation}
$C$ is closed iff it is the union of a countable set and a set closed in the usual topology on $\R$.
\end{textequation}

Now, let $x_1,x_2\in X$ be distinct, and let $\phi :\R \rightarrow (0,1)$ be any homeomorphism (in the usual topology).  Now define $f:X\rightarrow [0,1]$ by
\begin{equation}
f(x)\coloneqq \begin{cases}0 & \text{if }x=x_1 \\ 1 & \text{if }x=x_2 \\ \phi (x) & \text{otherwise}\end{cases}.
\end{equation}
The preimage of any set in $[0,1]$ which contains neither $0$ nor $1$ will be closed because $\phi$ is a homeomorphism.  If it does contain $0$ or $1$, then the preimage will be the union of a closed set in the usual topology on $\R$ and a finite set, and hence will be closed.  Thus, $f$ is continuous, and so $X$ is perfectly-$T_2$.  We know check that $X$ is not $T_3$.

$\Q \subseteq X$ is closed because it is countable.  Any open neighborhood of $\Q$ must be of the form of a usual open neighborhood of $\Q$ with countably many points removed.  Of course, the only usual open neighborhood of $\Q$ is all of $\R$, and so every open neighborhood of $\Q$ is just $\R$ with countably many points removed.  On the other hand, $\sqrt{2}\notin \Q$ and any open neighborhood of $\sqrt{2}$ is a usual open neighborhood with countably many points removed.  Thus, by uncountability of $\R$, these two must intersection, and so you cannot separate $\Q$ and $\sqrt{2}$ with neighborhoods.  Thus, $X$ is not $T_3$.
\end{exm}
\begin{exm}[A space that is $T_3$ but not completely-$T_2$]\footnote{Evidently, this example originally comes from \cite{Thomas}, though I first heard about it from Brian M.~Scott on \href{http://math.stackexchange.com/questions/386742/making-tychonoff-corkscrew-in-counterexamples-in-topology-rigorous}{math.stackexchange}.}\label{ThomasTentSpace}
For $m\in 2\Z$, define
\begin{equation}
L_m\coloneqq \{ m\} \times [0,\tfrac{1}{2}).
\end{equation}
For $n\in 1+2\Z$ and $k\in \Z$ with $k\geq 2$, let $T_{n,k}$ be the \emph{legs} of an isosceles triangle in $\R ^2\times \R ^2$ with apex at $p_{n,k}\coloneqq \coord{n,1-\frac{1}{k}}$ and base $[n-(1-\frac{1}{k}),n+(1+\frac{1}{k})]\times \{ 0\}$ (including the end-points of the legs on the base but not the apex $p_{n,k}$).  Then, define
\begin{equation}
\begin{split}
Y_0 & \coloneqq \bigcup _{m\in 2\Z}L_m \\
Y_1 & \coloneqq \bigcup _{n\in 1+2\Z ;\ k\in \Z ,k\geq 2}\{ p_{n,k}\} \\
Y_2 & \coloneqq \bigcup _{n\in 1+2\Z ;\ k\in \Z ,k\geq 2}T_{n,k} \\
Y & \coloneqq Y_0\cup Y_1\cup Y_2.
\end{split}
\end{equation}
We define a topology on $Y$ by defining a neighborhood base at each point---see \cref{prp4.1.8}.  For $\coord{x,y}\in Y_2$, we declare a neighborhood base to be simply $\{ \coord{x,y}\}$.  For $\coord{x,y}=p_{n,k}$, we declare a neighborhood base of $\coord{x,y}$ to consist of subsets of the form $\{ p_{n,k}\} \cup S$ for $S\subseteq T_{n,k}$ cofinite in $T_{n,k}$.  For $\coord{x,y}\in L_m$ for $m\in 2\Z$, we declare a neighborhood base of $\coord{x,y}$ to consist of subsets of the form $\{ \coord{m,y}\}\cup S$ for $S\subseteq Y\cap \left( (m-1,m+1)\times \{ y\}\right)$ cofinite in $Y\cap \left( (m-1,m+1)\times \{ y\}\right)$.  It follows from \cref{prp4.1.8} that there is a unique topology for which $\mathcal{B}_{\coord{x,y}}$ is a neighborhood base of $\coord{x,y}\in Y$. 

Finally define
\begin{equation}
X\coloneqq Y\sqcup \{ p_1,p_2\}
\end{equation}
for distinct new points $p_1,p_2$.  To define a topology on $X$, we once again use \cref{prp4.1.8}.  A neighborhood base at $\coord{x,y}\in Y$ consists of precisely the same sets as it did before.  We furthermore declare that
\begin{equation}
\begin{split}
\mathcal{B}_{p_1} & \coloneqq \left\{ U_{p_1}^\alpha :\alpha \in \R \right\} \\
\mathcal{B}_{p_2} & \coloneqq \left\{ U_{p_2}^\alpha :\alpha \in \R \right\}
\end{split}
\end{equation}
to be neighborhood bases of $p_1$ and $p_2$ respectively, where
\begin{equation}
\begin{split}
U_{p_1}^\alpha & \coloneqq \left\{ \coord{x,y}\in Y:x<\alpha \right\} \\
U_{p_2}^\alpha & \coloneqq \left\{ \coord{x,y}\in Y:x>\alpha \right\} .
\end{split}
\end{equation}

We first check that $X$ is not completely-$T_2$ by showing that no continuous function can separate $p_1$ from $p_2$.

First suppose that $f(p_{n,k})=\alpha$.  Then,
\begin{equation}
f^{-1}(\alpha )\cap T_{n,k}=f^{-1}\left( \bigcap _{m\in \Z ^+}(\alpha -\tfrac{1}{m},\alpha +\tfrac{1}{m})\right) \cap T_{n,k}=\bigcap _{m\in \Z ^+}S_m,
\end{equation}
where
\begin{equation}
S_m\coloneqq f^{-1}((\alpha -\tfrac{1}{m},\alpha +\tfrac{1}{m}))\cap T_{n,k}
\end{equation}
is cofinite in $T_{n,k}$.  Hence,
\begin{equation}
T_{n,k}\setminus \left( f^{-1}(\alpha )\cap T_{n,k}\right) =\bigcup _{m\in \Z ^+}T_{n,k}\setminus S_m.
\end{equation}
In other words, the number of points in $T_{n,k}$ that map to a different value than $f(p_{n,k})$ is at most countable.  Less precisely, $f$ is constant on $T_{n,k}$ modulo a countable set.

Let $C_{n,k}$ denote the set of $y$-values in $[0,\tfrac{1}{k})$ for which there is some point in $T_{n,k}$ with that $y$-coordinate and that maps to $f(p_{n,k})$.  We just showed that this set is cocountable in $[0,1-\frac{1}{k})$, and hence it is certainly cocountable in $[0,\frac{1}{2})$.  Then, the intersection of them $\bigcap _{k\in \Z ,k\geq 2}C_{n,k}$ must in turn then be cocountable in $[0,\frac{1}{2})$, and in particular, be nonempty.  So, let $y_m\in L_m$ be such that there is some $q_{m-1,k}\in T_{m-1,k}$ and some $q_{m+1,k}\in T_{m+1,k}$ with $f(q_{m-1,k})=f(p_{m-1,k})$ and $f(q_{m+1,k})=f(p_{m+1,k})$.  Because the open neighborhoods of $y_m$ are cofinite in $((m-1,m+1)\times \{ y\} )\cap Y$, the sequence $k\mapsto q_{m-1,k}$ must eventually be in every neighborhood of $y_m$, and so we have $\lim _kq_{m-1,k}=y_m=\lim _kq_{m+1,k}$.  It follows that
\begin{equation}
f(y_m)=\lim _kf(q_{m+1,k})=\lim _kf(p_{m+1,k})=\lim _kf(q_{m+2,k})=f(y_{m+2}).
\end{equation}
The crux:  for each $m\in 2\Z$, there exists points $y_m\in L_m$ and $y_{m+2}\in L_{m+2}$ with $f(y_m)=f(y_{m+2})$.  By the definition of the open neighborhoods of $p_1$ and $p_2$, it follows that $f(p_1)=\lim _{m\to -\infty}f(y_m)=\lim _{m\to +\infty}f(y_m)=f(p_2)$, so that $p_1$ and $p_2$ are not completely-separated, and so $X$ is not completely-$T_2$.

We now check that $X$ is $T_3$.  We first check that it is $T_1$.  $p_1$ is closed because
\begin{equation}
\{ p_1\} =\bigcap _{\alpha \in \R}(U_{p_2}^\alpha )^{\comp}.
\end{equation}
Similarly for $p_2$.  Each point $\coord{m,y}\in L_m$ is closed because the union of all open sets not contained in $\mathcal{B}_{\coord{m,y}}$, $\mathcal{B}_{p_1}$, or $\mathcal{B}_{p_2}$ is precisely the complement of $\coord{m,y}$.  Similarly for the $p_{n,k}$s.  For $\coord{x,y}\in T_{n,k}$, the intersection over $\{ \coord{x',y'}\} ^{\comp}$ for $\coord{x',y'}\neq \coord{x,y}$ in $T_{n,k}$ is $\{ \coord{x,y}\}$ and so will be closed if $T_{n,k}$ is.  Each $T_{n,k}$ is open, however, (being the union of its points, which are open), and so $T_{n,k}$ is the complement of the union over $T_{n',k'}$ with $n'\neq n$ and $k'\neq k$ with $\{ p_1,p_2\}$ removed.

Now we just need to show that we can separate closed sets from points with open neighborhoods.  One thing to note is that we need only separate closed sets that are complements of some element of some $\mathcal{B}_p$ from points, because every closed set is an intersection of sets of this form.  So, let $p\in X$ and let $C\subseteq X$ be closed not containing $p$.  There is nothing to do but break it down into cases.

First suppose that $p=p_1$ and $C\subseteq Y\cup \{ p_2\}$.  If $C$ contained points of arbitrarily small $x$-coordinate, then because it is closed, it would have to contain $p_1$.  Thus, there is some $x_0\in \R$ such that the $x$-coordinate of every point (besides $p_2$ of course) is at least $x_0$.  Then we can take $U_{p_1}^{\frac{x_0}{8}}$ as our open neighborhood of $p_1$ and $U_{p_2}^{\frac{x_0}{4}}$ as our open neighborhood of $C$.  Similarly if our point is $p_2$.

We can easily separate points of $T_{n,k}$ from closed sets as these points themselves are open.

Now take $p=\coord{m,y}\in L_m$ and $C\subseteq X$ not containing $p$.  If $C$ intersects $((m-1,m+1)\times \{ y\} )\cap Y$ at more than finitely many points, then $y$ would be an accumulation point of $C$, and hence would be contained in $C$.  Thus, it must intersection $((m-1,m+1)\times \{ y\})\cap Y$ at at most finitely many points, in which case we can simply remove them to obtain an open neighborhood of $\coord{m,y}$.  These finitely many points are elements of $T_{m-1,k}$ or $T_{m+1,k}$ for some $k$, and so are in particular open.  Thus, the union of these finitely many points together with $U_{p_1}^{(m-1)-(1-\frac{1}{k})}$ and $U_{p_2}^{(m+1)+(1+\frac{1}{k})}$ then must contain $C$ (because these two neighborhoods of $p_1$ and $p_2$ contain all $T_{n,l}$ for $n\neq m-1,m+1$).

Finally if one of the points is some $p_{n,k}$, then using very similar logic as in the previous case, $C$ must intersect $T_{n,k}$ at at most finitely many points.  Removing these points from $T_{n,k}$ gives us an open neighborhood which is disjoint from the neighborhood formed from the union of these finitely many points (which are open) and the complement of $T_{n,k}$.
\begin{rmk}
I do not know of a name for this space, but in order to have something to refer to, I shall call it the \emph{Thomas Tent Space}\index{Thomas Tent Space}.\footnote{The $T_{n,k}$s are like ``tents'', and indeed, that is the word Scott used to describe them.}
\end{rmk}
\end{exm}
Thus, once you hit $T_{2\frac{1}{2}}$, you can increase your separation in one of two noncomparable ways:  you can make your space completely-$T_2$ or you can make your space $T_3$.  At the end of the section, we will summarize how all the separation axioms related to each other.
\begin{dfn}[$T_{3\frac{1}{2}}$]\label{T312}
$X$ is \emph{$T_{3\frac{1}{2}}$}\index{$T_{3\frac{1}{2}}$} iff it is $T_1$ and any closed set can be separated by closed neighborhoods from a point it does not contain.
\begin{rmk}
This is not universally accepted terminology.  Often people use the term $T_{3\frac{1}{2}}$ to refer to what I call completely-$T_3$.  I imagine this is the case because it turns out that my $T_{3\frac{1}{2}}$ is equivalent to $T_3$ (see the next proposition).
\end{rmk}
\begin{rmk}
You will notice a pattern with the terminology.  If $T_k$ means you can separate XYZ from ABC with neighborhoods, then $T_{k\frac{1}{2}}$ means you can separate XYZ from ABC with \emph{closed} neighborhoods; completely-$T_k$ means you can completely-separate XYZ from ABC; perfectly-$T_k$ means you can perfectly-separate XYZ from ABC.  This admittedly creates conflict with some people's terminology, but the terminology of separation axioms is so varied from source to source that it was impossible to come up with a naming system that didn't conflict with something.  I chose what I did because it is the most systematic that does not completely depart from the established nomenclature.
\end{rmk}
\end{dfn}
\begin{prp}\label{prp4.5.70}
If $X$ is $T_3$, then $X$ is $T_{3\frac{1}{2}}$
\begin{proof}
Suppose that $X$ is $T_3$.  Let $C\subseteq X$ be closed and let $x\in C^{\comp}$.  As $X$ is $T_3$, there is an open neighborhood $U$ of $C$ and an open neighborhood $V$ of $X$ which are disjoint.  Then, $V^{\comp}$ is a closed set not containing $X$, and so there is an open neighborhood $W_1$ of $V^{\comp}$ disjoint and an open neighborhood $W_2$ of $x$ which are disjoint.  Then,
\begin{equation}
C\subseteq U\subseteq \Cls (U)\subseteq V^{\comp}\subseteq W_1\text{ and }x\in W_2\subseteq \Cls (W_2)\subseteq W_1^{\comp},
\end{equation}
and so $U$ and $W_2$ are open neighborhoods of $C$ and $x$ respectively with disjoint closures, so that $X$ is $T_{3\frac{1}{2}}$ by \cref{prp4.5.13}.
\end{proof}
\end{prp}
\begin{dfn}[Completely-$T_3$]\label{CompletelyT3}
$X$ is \emph{completely-$T_3$}\index{Completely-$T_3$} iff it is $T_1$ and any closed set can be completely-separated from a point it does not contain.
\begin{rmk}
As noted above, sometimes people use the term $T_{3\frac{1}{2}}$ for this property.  This is also sometimes called \emph{tychonoff}\index{tychonoff}.
\end{rmk}
\end{dfn}
\begin{exm}[A space that is $T_{3\frac{1}{2}}$ but not completely-$T_3$]
The Thomas Tent Space $X$ from \cref{ThomasTentSpace} will do the trick.  We showed there that it is $T_3$, but not completely-$T_2$.  From \cref{prp4.5.70}, it follows that $X$ is $T_{3\frac{1}{2}}$.  However, it if were completely-$T_3$, then it would also be completely-$T_2$---a contradiction.  Therefore, $X$ is likewise not completely-$T_3$.
\end{exm}
\begin{dfn}[Perfectly-$T_3$]\label{PerfectlyT3}
$X$ is \emph{perfectly-$T_3$} iff it is $T_1$ and any closed set can be perfectly-separated from a point it does not contain.
\end{dfn}
\begin{exm}[A space that is completely-$T_3$ but not perfectly-$T_3$]\label{exm4.6.80}
The Uncountable Fort Space from \cref{UncountableFortSpace} will do the trick.  In \cref{exm4.5.48}, we showed that this was completely-$T_2$ but not perfectly-$T_2$.  As it is not perfectly-$T_2$, it is certainly not perfectly-$T_3$, though we still need to check that it is completely-$T_3$.

Recall that the Uncountable Fort Space was defined to be $X\coloneqq \R$ with the closed sets being precisely the finite sets and also the sets which contained $0$.

So, let $C\subseteq X$ be closed and let $x_0\in C^{\comp}$.  First let us do the case where $x_0=0$.  Then, we may define $f:X\rightarrow [0,1]$ by
\begin{equation}
f(x)\coloneqq \begin{cases}1 & \text{if }x\in C \\ 0 & \text{otherwise}\end{cases}.
\end{equation}
Then, $f^{-1}(1)=C$ is closed and $f^{-1}(0)$ contains $0\in X$, and so is closed.  Thus, $f$ is continuous.  Now consider the case where $0\in C$.  Then, we may define $f:X\rightarrow [0,1]$ by
\begin{equation}
f(x)\coloneqq \begin{cases}1 & \text{if }x=x_0 \\ 0 & \text{otherwise}\end{cases}.
\end{equation}
$f^{-1}(1)$ is just a point and so is closed and $f^{-1}(0)$ contains $0\in X$ and so is closed.  Finally, let us consider the case where $x_0\neq 0$ and $0\notin C$.  Then, we may define $f:X\rightarrow [0,1]$ by
\begin{equation}
f(x)\coloneqq \begin{cases}1 & \text{if }x\in C \\ 0 & \text{if }x=x_0 \\ \tfrac{1}{2} & \text{otherwise}\end{cases}.
\end{equation}
Then, $f^{-1}(1)=C$ is closed, $f^{-1}(0)=\{ x_0\}$ is finite and hence closed, and $f^{-1}(\frac{1}{2})$ contains $0\in X$ and is hence closed.  Thus, indeed, $X$ is completely-$T_3$.
\end{exm}

The separation axioms $T_0$ through perfectly-$T_2$ all had to do with separation of points.  All the $T_3$ separation axioms had to do with separating closed sets from points.  Finally, the $T_4$ axioms have to do with separating closed sets from closed sets.
\begin{dfn}[Normal]\label{Normal}
$X$ is \emph{normal}\index{Normal (topological space)} iff any two disjoint closed subsets can be separated by neighborhoods.
\begin{rmk}
Similarly as with the definition of regular (\cref{Regular}), we do not need to consider topological-distinguishability and mere separatedness.
\end{rmk}
\begin{rmk}
Similarly as with the term ``regular'', some authors reverse my conventions of normal and $T_4$.  The motivation of our choice of convention is the same as it was for $T_3$.
\end{rmk}
\end{dfn}
\begin{exm}[A space that is normal but not $T_0$]
The indiscrete topology on any set with at least two points is vacuously normal but not $T_0$.
\end{exm}
Stupid examples like this is why we almost always care about the case when we in addition impose the condition of being $T_1$.
\begin{dfn}[$T_4$]\label{T4}
$X$ is \emph{$T_4$}\index{$T_4$} iff it is $T_1$ and normal.
\begin{rmk}
Just as before, the condition of $T_1$ is imposed so that this is a strict specifization of being $T_3$.
\end{rmk}
\end{dfn}
There are at least two relatively large families of spaces that are $T_4$---metric spaces (which are in fact perfectly-$T-4$---see \cref{prp5.4.13}) and \emph{compact} spaces.
\begin{prp}\label{prp4.6.83}
If $X$ is compact, then it is $T_4$.
\begin{proof}
Suppose that $X$ is compact.  Then, closed subsets are quasicompact (\cref{exr4.2.33}), and hence can be separated by neighborhoods by \cref{exr4.6.39}.
\end{proof}
\end{prp}
\begin{prp}
If $X$ is $T_4$, then it is $T_{3\frac{1}{2}}$.
\begin{proof}
Suppose that $X$ is $T_4$.  Let $C\subseteq X$ be closed and let $x\in C^{\comp}$.  As $X$ is $T_4$, it is definitely $T_3$, and so we can separate $C$ and $x$ with open neighborhoods $U_1$ and $U_2$ containing $C$ and $x$ respectively.  Then, $U_1^{\comp}$ is a closed set disjoint from $C$, and so because $X$ is $T_4$, there is an open neighborhood $V_1$ of $C$ and an open neighborhood $V_2$ of $U_1^{\comp}$ which are disjoint.  Then,
\begin{equation}
\Cls (U_2)\subseteq U_1^{\comp}\subseteq V_2\text{ and }\Cls (V_1)\subseteq V_2,
\end{equation}
so that $\Cls (U_2)$ and $\Cls (V_1)$ are disjoint, so that $X$ is $T_{3\frac{1}{2}}$ by \cref{prp4.5.13}.
\end{proof}
\end{prp}
There are spaces that are perfectly-$T_3$ but not $T_4$; however, unfortunately we haven't yet the firepower to construct such a space yet (or at least prove it has the desired properties).  The counter-example we have in mind is \emph{Niemytzki's Tange Disk Topology}, and you can find it in \cref{NiemytzkisTangentDiskTopology}.

Recall that (\cref{ThomasTentSpace}) there are $T_3$ spaces that are \emph{not} completely-$T_2$.  This does not happen with $T_4$ and completely-$T_3$!  Fortunately, every $T_4$ space is completely-$T_3$, as we shall see in a moment (see \cref{UrysohnsLemma}).
\begin{dfn}[$T_{4\frac{1}{2}}$]\label{T412}
$X$ is \emph{$T_{4\frac{1}{2}}$}\index{$T_{4\frac{1}{2}}$} iff it is $T_1$ and any two disjoint closed subsets can be separated by closed neighborhoods.
\begin{rmk}
Just as with $T_{3\frac{1}{2}}$ (\cref{T312}), this terminology is not standard, presumably because it is actually just equivalent to $T_4$.
\end{rmk}
\end{dfn}
\begin{prp}\label{prp4.5.88}
If $X$ is $T_4$, then $X$ is $T_{4\frac{1}{2}}$.
\begin{proof}
Suppose that $X$ is $T_4$.  Let $C_1,C_2\subseteq X$ be disjoint closed subsets.  As $X$ is $T_4$, there are disjoint open neighborhoods $U_1$ and $U_2$ of $C_1$ and $C_2$ respectively.  Then, $U_1^{\comp}$ is a closed set not containing $C_1$, and so there are disjoint open neighborhoods $V$ and $W$ of $C_1$ and $U_1^{\comp}$ respectively.  Then,
\begin{equation}
C_2\subseteq U_2\subseteq \Cls (U_2)\subseteq U_1^{\comp}\subseteq W\text{ and }C_1\subseteq V\subseteq \Cls (V)\subseteq W^{\comp},
\end{equation}
and so $V$ and $U_2$ are open neighborhoods of $C_1$ and $C_2$ respectively with disjoint closures, so that $X$ is $T_{4\frac{1}{2}}$ by \cref{prp4.5.13}.
\end{proof}
\end{prp}
\begin{dfn}[Completely-$T_4$]\label{CompletelyT4}
$X$ is \emph{completely-$T_4$} iff it is $T_1$ and any two disjoint closed subsets can be completely-separated.
\end{dfn}
This is in fact \emph{equivalent} to being $T_4$, though this is relatively nontrivial, and the statement even has a name associated to it.  Before we prove that, however, we first present a useful `'lemma'.
\begin{prp}\label{prp4.5.91}
Let $X$ be a $T_1$ topological space.  Then,
\begin{enumerate}
\item \label{enm4.5.91.i}$X$ is $T_3$ iff whenever $U$ is an open neighborhood of $x\in X$, there is an open neighborhood $V$ of $x$ such that $x\in V\subseteq \Cls (V)\subseteq U$; and
\item \label{enm4.5.91.ii}$X$ is $T_4$ iff whenever $U$ is an open neighborhood of a closed subset $C\subseteq X$, there is an open neighborhood $V$ of $C$ such that $C\subseteq V\subseteq \Cls (V)\subseteq U$.
\end{enumerate}
\begin{proof}
We first prove \ref{enm4.5.91.i}.

$(\Rightarrow )$ Suppose that $X$ is $T_3$.  Let $U$ be an open neighborhood of $x\in X$.  Then, $U^{\comp}$ is a closed set that does not contain $x$, and so there are disjoint open neighborhoods $V$ and $W$ of $x$ and $U^{\comp}$ respectively.  Then,
\begin{equation}
x\in V\subseteq \Cls (V)\subseteq W^{\comp}\subseteq U.
\end{equation}

\blankline
\noindent
$(\Leftarrow )$ Suppose that whenever $U$ is an open neighborhood of $x\in X$, there is an open neighborhood $V$ of $x$ such that $x\in V\subseteq \Cls (V)\subseteq U$.  Let $C\subseteq X$ be closed and let $x\in C^{\comp}$.  Then, $C^{\comp}$ is an open neighborhood of $x$, and so there is an open neighborhood $V$ of $x$ such that
\begin{equation}
x\in V\subseteq \Cls (V)\subseteq C^{\comp},
\end{equation}
and so $V$ and $\Cls (V)^{\comp}$ are disjoint open neighborhoods of $x$ and $C$ respectively, so that $X$ is $T_3$.
\begin{exr}
Prove \ref{enm4.5.91.i}.
\end{exr}
\end{proof}
\end{prp}
\begin{thm}[Urysohn's Lemma]\index{Urysohn's Lemma}\footnote{Proof adapted from \cite[pg.~207]{Munkres}.}\label{UrysohnsLemma}
If $X$ is $T_4$, then $X$ is completely-$T_4$.
\begin{rmk}
This is often stated as ``If $C_1,C_2\subseteq X$ are disjoint closed subsets of a $T_4$ space $X$, then there is a continuous function $f:X\rightarrow [0,1]$ that is $0$ on $C_1$ and $1$ on $C_2$.''.
\end{rmk}
\begin{proof}
Suppose that $X$ is $T_4$.  Let $C_1,C_2\subseteq X$ be closed.

Let $\{ r_m:m\in \N \}$ be an enumeration of the rationals in $[0,1]$ with $r_0=1$ and $r_1=0$.  We define a collection of open sets $\{ U_{r_m}:m\in \N \}$ inductively.  During this process, we will apply \cref{prp4.5.91} repeatedly.

First of all, define $U_1\coloneqq C_2^{\comp}$.  Then, $U_1$ is an open neighborhood of $C_1$, and so there is some other open neighborhood $U_0$ of $C_1$ such that $C_1\subseteq U_0\subseteq \Cls (U_0)\subseteq U_1$.

Suppose that we have defined $U_{r_0},\ldots U_{r_m}$ such that $\Cls (U_p)\subseteq U_q$ if $r<q$ for $r,q\in \{ r_0,\ldots ,r_m\}$.  We wish to define $U_{m+1}$ so that this property still remains to be true.  As there are only finitely many rational numbers in $\{ r_0,\ldots ,r_m\}$ there is a largest $p_0\in \{ r_0,\ldots ,r_m\}$ with $p_0<r_{m+1}$ and similarly there is a smallest $q_0\in \{ r_0,\ldots ,r_m\}$ with $r_{m+1}<q_0$.  Take $U_{r_{m+1}}$ so that
\begin{equation}
\Cls (U_{p_0})\subseteq U_{r_{m+1}}\subseteq U_{q_0}.
\end{equation}
Proceeding inductively, this allows us to define $U_{r_m}$ for all $m\in \N$, and hence, we have defined $U_r$ for all $r\in \Q \cap [0,1]$.  By construction, it follows that
\begin{equation}
\Cls (U_p)\subseteq U_q
\end{equation}
for $p,q\in \Q \cap [0,1]$ for $p<q$.

For $x\in X$, define
\begin{equation}
Q_x\coloneqq \{ r\in \Q \cap [0,1]:x\in U_r\} .
\end{equation}
Note that $Q_x$ is empty iff $x\in C_2$.  We then in turn define
\begin{equation}
f(x)\coloneqq \begin{cases}1 & \text{if }x\in C_2 \\ \inf \left( Q_x\right) & \text{otherwise}\end{cases}.
\end{equation}
Of course $f(C_2)=\{ 1\}$.  Furthermore, if $x\in C_1$, then $x\in U_0$, and so indeed $f(x)=0$.  Thus, we need only check that $f$ is continuous.

So, let $x_0\in X$.  First suppose that $f(x_0)\neq 0,1$.  The other two cases are similar.  Let $\varepsilon >0$ be such that $B_\varepsilon (f(x_0))\subseteq [0,1]$ (if $f(x_0)=0$, for example, then you will instead use $[0,\varepsilon )$ in place of $B_\varepsilon (f(x_0))$).  Let $p,q\in \Q$ be such that
\begin{equation}
f(x_0)-\varepsilon <p<f(x_0)<q<f(x_0)+\varepsilon .
\end{equation}
Then, $U\coloneqq U_q\setminus \Cls (U_p)$ is open in $X$ and we claim that $f(U)\subseteq B_\varepsilon (x_0)$.  This will show that $f$ is continuous at $x_0$, and hence continuous as $x_0$ was arbitrary.

So, let $x\in U_q\setminus \Cls (U_p)$.  Then, $q\in Q_x$, and so $f(x)=\inf (Q_x)\leq q$.  On the other hand, as $x\notin U_p$, it cannot be in $U_r$ for any $r\leq p$ as $U_r\subseteq U_p$ for $r\leq p$.  Therefore, $p$ is a lower-bound for $Q_x$, and so $f(x)=\inf (Q_x)\geq p$.  Hence,
\begin{equation}
f(x_0)-\varepsilon <p\leq f(x)\leq p<f(x_0)+\varepsilon ,
\end{equation}
and we are done.
\end{proof}
\end{thm}
\begin{dfn}[Perfectly-$T_4$]\label{PerfectlyT4}
$X$ is \emph{perfectly-$T_4$} iff it is $T_1$ and any two disjoint closed can be perfectly-separated.
\begin{rmk}
For some reason, this is sometimes called $T_6$.  What is $T_5$ you ask?  Evidently $T_5$ means $T_1$ and every subspace is $T_4$.
\end{rmk}
\end{dfn}
\begin{exm}[A space that is completely-$T_4$ but not perfectly-$T_4$]
The Uncountable Fort Space from \cref{UncountableFortSpace} will once again do the trick.  In \cref{exm4.5.48}, we show that it is not perfectly-$T_2$, and so it is certainly not going to be perfectly-$T_4$.  We still need to check that it is completely-$T_4$.

Recall that the Uncountable Fort Space was defined to be $X\coloneqq \R$ with the closed sets being precisely the finite sets and also the sets which contained $0$.

So, let $C_1,C_2\subseteq X$ be disjoint closed sets.  Let us first do the case where neither $C_1$ nor $C_2$ contains $0\in X$.  Then, we may define $f:X\rightarrow [0,1]$ by
\begin{equation}
f(x)\coloneqq \begin{cases}0 & \text{if }x\in C_1 \\ 1 & \text{if }x\in C_2 \\ \tfrac{1}{2} & \text{otherwise}\end{cases}.
\end{equation}
The preimage of $0$ is $C_1$ is closed, the preimage of $1$ is $C_2$ is closed, and the preimage of $\frac{1}{2}$ contains $0\in X$ and so is closed.  Thus, this function is continuous.  Now suppose that $0\in C_1$.  Then, we may define $f:X\rightarrow [0,1]$ by
\begin{equation}
f(x)\coloneqq \begin{cases}0 & \text{if }x\in C_1 \\ 1 & \text{if }x\in C_2\end{cases}.
\end{equation}
The preimage of $0$ is $C_1$ is closed and the preimage of $1$ is $C_2$ is closed.  Thus, this function is continuous.
\end{exm}

\subsection{Summary}

We summarize what we have covered so far in this section.

First of all, there are several levels of separation between two different objects in a space
\begin{equation}
\begin{split}
\text{Distinct} & \Leftarrow \footnote{A two-point space with the indiscrete topology---see \cref{exm4.5.2}.}\text{Topologically-distinguishable}\Leftarrow \footnote{A two-point space with precisely one of the points open---see \cref{exm4.5.3}.}\text{Separated} \\
& \qquad \Leftarrow \footnote{A certain three-point space---see \cref{exm4.5.8}.}\text{Separated by neighborhoods}\Leftarrow \footnote{Another three-point space---see \cref{exm4.5.11}.}\text{Separated by closed neighborhoods} \\
& \qquad \qquad \Leftarrow \footnote{The Arens Square---see \cref{ArensSquare}.}\text{Completely-separated}\Leftarrow \footnote{The Uncountable Fort Space---see \cref{UncountableFortSpace}.}\text{Perfectly-separated}
\end{split}
\end{equation}
The arrows indicated implication of course, and all these implications are strict, as indicated by the examples referenced in the footnotes.

There are three `families' of separation axioms of spaces:  (i) separation of pairs of points, (ii) separation of closed sets from points, and (iii) and separation of disjoint closed sets.

In the first family:
\begin{enumerate}
\item Points being topologically-distinguishable is $T_0$ (\cref{T0}).
\item Points being separated is $T_1$ (\cref{T1}).
\item Points being separated by neighborhoods is $T_2$ (\cref{T2}).
\item Points being separated by closed neighborhoods is $T_{2\frac{1}{2}}$ (\cref{T212}).
\item Points being completely-separated is completely-$T_2$ (\cref{CompletelyT2}).
\item Points being perfectly-separated is perfectly-$T_2$ (\cref{PerfectlyT2}).
\end{enumerate}
In general, appending ``$\frac{1}{2}2$'', ``completely'', or ``perfectly'' to a separation axiom in this way changes whatever the separation axiom was now to ``separated by closed neighborhoods'', ``completely-separated'', and ``perfectly-separated'' respectively.

Closed sets and points, as well as closed sets and closed sets, are automatically separated, and so there are no separation axioms analogous to $T_0$ and $T_1$ for families (ii) and (iii).

For (ii) and (iii), in order for them to be directly comparable with (i), we require that points be closed (that is, we explicitly require spaces in families (ii) and (iii) to be $T_1$---see \cref{prp4.5.32}).  Without this extra assumption, the separation axioms are called ``regular'' and ``normal'' respectively.

Thus, for the second family:\footnote{Remember that we require the spaces to be $T_1$ in addition to these properties.}
\begin{enumerate}
\item Closed sets and points being separated by neighborhoods is $T_3$ (\cref{T3}).
\item Closed sets and points being separated by closed neighborhoods is $T_{3\frac{1}{2}}$ (\cref{T312}).
\item Closed sets and points being completely-separated is completely-$T_3$ (\cref{CompletelyT3}).
\item Closed sets and points being perfectly-separated is perfectly-$T_3$ (\cref{PerfectlyT3}).
\end{enumerate}

And similarly for the third family:
\footnote{Remember that we require the spaces to be $T_1$ in addition to these properties.}
\begin{enumerate}
\item Closed sets being separated by neighborhoods is $T_4$ (\cref{T3}).
\item Closed sets being separated by closed neighborhoods is $T_{4\frac{1}{2}}$ (\cref{T412}).
\item Closed sets being completely-separated is completely-$T_4$ (\cref{CompletelyT4}).
\item Closed set being perfectly-separated is perfectly-$T_4$ (\cref{PerfectlyT4}).
\end{enumerate}

We now illustrate how all these axioms are related to each other.  All implications are strict (unless otherwise indicated by a $\Leftrightarrow$), in which case the offending counter-example is given in the indicated footnote.  Perhaps the only real surprise\footnote{That's not to say that the counter-examples are easy---they're not (in fact, some of them are \emph{really fucking hard}, like, among-the-most-difficult-things-in-the-notes hard)---but rather that you probably expect that some counter-example exists, even if incredibly exotic.} is the equivalence of $T_{4\frac{1}{2}}$ and completely-$T_4$:  \nameref{UrysohnsLemma} (\cref{UrysohnsLemma}).\footnote{Though perhaps it's worth nothing that $T_{k\frac{1}{2}}$ is equivalent to completely-$T_k$ for $k=3,4$ but \emph{not} $k=2$.}
\begin{equation}\label{4.6.105}
\begin{tikzcd}
T_0 & & & & \\
T_1 \ar[u,Rightarrow,"\footnote{A two-point space in which precisely one point is open---see \cref{exm4.5.3}.}"] & & & & \\
T_2 \ar[u,Rightarrow,"\footnote{$\R$ with the cocountable topology---see \cref{exm4.2.8x}.}"] & & & & \\
T_{2\tfrac{1}{2}} \ar[u,Rightarrow,"\footnote{The Simplified Arens Square---see \cref{SimplifiedArensSquare}.}"] & T_3 \ar[l,Rightarrow,"\footnote{$\R$ with the cocountable extension topology---see \cref{CocountableExtensionTopology}.}"] & T_{3\tfrac{1}{2}} \ar[l,Leftrightarrow,"\footnote{See \cref{prp4.5.70}.}"] & T_4 \ar[l,Rightarrow,"\footnote{Niemytzki's Tangent Disk Topology---see \cref{NiemytzkisTangentDiskTopology}.}"] & T_{4\tfrac{1}{2}} \ar[l,Leftrightarrow,"\footnote{See \cref{prp4.5.88}.}"] \\
\text{Completely-}T_2 \ar[u,Rightarrow,"\footnote{The Arens Square---see \cref{ArensSquare}.}"] & & \text{Completely-}T_3 \ar[ll,Rightarrow,"\footnote{$\R$ with the cocountable extension topology---see \cref{CocountableExtensionTopology}.}"] \ar[u,Rightarrow,"\footnote{The Thomas Tent Space---see \cref{ThomasTentSpace}.}"] & & \text{Completely-}T_4 \ar[ll,Rightarrow,"\footnote{Niemytzki's Tangent Disk Topology---see \cref{NiemytzkisTangentDiskTopology}.}"] \ar[u,Leftrightarrow,"\footnote{Urysohn's Lemma, \cref{UrysohnsLemma}.}"] \\
\text{Perfectly-}T_2 \ar[u,Rightarrow,"\footnote{The Uncountable Fort Space--see \cref{UncountableFortSpace}.}"] & & \text{Perfectly-}T_3 \ar[ll,Rightarrow,"\footnote{$\R$ wit the cocountable extension topology---see \cref{CocountableExtensionTopology}.}"] \ar[u,Rightarrow,"\footnote{The Uncountable Fort Space---see \cref{UncountableFortSpace}.}"] & & \text{Perfectly-}T_4 \ar[ll,Rightarrow,"\footnote{Niemytzki's Tangent Disk Topology---see \cref{NiemytzkisTangentDiskTopology}.}"] \ar[u,Rightarrow,"\footnote{The Uncountable Fort Space---see \cref{UncountableFortSpace}.}"]
\end{tikzcd}.
\end{equation}

\section{The Intermediate and Extreme Value Theorems}

\subsection{Connectedness and the Intermediate Value Theorem}

You'll recall from calculus that the classical statement of the Intermediate Value Theorem is
\begin{textequation}
Let $f:[a,b]\rightarrow \R$ be continuous.  Then, for all $y$ between $f(a)$ and $f(b)$ (inclusive), there exists $x\in [a,b]$ such that $f(x)=y$.
\end{textequation}
We will see that the proper way to interpret this statement is that the image of $f$ is connected, so that if the image contains $f(a)$ and $f(b)$ (which it does by definition), then it must contain everything in-between as well.  Of course, in order to make this precise, we have to first define what it means to be connected.
\begin{dfn}[Connected and disconnected]\label{Connected}
Let $X$ be a topological space.  Then, $X$ is \emph{disconnected}\index{Disconnected} iff there exist disjoint nonempty open sets $U,V\subset X$ such that $X=U\cup V$.  $X$ is \emph{connected}\index{Connected} iff it is not disconnected.  A subset $S$ of $X$ is connected iff it is connected in its subspace topology.
\begin{rmk}
The intuition for the definition of disconnected of course is that we break-up the space into two separate pieces which have no overlap.
\end{rmk}
\begin{rmk}
Another way to say this is that $X$ is disconnected iff it has a partition with two open sets.
\end{rmk}
\begin{rmk}
Note that it is \emph{not} the case that $S$ is disconnected iff $S=U\cup V$ for nonempty disjoint subsets $U,V\subseteq X$.  The reason for the difference is that \emph{subsets of $S$ which are open in $S$ need not be open in $X$} (see \cref{exm4.1.14}).
\end{rmk}
\end{dfn}
\begin{exm}
Define $S\coloneqq [0,1]\cup [2,3]\subseteq \R \eqqcolon X$.  Then, $X$ is certainly disconnected because $U\coloneqq [0,1]$ and $V\coloneqq [2,3]$ are nonempty disjoint open subsets of $S$ such that $S=[0,1]\cup [2,3]$.  Note, however, that if we required $U$ and $V$ to be open \emph{in $X\coloneqq \R$}, then this would not show that $S$ is disconnected.  This is why we require that $U$ and $V$ are \emph{open in the subspace topology of $S$}, instead of open in the ambient space.
\end{exm}
\begin{prp}
Let $X$ be a topological space and let $\mathcal{U}$ be a collection of connected subsets of $X$ with nonempty intersection.  Then, $\bigcup _{U\in \mathcal{U}}U$ is connected.
\begin{proof}
To simplify notation, let us write $X'\coloneqq \bigcup _{U\in \mathcal{U}}U$.  We proceed by contradiction:  suppose that $\bigcup _{U\in \mathcal{U}}U$ is disconnected, so that we may write
\begin{equation}
X'=V\cup W
\end{equation}
for $V,W\subseteq X'$ open, nonempty, and disjoint.  Let $x_0\in \bigcap _{U\in \mathcal{U}}$ and without loss of generality assume that $x_0\in V$.  For each $U\in \mathcal{U}$, let us write
\begin{equation}
U_V\coloneqq U\cap V\text{ and }U_W\coloneqq U\cap W,
\end{equation}
so that
\begin{equation}
U=U_V\cup U_W
\end{equation}
for all $U\in \mathcal{U}$.  As $U$ is connected, it follows that, for each $U$, either $U_V$ or $U_W$ is empty.  However, we know that $x_0\in U_V$, and so in fact, we must have that $U_W=\emptyset$ for all $U\in \mathcal{U}$, which in turn implies that $W=\emptyset$:  a contradiction.
\end{proof}
\end{prp}
\begin{dfn}[Connected component]
Let $X$ be a topological space and let $x_1,x_2$.  Then, $x_1$ and $x_2$ are \emph{connected} (to each other) iff there exists a connected set $U\subseteq X$ with $x_1,x_2\in U$.
\begin{prp}
The relation of being connected to is an equivalence relation on $X$.
\begin{proof}
$x$ is connected to itself because $\{ x\}$ is connected.  The relation is symmetric because the definition of the relation is symmetric.  If $x_1$ is connected to $x_2$ and $x_2$ is connected to $x_3$, then there is some connected set $U$ which contains $x_1$ and $x_2$, and there is some connected set $V$ which contains $x_2$ and $x_3$.  As $U$ and $V$ both contain $x_2$, it follows from the previous proposition that $U\cup V$ is connected, and hence $x_1$ is connected to $x_3$.
\end{proof}
\end{prp}
A \emph{connected component}\index{Connected component} of $X$ is an equivalence class of some point with respect to the relation of being connected to.
\end{dfn}
\begin{dfn}[Totally disconnected]
Let $X$ be a topological space.  Then, $X$ is \emph{totally-disconnected} iff every nonempty connected component of $X$ is a point.
\end{dfn}
\begin{prp}
Let $X$ be a discrete topological space.  Then, $X$ is totally-disconnected.
\begin{proof}
Let $U\subseteq X$ have at least two distinct points $x_1$ and $x_2$.  Then,
\begin{equation}
U=\{ x_1\} \cup (U\setminus \{ x_1\} ),
\end{equation}
and as every subset in a discrete space is open, it follows that $U$ is disconnected.
\end{proof}
\end{prp}
\begin{exm}[$\N$, $\Z$, and $\Q$ are totally-disconnected]
That $\N$ and $\Z$ are totally-disconnected follows from the fact that they are discrete.

$\Q$ is also totally-disconnected,\footnote{In particular, there are totally-disconnected spaces which are not discrete.} but this is more difficult to see.  Let $U\subseteq \Q$ have at least two distinct points $q_1$ and $q_2$.  Without loss of generality, suppose that $q_1<q_2$.  Then, by `density' of $\Q ^{\comp}$ in $\R$ (\cref{thm3.3.76}), there is some $x\in \Q ^{\comp}$ with $q_1<x<q_2$.  Define
\begin{equation}
V\coloneqq (-\infty ,x)\cap U \text{ and }W\coloneqq (x,\infty )\cap U .
\end{equation}
Both $V$ and $W$ are open by the definition of the subspace topology (\cref{SubspaceTopology}) of $U$, and both are nonempty because $q_1<x$ and $q_2>x$.  Thus, as $U=V\cup W$, $U$ is disconnected, and hence $\Q$ is totally-disconnected.
\end{exm}

We mentioned at the beginning of this section that the proper way to interpret the Intermediate Value Theorem is the statement that the image of connected sets are connected.  There is one other thing we need to check though---we need to check that intervals in $\R$ are in fact connected.
\begin{thm}\label{thm4.5.14}
Let $I\subseteq \R$.  Then, $I$ is connected iff it is an interval.
\begin{rmk}
In particular, $\R$ itself is connected, in contrast with $\N$, $\Z$, and $\Q$.
\end{rmk}
\begin{proof}
$(\Rightarrow )$ Suppose that $I$ is connected.  Let $a,b\in I$ with $a\leq b$ and let $x\in \R$ with $a\leq x\leq b$.  We must show that $x\in I$.  If either $x=a$ or $x=b$, we are done, so we may as well suppose that $a<x<b$.  We proceed by contradiction:  suppose that $x\notin I$.  Then,
\begin{equation}
I=(I\cap (-\infty ,x))\cup (I\cap (x,\infty )),
\end{equation}
and so as $I$ is connected, we must have that either $I\cap (-\infty ,x)$ is empty or $I\cap (x,\infty )$ is empty.  But $a$ is in the former and $b$ is in the latter:  a contradiction.

\blankline
\noindent
$(\Leftarrow )$ Suppose that $I$ is an interval.  As $I$ is an interval, by \cref{prp3.3.70}, we have that $I=[(a,b)]$ for $a=\inf (I)$ and $b=\sup (I)$.\footnote{Recall that this notation just means that the end-points can be either open or closed; see the remark in \cref{prp3.3.70}.}  We wish to show that $I$ is connected.  If $a=b$, then either $I=\emptyset$ or $I=\{ a\}$, in which case $I$ is trivially connected.  Therefore, we may assume without loss of generality that $a<b$.

We proceed by contradiction:  suppose that $I$ is disconnected.  The, we have that $I=U\cup V$ for $U,V\subseteq I$ open in $I$ disjoint and nonempty.  As $U$ and $V$ are open in $I$ and cover $I$, we must have that at least one of them contains an open neighborhood of $a$, so without loss of generality, suppose that $[(a,x)\subseteq U$ for some $x\in \R$ with $a<x\leq b$.  Now define
\begin{equation}
S\coloneqq \left\{ x\in I :[(a,x)\subseteq U\right\} .
\end{equation}
We just showed that this set is nonempty.  It is also bounded above by $b$ as $b$ is in particular a bound of $I$.  Therefore, it has a supremum.  We wish to show that $\sup (S)=b$.

Note that $a<\sup (S)\leq b$, and so, because $I$ is an interval, either $\sup (S)\in I$ or $\sup (S)=b$.  In the latter case, we are done, so we may as well assume that $\sup (S)\in I$.

In the case we are considering in which $\sup (S)\in I$, we show that in fact $\sup (S)\in U$.  We proceed by contradiction:  suppose that $\sup (S)\in V$ (here is where we use the fact that $\sup (S)\in I$.  As $V$ is open, there is a neighborhood of $\sup (S)$ completely contained in $V$.  On the other hand, by \cref{prp1.4.11}, this neighborhood has to contain some element of $S$, which in turn would imply that it would have to contain some element of $U$.  But then $U$ intersects $V$:  a contradiction.  Therefore, $\sup (S)\in U$.

We now finish the proof that $\sup (S)=b$.  We certainly have that $\sup (S)\leq b$ as $b$ is an upper-bound of $S$.  Thus, it suffices to show that $\sup (S)\geq b$.  To show this, we proceed by contradiction:  suppose that $\sup (S)<b$.  Then, because $U$ is open, there is some $\varepsilon >0$ such that $(\sup (S)-\varepsilon ,\sup (S)+\varepsilon )\subseteq U$.  By \cref{prp1.4.11}, there must be some $x\in S$ with $\sup (S)-\varepsilon <x\leq \sup (S)$, so that $[(a,x)\subseteq U$.  But then,
\begin{equation}
[(a,x))\cup (\sup (S)-\varepsilon ,\sup (S)+\varepsilon )=[(a,\sup (S)+\varepsilon )\subseteq U,
\end{equation}
and so, in particular, there is some $x'>\sup (S)$ such that $[(a,x')\subseteq U$:  a contradiction.  Therefore, we must have that $\sup (S)=b$.

Now that we have finally succeeded in showing that $\sup (S)=b$, we finish the proof by coming to a contradiction of the assumption of disconnectedness.  As $\sup (S)=b$, this means that for every $x\in I$ with $a<x<b$, we have that $[(a,x)\subseteq U$, and hence
\begin{equation}
\bigcup _{a<x<b}[(a,x)=[(a,b)\subseteq U.
\end{equation}
Thus, either $V=\{ b\}$ or $V=\emptyset$:  a contradiction of being open in $I$ or being nonempty respectively.
\end{proof}
\end{thm}
And now we finally get to the statement of the `true' Intermediate Value Theorem.
\begin{thm}[Intermediate Value Theorem]\label{IntermediateValueTheorem}\index{Intermediate Value Theorem}
Let $f:X\rightarrow Y$ be a continuous function and let $S\subseteq X$ be connected.  Then, $f(S)$ is connected.
\begin{proof}
We proceed by contradiction:  suppose that $f(S)$ is disconnected.  Then, $f(S)=U\cup V$ for $U,V\subseteq f(S)$ open in $f(S)$ disjoint and nonempty.  By continuity, we have that\footnote{Also recall that $f^{-1}(f(S))\supseteq S$---see \cref{exrA.1.47}\ref{enmA.1.47.ii}.}
\begin{equation}
S\supseteq \footnote{This follows from the fact that we are applying the preimage of \emph{the restriction} of $f$ to $S$.}\restr{f}{S}^{-1}(U)\cup \restr{f}{S}^{-1}(V)\supseteq S,
\end{equation}
and so
\begin{equation}
S=\restr{f}{S}^{-1}(U)\cup \restr{f}{S}^{-1}(V).
\end{equation}
As $U$ and $V$ are open in $S$ and $f$ is continuous, $\restr{f}{S}^{-1}(U)$ and $\restr{f}{S}^{-1}(V)$ are open in $S$.  They also must be disjoint, for a point which lied in their intersection would be mapped into $U\cap S$ via $f$.  Therefore, because $S$ is connected, we have that either $\restr{f}{S}^{-1}(U)$ or $\restr{f}{S}^{-1}(V)$ is empty, which implies respectively that either $U$ or $V$ is empty:  a contradiction.
\end{proof}
\end{thm}
As a corollary of this (and the fact that a subnet of $\R$ is connected iff it is an interval---see \cref{thm4.5.14}), we have the classical statement of the Intermediate Value Theorem.
\begin{crl}[Classical Intermediate Value Theorem]\label{ClassicalIntermediateValueTheorem}\index{Classical Intermediate Value Theorem}
Let $f:[a,b]\rightarrow \R$ be continuous.  Then, $f([a,b])$ is an interval.  In particular, any element between $f(a)$ and $f(b)$ is in the image of $f$.
\begin{proof}
The ``in particular'' part follows from the definition of an interval, \cref{Interval}.

$[a,b]$ is connected by \cref{thm4.5.14}, and so, by the Intermediate Value Theorem, $f([a,b])$ is connected, and so by \cref{thm4.5.14} again, is an interval.
\end{proof}
\end{crl}

\subsection{Quasicompactness and the Extreme Value Theorem}

You'll recall from calculus that the classical statement of the Extreme Value Theorem is
\begin{textequation}
Let $f:[a,b]\rightarrow \R$ be continuous.  Then, there exists $x_1,x_2\in [a,b]$ such that $f(x_1)=\inf _{x\in [a,b]}\left\{ f(x)\right\}$ and $f(x_2)=\sup _{x\in [a,b]}\left\{ f(x)\right\}$.  In other words, continuous functions \emph{attain} their maximum and minimum on closed intervals.
\end{textequation}
This is actually a special case of (or follows easily from) a \emph{much} more general, elegant statement.
\begin{thm}[Extreme Value Theorem]\label{ExtremeValueTheorem}\index{Extreme Value Theorem}
Let $f:X\rightarrow Y$ be continuous and let $K\subseteq X$.  Then, if $K$ is quasicompact, then $f(K)$.
\begin{rmk}
In other words, the continuous image of a quasicompact set is quasicompact.
\end{rmk}
\begin{proof}
Suppose that $K$ is quasicompact.  Let $\mathcal{U}$ be an open cover of $f(K)$.  Then, $f^{-1}(\mathcal{U})\coloneqq \left\{ f^{-1}(U):U\in \mathcal{U}\right\}$ is an open cover of $K$, and therefore it has a finite subcover $\{ f^{-1}(U_1),\ldots f^{-1}(U_m)\}$.  In other words,
\begin{equation}
K\subseteq f^{-1}(U_1)\cup \cdots \cup f^{-1}(U_m),
\end{equation}
and hence
\begin{equation}
\begin{split}
f(K) & \subseteq f\left( f^{-1}(U_1)\cup \cdots \cup f^{-1}(U_m)\right) =\footnote{By \cref{exrA.1.30}\ref{enmA.1.30.iii}.}f\left( f^{-1}(U_1)\right) \cup \cdots \cup f\left( f^{-1}(U_m)\right) \\
& =\footnote{By \cref{exrA.1.47}\ref{enmA.1.47.i}.}\subseteq U_1\cup \cdots \cup U_m,
\end{split}
\end{equation}
so that $\{ U_1,\ldots ,U_m\}$ is a finite subcover of $\mathcal{U}$, and hence $f(K)$ is quasicompact.
\end{proof}
\end{thm}
And now we can present the classical version of the theorem.
\begin{crl}[Classical Extreme Value Theorem]\label{ClassicalExtremeValueTheorem}\index{Classical Extreme Value Theorem}
Let $f:[a,b]\rightarrow \R$ be continuous.  Then, $f([a,b])$ is closed and bounded.  In particular, it attains a maximum and minimum on $[a,b]$.
\begin{rmk}
I suppose that usually in calculus people don't make any statement regarding the image being an interval---this is separately stated as the Intermediate Value Theorem.
\end{rmk}
\begin{proof}
The ``in particular'' is a result of \cref{exr3.4.27}, the statement that closed bounded sets contain their supreumum and infimum.

By the \nameref{HeineBorelTheorem} (\cref{HeineBorelTheorem}), $[a,b]$ is quasicompact.  Therefore, by the Extreme Value Theorem, $f([a,b])$ is quasicompact, and therefore, closed and bounded, again by the \nameref{HeineBorelTheorem}.
\end{proof}
\end{crl}

In fact, we may as well just combine the classical statements into one.
\begin{crl}[Classical Intermediate-Extreme Value Theorem]\label{ClassicalIntermediateExtremeValueTheorem}\index{Classical Intermediate-Extreme Value Theorem}
Let $f:[a,b]\rightarrow \R$ be continuous.  Then, $f([a,b])$ is a closed, bounded, interval.
\end{crl}

\section{Local properties}

For most topological properties, there is a ``local'' version.
\begin{mdf}[Locally XYZ]\label{LocallyXYZ}
A topological space is \emph{locally XYZ}\index{Locally XYZ} iff each point has a neighborhood base consisting of sets that are XYZ.
\end{mdf}
Of particular importance are the notions of local connectedness, local quasicompactness, and locally (completely/perfectly)-$T_a$.
\begin{exr}
Show that if a space is $T_2$ then it is locally $T_2$.  Find a counter-example to show that converse is false.
\end{exr}
\begin{exr}
Show that if a space is locally quasicompact and $T_2$, then it is locally compact.  Find a counter-example to show the converse is false.
\end{exr}
The following is the result related to locally quasicompactness that will be used in the proof of \nameref{HaarHowesTheorem}.
\begin{prp}\label{prp5.2.4}
Let $X$ be a locally compact space, let $K\subseteq X$ be quasicompact, and let $U\subseteq X$ contain $X$.  Then, there is an open set $V\subseteq X$ with compact closure such that
\begin{equation}
K\subseteq V\subseteq \Cls (V)\subseteq U.
\end{equation}
\begin{proof}
We leave this as an exercise.
\begin{exr}
Complete the proof yourself.
\end{exr}
\end{proof}
\end{prp}


\chapter{Uniform spaces}\label{chp5}

\section{Motivation}

A uniform space is the most general context in which one can talk about concepts such as uniform continuity, uniform convergence, cauchyness, completeness, etc..  To formalize this notion, we will equip a set with a distinguished set of covers, called \emph{uniform covers}.  The example you should always keep in the back of your mind is the collection of all $\varepsilon$-balls for a \emph{fixed} $\varepsilon$:  $\mathcal{U}_\varepsilon \coloneqq \left\{ B_\varepsilon (x):x\in \R \right\}$.  The idea is that, somehow, all of the sets in the same uniform cover are of the `same size'.

Having specified the uniform covers, we will then be able to say things like a net $\lambda \mapsto x_\lambda$ is cauchy iff for every uniform cover $\mathcal{U}$, there is some $U\in \mathcal{U}$ such that $\lambda \mapsto x_\lambda$ is eventually contained in $U$.  In the case that the collection of uniform covers is $\left\{ \mathcal{U}_\varepsilon :\varepsilon >0\right\}$,\footnote{Disclaimer:  The collection of all the $\mathcal{U}_{\varepsilon}$ is not actually a uniformity but rather a \emph{uniform base}---see \cref{UniformBase}.} you can check that this is precisely the definition of cauchyness we had in $\R$ (\cref{dfn3.3.26}).

Moreover, the generalization from $\R$ to uniform spaces is not a needless abstraction.  Indeed, I am \emph{required} to cover metric spaces (\cref{MetricSpace}) in this course, and this is just a very special type of uniform space.  Indeed, essentially every topological space we look at in these notes---besides ones cooked up for the express purpose of producing a counter-example---has a canonical uniformity.  On the other hand, it is certainly not the case that every topological space we encounter will be a metric space.  For example, something as simple as all continuous functions from $\R$ to $\R$ has no canonical metric,\footnote{For what it's worth, I believe the topology is metrizable (homeomorphic to a metric space), but certainly not with any metric you would like to work with, much less a canonical one.} but is trivially a uniform space (because it is a topological group---see \cref{dfnB.7}).

\section{Basic definitions and facts}

A uniform space will wind-up being a set equipped with a special set of covers, the \emph{uniform covers}.  Of course, however, as you should expect, we cannot just take \emph{any} collection of covers and declare them to be the uniform covers---the collection of uniform covers has to satisfy certain reasonable properties, analogous to the properties satisfied by the collections of all $\varepsilon$-balls.  They key requirement is that the collection of uniform covers has to be \emph{downward-directed} with respect to a relation called \emph{star-refinement}.\footnote{You are not expected to know what either of these terms mean yet.}  Thus, before getting to the definition of a uniform space itself, we must say what we mean by ``star-refinement'' (we will say what we mean by ``downward-directed'' in the definition of a uniform space itself).

\subsection{Star-refinements}

\begin{dfn}[Star]\label{Star}
Let $X$ be a set, let $S\subseteq X$, and let $\mathcal{U}$ be a cover of $X$.  Then, the \emph{star}\index{Star} of $S$ with respect to $\mathcal{U}$, $\Star _{\mathcal{U}}(S)$,\index[notation]{$\Star _{\mathcal{U}}(S)$} is defined by
\begin{equation}
\Star _{\mathcal{U}}(S)\coloneqq \bigcup _{U\in \mathcal{U}\st U\cap S\neq \emptyset}U.
\end{equation}
The star of a point is the star of its singleton and denoted $\Star _{\mathcal{U}}(x)$\index[notation]{$\Star _{\mathcal{U}}(x)$}.
\begin{rmk}
In other words, the star of a set with respect to a cover is the union of all elements of the cover which intersect the set.
\end{rmk}
\end{dfn}
\begin{exm}[The star of the preimage is not the preimage of the star]
One might hope for the preimage of the star to be the star of the preimage, that is, for $f:X\rightarrow Y$, $V\subseteq Y$, and $\mathcal{V}$ a cover of $Y$, that
\begin{equation}
f^{-1}\left( \Star _{\mathcal{V}}(V)\right) =\Star _{f^{-1}(\mathcal{V})}(f^{-1}(V)).
\end{equation}
Unfortunately, this is not necessarily the case.  For example, take $f$ to be the inclusion $\R \hookrightarrow \R ^2$ (with image the $x$-asix), take $V$ to be any subset of $\R ^2$ which does not intersect the $x$-axis (e.g.~$V=\{ \coord{x,y}\in \R ^2:y=1\}$), and take $\mathcal{V}\coloneqq \{ \R ^2\}$, namely the cover of $\R ^2$ consisting of only $\R ^2$ itself.  Then,
\begin{equation}
\Star _{\mathcal{V}}(V)=\R ^2,
\end{equation}
and so
\begin{equation}
f^{-1}\left( \Star _{\mathcal{V}}(V)\right) =\R .
\end{equation}
On the other hand,
\begin{equation}
\Star _{f^{-1}(\mathcal{V})}(f^{-1}(V))=\emptyset 
\end{equation}
simply because $f^{-1}(V)=\emptyset$.
\end{exm}
Despite this, we always have one inclusion.
\begin{prp}\label{prpC.2.3}
Let $f:X\rightarrow Y$ be a function, let $V\subseteq Y$, and let $\mathcal{V}$ be a cover of $Y$.  Then,
\begin{equation}
\Star _{f^{-1}(\mathcal{V})}(f^{-1}(V))\subseteq f^{-1}\left( \Star _{\mathcal{V}}(V)\right) .
\end{equation}
Furthermore, if $f$ satisfies $f(f^{-1}(S))=S$ for all $S\subseteq X$, then we have equality.
\begin{rmk}
A sufficient condition for $f(f^{-1}(S))=S$ is for $f$ to be surjective because then $f$ has a right-inverse---see \cref{exrA.1.9}\ref{enmA.1.9.ii}
\end{rmk}
\begin{proof}
Note that we have that (\cref{exrA.1.30}\ref{enmA.1.30.ii} and \cref{exrA.1.47}\ref{enmA.1.47.i}) $V\cap V_0\supseteq f(f^{-1}(V)\cap f^{-1}(V_0))$.  Therefore, if $f^{-1}(V)$ intersects $f^{-1}(V_0)$, it must be the case that $V$ intersects $V_0$.  Furthermore, if we have that $f(f^{-1}(S))=S$, then we would in fact that that $V\cap V_0=f(f^{-1}(V)\cap f^{-1}(V_0))$, so that in this case $V$ intersects $V_0$ iff $f^{-1}(V)$ intersects $f^{-1}(V_0)$.
\begin{equation}
\begin{split}
\Star _{f^{-1}(\mathcal{V})}(f^{-1}(V_0)) & =\bigcup _{V\in \mathcal{V}\st f^{-1}(V)\cap f^{-1}(V_0)\neq \emptyset}f^{-1}(V) \\
& =\footnote{\cref{exrA.1.30}\ref{enmA.1.30.i}}f^{-1}\left( \bigcup _{V\in \mathcal{V}\st f^{-1}(V)\cap f^{-1}(V_0)\neq \emptyset}V\right) \\
& \subseteq \footnote{Here we are using the fact that $f^{-1}(V)$ intersects $f^{-1}(V_0)$ implies that $V$ intersects $V_0$.  Also note that we have equality here if $f(f^{-1}(S))=S$.}f^{-1}\left( \bigcup _{V\in \mathcal{V}\st V\cap V_0\neq \emptyset}V\right) =f^{-1}\left( \Star _{\mathcal{V}}(V_0)\right) .
\end{split}
\end{equation}
\end{proof}
\end{prp}
We also have the `dual' counter-example and result for the image.
\begin{exm}[The star of the image is not the image of the star]
Take $f:\R ^2\rightarrow \R$ to be the projection onto the $x$-axis, define
\begin{equation}
\mathcal{U}\coloneqq \{ [m+y,m+1+y]\times \{ y\} :m\in \Z ,\ y\in \R \} ,
\end{equation}
and $U_0\coloneqq (0,1)\times \{ 0\}$.  Then,
\begin{equation}
\Star_{\mathcal{U}}(U_0)=[0,1]\times \{ 0\} ,
\end{equation}
and so
\begin{equation}
f\left( \Star _{\mathcal{U}}(U_0)\right) =[0,1].
\end{equation}
On the other hand,
\begin{equation}
f(\mathcal{U})=\{ [m+y,m+1+y]:m\in \Z ,\ y\in \R \} 
\end{equation}
and $f(U_0)=(0,1)$, and so $f(U_0)$ intersects $[m+y,m+1+y]$ for $m=0$ and $-1<y<1$, and so
\begin{equation}
\Star _{f(\mathcal{U})}f(U_0)\supseteq \bigcup _{-1<y<1}[y,y+1]=(-1,2),
\end{equation}
which is strictly larger than $f\left( \Star _{\mathcal{U}}(U_0)\right)$.
\end{exm}
\begin{prp}\label{prp4.2.17}
Let $f:X\rightarrow Y$ be a function, let $U\subseteq X$, and let $\mathcal{U}$ be a cover of $X$.  Then,
\begin{equation}
\Star _{f(\mathcal{U})}(f(U))\supseteq f(\Star _{\mathcal{U}}(U)).
\end{equation}
Furthermore, if $f$ is surjective and satisfies $f^{-1}(f(S))=S$ for all $S\subseteq X$, then we have equality.
\begin{rmk}
Requiring that $f$ be surjective is not really a big deal---we need $f$ to be surjective for the image of a cover to be a cover anyways.
\end{rmk}
\begin{rmk}
A sufficient condition for $f^{-1}(f(S))=S$ is for $f$ to be injective because then $f$ has a left-inverse---see \cref{exrA.1.9}\ref{enmA.1.9.i}.
\end{rmk}
\begin{proof}
Note that we have that (\cref{exrA.1.30}\ref{enmA.1.30.iv} and \cref{exrA.1.47}\ref{enmA.1.47.ii}) $U\cap U_0\subseteq f^{-1}(f(U)\cap f(U_0))$.  Therefore, if $U$ intersects $U_0$, it must be the case that $f(U)$ intersects $f(U_0)$.  Furthermore, if we have that $f$ is surjective and $f^{-1}(f(S))=S$ for all $S\subseteq X$, then we have that
\begin{equation}
U\cap U_0=f^{-1}(f(U))\cap f^{-1}(f(U_0))=f^{-1}(f(U)\cap f(U_0)),
\end{equation}
and hence
\begin{equation}
f(U\cap U_0)=f(U)\cap f(U_0),
\end{equation}
so in this case $U$ intersects $U_0$ iff $f(U)$ intersects $f(U_0)$.  Hence,
\begin{equation}
\begin{split}
f(\Star _{\mathcal{U}}(U)) & =f\left( \bigcup _{U'\in \mathcal{U}\st U'\cap U\neq \emptyset}U'\right) =\footnote{\cref{exrA.1.30}\ref{enmA.1.30.iii}}\bigcup _{U'\in \mathcal{U}\st U'\cap U\neq \emptyset}f(U') \\
& \subseteq \footnote{Here we are using the fact that $U$ intersects $U_0$ implies that $f(U)$ intersects $f(U_0)$.  Also note that we have equality here if $f$ is surjective and $f^{-1}(f(S))=S$.}\bigcup _{U'\in \mathcal{U}\st f(U')\cap f(U)\neq \emptyset}f(U')=\Star _{f(\mathcal{U})}f(U).
\end{split}
\end{equation}
\end{proof}
\end{prp}

\begin{dfn}[Refinement and star-refinement]\label{dfnC.1}
Let $X$ be a set, and let $\mathcal{U}$ and $\mathcal{V}$ be covers on $X$.
\begin{enumerate}
\item $\mathcal{U}$ is a \emph{refinement}\index{Refinement} of $\mathcal{V}$, written $\mathcal{U}\preceq \mathcal{V}$\index[notation]{$\mathcal{U}\preceq \mathcal{V}$} iff for every $U\in \mathcal{U}$ there is some $V\in \mathcal{V}$ such that $U\subseteq V$.
\item $\mathcal{U}$ is a \emph{star-refinement}\index{star-refinement} of $\mathcal{V}$, written $\mathcal{U}\llcurly \mathcal{V}$\index[notation]{$\mathcal{U}\llcurly \mathcal{V}$} iff for every $U\in \mathcal{U}$ there is a $V\in \mathcal{V}$ such that $\Star _{\mathcal{U}}(U)\subseteq V$.
\end{enumerate}
\begin{rmk}
The intuition is that every element of $\mathcal{U}$ is small enough to be contained in some element of $\mathcal{V}$.
\end{rmk}
\begin{rmk}
In other words, $\mathcal{U}$ is a star-refinement of $\mathcal{V}$ iff for all $U\in \mathcal{U}$, there is some $V\in \mathcal{V}$ such that, whenever $U'\in \mathcal{U}$ intersects $U$, it follows that $U'\subseteq V$.  The intuition for star-refinements is the same as for refinements, except that a star-refinement is \emph{much} finer than a mere refinement.
\end{rmk}
\end{dfn}
\begin{exr}
Show that $\preceq$ is a preorder, but not a partial-order.
\end{exr}
\begin{exr}\label{exr4.2.22}
Show that $\llcurly$ is transitive, but not even reflexive.
\end{exr}
Any two covers always have a common refinement.  In fact, they have a canonical (but not unique!) largest one.
\begin{dfn}[Meet of covers]
Let $X$ be a set, and let $\mathcal{U}$ and $\mathcal{V}$ be covers of $X$.  Then, the \emph{meet}\index{Meet (of covers)} of $\mathcal{U}$ and $\mathcal{V}$, $\mathcal{U}\wedge \mathcal{V}$\index[notation]{$\mathcal{U}\wedge \mathcal{V}$}, is defined by
\begin{equation}
\mathcal{U}\wedge \mathcal{V}\coloneqq \{ U\cap V:U\in \mathcal{U}\text{ and }V\in \mathcal{V}\} .
\end{equation}
\begin{rmk}
The term ``meet'' and notation ``$\mathcal{U}\wedge \mathcal{V}$'' is notation taken from the theory of partially-ordered sets where $x\wedge y$ (the \emph{meet}) of $x$ and $y$ is defined to be $\inf \{ x,y\}$.  In our case, however, this is abuse of notation and terminology as $\preceq$ is not a partial-order (and so infinma need not be unique---see \cref{exr1.4.4}).
\end{rmk}
\end{dfn}
\begin{exr}
Show that (i) $\mathcal{U}\wedge \mathcal{V}\preceq \mathcal{U},\mathcal{V}$; and (ii) if $\mathcal{W}$ refines both $\mathcal{U}$ and $\mathcal{V}$, then it refines $\mathcal{U}\wedge \mathcal{V}$.  Find an example to show that it is \emph{not} the unique such cover with these two properties.
\end{exr}
\begin{exr}
Does every cover have \emph{any} maximal star-refinement?
\end{exr}

Later it will be useful to know that taking the preimage preserves (star-)refinement.  But first, we need to be a bit careful about what we mean by the image and preimage of a cover.
\begin{dfn}[Image and preimage of a cover]
Let $f:X\rightarrow Y$ be a function, let $\mathcal{U}$ be a cover of $X$, and $\mathcal{V}$ be a cover of $Y$.  Then, the \emph{image} of $\mathcal{U}$, $f(\mathcal{U})$\index[notation]{$f(\mathcal{U})$}, is defined by
\begin{equation}
f(\mathcal{U})\coloneqq \{ f(U):U\in \mathcal{U}\} .
\end{equation}
The \emph{preimage} of $\mathcal{V}$, $f^{-1}(\mathcal{V})$\index[notation]{$f^{-1}(\mathcal{V})$}, is defined by
\begin{equation}
f^{-1}(\mathcal{V})\coloneqq \{ f^{-1}(V):V\in \mathcal{V}\} .
\end{equation}
\begin{rmk}
We needed to make these definitions because, technically speaking, we only defined the image an preimage of \emph{subsets} of $X$ and $Y$ respectively.  As $\mathcal{U}$ and $\mathcal{V}$ are subsets of $2^X$ and $2^Y$ respectively, not $X$ and $Y$, to talk about their `usual' image and preimage, we would need to have a function from $2^X$ to $2^Y$.\footnote{Actually, given a function $f:X\rightarrow Y$, we obtain a function from $2^X$ to $2^Y$ \emph{and} a function from $2^Y$ to $2^X$---$f:2^X\rightarrow 2^Y$ (the function that sends a set to its image) and $f^{-1}:2^Y\rightarrow 2^X$ (the function that sends a set to its preimage) respectively.  The preimage of a cover is the image of the cover with respect to the preimage function $f^{-1}:2^Y\rightarrow 2^X$.  Likewise, the image of a cover is the image of the cover with respect to the image function $f:2^X\rightarrow 2^Y$.}  In particular, note that the definition of the preimage of a cover is \emph{not} $\{ U\in 2^X:f(U)\in \mathcal{V}\}$.
\end{rmk}
\end{dfn}
\begin{prp}\label{prpB.2.12}
Let $f:X\rightarrow Y$ be a function and let $\mathcal{U}$ and $\mathcal{V}$ be covers of $Y$ such that $\mathcal{U}\preceq \mathcal{V}$ ($\mathcal{U}\llcurly \mathcal{V}$).  Then, $f^{-1}(\mathcal{U})\preceq f^{-1}(\mathcal{V})$ ($f^{-1}(\mathcal{U})\llcurly f^{-1}(\mathcal{V})$).
\begin{proof}
We first do the case with $\mathcal{U}\preceq \mathcal{V}$.  Let $f^{-1}(U)\in f^{-1}(\mathcal{U})$ with of course $U\in \mathcal{U}$.  Then, as $\mathcal{U}\preceq \mathcal{V}$, there is some $V\in \mathcal{V}$ such that $U\subseteq V$.  Then, $f^{-1}(U)\subseteq f^{-1}(V)$, and so $f^{-1}(\mathcal{U})\preceq f^{-1}(\mathcal{V})$.

Now we do the case $\mathcal{U}\llcurly \mathcal{V}$.  Let $f^{-1}(U)\in f^{-1}(\mathcal{U})$ with of course $U\in \mathcal{U}$.  Then, as $\mathcal{U}\llcurly \mathcal{V}$, there is some $V\in \mathcal{V}$ such that $\Star _{\mathcal{U}}(U)\subseteq V$.  Hence,
\begin{equation}
\Star _{f^{-1}(\mathcal{U})}(f^{-1}(U))\subseteq f^{-1}\left( \Star _{\mathcal{U}}(U)\right) \subseteq f^{-1}(V),
\end{equation}
where we have applied \cref{prpC.2.3}, and so $\mathcal{U}\llcurly \mathcal{V}$.
\end{proof}
\end{prp}
\begin{exr}
Show that if $\mathcal{U}\preceq \mathcal{V}$, then $f(\mathcal{U})\preceq f(\mathcal{V})$.
\end{exr}
Unfortunately, however, in general, it will not be the case that the image preserves star-refinements.
\begin{exr}
Find an example of covers $\mathcal{U}$ and $\mathcal{V}$ with $\mathcal{U}\llcurly \mathcal{V}$, but $f(\mathcal{U})$ not a star-refinment of $f(\mathcal{V})$.
\end{exr}
However, in special case, it will.
\begin{prp}\label{prp4.2.17x}
Let $f:X\rightarrow Y$ be a function and let $\mathcal{U}$ and $\mathcal{V}$ be covers of $X$ such that $\mathcal{U}\llcurly \mathcal{V}$.  Then, if $f$ is surjective and $f^{-1}(f(U))=U$ for all $U\in \mathcal{U}$, then $f(\mathcal{U})\llcurly f(\mathcal{V})$.
\begin{proof}
Suppose that $f$ is surjective and $f^{-1}(f(S))=S$ for all $S\subseteq X$.
\begin{exr}
Using the fact that $f$ preserves stars by \cref{prp4.2.17} to show that $f(\mathcal{U})\llcurly f(\mathcal{V})$.
\end{exr}
\end{proof}
\end{prp}

\subsection{Uniform spaces}

\begin{dfn}[Uniform space]\label{UniformSpace}
A \emph{uniform space}\index{Uniform space} is a set $X$ equipped with a nonempty collection $\widetilde{\mathcal{U}}$ of covers, the \emph{uniformity}\index{Uniformity}, such that
\begin{enumerate}
\item \label{UniformSpace.UpwardClosed}(Upward-closed)\index{Upward-closed} if $\mathcal{U}\in \widetilde{\mathcal{U}}$ and $\mathcal{U}\llcurly \mathcal{V}$, then $\mathcal{V}\in \widetilde{\mathcal{U}}$; and
\item \label{UniformSpace.DownwardDirected}(Downward-directed)\index{Downward-directed} if $\mathcal{U},\mathcal{V}\in \widetilde{\mathcal{U}}$, then there is some $\mathcal{W}\in \widetilde{\mathcal{U}}$ such that $\mathcal{W}\llcurly \mathcal{U}$ and $\mathcal{W}\llcurly \mathcal{V}$.
\end{enumerate}
\begin{rmk}
The elements of $\widetilde{\mathcal{U}}$ are \emph{uniform covers}\index{Uniform covers}.
\end{rmk}
\begin{rmk}
The intuition is that, in a given uniform cover $\mathcal{U}$, every element of $\mathcal{U}$ is `of the same size' (think $\mathcal{U}\coloneqq \{ B_{\varepsilon}(x):x\in \R \}$ for a \emph{fixed} $\varepsilon >0$).
\end{rmk}
\begin{rmk}
Note that, by taking $\mathcal{U}=\mathcal{V}$ in \ref{UniformSpace.DownwardDirected}, we see that, in particular, every uniform cover is star-refined by some other uniform cover.
\end{rmk}
\begin{rmk}
Note that the cover $\{ X\}$ is an element of every uniformity.  This follows from the fact that any collection of uniform covers is required to be nonempty and the fact that collections of uniform covers are upward-closed with respect to star-refinement.  (We mention this because sometimes that $\{ X\}$ is a uniform cover is taken as an axiom, in place of the requirement that the collection of uniform covers simply be nonempty.)
\end{rmk}
\end{dfn}
Of incredible importance is that uniformities \emph{define} a canonical topology.  Thus, we can think of uniform spaces as topological spaces with \emph{extra structure}.
\begin{prp}[Uniform topology]\label{UniformTopology}
Let $\coord{X,\widetilde{\mathcal{U}}}$ be a uniform space.  Then, for $x\in X$,
\begin{equation}
\mathcal{B}_x\coloneqq \left\{ \Star _{\mathcal{U}}(x):\mathcal{U}\in \widetilde{\mathcal{U}}\right\}
\end{equation}
is a neighborhood base at $x$.  The topology defined by this neighborhood base is the \emph{uniform topology}\index{Uniform topology} on $X$ with respect to $\widetilde{\mathcal{U}}$.
\begin{rmk}
Unless otherwise stated, uniformities are \emph{always} equipped with the uniform topology.
\end{rmk}
\begin{proof}
Let $\Star _{\mathcal{U}_1}(x),\Star _{\mathcal{U}_2}(x)\in \mathcal{B}_x$.  Let $\mathcal{U}_3$ be a common star-refinement of both $\mathcal{U}_1$ and $\mathcal{U}_2$.  We wish to show that
\begin{equation}
\Star _{\mathcal{U}_3}(x)\subseteq \Star _{\mathcal{U}_1}(x),\Star _{\mathcal{U}_2}(x).
\end{equation}
By $1\leftrightarrow 2$ symmetry, it suffices to just prove one of these inclusions.  By definition, we have
\begin{equation}
\Star _{\mathcal{U}_3}(x)\coloneqq \bigcup _{U\in \mathcal{U}_3\st x\in U}U.
\end{equation}
So, let $U\in \mathcal{U}_3$ contain $x$.  Because $\mathcal{U}_3$ star-refines $\mathcal{U}_1$, there is some $V\in \mathcal{U}_1$ such that
\begin{equation}
\Star _{\mathcal{U}_3}(U)\subseteq V.
\end{equation}
In particular, $U\subseteq V$.  Then, $V$ contains $x$, and so $V\subseteq \Star _{\mathcal{U}_1}(x)$, and so $U\subseteq \Star _{\mathcal{U}_1}(x)$.  It follows that
\begin{equation}
\Star _{\mathcal{U}_3}(x)\subseteq \Star _{\mathcal{U}_1}(x),
\end{equation}
and we are done.
\begin{rmk}
Note that this proof did not make use of the upward-closed axiom.  Thus, in fact, uniform bases (see below in \cref{UniformBase}) suffice to define the uniform topology as well.
\end{rmk}
\end{proof}
\end{prp}
\begin{exm}[Discrete and indiscrete uniform spaces]
Just as with topological spaces, we can always put the largest and the smallest uniformity on a set $X$.  The former case, in which every cover of $X$ is a uniform cover, is the \emph{discrete uniformity}\index{Discrete uniformity}, and the latter, in which the only uniform cover is $\{ X\}$, is the \emph{indiscrete uniformity}\index{Indiscrete uniformity}.
\begin{exr}
Show that the uniform topology with respect to the discrete uniformity is the discrete topology and that the uniform topology with respect to the indiscrete uniformity is the indiscrete topology.
\end{exr}
\end{exm}

Just as we have continuous maps between topological spaces, we have \emph{uniformly}-continuous maps between uniform spaces.
\begin{dfn}[Uniformly-continuous function]
Let $f:X\rightarrow Y$ be a function between uniform spaces.  Then, $f$ is \emph{uniformly-continuous} iff the preimage of every uniform cover is a uniform cover.
\end{dfn}
\begin{exm}[The category of uniform spaces]
The category of uniform spaces is the category $\Uni$\index[notation]{$\Uni$} whose collection of objects $\Uni _0$ is the collection of all uniform spaces, for every uniform space $X$ and uniform space $Y$ the collection of morphisms from $X$ to $Y$, $\Mor _{\Uni}(X,Y)$, is precisely the set of all uniformly-continuous functions from $X$ to $Y$, composition is given by ordinary function composition, and the identities of the category are the identity functions.
\begin{exr}
Show that the composition of two uniformly-continuous functions is uniformly-continuous.
\begin{rmk}
Note that this is something you need to check in order for $\Uni$ to actually form a category $(\Mor _{\Uni}(X,Y)$ needs to be closed under composition).  You also need to verify the identity function is uniformly-continuous, but this is trivial (the preimage of a cover is itself, so\textellipsis ).
\end{rmk}
\end{exr}
\end{exm}
\begin{dfn}[Uniform-homeomorphism]\label{UniformHomeomorphism}
Let $f:X\rightarrow Y$ be a function between uniform spaces.  Then, $f$ is a \emph{uniform-homeomorphism}\index{Uniform-homeomorphism} iff it is an isomorphism in $\Uni$.
\end{dfn}
\begin{exr}
Show that a function is a uniform-homeomorphism iff (i) it is bijective, (ii) it is uniformly-continuous, and (iii) its inverse is uniformly-continuous.
\end{exr}
\begin{exr}
Show that if a function is uniformly-continuous, then it is continuous.
\end{exr}
\begin{exr}
Find an example of a function that is bijective and uniformly-continuous, but not a uniform-homeomorphism.
\end{exr}

\subsection{Uniform bases, generating collections, and the initial and final uniformities}

It is usually convenient to not specify every uniform cover explicitly, but rather, to specify a certain collection of uniform covers and then take the uniformity `generated' by this collection.  This is analogous to how it is often convenient to only specify a base for a topology.
\begin{dfn}[Uniform base]\label{UniformBase}
Let $X$ be a uniform space and let $\widetilde{\mathcal{B}}$ be a collection of uniform covers of $X$.  Then, $\widetilde{\mathcal{B}}$ is a \emph{uniform base}\index{Uniform base} for the uniformity on $X$ iff the statement that a cover $\mathcal{U}$ is a uniform cover is equivalent to the statement that there is some $\mathcal{B}\in \widetilde{\mathcal{B}}$ such that $\mathcal{B}\llcurly \mathcal{U}$.
\begin{rmk}
You should compare this to the definition of a base for a topology (\cref{Base}).
\end{rmk}
\end{dfn}
And just like with bases, the real reason uniform bases are important is because they allow us to \emph{define} uniformities.  Thus, same as before, it is important to know when a collection of covers of a set form a uniform base for some uniformity.
\begin{prp}\label{prp4.3.2}
Let $X$ be a set and let $\widetilde{\mathcal{B}}$ be a nonempty collection of covers of $X$.  Then, there exists a unique uniformity for which $\widetilde{\mathcal{B}}$ is a uniform base iff $\widetilde{\mathcal{B}}$ is downward-directed with respect to $\llcurly$.
\begin{rmk}
Just as we did for bases, if a set $X$ does not a priori come with a uniformity, we will still refer to any collection of covers that is downward-directed with respect to $\llcurly$ as a \emph{uniform base}.
\end{rmk}
\begin{proof}
$(\Rightarrow )$ Suppose that there exists a uniformity for which $\widetilde{\mathcal{B}}$ is a uniform base.  Let $\mathcal{B},\mathcal{C}\in \widetilde{\mathcal{B}}$.  Then, there is certainly some uniform cover $\mathcal{U}$ which star-refines both $\mathcal{B}$ and $\mathcal{C}$ (recall that covers in $\widetilde{\mathcal{B}}$ are a priori taken to be uniform covers).  Thus, we will be done if we can show that $\mathcal{U}\in \widetilde{\mathcal{B}}$.  However, because $\mathcal{U}$ is a uniform cover and $\widetilde{\mathcal{B}}$ is a uniform base, there is some $\mathcal{D}\in \widetilde{\mathcal{B}}$ such that $\mathcal{D}\llcurly \mathcal{U}$.  As $\mathcal{U}$ star-refines both $\mathcal{B}$ and $\mathcal{C}$, it follows that $\mathcal{D}$ does as well.

\blankline
\noindent
$(\Leftarrow )$ Suppose that $\widetilde{\mathcal{B}}$ is downward-directed with respect to $\llcurly$.  Define $\widetilde{\mathcal{U}}$ to be the collection of covers that are star-refined by some element of $\widetilde{\mathcal{B}}$.  As $\widetilde{\mathcal{B}}$ is nonempty, so to is $\widetilde{\mathcal{U}}$ (it must contain $\{ X\}$).  By the definition of uniform bases, this was the only possibility.  As $\widetilde{\mathcal{B}}$ is nonempty and downward-directed with respect to $\llcurly$, it follows that $\widetilde{\mathcal{U}}$ contains $\widetilde{\mathcal{B}}$, and in particular is nonempty.  $\widetilde{\mathcal{U}}$ is upward-closed with respect to $\llcurly$ because if $\mathcal{U}$ is a uniform cover and star-refines $\mathcal{V}$, then there is some $\mathcal{B}\in \widetilde{\mathcal{B}}$ that star-refines $\mathcal{U}$, and hence in turn star-refines $\mathcal{V}$.  We now check that it is downward-directed with respect to $\llcurly$.  If $\mathcal{U}$ and $\mathcal{V}$ are covers, then there are $\mathcal{B},\mathcal{C}\in \widetilde{\mathcal{B}}$ that star-refine $\mathcal{U}$ and $\mathcal{V}$ respectively.  Because $\widetilde{\mathcal{B}}$ is downward-directed, there is then some $\mathcal{D}\in \widetilde{\mathcal{B}}$ which star-refines both $\mathcal{B}$ and $\mathcal{C}$, and hence both $\mathcal{U}$ and $\mathcal{V}$.
\end{proof}
\end{prp}
As we mentioned above at the end of the proof of \cref{UniformTopology}, uniform bases define the uniform topology just as well as the entire uniformity.
\begin{crl}
Let $\widetilde{\mathcal{B}}$ be a uniform base for the uniform space $X$.  Then, for $x\in X$,
\begin{equation}
\mathcal{B}_x\coloneqq \left\{ \Star _{\mathcal{U}}(x):\mathcal{U}\in \widetilde{\mathcal{U}}\right\}
\end{equation}
is a neighborhood base at $x$ for the uniform topology.
\end{crl}
\begin{exr}
Show that $\widetilde{\mathcal{U}}\coloneqq \left\{ \{ x\} :x\in X\right\}$ is a uniform base for the discrete uniformity.
\begin{rmk}
That is, the collection consisting of just a \emph{single} open cover, which itself is just the collection of all singletons, forms a uniform base.  In other words, you need to check that $\mathcal{U}$ star-refines itself.\footnote{Of course, while $\llcurly$ in general is not reflexive, that doesn't mean we can't at least have $\mathcal{U}\llcurly \mathcal{U}$ \emph{some} of the time.}
\end{rmk}
\begin{rmk}
The `dual' result for the indiscrete uniformity is trivial---the indiscrete uniformity by definition only has a single cover to begin with (namely $\{ X\}$), and that single cover certainly forms a uniform base for itself.
\end{rmk}
\end{exr}
We will want to check that two collections of uniform covers are the same by just looking at uniform bases.  We did not present it because we did not need to make use of it, but of course there is an analogous result for bases of topological spaces.
\begin{prp}\label{prp1.6}
Let $X$ be a set, and let $\widetilde{\mathcal{B}}$ and $\widetilde{\mathcal{C}}$ be uniform bases on $X$.  Then, $\widetilde{\mathcal{B}}$ and $\widetilde{\mathcal{C}}$ determine the same uniformity iff for every $\mathcal{B}\in \widetilde{\mathcal{B}}$, there is some $\mathcal{C}\in \widetilde{\mathcal{C}}$ with $\mathcal{C}\llcurly \mathcal{B}$; and for every $\mathcal{C}\in \widetilde{\mathcal{C}}$, there is some $\mathcal{B}\in \widetilde{\mathcal{B}}$ with $\mathcal{B}\llcurly \mathcal{C}$.
\begin{proof}
$(\Rightarrow )$ Suppose that $\widetilde{\mathcal{B}}$ and $\widetilde{\mathcal{C}}$ determine the same uniformity.  Let $\mathcal{B}\in \widetilde{\mathcal{B}}$.  Then, $\mathcal{B}$ is in particular in the uniformity generated by $\widetilde{\mathcal{B}}$, and hence in the uniformity generated by $\widetilde{\mathcal{C}}$.  Thus, there is some $\mathcal{C}\in \widetilde{\mathcal{C}}$ such that $\mathcal{C}\llcurly \mathcal{B}$.  By symmetry $\widetilde{\mathcal{B}}\leftrightarrow \widetilde{\mathcal{C}}$, the other result is true as well.

\blankline
\noindent
$(\Leftarrow )$ Suppose that for every $\mathcal{B}\in \widetilde{\mathcal{B}}$, there is some $\mathcal{C}\in \widetilde{\mathcal{C}}$ with $\mathcal{C}\llcurly \mathcal{B}$; and for every $\mathcal{C}\in \widetilde{\mathcal{C}}$, there is some $\mathcal{B}\in \widetilde{\mathcal{B}}$ with $\mathcal{B}\llcurly \mathcal{C}$.  Let $\mathcal{U}$ be a uniform cover in the uniformity determined by $\widetilde{\mathcal{B}}$.  Then, there is some $\mathcal{B}\in \widetilde{\mathcal{B}}$ such that $\mathcal{B}\llcurly \mathcal{U}$.  By the hypothesis, then, there is some $\mathcal{C}\in \widetilde{\mathcal{C}}$ with $\mathcal{C}\llcurly \mathcal{B}\llcurly \mathcal{U}$.  Thus, $\mathcal{U}$ is in the uniformity determined by $\widetilde{\mathcal{C}}$.  By $\widetilde{\mathcal{B}}\leftrightarrow \widetilde{\mathcal{C}}$ symmetry, the reverse inclusion is also true.
\end{proof}
\end{prp}
We will also want to check whether a function is uniformly-continuous by simply looking at a uniform base.
\begin{prp}\label{prpB.3.4}
Let $f:(X,\widetilde{\mathcal{U}})\rightarrow (Y,\widetilde{\mathcal{V}})$ be a function between uniform spaces and let $\widetilde{\mathcal{C}}$ be a uniform base for $\widetilde{\mathcal{V}}$.  Then, $f$ is uniformly-continuous iff $f^{-1}(\mathcal{C})\in \widetilde{\mathcal{U}}$ for each $\mathcal{C}\in \widetilde{\mathcal{C}}$.
\begin{proof}
$(\Rightarrow )$ There is nothing to check (because every open cover in a uniform base is itself a uniform cover).

\blankline
\noindent
$(\Leftarrow )$ Suppose that $f^{-1}(\mathcal{C})\in \widetilde{\mathcal{U}}$ for each $\mathcal{C}\in \widetilde{\mathcal{C}}$.  We need to show that the preimave of \emph{every} uniform cover is a uniform cover.  So, let $\mathcal{V}\in \widetilde{\mathcal{V}}$.  Then, there is some $\mathcal{C}\in \widetilde{\mathcal{C}}$ such that $\mathcal{C}\llcurly \mathcal{V}$.  Then, by \cref{prpB.2.12}, it follows that $f^{-1}(\mathcal{C})\llcurly f^{-1}(\mathcal{V})$.  As $f^{-1}(\mathcal{C})\in \widetilde{\mathcal{U}}$ and $\widetilde{\mathcal{U}}$ is upward-closed with respect to $\llcurly$, it follows that $f^{-1}(\mathcal{V})\in \widetilde{\mathcal{U}}$, so that $f$ is uniformly-continuous.
\end{proof}
\end{prp}

A uniform space does not start its life as a topological space---instead, it obtains a canonical topology from its uniformity.  In particular, it does not make sense a priori to just restrict to open covers.  On the other hand, once we specify the uniform covers, a topology is determined, and then, it turns out (as the following proposition shows), that it suffices to just look at \emph{open} uniform covers, more precisely, those uniform covers obtained by taking the interior of the covers in your uniform base.
\begin{prp}\label{lma5.1.16}
Let $\widetilde{\mathcal{B}}$ be a uniform base on a set $X$.  Then, (i) $\Int (\mathcal{B})\coloneqq \{ \Int (B):B\in \mathcal{B}\}$ is (still) a cover of $X$ and (ii) $\Int (\widetilde{\mathcal{B}})\coloneqq \left\{ \Int (\mathcal{B}):\mathcal{B}\in \widetilde{\mathcal{B}}\right\}$ is (still) a uniform base on $X$ that generates the same uniform base as $\widetilde{\mathcal{B}}$.
\begin{proof}
Let $\mathcal{B}\in \widetilde{\mathcal{B}}$ and let $x\in X$.  Let $\mathcal{C}$ be a star-refinement of $\mathcal{B}$.  Let $C\in \mathcal{C}$ contain $x$ and let $B\in \mathcal{B}$ be such that $\Star _{\mathcal{C}}(C)\subseteq B$.  Then,
\begin{equation}
x\in \Star _{\mathcal{C}}(x)\subseteq \Star _{\mathcal{C}}(C)\subseteq B,
\end{equation}
and so $x\in \Int (B)$, and so indeed $\Int (\mathcal{B})$ is a cover of $X$.

Let $\mathcal{B},\mathcal{C}\in \widetilde{\mathcal{B}}$ and let $\mathcal{D}$ be a common star-refinement of $\mathcal{B}$ and $\mathcal{C}$.  We show that $\Int (\mathcal{D})$ is a common star-refinement of $\Int (\mathcal{B})$ and $\Int (\mathcal{C})$.  Because of $\mathcal{B}\leftrightarrow \mathcal{C}$ symmetry, it suffices to show that it is a star-refinement of $\Int (\mathcal{C})$.  So, let $\Int (D)\in \Int (\mathcal{D})$.  Then, there is some $C\in \mathcal{C}$ such that $\Star _{\mathcal{D}}(D)\subseteq C$, and so because union of the interiors is contained in the interior of the union (\cref{exr3.4.53}\ref{enm3.4.53.ii}), we have
\begin{equation}
\Star _{\Int (\mathcal{D})}(\Int (D))\subseteq \Star _{\Int (\mathcal{D})}(D)\subseteq \Int (C),
\end{equation}
and so indeed $\Int (\mathcal{D})$ star-refines $\Int (\mathcal{D})$.

It remains to show that $\widetilde{\mathcal{B}}$ and $\Int (\widetilde{\mathcal{B}})$ induce the same uniform structure.  To show this, we apply \cref{prp1.6}.  Let $\mathcal{B}\in \widetilde{\mathcal{B}}$ and let $\mathcal{C}$ be a star-refinement of $\mathcal{B}$.  Then, $\Int (\mathcal{C})$ certainly star-refines $\mathcal{B}$.  For the other direction, let $\mathcal{D}$ be a star-refinement of $\mathcal{C}$.  We show that $\mathcal{D}$ star-refinement $\Int (\mathcal{B})$.  Let $D\in \mathcal{D}$ and let $C\in \mathcal{C}$ be such that $\Star _{\mathcal{D}}(D)\subseteq C$.  Let $B\in \mathcal{B}$ be such that $\Star _{\mathcal{C}}(C)\subseteq B$.  Let $x\in \Star _{\mathcal{D}}(D)$.  Then,
\begin{equation}
x\in C\subseteq \Star _{\mathcal{C}}(x)\subseteq \Star _{\mathcal{C}}(C)\subseteq B,
\end{equation}
and so indeed $x\in \Int (B)$.
\end{proof}
\end{prp}

As the idea of a uniformity is inherently `global' in nature, there really isn't a way to define a uniformity that is analogous to the method of defining a topology by specifying neighborhood bases.\footnote{Of course one cannot hope to make a statement like this precise (What does it mean for a method of defining a uniformity to be ``analogous to'' a method of defining a topology?), but hopefully the intuition is clear.  All elements in a uniform cover, no matter where they are in the space, are supposed to be thought of as the same size.  How could one hope to encode the idea of two sets living `far away' are of the same size by the specification of local information alone?}  There is, of course, a way to specifying a uniformity by merely declaring a given collection of covers to be uniform covers.  This is of course analogous to the specifying of a generating collection of a topology.
\begin{prp}[Generating collection of covers]
Let $X$ be a set and let $\widetilde{\mathcal{S}}$ be a nonempty collection of covers of $X$.  Then, there exists a unique uniformity $\widetilde{\mathcal{U}}$ on $X$, the uniformity \emph{generated by}\index{Generate (a uniformity)} by $\widetilde{\mathcal{S}}$, such that
\begin{enumerate}
\item $\widetilde{\mathcal{S}}\subseteq \widetilde{\mathcal{U}}$; and
\item if $\widetilde{\mathcal{U}}'$ is any other uniformity containing $\widetilde{\mathcal{S}}$, it follows that $\widetilde{\mathcal{U}}\subseteq \widetilde{\mathcal{U}}'$.
\end{enumerate}
Furthermore, the collection of all finite meets form a uniform base for this uniformity.  $\widetilde{\mathcal{S}}$ is the \emph{generating collection}\index{Generating collection (of covers)}.
\begin{proof}
We leave the proof as an exercise.
\begin{exr}
Prove this result, using the proof of \cref{GeneratingCollection} (the analogous result for topological spaces) as guidance.
\end{exr}
\end{proof}
\end{prp}
With topological spaces, we could define a topology by specifying the closure or interior, or defining a notion of convergence.  To the best of my knowledge, there is no analogous method for any of these definitions of defining uniformities.  As for the initial and final topologies, however, there are most certainly analogous constructions.
\begin{prp}[Initial uniformity]\label{InitialUniformity}
Let $X$ be a set, let $\mathcal{Y}$ be an indexed collection of uniform spaces, and for each $Y\in \mathcal{Y}$ let $f_Y:X\rightarrow Y$ be a function.  Then, there exists a unique uniformity $\widetilde{\mathcal{U}}$ on $X$, the \emph{initial uniformity} with respect to $\{ f_Y:Y\in \mathcal{Y}\}$, such that
\begin{enumerate}
\item $f_Y:X\rightarrow Y$ is uniformly-continuous with respect to $\widetilde{\mathcal{U}}$ for all $Y\in \mathcal{Y}$; and
\item if $\widetilde{\mathcal{U}}'$ is another uniformity for which each $f_Y$ is uniformly-continuous, then $\widetilde{\mathcal{U}}\subseteq \widetilde{\mathcal{U}}'$.  Furthermore, if $\widetilde{\mathcal{S}}_Y$ generates the uniformity on $Y$, then the collection
\begin{equation}
\{ f_Y^{-1}(\mathcal{U}):Y\in \mathcal{Y},\ \mathcal{U}\in \widetilde{\mathcal{S}}_Y\}
\end{equation}
generates $\widetilde{\mathcal{U}}$.
\end{enumerate}
\begin{rmk}
In other words, the initial uniformity is the smallest uniformity for which each $f_Y$ is uniformly-continuous.
\end{rmk}
\begin{rmk}
But what about the largest such uniformity?  Well, the largest such uniformity is always going to be the discrete uniformity, which is not very uninteresting.  This is how you remember whether the initial uniformity is the smallest or largest---it can't be the largest because the discrete uniformity always works.
\end{rmk}
\begin{rmk}
In particular,
\begin{equation}
\{ f_Y^{-1}(\mathcal{U}):Y\in \mathcal{Y},\ \mathcal{U}\text{ a uniform cover of }Y\text{.}\}
\end{equation}
generates this with the initial uniformity.
\end{rmk}
\begin{proof}
We leave the proof as an exercise.
\begin{exr}
Prove this result, using the proof of \cref{InitialTopology} (the defining result of the initial topology) as guidance.
\end{exr}
\end{proof}
\end{prp}
Of course, we have a result that is perfectly analogous to \cref{prp3.4.6} (a function is continuous iff its composition with each $f_Y$ is continuous).
\begin{prp}\label{prp4.2.54}
Let $X$ have the initial uniformity with respect to the collection $\{ f_Y:Y\in \mathcal{Y}\}$, let $Z$ be a uniform space, and let $f:Z\rightarrow X$ be a function.  Then, $f$ is uniformly-continuous iff $f_Y\circ f$ is uniformly-continuous for all $Y\in \mathcal{Y}$.  Furthermore, the initial uniformity is the unique uniformity with this property.
\begin{proof}
We leave the proof as an exercise.
\begin{exr}
Prove this result, using the proof of \cref{prp3.4.6} (the analogous result for the initial topology) as guidance.
\end{exr}
\end{proof}
\end{prp}
And just as we had with topological spaces, there is a `dual' version of the initial uniformity.
\begin{prp}[Final uniformity]\label{FinalUniformity}
Let $X$ be a set, let $\mathcal{Y}$ be an indexed collection of uniform spaces, and for each $Y\in \mathcal{Y}$ let $f_Y:Y\rightarrow X$ be a function.  Then, there exists a unique uniformity $\widetilde{\mathcal{U}}$ on $X$, the \emph{final uniformity}\index{Final uniformity} with respect to $\{ f_Y:Y\in \mathcal{Y}\}$, such that
\begin{enumerate}
\item $f_Y:Y\rightarrow X$ is uniformly-continuous with respect to $\widetilde{\mathcal{U}}$; and
\item if $\widetilde{\mathcal{U}}'$ is another uniformity for which each $f_Y$ is uniformly-continuous, then $\widetilde{\mathcal{U}}\supseteq \widetilde{\mathcal{U}}'$.
\end{enumerate}
Furthermore,
\begin{equation}
\widetilde{\mathcal{U}}=\{ \mathcal{U}\text{ a cover of }X:f_Y^{-1}(\mathcal{U})\text{ is a uniform cover for all }Y\in \mathcal{Y}\text{.}\} .
\end{equation}
\begin{rmk}
In other words, the final uniformity is the largest uniformity for which each $f_Y$ is uniformly-continuous.
\end{rmk}
\begin{rmk}
But what about the smallest such uniformity?  Well, the smallest such uniformity is always going to be the indiscrete uniformity, which is not very interest.  This is how you remember whether the final uniformity is the smallest or largest---it can't be the smallest because the indiscrete uniformity always works.
\end{rmk}
\begin{proof}
We leave the proof as an exercise.
\begin{exr}
Prove this result, using the proof of \cref{FinalTopology} (the defining result of the final topology) as guidance.
\end{exr}
\end{proof}
\end{prp}
And the result `dual' to \cref{prp4.2.54}:
\begin{prp}
Let $X$ have the final uniformity with respect to the collection $\{ f_Y:Y\in \mathcal{Y}\}$, let $Z$ be a uniform space, and let $f:X\rightarrow Z$.  Then, $f$ is uniformly-continuous iff $f_Y\circ f$ is uniformly-continuous for all $Y\in \mathcal{Y}$.  Furthermore, the final uniformity is the unique uniformity with this property.
\begin{proof}
We leave the proof as an exercise.
\begin{exr}
Prove this result, using the proof of \cref{prp3.4.34x} (the analogous result for the final topology) as guidance 
\end{exr}
\end{proof}
\end{prp}
Of course, just as with topological spaces, a key application of the initial and final uniformities is that they provide canonical uniformities on subsets, quotients, products, and disjoint-unions.  The definitions and results are completely analogous to the case of topological spaces, and so we omit stating them explicitly.

After having discussed the real numbers themselves as a uniform space, we show below (\cref{exm4.2.85}) that functions even as nice as polynomials are not uniformly continuous.  On the other hand, when restricted to \emph{quasicompact} sets, all continuous functions are uniformly-continuous.
\begin{prp}\label{prp4.2.73}
Let $f:X\rightarrow Y$ be a continuous function between uniform spaces and let $K\subseteq X$ be quasicompact.  Then, $\restr{f}{K}:K\rightarrow Y$ is uniformly-continuous.
\begin{rmk}
Ideally we would have presented this result shortly after giving the definition of uniformly-continuous functions, however, we do technically need the notion of the subspace uniformity to state this result.
\end{rmk}
\begin{proof}
Let $\mathcal{V}$ be a uniform cover of $Y$.  We would like to show that $\restr{f}{K}^{-1}(\mathcal{V})=f^{-1}(\mathcal{V})\wedge \{ K\}$ is a uniform-cover of $K$.  To do this, by upward-closedness, it suffices to find a uniform-cover of $K$ which star-refines $f^{-1}(\mathcal{V})\wedge \{ K\}$.

$f^{-1}(\mathcal{V})$, while not necessarily a uniform cover of $X$, will certainly be an open cover, and in particular will be an open cover of $K$.  So, for $x\in K$, let $V_x\in \mathcal{V}$ be such that $x\in f^{-1}(V_x)$.  Then, choose a uniform cover $\mathcal{U}_x$ of $X$ such that
\begin{equation}
\Star _{\mathcal{U}_x\wedge \{ K\}}(x)\subseteq f^{-1}(V_x)\cap K.
\end{equation}
As
\begin{equation}
\left\{ \Star _{\mathcal{U}_x\wedge \{ K\}}(x):x\in K\right\}
\end{equation}
is an open cover of $K$, there is a finite subcover.  So, let $x_1,\ldots ,x_m\in K$ be such that
\begin{equation}
\left\{ \Star _{\mathcal{U}_{x_k}\wedge \{ K\}}(x_k):1\leq k\leq m\right\}
\end{equation}
is an open cover of $K$.  Let $\mathcal{U}$ be a common star-refinement of each $\mathcal{U}_k$\footnote{It is here that the finiteness given to us by quasicompactness is key.}.  Then,
\begin{equation}
\Star _{\mathcal{U}_0\wedge \{ K\}}(x)\subseteq f^{-1}(V_x)\cap K
\end{equation}
for all $x\in K$.

Let $\mathcal{U}$ be in turn a star-refinement of $\mathcal{U}_0$.  We show that $\mathcal{U}\wedge \{ K\}$ is a star-refinement of $f^{-1}(\mathcal{V})\wedge \{ K\}$.  So, let $U\in \mathcal{U}$.  Let $U_0\in \mathcal{U}_0$ be such that $\Star _{\mathcal{U}}(U)\subseteq U_0$.  Let $x\in U$.  Then,
\begin{equation}
\Star _{\mathcal{U}\wedge \{ K\}}(U\cap K)\subseteq U_0\cap K\subseteq \Star _{\mathcal{U}_0\wedge \{ K\}}(x)\subseteq f^{-1}(V_x)\cap K.
\end{equation}
\end{proof}
\end{prp}

\horizontalrule

\begin{exm}[The real numbers]
The real numbers have a canonical uniformity (and in fact, we will see below that this is just a special base of a more general construction):  let $\varepsilon >0$ and define
\begin{equation}
\mathcal{U}_{\varepsilon}\coloneqq \left\{ B_{\varepsilon}(x):x\in \R \right\}
\end{equation}
\index[notation]{$\mathcal{U}_{\varepsilon}$} and
\begin{equation}
\widetilde{\mathcal{U}}\coloneqq \left\{ \mathcal{U}_{\varepsilon}:\varepsilon >0\right\} .
\end{equation}
\begin{exr}
Show that $\widetilde{\mathcal{U}}$ is a uniform base on $\R$.
\end{exr}
\begin{exr}
Show that $f:\R \rightarrow \R$ is uniformly-continuous iff for every $\varepsilon >0$ there is some $\delta >0$ such that $f(B_{\delta}(x))\subseteq B_{\varepsilon}(f(x))$ for every $x\in \R$.
\begin{rmk}
Compare this with the condition for $f:\R \rightarrow \R$ being \emph{continuous at $a\in \R$} given in \cref{exr3.4.5}\ref{enm3.4.5.iii}.  We will spell-it-out here for convenience:
\begin{textequation}
$f:\R \rightarrow \R$ is continuous iff for every $x\in \R$ and for every $\varepsilon >0$ there is some $\delta >0$ such that $f(B_{\delta}(x))\subseteq B_{\varepsilon}(f(x))$.
\end{textequation}
The key difference between continuity and uniform-continuity is \emph{the location in which the quantification ``for every $x\in \R$'' appears}.  In the former (just continuous case), your choice of $\delta$ is \emph{allowed to depend on $x$}, whereas to be uniformly-continuous, \emph{a single $\delta$ has to `work' for every $x\in \R$}.
\end{rmk}
\begin{rmk}
The result you just proved characterizing uniform-continuity in $\R$ is often taken as the definition of uniform-continuity.  Had we studied uniform-continuity in the context of just the real numbers first (as opposed to in the context of uniform spaces), we would have done the same.  My personal feeling, however, is that uniform continuity is not that incredibly important, at least not to the point where it is worth going out of our way to discuss it just in the context of $\R$.  The real reason we discuss uniform spaces is for the purpose of discussing cauchyness and completeness, not uniform continuity per se (and also of course because a huge collection of examples of topological spaces are canonically uniform spaces).
\end{rmk}
\end{exr}
\end{exm}
\begin{exm}[A uniformly-continuous function]
By \cref{prp4.2.73}, any continuous function restricted to a quasicompact set will be uniformly-continuous, so, for example the function $x\mapsto x^2$ is uniformly-continuous on $[0,1]$.  However, be careful:  it is not uniformly-continuous on all of $\R$.
\end{exm}
\begin{exm}[A continuous function that is not uniformly-continuous]\label{exm4.2.85}
Define $f:\R \rightarrow \R$ by $f(x)\coloneqq x^2$.  Of course $f$ is continuous (because it is the product of continuous functions---see \cref{exr3.4.12}).

On the other hand, we show that $f$ does not satisfy the condition given in the previous exercise.  Take $\varepsilon \coloneqq 1$.  Then, if $f$ were uniformly-continuous, there should be some $\delta >0$ such that
\begin{equation}
\left\{ x^2:\abs{x-x_0}<\delta \right\} \eqqcolon f(B_\delta (x_0))\subseteq B_{\varepsilon}(f(x_0))\coloneqq\left\{ x\in \R :\abs{x-x_0^2}<1\right\} 
\end{equation}
for all $x_0\in \R$.  However, $(x_0+\frac{1}{2}\delta )^2$ is an element of the left-hand side, but
\begin{equation}
\left( x_0+\tfrac{1}{2}\delta \right) ^2-x_0^2=\delta x_0+\tfrac{1}{4}\delta ^2
\end{equation}
is not less than $1$ in general (for example, for $x_0=\frac{1}{\delta}(1-\tfrac{1}{4}\delta ^2)$.

On the other hand, by \cref{prp4.2.73}, $f$ restricted to any closed interval is uniformly-continuous.
\end{exm}

\section{Semimetric spaces and topological groups}

As was previously mentioned, one big motivation for studying uniform spaces is that a huge collection of very important examples of topological spaces admit a canonical uniformity.  Two such families of spaces that we will study are \emph{semimetric spaces} and \emph{topological groups}.

\subsection{Semimetric spaces}

Before we talk about any sort of uniformity, we had better first say what we mean by \emph{semimetric space}.
\begin{dfn}[Semimetric and metric]\label{Semimetric}
Let $X$ be a set.  Then, a \emph{semimetric}\index{Semimetric} on $X$ is a function $\metric:X\times X\rightarrow \R _0^+$ such that
\begin{enumerate}
\item (Symmetry) $\metric[x][y]=\metric[y][x]$; and
\item (Triangle Inequality) $\metric[x][z]\leq \metric[x][y]+\metric[y][z]$.
\end{enumerate}
$\metric$ is a \emph{metric}\index{Metric} if furthermore (Definiteness) $\metric[x][y]=0$ implies $x=y$.
\begin{rmk}
Semimetrics are also sometimes called \emph{pseudometric}\index{Pseudometrics}.  However, the term seminorm (something we haven't discussed yet---see \cref{Seminorm}) is actually much more common than either of these terms, and as metrics are to norms (also something we haven't discussed yet---see \cref{Seminorm} again) as semimetrics/pseudometrics are to seminorms, we feel as if the terminology ``semimetric'' is more appropriate.
\end{rmk}
\begin{rmk}
It is much more common to denote (semi)metrics by ``$d(-,-)$'', however, this conflicts with our conventions of reserving the letter ``$d$'' for dimension and differentials.
\end{rmk}
\end{dfn}
\begin{exm}
Let $X\coloneqq \R$ and define $\metric[x][y]\coloneqq \abs{x-y}$.  Then, $\metric$ is in fact a metric.
\begin{rmk}
Of course, this is where the notation $\metric$ in general comes from.
\end{rmk}
\end{exm}
\begin{exm}[A semimetric that is not a metric]\label{exm4.4.3}
Let $X$ be a topological space and let $K\subseteq X$ be quasicompact.  For $f,g\in \Mor _{\Top}(X,\R )$, we define
\begin{equation}\label{4.4.4}
\metric[f][g]_K\coloneqq \sup _{x\in K}\{ \abs{f(x)-g(x)}\} .\footnote{We require that $K$ be quasicompact so that $f-g$ is bounded on $K$ (by the \nameref{ExtremeValueTheorem} (\cref{ExtremeValueTheorem}))}.
\end{equation}
In general, this will not be a metric.  For example, take $X\coloneqq \R$ and $K\coloneqq [0,1]$.  Then,  $\abs{f,0}_K=0$ iff $\restr{f}{[0,1]}=0$, but of course, there are many nonzero real-valued continuous functions on $\R$ that vanish on $[0,1]$ (by Urysohn's Lemma (\cref{UrysohnsLemma}), for example, if you want to use a sledgehammer (or maybe just a hammer?) to swat a fly).
\end{exm}
\begin{dfn}[Semimetric space]\label{Semimetric space}
A \emph{semimetric space}\index{Semimetric space} is a set $X$ equipped with a collection $\mathcal{D}$ of semimetrics such that if $\metric[x][y]=0$ for all $\metric \in \mathcal{D}$, then $x=y$.
\begin{rmk}
For some reason, it seems that semimetric spaces are also referred to as \emph{gauge spaces}\index{Gauge spaces}.  Off the top of my head, I can think of at least two other distinct ways in which the term ``gauge'' is used in mathematics, and so I would recommend not using this terminology
\end{rmk}
\end{dfn}
\begin{dfn}[Metric space]\label{MetricSpace}
A \emph{metric space}\index{Metric space} $(X,\abs{-,-})$ is a semimetric space $(X,\mathcal{D})$ in which $\mathcal{D}$ is a singleton, $\mathcal{D}=\{ \metric \}$.
\begin{rmk}
Of course, the definiteness condition in the definition of a semimetric space forces $\abs{-,-}$ to be a metric.
\end{rmk}
\begin{rmk}
The reason we take $\mathcal{D}$ to be a singleton instead of just an arbitrary collection of \emph{metrics} is to agree with standard terminology (metric spaces are almost always taken to be sets equipped with a (\emph{single}) metric).
\end{rmk}
\end{dfn}
\begin{exm}[A semimetric space that is not a metric space]
Let $X$ be a topological space, and for $K\subseteq X$ quasicompact nonempty, let $\metric _K$ be the semimetric on $\Mor _{\Top}(X,\R )$ in \eqref{4.4.4}, that is
\begin{equation}
\metric[f][g]\coloneqq \sup _{x\in K}\{ \abs{f(x)-g(x)}\} .
\end{equation}
We already know from \cref{exm4.4.3} that each $\metric _K$ is a semimetric on $\Mor _{\Top}(X,\R )$.  What we need to show that $\metric[f][g]_K=0$ for all $K\subseteq X$ quasicompact implies $f=g$.  This however follows from the fact that $K=\{ x\}$ for $x\in X$ is quasicompact (do you see why?).

As $\mathcal{D}$ clearly contains more than one element (at least so long as $X$ contains more than one point), you might think that this shows that this cannot be a metric space.  However, the real question is \emph{is it uniformly-homeomorphic} to a metric space?\footnote{Of course, this doesn't quite make sense yet as we have not put a uniformity on semimetric spaces.}  While not a proof, it is clear that it should not be as, unless $X$ is quasicompact itself, no element of $\mathcal{D}$ will actually be a metric.
\end{exm}

It's worth noting that, in a metric space, for every closed subset, the distance (as defined below in \eqref{4.8.50} from a point to the closed subset is a continuous function.
\begin{prp}\label{prp4.8.49}
Let $\coord{X,\metric}$ be a metric space and let $C\subseteq X$.  Then, the function $\dist _C:X\rightarrow \R$ defined by
\begin{equation}\label{4.8.50}
\dist _C(x)\coloneqq \inf _{c\in C}\{ \metric[x][c]\} 
\end{equation}\index[notation]{$\dist _C(x)$}
is uniformly-continuous and furthermore $\dist _C^{-1}(0)=C$.
\begin{proof}
Let $\varepsilon >0$.  Let $x_1,x_2\in X$ lie in some $\varepsilon$ ball.  Choose some $c\in C$ such that $\metric[x_1][c]-\dist_C(x_1)<\varepsilon$..  Then,
\begin{equation}
\dist _C(x_2)\leq \metric[x_2][c]\leq \metric[x_2][x_1]+\metric[x_1][c]<2\varepsilon +\dist _C(x_1),
\end{equation}
and so
\begin{equation}
\dist _C(x_2)-\dist _C(x_1)<2\varepsilon .
\end{equation}
By $1\leftrightarrow 2$ symmetry, we also have that
\begin{equation}
\dist _C(x_1)-\dist _C(x_2)<2\varepsilon ,
\end{equation}
and hence
\begin{equation}
\abs{\dist _C(x_1)-\dist _C(x_2)}<2\varepsilon .
\end{equation}
This shows that $\dist _C$ is uniformly-continuous.

Of course $C\subseteq \dist _C^{-1}(0)$.  On the other hand, if $x\in \dist _C^{-1}(0)$, then $x$ is an accumulation point of $C$, and hence contained in $C$.  Thus, $C=\dist _C^{-1}(0)$.
\end{proof}
\end{prp}
\begin{prp}\label{prp5.4.13}
Metric spaces are uniformly-perfectly-$T_4$.
\begin{proof}
Let $X$ be a metric space and let $C_1,C_2\subseteq X$ be closed and disjoint.
\begin{exr}
Show that there is some $f_1:X\rightarrow [0,\frac{1}{2}]$ uniformly-continuous, equal to $0$ precisely on $C_1$, and equal to $\frac{1}{2}$ on $C_2$.  Similarly, show that there is some $f_2:X\rightarrow [0,\frac{1}{2}]$ uniformly-continuous, equal to $\frac{1}{2}$ precisely on $C_2$, and equal to $0$ on $C_1$.
\end{exr}
Define $f\coloneqq f_1+f_2$.  Then, this is certainly $0$ on $C_1$ and $1$ on $C_2$.  Conversely, suppose that $f(x)=0$.  Then, in particular, $f_1(x)=0=f_2(x)=0$, and so in particular $x\in C_1$.  On the other hand, suppose that $f(x)=1$.  Then, we must have in particular that $f_2(x)=\frac{1}{2}$, which implies that $x\in C_2$.  Thus, $f^{-1}(0)=C_1$ and $f^{-1}(1)=C_2$, and hence $X$ is perfectly-$T_4$.
\end{proof}
\end{prp}

Now that we've gotten that out of the way, we are ready to equip semimetric spaces with a topology and uniformity.
\begin{dfn}[Uniformity on a semimetric space]\label{dfnB.10}
\begin{savenotes}
Let $(X,\mathcal{D})$ be a semimetric space, and equip $X$ with the uniformity generated by the uniform base defined by
\begin{equation}\label{B.11}
\widetilde{\mathcal{B}}_{\mathcal{D}}\coloneqq \left\{ U_{\varepsilon _1,\ldots ,\varepsilon _m}^{\metric _1,\ldots ,\metric _m}:m\in \Z ^+;\ \metric _1,\ldots ,\metric _m\in \mathcal{D};\ \varepsilon _1,\ldots ,\varepsilon _m>0\right\} ,\index[notation]{$\widetilde{\mathcal{U}}_{\mathcal{D}}$}
\end{equation}
where
\begin{equation}\label{1.12}
\mathcal{B}_{\varepsilon _1,\ldots ,\varepsilon _m}^{\metric _1,\ldots ,\metric _m}\coloneqq \left\{ B_{\varepsilon _1,\ldots ,\varepsilon _m}^{\metric _1,\ldots ,\metric _m}(x):x\in X\right\}
\end{equation}
and
\begin{equation}\label{1.13}
B_{\varepsilon _1,\ldots ,\varepsilon _m}^{\metric _1,\ldots ,\metric _m}\coloneqq \left\{ y\in X:\metric[y][x]_1<\varepsilon _1,\ldots ,\metric[y][x]_m<\varepsilon _m\right\} .
\end{equation}
\begin{exr}
Show that $\widetilde{\mathcal{B}}_{\mathcal{D}}$ is indeed a uniform base.
\end{exr}
\end{savenotes}
\end{dfn}
\begin{exm}[Discrete metric]
Not only does the discrete topology come from a uniformity, but so to does the discrete uniformity in turn come from a metric.

Let $X$ be a set and for $x,y\in X$, define
\begin{equation}
\metric[x][y]\coloneqq \begin{cases}0 & \text{if }x=y \\ 1 & \text{otherwise}\end{cases}.
\end{equation}
\begin{exr}
Show that $\metric$ is indeed a metric on $X$.
\end{exr}
\begin{exr}
Show that the uniformity defined by $\metric$ is the discrete uniformity.
\end{exr}
\end{exm}
\begin{exr}
Why does the indiscrete uniformity (on a set with at least two elements) not come from a metric?
\end{exr}
The following immediately follows from our result \cref{prpB.3.4} characterizing uniform-continuity in terms of uniform bases.
\begin{prp}\label{prp4.8.54}
Let $\coord{X,\widetilde{\mathcal{U}}}$ be a uniform space, let $\coord{Y,\metric}$ be a metric space, and let $f:X\rightarrow Y$.  Then, $f$ is uniformly-continuous iff for every $\varepsilon >0$ there is some $\mathcal{U}\in \widetilde{\mathcal{U}}$ such that for every $U\in \mathcal{U}$, whenever $x_1,x_2\in U$, it follows that $\metric[f(x_1)][f(x_2)]<\varepsilon$.
\end{prp}

But before we head onto topological groups, what about the morphisms in the category of semimetric spaces, you ask?  Good question.
\begin{dfn}[Bounded map (of semimetric spaces)]\label{BoundedMap}
Let $f:\coord{X,\mathcal{D}}\rightarrow \coord{Y,\mathcal{E}}$ be a function between semimetric spaces.  Then, $f$ is \emph{bounded}\index{Bounded map (of semimetric spaces)} iff for every $\metric _0\in \mathcal{E}$, there are \emph{finitely-many} $\metric _1,\ldots ,\metric _m\in \mathcal{D}$ and constants $K_1,\ldots ,K_m\geq 0$ such that
\begin{equation}
\metric[f(x_1)][f(x_2)]_0\leq K_1\metric[x_1][x_2]_1+\cdots +K_m\metric[x_1][x_2]_m
\end{equation}
for all $x_1,x_2\in X$.
\begin{rmk}
If $X$ and $Y$ are metric spaces with metric $\metric _X$ and $\metric _Y$ respectively, this condition reads just
\begin{equation}
\metric[f(x_1)][f(x_2)]_Y\leq K\metric[x_1][x_2]_X.
\end{equation}
In this case, $f$ is called \emph{lipschitz-continuous}\index{Lipschitz-continuous}\footnote{Dear lord.  His name is a juxtaposition of the word ``lip'' and the word ``shits''.  You have my sympathies, sir\textellipsis .}
\end{rmk}
\begin{rmk}
Of all the categories we've come across, that the bounded maps are the `right' notion of morphism between semimetric spaces is probably the least obvious.\footnote{Of course, we can declare any collection of morphisms we like.  It's just that, taking the morphisms to be \emph{all} functions when the objects are groups (for example) is not particularly useful---the category won't be able to tell that the groups are groups!}  The motivation for the definition is that, for seminormed vector spaces, this definition is equivalent to continuity---see \cref{exr4.3.57}.  Perhaps a simpler explanation is that it guarantees uniform-continuity.
\end{rmk}
\end{dfn}
\begin{exr}
Show that bounded maps between semimetric spaces are uniformly-continuous.
\end{exr}
\begin{exm}[The category of semimetric spaces]
The category of semimetric spaces is the category $\Semi \Met$ whose collection of objects $\Semi \Met _0$ is the collection of all semimetric spaces, for every semimetric space $X$ and semimetric space $Y$ the collection of all morphisms from $X$ to $Y$, $\Mor _{\Semi \Met}(X,Y)$, is precisely the set of all bounded maps from $X$ to $Y$, composition is given by ordinary function composition, and the identities of the category are the identity functions.
\begin{rmk}
Every semimetric space is canonically a uniform space---see \cref{dfnB.10}.  By the previous exercise, every bounded map is likewise uniformly-continuous.  Therefore, in fact, the category $\Semi \Met$ \emph{embeds} in $\Uni$.\footnote{The thing to take note of is that \emph{both} the objects \emph{and} the morphisms have to be contained in $\Uni$.}
\end{rmk}
\end{exm}
\begin{exm}[A lipschitz-continuous function]
The function $x\mapsto x$ from $\R$ to $\R$.
\end{exm}
\begin{exm}[A uniformly-continuous function that is not lipschitz-continuous]\label{exm4.3.34}
Define $f:[0,1]\rightarrow \R$ by $f(x)\coloneqq \sqrt{x}$.  This function is continuous on $[0,1]$, and hence uniformly-continuous because $[0,1]$ is quasicompact by the \nameref{HeineBorelTheorem} (and by \cref{prp4.2.73}).  On the other hand, to show that it is \emph{not} lipschitz-continuous, we need to show that
\begin{equation}
\frac{\sqrt{x}-\sqrt{y}}{x-y}
\end{equation}
is \emph{not} bounded for $x,y\in [0,1]$ distinct.  However, simply take $x=0$.  Then, we need to show that
\begin{equation}
\frac{\sqrt{y}}{y}=\frac{1}{\sqrt{y}}
\end{equation}
is not bounded on $[0,1]$.  Equivalently, you can show that $\lim _{y\to 0^+}\sqrt{y}=0$.\footnote{We have technically not defined one-sided limits.  If this bothers you, it's not a bad exercise to try to come-up with the definition yourself.}
\end{exm}

\subsection{Topological groups}

Before we talk about any sort of uniformity, we had better first define what we mean by a topological group.
\begin{dfn}[Topological group]\label{TopologicalGroup}
A \emph{topological group} is a group $\coord{G,\cdot ,1,\blank ^{-1}}$ (\cref{Group}) equipped with a topology such that
\begin{enumerate}
\item $\cdot :G\times G\rightarrow G$ is continuous; and
\item $\blank ^{-1}:G\rightarrow G$ is continuous.
\end{enumerate}
\begin{rmk}
That is to say, a topological group is a thing that is both a group and a topological space, subject to a couple of `compatibility' axioms that demand that the two structures `work together'.  This is very analogous to our definition of preordered rgs (\cref{dfn1.1.38})---a preordered rg is both a preordered set and a rg subject to a couple of ``compatibility conditions''.  This idea is not uncommon throughout all of mathematics, and, as you might have expected by this point, can be unified with the use of categories.
\end{rmk}
\begin{rmk}
Recall that in a remark of the definition of a group (\cref{Group}), we made a slight deal about ``having inverses'' not being stated as an \emph{extra property} but rather as \emph{extra structure}.  This is one reason why.  When we go to define a topological group, the the operation of taking inverses should be thought of as just that---an operation, on the same footing as the product.  If we think of the operation of taking inverses as on the same footing as the product, then we almost have to also assume that the inverse operation is likewise continuous, whereas if it were thought of just as an existence property, it would not make as much sense to do this.
\end{rmk}
\end{dfn}
\begin{exm}[The category of topological groups]
The category of topological groups is the category $\Top \Grp$\index[notation]{$\Top \Grp$} whose collections of objects $\Top \Grp _0$ is the collection of all topological groups, for every topological group $G$ and topological group $H$ the collection of morphisms from $G$ to $Y$, $\Mor _{\Top \Grp}(G,H)$, is precisely the set of all continuous group homomorphisms from $G$ to $H$, composition is given by ordinary function composition, and the identities of the category are the identity functions.
\end{exm}
\begin{exm}[A topological group that is not $T_0$]
\begin{savenotes}
Define $G\coloneqq \R /\Q$.\footnote{This is the quotient rng construction---see \cref{IdealsAndQuotientGroups}.}  Let $\q :\R \rightarrow G$ be the quotient map (i.e.~the map that sends an element to its equivalence class) and equip $\R /\Q$ with the quotient topology.
\begin{exr}
Show that $+:G\times G\rightarrow G$ and $\blank ^{-1}:G\rightarrow G$ are continuous.
\end{exr}
We now check that the quotient topology on $G$ is not $T_0$.  So, let $x\in \R$ be irrationals.  We wish to show every open neighborhood of $x+\Q$ contains $0+\Q$ and conversely.  So, let $U\subseteq x+\Q$ be open.  Then, by definition, $\q ^{-1}(U)$ is an open neighborhood of $x$, and so by density, must contain some rational number $r\in \q ^{-1}(U)$.  But then, $0+\Q =r+\Q \in \q \left( \q ^{-1}(U)\right) \subseteq U$.  On the other hand, if $U$ is an open neighborhood of $0+\Q$, $\q ^{-1}(U)$ is an open neighborhood of $0$, and so by density again, must contain some $\varepsilon >0$ so that $x-\varepsilon$ is rational.  But if $x-\varepsilon \in \Q$, then $x+\Q =\varepsilon +\Q \in \q \left( \q ^{-1}(U)\right) \subseteq U$.
\begin{rmk}
We show in the next section that every $T_0$ uniform space is in fact completely-$T_3$.  Thus, this serves as an example of a uniform space which is not completely-$T_3$.
\end{rmk}
\end{savenotes}
\end{exm}
A large number of examples of topological groups arise from totally-ordered rngs (or more generally, totally-ordered commutative groups).
\begin{exr}\label{exr4.8.58}
Let $G$ be a totally-ordered commutative group.  Show that $G$ is a topological group with respect to the order topology.
\end{exr}

Before we put a uniformity on topological groups, it will be useful to know at least one basic fact about them.
\begin{prp}\label{prp4.8.59}
\begin{savenotes}
Let $G$ be a topological group and let $U$ be a neighborhood of the identity.  Then, there exists an open neighborhood $V$ of the identity such that (i) $VV\subseteq U$ and (ii) $V^{-1}=V$.
\begin{rmk}
The notation means what you think it means:  $VV\coloneqq \{ v_1v_2:v_1\in V,\ v_2\in V\}$ (note how this is not the same as $V^2$) and $V^{-1}\coloneqq \{ v^{-1}:v\in V\}$.
\end{rmk}
\begin{proof}
Regarding the group operation $\cdot$ as a function from $G\times G$ to $G$, we know that $+^{-1}(U)$ is an open neighborhood of $\coord{1,1}$ in $G\times G$, and therefore we have that $V\times W\subseteq \cdot ^{-1}(U)$ for some $V,W\subseteq G$ open neighborhoods of the identity (by the definition of the product topology \cref{ProductTopology}).   Replace $V$ with $V\cap W$, another open neighborhood of the identity, so that $V\times V\subseteq \cdot ^{-1}(U)$.  In other words, $VV\subseteq U$.  Now do this exact same construction again and find another open neighborhood of the identity $W$ (replacing our `old' $W$) with $WW\subseteq V$.  Now define
\begin{equation}
W'\coloneqq W\cap [\blank ^{-1}](W^{-1}),
\end{equation}
that is, the intersection of $W$ with the preimage of $W^{-1}$ under the inverse function $\blank ^{-1}:G\rightarrow G$.\footnote{Yes, I am aware that this notation is ridiculously obtuse.}  This will be yet another open neighborhood of the identity, with both $W',(W')^{-1}\subseteq W$.  Finally, define
\begin{equation}
W''\coloneqq (W')(W')^{-1}.
\end{equation}
This certainly satisfies $(W'')^{-1}=W''$, and furthermore,
\begin{equation}
W''W''\coloneqq (W')(W')^{-1}(W')(W')^{-1}\subseteq WWWW\subseteq VV\subseteq U.
\end{equation}
\end{proof}
\end{savenotes}
\end{prp}

\begin{dfn}[Uniformity on a topological group]\label{dfnB.7}
Let $G$ be a topological group, and equip $G$ with the uniformityd generated by the uniform base defined by
\begin{equation}\label{B.8}
\widetilde{\mathcal{B}}_G\coloneqq \left\{ \mathcal{B}_U:U\ni 1\text{ is open.}\right\} ,\index[notation]{$\widetilde{\mathcal{B}}_G$}
\end{equation}
where
\begin{equation}\label{1.9}
\mathcal{B}_U\coloneqq \left\{ gU:g\in G\right\} .
\end{equation}
\index[notation]{$\mathcal{B}_U$}
\begin{exr}
Show $\widetilde{\mathcal{B}}_G$ is indeed a uniform base.
\end{exr}
\begin{rmk}
Note that we could have equally well taken the covers $U_G$ for only $U\in \mathcal{N}$, $\mathcal{N}$ a fixed neighborhood base of the identity (by \cref{prp1.6}).
\end{rmk}
\end{dfn}
We have a potential problem here---$G$ started its life as a topological group, and in particular, as a topological space.  We then equipped it with a uniform structure, from which it obtains the uniform topology.  The question arises:  ``Are these topologies the same, and if not, which one should we use?''.  Fortunately, it turns out that they are the same.
\begin{exr}
Show that the topology on a topological group agrees with the uniform topology induced by the uniform base in \eqref{B.8}.
\end{exr}

We have yet another potential problem here---$\R$ is both a metric space and a topological group (with respect to $+$), so which uniformity should we use?  Fortunately, we needn't worry about this, because the two uniformities are the same.
\begin{exr}
Show that $\widetilde{\mathcal{B}}_{\coord{\R ,\metric}}$ and $\widetilde{\mathcal{B}}_{\coord{\R ,+}}$ define the same uniformity on $\R$.
\end{exr}

You might say that functional analysis is the study of topological vector spaces (the term ``functional'' a result of the fact that many `spaces' of functions are topological vector spaces).  As vector spaces are in particular a group (just forget about the scalars), everything we say regarding the uniformities of topological groups also applies to uniformities of topological vector spaces.  Thus, a knowledge of uniform spaces is very useful when studying functional analysis.  In particular, the following result is used ubiquitously (to the point where it is so common that it is not really even mentioned)
\begin{prp}\label{prpB.10}
\begin{savenotes}
Let $f:G\rightarrow H$ be a group homomorphism between topological groups and let $x_0\in G$.  Then, if $f$ is continuous at $x_0$, then $f$ is uniformly-continuous.
\begin{rmk}
This shows that the category $\Top \Grp$ \emph{embeds}\footnote{We have not defined what precisely this means for categories, but with a little mathematical maturity, you can probably figure it out.  In any case, it's okay if you don't know the precise definition} into the category $\Uni$.  We already knew that the \emph{objects} `embedded' (from the canonical uniformity on topological groups given in \cref{dfnB.7})---this result tells us furthermore that the morphisms `embed' as well.
\end{rmk}
\begin{proof}
\Step{Make hypotheses}
Suppose that $f$ is continuous at $x_0$.

\Step{Show that $f$ is continuous.}
We first show that $f$ is continuous (as opposed to just continuous at $x_0$).  To show that, we show that $f$ is continuous at $x\in G$ for arbitrary $x$.  Let $V$ be a neighborhood of $f(x)\in H$.  Then, $f(x_0)f(x)^{-1}V$ is a neighborhood of $f(x_0)\in H$.  As $f$ is continuous at $x_0$, it follows that $f^{-1}\left( f(x_0)f(x)^{-1}V\right)$ is a neighborhood of $x_0$, and so $xx_0^{-1}f^{-1}\left( f(x_0)f(x)^{-1}V\right)$ is a neighborhood of $x$.\footnote{The juxtaposition here is being used to denote multiplication in the group.  Be careful not to confuse preimages with inverse elements (even though the same symbol is used, the context makes the notation unambiguous).}  However,
\begin{equation}
\begin{split}
f\left( xx_0^{-1}f^{-1}\left( f(x_0)f(x)^{-1}V\right) \right) & =f(x)f(x_0)^{-1}f\left( f^{-1}\left( f(x_0)f(x)^{-1}V\right) \right) \\
& \subseteq f(x)f(x_0)^{-1}f(x_0)f(x)^{-1}V=V,
\end{split}
\end{equation}
so that
\begin{equation}
xx_0^{-1}f^{-1}\left( f(x_0)f(x)^{-1}V\right) \subseteq f^{-1}(V),
\end{equation}
so that $f^{-1}(V)$ is a neighborhood of $x$, so that $f$ is continuous at $x$.

\Step{Show that $f$ is uniformly-continuous.}
To show that $f$ is uniformly-continuous, we apply \cref{prpB.3.4}.  So, let $V\subseteq H$ be an open neighborhood of the identity and consider the cover $\mathcal{U}_V\coloneqq \left\{ hV:h\in H\right\}$.  To show that $f^{-1}(\mathcal{U}_V)$ is a uniform cover, it suffices to find an open neighborhood $U\subseteq G$ of the identity such that $\mathcal{U}_U\llcurly f^{-1}(\mathcal{U}_V)$.  Take $U'$ to be an open neighborhood of the identity such that $U'U'\subseteq f^{-1}(V)$, and then in turn take $U$ to be an open neighborhood of the identity such that (i) $UU\subseteq U'$ and (ii) $U=U^{-1}$ (which we may do by \cref{prp4.8.59}).  We wish to show that
\begin{equation}\label{B.4.7}
\Star _{\mathcal{U}_U}(U)=\bigcup _{x\in G\st xU\cap U\neq \emptyset}xU\subseteq f^{-1}(V).
\end{equation}
It will follow from this (see \eqref{4.8.72}) that $\mathcal{U}_U\llcurly f^{-1}(\mathcal{U}_V)$.  So, let $x\in G$ be such that $xU\cap U\neq \emptyset$.  Then, there are $u_1,u_2\in U$ such that $xu_1=u_2$, so that $x=u_2u_1^{-1}\in UU^{-1}=UU\subseteq U'$.  Thus, $xU\subseteq U'U'\subseteq f^{-1}(V)$.  \eqref{B.4.7} follows from this.  From this, we have
\begin{equation}\label{4.8.72}
\begin{split}
\Star _{\mathcal{U}_U}(x_0U) & =\bigcup _{x\in G\st xU\cap x_0U\neq \emptyset}xU=\bigcup _{x\in G\st xU\cap U\neq \emptyset}x_0xU=x_0\Star _{\mathcal{U}_U}(U) \\
& \subseteq x_0f^{-1}(V)\subseteq f^{-1}(f(x_0)V)\in f^{-1}(\mathcal{U}_V),
\end{split}
\end{equation}
so that $\mathcal{U}_U\llcurly f^{-1}(\mathcal{U}_V)$.
\end{proof}
\end{savenotes}
\end{prp}

\subsection{Topological vector spaces and algebras}

An \emph{incredibly} family of examples of topological groups are the topological vector spaces.
\begin{dfn}[Topological vector space]
A \emph{topological vector space}\index{Topological vector space} is real vector space $\coord{V,+,0,\R ,\cdot}$ such that
\begin{enumerate}
\item $\coord{V,+,0}$ is a topological group; and
\item $\cdot :\R \times V\rightarrow V$ is continuous.
\end{enumerate}
\begin{rmk}
Of course, this definition makes sense if we were to replace $\R$ with any topological field\footnote{What do you think the definition of a topological field should be?}, but for our purposes, restricting ourselves to working over the reals will be sufficient.  The other case of most interest is over $\C$, but we have not even defined the complex numbers.  Most of the time, our topology comes from a collection of seminorms (see \cref{Seminorm} below), in which case it takes a fair amount of effort to work over topological fields whose topology does not come from a norm (at least in the traditional sense of the word).
\end{rmk}
\end{dfn}
One big reason why we are interested in topological vector spaces is because almost all of the examples of semimetrics we counter actually come \emph{seminorms}.
\begin{dfn}[Seminorm and norm]\label{Seminorm}
Let $V$ be a real vector space.  Then, a \emph{seminorm}\index{Seminorm} on $V$ is a function $\norm :V\rightarrow \R _0^+$ such that
\begin{enumerate}
\item (Homogeneity) $\norm[\alpha v]=\norm[\alpha ]\norm[v]$ for $\alpha \in \R$ and $v\in V$;
\item (Triangle Inequality) $\norm[v_1+v_2]\leq \norm[v_1]+\norm[v_2]$.
\end{enumerate}
$\norm$ is a \emph{norm} if furthermore (Definiteness) $\norm[x]=0$ implies $x=0$.
\end{dfn}
\begin{dfn}[Semimetric induced by a seminorm]
Let $V$ be a real vector space, let $\norm$ be a seminorm on $V$, and let $v_1,v_2\in V$.  Then, the \emph{semimetric induced by $\norm$}, $\metric$, is defined by
\begin{equation}
\metric[v_1][v_2]\coloneqq \norm[v_1-v_2].
\end{equation}
\begin{exr}
Check that $\metric$ is indeed a semimetric.  Show that if $\norm$ is a norm then $\metric$ is a metric.
\end{exr}
\begin{rmk}
Intuitively, the seminorm of something is like its `size' and semimetric is like `distance'---the `distance' between two vectors is the `size' of their difference.
\end{rmk}
\end{dfn}
\begin{dfn}[Seminormed vector space and normed vector space]\label{SeminormedVectorSpace}
A \emph{seminormed vector space}\index{Seminormed vector space} is a real vector space $V$ together with a collection of seminorms $\mathcal{D}$ such that $\coord{V,\mathcal{D}}$ is a semimetric space.  $\coord{V,\mathcal{D}}$ is a \emph{normed vector space}\index{Normed vector space} iff it is furthermore a metric space.
\end{dfn}
As by now you should have expected, the morphisms of seminormed vector spaces are the bounded (\cref{BoundedMap}) linear maps.
\begin{exm}[The category of seminormed vector spaces]
The category of seminormed vector spaces is the category $\Semi \Vect$ whose collection of objects $\Semi \Vect _0$ is the collection of all seminormed vector spaces and for every seminormed vector space $V$ and seminormed vector space $W$ the collection of all morphisms from $V$ to $W$, $\Mor _{\Semi \Vect}(V,W)$ is preicsely the set of all bounded linear maps from $V$ to $W$, composition is givne by ordinary function composition, and the identities of the category are the identity.
\end{exm}
\begin{exr}
Let $f:\coord{V,\mathcal{D}}\rightarrow \coord{W,\mathcal{E}}$ be a linear map between seminormed vector spaces.  Show that $f$ is bounded iff for every $\norm _0\in \mathcal{E}$ there are \emph{finitely-many} $\norm _1,\ldots ,\norm _m\in \mathcal{D}$ and constants $K_1,\ldots ,K_m\geq 0$ such that
\begin{equation}
\norm[f(v)]_0\leq K_1\norm[v]_1+\cdots +K_m\norm[v]_m
\end{equation}
for all $v\in V$.
\begin{rmk}
Compare this with the definition of bounded maps of semimetric spaces (\cref{BoundedMap}).  Essentially this boils down to the statement that, to check that $f$ is bounded, it suffices to check for $x_2=0$ and $x_1\eqqcolon v$ arbitrary (in the notation of \cref{BoundedMap}).
\end{rmk}
\end{exr}
Note that a priori a seminormed vector space is not a topological vector space.  However, being in metric a semimetric space, it is in fact a uniform space, and so in turn is equipped with its uniform topology.  The question is then whether it is a topological vector space with respect to the uniform topology.  Of course, the answer is in the affirmative.  Once we know that the seminormed vector space $V$ is likewise a topological vector space, we know in turn that its underlying topological group $\coord{V,+,0,-}$ induces in turn yet another uniform structure, and so a new question arises as to whether or not this uniform structures agrees with the one induced from the semimetric space structure.  Of course, the answer to this is likewise in the affirmative.
\begin{exr}
Let $V$ be a seminormed vector space.  Show that $V$ is a topological vector space with respect to the uniform topology induced by the semimetric uniformity.
\end{exr}
\begin{exr}
Let $V$ be a seminormed vector space.  Show that the uniformity induced by the topological group structure $\coord{V,+,0,-}$ is the same as the semimetric uniformity.
\end{exr}
As a mater of fact, the morphisms don't care whether you're thinking of things as a semimetric space or as a topological group either.
\begin{exr}\label{exr4.3.57}
Let $f:\coord{V,\mathcal{D}}\rightarrow \coord{W,\mathcal{E}}$ be a \emph{linear} map between two seminormed vector spaces.  Show that it is continuous iff it is bounded.
\end{exr}
\begin{displayquote}
Unless otherwise stated, seminormed vector spaces are always equipped with the uniformity induced by the semimetric space structure (or, equivalently, the uniformity induced by the topological group structure).
\end{displayquote}

In fact, a lot of examples of seminormed vector spaces have \emph{even more} structure, namely, the structure of an algebra.
\begin{dfn}[Algebra]\label{Algebra}
An \emph{algebra}\index{Algebra} is a set $A$ equipped with the structure of a vector space over a field $F$ and the structure of a ring such that
\begin{enumerate}
\item $(\alpha _1\alpha _2)\cdot a=\alpha _1\cdot (\alpha _2\cdot a)$ for $\alpha _1,\alpha _2\in F$ and $a\in A$; and
\item $\alpha \cdot (a_1a_2)=(\alpha \cdot a_1)a_2$ for $\alpha \in F$ and $a_1,a_2\in A$.
\end{enumerate}
\begin{rmk}
That is, an algebra is both a vector space and a ring subject to a couple of compatibility axioms.
\end{rmk}
\end{dfn}
\begin{dfn}[Homomorphism (of algebras)]\label{HomomorphismOfAlgebras}
Let $A$ and $B$ be algebras and let $f:A\rightarrow B$ be a function.  Then, $f$ is a \emph{homomorphism}\index{Homomorphism (of algebras)} iff $f$ is both a linear map of the underlying vectors spaces and a ring homomorphism of the underlying vector spaces.
\end{dfn}
\begin{exm}[The category of algebras over a field $F$]
The category of algebras over a field $F$ is the category $\Alg _F$ whose collection of objects is the collection of all algebras over $F$, for every algebra $A$ and algebra $B$ over $F$ the collection of all morphisms from $X$ to $Y$, $\Mor _{\Alg _F}(A,B)$, is precisely the set of all homomorphisms from $A$ to $B$s, composition is given by ordinary function composition, and the identities of the categories are the identity functions.
\end{exm}
\begin{dfn}[Seminormed algebra]\label{SeminormedAlgebra}
A \emph{seminormed algebra}\index{Seminormed algebra} is an algebra whose underlying vector space is a seminormed vector space such that $\norm[a_1a_2]\leq \norm[a_1]\norm[a_2]$ for $a_1,a_2\in A$.
\end{dfn}
\begin{exm}[The category of seminormed algebras]
The category of seminormed algebras is the category $\Semi \Alg$ whose collection of objects $\Semi \Alg _0$ is the collection of all seminormed algebras, for every seminormed algebras $A$ and seminormed algebra $B$ the collection of all morphisms from $A$ to $B$, $\Mor _{\Semi \Alg}(A,B)$, is precisely the set of all bounded homomorphisms, composition is given by ordinary function composition, and the identities of the category are the identity functions.
\end{exm}
With these new definitions in hand, we now present an incredibly important example of a seminormed algebra, an example that was a large part of the motivation for introducing seminormed algebras at all.
\begin{exm}[Uniform convergence on quasicompact subsets]\label{exm4.3.60}
Let $X$ be a topological space and define
\begin{equation}
A\coloneqq \Mor _{\Top}(X,\R ).
\end{equation}
Pointwise addition and pointwise scalar multiplication gives $A$ the structure of a real vector space.  Pointwise multiplication gives $A$ in turn the structure of a real algebra.  The collection $\{ \norm _K:K\subseteq X\text{ quasicompact}\}$, where
\begin{equation}
\norm[f]_K\coloneqq \sup _{x\in K}\{ \abs{f(x)}\} 
\end{equation}
then gives $A$ the structure of a seminormed algebra.  It is thus canonically a uniform space (and in turn a topological space).  If $X$ itself is quasicompact, convergence in $\Mor _{\Top}(X,\R )$ is called \emph{uniform convergence}\index{Uniform convergence}.\footnote{Despite the name and the context in which we're presenting it, uniform convergence actually has nothing to do with uniform spaces per se (in contrast to uniform continuity, for example).  $\Mor _{\Top}(X,\R )$ has a topology, and hence a notion of convergence, which we happen to call ``uniform convergence''.  In particular, we only needed to equip $\Mor _{\Top}(X,\R )$ with a topology to define uniform convergence.  The reason we waited until the chapter on uniform spaces, of course, is because $\Mor _{\Top}(X,\R )$ obtains its topology from a family of semimetrics (or in the case $X$ is quasicompact, just a single metric), not because of any direct connection with uniform convergence and uniform spaces.}  Thus, in the general case, people refer to convergence in the topological space $\Mor _{\Top}(X,\R )$ as \emph{uniform convergence on quasicompact subsets}.

The reason the case $X$ quasicompact is special is because, in this case, it is actually isomorphic (in the category of seminormed algebras) to a normed algebra.
\begin{exr}
Let $X$ be quasicompact.  Show that
\begin{equation}
\id _X:\coord{X,\{ \norm _X\}}\rightarrow \coord{X,\{ \norm _K:K\subseteq X\text{ quasicompact}\}} .
\end{equation}
is an isomorphism in the category of seminormed algebras.
\begin{rmk}
The point is that, if all we care about is the seminormed algebra structure, we may always assume without loss of generality that $\Mor _{\Top}(X,\R )$ is in fact a \emph{normed} algebra with the single norm being given by $\norm[f]_X\coloneqq \sup _{X\in X}\{ \abs{f(x)}\}$.
\end{rmk}
\end{exr}
\begin{displayquote}
Unless otherwise stated, $\Mor _{\Top}(X,\R )$ is a always given the structure of a seminormed algebra, the algebra structure defined pointwise and the seminorms being $\norm _K$ for $K\subseteq X$ quasicompact.
\end{displayquote}

Of \emph{incredible} importance is that this space is in fact complete (at least for so-called \emph{quasicompactly-generated} spaces).  Of course, we need to first actually define what we mean by complete, and so we postpone this result---see \cref{thm4.5.6}.
\end{exm}

\section{$T_0$ uniform spaces are uniformly-completely-$T_3$}

Our goal in this subsection is to show that all $T_0$ uniform spaces are uniformly-$T_3$.  Of course, to prove this, we had better say what uniformly-completely-$T_3$ means.

\subsection{Separation axioms in uniform spaces}

Throughout this subsection, let $S_1,S_2\subseteq X$ be \emph{disjoint} subsets of a uniform space $X$.
\begin{dfn}[Uniformly-distinguishable]\label{UniformlyDistinguishable}
$S_1$ and $S_2$ are \emph{uniformly-distinguishable}\index{Uniformly-distinguishable} iff there is some uniform cover $\mathcal{U}$ for which $\Star _{\mathcal{U}}(S_1)\neq \Star _{\mathcal{U}}(S_2)$.
\end{dfn}
\begin{dfn}[Uniformly-separated]\label{UniformlySeparated}
$S_1$ and $S_2$ are \emph{uniformly-separated}\index{Uniformly-separated} iff there is some uniform cover $\mathcal{U}$ for which $\star _{\mathcal{U}}(S_1)$ is disjoint from $\Star _{\mathcal{U}}(S_2)$.
\begin{rmk}
Note that this is analogous to the topological condition of being ``separated by neighborhoods'' (\cref{SeparatedByNeighborhoods})---there is not really any uniform analgoe of just being plain separated (\cref{Separated}).
\end{rmk}
\end{dfn}
\begin{dfn}[Uniformly-completely-separated]\label{UniformlyCompletelySeparated}
$S_1$ and $S_2$ are \emph{uniformly-completely-separated}\index{Uniformly-completely-separated} iff there is a uniformly-continuous function $f:X\rightarrow [0,1]$ such that $\restr{f}{S_1}=0$ and $\restr{f}{S_2}=1$.
\end{dfn}
\begin{exr}
Show that if $S_1$ and $S_2$ are uniformly-completely-separated, then they are uniformly-separated.
\end{exr}
\begin{dfn}[Uniformly-perfectly-separated]\label{UniformlyPerfectlySeparated}
$S_1$ and $S_2$ are \emph{uniformly-perfectly-separated}\index{Uniformly-perfectly-separated} iff there is a uniformly-continuous function $f:X\rightarrow [0,1]$ such that $S_1=f^{-1}(0)$ and $S_2=f^{-1}(1)$.
\end{dfn}
\begin{dfn}[Uniformly-$T_0$]\label{UniformlyT0}
$X$ is \emph{uniformly-$T_0$}\index{Uniformly-$T_0$} iff any two distinct points are uniformly-distinguishable.
\end{dfn}
\begin{dfn}[Uniformly-$T_2$]\label{UniformlyT2}
$X$ is \emph{uniformly-$T_2$}\index{Uniformly-$T_2$} iff any two distinct points can be uniformly-separated.
\end{dfn}
\begin{dfn}[Uniformly-completely-$T_2$]\label{UniformlyCompletelyT2}
$X$ is \emph{uniformly-completely-$T_2$}\index{uniformly-completely-$T_2$} iff any two distinct points can be uniformly-completely-separated.
\end{dfn}
\begin{dfn}[Uniformly-perfectly-$T_2$]\label{UniformlyPerfectlyT2}
$X$ is \emph{uniformly-perfectly-$T_2$}\index{uniformly-perfectly-$T_2$} iff any two distinct points can be uniformly-perfectly-separated.
\end{dfn}
\begin{dfn}[Uniformly-$T_3$]\label{UniformlyT3}
$X$ is \emph{uniformly-$T_3$}\index{Uniformly-$T_3$} iff it is $T_1$ and any closed set and a point not contained in it can be uniformly-separated.
\end{dfn}
\begin{dfn}[Uniformly-completely-$T_3$]\label{UniformlyCompletelyT3}
$X$ is \emph{uniformly-completely-$T_3$}\index{Uniformly-completely-$T_3$} iff it is $T_1$ and any closed set and a point not contained in it can be uniformly-completely-separated.
\end{dfn}
\begin{dfn}[Uniformly-perfectly-$T_3$]\label{UniformlyPerfectlyT3}
$X$ is \emph{uniformly-perfectly-$T_3$}\index{Uniformly-perfectly-$T_3$} iff it is $T_1$ and any closed set and a point not contained in it can be uniformly-perfectly-separated.
\end{dfn}
\begin{dfn}[Uniformly-$T_4$]\label{UniformlyT4}
$X$ is \emph{uniformly-$T_4$}\index{Uniformly-$T_4$} iff it is $T_1$ and any two disjoint closed subsets can be uniformly-separated.
\end{dfn}
\begin{dfn}[Uniformly-completely-$T_4$]\label{UniformlyCompletelyT4}
$X$ is \emph{uniformly-completely-$T_4$}\index{Uniformly-completely-$T_4$} iff it is $T_1$ and any two disjoint closed subsets can be uniformly-completely-separated.
\end{dfn}
\begin{dfn}[Uniformly-perfectly-$T_4$]\label{UniformlyPerfectlyT4}
$X$ is \emph{uniformly-perfectly-$T_4$}\index{Uniformly-perfectly-$T_4$} iff it is $T_1$ and any closed set and any two disjoint closed subsets can be can be uniformly-perfectly-separated.
\end{dfn}
The goal of this section is to prove that all of the from uniformly-$T_0$ to uniformly-completely-$T-3$ (that is, $T_0$ implies uniformly-completely-$T_3$---see \cref{crl4.4.16}).  We also present counter-examples to show that all these equivalent axioms are strictly weaker than both uniformly-$T_4$ (see \cref{exm4.4.20}) and uniformly-perfectly-$T_4$ (see \cref{exm4.4.23}).

\subsection{The key result}

We actually prove a stronger result which requires the notion of the \emph{diameter} of a set (in a metric space).
\begin{dfn}[Diameter]\label{Diameter}
Let $\coord{X,\metric}$ be a metric space and let $S\subseteq X$.  Then, the \emph{diameter}\index{Diameter} of $S$, $\diam (S)$, is defined by
\begin{equation}
\diam (S)\coloneqq \sup _{x,y\in S}\{ \metric[x][y]\} .
\end{equation}
\begin{rmk}
Of course, it may be the case that $\diam (S)=\infty$.
\end{rmk}
\end{dfn}
And now we are ready to state our key result.
\begin{thm}
\begin{savenotes}
Let $\mathcal{U}$ be a uniform cover of a $T_0$ uniform space $X$.  Then, there exists a metric space $Y$ and a uniformly-continuous surjective function $\q :X\rightarrow Y$ such that, if $\diam (S)<1$ for $S\subseteq Y$, then $\q ^{-1}(S)$ will be contained in some element of $\mathcal{U}$.
\begin{proof}\footnote{Proof adapted from \cite[pg.~8]{Isbell}.}
To construct $Y$, we shall put a semimetric on $X$ and then take the quotient set with respect to the equivalence relation of `being infinitely close to each other'.

\Step{Construct a sequence of star-refinements of $\mathcal{U}$}
Let us write $\mathcal{U}_0\coloneqq \mathcal{U}$.  Then, we take a star-refinement $\mathcal{U}_1$ of $\mathcal{U}_0$, in turn another star-refinement $\mathcal{U}_2$ of $\mathcal{U}_1$, and so on.

\Step{Define $\ell (x_1,x_2)$ for $x_1,x_2\in X$}\label{stp4.8.76.2}
Define
\begin{equation}
\ell (x_1,x_2)\coloneqq \begin{cases}2 & \text{if }x_2\notin \Star _{\mathcal{U}_0}(x_1) \\ 2^{1-\max \{ m\in \N :x_2\notin \Star _{\mathcal{U}_m}(x_1)\}} & \text{otherwise}\end{cases}.
\end{equation}
Note that this in particular implies that $\ell (x_1,x_2)=0$ if $x_2\in \Star _{\mathcal{U}_m}(x_1)$ for all $m\in \N$.  (We do need to make the other extreme case explicit as the maximum of the empty-set is $-\infty$.)  Thus, the statement that $X$ is $T_0$ (together with the fact that stars for a base for the topology (\cref{UniformTopology})) implies that either $\ell (x_1,x_2)>0$ or $\ell (x_2,x_1)>0$.

\Step{Define $\ell (\mathcal{P})$ for paths $\mathcal{P}$}
For the purposes of this proof, a \emph{path} from $x_1$ to $x_2$ will be a finite sequence of points $(x^\infty,x^1,\ldots ,x^m)$ with $x^\infty=x_1$ and $x^m=x_2$.\footnote{The superscripts (as opposed to subscripts) if obviously for the purpose of not conflicting with the subscripts on $x_1$ and $x_2$.}  If $\mathcal{P}=(x^\infty,\ldots ,x^m)$, then we define $\ell (\mathcal{P})\coloneqq \ell (x^\infty,x^1)+\ell (x^1,x^2)+\cdots +\ell (x^{m-1},x^m)$.   We shall call this the \emph{length} of the path.

\Step{Define the semimetric}
Finally, we define
\begin{equation}
\metric[x_1][x_2]\coloneqq \min \{ \inf \left( \left\{ \ell (\mathcal{P}):\mathcal{P}\text{ is a path from }x_1\text{ to }x_2\text{ or a path from }x_2\text{ to }x_1\text{.}\right\} \right) ,1\} .
\end{equation}

\Step{Show that this is in fact a semimetric}
From the definition, we have that $\metric$ is symmetric (this is the reason for putting the ``or'' in the definition---note that the definition of $\ell (x_1,x_2)$ is not manifestly symmetric).  The triangle inequality follows from the fact that a path from $x_1$ to $x_3$ and a path from $x_3$ to $x_2$ gives us a path from $x_1$ to $x_2$, with the length of this new path being the sum of the lengths of the other two.  Thus, $\metric$ is in fact a semimetric.

\Step{Construct $Y$}
Define $x_1\sim x_2$ iff $\metric[x_1][x_2]=0$.  That this is an equivalence relation follows from the fact that $\metric$ is a semimetric.  Thus, we may define
\begin{equation}
Y\coloneqq X/\sim .
\end{equation}

\Step{Construct the metric on $Y$}
We abuse notation and write the induced metric on $Y$ with the same symbol $\metric$ as the semimetric on $X$:
\begin{equation}
\metric[[x_1]_{\sim}][[x_2]_{\sim}]\coloneqq \metric[x_1][x_2].
\end{equation}
\begin{exr}
Check that $\metric$ on $Y$ is well-defined.
\end{exr}

\Step{Show that this in fact a metric}
$\metric$ on $Y$ is automatically symmetric and satisfies the triangle inequality because $\metric$ on $X$ does.  Furthermore, if $\metric[[x_1]_{\sim}][[x_2]_{\sim}]=0$, then $\metric[x_1][x_2]=0$, and so $x_1\sim x_2$ by the definition of $\sim$.  Thus, $\metric$ is indeed a metric on $Y$.

\Step{Define $\q :X\rightarrow Y$}
We take $\q :X\rightarrow Y$ to be the quotient map:  $\q (x)\coloneqq [x]_{\sim}$.  Of course $\q$ is surjective (all quotient maps are).

\Step{Show that $\q$ is uniformly-continuous}
We apply \cref{prp4.8.54} which characterizes uniform-continuity for functions whose codomain is a metric space.  So, let $\varepsilon >0$.  We must find a uniform cover $\mathcal{U}$ of $X$ such that for every $U\in \mathcal{U}$, whenever $x_1,x_2\in U$, it follows that $\metric[\q (x_1)][\q (x_2)]<\varepsilon$.  It suffices to show this for $\varepsilon \coloneqq 2^{1-m}$.  We show that $\mathcal{U}_m$ is a uniform cover that `works'.  So, let $U\in \mathcal{U}_m$ and let $x_1,x_2\in U$.   Then, in particular, $x_2\in \Star _{\mathcal{U}_m}(x_1)$, and so
\begin{equation}
\metric[\q (x_1)][\q (x_2)]\coloneqq \metric[x_1][x_2]<2^{1-m}\eqqcolon \varepsilon .
\end{equation}

\Step{Finish the proof by proving the desired property of $\q$}
Let $S\subseteq Y$ and suppose that $\diam (S)<1$.  We wish to show that there is some $U\in \mathcal{U}_0$ such that $S\subseteq U$.  It suffices to show that for $x_1,x_2\in S$, there is some $U_{x_1,x_2}\in \mathcal{U}_1$ such that $x_1,x_2\in U_{x_1,x_2}$.  This is because, if this is true, then $S\subseteq \Star _{\mathcal{U}_1}(x_1)$, which in turn is contained in some element of $\mathcal{U}_0$ because $\mathcal{U}_1$ star-refines $\mathcal{U}_0$.

To show this, it suffices to show that if $\metric[x_1][x_2]<2^{1-m}$ for $m\in \Z ^+$, then there is some $U\in \mathcal{U}_m$ such that $x_1,x_2\in U$.  So, let $x_1,x_2\in X$ be such that $\metric[x_1][x_2]<2^{1-m}$.  Then, without loss of generality, there is some path $(x^\infty,\ldots ,x^n)$ from $x_1$ to $x_2$ with
\begin{equation}\label{4.8.53}
\ell (x^\infty,x^1)+\cdots +\ell (x^{n-1},x^n)<2^{1-m}.
\end{equation}
It thus suffices to show that, whenever \eqref{4.8.53} holds, there is some $U\in \mathcal{U}_m$ with $x^\infty,x^n\in U$.  We prove this by induction on $n$.  For $n=1$, \eqref{4.8.53} implies that $\ell (x_1,x_2),\ell (x_2,x_1)<2^{1-m}$, as was mentioned above in \cref{stp4.8.76.2}, because $X$ is $T_0$, at least one of these is strictly positive---without loss of generality suppose that $\ell (x_1,x_2)>0$.  Then, the fact that $\ell (x_1,x_2)<2^{1-m}$ implies that
\begin{equation}
2^{1-\max \{ o\in \N :x_2\notin \Star _{\mathcal{U}_o}(x_1)\}}<2^{1-m},
\end{equation}
so that
\begin{equation}
m\leq \max \{ o\in \N :x_2\notin \Star _{\mathcal{U}_o}(x_1)\} =\footnote{This equality implicitly uses the fact that $\ell (x_1,x_2)>0$, so that $\{ o\in \N :x_2\in \Star _{\mathcal{U}_o}\}$ is bounded above.}\min \{ o\in \N :x_2\in \Star _{\mathcal{U}_o}(x_1)\} ,
\end{equation}
which implies that $x_2\in \Star _{\mathcal{U}_{m+1}}(x_1)$, which implies that there is some $U\in \mathcal{U}_{m+1}$ such that $x_1,x_2\in U$.  Thus, this does the case for $n=1$.  (Note that in fact we can take $U\in \mathcal{U}_{m+1}$---this will be important later.)

Now assume the result is true for all $k\leq n$.  We wish to prove the result for $n+1$.

We must have that $\ell (x^\infty,x^1)<2^{1-n}$, because otherwise we would have to have that $\ell (x^k,x^{k+1})$ for $k\geq 1$, in which case $x^k$ and $x^{k+1}$ lie in some $U\in \mathcal{U}_o$ for $o$ arbitrarily large.  We can then guarantee that $x^k$ for $k\geq 1$ are obtained in some element of $\mathcal{U}_{m+1}$, and hence, as $x^\infty$ and $x^1$ are obtained in some element of $\mathcal{U}_{m+1}$, everything is contained in some element of $\mathcal{U}_m$.

Thus, without loss of generality assume that $\ell (x^\infty,x^1)<2^{1-n}$.  Then, the inequality \eqref{4.8.53} implies that there is some $k_0$ such that
\begin{equation}
\ell (x^\infty,x^1)+\cdots \ell (x^{k_0-1},x^{k_0})\leq 2^{-m}.
\end{equation}
(This is the same inequality with $m$ one larger).  Similarly, there is some $l_0$ such that
\begin{equation}
\ell (x^{l_0},x^{l_0+1})+\cdots +\ell (x^{n-1},x^n)\leq 2^{-m}.
\end{equation}
It then follows that we must also have that
\begin{equation}
\ell (x^{k_0},x^{k_0+1})+\cdots +\ell (x^{l_0-1},x^{l_0}).
\end{equation}
By the induction hypotheses, we then must have in particular that there are $U_1,U_2,U_3\in \mathcal{U}_{m+1}$ such that $x^\infty,x^{k_0}\in U_1$, $x^{l_0},x^n\in U_2$, and $x^{k_0},x^{l_0}\in U_3$.  Then, there is some $U\in \mathcal{U}_m$ such that $\Star _{\mathcal{U}_{m+1}}(U_3)\subseteq U$.  However, as $U_1,U_2\subseteq \Star _{\mathcal{U}_{m+1}}(U_3)$, this completes the proof.
\end{proof}
\end{savenotes}
\end{thm}
From this, that every uniform space is uniformly-completely-$T_3$ follows relatively easily.
\begin{crl}\label{crl4.4.16}
Let $X$ be a $T_0$ uniform space.  Then, $X$ is uniformly-completely-$T_3$.
\begin{proof}\footnote{Proof adapted from \cite[pg.~8]{Isbell}.}
Let $X$ be a uniform space.  We first show that $X$ is $T_1$.  We know that $X$ is $T_0$, and so by \cref{prp4.6.53} (regular $T_0$ spaces are $T_2$), $X$ is $T_2$, hence $T_1$.

We now show that uniformly-continuous functions on $X$ can separate closed sets from points.  So, let $C\subseteq X$ be closed, and let $x_0\in C^{\comp}$.  As $C$ is closed, there must be some neighborhood of $x_0$ that does not intersect $C$ (otherwise, $x_0$ would be an accumulation point of $C$).  Then, because stars for a basis for the topology (\cref{UniformTopology}), there is a uniform cover $\mathcal{U}$ such that
\begin{equation}\label{4.8.90}
\Star _{\mathcal{U}}(x_0)\subseteq C^{\comp}.
\end{equation}

Now apply the previous theorem for the uniform cover $\mathcal{U}$, so that there is a metric space $\coord{Y,\metric}$ and a uniformly-continuous map $\q :X\rightarrow Y$ such that, if $\diam (S)<1$ for $S\subseteq Y$, it follows that $\q ^{-1}(S)$ is contained in some element of $\mathcal{U}$.  From \eqref{4.8.90}, it follows that $C\cup \{ x_0\}$ is not contained in any element of $\mathcal{U}$, and so
\begin{equation}\label{4.8.97}
\diam (\q (C)\cup \{ \q (x_0)\})\geq 1.
\end{equation}
Define $f:X\rightarrow [0,1]$ by
\begin{equation}
f(x)\coloneqq \footnote{See \eqref{4.8.50} for the definition of $\dist _C$.}\max \{ \dist _C(x) ,1\} .
\end{equation}
This is uniformly-continuous because $\dist _C$ is.  \eqref{4.8.97} implies that $\dist _C(x_0)\geq 1$, and so $f(x_0)=1$.  We showed in \cref{prp4.8.49} that $\dist _C(C)=0$.
\end{proof}
\end{crl}

\subsection{The counter-examples}

We know from the diagram \eqref{4.6.105}, that if we are to `do any better' in terms of separation axioms, we would be able to prove that every uniform space is either perfectly-$T_3$ or completely-$T_4$ (which is equivalent to $T_4$ by Urysohn's Lemma, \cref{UrysohnsLemma}).  Unfortunately, however, there exist counter-examples to both these separation axioms.
\begin{exm}[A uniform space that is not perfectly-$T_3$]\label{exm4.4.20}
The Uncountable Fort Space $X$ of \cref{UncountableFortSpace} will do just fine yet again.  We already know that this space is not perfectly-$T_3$ from \cref{exm4.6.80}.  Thus, all that remains to be done is to equip $X$ with a uniformity that generates the Uncountable Fort Space Topology.

Define
\begin{equation}
\widetilde{\mathcal{B}}\coloneqq \left\{ f^{-1}(\mathcal{B}_{\varepsilon}):f\in \Mor _{\Top}(X,\R ),\ \varepsilon >0\right\} .
\end{equation}
\begin{exr}
Show that $\widetilde{\mathcal{B}}$ is a uniform base for the Uncountable Fort Space Topology.
\end{exr}
\end{exm}
\begin{exm}[A uniform space that is not $T_4$]\label{exm4.4.23}
\begin{savenotes}
We define a topology on $\Mor _{\Set}(\R ,\R )$,\footnote{$\Mor _{\Set}(\R ,\R )$ is our fancy-schmancy notation for the set of all functions from $\R$ to $\R$.} the \emph{topology of pointwise convergence}.  We use \nameref{KelleysConvergenceTheorem}, \cref{KelleysConvergenceTheorem}, to do it.

For $f_\infty \in \Mor _{\Set}(\R ,\R )$ and a let $\lambda \mapsto f_\lambda \in \Mor _{\Set}(\R ,\R )$.  Then,
\begin{textequation}
we declare that the net $\lambda \mapsto f_\lambda$ converges to $f_\infty$ iff the net $\lambda \mapsto f_\lambda (x)$ converges to $f_\infty (x)$ for every $x\in \R$.
\end{textequation}
\begin{exr}
Show that this definition satisfies the axioms of \nameref{KelleysConvergenceTheorem}, and so define a topology on $\Mor _{\Set}(\R ,\R )$.
\end{exr}
\begin{exr}
Show that ${\Mor _{\Set}(\R ,\R ),+}$ is a topological group, where $+$ is defined pointwise:
\begin{equation}
[f_1+f_2](x)\coloneqq f_1(x)+f_2(x).
\end{equation}
\end{exr}
Thus, $\Mor _{\Set}(\R ,\R )$ is canonically a uniform space (and hence completely-$T_3$).

We now shows that $\Mor _{\Set}(\R ,\R )$ is not completely-$T_4$ with respect to this topology.  To do this, we first show that $\Mor _{\Set}(\R ,\Z )\subseteq \Mor _{\Set}(\R ,\R )$ is not completely-$T_4$.\footnote{The proof of this is adapted from \cite[pg.~206]{Munkres}.}

For $m\in \Z$, define
\begin{equation}
P_m\coloneqq \left\{ f\in \Mor _{\Set}(\R ,\Z ):\restr{f}{[f^{-1}(m)]^{\comp}}\text{ is injective.}\right\} ,
\end{equation}
that is, the set of all functions that are injective `modulo sending more than one point to $m$'.  We show that $P_0$ and $P_1$ are closed and disjoint, but cannot be separated by open neighborhoods.

We first check that $P_0$ is closed (the proof that $P_1$ is closed is nearly identical).  So, let $\lambda \mapsto f_\lambda \in P_0$ converge to $f_\infty$.  Suppose that $f_\infty (x_1)=f_\infty (x_2)$ is distinct from $0$.  Then, by our definition of convergence and the fact that our functions are taking values in the integers, it must be the case that $\lambda \mapsto f_\lambda (x_i)$ is eventually equal to $f_\infty (x_i)$ for $i=1,2$.  Then, in particular, we will have that $f_{\lambda _0}(x_1)=f_{\lambda _0}(x_2)$ for $\lambda _0$ sufficiently large, and hence $x_1=x_2$.  Thus, $f_\infty \in P_0$.

We now check that $P_0$ and $P_1$ are disjoint.  If $f$ is injective on the complement of $f^{-1}(0)$ (i.e.~if $f\in P_0$), then this complement must be countable (because the image of $f$ lies in $\Z$).  In particular, there must be at least two elements in $f^{-1}(0)$, and so $f(x_1)=0=f(x_2)$ for $x_1\neq x_2$.  But then $f^{-1}$ cannot be injective on the complement of $f^{-1}(1)$, and so $f\notin P_1$.  Thus, $P_0$ is disjoint from $P_1$.

Let $A\coloneqq \{ \alpha _0,\alpha _1,\alpha _2,\ldots \}$ be a countably-infinite subset of $\R$, and for $S\subseteq A$ a finite, let us define
\begin{equation}
U_{S,f}\coloneqq \{ g\in \Mor _{\Set}(\R ,\Z ):\restr{g}{S}=\restr{f}{S}\} .
\end{equation}
The complement of this is the collection of all functions which differ from $f$ at at least one point of $S$.  Because the functions take their value in $\Z$, however, if you take a net of such functions, the limit (if it has one) must still disagree with $f$ at least one point.  Therefore, $U_{S,f}^{\comp}$ is closed, and hence $U_{S,f}^{\comp}$ is open.  Moreover, as
\begin{equation}
U_{S,f}\cap U_{T,f}=U_{S\cap T,f},
\end{equation}
it follows from \cref{prp4.1.8} that this is a neighborhood base for $f$.  If fact, if we restrict ourselves to only taking $S$ from a given infinite subset of $A$, we still get a neighborhood base.

Now, let $U$ and $V$ be open neighborhoods of $P_0$ and $P_1$ respectively.  Of course, we seek to show that $U$ and $V$ must intersect.  We seek to construct a sequence of functions $f_k\in U$ and a countably-infinite collection of finite subsets $B_k=\{ \alpha _0,\ldots ,\alpha _{m_k}\}$ of $A$ such that $U_{B_k,f_k}\subseteq U$ and
\begin{equation}\label{5.5.29}
f_k(x)\coloneqq \begin{cases} k & \text{if }x=\alpha _k\in B_{k-1} \\ 0 & \text{otherwise}\end{cases}.
\end{equation}
We do so inductively.  (We take $B_0\coloneqq \emptyset$.)

Take $f_1\coloneqq 0$, so that of course $f_1\in P_0$.  Thus, because $\left\{ U_{S,f_1}:S\subseteq A\text{ finite.}\right\}$ is a neighborhood base at $f_1$, there must be some finite subset $B_1\subseteq A$ such that $U_{B_1,f_1}\subseteq U$.  In fact, we can enlarge $B_1$ so that it is of the form $B_1=\{ \alpha _0,\ldots ,\alpha _{m_1}\}$.  Now define $f_2$ according to \eqref{5.5.29}, that is
\begin{equation}
f_2(x)\coloneqq \begin{cases}k & \text{if }x=\alpha _k\in \in B_1 \\ 0 & \text{otherwise}\end{cases}.
\end{equation}
Then, $f_2\in U$, and so there must be some finite set $B_2\subseteq A$ such that $U_{B_2,f_2}\subseteq U$.  By enlarging $B_2$ if necessary, we can guarantee it is of the form $B_2=\{ \alpha _0,\ldots ,\alpha _{m_2}\}$ with $m_2>m_1$.  Then, we may define
\begin{equation}
f_3(x)\coloneqq \begin{cases}2 & \text{if }x\in B_2\setminus B_1 \\ 1 & \text{if }x\in B_1 \\ 0 & \text{otherwise}\end{cases}.
\end{equation}
Then, for the same reason as before, $f_3\in U$, and so there is some finite set $B_3\subseteq A$ (that is without loss of generality of the form $B_3=\{ \alpha _0,\ldots ,\alpha _{m_3}\}$ with $m_3>m_2$) and $U_{B_3,f_3}\subseteq U$.  Continue this process inductively.

Now define $g\in \Mor _{\Set}(\R ,\Z )$ by
\begin{equation}
g(x)\coloneqq \begin{cases}k & \text{if }x=\alpha _k\in A \\ 1 & \text{otherwise}\end{cases}.
\end{equation}
Then, $g\in P_1$, and so there is some finite set $B\subseteq A$ such that $U_{B,g}\subseteq V$.  Let $m$ be sufficiently large so that $B\subseteq B_m$.  Then, $f_m\in U_{B,g}\subseteq V$ and $f_m\in U_{B_m,f_m}\subseteq U$, and so, in particular, lies in $U\cap V$.

This shows that $\Mor _{\Set}(\R ,\Z )$ is not $T_4$, but we still must show that $\Mor _{\Set}(\R ,\Z )$ itself is not $T_4$.  This will follow from the following lemma.
\begin{lma}
Let $X$ be $T_4$ and let $C\subseteq X$ be closed.  Then, $X$ is $T_4$.
\begin{rmk}
Note that this doesn't hold in general.  For example, this example shows that $\prod _\R \R \subseteq \prod _\R [-\infty ,\infty ]$ is not normal even though $\prod _\R [-\infty ,\infty ]$ is (it is compact by \cref{exr4.6.38} and \nameref{TychnoffsTheorem}, \cref{TychnoffsTheorem}, and hence $T_4$ by \cref{prp4.6.83}.
\end{rmk}
\begin{proof}
Let $C_1,C_2\subseteq C$ be disjoint and closed.  Then, by definition of the subspace topology (\cref{SubspaceTopology}), $C_1=C_1'\cap C$ and $C_2=C_2'\cap C$ for $C_1',C_2'\subseteq X$ closed, and so $C_1$ and $C_2$ are themselves closed in $X$ because $C$ is closed.  Thus, because $X$ is $T_4$, $C_1$ and $C_2$ can be separated by neighborhoods in $X$, and hence can be separated by neighborhoods in $C$.
\end{proof}
\end{lma}
\end{savenotes}
\end{exm}

Finally, we are ready to begin discussing cauchyness and completeness in the general context of uniform spaces.

\section{Cauchyness and completeness}

\begin{dfn}[Cauchyness]\label{Cauchyness}
Let $\coord{X,\widetilde{\mathcal{U}}}$ be a uniform space and let $\lambda \mapsto x_\lambda$ be a net.  Then, we say that $\lambda \mapsto x_\lambda$ is \emph{cauchy}\index{Cauchy (in uniform spaces)} iff for every $\mathcal{U}\in \widetilde{\mathcal{U}}$, there is some $U\in \mathcal{U}$ such that $\lambda \mapsto x_\lambda$ is eventually contained in $U$.
\begin{rmk}
You should compare this with our definition of cauchyness in $\R$, \cref{dfn3.3.26}.  As the collection of all $\varepsilon$-balls for all $\varepsilon >0$ forms a uniform base, if we show that we can replace the entire uniformity in the test of cauchyness, then the definition we gave before in \cref{dfn3.3.26} will be a word-for-word special case of this definition.  Indeed, part of the motivation for phrasing the definition in \cref{dfn3.3.26} the way we did was to make the transition to this higher level of generality as transparent as possible.  See the paragraphs that follow \cref{dfn3.3.26} for a few more comments regarding this.
\end{rmk}
\end{dfn}
To check whether a net is cauchy, it suffices to check on just a uniform base for the collection of uniform covers.
\begin{prp}\label{prpB.16}
Let $X$ be a set, let $\widetilde{\mathcal{B}}$ be a uniform base on $X$, and let $\lambda \mapsto x_\lambda$ be a net.  Then, $\lambda \mapsto x_\lambda$ is cauchy iff for every $\mathcal{B}\in \widetilde{\mathcal{B}}$ there is some $B\in \mathcal{B}$ such that $\lambda \mapsto x_\lambda$ is eventually contained in $B$.
\begin{rmk}
With this equivalence, the definition we gave before for cauchyness in $\R$ in \cref{dfn3.3.26} is literally verbatim equivalent to this definition upon replacement of $\widetilde{\mathcal{B}}$ with $\{ \mathcal{B}_\varepsilon :\varepsilon >0\}$ and of $\mathcal{B}$ with $\mathcal{B}_\varepsilon \coloneqq \{ B_\varepsilon (x):x\in \R \}$.
\end{rmk}
\begin{proof}
$(\Rightarrow )$ There is nothing to check.

\blankline
\noindent
$(\Leftarrow )$ Suppose that for every $\mathcal{B}\in \widetilde{\mathcal{B}}$ there is some $B\in \mathcal{B}$ such that $\lambda \mapsto x_\lambda$ is eventually contained in $B$.  Denote the uniformity on $X$ by $\widetilde{\mathcal{U}}$.  Let $\mathcal{U}\in \widetilde{\mathcal{U}}$.  Then, there is some $\mathcal{B}\in \widetilde{\mathcal{B}}$ such that $\mathcal{B}\llcurly \mathcal{U}$.  Thus, there is some $B\in \mathcal{B}$ such that $\lambda \mapsto x_\lambda$ is eventually contained in $B$.  As $\mathcal{B}\llcurly \mathcal{U}$, there is some $U\in \mathcal{U}$ such that $\Star _{\mathcal{B}}(B)\subseteq U$.   In particular, $B\subseteq U$, and so if $\lambda \mapsto x_\lambda$ is eventually contained in $B$, it is certainly eventually contained in $U$.
\end{proof}
\end{prp}
From this, we obtain relatively nice description of what it means to be cauchy in our two large families of examples, namely semimetric spaces and topological groups.
\begin{exr}\label{exr4.5.3x}
Let $\coord{X,\mathcal{D}}$ be a semimetric space and let $\lambda \mapsto x_\lambda \in X$ be a net.  Show that $\lambda \mapsto x_\lambda$ is cauchy iff for every $\metric \in \mathcal{D}$ and for every $\varepsilon >0$, $\lambda \mapsto x_\lambda$ is eventually contained in $B_{\varepsilon}^{\metric}(x)$ for \emph{some} $x\in X$.
\begin{rmk}
In other words, a net in a semimetric space is cauchy iff it is cauchy with respect to each semimetric.
\end{rmk}
\end{exr}
\begin{exr}
Let $G$ be a topological group and let $\lambda \mapsto g_\lambda \in G$ be a net.  Show that $\lambda \mapsto g_\lambda$ is cauchy iff for every open neighborhood $U$ of the identity there is \emph{some} $g\in G$ such that $\lambda \mapsto g_\lambda$ is eventually contained in $gU$.
\end{exr}

Just as continuous functions preserve convergence, so to do uniformly-continuous functions preserve cauchyness.
\begin{exr}\label{exr4.5.3}
Let $f:X\rightarrow Y$ be uniformly-continuous and let $\lambda \mapsto x_\lambda \in X$ be cauchy.  Show that $\lambda \mapsto f(x_\lambda)$ is cauchy.
\end{exr}
Warning:  Continuous functions do \emph{not} necessarily preserve cauchyness.
\begin{exm}[A continuous image of a cauchy net need not be cauchy]
Let $X \coloneqq (-\frac{\uppi}{2},\frac{\uppi}{2})$ and define $m\mapsto \arctan (m)\in X$.  This is certainly cauchy as, for every $\varepsilon >0$, it is eventually contained in $(\frac{\uppi}{2}-\varepsilon ,\frac{\uppi}{2})$.  On the other hand, its image under the continuous function $\tan :X\rightarrow \R$ is not even bounded, much less cauchy.
\end{exm}

Of course, if we know what it means for nets to be cauchy, then we likewise have a notion of what it means for uniform spaces to be \emph{complete}.
\begin{dfn}[Completeness]\label{Completeness}
A uniform space is \emph{complete}\index{Complete (uniform space)} iff every cauchy net converges.
\begin{rmk}
In case there might be some confusion (e.g.~if the topology of the underlying uniform space comes from a totally-ordered set), then we shall say \emph{cauchy-complete} in contrast to \emph{dedekind complete}.
\end{rmk}
\end{dfn}
\begin{exr}
Show that every compact metric space is complete.
\end{exr}

\subsection{Completeness of $\Mor _{\Top}(X,\R )$}

We mentioned back above in \cref{exm4.3.60} that $\Mor _{\Top}(X,\R )$ is complete.  We now prove this.
\begin{thm}\label{thm4.5.6}
Let $X$ be a topological space that has the property that a subset is open iff its intersection with each quasicompact subset $K$ is open in $K$.  $\Mor _{\Top}(X,\R )$ is complete.
\begin{rmk}
In particular, a the limit of a uniformly convergent net of continuous functions is continuous.  This need not be the case if the convergence is just pointwise---see \cref{exm4.5.12x}.
\end{rmk}
\begin{rmk}
This condition on $X$ is called \emph{quasicompactly-generated}\index{Quasicompactly-generated}.  For example, the cocountable topology on $\R$ is \emph{not} quasicompactly-generated---see \cref{exm4.5.12}.
\end{rmk}
\begin{proof}
\Step{Prove the result for $X$ quasicompact}
We first take $X$ to be quasicompact.  In this case, the uniform structure on $\Mor _{\Top}(X,\R )$ is the same as that generated by the single norm $\norm _X$.  Therefore, by \cref{exr4.5.3x} (cauchyness in semimetric spaces), to show that $\Mor _{\Top}(X,\R )$ is complete, it suffices to show that every net that is cauchy with respect to $\norm _X$ converges.  So, suppose that $\lambda \mapsto f_\lambda \in \Mor _{\Top}(X,\R )$ is cauchy.  As
\begin{equation}
\abs{f_{\lambda _1}(x)-f_{\lambda _2}(x)}\leq \norm[f_{\lambda _1}-f_{\lambda _2}],
\end{equation}
it follows that, for each $x\in X$, the net $\lambda \mapsto f_\lambda (x)\in \R$ is cauchy.  As $\R$ is complete, this net has a limit.  Call this limit $f_\infty (x)$.  We need to check two things:  (i) that $x\mapsto f_\infty (x)$ is continuous (so that indeed $f_\infty \in \Mor _{\Top}(X,\R )$, and (ii) that $\lambda \mapsto f_\lambda$ converges to $f_\infty$ in $\Mor _{\Top}(X,\R )$.

We first show that $f_\infty$ is continuous.  Let $\varepsilon >0$.  Let $\lambda _0$ be such that, whenever $\lambda _1,\lambda _2\geq \lambda _0$, it follows that $\norm[f_{\lambda _1}-f_{\lambda _2}]<\varepsilon$.  Let $U$ be an open neighborhood of $x_\infty$ such that $f_{\lambda _0}(U)\subseteq B_{\varepsilon}(f_{\lambda _0}(x_\infty ))$.  Let $x\in U$.  Let $\lambda _1,\lambda _2\geq \lambda _0$ be such that $\abs{f_{\lambda _1}(x)-f_\infty (x)},\abs{f_{\lambda _2}(x_\infty )-f_\infty (x_\infty )}<\varepsilon$.  Then,
\begin{equation}
\begin{split}
\abs{f_\infty (x)-f_\infty (x_\infty )} & \leq \abs{f_\infty (x)-f_{\lambda _1}(x)}+\abs{f_{\lambda _1}(x)-f_{\lambda _0}(x)}+\abs{f_{\lambda _0}(x)-f_{\lambda _0}(x_\infty )} \\
& \qquad +\abs{f_{\lambda _0}(x_\infty )-f_{\lambda _2}(x_\infty )}+\abs{f_{\lambda _2}(x_\infty )-f_\infty (x_\infty )}<5\varepsilon .
\end{split}
\end{equation}
Thus, $f_\infty$ is continuous.

We now check that $\lambda \mapsto f_\lambda$ converges to $f_\infty$ in $\Mor _{\Top}(X,\R )$.  Let $\varepsilon >0$.  Now that we know that $f_\infty$ is continuous, for each $x\in X$, there is some open neighborhood $U_x$ of $x$ such that $f_\infty U_x)\subseteq B_{\varepsilon}(f_\infty (x))$.  Then, 
\begin{equation}
\left\{ U_x:x\in X\right\} 
\end{equation}
is an open cover of $X$.  Therefore, there is a finite subcover, $U_{x_1},\ldots ,U_{x_m}$.  Thus, we may choose $\lambda _0$ such that, whenever $\lambda \geq \lambda _0$, it follows that $\abs{f_\lambda (x_k)-f_\infty (x_k)}<\varepsilon$ for all $1\leq k\leq m$.  Now let $x\in X$ be arbitrary.  Without loss of generality, assume that $x\in U_1$.  Then, whenever $\lambda \geq \lambda _0$, it follows that
\begin{equation}
\abs{f_\lambda (x)-f_\infty (x)}\leq \abs{f_\lambda (x)-f_\lambda (x_1)}+\abs{f_\lambda (x_1)+f_\infty (x_1)}<2\varepsilon .
\end{equation}
Taking the supremum over $x$, we find that
\begin{equation}
\norm[f_\lambda -f_\infty ]<\varepsilon ,
\end{equation}
so that indeed $\lambda \mapsto f_\lambda$ converges to $f_\infty$.

\Step{Prove the result in general}
We now do the general case, in which case $X$ is not necessarily quasicompact.  So, let $\lambda \mapsto f_\lambda \in \Mor _{\Top}(X,\R )$ be cauchy.  Then, by \cref{exr4.5.3} again, we must have that $\lambda \mapsto \restr{f_\lambda}{K}\Mor _{\Top}(K,\R )$ is cauchy for each quasicompact subset $K\subseteq X$.  Therefore, by quasicompact case, $\lambda \mapsto f_\lambda$ converges to its pointwise limit $f_\infty$ on each quasicompact subset.  As $f_\infty$ is continuous on each quasicompact subset, the intersection of the preimage of every open set with every quasicompact subset of $X$ is open, and hence, by hypothesis, is open.  Therefore, $f_\infty$ is continuous.  Furthermore, $\lambda \mapsto f_\lambda$ converges to $f_\infty$ in $\Mor _{\Top}(X,\R )$ because it converges to $f_\infty$ on each quasicompact subset.
\end{proof}
\end{thm}
Now that we've just proven what goes right, we turn to the more interesting side of things---what can go wrong.
\begin{exm}[A pointwise limit of continuous functions that is not continuous]\label{exm4.5.12x}
Let $f_m:[0,1]\rightarrow \R$ be defined by $f_m(x)\coloneqq x^m$.  Then, of course each $f_m$ is continuous.  On the other hand,
\begin{equation}
\lim _mf(x)=\begin{cases}0 & \text{if }x\in [0,1) \\ 1 & \text{if }x=1\end{cases},
\end{equation}
which is clearly not continuous.
\end{exm}
\begin{exm}[A discontinuous function that is continuous on every quasicompact subset]\label{exm4.5.12}
Take $X\coloneqq \R$ and equip it with our good old-buddy, the cocountable topology.  We claim that a subset of $X$ is quasicompact iff it is finite.  Finite subsets are always quasicompact.  On the other hand, take $K\subseteq X$ infinite.  Then, there is in particular a countably-infinite subset $\{ x_0,x_1,x_2,\ldots \} \subseteq K$.  Define $C_m\coloneqq \{ x_k:k\geq m\}$ and $\mathcal{C}\coloneqq \{ C_m:m\in \N \}$.  Then, $\mathcal{C}$ is a collection of closed subsets of $X$.  Furthermore, the intersection of any finitely many of them intersects $K$.  On the other hand, the intersection of all of them is empty.  Therefore, $K$ is not quasicompact.  In particular, $X$ is \emph{not} quasicompactly-generated.\footnote{Finite subsets of $X$ are discrete, and so from what we just showed, it follows that all quasicompact subsets of $X$ are discrete.  Therefore, $S\cap K$ is always open in $K$, regardless of whether $S$ is open or not.}

Now let $f:X\rightarrow \R$ be any discontinuous function.  For example, the Dirichlet Function is discontinuous with respect to the cocountable topology (because $\Q ^{\comp}$ is not closed).  On the other hand, finite subsets of $X$ are discrete, and so $f$ restricted to finite subsets must be continuous.
\end{exm}
We can use this trick to find an example of a space for which $\Mor _{\Top}(X,\R )$ is not complete.
\begin{exm}[A topological space for which $\Mor _{\Top}(X,\R )$ is not complete]
Take $X\coloneqq \R$ equipped with the cocountable extension topology.  Recall that this means that the only closed sets are (i) $X$ itself, (ii) countable subsets, and (iii) subsets which are closed in the usual topology of $\R$.

The same proof in the previous example shows that the only quasicompact subsets of $X$ are the finite sets.  Let $f:X\rightarrow \R$ be the Dirichlet function.  As $f^{-1}(-1)=\Q ^{\comp}$ is not closed, $f$ is not continuous.  We construct a net of functions in $\Mor _{\Top}(X,\R )$ converging to $f$ uniformly on each quasicompact (i.e.~each finite) subset of $X$.

Our directed set $\Lambda$ is the collection of all finite subsets of $X$ ordered by inclusion.  For $S\subseteq \Lambda$, let us write $S=\{ x_1,\ldots ,x_m\}$ with $x_k<x_{k+1}$ and define
\begin{equation}
f_S(x)\coloneqq \begin{cases}f(x) & \text{if }x\in S \\ f(x_1) & \text{if }x<x_1 \\ \frac{f(x_{k+1}-f(x_k)}{x_{k+1}-x_k}(x-x_k) & \text{for }x_k<x<x_{k+1} \\ f(x_m) & \text{if }x_m<x\end{cases},
\end{equation}
That it, is is a constant $f(x_1)$ for all $x\leq x_1$, and similarly for $x\geq x_m$.  For $x$ between $x_k$ and $x_{k+1}$, is it a just the line segment going from $f(x_k)$ at $x=x_k$ to $f(x_{k+1})$ at $x=x_{k+1}$.  By construction, this is continuous with respect to the usual topology, and hence continuous with respect to the cocountable extension topology.
\end{exm}
The key result of this subsection is that $\Mor _{\Top}(X,\R )$ is complete for $X$ quasicompactly-generated.  If turns out that the converse of this is \emph{false}.
\begin{exm}[A space for which $\Mor _{\Top}(X,\R )$ is complete yet is not quasicompactly-generated]
Take $X\coloneqq \R$ equipped with the cocountable topology.  We show that every element of $\Mor _{\Top}(X,\R )$ is constant.

Let $f:X\rightarrow \R$ be continuous.  If $f$ is not constant, then $f^{-1}(a)$ is proper for every $a\in \R$, and hence countable.  The image cannot be countable then, because if it were, $\R$ would be a countable union of countable sets, and hence countable.  Furthermore, if the image were contained in a \emph{proper} closed subset of $\R$, then by taking the preimage, we would have a countable set that is the uncountable union of nonempty disjoint sets.  Thus, the image must be uncountable with closure all of $\R$.  Thus, every open interval of $\R$ intersects the image.  In particular, as $\R =\bigcup _{m\in \Z}[m,m+1]$, some interval $[m,m+1]$ must intersect the image at uncountable many points.  But then,
\begin{equation}
\text{countable set}=f^{-1}([m,m+1])=\bigcup _{a\in f(X)\cap [m,m+1]}f^{-1}(a)
\end{equation}
is an uncountable union of disjoint sets--- a contradiction.  Therefore, $f$ is constant.

A cauchy net of constant functions amounts to a cauchy net of real numbers, which converges, and so the original cauchy net of constant functions converges to this constant function.  Therefore, $\Mor _{\Top}(X,\R )$ is complete.
\end{exm}

\subsection{The completion}

And now we show that every uniform space can be \emph{completed}.\footnote{Of course, we already know that $\Q$ is not complete (see \cref{prp3.3.59,prp3.3.68}), and so in general there will most certainly be some `completing' to be done.}
\begin{thm}[Completion]\label{Completion}
\begin{savenotes}
Let $X$ be a uniform space.  Then, there exists a complete uniform space $\Cmp (X)$, the \emph{completion}\index{Cauchy completion}\index{Completion (of a uniform space)} of $X$, such that
\begin{enumerate}
\item $\Cmp (X)$ contains $X$; and
\item if $Y$ is any other complete uniform space which contains $X$, then $Y$ contains $\Cmp (X)$.
\end{enumerate}
Furthermore,
\begin{enumerate}
\item $\Cmp (X)$ is unique up to uniform homeomorphism;\footnote{This of course justifies our use of the notation $\Cmp (X)$.} and
\item $X$ is dense in $\Cmp (X)$.
\end{enumerate}
\begin{rmk}
You will see in the proof that this is the ``cauchy sequence construction'' we mentioned in a footnote right before our proof of existence of the real numbers (\cref{RealNumbers}).  It turns out that, in the case of $\R$, this will gives us the right answer, but that the passage from $\Q$ to $\R$ should really be thought of as the \emph{dedekind completion}, not the \emph{cauchy completion}.
\end{rmk}
\begin{proof}
By replacing $X$ with $\TZero (X)$ if necessary, we may without loss of generality assume that $X$ is $T_0$.

\Step{Define an equivalence relation $\sim$ on cauchy nets}\label{stpB.5.6.1}
Define first as a set
\begin{equation}
X'\coloneqq \left\{ \lambda \mapsto x_\lambda \in X:\lambda \mapsto x_\lambda \text{ is cauchy.}\right\} 
\end{equation}
For $\lambda \mapsto x_\lambda$ and $\mu \mapsto y_\mu$ cauchy nets, define $\lambda \mapsto x_\lambda \sim \mu \mapsto y_\mu$ iff open subsets of $X$ eventually contain $\lambda \mapsto x_\lambda$ iff they eventually contain $\mu \mapsto y_\mu$.

\Step{Show that $\sim$ is an equivalence relation}
That $\lambda \mapsto x_\lambda \sim \lambda \mapsto x_\lambda $ is tautological.  The definition of $\sim$ is $\lambda \mapsto x_\lambda \leftrightarrow \mu \mapsto y_\mu$ symmetric, and so of course $\lambda \mapsto x_\lambda \sim \mu \mapsto y_\mu $ implies $\mu \mapsto y_\mu \sim \lambda \mapsto x_\lambda$.

As for transitivity, suppose that $\lambda \mapsto x_\lambda \sim \mu \mapsto y_\mu $ and $\mu \mapsto y_\mu \sim \nu \mapsto z_\nu$.  We wish to show that $\lambda \mapsto x_\lambda \sim \nu \mapsto z_\nu $.  Let $U\subseteq X$ be open.  We must show that $U$ eventually contains $\lambda \mapsto x_\lambda$ iff it eventualy contains $\nu \mapsto z_\nu$.  By $\lambda \mapsto x_\lambda \leftrightarrow \nu \mapsto z_\nu$, it suffices to show only one of these directions.  So, suppose that $U$ eventually contains $\lambda \mapsto x_\lambda$.  Then, $U$ eventually contains $\mu \mapsto y_\mu$ (because $\lambda \mapsto x_\lambda \sim \mu \mapsto y_\mu$), and so $U$ eventually contains $\nu \mapsto z_\nu$ (because $\mu \mapsto y_\mu \sim \nu \mapsto z_\nu$).

\Step{Define $\Cmp (X)$ as a set}
We define
\begin{equation}
\Cmp (X)\coloneqq X'/\sim .
\end{equation}

\Step{Define a uniformity on $X'$.}
Denote the uniformity on $X$ by $\widetilde{\mathcal{U}}$.  For every uniform cover $\mathcal{U}\in \widetilde{\mathcal{U}}$, we define a corresponding cover $\mathcal{U}'$ of $X'$.  For $\mathcal{U}\in \widetilde{\mathcal{U}}$ and $U\in \mathcal{U}$, define
\begin{equation}\label{B.26}
\begin{split}
U' & \coloneqq \left\{ x\in X':x\text{ is eventually contained in }U\text{.}\right\} \\
\mathcal{U}' & \coloneqq \{ U':U\in \mathcal{U}\} \\
\widetilde{\mathcal{U}}' & \coloneqq \{ \mathcal{U}':\mathcal{U}\in \widetilde{\mathcal{U}}\} .
\end{split}
\end{equation}
We wish to show that $\widetilde{\mathcal{U}}'$ is a uniform base on $X'$.\footnote{The ``B'' is of course to remind us that we don't know this to be a uniformity per se, only a uniform base.}  This will then define a topology and compatible uniformity for each each $\mathcal{U}'$ is a uniform cover by \cref{UniformTopology} and \cref{prp4.3.2} respectively.  To show this, we must show (i) that each element of $\widetilde{\mathcal{U}}'$ is in fact a cover of $X'$ and (ii) that $\widetilde{\mathcal{U}}'$ is downward-directed with respect to star-refinement.

We first check that each $\mathcal{U}'$ is in fact a cover of $X'$.  So, let $\mathcal{U}'\in \widetilde{\mathcal{U}}'$ be arbitrary and let $x\in X'$.  As $x$ is cauchy, there is some $U\in \mathcal{U}$ such that $x$ is eventually contained in $U$, so that $x\in U'$, and hence $\mathcal{U}'$ covers $X'$.

We now check that $\widetilde{\mathcal{U}}'$ is downward-directed with respect to star-refinement.  So, let $\mathcal{U}',\mathcal{V}'\in \widetilde{\mathcal{U}}'$.  Let $\mathcal{W}$ be a common star-refinement of $\mathcal{U}$ and $\mathcal{V}$.  We wish to show that $\mathcal{W}'$ is a common star-refinement of $\mathcal{U}'$ and $V'$.  By $\mathcal{U}\leftrightarrow \mathcal{V}$ symmetry, it suffices to prove that $\mathcal{W}'$ is a star-refinement of $\mathcal{U}'$.  So, let $W_0'\in \mathcal{W}'$.  As $\mathcal{W}$ is a star-refinement of $\mathcal{U}$, it follows that there is some $U_0\in \mathcal{U}$ such that $\Star _{\mathcal{W}}(W_0)\subseteq U_0$.  We wish to show that $\Star _{\mathcal{W}'}(W_0')\subseteq U_0'$.  So, let $W'\in \mathcal{W}'$ intersect $W_0'$.  Then, there is a net that is eventually contained in both $W$ and $W_0$, so that, in particular, $W$ and $W_0$ intersect.  It follows that $W\subseteq U_0$, and hence in turn, that any net eventually contained in $W$ is eventually contained in $U_0$, so that $W'\subseteq U_0'$, and hence $\Star _{\mathcal{W}'}(W_0')\subseteq U_0'$ as desired.

This completes the proof that $\widetilde{\mathcal{U}}'$ is a uniform base on $X'$.

\Step{Show that the quotient map $\q :X'\rightarrow \Cmp (X)$ satisfies $\q ^{-1}(\q (U'))=U'$}\label{Completion.5}
As it is always the case that $U'\subseteq \q ^{-1}(\q (U'))$, it suffices to show that $\q ^{-1}(\q (U'))\subseteq S$.  So, let $x\in \q ^{-1}(\q (U'))$.  Then, $x$ is a cauchy net and there is another cauchy net $y\in U'$ such that $x\sim y$.  That is, open sets eventually contain $x$ iff they eventually contain $y$.  However, as $y\in U'$, by definition of $U'$ (\eqref{B.26}), $y$ is eventually contained in $U$, and so $x$ is eventually contained in $U$, and so $x\in U'$.  Hence, $\q ^{-1}(\q (U'))\subseteq U'$, and we are done.

\Step{Define a uniform base on $\Cmp (X)$}
For every uniform cover $\mathcal{U}\in \widetilde{\mathcal{U}}$, we define a corresponding cover $\Cmp (\mathcal{U})$ of $\Cmp (X)$.  For $\mathcal{U}\in \widetilde{\mathcal{U}}$ and $U\in \mathcal{U}$, define\footnote{As you might have guessed, this is just the quotient uniformity induced from the one on $X'$ written out exlicitly.}
\begin{equation}
\begin{split}
\Cmp (U) & \coloneqq \q (U') \\
\Cmp (\mathcal{U}) & \coloneqq \q (\mathcal{U}')\coloneqq \{ \Cmp (U):U\in \mathcal{U}\} \\
\widetilde{\Cmp (\mathcal{U})} & \coloneqq \q (\widetilde{\mathcal{U}}') \coloneqq \{ \Cmp (\mathcal{U}):\mathcal{U}\in \widetilde{\mathcal{U}}\} .
\end{split}
\end{equation}
We claim that $\widetilde{\Cmp (\mathcal{U})}$ is a uniform base on $\Cmp (X)$.  Certainly each $\Cmp (\mathcal{U})$ is a cover of $\Cmp (X)$ because $\mathcal{U}'$ is a cover of $X'$ and $\q$ is surjective.

It follows from \cref{prp4.2.17x} that $\q$ preserves star-refinement (the purpose of the previous step was to verify the requisite hypotheses of \cref{prp4.2.17x}), and so that $\Cmp (\widetilde{\mathcal{U}})$ is a uniform base follows from the fact that that $\widetilde{\mathcal{U}}'$ was a uniform base on $X'$.

\Step{Show that $\Cmp (X)$ is complete.}
Let $\lambda \mapsto \q (x^\lambda )$ be cauchy in $\Cmp (X)$.  By definition of cauchyness and our uniformity on $\Cmp (X)$, this means that, for every uniform cover $\Cmp (\mathcal{U})\in \Cmp (\widetilde{\mathcal{U}})$, there is some $\Cmp (U)\in \Cmp (\mathcal{U})$ such that $\lambda \mapsto \q (x^\lambda )$ is eventually contained in $\Cmp (U)\coloneqq \q (U')$.  Thus, for each $x^\lambda \in X'$ for $\lambda$ sufficiently large, there is some $y^\lambda \in U'$ with $x^\lambda \sim y^\lambda$.  However, by definition of $U'$, $U$ eventually contains $y^\lambda$, and hence, because $x^\lambda \sim y^\lambda$, eventually contains $x^\lambda$, and so in fact $x^\lambda \in U'$.  Thus, we have a net $\lambda \mapsto x^\lambda \in X'$ that has the property that, for every uniform cover $\mathcal{U}\in \widetilde{\mathcal{U}}$, there is some $U\in \mathcal{U}$ such that $\lambda \mapsto x^\lambda$ is eventually contained in $U'$.

Let us denote the domain of $\lambda \mapsto x^\lambda$ by $\Lambda$.  Now, each $x^\lambda$ is itself a net, and so let us denote its domain by $M^\lambda$.  Define $x^\infty \in X'$ to be the net
\begin{equation}
\Lambda \times \prod _{\lambda \in \Lambda}M^\lambda \ni (\lambda ,\mu )\mapsto (x^\lambda )_{\mu ^\lambda}\in X.
\end{equation}
We first must check that this is cauchy, so that indeed $x^\infty \in X'$.

So, let $\mathcal{U}\in \widetilde{\mathcal{U}}$ be a uniform cover.  Then, there is some $U\in \mathcal{U}$ such that $\lambda \mapsto x^\lambda$ is eventually contained in $U'$.  So, let $\lambda _0$ be such that, whenever $\lambda \geq \lambda _0$, it follows that $x^\lambda \in U'$.  For all such $\lambda$, the net $x^\lambda$ itself must be eventually contained in $U$, so let $\mu _0^\lambda$ be such that, whenever $\mu ^\lambda \geq \mu _0^\lambda$, it follows that $(x^\lambda )_{\mu ^\lambda}$.  (For $\lambda$ not at least $\lambda _0$, $\mu _0^\lambda$ may be anything).  Now, suppose that $(\lambda ,\mu )\geq (\lambda _0,\mu _0)$.  Then,
\begin{equation}
(x^\lambda )_{\mu ^\lambda}\in U
\end{equation}
Thus, $(\lambda ,\mu )\mapsto (x^\lambda )_{\mu ^\lambda}$ is cauchy.

We now show that $\lambda \mapsto \q (x^\lambda )$ converges to $\q (x^\infty )$.  To show this, as stars form neighborhood bases, it suffices to show that $\lambda \mapsto \q (x^\lambda )$ is eventually contained in $\Star _{\Cmp (\mathcal{U})}(\q (x^\infty ))$ for all $\mathcal{U}\in \widetilde{\mathcal{U}}$.  As $\q$ is surjective, the preimage of a star is equal to the star of the preimage, and so it suffices to show that $\lambda \mapsto x^\lambda$ is eventually contained in $\Star _{\mathcal{U}'}(x^\infty )$ for all $\mathcal{U}\in \widetilde{\mathcal{U}}$.  So, let $\mathcal{U}\in \widetilde{\mathcal{U}}$ be a uniform cover.  Then, there is some $U\in \mathcal{U}$ such that $\lambda \mapsto x^\lambda$ is eventually contained in $U'$.  Thus, we will be done if we can show that $x^\infty \in U'$ (so that then $U'\subseteq \Star _{\mathcal{U}'}(x^\infty )$.  To show this, we must show that $x^\infty$ is eventually contained in $U$.  However, the exact same argument that was used above to show that $x^\infty$ was cauchy shows precisely this.  Therefore, $\lambda \mapsto x^\lambda$ converges to $x^\infty$.

\Step{Show that $\Cmp (X)$ contains $X$}
Of course, when we say that $\Cmp (X)$ ``contains'' $X$, what we really means is that there is a subset of $\Cmp (X)$ which is uniformly-homeomorphic to $X$.  So, we define $\iota :X\rightarrow \Cmp (X)$, and show that it is a uniform-homeomorphism onto its image.\footnote{Its image will be equipped with the subspace uniformity, that is, the initial uniformity (see \cref{InitialUniformity}) with respect to the inclusion into $\Cmp (X)$.}

Define $\c :X\rightarrow X'$ by $\c (x)\coloneqq (\lambda \mapsto x_\lambda \coloneqq x)$, that is, $\c$ sends $x$ to the constant net with value $x$.  (Constant nets converge, and hence are in particular cauchy.)  Then we define $\iota \coloneqq \q \circ \c$.  We first show that this is injective.  Suppose that $\iota (x_1)=\iota (x_2)$.  Then, every neighborhood that eventually contains the constant net $x_1$ eventually contains the constant net $x_2$.  In other words, open sets in $X$ contain $x_1$ iff they contain $x_2$, which implies that $x_1=x_2$ because $X$ is $T_0$.

We now check that $\iota$ is uniformly-continuous.  We claim that $\iota ^{-1}(\Cmp (\mathcal{U})=\mathcal{U}$.  However, using result that $\q ^{-1}(\q (U'))=U'$ from \cref{Completion.5}, we have that
\begin{equation}
\iota ^{-1}(\Cmp (\mathcal{U}))=\c ^{-1}\left( \q ^{-1}\left( \q (\mathcal{U}')\right) \right) =\c ^{-1}(\mathcal{U}')=\{ \c ^{-1}(U'):U\in \mathcal{U}\} .
\end{equation}
However, the only constant nets which are eventually contained in $U$ are the elements of $U$ themselves, and so $\c ^{-1}(U')=U$, and so indeed
\begin{equation}
\iota ^{-1}(\Cmp (\mathcal{U}))=\mathcal{U}.
\end{equation}
The subspace uniformity induced on $\iota (X)$ is simply
\begin{equation}\label{4.5.13}
\left\{ \Cmp (\mathcal{U})\wedge \{ \iota (X)\} :\mathcal{U}\in \widetilde{\mathcal{U}}\right\} ,
\end{equation}
that is, a cover of $\iota (X)$ is uniform iff it is a cover that is obtained from a uniform cover of $\Cmp (X)$ by simply restricting that cover to $\iota (X)$.  Because the preimage of a cover with respect its inverse is the same as the image of that uniform cover, to show that the inverse of $\iota :X\rightarrow \iota (X)$ is uniformly-continuous, it suffices to show that $\iota (\mathcal{U})$ is a uniform cover on $\iota (X)$.  By \eqref{4.5.13}, it thus suffices to show that
\begin{equation}\label{4.5.14}
\iota (\mathcal{U})=\Cmp (\mathcal{U})\wedge \{ \iota (X)\} .
\end{equation}
However,
\begin{equation}
\begin{split}
\Cmp (\mathcal{U}) \wedge \{ \iota (X)\} & =\left\{ \q (U')\cap \q (\c (X)):U\in \mathcal{U}\right\}  =\footnote{Careful:  $f(X)\cap f(Y)=f(X\cap Y)$ does not hold in general.  It does, however, hold for $\q$ because $\q$ is surjective and satisfies $\q ^{-1}(\q (U))=U$.}\left\{ \q (U'\cap \c (X)):U\in \mathcal{U}\right\} \\
& =\left\{ \iota (U):U\in \mathcal{U}\right\} ,
\end{split}
\end{equation}
which demonstrates the truth of \eqref{4.5.14}.

\Step{Show that any other complete uniform space that contains $X$ contains $\Cmp (X)$}
Let $Y$ be some other complete uniform space that contains $X$.  Let $x\in X'$ be a cauchy net in $X$ and let $x_\infty$ be its (unique) limit in $Y$.  Define $\kappa :\Cmp (X)\rightarrow Y$ by $\kappa (\q (x)) \coloneqq x_\infty$.
\begin{exr}
Show that $\kappa$ is well-defined and is a uniform-homeomorphism onto its image.
\end{exr}

\Step{Show that $\Cmp (X)$ is unique}
Let $Y$ be another complete uniform space which (i) contains $X$ and (ii) is contained in every other complete uniform space that contains $X$.  From this, we know that $Y$ is contained in $\Cmp (X)$.  On the other hand, we already knew that $\Cmp (X)$ was contained in $Y$.  Therefore, $\Cmp (X)=Y$.

\Step{Show that $X$ is dense in $\Cmp (X)$}
Let $\q (x)\in \Cmp (X)$.
\begin{exr}
Show that $\lambda \mapsto \iota (x_\lambda )$ converges to $\q (x)$ in $\Cmp (X)$.
\end{exr}
It follows from this exercise that $\Cls (\iota (X))=\Cmp (X)$, and so that $X$ is dense in $\Cmp (X)$.
\end{proof}
\end{savenotes}
\end{thm}
One thing that we will frequently want to do is extend a given function to its completion.  If the codomain is complete as well, we can do this, and in fact, we can do it whenever we have a continuous function defined on a dense subspace.
\begin{prp}
\begin{savenotes}
Let $S\subseteq X$ be a dense subset of a topological space, let $Y$ be a complete $T_0$ uniform space, and let $f:S\rightarrow Y$ be uniformly-continuous.  Then, there exists a unique uniformly-continuous map $g:X\rightarrow Y$ such that $\restr{g}{S}=f$.
\begin{rmk}
Mere continuity does not suffice, even in the nicest of cases.  See the counter-example below (\cref{exm4.5.20x}).
\end{rmk}
\begin{proof}
Let $x\in X$.  As $\Cls (S)=X$, there is a net $\lambda \mapsto x_\lambda \in X$ converging to $x$.  Pick any such net\footnote{Our proof of uniqueness will show that this choice ultimately does not matter.}.  By \cref{exr4.5.3},  $\lambda \mapsto f(x_\lambda )$ is cauchy in $Y$, so that we may simply take its limit (because $Y$ is complete and is $T_0$, and hence $T_2$, so that limits are unique---see \cref{prp4.5.37}).  So, let us define $g:X\rightarrow Y$ by
\begin{equation}
g(x)\coloneqq \lim _\lambda f(x_\lambda ).
\end{equation}
\begin{exr}
Show that if $\mu \mapsto y_\mu$ also converges to $x$, then $\lim _\lambda f(x_\lambda )=\lim _\mu f(x_\mu )$.
\end{exr}
This exercises established uniqueness.  Thus, the choice of net does not matter, and so for $x\in S$, we may simply take the constant net $\lambda \mapsto x_\lambda \coloneqq x$, so that $g$ is indeed an extension of $f$.
\begin{exr}
Show that $g$ is uniformly-continuous.
\end{exr}
\end{proof}
\end{savenotes}
\end{prp}
\begin{exm}[A real-valued continuous function on a dense subspace that does \emph{not} extend]\label{exm4.5.20x}
In the notation of the previous proposition, take $S\coloneqq \R$, $X\coloneqq [-\infty ,+\infty]$, $Y\coloneqq \R$, and $f\coloneqq \id _{\R}$.  If this had an extension to all of $X$, then in particular the sequence $m\mapsto x_m\coloneqq m$ would have to converge in $\R$.
\begin{rmk}
In fact, if perhaps you thought you could make use of the fact that continuous functions restricted to quasicompact subsets are uniformly-continuous (\cref{prp4.2.73}) to prove the result in special cases, this even provides a counter-example in which every point of $X$ has a compact neighborhood.
\end{rmk}
\end{exm}
Among other things, the significance of this result is that group operations of topological groups extend uniquely to their completions.

Having shown that uniform spaces always have completions, finally, we may return to an unresolved issue all the way back from \cref{chp1}.
\begin{exm}[A nonzero totally-ordered cauchy-complete field distinct from $\R$]
Recall the field of rational functions with coefficients in the reals, $\R (x)$, from \cref{exm2.3.12}.  Being a totally-ordered field, it is in particular a topological group (\cref{exr4.8.58}) with respect to its underlying commutative group $\coord{\R (x),+,0,-}$, and so has a canonical uniform structure.  Thus, we may complete to form the complete topological field $\Cmp (\R (x))$.
\begin{exr}
Extend the order on $\R (x)$ to $\Cmp (\R (x))$ so that $\Cmp (\R (x))$ is a totally-ordered field containing $\R (x)$.
\end{exr}
By construction then, $\Cmp (\R (x))$ is a nonzero totally-ordered cauchy-complete field.  Not only is it distinct from $\R$ it cannot even embed in $\R$ as, if it did, then so to $\R (x)$ would embed in $\R$ (as it embeds in $\Cmp (\R (x))$), and hence $\R (x)$ would be archimedean---a contradiction of \cref{exm2.3.12}.
\end{exm}

In conclusion:
\begin{textequation}
If a space is quasicompactly-generated, then $\Mor _{\Top}(X,\R )$ is complete.  Furthermore, if $\Mor _{\Top}(X,\R )$ is complete, then $X$ is a topological space.\footnote{Uhm, duh.  The content here is not in the implication, but rather in the counter-example.}  Both of these implications are strict:  the reals with the cocountable topology show that the first implication is strict, and the reals with the cocountable extension topology show that the second implication is strict.
\end{textequation}

\subsection{Complete metric spaces}

We present here two important results that are specific to complete metric spaces.

\subsubsection{The Baire Category Theorem}

The Baire Category Theorem is an important result concerning complete \emph{metric} spaces.  It has many important applications, most of which we haven't the time to present.  One important application for us, however, is that it will allow us to finally wrap up the separation axiom counter-examples (see \cref{NiemytzkisTangentDiskTopology}) below.
\begin{thm}[Baire Category Theorem]\label{BaireCategoryTheorem}
Let $X$ be a complete metric space.  Then,
\begin{enumerate}
\item the countable intersection of open dense subsets of $X$ is dense; and
\item the countable union of closed sets with empty interior has empty interior.
\end{enumerate}
\begin{rmk}
These conclusions are equivalent, the equivalence being obtained by taking the complement of the conclusion.  The former is arguably a bit more intuitive to prove, while the latter the form that is probably more frequently used in concrete situations.
\end{rmk}
\begin{rmk}
Warning:  This is \emph{false} in general for complete uniform spaces---see \cref{exm4.5.2x}.
\end{rmk}
\begin{rmk}
The word ``category'' in the name has nothing to do with categories---the terminology it refers to is archaic.
\end{rmk}
\begin{proof}
For $m\in \N$, let $U_m\subseteq X$ be an open dense subset.  We wish to show that
\begin{equation}
U\coloneqq \bigcap _{m\in \N}U_m
\end{equation}
is dense.  The definition of density is that the closure is equal to all of $X$, in other words, that every point of $X$ is an accumulation point, or in other words, that every open subset intersects the dense set.  So, let $V\subseteq X$ be open.  We wish to show that $V$ intersects $U$.

As $U_0$ is dense, $V$ intersects $U_0$, say at $x_0\in U_0\cap V$.  As $U_0\cap V$ is open.  We can fit an $\varepsilon$-ball around $x_0$ inside $U_0\cap V$.  In fact, as $X$ is perfectly-$T_4$, and in particular $T_3$, we can find some $\varepsilon _0>0$ such that
\begin{equation}
\Cls \left( B_{\varepsilon _0}(x_0)\right) \subseteq U_0\cap V.
\end{equation}
In fact, by making $\varepsilon _0$ smaller if necessary, we may without loss of generality assume that $\varepsilon _0<2^{-0}=1$.

Now, because $U_1$ is dense, there is some $x_1\in U_1\cap B_{\varepsilon _0}(x_0)$, and so, just the same as before, there is some $0<\varepsilon <1<2^{-1}$ such that
\begin{equation}
\Cls \left( B_{\varepsilon _1}(x_1)\right) \subseteq U_1\cap B_{\varepsilon _0}(x_0).
\end{equation}
Proceeding inductively, we can find $x_m\in X$ and $0<\varepsilon _m<2^{-m}$ such that
\begin{equation}
\Cls \left( B_{\varepsilon _m}(x_m)\right) \subseteq U_{m-1}\cap B_{\varepsilon _{m-1}}(x_{m-1}).
\end{equation}

We now check that $m\mapsto x_m$ is cauchy.  Let $\varepsilon >0$.  Choose $m\in \N$ such that $\varepsilon _m<\varepsilon$.  Suppose that $n\geq m$.  Then, $x_n\in B_{\varepsilon _n}(x_n)\subseteq B_{\varepsilon _m}(x_m)\subseteq B_{\varepsilon}(x_m)$.  Thus, $m\mapsto x_m$ is eventually contained in some $\varepsilon$ ball, and is hence cauchy.  As $X$ is complete, it converges (to a unique limit) and so we may define
\begin{equation}
x_\infty \coloneqq \lim _mx_m.
\end{equation}

We claim that $x\in U\cap V$.  As explained above, this will complete the proof.  $m\mapsto x_m$ is eventually contained in $\Cls (B_{\varepsilon _0}(x_0))\subseteq V$, and so $x_\infty \in \Cls (B_{\varepsilon _0}(x_0))\subseteq V$.  Similarly, $m\mapsto x_m$ is eventually contained in $\Cls \left( B_{\varepsilon _m}(x_m)\right) \subseteq U_{m_1}$, and so, same as before, $x_\infty \in U_{m_1}$.  Hence, $x_\infty \in U$, and we are done.
\end{proof}
\end{thm}

\begin{exm}[A complete uniform space which is not a baire space]\begin{savenotes}\footnote{A \emph{baire space}\index{Baire space} is a topological space in which the conclusion of the \nameref{BaireCategoryTheorem} holds.  This example was inspired by \href{http://mathoverflow.net/questions/212308/baire-category-theorem-for-complete-uniform-spaces}{priel's answer} on mathoverflow.net.  Thanks to Nate Eldredge for nudging me towards the correct proof.}\label{exm4.5.2x}
Define
\begin{equation}
X\coloneqq \{ f:\Z ^+ \rightarrow [0,1] :f(m)=0\text{ for all but finitely many }m\in \N \text{.}\} 
\end{equation}
We define a uniformity on $X$ as follows.  First of all, for $m\in \Z ^+$, define
\begin{equation}
X_m\coloneqq \underbrace{[0,1] \times \cdots \times [0,1]}_{m}
\end{equation}
equipped with the product uniformity.  Note that $X_m$ embeds in $X$ via $\iota _m:X_m\rightarrow X$ defined by
\begin{equation}
\iota _m(\coord{x_1,\ldots ,x_m})\coloneqq k\mapsto \begin{cases}x_k & \text{if }k\leq m \\ 0 & \text{otherwise}\end{cases},
\end{equation}
that is, $\coord{x_1,\ldots ,x_m}$ is sent to the function from $\Z ^+$ into $[0,1]$ which sends $k$ to $x_k$ for $k\leq m$ and $0$ otherwise.  We then equip $X$ with the final uniformity with respect to the collection $\{ \iota _m:m\in \Z ^+\}$.

We first wish to show that $\Lambda \ni \lambda \mapsto x_\lambda$ is eventually contained in $X_{m_0}$ for some $m_0\in \Z ^+$.  To show this, we proceed by contradiction:  suppose that for every $m\in \N$ and every $\lambda$ there is some $\lambda _{m,\lambda}\geq \lambda$ such that $x_{m,\lambda}\notin X_m$.  Define $\Lambda '\coloneqq \Z ^+\times \Lambda$.  Then, $\Lambda '\ni \coord{m,\lambda}\mapsto x_{\lambda _{m,\lambda}}$ is a subnet of $\lambda \mapsto x_\lambda$, and hence is in turn cauchy.  Thus, for every $\varepsilon _1,\varepsilon _2,\varepsilon _3,\ldots >0$, there is some $\coord{m_0,\lambda _0}$ such that, whenever $\coord{m,\lambda},\coord{n,\mu}\geq \coord{m_0,\lambda _0}$, it follows that
\begin{equation}
\abs{x_{\lambda _{m,\lambda}}(k)-x_{\lambda _{n,\mu}}(k)}<\varepsilon _k
\end{equation}
for all $k\in \Z ^+$.  As $x_{\lambda _{m_0,\lambda _0}}(k)=0$ for all $k$ sufficiently large, we find that
\begin{equation}
x_{\lambda _{n,\mu}}(k)<\varepsilon _k
\end{equation}
for all $k$ sufficiently large, say for $k\geq k_0$, and all $\coord{n,\mu}\geq \coord{m_0,\lambda _0}$.  Take $n\coloneqq \max \{ k_0,m_0\}$.  As $x_{\lambda _{n,\mu}}\notin X_n$, there is some $l>n\geq k_0$ such that $x_{\lambda _{n,\mu}}(l)\neq 0$.  Then,
\begin{equation}
0<x_{\lambda _{n,\mu}}(l)<\varepsilon _l.
\end{equation}
As $\varepsilon _l$ is arbitrary, this is a contradiction.  Thus, there is some $m_0\in \Z ^+$ such that $\lambda \mapsto x_\lambda$ is eventually contained in $X_{m_0}$.

However, $X_{m_0}$, being a finite product of compact metric spaces, is a compact metric space, and hence complete, so that $\lambda \mapsto x_\lambda$ converges in $X_{m_0}$, and hence in $X$.  Therefore, $X$ is complete.

We now wish to check that $X$ is not a baire space.  From the definition, we have that
\begin{equation}
X=\bigcup _{m\in \Z ^+}X_m.
\end{equation}
\begin{exr}
Show that $X_m$ is closed in $X$.
\end{exr}
Thus, to show that $X$ is not a baire space, it suffices to show that each $X_m$ has empty interior.  So, let $x\in X_m$.  We show that every open neighborhood around $x$ contains an element of $X_{m+1}$ that is not contained in $X_m$.  Let us write
\begin{equation}
x=\coord{x_1,x_2,\ldots ,x_m,0,0,0,\ldots}
\end{equation}
Then, for every neighborhood $U$ of $x$, there will be some $\varepsilon _0>0$ sufficiently small so that
\begin{equation}
x=\coord{x_1,x_2,\ldots ,x_m,\varepsilon _0,0,0,\ldots}\in U,
\end{equation}
so that $x$ is not in the interior of $X_m$, so that each $X_m$ has empty interior.
\end{savenotes}
\end{exm}

Finally, we are able to tie-up our one loose end with the separation axioms.
\begin{exm}[A space that is perfectly-$T_3$ but not $T_4$]\footnote{This example comes from \cite[pg.~100]{Steen}.}\label{NiemytzkisTangentDiskTopology}
Define $X\coloneqq \R \times \R _0^+$, the upper-half plane, and define a base for a topology on $X$ by
\begin{equation}
\begin{split}
\mathcal{B} & \coloneqq \left\{ B_\varepsilon (\coord{x,y})\subseteq \R \times \R ^+:\varepsilon >0,\ \coord{x,y}\in \R \times \R ^+\right\} \\
& \qquad \cup \left\{ B_\varepsilon (\coord{x,\varepsilon})\cup \{ \coord{x,0}\} :x\in \R ,\ \varepsilon >0\right\} .
\end{split}
\end{equation}
That is, we take all $\varepsilon$-balls contained in the (strict) upper half-plane together with all $\varepsilon$-balls in the upper half-plane which are `tangent' to the $x$-axis (together with the point of tangency).

We first check that $X$ is perfectly-$T_3$.  So, let $C\subseteq X$ be closed and let $\coord{x_0,y_0}\in C^{\comp}$.  The subspace topology of $\R ^+\times \R ^+\subseteq X$ is just the usual topology, which, being a metric space, is perfectly-$T_4$.  Thus, we only need to check the case where $y_0=0$.  Let $\varepsilon >0$ be such that $B_{\varepsilon}(\coord{x_0,\varepsilon})$ is disjoint from $C$.  Then define $f:X\rightarrow [0,1]$ by
\begin{equation}
f(\coord{x,y})\coloneqq \begin{cases}0 & \text{if }\coord{x,y}=\coord{x_0,0} \\ 1 & \text{if }\coord{x,y}\notin B_{\varepsilon}(\coord{x_0,\varepsilon})\cup \{ \coord{x_0,0}\} \\ \frac{(x-x_0)^2+y^2}{2\varepsilon y}\end{cases}.
\end{equation}
For $0<a<1$, $f^{-1}([0,a)]$ is $B_{\varepsilon a}(\coord{x,\varepsilon a})\cup \{ \coord{x,0}\}$, hence open; $f^{-1}((a,1])$ is $X\setminus \Cls (B_{\varepsilon a}(\coord{x,\varepsilon a}))$, hence open; and so $f^{-1}(a,b)=f^{-1}((a,1])\cap f^{-1}([0,b))$ is open.

We now check that $X$ is not $T_4$.  To do this, we show that $\Q ,\R \setminus \Q \subseteq \R \times \R _0^+$ (as subsets of the $x$-axis) are closed and cannot be separated by neighborhoods.\footnote{We write $\Q$ and $\R \setminus \Q$ instead of $\Q \times \{ 0\}$ and $(\R \setminus \Q )\times \{ 0\}$.}  In fact, we check that every subset $S\subseteq \R \times \{ 0\}$ is closed.  To show that, we show that $S^{\comp}$ is open.  For $\coord{x,y}\in S^{\comp}$, if $y>0$, then certainly we can put an $\varepsilon$-ball around it that does not intersect the $x$-axis, and hence does not intersection $S$.  On the other hand, for $\coord{x,0}\in S^{\comp}$, $B_{\varepsilon}(\coord{x,\varepsilon})\cup \{ \coord{x,0}\}$ intersects the $x$-axis only at $\coord{x,0}$, and so does not intersect $S$.  Therefore, $S^{\comp}$ is open, and so $S$ is closed.

We now check that $\Q ^{\comp}$ and $(\R \setminus \Q )^{\comp}$ cannot be separated by neighborhoods.\footnote{Note that we cannot write $\Q ^{\comp}$ to denote the irrationals as usual because, in this context, $\Q ^{\comp}$ means $(\R \times \R _0^+)\setminus \Q$.}  Let $U$ be an open neighborhood o $\R \setminus \Q$.  We show that there is some point $q_0\in \Q$ every neighborhood of which intersects $U$.

For $x\in \R \setminus \Q$, let $\varepsilon _x>0$ be such that
\begin{equation}
\R \setminus \Q \ni x\in B_{\varepsilon _x}(\coord{x,\varepsilon _x})\cup \{ x\} \subseteq U.
\end{equation}
For $m\in \Z ^+$, define
\begin{equation}
S_m\coloneqq \{ x\in \R \setminus \Q :\varepsilon _x>\tfrac{1}{m}\} .
\end{equation}
Then,
\begin{equation}
\R =\bigcup _{m\in \Z ^+}S_m\cup \bigcup _{x\in \Q}\{ x\},
\end{equation}
and so
\begin{equation}
\R =\bigcup _{m\in \Z ^+}\Cls _{\R}(S_m)\cup \bigcup _{x\in \Q}\{ x\} .\footnote{The subscript $\R$ here is to indicate that we are taking the closure with respect to the usual topology on $\R$.}
\end{equation}
As $\R$ is a complete metric space, by the \nameref{BaireCategoryTheorem}, there must be some $m_0\in \Z ^+$ such that $\Cls _{\R}(S_m)$ does \emph{not} have empty interior.  So, let $(a,b)\subseteq \Cls _{\R}(S_m)$ and let $\Q \ni q_0\in (a,b)$.  Then, for every $\varepsilon >0$, $(q_0-\varepsilon ,q_0+\varepsilon )$ intersects $S_m$, say at $x_\varepsilon$.  But then, for all $\varepsilon$ sufficiently small,
\begin{equation}
\coord{x_\varepsilon ,\varepsilon}\in B_{\varepsilon}(\coord{q_0,\varepsilon})\cap B_{\tfrac{1}{m}}(\coord{x_\varepsilon ,\tfrac{1}{m}})\subseteq B_{\varepsilon}(\coord{q_0,\varepsilon})\cap U.
\end{equation}
Thus, every neighborhood of $\coord{q_0,0}\in \Q$ intersects $U$.
\begin{rmk}
This is \emph{Niemytzki's Tangent Disk Topology}\index{Niemytzki's Tangent Disk Topology}.
\end{rmk}
\end{exm}

\subsubsection{Banach Fixed-Point Theorem}

We finish the chapter with an application of the \nameref{BaireCategoryTheorem} that will prove useful to us later when we study differentiation.
\begin{thm}[Banach Fixed-Point Theorem]\index{Banach Fixed-Point Theorem}\label{BanachFixedPointTheorem}
Let $\coord{X,\metric}$ be a metric space and let $f:X\rightarrow X$ be such that
\begin{equation}\label{4.5.70}
\metric[f(x_1)][f(x_2)]\leq M\metric[x_1][x_2]
\end{equation}
for some $0\leq 1<M$.  Then, there is \emph{at most one} point $x_0\in X$ such that $f(x_0)=x_0$.  If $X$ is nonempty and complete, then there is \emph{exactly one} $x_0\in X$ such that $f(x_0)=x_0$.
\begin{rmk}
Such an $x_0$ is called a \emph{fixed-point}, hence the name of the theorem.  Thus, the theorem tells us that (i) fixed-points, if they exist, have to be unique; and (ii) in the (nonempty) complete case, there has to be some fixed-point (and hence exactly one fixed-point).
\end{rmk}
\begin{rmk}
\eqref{4.5.70} is just the the statement that $f$ is lipschitz-continuous \emph{for a constant} $M<1$.  Such maps are called \emph{contraction-mapppings}, hence the alternative name for this theorem, the \emph{Contraction-Mapping Theorem}\index{Contraction-Mapping Theorem}.
\end{rmk}
\begin{proof}
Let $x_1,x_2\in X$ be two fixed points of $f$.  Then,
\begin{equation}
\metric[x_1][x_2]=\metric[f(x_1)][f(x_2)]\leq M\metric[x_1][x_2],
\end{equation}
and hence, if $\metric[x_1][x_2]\neq 0$, we would have (because $M<1$)
\begin{equation}
\metric[x_1][x_2]<\metric[x_1][x_2]:
\end{equation}
a contradiction.  Therefore, $\metric[x_1][x_2]=0$, and hence $x_1=x_2$.

Now take $X$ to be nonempty and complete.  We construct an actual fixed point of $f$.  Let $x_0\in X$ (there is such a point because $X$ is nonempty).  For $m\in \Z ^+$, define
\begin{equation}
x_m\coloneqq f(x_{m-1}).
\end{equation}
We wish to show that the sequence $m\mapsto x_m$ is cauchy.  If we can do so, then its limit $x_\infty$ must exist, and so by taking the limit of the previous equation, we would find that $x_\infty =f(x_\infty )$ ($f$ is lipschitz-continuous, hence uniformly-continuous, hence continuous).  Thus, it suffices to show that $m\mapsto x_m$ is cauchy

To see this, we first notice that\footnote{We're using $y$s instead of $x$s because those symbols are already used-up.}
\begin{equation}
\metric[y_1][y_2]\leq \metric[y_1][f(y_1)]+\metric[f(y_1)][f(y_2)]+\metric[f(y_2)][y_2]\leq \metric[y_1][f(y_1)]+M\metric[y_1][y_2]+\metric[f(y_2)][y_2],
\end{equation}
and so
\begin{equation}\label{4.5.75}
\metric[y_1][y_2]\leq \frac{1}{1-M}\left( \metric[y_1][f(y_1)]+\metric[f(y_2)][y_2]\right) 
\end{equation}
Also note that
\begin{equation}
\metric[f^m(y_1)][f^m(y_2)]\leq M^m\metric[y_1][y_2],
\end{equation}
which follows of course from just applying \eqref{4.5.70} inductively.  Taking $y_1\coloneqq f^m(x_0)$ and $y_2\coloneqq f^n(x_0)$ in \eqref{4.5.75}, we find
\begin{equation}
\begin{split}
\metric[x_m][x_n] & \leq \frac{1}{1-M}\left( \metric[x_m][x_{m+1}]+\metric[x_{n+1}][x_n]\right) \leq \frac{1}{1-M}\left( M^m\metric[x_0][f(x_0)]+M^n\metric[f(x_0)][x_0]\right) \\
& =\frac{M^m+M^n}{1-M}\metric[x_0][f(x_0)].
\end{split}
\end{equation}
Because $M<1$, we can make $\frac{M^m+M^n}{1-M}$ arbitrarily small by taking $m$ and $n$ sufficiently large.\footnote{If you are not comfortable with this amount of detail, I suggest you fill in the gaps.  You will want to get to the point where you feel comfortable just asserting cauchyness after obtaining an inequality like this.}  Hence, this sequence is cauchy, and we are done.
\end{proof}
\end{thm}
\begin{exm}[A contraction mapping with no fixed-point]
Take $X\coloneqq \R ^+$ and define $f(x)\coloneqq \frac{1}{2}x$.  The fixed-point `should' be $0$, but $0\notin \R ^+$.  You can turn this intuition into a proof that this map indeed has no fixed point, and we recommend you try to do so.
\end{exm}


\chapter{Integration}\label{chp5xx}

So, first things first---fuck the riemann integral.  Seriously.  The only argument pro-riemann-integral is that it is easier.  What a ridiculous argument.  This is math, dude.  If you choose to do things because they're easy, you're in the wrong subject.  Moreover, I would argue that this is not even true---if you set things up right, you can literally \emph{define} the (lebesgue) integral to be the area (measure) under the curve.  Or, if you prefer, you can take a limit over the size of a partition of the sum of the areas of the rectangles corresponding to the subsets of the partition (the riemann integral).  Are you really going to sit here and try to argue that this is easier to teach?  I call bullshit.  And besides, if you're going to become a mathematician, you have to learn the lebesgue integral at some point anyways\textellipsis why learn something only to have to relearn it later?

Okay, so now that my rant is out of the way, let's actually do some mathematics.

\section{Measure theory}

All of integration theory ultimately boils down to measure theory.  The definition of the integral itself is relatively easy.  In fact, the definition of abstract measure spaces is even easier.  There's really no question that writing down the definition of the lebesgue integral is \emph{significantly} easier than that of the riemann integral.  What is a bit tricky, however, is constructing specific measures.  In our case, we will primarily be concerned with constructing lebesgue measure (on $\R ^d$), and this is really the only part that is a bit tricky.  Before we get there though, we will present the theory of measure spaces, and then get to lebesgue measure in the next section.

The intuition behind measure is actually quite easy---a measure is just an axiomatization of our intuition about notion of things like length, area, and volume.  Before we define a measure, it will be convenient to introduce a couple of terms.

\subsection{Basic terminology}

\begin{dfn}
Let $X$ be a set, let $\meas :2^X\rightarrow [0,\infty ]$, and let $\mathcal{M}\subseteq 2^X$.  
\begin{enumerate}
\item If $\mathcal{U}$ is a cover of $X$ on which $\meas$ is constant, then we say that $\mathcal{U}$ is \emph{uniformly-measurable}\index{Uniformly-measurable cover} with respect to $\meas$ and we denote by $\meas (\mathcal{U})$ its constant value on $\mathcal{U}$;
\item $\meas$ is \emph{subadditive}\index{Subadditive} on $\mathcal{M}$ iff for $\{ M_m:m\in \N \} \subseteq \mathcal{M}$ we have
\begin{equation}\label{5.1.2}
\meas \left( \bigcup _{m\in \N}M_m\right) \leq \sum _{m\in \N}\meas (M_m);
\end{equation}
\item $\meas$ is \emph{additive}\index{Additive (measure)} on $\mathcal{M}$ iff for $\{ M_m:m\in \N \} \subseteq \mathcal{M}$ a \emph{disjoint} collection we have
\begin{equation}\label{5.1.3}
\meas \left( \bigcup _{m\in \N}M_m\right) =\sum _{m\in \N}\meas (M_m).
\end{equation}
\end{enumerate}
$\meas$ is simply just subadditive (resp.~additive) if it is subadditive (resp.~additive) on all of $X$.
\end{dfn}
\begin{exr}\label{exr5.1.4}
Show that if $\meas$ is additive on $\mathcal{M}$ then it is subadditive on $\mathcal{M}$.
\begin{rmk}
There is something to show here.  While \eqref{5.1.3} itself is obviously a stronger condition than \eqref{5.1.2}, it is also only assumed for \emph{disjoint} collections.  The problem then is to show that, if \eqref{5.1.3} holds for disjoint collections, then \eqref{5.1.2} holds for \emph{all} collections.
\end{rmk}
\end{exr}
\begin{dfn}[Boolean algebra]
A \emph{boolean algebra}\index{Boolean algebra} is a set $X$ equipped with two binary operations, $\vee$ and $\wedge$, a unary operation $\neg$, and identities for $\vee$ and $\wedge$, $0$ and $1$ respectively, such that
\begin{enumerate}
\item $\coord{X,\vee ,0}$ and $\coord{X,\wedge ,1}$ are both monoids;
\item $\vee$ distributes over $\wedge$ and $\wedge$ distributes over $\vee$; and
\item $x\vee \not x=1$ and $x\wedge \not x=0$.
\end{enumerate}
\begin{rmk}
The example you should keep in mind, and indeed, the only example of relevance to us (together with its subalgebras) is $\coord{2^X,\cup ,\cap ,\blank ^{\comp},\emptyset ,X}$.
\end{rmk}
\end{dfn}
\begin{exr}
Let $\mathcal{M}\subseteq 2^X$.  Show that $\mathcal{M}$ is a subalgebra\footnote{That is, it itself is a boolean algebra with respect to union, intersection, and complementation.} of $2^X$ iff it is closed under finite union and complementation.
\end{exr}
A construction we'll be making use of a lot is the boolean algebra of sets \emph{generated} by a given collection.
\begin{prp}
Let $\mathcal{S}\subseteq 2^X$.  Then, there exists a unique subalgebra $\mathcal{M}$ of $2^X$, the boolean algebra \emph{generated by} $\mathcal{S}$, such that
\begin{enumerate}
\item $\mathcal{S}\subseteq \mathcal{M}$; and
\item if $\mathcal{M}'$ is any other subalgera containing $\mathcal{S}$, it follows that $\mathcal{M}\subseteq \mathcal{M}'$.
\end{enumerate}
Furthermore, $\mathcal{M}$ is the collection of all sets that can be written as a finite union of finite intersections of sets coming from $\mathcal{S}$ or complements of sets in $\mathcal{M}$.
\begin{proof}
We leave this as an exercise.
\begin{exr}
Complete the proof yourself.
\end{exr}
\end{proof}
\end{prp}

\subsection{Main definitions}

\begin{dfn}[Outer measure]\label{OuterMeasure}
Let $X$ be a set.  An \emph{outer measure}\index{Outer measure} on $X$ is a function $\meas :2^X\rightarrow [0,\infty ]$ such that
\begin{enumerate}
\item $\meas (\emptyset )=0$;
\item (Nondecreasing)\label{Measure.Monotonicity} $\meas :\coord{2^X,\subseteq}\rightarrow [0,\infty ]$ is nondecreasing;\footnote{Concretely, this means that $\meas (S)\leq \meas (T)$ if $S\subseteq T$.}; and
\item (Subadditivity) $\meas$ is subadditive.
\end{enumerate}
\begin{rmk}
Note that we allow the measure of sets to be infinite.  This is incredibly important---for example, we will want $\meas (\R )=\infty$ (for lebesgue measure anyways).
\end{rmk}
\begin{rmk}
As a consequence of this, we needn't worry about convergence in the third axiom (see \eqref{5.1.2}).  As a matter of fact, we definitely want to allow this sum to diverge---think about what the measure of $\bigcup _{m\in \Z}(m,m+1)$ should be.
\end{rmk}
\end{dfn}
\begin{dfn}[Uniform measure]\label{Measure}
Let $\coord{X,\widetilde{\mathcal{U}}}$ be a uniform space.  A \emph{uniform measure}\index{Uniform measure} on $X$ is an outer measure on $X$ for which there exists a uniform base $\widetilde{\mathcal{B}}$ of $X$ consisting of uniformly-measurable covers that is additive on the boolean algebra generated by $\bigcup _{\mathcal{B}\in \widetilde{\mathcal{B}}}\mathcal{B}$.  $\widetilde{\mathcal{B}}$ is a \emph{uniformly-measurable} base for $\coord{X,\meas}$.
\begin{rmk}
Think about what having a uniform base of uniformly-measurable covers means for a metric space---if we take as a uniform base the collection of all covers by $\varepsilon$-balls, then this is just the statement that every $\varepsilon$-ball has to have the same measure.
\end{rmk}
\begin{rmk}
Note that you definitely do not want to require \emph{every} uniform cover be uniformly-measurable.  For example, in a metric space, by upward-closedness the collection of all $\varepsilon$-balls together with a single $2\varepsilon$-ball will also be a uniform-cover---we definitely do not want to require that a $2\varepsilon$ ball has the same measure as an $\varepsilon$-ball.
\end{rmk}
\end{dfn}
\begin{displayquote}
At some point in the near future, we will be doing arithmetic with $\infty$---for example, what should the measure of $\R \times \{ 0\}$ in $\R ^2$ be?  Of course, from our definition of product measures, this will turn out to be $\infty \cdot 0$.  We hence declare that
\begin{equation}
\infty \cdot 0\coloneqq 0\eqqcolon 0\cdot \infty .
\end{equation}
There are other arithmetic notions we have to technically define (e.g.~$x+\infty=\infty$), but this is the only nonobvious one.
\end{displayquote}
\begin{exm}[The zero measure]
Let $\coord{X,\widetilde{\mathcal{U}}}$ be a uniform space and define $\meas :2^X\rightarrow [0,\infty ]$ by $\meas (S)\coloneqq 0$.  How terribly interesting.
\end{exm}
\begin{exm}[The infinite measure]
Let $\coord{X,\widetilde{\mathcal{U}}}$ be any uniform space, let $\widetilde{\mathcal{B}}$ be a uniform base no cover of which contains the empty-set, and define $\meas :2^X\rightarrow [0,\infty ]$ by
\begin{equation}
\meas (S)\coloneqq \begin{cases}0 & \text{if }S=\emptyset \\ \infty & \text{otherwise}\end{cases}.
\end{equation}
Dear god, this example is even more interesting than the last one.
\end{exm}
\begin{exm}[The counting measure]
Let $\coord{X,\widetilde{\mathcal{U}}}$ be a discrete uniform space (so that the set which a single cover, the cover by singletons, forms a uniform base), and for $S\subseteq X$ define $\meas (S)\coloneqq \abs{S}$, that is, the cardinality of $S$.
\begin{rmk}
This is actually incredibly important, as we shall see that sums are just integrals with respect to the counting measure.
\end{rmk}
\end{exm}
Before we get to any examples more interesting than this, we will first have to develop a bit of theory.
\begin{prp}\label{prp5.1.9}
Let $\meas$ be a measure on a uniform space $X$ with uniformly-measurable base $\widetilde{\mathcal{B}}$ and let $\mathcal{B},\mathcal{C}\in \widetilde{\mathcal{B}}$.  Then, if $\mathcal{B}\preceq \mathcal{C}$ (and in particular if $\mathcal{B}\llcurly \mathcal{C}$), then $\meas (\mathcal{B})\leq \meas (\mathcal{C})$.
\begin{proof}
Let $B\in \mathcal{B}$.  Then, there is some $C\in \mathcal{C}$ be such that $B\subseteq C$.  Thus, we have
\begin{equation}
\meas (\mathcal{B})\coloneqq \meas (B)\leq \meas (C)\eqqcolon \meas (\mathcal{C}).
\end{equation}
\end{proof}
\end{prp}
Our first relatively significant result, which we shall use to define lebesgue measure on $\R ^d$, is that, to define a measure, it suffices to define a measure on just the sets in a given uniform base.
\begin{thm}[Carath\'{e}odory's Extension Theorem]\label{CaratheodorysExtensionTheorem}
\begin{savenotes}
Let $X$ be a uniform space with uniform base $\widetilde{\mathcal{B}}$, let $\mathcal{M}$ be the collection of sets that are a finite union and intersection of elements of $\widetilde{\mathcal{B}}$ and their complements, and let $\meas :\mathcal{M}\rightarrow [0,\infty ]$ be such that
\begin{enumerate}
\item \label{CaratheodorysExtensionTheorem.i}$\meas (\emptyset )=0$;
\item \label{CaratheodorysExtensionTheorem.iii}every $\mathcal{B}\in \widetilde{\mathcal{B}}$ is uniformly-measurable with respect to $\meas$; and
\item \label{CaratheodorysExtensionTheorem.iv}$\meas$ is additive.
\end{enumerate}
Then,
\begin{equation}
\meas (S)\coloneqq \inf \left\{ \sum _{m\in \N}\meas (B_m):S\subseteq \bigcup _{m\in \N}M_m\text{ and  each }M_m\in \mathcal{M}\text{ (or }M_m=\emptyset \text{).}\right\} ,\footnote{The point of allowing each $M_m$ to be empty is to allow for finite unions as well.}
\end{equation}
is a measure on $X$ which agrees with $\meas$ on $\mathcal{M}$.
\begin{rmk}
These are just verbatim three out of four of the axioms of a measure.  The point is that, to define a measure, it suffices to define them on sets coming from a given uniform base.  Furthermore, you also do not have to check subadditivity of your original definition.
\end{rmk}
\begin{rmk}
Warning:  This measure need not be unique in general---see \cref{exm5.1.22}.
\end{rmk}
\begin{proof}
\Step{Introduce notation}
For the time being, let us write
\begin{equation}
\breve{\meas}(S)\coloneqq \inf (M(S))
\end{equation}
with the breve to distinguish it from what we started with, where
\begin{equation}
M(S) \coloneqq \left\{ \sum _{m\in \N}\meas (B_m):S\subseteq \bigcup _{m\in \N}M_m\text{ and  each }M_m\in \mathcal{M}\text{ (or }M_m=\emptyset \text{).}\right\} .
\end{equation}
(Once we show that this indeed agrees with what we started with, we shall drop the breve.)

\Step{Show that $\breve{\meas}(S)=\meas (S)$ for $S\in \mathcal{M}$}
First of all, as $\meas (S)\in M(S)$, we have that $\breve{\meas}(S)\leq \meas (S)$.  In particular, if $\breve{\meas}(S)=\infty$, then $\meas (S)=\infty$ as well, and so we may as well assume that $\breve{\meas}(S)$ is finite.  For the other inequality, let $\varepsilon >0$, and let $M_m\in \mathcal{M}$ (or empty) be such that (i) $S\subseteq \bigcup _{m\in \N}M_m$ and (ii)
\begin{equation}
\breve{\meas}(S)\leq \sum _{m\in \N}\meas (M_m)<\breve{\meas}(S)+\varepsilon .
\end{equation}
Define
\begin{equation}
M_m'\coloneqq S\cap \left( M_m\setminus \bigcup _{k=0}^{m-1}M_k\right) .
\end{equation}
Note that (i) each $M_m'\in \mathcal{M}$, (ii) $M_m'\subseteq M_m$, (iii) the $M_m'$s are disjoint, and (iv) $S=\bigcup _{m\in \N}M_m'$.  Thus,
\begin{equation}
\meas (S)=\sum _{m\in \N}\meas (M_m')\leq \sum _{m\in \N}\meas (M_m)<\breve{\meas}(S)+\varepsilon .
\end{equation}
Hence, $\meas (S)\leq \breve{\meas (S)}$, and so $\meas (S)=\breve{\meas}(S)$.  Thus, hereafter, we drop the breve.

\Step{Show that $\meas (S)\geq 0$}
As $M(S)\subseteq [0,\infty ]$, we always have that $\meas (S)\geq 0$.

\Step{Show that $\meas (\emptyset )=0$}
The empty-set itself is a countable cover of $\emptyset$, so that $0\in M(\emptyset )$, and hence $\meas (\emptyset )\leq 0$.  Of course, we already know that $\meas (\emptyset )\geq 0$, and so $\meas (\emptyset )=0$.

\Step{Show that $\meas$ is nondecreasing}
If $S\subseteq T$, then $M(S)\supseteq M(T)$, and so, taking the infimum, we have that $\meas (S)\leq \meas (T)$.

\Step{Show that $\meas$ is subadditive on $2^X$}
Let $\{ M_m:m\in \N \} \subseteq 2^X$.  If $\meas (M_m)=\infty$ for any $m\in \N$, then because $\meas$ is nondecreasing, we would likewise have that
\begin{equation}
\meas \left( \bigcup _{m\in \N}M_m\right) \geq \meas (M_m)=\infty ,
\end{equation}
and so in this case we are done.  Thus, we may as well assume without loss of generality that each $\meas (M_m)$ is finite.  Let $\varepsilon >0$.  Then,\footnote{This is why we needed finiteness.} there are $M_{m,n}\in \mathcal{M}\cup \{ \emptyset \}$, such that
\begin{equation}
\meas _(M_m)\leq \sum _{n\in \N}\meas (M_{m,n})<\meas (M_m)+\tfrac{\varepsilon}{2^m}.
\end{equation}
As $\bigcup _{m\in \N}M_m\subseteq \bigcup _{m,n\in \N}M_{m,n}$, we thus have that
\begin{equation}
\meas \left( \bigcup _{m\in \N}M_m\right) \leq \sum _{m,n\in \N}\meas (M_{m,n})<\sum _{m\in \N}\left[ \meas (M_m)+\tfrac{\varepsilon}{2^m}\right] =\sum _{m\in \N}\meas (M_m)+\varepsilon .
\end{equation}
Thus, indeed,
\begin{equation}
\meas _{\mathcal{B}}\left( \bigcup _{m\in \N}M_m\right) \leq \sum _{m\in \N}\meas (M_m).
\end{equation}

\Step{Conclude that $\meas$ is a measure on $X$}
The only remaining axiom that needs verifying is that it is additive on $\bigcup _{\mathcal{B}\in \widetilde{\mathcal{B}}}\mathcal{B}$, however, this was assumed, and so there is nothing to prove.\footnote{Well, I suppose we are technically using the fact that our new measure agrees with the old on $\mathcal{M}$, which contains $\bigcup _{\mathcal{B}\in \widetilde{\mathcal{B}}}\mathcal{B}$.}
\end{proof}
\end{savenotes}
\end{thm}
\begin{exm}[The extension need not be unique]\label{exm5.1.22}
Take $X\coloneqq \R$ and $\widetilde{\mathcal{B}}\coloneqq \{ \mathcal{B}_\varepsilon :\varepsilon >0\}$.  Then, the only noninfinite set which is contained in the collection $\mathcal{M}$ of sets that are a finite union and intersection of elements of $\widetilde{\mathcal{B}}$ and their complements is the empty-set.  Define $\meas :\mathcal{M}\rightarrow [0,\infty ]$ such that
\begin{equation}
\meas (S)\coloneqq \begin{cases}0 & \text{if }S=\emptyset \\ \infty & \text{otherwise}\end{cases}.
\end{equation}
Then, the counting measure and the infinite measure are two distinct extensions of $\meas$ to all of $2^X$.
\end{exm}

\begin{dfn}[Regular measure]\label{RegularMeasure}
Let $X$ be a uniform space and let $\meas :2^X\rightarrow [0,\infty ]$ be a uniform measure.  Then, $\meas$ is \emph{regular}\index{Regular measure}
\begin{enumerate}
\item $\meas$ is finite on quasicompact subsets;
\item (Outer-regular) for $S\subseteq X$,
\begin{equation}
\meas (S)=\inf \{ \meas (U):S\subseteq U,\ U\text{ open.}\} ;\text{ and }
\end{equation}
\item (Inner-regular on open sets) for $U\subseteq X$ open,
\begin{equation}
\meas (U)=\sup \{ \meas (K):K\subseteq U,\ K\text{ quasicompact.}\} .
\end{equation}
\end{enumerate}
\end{dfn}
\begin{dfn}[Positively-separated]\label{PositivelySeparated}
Let $X$ be a set, let $\mathcal{U}$ be a cover of $X$, and let $S,T\subseteq X$.  Then, $S$ and $T$ are \emph{positively-separated} with respect to $\mathcal{U}$ iff no element of $\mathcal{U}$ intersects both $S$ and $T$.
\end{dfn}

\begin{dfn}[Isogeneous space]\label{IsogeneousSpace}
An \emph{isogeneous space} is a uniform space $X$ equipped with a group of uniform-homeomorphisms $H$ such that
\begin{equation}
\widetilde{\mathcal{B}}_H\coloneqq \{ \mathcal{B}_U\} \text{ where }\mathcal{B}_U\coloneqq \{ h(U):h\in H\} \text{ and }U\subseteq X\text{ open.}\right\} .
\end{equation}
is a uniform base for $X$.
\begin{rmk}
$\widetilde{\mathcal{B}}_H$ is the \emph{isogeneous base}\index{Isogeneous base} and each $\mathcal{B}_U$ is an \emph{isogeneous cover}\index{Isogeneous cover}.
\end{rmk}
\begin{rmk}
The example you should have in mind here is that of a topological group $G$.  In this case, the uniform base is the canonical one, $\widetilde{\mathcal{B}}\coloneqq \{ \mathcal{B}_U\}$ with $\mathcal{B}_U\coloneqq \{ gU:g\in U\}$ for $U$ an open neighborhood of the identity, and the set of all uniform homeomorphisms is the set of all left translations, $H\coloneqq \{ h_g:g\in G\}$ where $h_g(x)\coloneqq gx$.  Of course, you can also choose right-translations over left-translations (in both $\widetilde{\mathcal{B}}$ and $H$!) if you so desire.
\end{rmk}
\begin{rmk}
What do you think the morphisms of isogeneous spaces should be?
\end{rmk}
\end{dfn}
\begin{exr}
Let $G$ be a topological group and let $H\coloneqq \{ h_g:G\in G\}$, where $h_g:G\rightarrow G$ is defined by $h_g(x)\coloneqq gx$.  Show that $\coord{G,H}$ is an isogeneous space.
\end{exr}
\begin{thm}[Howe's Theorem]\index{Howe's Theorem}\label{HowesTheorem}
\begin{savenotes}
Let $\coord{X,H}$ be a $T_0$ isogeneous space. Then, if $X$ has a quasicompact set $K$ with nonempty interior,\footnote{So that then there $\mathcal{B}_{\Int (K)}$, $K$ the quasicompact set with nonempty interior, is an isogeneous cover.} then there exists a unique regular $\meas$ measure on $X$ such that (i) each cover in $\widetilde{\mathcal{B}}_H$ is a uniformly-measurable base with respect to $\meas$ and (ii) $\meas (K)=1$.
\begin{rmk}
You should think of $K$ has a set with which we can `compare' all other sets to get a ``measure'' of `size'.  The condition that it be quasicompact you can think of the condition that the measure of $K$ be finite, and the condition that it have nonempty interior you can think of the condition that the measure of $K$ be positive.  Intuitively, you can imagine that if $\meas (K)=\infty$ or $\meas (K)=0$, then it will be essentially impossible to compare `compare' the `size' of other sets to $K$.
\end{rmk}
\begin{rmk}
If $G$ is a topological group and has a quasicompact set with nonempty interior, in this classical case, the resulting measure is called a (left) \emph{haar measure}\index{Haar measure} (for the symmetries being \emph{left}-translation).  In particular, \emph{lebesgue measure} will be the haar measure for the topological group $\coord{\R ,^d+,0,-}$.\footnote{If the group is commutative, the symmetries by left translation and right translations are the same, so left vs.~right does not matter.}
\end{rmk}
\begin{proof}
\Step{Make hypotheses and introduce notation}
Suppose that $X$ has a quasicompact set $K_0$ with nonempty interior.  Denote the uniform topology on $X$ by $\mathcal{U}$ and denote the collection of all quasicompact subsets of $X$ by $\mathcal{K}$

\Step{Define $(K:U)$ for $K\in \mathcal{K}$ and $U\in \mathcal{U}$}
The cover $\mathcal{B}_U\coloneqq \{ h(U):h\in H\}$ is an open cover of $K$, and so there are a finite subcover.  Let $(K:U)$ denote the cardinality of the smallest such subcover.

\Step{Define $\mathrm{H}_U\mathcal{K}\rightarrow \R _0^+$}
For $U\in \mathcal{U}$, define $\mathrm{H}_U:\mathcal{K}\rightarrow \R _0^+$ by
\begin{equation}
\mathrm{H}_U(K)\coloneqq \frac{(K:U)}{(K_0:U)}.\footnote{$K_0$ is nonempty, and so cannot be covered by anything empty.  Therefore, $(K_0:U)\geq 1$, and in particular, is not $0$.}
\end{equation}

\Step{Show that $\mathrm{H}_U(K)\leq (K:\Int (K_0))$}
We now check that $\mathrm{H}_U(K)\leq (K:\Int (K_0))$, that is, $(K:U)\leq (K:\Int (K_0))(K_0:U)$.  Let us temporarily write $m\coloneqq (K:\Int (K_0))$ and $n\coloneqq (K_0:U)$.  There are thus $h_1,\ldots ,h_m\in H$ such that $\{ h_1(\Int (K_0)),\ldots ,h_m(\Int (K_0))\}$ covers $K$.  There are also $h_1',\ldots ,h_n'\in H$ such that $\{ h_1(U),\ldots ,h_n(U)\}$ covers $K$.  Therefore,
\begin{equation}
K\subseteq \bigcup _{k=1}^mh_k(\Int (K_0))\subseteq \bigcup _{k=1}^mh_k(K_0)\subseteq \bigcup _{k=1}^mh_k\left( \bigcup _{l=1}^nh_l'(U)\right) =\bigcup _{k=1}^m\bigcup _{l=1}^n[h_k\circ h_l'](U)
\end{equation}
Hence, $K$ is covered by $mn$ elements of $\mathcal{B}_U$,\footnote{Here we are using the fact that $H$ is closed under composition, so that $h_k\circ h_l'\in h$.} and hence $(K:U)\leq mn\coloneqq (K:\Int (K_0))(K_0:U)$.

\Step{Define $\mathrm{H}$}
Define $T\coloneqq \prod _{K\in \mathcal{K}}[0,(K:K_0)]$.  Each $\mathrm{H}_U$ may be thought of as a point in $T$, whose component at $K\in \mathcal{K}$ is $\mathrm{H}_U(K)\in [0,(K:K_0)]$.\footnote{That was sort of the point of the previous step.}  Thus, for $U\in \mathcal{U}$, let us define
\begin{equation}
C_U\coloneqq \Cls \left( \left\{ \mathrm{H}_U:\mathcal{U}\ni V\subseteq U\right\} \right) 
\end{equation}
and
\begin{equation}
\mathcal{C}\coloneqq \{ C_U:U\in \mathcal{U}\} .
\end{equation}
We wish to show that the intersection of any finitely many elements of $\mathcal{C}$ is nonempty.  Then, because $T$ is quasicompact by \nameref{TychnoffsTheorem} (\cref{TychnoffsTheorem}), it will follow that the intersection over \emph{all} elements in $\mathcal{C}$ will be nonempty.

This is actually really easy, however, because for $U_1,\ldots ,U_m\in \mathcal{U}$, we have that
\begin{equation}
\mathrm{H}_{U_1\cap \cdots \cap U_m}\in \bigcap _{k=1}^mC_{U_k}.
\end{equation}
Therefore, by quasicompactness, there is some
\begin{equation}
\mathrm{H}\in \bigcap _{U\in \mathcal{U}}C_U.
\end{equation}

\Step{Show that $\mathrm{H}(K_1)\leq \mathrm{H}(K_2)$ if $K_1\subseteq K_2$}\label{Haar.4}
Let $K_1,K_2\in \mathcal{K}$ be such that $K_1\subseteq K_2$.  We first show that, for each $U\in \mathcal{U}$, $\mathrm{H}_U(K_1)\leq \mathrm{H}_U(K_2)$.  But this is trivial, because the covering of $K_2$ with $(K_2:U)$ elements of $\mathcal{B}_U$ is also a covering of $K_1$ with $(K_2:U)$ elements of $\mathcal{B}_U$, so that $(K_1:U)\leq (K_2:U)$, and hence $\mathrm{H}_U(K_1)\leq \mathrm{H}_U(K_2)$.

Thinking of elements $f$ of $T$ as functions from $\mathcal{K}$ to $\R$, consider the map\footnote{For each $K_1,K_2\in \mathcal{K}$ with $K_1\subseteq K_2$, we have such a map.} that sends $f\in T$ to $f(K_2)-f(K_1)$.  This is a composition of continuous functions, and hence continuous.\footnote{The first map from $T$ into $\R \times \R$ is the projection of $f\in T$ onto the $K_1^{\text{th}}$ coordinate in the first coordinate and the projection of $f\in T$ onto $K_2^{\text{th}}$ coordinate in the second coordinate.  This map is continuous because it is continuous in each coordinate.  Each coordinate is continuous because projections are continuous.  The first map is followed by the map from $\R \times \R$ into $\R$ given by subtraction, which is continuous because we know that $\coord{R,+,0,-}$ is a topological group.}  This map is also nonnegative on each $C_U$ because $\mathrm{H}_U(K_1)\leq \mathrm{H}_U(K_2)$ for each $U\in \mathcal{U}$ (we need continuity so that we know it is nonnegative on the \emph{closure} of $\{ \mathrm{H}_V:V\subseteq U\}$).  As $\meas$ is an element of each $C_U$, it follows that this map is also nonnegative at $\meas$, so that $\mathrm{H}(K_1)\leq \mathrm{H}(K_2)$.

\Step{Show that $\mathrm{H}(K_1\cup K_2)\leq \mathrm{H}(K_1)+\mathrm{H}(K_2)$}\label{Haar.5}
Let $K_1,K_2\in \mathcal{K}$.  We first show that $\mathrm{H}_U(K_1\cup K_2)\leq \mathrm{H}_U(K_1)+\mathrm{H}_U(K_2)$ for each $U\in \mathcal{U}$.  This is trivial, because a covering of $K_1$ with $(K_1:U)$ elements of $\mathcal{B}_U$ together with a covering of $K_2$ with $(K_2:U)$ elements of $\mathcal{B}_U$ is a cover of $K_1\cup K_2$ with $(K_1:U)+(K_2:U)$ elements of $\mathcal{B}_U$, so that $(K_1\cup K_2:U)\leq (K_1:U)+(K_2:U)$.  It follows that $\mathrm{H}_U(K_1\cup K_2)\leq \mathrm{H}_U(K_1)+\mathrm{H}_U(K_2)$.

Proceeding similarly as in \cref{Haar.4}, the map that sends $f\in T$ to $f(K_1)+f(K_2)-f(K_1\cup K_2)$ is continuous and nonnegative on each $C_U$, and hence is nonnegative for $\mathrm{H}\in T$.  Thus, $\mathrm{H}(K_1\cup K_2)\leq \mathrm{H}(K_1)+\mathrm{H}(K_2)$.

\Step{Show that $\mathrm{H}_U(K_1\cup K_2)=\mathrm{H}_U(K_1)+\mathrm{H}_U(K_2)$ if $K_1$ and $K_2$ are positively-separated with respect to $\mathcal{B}_U$}
Let $K_1,K_2\in \mathcal{K}$ be positively-separated with respect to $\mathcal{B}_U$.  We have already shown that $\mathrm{H}_U(K_1\cup K_2)\leq \mathrm{H}_U(K_1)+\mathrm{H}_U(K_2)$, so it suffices to show that $\mathrm{H}_U(K_1)+\mathrm{H}_U(K_2)\leq \mathrm{H}_U(K_1\cup K_2)$.  In other words, it suffices to show that $(K_1:U)+(K_2:U)\leq (K_1\cup K_2:U)\eqqcolon m$.  Let $h_1(U),\ldots ,h_m(U)\in \mathcal{B}_U$ be a cover of $K_1\cup K_2$.  By hypothesis,\footnote{This is the definition of positively-separated---see \cref{PositivelySeparated}.} every single one of these can only intersect $U_1$ or $U_2$, but not both.  Thus, after relabeling if necessary, the first $k$ of these guys will form a cover of $K_1$ and the latter $m-k$ will form a cover of $K_2$.  Thus, $(K_1:U)\leq k$ and $(K_2:U)\leq m-k$, and so $(K_1:U)+(K_2:U)\leq k+(m-k)=m\coloneqq (K_1\cup K_2:U)$, which completes this step.

\Step{Define the measure $\meas$ on all open subsets of $X$}
For $U\subseteq X$ open, define
\begin{equation}\label{5.1.46}
\meas (U)\coloneqq \sup \{ \mathrm{H}(K):K\subseteq U,\ K\in \mathcal{K}\} .
\end{equation}

\Step{Extend $\meas$ to all subsets of $X$}
Now, for an arbitrary subsets $S$ of $X$, define
\begin{equation}
\meas (S)\coloneqq \inf \{ \meas (U):S\subseteq U,\ U\in \mathcal{U}\} .
\end{equation}
\begin{exr}
Show that this agrees with \eqref{5.1.46} when $S$ is open, so that this is indeed an extension.
\end{exr}

\Step{Show that $\meas$ is an outer-measure}
\begin{exr}
Check that $\meas (\emptyset )=0$ and that $\meas$ is nondecreasing.
\end{exr}

We now check that it is subadditive.  To prove this, we will first need a lemma.
\begin{lma}
Let $X$ be $T_2$, let $K\subseteq X$ be quasicompact, and let $U_1,U_2\subseteq X$ be open and such that $K\subseteq U_1\cup U_2$>  Then, there are quasicompact subsets $K_1,K_2\subseteq X$ such that (i) $K_1\subseteq U_1$, (ii) $K_2\subseteq U_2$, and (iii) $K=K_1\cup K_2$.
\begin{proof}
We leave this as an exercise.
\begin{exr}
Complete the proof yourself.
\end{exr}
\end{proof}
\end{lma}
Having (hopefully) proved the lemma, we now show subadditivity for \emph{open} sets.  (We will then prove subadditivity in general.)  So, let $\{ U_m:m\in \N \}$ be a countable collection of open sets of X.  Let $K\subseteq \bigcup _{m\in \N}U_m$.  Then, there is some $m_K\in \N$ such that $K\subseteq \bigcup _{k=1}^{m_K}U_k$.  By applying this lemma inductively then, we may find quasicompact sets $K_1,\ldots ,K_m$ such that (i) $K_k\subseteq U_k$ for $1\leq k\leq m$ and $K=\bigcup _{k=1}^mK_k$.  Using the fact that we have already proved finite `subadditivity' (of $\mathrm{H}$) for quasicompact sets (\cref{Haar.5}), we find that
\begin{equation}
\mathrm{H}(K)\leq \sum _{k=1}^m\mathrm{H}(K_k)\leq \sum _{k=1}^m\meas (U_k)\leq \sum _{m\in \N}\meas (U_m).
\end{equation}
Taking the $\sup$ of $K$, we find that
\begin{equation}
\meas \left( \bigcup _{m\in \N}U_m\right) \coloneqq \sup \left\{ \mathrm{H}(K):K\subseteq \bigcup _{m\in \N}U_m,\ K\in \mathcal{K}\right\} \leq \sum _{m\in \N}\meas (U_m).
\end{equation}

Having proved subadditivity for open sets, we now prove it for arbitrary sets.  So, let $\{ S_m:m\in \N \}$ be an arbitrary countable collection of subsets of $X$.  If $\sum _{m\in \N}\meas (S_m)=\infty$, then there is nothing to show, and so we may as well suppose without loss of generality that $\sum _{m\in \N}\meas (S_m)<\infty$.  Let $\varepsilon >0$ and for each $m\in \N$ pick an open set $U_m$ such that (i) $S_m\subseteq U_m$ and (ii) $\meas (S_m)\leq \meas (U_m)<\meas (S_m)+\frac{\varepsilon}{2^m}$.  Then, using subadditivity for open sets, we find
\begin{equation}
\meas \left( \bigcup _{m\in \N}S_m\right) \leq \meas \left( \bigcup _{m\in \N}U_m\right) \leq \sum _{m\in \N}\meas (U_m)<\sum _{m\in \N}\left[ \meas (S_m)+\tfrac{\varepsilon}{2^m}\right] =\sum _{m\in \N}\meas (S_m)+2\varepsilon .
\end{equation}
Hence, as $\varepsilon >0$ was arbitrary, we have that
\begin{equation}
\meas \left( \bigcup _{m\in \N}S_m\right) \leq \sum _{M\in \N}\meas (S_m).
\end{equation}
Thus, $\meas$ is an outer-measure on $X$.

\Step{Show that each $\mathcal{B}_U$ is uniformly-measurable with respect to $\meas$}
Let $h\in H$.  We want to show that $\meas (h(U))=\meas (U)$.  Then, for any other $h'\in H$, we will have that $\meas (h(U))=\meas (U)=\meas (h'(U))$, so that indeed every element of $\mathcal{B}_U$ has the same measure.  However, $K$ is a quasicompact set contained in $U$ iff $h(K)$ is a quasicompact set contained in $h(U)$.\footnote{This implicitly uses the fact that $h^{-1}\in H$.}  Therefore, by the definition of $\meas (U)$ \eqref{5.1.46} it suffices to show that $\mathrm{H}(K)=\mathrm{H}(h(K))$.  To show this, we first show that $\mathrm{H}_U(K)=\mathrm{H}_U(h(K))$ for all $U\in \mathcal{U}$.  That is, we would like to show that $(K:U)=(h(K):U)$.  However, every cover of $K$ by elements of $\mathcal{B}_U$, $h_1(U),\ldots ,h_m(U)$, gives a cover of $h(K)$ by elements of $\mathcal{B}_U$ of the same cardinality, $h(h_1(U)),\ldots ,h(h_m(U))$.  It thus follows that $\mathrm{H}_U(K)=\mathrm{H}_U(h(K))$.

For $h\in H$ fixed, consider the map from $T$ to $\R$ that sends $f$ to $f(h(K))-f(K)$.  We just showed that this is $0$ on each $\mathrm{H}_U\in T$, and so it is $0$ on $C_U$, and so it is $0$ on $\mathrm{H}$, that is, $\mathrm{H}(K)=\mathrm{H}(h(K))$.

\end{proof}
\end{savenotes}
\end{thm}


\chapter{Differentiation}\label{chp5x}

Finally we are ready to begin our study of calculus proper.  We could have started this awhile ago, but we really needed certain facts about continuity, uniform convergence, etc.~before we could address all the facts we care about related to differentiation.  This thus led us to a relatively broad study of the most general spaces\footnote{Meh, basically anyways} in which these notions makes sense.  For the time being, however, we return to $\R$.

\section{Tensors and abstract index notation}

\begin{displayquote}
Throughout this chapter, all vector spaces will be finite-dimensional and \emph{real}.  For the remainder of this \emph{section}, $V$ and $W$ will always denote such vector spaces.  We omit proofs in this section under the assumption that you have either already seen the results or can prove them on your own if you so desired.
\end{displayquote}

We plan to do differentiation in $\R ^d$, and to do this (instead of just in $\R$), it will be useful to know some basic facts about linear algebra.  The real motivation for taking this small side route is that we want to use Penrose's \emph{abstract index notation}.\footnote{This is conceptually different, but mechanically very similar to Einstein's index notation.  You might say that abstract index notation is choice-free Einstein index notation (the choice of course being a choice of basis).}

\begin{displayquote}
If you don't understand the details of this section your first time through, that's fine.  Learn what you can, and then come back as you need to when the concepts come up in the actual differentiation part of the chapter.
\end{displayquote}

We first discuss the \emph{dual space} and \emph{tensor product}.
\begin{dfn}[Dual space]\label{DualSpace}
The \emph{dual space}\index{Dual space} of $V$, $V^{\dagger}$, is defined to be
\begin{equation}
V^{\dagger}\coloneqq \Mor _{\Vect _{\R}}(V,\R ).
\end{equation}
$V^{\dagger}$ has the structure of a vector space by defining addition and scalar multiplication pointwise.  The elements of $V^{\dagger}$ are \emph{linear functionals}\index{Linear functionals} or \emph{covectors}\index{Covector}.
\begin{rmk}
Though we have not precisely defined in yet, the category $\Vect _F$, for $F$ a field, is the category whose objects are vector fields over $F$ and whose morphisms are linear transformations.  Thus, $\Mor _{\Vect _{\R}}(V,\R )$ is our fancy-schmancy notation for the vector space of linear functions from $V$ into $\R$.
\end{rmk}
\begin{rmk}
In other words, the elements of $V^{\dagger}$ take in elements of $V$ and spit out numbers.  This is actually not that foreign of a concept---for example, the derivative takes in a vector (the direction in which to differentiate) and spits out a number (the directional derivative in that direction).
\end{rmk}
\begin{rmk}
We reserve the notation $V^*$ for the \emph{conjugate-dual} (something we won't see in these notes)---this is why we write $V^{\dagger}$ instead of $V^*$.
\end{rmk}
\begin{rmk}
If $V$ comes with a topology, you're only going to want to look at the \emph{continuous} linear functionals.  Of course, you can look at all of them (including the discontinuous ones), but in comparison this space will be an ugly beast of a motherfucker.\footnote{Can you tell I write not much differently from how I speak?  ;-)}
\end{rmk}
\end{dfn}
Of critical importance is that the dual of the dual is the original vector space.\footnote{Careful:  It will frequently be the case that $V^{\dagger}$ \emph{is} isomorphic to $V$, but in a noncanonical way.  On the other hand $(V^{\dagger})^{\dagger}$ and $V$ are \emph{canonically isomorphic}.  The way to make this intuition precise requires more category theory than is probably helpful.  Suffice it to say, the idea is to show that the `constructions' (read ``functors'') are isomorphic, not the objects themselves.}
\begin{prp}\label{prp5.1.4}
The map $V\ni v\mapsto \phi _v\in (V^\dagger )^\dagger$, where $\phi _v:V\rightarrow \R$ is defined by
\begin{equation}
\phi _v(\omega )\coloneqq \omega (v)
\end{equation}
is an isomorphism.
\begin{rmk}
Warning:  This is \emph{false} in infinite dimensions.  You will see that we show that this map is injective and linear, and so by the Rank-Nullity Theorem (something specific to finite-dimensions) is an isomorphism.
\end{rmk}
\end{prp}
\begin{dfn}[Tensor product]\label{TensorProduct}
Let $V$ and $W$ be finite-dimensional real vector spaces.  Then, the \emph{tensor product}\index{Tensor product (of vector spaces)} of $V$ and $W$, $V\otimes W$\index[notation]{$V\otimes W$} is defined to be the set of all functions $T:V^{\dagger}\times W^{\dagger}\rightarrow \R$ such that
\begin{enumerate}
\item for each fixed $\omega \in V^{\dagger}$, the map $\eta \mapsto T(\omega ,\eta )$ is linear; and
\item for each fixed $\eta \in W^{\dagger}$, the map $\omega \mapsto T(\omega ,\eta )$ is linear.
\end{enumerate}
Let $v\in V$ and $w\in W$.  Then, the \emph{tensor product}\index{Tensor product (of tensors)}, $v\otimes w\in V\otimes W$\index[notation]{$v\otimes w$}, is defined by
\begin{equation}
[v\otimes w](\omega ,\eta )\coloneqq \omega (v)\eta (w).
\end{equation}
\begin{rmk}
To clarify, there are tensor products of \emph{vector spaces}, and then there are tensor products of \emph{vectors themselves}.  The tensor product of two vectors `lives in' the tensor product of the corresponding vector spaces.  And in fact, \emph{everything} in $V\otimes W$, while \emph{not} of the form $v\otimes w$ itself necessarily, can be written as a finite sum of elements of this form---see \cref{prp5.1.8} below.  (Elements of the form $v\otimes w$ are sometimes called \emph{pure} or \emph{simple}, as opposed to, e.g.~, $v_1\otimes w_1+v_2\otimes w_2$).
\end{rmk}
\begin{rmk}
In other words, $V\otimes W$ is the set of all \emph{bilinear}\index{Bilinear} maps on $V^{\dagger}\times W^{\dagger}$, where bilinear means that, if you fix all arguments except one, you obtain a linear map.
\end{rmk}
\begin{rmk}
In practice, I find it easier to think of the tensor product as the vector space spanned by guys of the form $v\otimes w$ (as opposed to bilinear maps on the cartesian product of the duals---ick).
\end{rmk}
\begin{rmk}
This is neither the most general nor the most elegant definition of the tensor product.  The `right' way to define the tensor product is, as usual, by finding the properties which uniquely characterize it.  Maybe I will update the notes at a later date to include this, but my personal feeling right now is that this would take us a bit too far astray (after all, we're not studying linear algebra or tensors for their own sake---for us, they're just a tool to do calculus).
\end{rmk}
\end{dfn}
\begin{prp}\label{prp5.1.8}
$V\otimes W=$ is the span of $\{ v\otimes w:v\in V,\ w\in W\}$.
\end{prp}
\begin{dfn}[Tensor]\label{Tensor}
A \emph{tensor}\index{Tensor} of rank $\coord{k,l}$ over $V$ is an element of
\begin{equation}
\underbrace{V\otimes \cdots \otimes V}_k\otimes \underbrace{V^{\dagger}\otimes \cdots \otimes V^{\dagger}}_l
\end{equation}
$k$ is the \emph{contravariant rank}\index{Contravariant rank} and $l$ is the \emph{covariant rank}\index{Covariant rank}.  If $l=0$, then the tensor is \emph{contravariant}\index{Contravariant tensor}, and if $k=0$, then the tensor is \emph{covariant}\index{Covariant tensor}.  If $T$ is a tensor of rank $\coord{k,l}$, then we shall write
\begin{equation}
T\indices{^{a_1\cdots a_k}_{b_1\cdots b_l}}
\end{equation}
to help remind us what type of tensor this is.\footnote{Don't let the indices mislead you---there are no choices being made.  Everything is ``coordinate-free'', even later when we start manipulating the indices themselves---it's still all coordinate-free.}
\begin{rmk}
\emph{Do not be sloppy and not stagger your indices!}  If you do, you will eventually make a mistake.  For example, later we will be raising and lowering indices.  Suppose I start with $T^{ab}$, I lower to obtain $T_b^a$, and then I raise again to obtain $T^{ba}$---I should obtain the same thing, but in general $T^{ab}\neq T^{ba}$, and so I have an error.  It may seem obvious to the point of being silly when I point it out like this, but this is a mistake that is easy to make if there is a big long computation in between the raising and lowering (especially if it's more than just $a$ and $b$ floating around).  And of course, you will never have this problem if you stagger:  $T^{ab}$ goes to $T\indices{^a_b}$ goes back to $T^{ab}$.
\end{rmk}
\begin{rmk}
I claim that this is likewise not that foreign of a concept.  In fact, there are so many examples you are familiar with that I don't even want to put them in a remark, so see the next example.for example, vectors (written $v^a$) themselves are tensors of type $\coord{1,0}$, covectors (or linear functionals) (written $\omega _a$) are of type $\coord{0,1}$, the dot product (written $g_{ab}$) is a tensor of type $\coord{0,2}$ (it takes in two vectors and spits out a number), linear transformations (written $T\indices{^a_b}$ are tensors of type $\coord{1,1}$,\footnote{It takes in a single vector an spits out a linear map from $V^{\dagger}$ to $\R$, that is, another vector (by \cref{prp5.1.4}).}
\end{rmk}
\end{dfn}
\begin{exm}
\begin{enumerate}
Disclaimer:  While none of the examples themselves make use of things we haven't done yet, some of the notation does (e.g.~$v^a\omega _a$).  Read onwards and come back later if this really bothers you.
\item Vectors (written $v^a$) themselves are tensors of type $\coord{1,0}$.
\item Covectors (or linear functionals) (written $\omega _a$) are of type $\coord{0,1}$.  For $\omega$ a linear functional and $v$ a vector, $\omega (v)$ is written as $v^a\omega _a$.
\item The dot product (written $g_{ab}$) is a tensor of type $\coord{0,2}$---it takes in two vectors and spits out a number, written $v\cdot w=v^aw^bg_{ab}$.
\item Linear transformations (written $T\indices{^a_b}$) are tensors of type $\coord{1,1}$---it takes in a single vector and spits out another vector (written $v^a\mapsto T\indices{^a_b}v^b$).  Note that it is $T\indices{^a_b}$ and not $T\indices{^b_a}$---your convention could go either way, but in the convention we choose the indices that are contracted during composition are closer together.
\end{enumerate}
\end{exm}

There are three key constructions involving tensors that we will need, the \emph{tensor product}, \emph{contraction}, and the \emph{dual vector}.  The tensor product we have already done in \cref{TensorProduct},\footnote{Well, I suppose we have to define the tensor products of \emph{arbitrary} tensors, as opposed to just vectors, but the definition in general is just an extension of the one we've already written down.} and so we simply explain the how to write the tensor product in index notation.
\begin{displayquote}
The tensor product of $[T_1]\indices{^{a_1\ldots a_{k_1}}_{b_1\ldots b_{l_1}}}$ and $[T_2]\indices{^{a_1\ldots a_{k_2}}_{b_1\ldots b_{l_2}}}$ is denoted
\begin{equation}
[T_1]\indices{^{a_1\ldots a_{k_1}}_{b_1\ldots b_{l_1}}}[T_2]\indices{^{a_1\ldots a_{k_2}}_{b_1\ldots b_{l_2}}}.
\end{equation}
That is, you literally just juxtapose them.
\end{displayquote}
We now turn to \emph{contraction}.
\begin{dfn}[Contraction]\label{Contraction}
Let $T\indices{^{a_1\ldots a_k}_{b_1\ldots b_l}}\coloneqq [v_1]^{a_1}\cdots [v_k]^{a_k}[\omega _1]_{b_1}\cdots [\omega _l]_{b_l}$ be a tensor of rank $\coord{k,l}$ (recall (\cref{prp5.1.8}) that every tensor can be written as a sum of tensors of this form).  Then, the \emph{contraction}\index{Contraction} of $T$ along the $a_i$ and $b_j$ index is defined to be
\begin{equation}
\begin{multlined}
T\indices{^{a_1\ldots a_{i-1}a_{i+1}\ldots a_k}_{b_1\ldots b_{j-1}b_{j+1}\ldots b_l}}\coloneqq \\ \omega _j(v_i)\cdot [v_1]^{a_1}\cdots [v_{i-1}]^{a_{i-1}}[v_{i+1}]^{a_{i+1}}\cdots [v_k]^{a_k}[\omega _1]_{b_1}\cdots [\omega _{j-1}]_{b_{j-1}}[\omega _{j+1}]_{b_{j+1}}\cdots [\omega _l]_{b_l}.
\end{multlined}
\end{equation}
The contraction of tensor that is a sum of simple tensors is defined to be the sum of the contraction of those simple tensors.
\begin{rmk}
In general, the way one can `decompose' a general tensor as a sum of simple ones is not unique, so we must technically check that this is well-defined.
\end{rmk}
\begin{rmk}
This might \emph{look} atrocious, but it's actually quite simple.  Covectors take in vectors and spit-out numbers, and so the contraction of a tensor product in its $a_i$ and $b_j$ index is formed by plugging in the $i^{\text{th}}$ vector into the $j^{\text{th}}$ covector.
\end{rmk}
\begin{rmk}
Keep in mind that you can \emph{only} contract upper-indices (contravariant) with lower (covariant) ones.
\end{rmk}
\begin{rmk}
A couple of examples:  The only possible contraction of the tensor $v^a\omega _b$ is written $v^a\omega _a$ and this of course is just $\omega (v)$.  If $T\indices{^a_b}$ is a linear transformation and $v^a$ is a vector, then the contraction $T\indices{^a_b}v^b$ is just $T(v)$, that is, the image of $v$ under the linear transformation $T$.
\end{rmk}
\begin{rmk}
The index $k$ in $[v_k]^{a_k}$ is part of the name of the vector---the entire name is $v_k$, and then the notation $[v_k]^{a_k}$ reminds us that $v_k$ is a $\coord{1,0}$-tensor, i.e.~just a vector.
\end{rmk}
\end{dfn}
We need one more ingredient before we can get to actual differentiation, namely that of a \emph{metric}.
\begin{dfn}[Metric (on a vector space)]\label{MetricVectorSpace}
A \emph{metric}\index{Metric (on a vector space)} $g$ on $V$ is a covariant tensor of rank $2$ such that
\begin{enumerate}
\item \label{MetricVectorSpace.Symmetry}(Symmetry) $g(v_1,v_2)=g(v_2,v_1)$; and
\item \label{MetricVectorSpace.Nonsingularity}(Nonsingularity)\index{Nonsingular (metric)} the map from $V$ to $V^{\dagger}$ defined by $v\mapsto g(v,\blankdot )$, where $g(v,\blankdot )$ is the linear functional which sends $w$ to $g(v,w)$, is an isomorphism of vectors spaces.
\end{enumerate}
\begin{rmk}
If $v^a$ is a vector, then we write $v_a\coloneqq g_{ab}v^b$.  $v_a$ is the \emph{dual vector}\index{Dual vector} (which itself is not a vector---it's a covector) of $v^a$.  \emph{Nonsingualirty} is key because it allows us to reverse this process.  If $\omega _a$ is a covector, then because the map $v^a\mapsto v_a$ is an \emph{isomorphism}, there is a unique vector, written $\omega ^a$, that is equal to $\omega _a$ under this map.
\end{rmk}
\begin{rmk}
\ref{MetricVectorSpace.Symmetry} can be written $g_{ab}=g_{ba}$.  Also note that $g(v_1,v_2)=[v_1]^a[v_2]^bg_{ab}$.
\end{rmk}
\begin{rmk}
The idea of a notion of a metric on a vector space and a metric on a set (in the context of uniform space theory) have little to nothing to do with each other.  It is merely a coincidence of terminology that is so ingrained that even I dare not go against it.
\end{rmk}
\begin{rmk}
The term ``metric'' in this sense of the word should really not be thought of as a sort of distance, but rather as a sort of dot product.  Indeed, you can verify that the dot product is a metric, and furthermore, in a sense that we don't bother to make precise, every positive-definite metric (on a vector space) is equivalent to the usual euclidean dot product.  There is \emph{some} connection with the other notion of metric, however---positive-definite metrics give us norms (the square-root $g(v,v)$), which in turn gives us a metric (in the other sense).
\end{rmk}
\begin{rmk}
Nonsingularity is usually replaced with the requirement that $g(v,w)=0$ for all $w$ implies that $v=0$ (called \emph{nondegeneracy}\index{Nondegenrate (metric)}.  In finite dimensions, this is equivalent to nonsingularity (by the Rank-Nullity Theorem).  In infinite dimensions, however, they are not equivalent, and it is nonsingularity that we want (so that we can raise and lower indices).
\end{rmk}
\end{dfn}
It's worth nothing that, everything \emph{except} raising and lowering indices we can do without a metric.  To raise and lower indices, we do need that \emph{extra} structure.  In particular, if you pick a different metric, then your meaning of $v_a$ will change even though the metric does not appear explicitly in this notation.

In summary:
\begin{enumerate}
\item The tensor product of two vectors $v^a$ and $w^a$, written $v^aw^b$, is defined to be the bilinear map that sends the pair of covectors $\coord{\omega _a,\eta _a}$ to $(\omega _av^a)(\eta _aw^a)$.  In practice, it's not particularly helpful to think of what this is\footnote{When you add $2$ to $3$ do you think about $2$ being an equivalence class of sets with respect to the equivalence relation of isomorphism in the category of sets?  Here, I'll help you out:  No, you do not.}---in practice what matters is can you manipulate them.
\item A general tensor of rank $\coord{k,l}$is an element in the tensor product of $k$ copies of $V$ with $l$ copies of $V^{\dagger}$.
\item The definition of the tensor product of vectors can be extended to the tensor product of any tensors.  In index notation, this is denoted simply by juxtaposition.
\item We can contract indices.
\item If we have a metric, we can also raise and lower indices.
\end{enumerate}

\section{The definition}

One thing that I personally found conceptually confusing with differentiation in $\R ^d$ itself that was elucidated for me when passing to the study of more general manifolds was the distinction between a \emph{vector} and a \emph{point}.  The problem in $\R ^d$ of course is that the space of points and the space of factors are effectively the same thing, they are both given by a $d$-tuple of real numbers, when in fact, they are really playing quite different roles.  The points tell you ``where'' we are and the vectors tell you ``what direction'' to go in.  In a general manifold, the points that tell you ``where'' form the points of the space itself and the vectors do \emph{not} live in the entire space itself, but rather the tangent spaces.

Thus, while we have no intention of doing manifold theory in general,\footnote{In contrast to topology, for example, you do have to prove essentially all of your results in $\R ^d$ first and \emph{then} extend them to arbitrary manifolds, whereas in principle you can prove all the results about topology you ever wanted without even mentioning $\R$.  This is one reason among others why we do general topology but not manifold theory.} we will make use of some suggestive notation that comes from the theory.
\begin{displayquote}
Throughout this chapter, the symbol $\R ^d$ will be used to denote $d$-dimensional euclidean space \emph{as a metric space}.  For each $x\in \R ^d$, we define $\tangent[\R ^d][x]$, the \emph{tangent space at $x$ in $\R ^d$}\index{Tangent space} to be the \emph{metric vector space} $\R ^d$ (with metric being the dot product).  Furthermore, we declare that $\tangent[\R ^d][x_1]\neq \tangent[\R ^d][x_2]$ for $x_1\neq x_2$.\footnote{There are many ways to do this, but one way, for example, is to take $\tangent[\R ^d][x]\coloneqq \R ^d\times \{ x\}$.}  We will often, but not always, use abstract index notation for vectors $v^a\in \tangent[\R ^d][x]$ to help remind us that they are to be thought of as vectors instead of points.  It will sometimes be useful to assign other vectors spaces to each point (for example, for a function $f:\R ^d\rightarrow \R ^m\coloneqq V$).  In general, the assignment of a vector space $V_x$ to each point of $\R ^d$ will be called a \emph{vector bundle}\index{Vector bundle} on $\R ^d$.  The assignment to each point of the tangent space is a specal vector bundle called the \emph{tangent bundle}\index{Tangent bundle}.
\end{displayquote}
In particular, as sets, we might have that $\R ^d=\tangent[\R ^d][x]$, but that's it---the two objects don't even live in the same category, and so it doesn't even make sense to ask whether there is some isomorphism between them (unless you forget some of the structure, in which case you're actually changing the object).  For example, $\R ^d$ is just a metric space---you cannot add any two of its elements.  Tangent vectors, elements of $\tangent[\R ^d][x]$, on the other hand, we can add just fine.

To clarify, this is not actually how the definition goes in general.  The general definition of the tangent space requires us to first be able to talk about manifolds, which in turn requires us to know how differentiation in $\R ^d$ works, which of course we have not done yet.  Thus, we are making use of this notation only to help clarify the study of differentiation in $\R ^d$.  In principle, once enough of this theory has been developed so that we can talk about tangent spaces in general, we would replace the above with the `actual' definition.  Likewise, this is not the actual definition of vector bundles---for us, the term ``vector bundle'' is just a phrase in the English language that we are going to use to communicate mathematical ideas.  It requires basic manifold theory to define vector bundles properly.

This speak of tangent spaces allows us to make an important definition.
\begin{dfn}[Tensor field]\label{TensorField}
A \emph{tensor field}\index{Tensor field} of rank $\coord{k,l}$ is a function on $\R ^d$ whose value at $x$ is a rank $\coord{k,l}$ tensor on $\tangent[\R ^d][x]$.
\begin{rmk}
It's just an assignment of a tensor to every point in $\R ^d$.  For example, a vector field is an assignment of a vector to every point.
\end{rmk}
\end{dfn}

Finally, with this (hopefully elucidating) notation in hand, we can define the derivative.
\begin{dfn}[Derivative (of a function)]\index{Derivative}
\begin{savenotes}
Let $f:\R ^d\rightarrow \R$, let $x\in \R ^d$, and let $v^a\in \tangent[\R ^d][x]$.  Then, the \emph{derivative}\index{Derivative} of $f$ at $x$ in the direction $v^a$, $v^a\nabla _af(x)$, is defined by
\begin{equation}\label{DifferenceQuotient}
v^a\nabla _af(x)\coloneqq \lim _{h\to 0}\frac{f(x+hv)-f(x)}{h}.
\end{equation}
If this limit exists, then $f$ is \emph{differentiable}\index{Differentiable} at $x$ in the direction $v^a$.  If $f$ is differentiable at $x$ for all $v^a$, then $f$ is differentiable at $x$.  If $f$ is differentiable at $x$ for all $x$, then $f$ is differentiable.  The expression on the right-hand side of \eqref{DifferenceQuotient} is the \emph{difference quotient}\index{Difference Quotient}.
\begin{rmk}
The notation $v^a\nabla _af$ is obviously suggestive.  For fixed $f$ and $x$, the map $v^a\mapsto [v^a\nabla _af](x)$ is linear in $v^a$, and so defines a \emph{linear functional}, that is to say, $\nabla _af(x)\in \tangent[\R ^d][x]^{\dagger}$, or equivalently, that $\nabla _af$ is a covector field on $\R ^d$.  This covector field is the \emph{gradient}\index{Gradient} of $f$ at $x$.
\end{rmk}
\begin{rmk}
In one dimension, there is essentially only one choice of $v^a$---everything else is a scalar multiple.  Thus, in one dimension, we \emph{always} take $v^a=1\in \tangent[x][\R ]\cong _{\Vect _\R}\R$ and write
\begin{equation}
\frac{\dif}{\dif x}f(x)\coloneqq v^a\nabla _af(x).
\end{equation}
\end{rmk}
\begin{rmk}
Perhaps more accurate notation would have been $[v^a\nabla _af](x)$, that is, $v^a\nabla _af$ is a function (in the case that $f$ is differentiable anyways), and so $v^a\nabla _af(x)$ is the value of $v^a\nabla _af$ at $x$, as opposed to, the derivative of the function $f(x)$, which is just $0$ of course.\footnote{I know to some this may seem pedantic, but $f$ is the function, $f(x)$ is the value at $x$ of the function, so that $f(x)$ is just a number.}  The point is:  you must compute the derivative, and \emph{then} plug-in $x$.  It might seem silly in such a simple context, but in much more complicated contexts I myself have made this very mistake.  For example, suppose you are computing a functional derivative (to find a noether charge, say) of some action functional in physics, and you want to see that this quantity is conserved `on-shell' (i.e.~when the equations of motion hold)---you cannot use the equations of motion before you finish computing the noether charge `off-shell':  that's cheating (and more importantly, possibly just plain wrong)!\footnote{Once again, if the physics analogy is meaningless to you, just ignore it.}
\end{rmk}
\begin{rmk}
Suppose that $f$ is differentiable.  Then, $v^a\nabla _af$ itself is a function on all of $\R ^d$, and so we may differentiate\footnote{If the limit exists, of course} this as well to obtain
\begin{equation}
w^b\nabla _b(v^a\nabla _af)=w^bw^a\nabla _b\nabla _af.
\end{equation}
(Warning:  This equality will not hold in general if $v^a$ is a nonconstant function of $x$.)  Just as $\nabla _af(x)$ defined a covector on $\tangent[\R ^d][x]$, so to does $\nabla _b\nabla _af(x)$ defines a covariant $2$-tensor on $\tangent[\R ^d][x]$.\footnote{Note that a covariant $2$-tensor is an element of $\tangent[\R ^d][x]^\dagger \otimes \tangent[\R ^d][x]^\dagger$.  When we say it is a ``covariant $2$-tensor \emph{on} $\tangent[\R ^d][x]$'', this is what we mean---it's neither a (real-valued) function nor an element of $\tangent[\R ^d][x]$.}
\end{rmk}
\begin{rmk}
If all higher derivatives of $f$ exist, then $f$ is \emph{smooth}\index{Smooth}.
\end{rmk}
\begin{exr}
Show that $v^a\mapsto v^a\nabla _af(x)$ is in fact linear.
\end{exr}
\end{savenotes}
\end{dfn}
We can use the definition of the derivative of a function to define the derivative for \emph{all} tensor fields.  The idea is that, by plugging-in enough vectors and covectors, all tensor fields reduce to just a function.
\begin{dfn}[Derivative (of tensors)]\label{DerivativeTensor}
Let $T\indices{^{a_1\ldots a_k}_{b_1\ldots b_l}}$ be a tensor field of rank $\coord{k,l}$ on $\R ^d$.  Then, the \emph{gradient}\index{Gradient (of a tensor field)}, $\nabla _aT\indices{^{a_1\ldots a_k}_{b_1\ldots b_l}}$, is defined by
\begin{equation}
\begin{multlined}
v^a[\omega _1]_{a_1}\cdots [\omega _k]_{a_k}[v_1]^{b_1}\cdots [v_l]^{b_l}\nabla _aT\indices{^{a_1\ldots a_k}_{b_1\ldots b_l}}= \\ v^a\nabla _a\left( [\omega _1]_{a_1}\cdots [\omega _k]_{a_k}[v_1]^{b_1}\cdots [v_l]^{b_l}\nabla _aT\indices{^{a_1\ldots a_k}_{b_1\ldots b_l}}\right) 
\end{multlined}
\end{equation}
for all covectors $\omega _1,\ldots ,\omega _k$ and vectors $v,v_1,\ldots ,v_l$.
\begin{rmk}
This makes sense because the thing inside the gradient on the right-hand side is just a function.
\end{rmk}
\begin{rmk}
Among other things, we can how differentiate functions which take their values in $\R ^m$, as we just interpret such a function as a vector field on $\R ^d$.\footnote{We are cheating a bit.  $\R ^m$ is \emph{not} the tangent space of any point---you can tell because it might not even have the right dimension.  The appropriate way to deal with this is to attach a new vector space $V_x$ to each point, where $V_x\cong _{\Vect _{\R}}(\R ^e)$, and then interpret the value $f(x)$ as an element of $V_x$.  Fortunately, this has no effect on the above definition.  For such functions, I will write $f^\alpha$, to remind us that $f^\alpha$ lives in a different vector bundle than usual.}
\end{rmk}
\begin{rmk}
Thus, the gradient in particular shifts the covariant rank of a tensor up by $1$ (for example, as functions are just $\coord{0,0}$ tensors, it takes functions to covector fields).
\end{rmk}
\end{dfn}

\section{Basic facts}

\begin{exr}[A function that is not differentiable]
Find an example of a function that is not differentiable.
\end{exr}
\begin{exr}[A function differentiable in one but not all directions]
Find a function $f:\R ^2\rightarrow \R$ that is differentiable along the $x$-axis at the origin, but not along the $y$-axis.
\end{exr}
\begin{exr}[A continuous function that is not differentiable]
Find an example a function that is continuous but not differentiable.
\end{exr}
What you will not be able to do, however, is find an example of a differentiable function that is not continuous.
\begin{prp}
Let $f:\R ^d\rightarrow \R$ be a function and let $x\in \R ^d$.  Then, if $f$ is differentiable at $x_0$, then it is continuous at $x_0$.
\begin{proof}
Consider
\begin{equation}
f(x)-f(x_0)=\left( \frac{f(x)-f(x_0)}{x-x_0}\right) (x-x_0).
\end{equation}
Taking the limit of both sides as $x\to x_0$, because $\lim _{x\to x_0}\frac{f(x)-f(x_0)}{x-x_0}=f'(x_0)$ exists (and is finite), we find that $\lim _{x\to x_0}(f(x)-f(x_0))=0$, so that $f$ is continuous at $x_0$.
\end{proof}
\end{prp}

\begin{prp}[Algebraic Derivative Theorems]\index{Algebraic Derivative Theorems}\label{AlgebraicDerivativeTheorems}
Let $f,g:\R ^d\rightarrow \R$ be differentiable.\footnote{Everything works just as well (with the same proof) if you just assume it is differentiable in some direction at some point, but the notation is more tedious.} and let $\alpha \in \R$.  Then,
\begin{enumerate}
\item \label{AlgebraicDerivativeTheorems.Linearity}(Linearity) $\nabla _a(f+g)=\nabla _a+\nabla _ag$;
\item \label{AlgebraicDerivativeTheorems.Homogeneity}(Homogeneity) $\nabla _a(\alpha f)=\alpha \nabla _af$;
\item \label{AlgebraicDerivativeTheorems.ProductRule}(Product Rule)\index{Product Rule}$\nabla _a(fg)=(\nabla _af)g+f(\nabla _ag)$;\footnote{I would get in the habit of not mixing-up the order of $f$ and $g$ (e.g.~by writing $(\nabla _af)g+(\nabla _ag)f$ or something of the like).  It won't matter for us, but it can and will latter when you're working with things that are not commutative (the cross-product of vectors is probably the most elementary example).} and
\item \label{AlgebraicDerivativeTheorems.QuotientRule}(Quotient Rule)\index{Quotient Rule} $\nabla _a\left( \frac{f}{g}\right) =\frac{(\nabla _af)g-f(\nabla _ag)}{g^2}$ whenever $g\neq 0$.
\end{enumerate}
\begin{rmk}
The first three are true just as well for $f$ and $g$ arbitrary tensor fields, with essentially the same exact proofs (the juxtaposition denotes the tensor product of course).  The Quotient Rule does not make sense, however, as in general you cannot invert tensors.
\end{rmk}
\begin{proof}
\ref{AlgebraicDerivativeTheorems.Linearity} and \ref{AlgebraicDerivativeTheorems.Homogeneity} follows straight from the corresponding results above limits---see \cref{AlgebraicLimitTheorems}\ref{enmAlgebraicLimitTheorems.i} and \cref{AlgebraicLimitTheorems}\ref{enmAlgebraicLimitTheorems.ii}.

As for \ref{AlgebraicDerivativeTheorems.ProductRule}, we have
\begin{equation}
\begin{multlined}
\frac{f(x+hv)g(x+hv)-f(x)g(x)}{h}=\footnote{We added and subtracted $f(x)g(x+hv)$.} \\ \left( \frac{f(x+hv)-f(x)}{h}\right) g(x+hv)+f(x)\left( \frac{g(x+hv)-g(x)}{h}\right) ,
\end{multlined}
\end{equation}
and so taking limits gives us the Product Rule.

Similarly, the proof of the Quotient Rule amounts to just algebraic manipulation of the difference quotient:
\begin{equation}
\frac{\frac{f(x+hv)}{g(x+hv)}-\frac{f(x)}{g(x)}}{h}=\frac{\left( \frac{f(x+hv)-f(x)}{h}\right) g(x)-f(x)\left( \frac{g(x+hv)-g(x)}{h}\right)}{g(x+hv)g(x)}.
\end{equation}
\end{proof}
\end{prp}
\begin{prp}[Chain Rule]\index{Chain rule}\label{ChainRule}
Let $f^\alpha :\R ^d\rightarrow \R ^m$ and $g^\mu :\R ^m\rightarrow \R ^n$ be differentiable.\footnote{The $\mu$ index is used to remind us that $g^\mu$ lives in a $\R ^n$ (as opposed to $\R ^m$ or $\R ^d$).}  Then,
\begin{equation}
\nabla _a[g\circ f]^\mu (x)=\nabla _\alpha g^\mu (f(x))\nabla _af^\alpha (x).
\end{equation}
\begin{rmk}
One of the things I really love about index notation is that it almost dictates what the answer has to be.  How many ways can construct a tensor with one $\R ^d$ covariant index and one $\R ^n$ contravariant index using only $f^\alpha$, $g^\mu$, and their derivatives?
\end{rmk}
\begin{proof}
To prove this, by the definition of the derivative of tensors, we need to show that
\begin{equation}
\omega _\mu \nabla _a[g\circ f]^\mu (x)=\omega _\mu \nabla _\alpha g^\mu (f(x))\nabla _af^\alpha (x)
\end{equation}
for all covectors (living in $\R ^n$) $\omega _\mu$.  In particular, it suffices to prove the result for $g:\R ^m\rightarrow \R$ (because now $\omega _\mu g^\mu :\R ^m\rightarrow \R$).

Let $v^a$ be a constant vector field on $\R ^d$.  Then, what we actually want to show is
\begin{equation}
v^a\nabla _a[g\circ f](x)=\left( \nabla _\alpha g(f(x))\right) \left( v^a\nabla _af^\alpha (x)\right) ,
\end{equation}
as $v^a$ is arbitrary.  On one hand
\begin{equation}\label{5.2.18}
v^a\nabla _a[g\circ f](x)=\lim _{h\to 0}\frac{g\left( f(x+hv)\right) -g(f(x))}{h}
\end{equation}
and on the other hand
\begin{equation}
\left( \nabla _\alpha g(f(x))\right) \left( v^a\nabla _af^\alpha (x)\right) =\lim _{h\to 0}\frac{1}{h}\left[ g\left( f(x)+hv^a\nabla _af(x)\right) -g(f(x))\right]
\end{equation}
Thus, we want to show that
\begin{equation}
f(x+hv)=f(x)+hv^a\nabla _af(x)\text{ as }h\to 0,
\end{equation}
but of course this is just the very definition of the derivative.
\end{proof}
\end{prp}

\section{The exponential function}

So, we've proven several properties about how to manipulate derivatives, but what is there to differentiate?  Polynomials?  That's no fun.  Let's find a function even easier to differentiate.  In fact, let's see if we can find a function that is equal to its own derivative
\begin{equation}
\tfrac{\dif}{\dif x}f(x)=f(x).
\end{equation}
Don't be a smart-ass---zero doesn't count.  Of course, you already know the answer---or so you think you do.  Can you define $\exp (x)$?  For what it's worth, you do have the tools to do so at this point, but it's quite likely that someone told you once upon a time that $\e \approx 2.718\ldots $ and then $\exp (x)\coloneqq \e ^x$.  If it's not clear to you at this point that this is just complete and utter nonsense, then apparently I'm not very good at writing mathematical exposition.\footnote{Or maybe you're just stupid.  (Disclaimer:  That was a joke.)}  What you could do, however, is define $\exp$ by its power-series, $\exp (x)\coloneqq \sum _{m\in M}\frac{x^m}{m!}$, but where did that formula come from?  Your ass?  No.  The proper way to define the exponential function is that it is the unique function from $\R$ to $\R$ that (i) is equal to its own derivative and (ii) is $1$ at $0$.\footnote{The unique function that that is equal to its own derivative and is equal to $0$ at $0$ is the function that is everywhere $0$.  $1$ is the next most obvious choice, as opposed to, say $\sqrt{\uppi}$.}  Of course, as always, we can't just go around asserting things like this exist willy-nilly.  Who can even comprehend the chaos that might ensue?  We must \emph{prove} that such a thing exists, and that only one such thing exists.  The theorem that does this for us (and a whole lot more) is Picard's Existence Theorem.
\begin{thm}[Picard's Existence Theorem]\index{Picard's Existence Theorem}\label{PicardsExistenceTheorem}
Let $F:\R \times \R \rightarrow \R$ be a function such that
\begin{enumerate}
\item for each fixed $y\in \R$, the map $x\mapsto F(x,y)$ is continuous; and
\item for each fixed $x\in \R$, the map $y\mapsto F(x,y)$ is lipschitz-continuous,\footnote{The definition is in \cref{BoundedMap} in case you've forgotten (or just plain missed it).}
\end{enumerate}
and let $\coord{x_0,y_0}\in \R \times \R$.  Then, for some $\varepsilon _0$, there is a unique function $f:(x_0-\varepsilon _0,x_0+\varepsilon _0)\rightarrow \R$ such that
\begin{equation}
\frac{\dif}{\dif x}f(x)=F\left( x,f(x)\right) \text{ and }f(x_0)=y_0.
\end{equation}
\begin{rmk}
Picard's theorem is about the existence and uniqueness of solutions to \emph{differential equations}.  The $F$ is supposed to be thought of as the differential equation itself, or at least everything in the differential equation that does not contain a derivative.  For example, in our case of primary interest ($\frac{\dif}{\dif x}f(x)=f(x)$), $F$ will be just $F(x,y)\coloneqq y$.
\end{rmk}
\begin{rmk}
The ``there exists some $\varepsilon _0$'' business is a result of the fact that we may not be able to find a solution to the differential equation on all of $\R$---in general, we can only do so on a neighborhood of where we started ($x=x_0$).  For example, consider the differential equation $\frac{\dif}{\dif x}f(x)=-f(x)^2$ (so for $F(x,y)\coloneqq -y^2$) with initial value $x_0=1=y_0$.  The unique solution will be $f(x)=\frac{1}{x}$, which you cannot extend past $0$ (so that $\varepsilon _0=1$ is the best we can do).
\end{rmk}
\end{thm}



\appendix

\chapter{Sets and categories}

\section{Basic set theory}

\subsection{What is a set?}\label{sbsA.1.1}

For the most part, we will completely ignore any set-theoretic concerns in these notes, but before we do just blatantly ignore any potential issues, we should first probably (attempt to) justify this dismissal.

What is a set?  Of course, intuitively, a set is just a thing that `contains' a bunch of other things, but this itself is not a precise mathematical definition, so how do we come up with a precise mathematical definition of the idea of a set?  One way to do this would be to attempt to develop an axiomatic set theory, but there is a certain `circularity' problem in doing this.

The term ``axiomatic set theory'' here refers to any collection of axioms which attempt to make precise the intuitive idea of a set.  In a given theory, however, the symbols which we make use of to write down the axioms themselves form a \emph{set}.  The point is that, in attempting to write down a mathematically precise definition of a set, one must make use of the naive notion of a set.

Of course this example might not be very convincing.  Why not just not think of all the symbols together and just think of them individually?  It is true that if you fudge things around a bit you may be able to convince yourself that you're not really making use of the naive notion of a set here.  That being said, even if you can convince yourself that you can get around the problem of first requiring a `set' of symbols, sooner or later, in attempting to make sense out of an axiomatic set theory, you will need to make use of the naive notion of a set.

Because of this, we consider the idea of a set to be so fundamental as to be undefinable, and we simply assume that we can freely work with this intuitive idea of a collection of things all thought of as one thing, namely a set.

One has to be careful however.  Naive set theory has paradoxes, a famous example of which is Russel's paradox\index{Russel's paradox}.  Consider for example the set\footnote{Hopefully you have seen notation like this before.  If not, really quickly skip ahead to \cref{sbsA.1.2} \nameref{sbsA.1.2} to look-up the meaning of this notation.}
\begin{equation}\label{A.1.1}
X\coloneqq \left\{ Y:Y\notin Y\right\} .
\end{equation}
Is $X\in X$?  One resolution of this paradox is that it is nonsensical to construct the set of \emph{all} things satisfying a certain property.  Whenever you construct a set in this manner, your objects have to be already `living inside' some other set.  For example, we can write
\begin{equation}
X\coloneqq \left\{ Y\in Z:Y\notin Y\right\}
\end{equation}
for some fixed set $Z$.  Russel's paradox now becomes the statement that $X\notin Z$.

This is still somehow not enough.  For example, if you turn to \cref{exm1.2.2}, the category of sets, you'll see that we do need to make use of the notion of the collection of all sets.  To get around this, we think of the $X$ of \eqref{A.1.1} as just being on an entirely new `level' of set:  for example, one way to get around this is to understand that $Y$ varies over all \emph{sets} and then to just interpret Russel's paradox as saying that $X$ is a set-like object that is not itself a set (often called a \emph{proper class}\index{Proper class}).

\begin{rmk}
Do not freak-out if we refer to a set-like object, which is in fact not as a set, as a ``set''.  For us, this is just a convenient abuse of language.  Ultimately, whether or not a set-like object is truly a set or not won't matter too much to us.
\end{rmk}

The content of this section was meant only to convince you that (i) there is no way of getting around the fact that the idea of collecting things together is undefinably fundamental, and that (ii) ultimately this naive idea is not paradoxical (if you cheat a little).

Disclaimer:  I am neither a logician nor a set-theorist, so take what I say with a grain of salt.

\subsection{The absolute basics}\label{sbsA.1.2}

\subsubsection{Some comments on logical implication}

The word \emph{iff}\index{Iff} is short-hand for the phrase \emph{if and only if}.  So, for example, if $A$ and $B$ are statements, then the sentence ``$A$ iff $B$.'' is logically equivalent to the two sentences ``$A$ if $B$.'' and ``$A$ only if $B$.''.  In symbols, we write $B\Rightarrow A$ and $A\Rightarrow B$ respectively.  The former logical implication is perhaps more obvious; the other might be slightly trickier to translate from the English to the mathematics.  The way you might think about it is this:  if $A$ is true, then, because $A$ is true \emph{only if} $B$ is true, it must have been the case that $B$ was true too.  Thus, ``$A$ only if $B$.'' is logically equivalent to ``$A$ implies $B$.''.

For us, the term \emph{statement}\index{Statement} will refer to something that is either true or false.  If $A$ and $B$ are statements, then $A\Rightarrow B$ are statements:  $\text{True}\Rightarrow \text{True}$ is considered true, $\text{True}\Rightarrow \text{False}$ is considered false, $\text{False}\Rightarrow \text{True}$ is considered true, and $\text{False}\Rightarrow \text{False}$.  Hopefully the first two of these make sense, but how does one understand why it should be the case that $\text{False}\Rightarrow \text{True}$ is true?  To see this, I think it helps to first note the following.\footnote{The symbol ``$\forall$\index[notation]{$\forall$}'' in English reads ``for all''.  Similarly, the symbol ``$\exists$\index[notation]{$\exists$}'' is read as ``there exists''.}
\begin{textequation}[A.1.3]
``$\forall x\in X,P(x)$.'' is logically equivalent to ``$x\in X\Rightarrow P(x)$.'',
\end{textequation}
where $P(x)$ is a statement that depends on $x$.

Now consider the following example in English.
\begin{textequation}
Every pig on Mars owns a shotgun.
\end{textequation}
Is this statement true or false?  Under the (hopefully legitimate assumption) that there is no pig on Mars at all, my best guess is that most native English speakers would say that this is a true statement.  In any case, this is mathematics, not linguistics, and for the sake of definiteness, we simply declare a statement such at this to be \emph{vacuously true} (unless of course there are pigs on Mars, in which case we would need to determine if they all owned shotguns).  This example is meant to convince you that, in the case that $X$ is empty, it is reasonable to declare the statement $\forall x\in X,P(x)$ to be true for tautological reasons.

Now, appealing back to \eqref{A.1.3}, hopefully it now also seems reasonable to declare statements of the form $\text{False}\Rightarrow Q$ to be true, likewise for tautological reasons.

\subsubsection{Sets}

The idea of a set is something that contains other things.
\begin{textequation}
If $X$ is a \emph{set} which contains an \emph{element} $x$, then we write $x\in X$.  Two sets are equal iff they contain the same elements.
\end{textequation}
\begin{dfn}[Empty-set]
The \emph{empty-set}\index{Empty-set}, $\emptyset$, is the set $\emptyset \coloneqq \{ \}$.  That is, it is the set which contains no elements.
\end{dfn}
\begin{rmk}
If ever you see an equals sign with a colon in front of it (e.g.~in ``$\emptyset \coloneqq \{ \}$''), it means that the equality is true \emph{by definition}.  This is used in definitions themselves, but also outside of definitions to serve as a reminder as to why the equality holds.\index[notation]{$\coloneqq $}
\end{rmk}
\begin{dfn}[Subset]
Let $X$ and $Y$ be sets.  Then, $X$ is a \emph{subset}\index{Subset} of $Y$ iff whenever $x\in X$ it is also the case that $x\in Y$.  If $X$ is a subset of $Y$, we write $X\subseteq Y$.
\end{dfn}
\begin{rmk}
Generally speaking we put slashes through symbols to indicate that the statement that would have been conveyed without the slash is false.  For example, $x\notin X$ means that $x$ is not an element of $X$, the statement that $X\not \subseteq Y$ means that $X$ is not a subset of $Y$, etc..
\end{rmk}
\begin{exr}
Let $X$ and $Y$ be sets.  Show that $X=Y$ iff $X\subseteq Y$ and $Y\subseteq X$.
\end{exr}
\begin{dfn}[Proper subset]\label{ProperSubset}
Let $X$ be a subset of $Y$.  Then, $X$ is a\emph{proper}\index{Proper subset}, and write $X\subset Y$\index[notation]{$X\subset Y$}, iff there is some $y\in Y$ that is not also in $X$.
\end{dfn}
\begin{displayquote}
Let $X$ be a set, let $\mathcal{P}$ be a property that an element in $X$ may or may not satisfy, and let us write $\mathcal{P}(x)$ iff $x$ satisfies the property $\mathcal{P}$.  Then, the notation
\begin{equation*}
\left\{ x\in X:\mathcal{P}(x)\right\}
\end{equation*}
is read ``The set of all elements in $X$ such that $\mathcal{P}(x)$.'' and represents a set whose elements are precisely those elements of $X$ for which $\mathcal{P}$ is true.  Sometimes this is also written as
\begin{equation*}
\left\{ x\in X|\mathcal{P}(x)\right\} ,
\end{equation*}
but our personal opinion is that this can look ugly (or even slightly confusing) if, for example, $\mathcal{P}(x)$ contains an absolute value in it:
\begin{equation*}
\left\{ x\in \R |\abs{x}<1\right\} .
\end{equation*}
\end{displayquote}
\begin{dfn}[Complement]
Let $X$ and $Y$ be sets.  Then, the \emph{complement}\index{Complement} of $Y$ in $X$, $X\setminus Y$\index[notation]{$X\setminus Y$}, is
\begin{equation}
X\setminus Y\coloneqq \{ x\in X:x\notin Y\} .
\end{equation}
If $X$ is clear from context, sometimes we write $Y^{\comp}\coloneqq X\setminus Y$\index[notation]{$Y^{\comp}$}.
\end{dfn}
\begin{dfn}[Union and intersection]
Let $A,B$ be subsets of a set $X$.  Then, the \emph{union}\index{Union} of $A$ and $B$, $A\cup B$\index[notation]{$A\cup B$}, is
\begin{equation}
A\cup B\coloneqq \left\{ x\in X:x\in A\text{ or }x\in B\right\} .
\end{equation}
The \emph{intersection}\index{Intersection} of $A$ and $B$, $A\cap B$\index[notation]{$A\cap B$}, is
\begin{equation}
A\cap B\coloneqq \left\{ x\in X:x\in A\text{ and }x\in B\right\} .
\end{equation}
\end{dfn}
\begin{dfn}[Disjoint and intersecting]
Let $A,B$ be subsets of a set $X$.  Then, $A$ and $B$ are \emph{disjoint}\index{Disjoint} iff $A\cap B=\emptyset$.  $A$ and $B$ \emph{intersect}\index{Intersect} (or \emph{meet}\index{Meet}) iff $A\cap B\neq \emptyset$.
\end{dfn}
\begin{exr}[De Morgan's Laws]\index{De Morgan's Laws}\label{DeMorgansLaws}
Let $\{ S_i\subseteq X:i\}$ be a collection of subsets of a set $X$.  Show that
\begin{equation}
\left( \bigcup _iS_i\right) ^{\comp}=\bigcap _iS_i^{\comp}\text{ and }\left( \bigcap _iS_i\right) ^{\comp}=\bigcup _iS_i^{\comp}.
\end{equation}
\end{exr}

The union and intersection of two sets are ways of constructing new sets, but one important thing to keep in mind is that, a priori, the two sets $A$ and $B$ are assumed to be contained within another set $X$.  But how do we get entirely new sets without already `living' inside another?  There are several ways to do this.
\begin{dfn}[Cartesian-product]\label{CartesianProduct}
Let $X$ and $Y$ be sets.  Then, the \emph{cartesian-product}\index{Cartesian-product} of $X$ and $Y$, $X\times Y$\index[notation]{$X\times Y$}, is
\begin{equation}
X\times Y\coloneqq \left\{ \coord{x,y}:x\in X,y\in Y\right\} .
\end{equation}
\begin{rmk}
If you really insist upon everything being defined in terms of sets we can take
\begin{equation}
\coord{x,y}\index[notation]{$\coord{x,y}$}\coloneqq \left\{ x,\{ x,y\} \right\} .
\end{equation}
The reason we use the notation $\coord{x,y}$ as opposed to the probably more common notation $(x,y)$ is to avoid confusion with the notation for open intervals.
\end{rmk}
\begin{rmk}
If $Y=X$, then it is common to write $X^2\coloneqq X\times X$, and similarly for products of more than two sets (e.g.~$X^3\coloneqq X\times X\times X$).  Elements in finite products are called \emph{tuples}\index{Tuple}.  For example, the elements of $X^2$ are $2$-tuples (or just \emph{ordered pairs}\index{Ordered pair}), the elements in $X^3$ are $3$-tuples, etc.
\end{rmk}
\end{dfn}
\begin{dfn}[Disjoint-union]\label{DisjointUnion}
Let $X$ and $Y$ be sets.  Then, the \emph{disjoint-union}\index{Disjoint-union} of $X$ and $Y$, $X\sqcup Y$\index[notation]{$X\sqcup Y$}, is
\begin{equation}
X\sqcup  Y\coloneqq \left\{ \coord{a,m}:m\in \{ 0,1\} ,\ a\in X\text{ if }m=0,\ a\in Y\text{ if }m=1\right\} .
\end{equation}
\begin{rmk}
Intuitively, this is supposed to be a copy of $X$ together with a copy of $Y$.  $a$ can come from either set, and the $0$ or $1$ tells us which set $a$ is supposed to come from.  Thus, we think of $X\subseteq X\sqcup Y$ as $X=\left\{ (a,0):a\in X\right\}$ and $Y\subseteq X\sqcup Y$ as $Y\left\{ (a,1):a\in Y\right\}$.
\end{rmk}
\end{dfn}
The key difference between the union and disjoint-union is that, in the case of the union of $A$ and $B$, an element that $x$ is both in $A$ and in $B$ is a \emph{single} element in $A\cup B$, whereas in the disjoint-union there will be two copies of it:  one in $A$ and one in $B$.  Hopefully the next example will help clarify this.
\begin{exm}[Union vs.~disjoint-union]
Define $A\coloneqq \{ a,b,c\}$ and $B\coloneqq \{ c,d,e,f\}$.  Then, $A\cup B=\{ a,b,c,d,e,f\}$.  On the other hand, $A\sqcup B=\{ a,b,c_A,c_B,d,e,f\}$, where $A\sqcup B\supseteq A=\{ a,b,c_A\}$ and $A\sqcup B\supseteq B=\{ c_B,d,e,f\}$.
\end{exm}
\begin{dfn}[Power set]
Let $X$ be a set.  Then, the \emph{power set}\index{Power set} of $X$, $2^X$\index[notation]{$2^X$}, is the set of all subsets of $X$,
\begin{equation}
2^X\coloneqq \left\{ A:A\subseteq X\right\} .
\end{equation}
\begin{rmk}
We will discuss the motivation for this notation in the next subsection (see \cref{exrA.1.26x}).
\end{rmk}
\end{dfn}

\subsection{Relations, functions, and orders}

We can then make the following definition.
\begin{dfn}[Relation]
A \emph{relation}\index{Relation} between two sets $X$ and $Y$ is a subset $R$ of $X\times Y$.  For a given relation $R$, we then write $x\sim _Ry$\index[notation]{$x\sim _Ry$}, or just $x\sim y$\index[notation]{$x\sim y$} if $R$ is clear from context iff $(x,y)\in R$.  Often we will simply refer to the relation by the symbol $\sim$ instead of $R$.
\end{dfn}
\begin{dfn}[Composition]\label{Composition}
Let $X$, $Y$, and $Z$ be sets, and let $R$ be a relation on $X$ and $Y$, and let $S$ be a relation on $Y$ and $Z$.  Then, the \emph{composition}\index{Composition}, $S\circ R$\index[notation]{$S\circ R$}, of $R$ and $S$ is the relation on $X$ and $Z$ defined by
\begin{equation}
S\circ R\coloneqq \left\{ \coord{x,z}\in X\times Z:\exists y\in Y\text{ such that }\coord{x,y}\in X\times Y\text{ and }\coord{y,z}\in Y\times Z\text{.}\right\} .
\end{equation}
\begin{rmk}
You will see in the next definition that a function is in fact just a very special type of relation, in which case, this composition is exactly the composition that you (hopefully) know and love.
\end{rmk}
\end{dfn}

There are several different important types of relations.  Perhaps the most important is the notion of a function.
\begin{dfn}[Function]
A \emph{function}\index{Function} from a set $X$ to a set $Y$ is a relation $f$ that has the property that for each $x\in X$ there is exactly one $y\in Y$ such that $x\sim _fy$.  For a given function $f$, we denote by $f(x)$ that unique element of $Y$ such that $x\sim _ff(x)$.  $X$ is the \emph{domain}\index{Domain (of a function)} of $f$ and $Y$ is the \emph{codomain}\index{Codomain (of a function)} of $f$.  The set of all functions from $X$ to $Y$ is denoted $Y^X$\index[notation]{$Y^X$}.
\begin{rmk}
The motivation for this notation is that, if $X$ and $Y$ are finite sets, then the cardinality (see \cref{chp1} \nameref{chp1}) of the set of all functions from $X$ to $Y$ is $\abs{Y}^{\abs{X}}$.
\end{rmk}
\end{dfn}
\begin{exm}[Identity function]
For every set $X$ (including the empty-set), there is a function, $\id _X:X\rightarrow X$, the \emph{identity function}\index{Identity function}, defined by
\begin{equation}
\id _X(x)\coloneqq x.
\end{equation}
\end{exm}
\begin{dfn}[Inverse function]
Let $f:X\rightarrow Y$ and $g:Y\rightarrow X$ be functions.  Then, $g$ is a \emph{left-inverse}\index{Left-inverse (of a function)} of $f$ iff $g\circ f=\id _X$; $g$ is a \emph{right-inverse}\index{Right-inverse (of a function)} of $f$ iff $f\circ g=\id _Y$; $g$ is a \emph{two-sided-inverse}\index{Two-sided-inverse (of a function)}, or just \emph{inverse}\index{Inverse (of a function)}, iff $g$ is both a left- and right-inverse of $f$.
\begin{exr}
Let $g$ and $h$ be two (two-sided)-inverses of $f$.  Show that $g=h$.
\end{exr}
Because of the uniqueness of two-sided-inverses, we may write $f^{-1}$\index[notation]{$f^{-1}$} for the unique two-sided-inverse of $f$.
\end{dfn}
\begin{exr}
Provide examples to show that left-inverses and right-inverses need not be unique.
\end{exr}
\begin{exr}\label{exrA.1.23}
Let $X$ be a nonempty set.
\begin{enumerate}
\item \label{enmA.1.23.i}Explain why there is \emph{no} function $f:X\rightarrow \emptyset$.
\item \label{enmA.1.23.ii}Explain why there is \emph{exactly one} function $f:\emptyset \rightarrow X$.
\item \label{enmA.1.23.iii}How many functions are there $f:\emptyset \rightarrow \emptyset$?
\end{enumerate}
\end{exr}
\begin{dfn}[Image]
Let $f:X\rightarrow Y$ be a function and let $S\subseteq X$.  Then, the \emph{image}\index{Image} of $S$ under $f$, $f(S)$, is
\begin{equation}
f(S)\coloneqq \left\{ f(x):x\in X\right\} .
\end{equation}
The \emph{range}\index{Range} of $f$, $f(X)$, is the image of $X$ under $f$.
\begin{rmk}
Note the difference between range and codomain.  For example, consider the function $f:\R \rightarrow \R$ defined by $f(x)\coloneqq x^2$.  Then, the codomain is $\R$ but the range is just $[0,\infty )$.  In fact the range and codomain are the same precisely when $f$ is surjective (see \cref{exrA.1.32}).
\end{rmk}
\end{dfn}
\begin{dfn}[Preimage]
Let $f:X\rightarrow Y$ be a function and let $T\subseteq Y$.  Then, the \emph{preimage} of $T$ under $f$, $f^{-1}(T)$, is
\begin{equation}
f^{-1}(T)\coloneqq \left\{ x\in X:f(x)\in T\right\} .
\end{equation}
\end{dfn}
\begin{exr}\label{exrA.1.30}
Let $f:X\rightarrow Y$ be a function, and let $\mathcal{S}$ and $\mathcal{T}$ be a collection of subsets of $X$ and $Y$ respectively.  Show that the following statements are true.
\begin{enumerate}
\item \label{enmA.1.30.i}$f^{-1}\left( \bigcup _{T\in \mathcal{T}}T\right) =\bigcup _{T\in \mathcal{T}}f^{-1}(T)$.
\item \label{enmA.1.30.ii}$f^{-1}\left( \bigcap _{T\in \mathcal{T}}T\right) =\bigcap _{T\in \mathcal{T}}f^{-1}(T)$.
\item \label{enmA.1.30.iii}$f\left( \bigcup _{S\in \mathcal{S}}S\right) =\bigcup _{S\in \mathcal{S}}f(S)$
\item \label{enmA.1.30.iv}$f\left( \bigcap _{S\in \mathcal{S}}S\right) \subseteq \bigcap _{S\in \mathcal{S}}f(S)$.
\end{enumerate}
Find an example to show that we need not have equality in \ref{enmA.1.30.iv}.
\end{exr}
\begin{exr}
Let $f:X\rightarrow Y$ be a function and let $T\subseteq Y$.  Show that $f^{-1}(T^{\comp})=f^{-1}(T)^{\comp}$.  For $S\subseteq X$, find examples to show that we need not have either $f(S^{\comp})\subseteq f(S)^{\comp}$ nor $f(S)^{\comp}\subseteq f(S^{\comp})$.
\end{exr}
\begin{dfn}[Injectivity, surjectivity, and bijectivity]
Let $f:X\rightarrow Y$ be a function.  Then,
\begin{enumerate}
\item (Injective) $f$ is \emph{injective}\index{Injective} iff for every $y\in Y$ there is at most one $x\in X$ such that $f(x)=y$.
\item (Surjective) $f$ is \emph{surjective}\index{Surjective} iff for every $y\in Y$ there is at least one $x\in X$ such that $f(x)=y$.
\item (Bijective) $f$ is \emph{bijective}\index{Bijective} iff for every $y\in Y$ there is exactly one $x\in X$ such that $f(x)=y$.
\end{enumerate}
\end{dfn}
\begin{rmk}
It follows immediately from the definitions that a function $f:X\rightarrow Y$ is bijective iff it is both injective and surjective.
\end{rmk}
\begin{exr}
Let $f:X\rightarrow Y$ be a function.  Show that $f$ is injective iff whenever $f(x_1)=f(x_2)$ it follows that $x_1=x_2$.
\end{exr}
\begin{exr}\label{exrA.1.32}
Let $f:X\rightarrow Y$ be a function.  Show that $f$ is surjective iff $f(X)=Y$.
\end{exr}
\begin{exm}[The domain and codomain matter]
Consider the `function' $f(x)\coloneqq x^2$.  Is this `function' injective or surjective?  Defining functions like this may have been kosher back when you were doing mathematics that wasn't actually mathematics, but no longer.  The question does not make sense because you have not specified the domain or codomain.  For example, $f:\R \rightarrow \R$ is neither injective nor surjective, $f:\R _0^+\rightarrow \R$ is injective but not surjective, $f:\R \rightarrow \R _0^+$ is surjective but not injective, and $f:\R _0^+\rightarrow \R _0^+$ is both injective and surjective.  Hopefully this example serves to illustrate:  functions are not (just) `rules'---if you have not specified the domain and codomain, then \emph{you have not specified the function}.
\end{exm}
\begin{exr}
Let $f:X\rightarrow Y$ be a function between nonempty sets.  Show that
\begin{enumerate}\label{exrA.1.9}
\item \label{enmA.1.9.i}$f$ is injective iff it has a left inverse,
\item \label{enmA.1.9.ii}$f$ is surjective iff it has a right inverse, and
\item \label{enmA.1.9.iii}$f$ is bijective iff it has a (two-sided) inverse.
\end{enumerate}
\begin{rmk}
By \cref{exrA.1.23}\ref{enmA.1.23.ii}, there \emph{is} exactly one function from $\emptyset$ to $\{ \emptyset \}$.  This function is definitely injective as every element in the codomain has \emph{at most one} preimage.  On the other hand, there is \emph{no} function from $\{ \emptyset \}$ to $\emptyset$ (by \cref{exrA.1.23}\ref{enmA.1.23.i}), and so certainly no left-inverse to the function from $\emptyset$ to $\{ \emptyset \}$.  This is why we require the sets to be nonempty.
\end{rmk}
\end{exr}
\begin{exr}\label{exrA.1.10}
Show that
\begin{enumerate}
\item the composition of two injections is an injection,
\item the composition of two surjections is a surjection, and
\item the composition of two bijections is a bijection.
\end{enumerate}
\end{exr}
\begin{exr}\label{exrA.1.47}
Let $f:X\rightarrow Y$ be a function, let $S\subseteq X$, and let $T\subseteq Y$.  Show that the following statements are true.
\begin{enumerate}
\item \label{enmA.1.47.i}$f\left( f^{-1}(T)\right) \subseteq T$, with equality if $f$ is surjective.
\item \label{enmA.1.47.ii}$f^{-1}\left( f(S)\right) \supseteq S$, with equality if $f$ is injective.
\end{enumerate}
Find examples to show that we need not have equality in general.
\begin{rmk}
Maybe this is a bit silly, but I remember which one is which as follows.  First of all, write these both using $\subseteq$, not $\supseteq$, that is, $S\subseteq f^{-1}(f(S))$ and $f(f^{-1}(S))\subseteq S$.  Then, the ``$-1$'' is always closest symbol that represents being `smaller' (that is ``$\subseteq$'').
\end{rmk}
\end{exr}
\begin{exr}\label{exrA.1.27}
Let $X$ and $Y$ be sets, and let $x_0\in X$ and $y_0\in Y$.  If there is some bijection from $X$ to $Y$, show that in fact there is a bijection from $X$ to $Y$ which sends $x_0$ to $y_0$.
\end{exr}
\begin{exr}\label{exrA.1.28}
Let $X$ and $Y$ be sets.  Show that there is a bijection from $X\times Y$ to $\sqcup _{y\in Y}X$.
\end{exr}
\begin{exr}\label{exrA.1.26x}
Let $X$ be a set.  Construction a bijection from $2^X$, the power set of $X$, and $\{ 0,1\}^X$, the set of functions from $X$ into $\{ 0,1\}$.
\begin{rmk}
This is the motivation for the notation $2^X$ to denote the power set.
\end{rmk}
\end{exr}

\subsubsection{Arbitrary disjoint-unions and products}

\begin{dfn}[Disjoint-union (of collection)]\label{DisjointUnionCollection}
Let $\mathcal{X}$ be an indexed collection\footnote{By \emph{indexed collection}\index{Indexed collection} we mean a set in which elements are allowed to be repeated.  So, for example, $\mathcal{X}$ is allowed to contain two copies of $\N$.  The reason for the term ``\emph{indexed} collection'' is that indices are often used to distinguish between the two identical copies, e.g.,~$\mathcal{Y}=\{ \N _1,\N _2\}$---as sets are not allowed to `repeat' elements, we add the indices so that, strictly speaking, $\N _1\neq \N _2$ as elements of $\mathcal{X}$, even though they represent the same set.  (If this is confusing, don't think about it too hard---it's just a set where elements are allowed to be repeated.)} of sets.  Then, the \emph{disjoint-union}\index{Disjoint-union} over all $X\in \mathcal{X}$, $\coprod _{X\in \mathcal{X}}X$\index[notation]{$\coprod _{X\in \mathcal{X}}X$}, is
\begin{equation}
\coprod _{X\in \mathcal{X}}X\coloneqq \{ \coord{x,X}:X\in \mathcal{X}\, x\in X\} .
\end{equation}
\begin{rmk}
The intuition and way to think of notation is just the same as it was in the simpler case of the disjoint-union of two sets (\cref{DisjointUnion}).
\end{rmk}
\end{dfn}
\begin{dfn}[Restrictions (of functions defined on a disjoint-union)]
Let $\mathcal{X}$ be an indexed collection of sets, let $Y$ be a set, and let $f:\coprod _{X\in \mathcal{X}}X\rightarrow Y$ be a function.  Then, the \emph{restriction of $f$ to $X$}\index{Restriction (disjoint-union)}, $\restr{f}{X}:X\rightarrow Y$\index[notation]{$\restr{f}{X}$}, is defined by
\begin{equation}
\restr{f}{X}(x)\coloneqq f(\coord{x,X}).
\end{equation}
In particular, the \emph{inclusion}\index{Inclusion (disjoint-union)} is defined to be
\begin{equation}
\iota _X\coloneqq \restr{[\id _{\coprod _{X\in \mathcal{X}}}]}{X},
\end{equation}
that is, the restriction of the identity $\id _{\coprod _{X\in \mathcal{X}}X}:\coprod _{X\in \mathcal{X}}X\rightarrow \coprod _{X\in \mathcal{X}}X$.
\end{dfn}

\begin{dfn}[Cartesian-product (of collection)]\label{CartesianProductCollection}
Let $\mathcal{X}$ be an indexed collection of sets.  Then, the \emph{cartesian-product}\index{Cartesian-product} over all $X\in \mathcal{X}$, $\prod _{X\in \mathcal{X}}X$\index[notation]{$\prod _{X\in \mathcal{X}}X$}, is
\begin{equation}
\prod _{X\in \mathcal{X}}X\coloneqq \left\{ f:\mathcal{X}\rightarrow \coprod _{X\in \mathcal{X}}X:f(X)\in X\right\} .
\end{equation}
\begin{rmk}
Admittedly this notation is a bit obtuse.  The cartesian-product is still supposed to be thought of a collection of ordered-`pairs', except now the pairs aren't just pairs, but can be $3$, $4$, or even infinitely many `coordinates'.  The coordinates are indexed by elements of $\mathcal{X}$, and the $X$-coordinate for $X\in \mathcal{X}$ must lie in $X$ itself.  Thus, for example, $X_1\times X_2=\prod _{X\in \mathcal{X}}X$ for $\mathcal{X}=\{ X_1,X_2\}$.  The key that is probably potentially the most confusing is that the elements of $\mathcal{X}$ are playing more than one role:  on one hand, they index the coordinates, and on the other hand, they are the set in which the coordinates take their values.  Hopefully keeping in mind the case $\mathcal{X}=\{ X_1,X_2\}$ helps this make sense.  So, for example, in the statement ``$f(X)\in X$'', on the left-hand side, $X$ is being thought of as an `index', and on the right-hand side it is being thought of as the `space' in which a coordinate `lives'.  This is thus literally just the statement that the $X$-coordinate of $f\in \prod _{X\in \mathcal{X}}X$ must be an element of the set $X$.
\end{rmk}
\begin{rmk}
For $x\in \prod _{X\in \mathcal{X}}X$, we write $x_X\coloneqq x(X)$ for the \emph{$X$-component}\index{Component (cartesian-product)} or \emph{$X$-coordinate}\index{Coordinate (cartesian-product)}.
\end{rmk}
\end{dfn}
\begin{dfn}[Components (of functions into a product)]
Let $\mathcal{X}$ be an indexed collection of sets, let $Y$ be a set, and let $f:Y\rightarrow \prod _{X\in \mathcal{X}}X$ be a function, then the \emph{$X$-component}\index{Component (of a function into a product)}, $f_X:Y\rightarrow X$\index[notation]{$f_X$}, is defined by
\begin{equation}
f_X(y)\coloneqq f(y)_X.
\end{equation}
In particular, the \emph{projection}\index{Projection (cartesian-product)}, $\pi _X$\index[notation]{$\pi _X$}, is defined to be
\begin{equation}
\pi _X\coloneqq [\id _{\prod _{X\in \mathcal{X}}X}]_X,
\end{equation}
that is, it is the $X$-component of the identity $\id _{\prod _{X\in \mathcal{X}}X}:\prod _{X\in \mathcal{X}}X\rightarrow \prod _{X\in \mathcal{X}}X$.
\begin{rmk}
For example, in the case $f:Y\rightarrow X_1\times X_2$, then $f(y)=\coord{f_1(x),f_2(x)}$.
\end{rmk}
\end{dfn}

\horizontalrule

Before introducing other important special cases of relations, we must first introduce several properties of relations.
\begin{dfn}
Let $\sim$ be a relation on a set $X$.
\begin{enumerate}
\item (Reflexive) $\sim$ is \emph{reflexive}\index{Reflexive} iff $x\sim _Rx$ for all $x\in X$.
\item (Symmetric) $\sim$ is \emph{symmetric}\index{Symmetric} iff $x_1\sim x_2$ is equivalent to $x_2\sim x_1$ for all $x_1,x_2\in X$.
\item (Transitive) $\sim$ is \emph{transitive}\index{Transitive} iff $x_1\sim x_2$ and $x_2\sim _Rx_3$ implies $x_1\sim _Rx_2$.
\item (Antisymmetric) $\sim$ is \emph{antisymmetric}\index{Antisymmetric} iff $x_1\sim x_2$ and $x_2\sim x_1$ implies $x_1=x_2$.\footnote{Admittedly the terminology here with ``symmetric'' and ``antisymmetric'' is a bit unfortunate.}
\item (Total) $\sim$ is \emph{total}\index{Total} iff for every $x_1,x_2\in X$ either $x_1\sim x_2$ or $x_2\sim x_1$.
\end{enumerate}
\end{dfn}

\subsubsection{Equivalence relations}

\begin{dfn}[Equivalence relation]
An \emph{equivalence relation}\index{Equivalence relation} on a set $X$ is a relation $\sim$ on $X$ that is reflexive, symmetric, and transitive.
\end{dfn}
\begin{exm}[Integers modulo $m$]\label{exmA.1.53}
Let $m\in \Z ^+$ and let $x,y\in \Z$.  Then, $x$ and $y$ are \emph{congruent modulo $m$}, and write $x\cong y\mmod{m}$, iff $x-y$ is divisible by $m$.
\begin{exr}
Check that $\cong \mmod{m}$ is an equivalence relation.
\end{exr}
For example, $3$ and $10$ are congruent modulo $7$, $1$ and $-3$ are congruent modulo $4$, $-2$ and $6$ are congruent modulo $8$, etc..
\begin{rmk}
We will see a `better' way of viewing the integers modulo $m$ in \cref{exmA.1.117}.  It is better in the sense that it is much more elegant and concise, but requires a bit of machinery and will probably not be as transparent if you have never seen it before.  Thus, it is probably more enlightening, at least the first time, to see things spelled out in explicit detail.
\end{rmk}
\end{exm}
\begin{dfn}[Equivalence class]
Let $\sim$ be an equivalence relation on a set $X$ and let $x_0\in X$.  Then, the \emph{equivalence class}\index{Equivalence class} of $x_0$, denoted by $[x_0]_\sim$\index[notation]{$[x_0]_\sim$}, or just $[x_0]$\index[notation]{$[x_0]$} if $\sim$ is clear from context, is
\begin{equation}\label{A.1.10}
[x_0]_\sim \coloneqq \left\{ x\in X:x\sim x_0\right\} =\left\{ x\in X:x_0\sim x\right\} .
\end{equation}
\begin{rmk}
Note that the second equation of \eqref{A.1.10} uses the symmetry of the relation.
\end{rmk}
\end{dfn}
\begin{exm}[Integers modulo $m$]\label{exmA.1.57}
This is a continuation of \cref{exmA.1.53}.  For example, the equivalence class of $5$ modulo $6$ is
\begin{equation}
[5]_{\cong \mmod{6}}=\left\{ \ldots ,-1,5,11,17,\ldots \right\} ,
\end{equation}
the equivalence class of $-1$ modulo $8$ is
\begin{equation}
[1]_{\cong \mmod{8}}=\left\{ \ldots ,-17,-9,-1,7,15,\ldots \right\} ,
\end{equation}
etc..
\end{exm}
An incredibly important property of equivalence classes is that they form a partition of the set.
\begin{dfn}[Partition]\label{dfnA.1.11}
Let $X$ be a set.  Then, a \emph{partition}\index{Partition} of $X$ is a collection $\mathcal{X}$ of subsets of $X$ such that
\begin{enumerate}
\item \label{A.1.11.1}$X=\bigcup _{U\in \mathcal{X}}U$, and
\item \label{A.1.11.2}for $U_1,U_2\in \mathcal{X}$ either $U_1=U_2$ or $U_1$ is disjoint from $U_2$.
\end{enumerate}
\end{dfn}
\begin{prp}\label{prpA.1.12}
Let $\sim$ be an equivalence relation on a set $X$ and let $x_1,x_2\in X$.  Then, either (i) $x_1\sim x_2$ or (ii) $[x_1]_\sim$ is disjoint from $[x_2]_\sim$.
\begin{proof}
If $x_1\sim x_2$ we are done, so suppose that this is not the case.  We wish to show that $[x_1]_\sim$ is disjoint from $[x_2]_\sim$, so suppose that this is not the case.  Then, there is some $x_3\in X$ with $x_1\sim x_3$ and $x_3\sim x_2$.  Then, by transitivity $x_1\sim x_2$:  a contradiction.  Thus, it must be the case that $[x_1]_\sim$ is disjoint from $[x_2]_\sim$.
\end{proof}
\end{prp}
\begin{crl}\label{crlA.1.13}
Let $X$ be a set and let $\sim$ be an equivalence relation on $X$.  Then, the collection $\mathcal{X}\coloneqq \left\{ [x]_\sim :x\in X\right\}$ is a partition of $X$.
\begin{proof}
The previous proposition, \cref{prpA.1.12}, tells us that $\mathcal{X}$ has property \cref{A.1.11.2} of the definition of a partition, \cref{dfnA.1.11}.  Property \cref{A.1.11.1} follows from the fact that $x\in [x]_\sim$, so that indeed
\begin{equation}
X=\bigcup _{x\in X}[x]_\sim =\bigcup _{U\in \mathcal{X}}U.
\end{equation}
\end{proof}
\end{crl}
Conversely, a partition of a set defines an equivalence relation.
\begin{exr}\label{exrA.1.41}
Let $X$ be a set, let $\mathcal{X}$ be a partition of $X$, and define $x_1\sim x_2$ iff there is some $U\in \mathcal{X}$ such that $x_1,x_2\in U$.  Show that $\sim$ is an equivalence relation.
\end{exr}
\begin{exm}[Integers modulo $m$]\label{exmA.1.63}
This in turn is a continuation of \cref{exmA.1.57}.  The equivalence classes modulo $4$ are
\begin{equation}
\begin{split}
[0]_{\cong \mmod{4}} & =\left\{ \ldots ,-8,-4,0,4,8,\ldots \right\} \\
[1]_{\cong \mmod{4}} & =\left\{ \ldots ,-7,-3,1,5,9,\ldots \right\} \\
[2]_{\cong \mmod{4}} & =\left\{ \ldots ,-6,-2,2,6,10,\ldots \right\} \\
[3]_{\cong \mmod{4}} & =\left\{ \ldots ,-5,-1,3,7,11,\ldots \right\} .
\end{split}
\end{equation}
You can verify directly that (i) each integer appears in at least one of these equivalence classes and (ii) that no integer appears in more than one.  Thus, indeed, the set $\left\{ [0]_{\cong \mmod{4}},[1]_{\cong \mmod{4}},[2]_{\cong \mmod{4}},[3]_{\cong \mmod{4}}\right\}$ is a partition of $\Z$.
\end{exm}

Given a set $X$ with an equivalence relation $\sim$, we obtain a new set $X/\sim$, the collection of all equivalence classes of elements in $X$ with respect to $\sim$.
\begin{dfn}[Quotient set]\label{dfnA.1.42}
Let $\sim$ be an equivalence relation on a set $X$.  Then, the \emph{quotient of $X$ with respect to $\sim$}\index{Quotient set}, $X/\sim$, is defined as
\begin{equation}
X/\sim \coloneqq \left\{ [x]_\sim :x\in X\right\} .
\end{equation}
The function $\q :X\rightarrow X/\sim$ defined by $\q (x)\coloneqq [x]_\sim$ is the \emph{quotient function}\index{Quotient function}.
\end{dfn}
Of course the quotient function is surjective.  What's perhaps a bit more surprising is that \emph{every} surjective function can be viewed as the quotient function with respect to some equivalence relation.
\begin{exr}\label{exrA.1.81}
Let $\q :X\rightarrow Y$ be surjective and for $x_1,x_2\in X$, define $x_1\sim x_2$ iff $x_1,x_2\in \q ^{-1}(y)$ for some $y\in Y$.  Show that (i) $\sim$ is an equivalence relation on $X$ and (ii) that $\q (x)=[x]_\sim$.
\end{exr}
\begin{exm}[Integers modulo $m$]\label{exmA.1.69}
This in turn is a continuation of \cref{exmA.1.63}.  For example, the quotient set mod $5$ is
\begin{equation}
\Z /\cong \mmod{5}=\left\{ [0]_{\cong \mmod{5}},[1]_{\cong \mmod{5}},[2]_{\cong \mmod{5}},[3]_{\cong \mmod{5}},[4]_{\cong \mmod{5}}\right\} .
\end{equation}
\end{exm}

It is quite common for us, after having defined the quotient set, to want to define operations on the quotient set itself.  For example, we would like to be able to add integers modulo $24$ (we do this when telling time).  In this example, we could make the following definition.
\begin{equation}
[x]_{\cong \mmod{24}}+[y]_{\cong \mmod{24}}\coloneqq [x+y]_{\cong \mmod{24}}.
\end{equation}
This is okay, but before we proceed, we have to check that this definition is \emph{well-defined}.  That is, there is a potential problem here, and we have to check that this potential problem doesn't actually happen.  I will try to explain what the potential problem is.

Suppose we want to add $3$ and $5$ modulo $7$.  On one hand, we could just do the obvious thing $3+5=8$.  But because we are working with \emph{equivalence classes}, I should just as well be able to add $10$ and $5$ and get the same answer.  In this case, I get $10+5=15$.  At first glance, it might seem we got different answers, but, alas, while $8$ and $15$ are not the same integer, they \emph{are} congruent modulo $7$.

In symbols, if I take two integers $x_1$ and $x_2$ and add them, and you take two integers $y_1$ and $y_2$ \emph{with $y_1$ equivalent to $x_1$ and $y_2$ equivalent to $x_2$}, it had better be the case that $x_1+x_2$ is equivalent to $y_1+y_2$.  That is, the answer should not depend on the ``representative'' of the equivalence class we chose to do the addition.
\begin{exm}[Integers modulo $m$]
This in turn is a continuation of \cref{exmA.1.69}.  Let $m\in \Z ^+$, let $x_1,x_2\in \Z$, and define
\begin{equation}
[x_1]_{\cong \mmod{m}}+[x_2]_{\cong \mmod{m}}\coloneqq [x_1+x_2]_{\cong \mmod{m}}.
\end{equation}
We check that this is well-defined.  Suppose that $y_1\cong x_1\mmod{m}$ and $y_2\cong x_2\mmod{m}$.  We must show that $x_1+x_2\cong y_1+y_2\mmod{m}$.  Because $y_k\cong x_k\mmod{m}$, we know that $y_k-x_k$ is divisible by $m$, and hence $(y_1-x_1)+(y_2-x_2)=(y_1+y_2)-(x_1+x_2)$ is divisible by $m$.  But this is just the statement that $x_1+x_2\cong y_1+y_2\mmod{m}$, exactly what we wanted to prove.
\begin{exr}
Define multiplication modulo $m$ and show that is is well-defined.
\end{exr}
\end{exm}

\subsubsection{Preorders}

\begin{dfn}[Preorder]\label{dfnA.1.19}
A \emph{preorder}\index{Preorder} on a set $X$ is a relation $\leq$ on $X$ that is reflexive and transitive.  A set equipped with a preorder is a \emph{preordered set}\index{Preordered set}.
\end{dfn}
\begin{rmk}
Note that an equivalence relation is just a very special type of preorder.
\end{rmk}
\begin{exr}
Find an example of
\begin{enumerate}
\item a relation that is both reflexive and transitive (i.e.~a preorder),
\item a relation that is reflexive but not transitive,
\item a relation that is not reflexive but transitive, and
\item a relation that is neither reflexive nor transitive.
\end{enumerate}
\end{exr}
The notion of an \emph{interval} is obviously important in mathematics and you almost have certainly encountered them before in calculus.  We give here the abstract definition (see \cref{prp3.3.70} to see that this agrees with what you are probably familiar with).
\begin{dfn}[Interval]\label{Interval}
Let $\coord{X,\leq}$ be a preordered set and let $I\subseteq X$.  Then, $I$ is an \emph{interval}\index{Interval} iff for all $x_1,x_2\in I$ with $x_1\leq x_2$, whenever $x_1\leq x\leq x$, it follows that $x\in I$.
\begin{rmk}
In other words, $I$ is an interval iff everything in-between two elements of $I$ is also in $I$.
\end{rmk}
\end{dfn}
\begin{dfn}[Monotone]\label{dfnA.1.21}
Let $X$ and $Y$ be preordered sets and let $f:X\rightarrow Y$ be a function.  Then, $f$ is \emph{nondecreasing}\index{Nondecreasing} iff $x_1\leq x_2$ implies that $f(x_1)\leq f(x_2)$.  If the second inequality is strict for distinct $x_1$ and $x_2$, i.e.~if $x_1<x_2$ implies $f(x_1)<f(x_2)$, then $f$ is \emph{increasing}\index{Increasing}.  If the inequality is in the other direction, i.e.~if $f(x_1)\geq f(x_2)$, then $f$ is \emph{nonincreasing}\index{Nonincreasing}.  If it is both strict and reversed, i.e.~if $x_1<x_2$ implies $f(x_1)>f(x_2)$, then $f$ is \emph{decreasing}\index{Decreasing}.  $f$ is \emph{monotone} iff it is either nondecreasing or nonincreasing and $f$ is \emph{strictly monotone} iff it is either increasing or decreasing.
\begin{rmk}
Note that the $\leq$ that appears in $x_1\leq x_2$ is \emph{different} than the $\leq$ that appears in $f(x_1)\leq f(x_2)$:  the former is the preorder on $X$ and the latter is the preorder on $Y$.  We will often abuse notation in this manner.
\end{rmk}
\end{dfn}

In this course, we will almost always be dealing with preordered sets whose preorder is in addition antisymmetric.
\begin{dfn}[Partial-order]\label{dfnA.1.24}
A \emph{partial-order}\index{Partial-order} is an antisymmetric preorder.  A set equipped with a partial-order is a \emph{partially-ordered set}\index{Partially-ordered set} or a \emph{poset}\index{Poset}.
\end{dfn}
\begin{exm}[Power set]
The archetypal example of a partially-ordered set is given by the power set.  Let $X$ be a set and for $U,V\in 2^X$, define $U\leq V$ iff $U\subseteq V$.
\begin{exr}\label{exrA.1.26}
Check that $(2^X,\leq )$ is in fact a partially-ordered set.
\end{exr}
\end{exm}
\begin{exr}
What is an example of a preorder that is not a partial-order?
\end{exr}

While we will certainly be dealing with nontotal partially-ordered sets, totality of an ordering is another property we will commonly come across.
\begin{dfn}[Total-order]
A \emph{total-order}\index{Total-order} is a total partial-order.  A set equipped with a total-order is a \emph{totally-ordered set}\index{Totally-ordered set}.
\end{dfn}
\begin{exr}
What is an example of a partially-ordered set that is not a totally-ordered set.
\end{exr}

And finally we come to the notion of well-ordering, which is an incredibly important property of the natural numbers.
\begin{dfn}[Well-order]
A \emph{well-order}\index{Well-order} on a set $X$ is a total-order that has the property that every nonempty subset of $X$ has a smallest element.  A set equipped with a well-order is a \emph{well-ordered set}\index{Well-ordered set}. 
\end{dfn}
In fact, we do not need to assume a priori that the order is a total-order.  This follows simply from the fact that every nonempty subset has a smallest element.
\begin{prp}\label{prpA.1.51}
Let $X$ be a partially-ordered set that has the property that every nonempty subset of $X$ has a smallest element.  Then, $X$ is totally-ordered (and hence well-ordered).
\begin{proof}
Let $x_1,x_2\in X$.  Then, the set $\{ x_1,x_2\}$ has a smallest element.  If this element is $x_1$, then $x_1\leq x_2$.  If this element is $x_2$, then $x_2\leq x_1$.  Thus, the order is total, and so $X$ is totally-ordered.
\end{proof}
\end{prp}
\begin{exr}
What is an example of a totally-ordered set that is not a well-ordered set?
\end{exr}

\subsubsection{Zorn's Lemma}

We end this subsection with an incredibly important result known as \emph{Zorn's Lemma}.  At the moment, it's importance might not seem obvious, and perhaps one must see it in action in order to appreciate its significance.  For the time being at least, let me say this:  if ever you are trying to produce something maximal by adding things to a set one-by-one (e.g.~if you are trying to construct a basis by picking linearly-independent vectors one-by-one), but you are running into trouble because, somehow, this process will never stop, not even if you `go on forever':  give Zorn's Lemma a try.
\begin{dfn}[Upper-bound and lower-bound]
Let $\coord{X,\leq}$ be a preordered set, let $S\subseteq X$, and let $x\in X$.  Then, $x$ is an \emph{upper-bound}\index{Upper bound} iff $s\leq x$ for all $s\in S$.  $x$ is a \emph{lower-bound}\index{Lower-bound} iff $x\leq s$ for all $s\in S$.
\end{dfn}
\begin{dfn}[Maximum and minimum]
Let $\coord{X,\leq}$ be a preordered set and let $x\in X$.  Then, $x$ is a \emph{maximum}\index{Maximum} of $X$ iff $x$ is an upper-bound of all of $X$.  $x$ is a \emph{minimum}\index{Minimum} of $X$ iff $x$ is a lower-bound of all of $X$.
\end{dfn}
\begin{dfn}[Maximal and minimal]
Let $\coord{X,\leq}$ be a preordered set, let $S\subseteq X$, and let $x\in S$.  Then, $x$ is \emph{maximal}\index{Maximal} in $S$ iff whenever $y\in S$ and $y\geq x$, it follows that $x=y$.  $x$ is \emph{minimal}\index{Minimal} in $S$ iff whenever $y\in S$ and $y\leq x$ it follows that $x=y$.
\begin{rmk}
In other words, maximal means that there is no element in $S$ strictly greater than $x$ (and similarly for minimal).  Contrast this with maxi\emph{mum} and mini\emph{mum}:  if $x$ is a maximum of $S$ it means that $y\leq x$ for all $y\in S$ (and analogously for minimum).
\end{rmk}
\end{dfn}
\begin{exr}
Let $X$ be a \emph{partially}-ordered set and let $S\subseteq X$.
\begin{enumerate}
\item Show that every maximum of $S$ is maximal in $S$.
\item Show that $S$ has at most one maximum element.
\item Come up with an example of $X$ and $S$ where $S$ has two distinct maximal elements.
\end{enumerate}
\end{exr}
\begin{dfn}[Downward-closed and upward-closed]
Let $X$ be a preordered set and let $S\subseteq X$.  Then, $S$ is \emph{downward-closed}\index{Downward closed} in $X$ iff whenever $x\leq s\in S$ it follows that $x\in S$.  $S$ is \emph{upward-closed}\index{Upward closed} in $X$ iff whenever $x\geq s\in S$ it follows that $x\in S$.
\end{dfn}
\begin{prp}\label{prpA.1.56}
Let $X$ be a well-ordered set and let $S\subset X$ be downward-closed in $X$.  Then, there is some $s_0\in X$ such that $S=\left\{ x\in X:x<s_0\right\}$.
\begin{proof}
As $S$ is a proper subset of $X$, $S^{\comp}$ is nonempty.  As $X$ is well-ordered, it follows that $S^{\comp}$ has a smallest element $s_0$.  We claim that $S=\left\{ x\in :x<s_0\right\}$.  First of all, let $x\in X$ and suppose that $x<s_0$.  If it were \emph{not} the case that $x\in S$, then $s_0$ would no longer be the smallest element in $S^{\comp}$.  Hence, we must have that $x\in S$.  Conversely, let $x\in S$.  By totality, either $x\leq s_0$ or $s_0\leq x$.  As $x\in S$ and $s_0\in S^{\comp}$, we cannot have that $x=s_0$, so in fact, in the former case, we would have $x<s_0$, and we are done, so it suffices to show that $s_0\leq x$ cannot happen.  If $s_0\leq x$, then because $S$ is downward-closed in $X$ and $x\in S$, it would follows that $s_0\in S$:  a contradiction.  Therefore, it cannot be the case that $s_0\leq x$.
\end{proof}
\end{prp}
\begin{thm}[Zorn's Lemma]\label{ZornsLemma}\index{Zorn's Lemma}
Let $X$ be a partially-ordered set.  Then, if every well-ordered subset has an upper-bound, then $X$ has a maximal element.
\begin{proof}\footnote{Proof adapted from \cite{Grayson}.}
\Step{Make hypotheses}
Suppose that every well-ordered subset has an upper bound.  We proceed by contradiction:  suppose that $X$ has no maximal element.

\Step{Show that every well-ordered subset has an upper-bound \emph{not} contained in it}
Let $S\subseteq X$ be a well-ordered subset, and let $u$ be some upper-bound of $S$.  If there were no element in $X$ strictly greater than $u$, then $u$ would be a maximal element of $X$.  Thus, there is some $u'>u$.  It cannot be the case that $u'\in S$ because then we would have $u'\leq u$ because $u$ is an upper-bound of $S$.  But then the fact that $u'\leq u$ and $u\leq u'$ would imply that $u=u'$:  a contradiction.  Thus, $u'\notin S$, and so constitutes an upper-bound not contained in $S$.

\Step{Define $u(S)$}
For each well-ordered subset $S\subseteq X$, denote by $u(S)$ some upper-bound of $S$ not contained in $S$.

\Step{Define the notion of a $u$-set}
We will say that a well-ordered subset $S\subseteq X$ is a $u$-set iff $x_0=u\left( \left\{ x\in S:x<x_0\right\} \right)$ for all $x_0\in S$.

\Step{Show that for $u$-sets $S$ and $T$, either $S$ is downward-closed in $S$ or $T$ is downward closed in $S$}
Define
\begin{equation}
D\coloneqq \bigcup _{\substack{A\subseteq X \\ A\text{ is downward-closed in} S \\ B\text{ is downward-closed in }T}}A.
\end{equation}
That is, $D$ is the union of all sets that are downward-closed in both $S$ and $T$.

We first check that $D$ itself is downward-closed in both $S$ and $T$.  Let $d\in D$, let $s\in S$, and suppose that $s\leq d$.  As $d\in D$, it follows that $d\in A$ for some $A\subseteq X$ downward-closed in both $S$ and $T$.  As $A$ is in particular downward-closed in $S$, it follows that $s\in D$, and so $D$ is downward-closed in $S$.  Similarly it is downward-closed in $T$.

If $D=S$, then $S=D$ is downward-closed in $T$, and we are done.  Likewise if $D=T$, so we may as well assume that $D$ is a proper subset of both $S$ and $T$.  Then, by \cref{prpA.1.56}, there are $s_0\in S$ and $t_0\in T$ such that $\{ s\in S:s<s_0\} =D=\{ t\in T:t<t_0\}$.  Because $S$ and $T$ are $u$-sets, it follows that
\begin{equation}
s_0=u\left( \{ s\in S:s<s_0\} \right) =u\left( \{ t\in T:t<t_0\} \right) =t_0.
\end{equation}
Define $D\cup \{ s_0\} \eqqcolon D'\coloneqq D\cup \{ t_0\}$.
Let $d\in D'$, let $s\in S$, and suppose that $s\leq d$.  Either $d=s_0$ or $d\in D$.  In the latter case, $d<s_0$.  Either way, $d\leq s_0$, and so we have that $s\leq s_0$, and so either $s=s_0$ or $s<s_0$; either way, $s\in D'$.  The conclusion is that $D'$ is downward-closed in $S$.  It is similarly downward-closed in $T$.  By the definition of $D$, we must have that $D'\subseteq D$:  a contradiction.  Thus, it could not have been the case that $D$ was a proper subset of both $S$ and $T$.

\Step{Define $U$}
Define
\begin{equation}
U\coloneqq \bigcup _{\substack{S\subseteq X \\ S\text{ is a }u\text{-set}}}S.
\end{equation}
We claim that $U$ is a $u$-set.  It will then also be the case that $U'\coloneqq U\cup \{ u(U)\}$ is a $u$-set, and so, by the definition of $U$, we will have $U'\subseteq U$:  a contradiction, which will complete the proof.  Thus, it suffices to show that $U$ is a $u$-set.

\Step{Finish the proof by showing that $U$ is a $u$-set}
We first need to check that $U$ is well-ordered.  Let $A\subseteq U$ be nonempty.  For $S\subseteq X$ a $u$-set, define $A_S\coloneqq A\cap S$.  For each $A_S$ that is nonempty, denote by $a_S$ the smallest element in $A_S$.  Let $T\subseteq X$ be some other $u$-set.  Then, by the previous step, without loss of generality, $S$ is downward-closed in $T$.  In particular, $S\subseteq T$ so that $a_S\in T$.  Hence, $a_T\leq a_S$.  Then, because $S$ is downward-closed, $a_T\in S$, and hence $a_T\leq a_S$, and hence $a_T=a_S$.  This unique element is a smallest element of $A$.

Let $u_0\in U$.  All that remains to be shown is that $u_0=u\left( \{ x\in U:x<u_0\} \right)$.  To do this, we first show  that every $u$-set is downward-closed in $U$.

Let $S\subseteq X$ be a $u$-set, let $s\in S$, let $x\in U$, and suppose that $x\leq s$.  As $x\in U$, there is some $u$-set $T$ such that $x\in T$.  Then, by the previous step, either $S$ is downward-closed in $T$ or $T$ is downward-closed in $S$.  If the former case, then we have that $x\in S$ because $x\leq s$.  On the other hand, in the latter case, we have that $x\in S$ because $x\in T\subseteq S$.

Now we finally return to showing that $u_0=u\left( \{ x\in U:x<u_0\} \right)$.  By definition of $U$, $u_0\in S$ for some $u$-set $S$, and therefore, $u_0=u\left( \{ x\in S:x<u_0\} \right)$.  Therefore, it suffices to show that if $x\in U$ is less than $u_0$, then it is in $S$ (because then $\{ x\in S:x<u_0\} =\{ x\in U:x<u_0\}$.  This, however, follows from the fact that $S$ is downward-closed in $U$.
\end{proof}
\end{thm}

\subsection{Sets with algebraic structure}

\begin{dfn}[Binary operation]
A \emph{binary operation}\index{Binary operation} $\cdot$ on a set $X$ is a function $\cdot :X\times X\rightarrow X$.  It is customary to write $x_1\cdot x_2\coloneqq \cdot (x_1,x_2)$ for binary operations.
\begin{rmk}
Sometimes people say that \emph{closure} is an axiom.  This is not necessary.  That a binary operation on $X$ takes values \emph{in} $X$ implicitly says that the operation is closed.  That doesn't mean that you never have to check closure, however.  For example, in order to verify that the even integers $2\Z$ are a subrng (see \cref{dfnA.1.86}).
\end{rmk}
\end{dfn}
\begin{dfn}
Let $\cdot$ be a binary relation on a set $X$.
\begin{enumerate}
\item (Associative) $\cdot$ is \emph{associative}\index{Associative} iff $(x_1\cdot x_2)\cdot x_3=x_1\cdot (x_2\cdot x_3)$ for all $x_1,x_2,x_3\in X$.
\item (Commutative) $\cdot$ is \emph{commutative}\index{Commutative} if $x_1\cdot x_2=x_2\cdot x_1$ for all $x_1,x_2\in X$.
\item (Identity) An \emph{identity} of $\cdot$ is an element $1\in X$ such that $1\cdot x=x=x\cdot 1$ for all $x\in X$.
\item (Inverse) If $\cdot$ has an identity and $x\in X$, then an \emph{inverse} of $x$ is an element $x^{-1}\in X$ such that $x\cdot x^{-1}=1=x^{-1}\cdot x$.
\end{enumerate}
\end{dfn}

We first consider sets equipped just a single binary operation.
\begin{dfn}[Magma]
A \emph{magma}\index{Magma} is a set equipped with a binary operation.
\end{dfn}
\begin{exr}\label{exrA.1.34}
Let $\coord{X,\cdot }$ be a magma and let $x_1,x_2,x_3\in X$.  Show that $x_1=x_2$ implies $x_1\cdot x_3=x_2\cdot x_3$.
\begin{rmk}
My hint is that the solution is so trivial that it is easy to overlook.
\end{rmk}
\begin{rmk}
This is what justifies the `trick' (if you can call it that) of doing the same thing to both sides of an equation that is so common in algebra.
\end{rmk}
\begin{rmk}
Note that the converse is not true in general.  That is, we can have $x_1\cdot x_2=x_1\cdot x_3$ with $x_2\neq x_3$.
\end{rmk}
\end{exr}
\begin{dfn}[Semigroup]\label{Semigroup}
A \emph{semigroup}\index{Semigroup} is a magma $\coord{X,\cdot}$ such that $\cdot$ is associative.
\end{dfn}
\begin{dfn}[Monoid]\label{Monoid}
A \emph{monoid}\index{Monoid} $\coord{X,\cdot ,1}$ is a semigroup $\coord{X,\cdot}$ equipped with an identity $1\in X$.
\end{dfn}
\begin{exr}[Identities are unique]\label{exrA.1.77}
Let $X$ be a monoid and let $1,1'\in X$ be such that $1\cdot x=x=x\cdot 1$ and $1'\cdot x=x=x\cdot 1'$ for all $x\in X$.  Show that $1=1'$.
\end{exr}
\begin{dfn}[Group]\label{Group}
A \emph{group}\index{Group} is a monoid $\coord{X,\cdot ,1}$ equipped with a function $-^{-1}:X\rightarrow X$ so that $x^{-1}$ is an inverse of $x$ for all $x\in X$.
\begin{rmk}
Usually this is just stated as ``$X$ has inverses.''.  This isn't wrong, but this way of thinking about things doesn't generalize to universal algebra or category theory quite as well.  The way to think about this is that, inverses, like the binary operation (as well as the identity) is \emph{additional structure}.  This is in contrast to the axiom of associativity which should be thought of as a \emph{property} satisfied by an \emph{already existing} structure (the binary operation).
\end{rmk}
\end{dfn}
\begin{exr}[Inverses are unique]\label{exrA.1.79}
Let $X$ be a group, let $x\in X$, and let $y,z\in X$ both be inverses of $x$.  Show that $y=z$.
\end{exr}
\begin{exr}
Let $\coord{X,\cdot}$ be a group and let $x_1,x_2,x_3\in X$.  Show that if $x_1\cdot x_2=x_1\cdot x_3$, then $x_2=x_3$.
\begin{rmk}
Thus, the converse to \cref{exrA.1.34} holds in the case of a group.
\end{rmk}
\end{exr}
\begin{dfn}[Homomorphism (of magmas)]
Let $X$ and $Y$ be magmas and let $f:X\rightarrow Y$ be a function.  Then, $f$ is a \emph{homomorphism}\index{Homomorphism (of magmas)} iff $f(x_1\cdot x_2)=f(x_1)\cdot f(x_2)$ for all $x_1,x_2\in X$, and furthermore, in the case that both $X$ and $Y$ have identities, $f(1)=1$.
\begin{rmk}
Note that, once again, the $\cdot$ in $f(x_1\cdot x_2)$ is \emph{not} the same as the $\cdot$ in $f(x_1)\cdot f(x_2)$.  Confer the remark following the definition of a nondecreasing function, \cref{dfnA.1.21}.
\end{rmk}
\end{dfn}

We now move on to the study of sets equipped with \emph{two} binary operations.
\begin{dfn}[Rg]
A \emph{rg}\index{Rg} is a set equipped with two binary operations $\coord{X,+,\cdot}$ such that
\begin{enumerate}
\item $\coord{X,+}$ is a commutative monoid,
\item $\coord{X,\cdot}$ is a semigroup, and
\item $\cdot$ distributes over $+$, that is, $x_1\cdot (x_2+x_3)=x_1\cdot x_2+x_1\cdot x_3$ for all $x_1,x_2,x_3\in X$.
\end{enumerate}
\begin{rmk}
In other words, writing out what it means for $\coord{X,+}$ to be a commutative monoid and for $\coord{X,\cdot}$ to be a a semigroup, these three properties are equivalent to
\begin{enumerate}
\item $+$ is associative,
\item $+$ is commutative,
\item $+$ has an identity,
\item $\cdot$ is associative,
\item $\cdot$ distributes over $+$.
\end{enumerate}
\end{rmk}
\begin{rmk}
For $x\in X$ and $m\in \Z ^+$, we write $m\cdot x\coloneqq \underbrace{x+\cdots +x}_{m}$.  Note that we do \emph{not} make this definition for $m=0\in \N$.  An empty-sum is \emph{always} $0$ (by definition), but $0\cdot x$ need not be $0$ in a general rg (see the tropical integers in \cref{exm1.3.2}).
\end{rmk}
\begin{rmk}
I have actually never seen the term rg used before.  That being said, I haven't seen \emph{any} term to describe such an algebraic object before.  Nevertheless, I have seen both the terms rig and rng before (see below), and, well, given those terms, ``rg'' is pretty much the only reasonable term to give to such an algebraic object.  We don't have a need to work with rgs directly, but we will work with both rigs and rngs, and so it is nice to have an object of which both rigs and rngs are special cases.
\end{rmk}
\end{dfn}
\begin{dfn}[Rng]\label{dfnA.1.86}
A \emph{rng}\index{Rng} is a rg such that $\coord{X,+,0,-}$ is a commutative group, that is, a rg that has additive inverses.
\end{dfn}
\begin{exr}\label{exrA.1.43}
Let $\coord{X,+,0-,\cdot}$ be a rng and let $x_1,x_2\in X$.  Show the following properties.
\begin{enumerate}
\item $0\cdot x_1=0$.
\item $(-x_1)\cdot x_2=-(x_1\cdot x_2)$.
\end{enumerate}
\end{exr}
\begin{exm}[A rg that is not a rng]
The even natural numbers $2\N$ with their usual addition and multiplication is also an example of a rg that is not a rng.
\end{exm}
\begin{dfn}[Rig]\label{dfnA.1.33}
A \emph{rig}\index{Rig} is a rg such that $\coord{X,\cdot ,1}$ is a monoid, that is, a rg that has a multiplicative identity
\begin{rmk}
I believe it is more common to refer to rigs as \emph{semirings}.  I dislike this terminology because it suggests an analogy with semigroups, of which there is none.  The term rig is also arguably more descriptive---even if you didn't know what the term meant, you might have a good chance of guessing, especially if you had seen the term rng before.
\end{rmk}
\end{dfn}
\begin{dfn}[Characteristic]
Let $\coord{X,+,0,-,\cdot ,1}$ be a rig.  Then, either (i) there is some $m\in \Z ^+$ such that $m\cdot 1=0\in X$ or (ii) there is no such $m$.  In the former case, the smallest positive integer such that $m\cdot 1=0\in X$ is the \emph{characteristic}\index{Characteristic (of a rig)}, and in the latter case the \emph{characteristic} is $0$.  We denote the characteristic by $\Char (X)$.
\end{dfn}
\begin{exm}[A rg that is not a rig]
The even natural numbers $2\N$ with their usual addition and multiplication is a rg that is not a rig.
\end{exm}
\begin{dfn}[Ring]
A \emph{ring}\index{Ring} is a rg that is both a rig and a rng.
\end{dfn}
\begin{rmk}
The motivation for the terminology is as follows.  Historically, the term ``ring'' was the first to be used.  It is not uncommon for authors to use the term ring to mean both our definition and our definition minus the requirement of having a multiplicative identity.  To remove this ambiguity in terminology, we take the term ``ring'' to imply the existence of the identity and the removal of the ``i'' from the word is the term used for objects which do not necessarily have an identity.  Similarly, thinking of the ``n'' in ``ring'' as standing for ``negatives'', a rig is just a ring that does not necessarily posses additive inverses.
\end{rmk}
\begin{rmk}
Whenever we say that a rg is commutative, we mean that the \emph{multiplication} is commutative (this should be obvious---addition is always commutative).  Instead of saying referring to things as ``commutative rgs'' etc.~we will often shorten this to ``crg'' etc.\index{Crg}\index{Crig}\index{Cring}\index{Crng}.
\end{rmk}
\begin{rmk}
Note that it follows from \cref{exrA.1.43} that $-1\cdot x=-x$ for all $x\in X$, $X$ a ring.
\end{rmk}
\begin{exr}
Let $X$ be a ring and suppose that $0=1$.  Show that $X=\{ 0\}$.
\begin{rmk}
This is called the \emph{zero cring}\index{Zero cring}.
\end{rmk}
\end{exr}
\begin{dfn}[Integral]\label{dfnA.1.69}
A rg $\coord{X,+,\cdot ,0}$ is \emph{integral}\index{Integral} iff it has the property that, whenever $x\cdot y=0$, it follows that either $x=0$ or $y=0$.
\begin{rmk}
Usually the adjective ``integral'' is applied only to crings, in which case people refer to this as an \emph{integral domain}\index{Integral domain} instead of an integral cring.  As the natural numbers have this property (i.e.~$xy=0\Rightarrow x=0\vee y=0$) I wanted an adjective that would describe rgs with this property and ``integral'' was an obvious choice because of common use of the term ``integral domain''.  It is then just more systematic to refer to them as integral crings instead of integral domains.  This is usually not an issue because it is not very common to work with rigs or rgs (we of course need to because we construct the natural numbers from the ground up).
\end{rmk}
\end{dfn}
\begin{dfn}[Skew-field]
A \emph{Skew-field}\index{Skew-field} is a ring $\coord{X,+,\cdot ,0,-,1}$ such that $\coord{X\setminus \{ 0\} ,\cdot ,1,^{-1}}$ is a group.  That is, it is a ring such that every nonzero element has a multiplicative inverse.
\begin{rmk}
In other words, every nonzero element has a multiplicative inverse.
\end{rmk}
\begin{rmk}
Sometimes people use the term \emph{division ring}\index{Division ring} instead of skew-field.
\end{rmk}
\begin{exr}
Show that all skew-fields are integral.
\end{exr}
\end{dfn}
\begin{dfn}[Field]
A \emph{field}\index{Field} is a commutative skew-field.
\end{dfn}
\begin{exr}
Let $F$ be a field with positive characteristic $p$.  Show that $p$ is prime.
\end{exr}
\begin{dfn}[Homomorphisms (of rgs)]
Let $\coord{X,+,\cdot}$ and $\coord{Y,+,\cdot}$ be rgs and let $f:X\rightarrow Y$ be a function.  Then,$f$ is a \emph{homomorphism}\index{Homomorphism (of rgs)} iff $f$ is both a homomorphism (of magmas) from $\coord{X,+}$ to $\coord{Y,+}$ and from $\coord{X,\cdot}$ to $\coord{Y,\cdot}$.  Explicitly, this means that
\begin{equation}
f(x+y)=f(x)+f(y),f(0)=0,\text{ and }f(xy)=f(x)f(y),
\end{equation}
and furthermore, in the case that both $X$ and $Y$ are rigs, that
\begin{equation}
f(1)=1.
\end{equation}
\end{dfn}

\subsubsection{Quotient groups rngs}

It is probably worth noting that this subsubsection is of relatively low priority.  We present this information here essentially because it gives a more unified, systematic, sophisticated, and elegant way to view things presented in other places in the notes, but it is also not really strictly required to understand these examples.

If you have never seen quotient rngs before, it may help to keep in the back of your mind a concrete example as you work through the definitions.  We recommend you keep in mind the example $R\coloneqq \Z$ and $I\coloneqq m\Z$ (all multiplies of $m$) for some $m\in \Z ^+$.  In this case, the quotient $R/I$ is (supposed to be, and in fact will turn-out to be) the integers modulo $m$.  While this is a quotient rng, it is also of course a quotient group (just forget about the multiplication), so this example may also help you think about quotient groups as well.

As we shall use quotient groups to define quotient rngs, we do them first.  The first thing to notice is that every subgroup of a group induces an equivalence relation.
\begin{dfn}[Cosets (in groups)]\label{Cosets}
Let $G$ be a group, let $H$ be a subgroup, and let $g_1,g_2\in G$.  Then, we define $g_1\cong g_2\mmod{H}$\index[notation]{$g_1\cong g_2\mmod{H}$} iff $g_2^{-1}g_1\in H$.
\begin{exr}
Show that $\cong \mmod{H}$ is an equivalent relation.
\end{exr}
The equivalence class of $g$ with respect to this equivalence relation is the \emph{left $H$-coset}\index{Coset} of $g$ and written $gH$\index[notation]{$gH$}.  The quotient set (\cref{dfnA.1.42}) of $G$ with respect to this equivalence relation id denoted $G/H$.
\begin{rmk}
By changing the definition of the equivalent relation to ``\textellipsis iff $g_1g_2^{-1}\in H$'', then we obtain the corresponding definition of \emph{right $H$-cosets}, denoted by $Hg$\index[notation]{$Hg$}.  Of course, in general if the binary operation is not commutative, then $gH\neq Hg$.
\end{rmk}
\end{dfn}
For a subgroup $H$ of $G$, $G/H$ will always be a set.  However, in good cases, it will be more than just a set---it will be a group in its own right.
\begin{dfn}[Ideals and quotient groups]\label{IdealsAndQuotientGroups}
Let $G$ be a group, let $H\subseteq G$ be a subgroup, and let $g_1,g_2\in G$.  Define
\begin{equation}
(g_1H)\cdot (g_2)\coloneqq (g_1g_2)H.
\end{equation}
$H$ is an \emph{ideal}\index{Ideal} iff this is well-defined on the quotient set $G/H$.  In this case, $G/H$ is itself a group, the \emph{quotient group}\index{Quotient group} of $G$ modulo $H$.
\begin{rmk}
Recall that $gH$ represents the equivalent class of $g$ modulo $H$, and so, in particular, these definitions involve picking representatives of equivalence classes.  Thus, in order for these operations to make sense, they must be well-defined.  In general, they will not be well-defined, and we call $H$ an \emph{ideal} precisely in the ``good'' case where these operations to make sense.
\end{rmk}
\begin{rmk}
In the context of groups, it is \emph{much} more common to refer to ideals as \emph{normal subgroups}\index{Normal subgroups}.  As always, we choose the terminology we do because it is more universally consistent, even if less common.
\end{rmk}
\end{dfn}
There is an easy way to check that a subgroup is an ideal or not that does not require directly checking well-definedness.
\begin{exr}
Let $G$ be a group and let $H\subseteq G$ be a subset.  Show that $H$ is an ideal iff (i) it is a subgroup and (ii) $gHg^{-1}\subseteq H$ for all $g\in G$.
\end{exr}

And now we turn to quotient rngs, whose development is completely analogous.
\begin{dfn}[Cosets (in rngs)]
Let $R$ be a rng, let $S\subseteq R$ be a subrng, and let $r_1,r_2\in R$.  Then, we define $r_1\cong r_2\mmod{S}$ iff $r_1-r_2\in S$.
\begin{exr}
Show that $\cong \mmod{S}$ is an equivalence relation.
\end{exr}
The equivalence class of $r$ with respect to this equivalence relation is the $S$-\emph{coset} of $r$ and written $r+S$.  The quotient set of $R$ with respect to this equivalence relation is denoted $R/S$.
\end{dfn}
You can check that $m\Z$ is indeed a subrng of $\Z$ and that $\Z /m\Z$ consists of just $m$ cosets:  $0+m\Z ,1+m\Z ,\ldots ,(m-1)+m\Z$, though you are probably more familiar just writing this as $0\mmod{m},1\mmod{m},\ldots ,m-1\mmod{m}$.  Of course, however, $\Z /m\Z$ is more than just a set, it has a ring structure of its own, and in good cases, $R/S$ will obtain a canonical ring structure of its own as well.
\begin{dfn}[Ideals and quotient rngs]\label{IdealsAndQuotientRngs}
Let $R$ be a rng, let $S\subseteq R$ be a subrng, and let $r_1,r_2\in R$.  Define
\begin{equation}
(r_1+S)+(r_2+S)\coloneqq (r_1+r_2)+S\text{ and }(r_1+S)\cdot (r_2+S)\coloneqq (r_1\cdot r_2)+S.
\end{equation}
$S$ is an \emph{ideal}\index{Ideal} iff both of these operations are well-defined.  In this case, $R/S$ is the \emph{quotient rng}\index{Quotient rng} of $R$ modulo $S$.
\end{dfn}
Just as before, we have an easy way of checking of subsets are ideals.
\begin{exr}
Let $R$ be a rng and let $S\subseteq R$ be a subset.  Show that $S$ is an ideal iff (i) it is a subrng and (ii) $r\in R$ and $s\in S$ implies that $r\cdot s,s\cdot r\in S$.
\begin{rmk}
The second property is sometimes called ``absorbing'', because elements in the ideal `absorb' things into the ideal when you multiply them.
\end{rmk}
\end{exr}
\begin{exm}[Integers modulo $m$]\label{exmA.1.117}
Let $m\in \Z ^+$.
\begin{exr}
Show that $m\Z$ is an ideal in $\Z$.
\end{exr}
Then, the \emph{integers modulo $m$}\index{Integers modulo $m$} are defined to be the quotient cring $\Z /m\Z$.
\end{exm}

\section{Basic category theory}

First of all, a disclaimer:  it is probably not best pedagogically speaking to start with even the very basics of category theory.  While in principle anyone who has the prerequisites for these notes knows everything they need to know to understand category theory, it may be difficult to understand the motivation for things without a collection of examples to work with in the back of your mind.  Thus, if anything in this section does not make sense the first time you read through it, you should not worry---t will only be a problem if you do not understand ideas here as they occur later on in the text.  In fact, it is probably perfectly okay to completely skip this section and reference back to it as needed.  In any case, our main motivation for introducing category theory in a subject like this is simply that we would like to have more systematic language and notation.

\subsection{What is a category?}

In mathematics, we study many different types of objects:  sets, preordered sets, monoids, rngs, topological spaces, schemes, etc..\footnote{No, you are not expected to know what all of these are.}  In all of these cases, however, we are not only concerned with the objects themselves, but also with maps between them that `preserve' the relevant structure.  In the case of a set, there is no extra structure to preserve, and so the relevant maps are \emph{all} the functions.  In contrast, however, for topological spaces, we will see that the relevant maps are not all the functions, but instead all \emph{continuous} functions.\footnote{You might say that the entire point of the notion of a topological space is it is one of the most general contexts in which the notion of continuity makes sense.  We will see exactly how this works later in the body of the text.}  Similarly, the relevant maps between monoids are not all the functions but rather the \emph{homomorphisms}.\footnote{Once again, it is perfectly okay if you don't know what either a monoid or a homomorphism is.}  The idea then is to come up with a definition that deals with both the objects and the relevant maps, or morphisms, simultaneously.  This is the motivating idea of the definition of a category.
\begin{dfn}[Category]
A \emph{category}\index{Category} $\mathcal{C}$ is
\begin{enumerate}
\item a collection $\mathcal{C}_0$\index[notation]{$\mathcal{C}_0$} called the \emph{objects}\index{Objects} of $\mathcal{C}$;
\item together with, for each $A,B\in \mathcal{C}_0$, a collection $\Mor _{\mathcal{C}}(A,B)$\index[notation]{$\Mor _{\mathcal{C}}(A,B)$} called the \emph{morphisms}\index{Morphisms} from $A$ to $B$ in $\mathcal{C}$;\footnote{No, we do not require that $\Mor _{\mathcal{C}}(A,B)$ be a set.}
\item for each $A,B,C\in \mathcal{C}_0$, a function $\circ :\Mor _{\mathcal{C}}(B,C)\times \Mor _{\mathcal{C}}(A,B)\rightarrow \Mor _{\mathcal{C}}(A,C)$ called \emph{composition}\index{Composition};
\item and for each $A\in \mathcal{C}_0$ a distinguished element $\id _{A}\in \Mor _{\mathcal{C}}(A,A)$ called the \emph{identity}\index{Identity (in a category)} of $A$;
\end{enumerate}
such that
\begin{enumerate}
\item $\circ$ is `associative', that is, $f\circ (g\circ h)=(f\circ g)\circ h$ for all morphisms $f,g,h$ for which these composition make sense,\footnote{In case you're wondering, the quotes around ``associative'' are used because usually the word ``associative'' refers to a property that a binary operation has.  A binary operation on a set $S$ is, by definition, a function from $X\times X$ into $X$.  Composition however in general is a function from $X\times Y$ into $Z$ for $X\coloneqq \Mor _{\mathcal{C}}(B,C)$, $Y\coloneqq \Mor _{\mathcal{C}}(A,B)$ and $Z\coloneqq \Mor _{\mathcal{C}}(A,C)$, and hence not a binary operation.} and
\item $f\circ \id _A=f=\id _A\circ f$ for all $A\in \mathcal{C}_0$.
\end{enumerate}
\end{dfn}
The intuition here is that the objects $\mathcal{C}_0$ are the objects you are interested in studying, and for objects $A,B\in \mathcal{C}_0$, the morphisms $\Mor _{\mathcal{C}}(A,B)$ are the maps relevant to the study of the objects in $\mathcal{C}$.  For us, it will usually be the case that every element of $\mathcal{C}_0$ is a set equipped with extra structure (e.g.~a binary operation) and the morphisms are just the functions that `preserve' this structure (e.g.~homomorphisms).

At the moment, this might seem a bit abstract because of the lack of examples.  As you continue through the main text, you will encounter more examples of categories, which will likely elucidate this abstract definition.  However, even already we have a couple basic examples of categories.
\begin{exm}[The category of sets]\label{exm1.2.2}
The category of sets is the category $\Set$\index[notation]{$\Set$} whose collection of objects $\Set _0$ is the collection of all sets,\footnote{This is not paradoxical as the collection of all sets is not itself a set.} for every set $X$ and every set $Y$ the collection of morphisms from $X$ to $Y$, $\Mor _{\Set}(X,Y)$, is precisely the set of all functions from $X$ to $Y$,\footnote{Note that each $\Mor _{\Set}(X,Y)$ itself is a set.} composition is given by ordinary function composition, and the identities of the category are the identity functions.
\end{exm}
We also have another example at our disposal, namely the category of preordered sets.
\begin{exm}[The category of preordered sets]
The category of preordered sets is the category $\Pre$\index[notation]{$\Pre$} whose collection of objects $\Pre _0$ is the collection of all preordered sets, for every preordered set $X$ and preordered set $Y$ the collection of morphisms from $X$ to $Y$, $\Mor _{\Pre}(X,Y)$, is precisely the set of all nondecreasing functions from $X$ to $Y$, composition is given by ordinary function composition, and the identities of the category are the identity functions.
\end{exm}
The idea here is that the only structure on a preordered set is the preorder, and that the precise notion of what it means to `preserve' this structure is to be nondecreasing.  Of course, we could everywhere replace the word ``preorder'' (or its obvious derivatives) with ``partial-order'' or ``total-order'' and everything would make just as much sense.  Upon doing so, we would obtain the category of partially-ordered sets and the category of totally-ordered sets respectively.

We also have the category of magmas.
\begin{exm}[The category of magmas]
The category of magmas is the category $\Mag$\index[notation]{$\Mag$} whose collection of objects $\Mag _0$ is the collection of all magmas, for every magma $X$ and every magma $Y$ the collection of morphisms from $X$ to $Y$, $\Mor _{\Mag}(X,Y)$, is precisely the set of all homomorphisms from $X$ to $Y$, composition is given by ordinary function composition, and the identities of the category are the identity functions.
\end{exm}
Similarly, the idea here is that the only structure here is that of the binary operation (and possibly an identity element) and that it is the homomorphisms which preserve this structure.  Of course, we could everywhere here replace the word ``magma'' with ``semigroup'', ``monoid'', ``group'', etc.~and everything would make just as much sense.  Upon doing so, we would obtain the categories of semigroups, the category of monoids, and the category of groups respectively.

Finally we have the category of rgs.
\begin{exm}[The category of rgs]
The category of rgs is the category $\Rg$\index[notation]{$\Rg$} whose collection of objects $\Rg _0$ is the collection of all rgs, for every rg $X$ and every rg $Y$ the collection of morphisms from $X$ to $Y$, $\Mor _{\Rg}(X,Y)$, is precisely the set of all homomorphisms from $X$ to $Y$, composition is given by ordinary function composition, and the identities of the category are the identity functions.
\begin{rmk}
The same as before, we could have everywhere replaced the word ``rg'' with ``rig'', ``rng'', or ``ring''.  These categories are denoted $\Rig$\index[notation]{$\Rig$}, $\Rng$\index[notation]{$\Rng$}, and $\Ring$\index[notation]{$\Ring$} respectively.
\end{rmk}
\end{exm}

\subsection{Some basic notation}

The real reason we introduce the definition of a category in notes like these is that it allows us to introduce consistent notation and terminology throughout the text.  Had we forgone even the very basics of categories, we would still be able to do the same mathematics, but the notation and terminology would be much more ad hoc.
\begin{dfn}[Domain and codomain]
Let $f:A\rightarrow B$ be a morphism in a category.  Then, the \emph{domain}\index{Domain (of a morphism)} of $f$ is $A$ and the \emph{codomain}\index{Codomain (of a morphism)} of $f$ is $B$.
\begin{rmk}
Of course, these terms generalize the notions of domain and codomain for sets.
\end{rmk}
\end{dfn}
\begin{dfn}[Endomorphism]\label{Endomorphism}
Let $\mathcal{C}$ be a category and let $A\in \mathcal{C}_0$.  Then, an \emph{endomorphism}\index{Endomorphism} is a morphism $f\in \Mor _{\mathcal{C}}(A,A)$.  We write $\End _{\mathcal{C}}(A)\coloneqq \Mor _{\mathcal{C}}(A,A)$\index[notation]{$\End _{\mathcal{C}}(A)$} for the collection of endomorphisms on $A$.
\end{dfn}
In other words, ``endomorphism'' is just a fancy name for a morphism with the same domain and co-domain.

\begin{dfn}[Isomorphism]\label{Isomorphism}
Let $f:A\rightarrow B$ be a morphism in a category.  Then, $f$ is an \emph{isomorphism}\index{Isomorphism} iff it is invertible, i.e., iff there is a morphism $g:B\rightarrow A$ such that $g\circ f=\id _A$ and $f\circ g=\id _B$.  In this case, $g$ is an \emph{inverse} of $f$.  The collection of all isomorphisms from $A$ to $B$ is denoted by $\Iso _{\mathcal{C}}(A,B)$\index[notation]{$\Iso _{\mathcal{C}}(A,B)$}.
\end{dfn}
\begin{exr}[Inverses are unique]
Let $f:A\rightarrow B$ be a morphism in a category and let $g,h:B\rightarrow A$ be two inverses of $f$.  Show that $g=h$.
\begin{rmk}
As a result of this exercise, we may denote \emph{the} inverse of $f$ by $f^{-1}$.\footnote{If inverses were not unique, then the notation $f^{-1}$ would be ambiguous:  what inverse would we be referring to?}
\end{rmk}
\end{exr}
\begin{exr}
Show that a morphism in $\Mag$ is an isomorphism iff (i) it is bijective, (ii) it is a homomorphism, and (iii) its inverse is a homomorphism.
\end{exr}
\begin{exr}\label{exrA.2.11x}
Show that the inverse of a bijective homomorphism of magmas is itself a homomorphism.
\begin{rmk}
Thus, if you want to show a function is an isomorphism of magmas, in fact you only need to check (i) and (ii) of the previous exercise, because then you get (iii) for free.  (Of course, essentially the very same thing happens in $\Rg$ as well.)
\end{rmk}
\end{exr}
\begin{dfn}[Isomorphic]\label{dfnA.2.10}
Let $A,B\in \mathcal{C}_0$ be objects in a category.  Then, we $A$ and $B$ are \emph{isomorphic}\index{Isomorphic} iff there is an isomorphism from $A$ to $B$.  In this case, we write $A\cong _{\mathcal{C}}B$\index[notation]{$A\cong _{\mathcal{C}}B$}, or just $A\cong B$\index[notation]{$A\cong B$} if the category $\mathcal{C}$ is clear from context.
\end{dfn}
\begin{dfn}[Automorphisms]
Let $\mathcal{C}$ be a category and let $A\in \mathcal{C}_0$.  Then, an \emph{automorphism}\index{Automorphism} $f:A\rightarrow A$ is a morphism which is both an endomorphism and an isomorphism.  We write $\Aut _{\mathcal{C}}(A)\coloneqq \Iso _{\mathcal{C}}(A,A)$\index[notation]{$\Aut _{\mathcal{C}}(A)$} for the collection of automorphisms of $A$.
\end{dfn}
\begin{exr}\label{exr2.1.3}
Let $X$ and $Y$ be set.  Show that $\Iso _{\Set}(X,Y)$ is the set of all bijections from $X$ to $Y$.
\end{exr}
\begin{exr}\label{exrA.2.11}
Let $\mathcal{C}$ be a category.  Show that $\cong _{\mathcal{C}}$ is an equivalence relation on $\mathcal{C}_0$.
\end{exr}


%----------------------------------------------------------------------------------------
%	BIBLIOGRAPHY
%----------------------------------------------------------------------------------------

\begin{thebibliography}{9}

\bibitem{Abbott}
Abbot, Stephen.  \textit{Understanding Analysis}.  Springer-Verlag UTM.  (2010).

\bibitem{Cohn}
Cohn, Donald L..  \textit{Measure Theory}, 2nd Ed..  Springer.  (2013).

\bibitem{Coleman}
Coleman, Mark.  \textit{Continuous functions nowhere differentiable}.  \href{http://www.maths.manchester.ac.uk/~mdc/MATH20101/notesPermanant/Contsnondiff.pdf}{Constnondiff.pdf}

\bibitem{Gelbaum}
Gelbaum, Bernard R.; Olmsted, John M.~H..  \textit{Counterexamples in Analysis}.  Dover Publications Inc..  (1992).

\bibitem{GleasonMeasure}
Gleason, Jonatahn.  \textit{A homeomorphism of $\R$ that doesn't preserve lebesgue measurability}.  (2009).  \url{https://www.academia.edu/14753183/A_homeomorphism_of_mathbb_R_that_doesnt_preserve_lebesgue_measurability}.

\bibitem{GleasonHaar}
Gleason, Jonathan.  \textit{Existence and uniqueness of haar measure}.  (2010).  \url{https://www.academia.edu/740208/Existence_and_Uniqueness_of_Haar_Measure}.

\bibitem{Grayson}
Grayson, Daniel R..  \textit{Zorn's Lemma}.

\bibitem{Hart}
Hart, K.~P.; Nagata, J.; Vaughn, J.~E..  \textit{Encyclopedeia of General Topology}.  Elsevier Science.  (2004).

\bibitem{Honig}
H\"{o}nig, Chaim Samuel.  \textit{Proof of the well-ordering of cardinal numbers}

\bibitem{Howes}
Howes, Norman R..  \textit{Modern Analysis and Toology}.  Springer-Verlag Universitext.  (1991).

\bibitem{Isbell}
Isbell, J.~R.~.  \textit{Uniform spaces}.  American Mathematical Society.  (1964).

\bibitem{Kelley}
Kelley, John L..  \textit{General Topology}.  Springer-Verlag GTM.  (1955).

\bibitem{mathoverflow}
\href{http://mathoverflow.net/}{mathoverflow}

\bibitem{stackexchange}
\href{http://math.stackexchange.com/}{math.stackexchange}

\bibitem{Munkres}
Munkres, James.  \textit{Topology}, 2nd Ed..  Prentice Hall.  (2000).

\bibitem{Pugh}
Pugh, Charles Chapman.  \textit{Real Mathematical Analysis}.  Springer-Verlag UTM.  (2002).

\bibitem{Ross}
Ross, Kenneth.  \textit{Elementary Analysis:  The Theory of Calculus}.  Springer-Verlag UTM.  (2013).

\bibitem{Rudin}
Rudin, Walter.  \textit{Principles of Mathematical Analysis}, 3rd Ed..  McGraw-Hill.  (1976).

\bibitem{BigRudin}
Rudin, Walter.  \textit{Real and Complex Analysis}, 3rd Ed..  McGraw-Hill.  (1987).

\bibitem{Steen}
Steen, Lynn Arthur.  Seebach Jr., J.~Arthur.  \textit{Counterexamples in Topology}.  Holt, Rinehart and Winston Inc..  (1970).

\bibitem{Stein}
Stein, Elias M.  Shakarchi, Rami.  \textit{Real Analysis---Meas Theory, Integration, and Hilbert Spaces}.  Princeton University Press.  (2005).

\bibitem{Thomas}
Thomas, J..  \textit{A regular space, not completely regular}.  The American Mathematical Monthly, Vol.~76, No.~2, pp.~181--182.  (1969).

\bibitem{Wikipedia}
Wikipedia

\end{thebibliography}

%----------------------------------------------------------------------------------------
%	INDEX
%----------------------------------------------------------------------------------------

\cleardoublepage
\phantomsection
\setlength{\columnsep}{0.75cm}
\addcontentsline{toc}{chapter}{\textcolor{ocre}{Index}}
\printindex
\printindex[notation]

%----------------------------------------------------------------------------------------

\end{document}
\usepackage{xparse}

% 'Punctuation'
\DeclarePairedDelimiter{\abs}{\lvert}{\rvert}
\newcommand{\blank}{{-}}
\newcommand{\blankdot}{{\cdot}}
\newcommand{\comma}{{,}}
\DeclarePairedDelimiter{\coord}{\langle}{\rangle}

% Operations
\DeclareMathOperator{\Ad}{Ad}
\DeclareMathOperator{\ad}{ad}
\DeclareMathOperator{\Aut}{Aut}
\DeclareMathOperator{\CO}{CO}
\newcommand{\Cech}{\check{\mathrm{C}}}
\newcommand{\Cl}{\mathcal{C}\ell}
\DeclareMathOperator{\Cls}{Cls}
\DeclareMathOperator{\Cmp}{Cmp}
\DeclareMathOperator{\co}{co}
\renewcommand{\c}{\mathrm{c}}
\newcommand{\comp}{\mathsf{c}}
\DeclareMathOperator{\Char}{Char}
\DeclareMathOperator{\Coker}{Coker}
\DeclareMathOperator{\coker}{coker}
\DeclareMathOperator{\colim}{colim}
\DeclareMathOperator{\csch}{csch}
\newcommand{\Curl}{\vec{\nabla}\times}
\DeclareMathOperator{\D}{D}
\newcommand{\Div}{\vec{\nabla}\cdot}
\DeclareMathOperator{\diag}{diag}
\DeclareMathOperator{\diam}{diam}
\newcommand{\dif}{\mathop{}\!\mathrm{d}}
\DeclareMathOperator{\dist}{dist}
\DeclareMathOperator{\EV}{E}
\DeclareMathOperator{\End}{End}
\DeclareMathOperator{\Ext}{Ext}
\DeclareMathOperator{\Erf}{Erf}
\DeclareMathOperator{\Gal}{Gal}
\newcommand{\Grad}{\vec{\nabla}}
\DeclareMathOperator{\Hom}{Hom}
\DeclareMathOperator{\Ima}{Im}
\DeclareMathOperator{\Imag}{Im}
\DeclareMathOperator{\Index}{Index}
\DeclareMathOperator{\Inn}{Inn}
\DeclareMathOperator{\Iso}{Iso}
\DeclareMathOperator{\id}{id}
\DeclareMathOperator{\Int}{Int}
\DeclareMathOperator{\Ker}{Ker}
\DeclareMathOperator{\lcm}{lcm}
\DeclareMathOperator{\Lie}{Lie}
\DeclareMathOperator{\MaxSpec}{MaxSpec}
\DeclareMathOperator{\Mor}{Mor}
\newcommand{\m}{\mathrm{m}}
\newcommand{\meas}{\m}
\NewDocumentCommand{\metric}{o o}{\IfNoValueTF{#1}{\IfNoValueTF{#2}{\abs{\blankdot \comma \blankdot}}{\abs*{\blankdot \comma \, #2}}}{\IfNoValueTF{#2}{\abs*{#1\comma \blankdot}}{\abs*{#1\comma \, #2}}}}
\newcommand{\mmod}[1]{\left( \mathrm{mod}\, #1\right)}
\DeclareMathOperator{\Nil}{Nil}
\NewDocumentCommand{\norm}{o}{\IfNoValueTF{#1}{\abs{\blankdot}}{\abs*{#1}}}
\renewcommand{\o}{\mathrm{o}}
\DeclareMathOperator{\op}{op}
\DeclareMathOperator{\ord}{ord}
\newcommand{\q}{\mathrm{q}}
\DeclareMathOperator{\Real}{Re}
\DeclareMathOperator{\Res}{Res}
\newcommand\restr[2]{{% we make restriction an ordinary symbol
  \left.\kern-\nulldelimiterspace % automatically resize the bar with \right
  #1 % the function
%  \vphantom{\big|} % pretend it's a little taller at normal size
  \right|_{#2} % this is the delimiter
  }}
\DeclareMathOperator{\rk}{rk}
\DeclareMathOperator{\sgn}{sgn}
\DeclareMathOperator{\Spec}{Spec}
\DeclareMathOperator{\Star}{Star}
\DeclareMathOperator{\supp}{supp}
\DeclareMathOperator{\Syl}{Syl}
\DeclareMathOperator{\T}{T}
\DeclareMathOperator{\Tp}{Tp}
\DeclareMathOperator{\TZero}{T0}
\NewDocumentCommand{\tangent}{m m}{\T _{#2}(#1)}    % For tangent space
\DeclareMathOperator{\tp}{tp}
\DeclareMathOperator{\Tr}{Tr}
\DeclareMathOperator{\tr}{tr}

\makeatletter
\DeclareFontFamily{U}{mathx}{\hyphenchar\font45}
\DeclareFontShape{U}{mathx}{m}{n}{
      <5> <6> <7> <8> <9> <10>
      <10.95> <12> <14.4> <17.28> <20.74> <24.88>
      mathx10
      }{}
\DeclareSymbolFont{mathx}{U}{mathx}{m}{n}
\def\mathabx@undefine#1{\let#1=\undefined}%
\mathabx@undefine{\prod}
\DeclareMathSymbol{\prod}   {1}{mathx}{"B1}
\mathabx@undefine{\coprod}
\DeclareMathSymbol{\coprod} {1}{mathx}{"B2}
\makeatother % The preceding lines are to use the symbols \prod and \coprod from mathabx package (mathptmx disallows \coprod by default for some reason, also use \prod from mathabx for consistency)

\makeatletter
\def\re@DeclareMathSymbol#1#2#3#4{%
    \let#1=\undefined
    \DeclareMathSymbol{#1}{#2}{#3}{#4}}
% no OMX used 
\expandafter\ifx\csname npxmath@scaled\endcsname\relax
  \let\npxmath@@scaled\@empty%
\else
  \edef\npxmath@@scaled{s*[\csname npxmath@scaled\endcsname]}%
\fi
\DeclareFontEncoding{LMX}{}{}
\DeclareFontSubstitution{LMX}{npxexx}{m}{n}
\DeclareFontFamily{LMX}{npxexx}{}
\DeclareFontShape{LMX}{npxexx}{m}{n}{<-> \npxmath@@scaled zplexx}{}
\DeclareSymbolFont{lettersA}{U}{npxmia}{m}{it}
\re@DeclareMathSymbol{\epsilonup}{\mathord}{lettersA}{15}
\makeatother
\renewcommand{\epsilon}{\epsilonup} % The preceding lines are to used the symbol \epsilonup from the newpx package.  We then use this for \epsilon as the other fonts use the ssame symbol for \epsilon and \varepsilon

% Relations
\DeclareFontFamily{U}{mathb}{\hyphenchar\font45}
\DeclareFontShape{U}{mathb}{m}{n}{
      <5> <6> <7> <8> <9> <10> gen * mathb
      <10.95> mathb10 <12> <14.4> <17.28> <20.74> <24.88> mathb12
      }{}
\DeclareSymbolFont{mathb}{U}{mathb}{m}{n}
\DeclareMathSymbol{\llcurly}     {3}{mathb}{"CE}
\DeclareMathSymbol{\ggcurly}     {3}{mathb}{"CF} % The preceding lines are to use just the symbols \llcurly and \ggcurly from mathabx package (which itself conflicts with amsmath)

\newcommand*{\llcurlyeq}{\mathrel{\vbox{\offinterlineskip\hbox{\scalebox{.9}{$\llcurly$}}\vskip-.6ex\hbox{$-$}\vskip-.75ex}}}
\newcommand*{\ggcurlyeq}{\mathrel{\vbox{\offinterlineskip\hbox{\scalebox{.9}{$\ggcurly$}}\vskip-.6ex\hbox{$-$}\vskip-.75ex}}}

\newcommand{\st}{\ensuremath{\text{ s.t.~}}}

% Constants
\newcommand{\e}{\mathrm{e}}
\newcommand{\im}{\mathrm{i}\text{\hspace{1pt}}}

% 'Sets'
\newcommand{\A}{\mathbb{A}}
\renewcommand{\C}{\mathbb{C}}
\newcommand{\F}{\mathbb{F}}
\newcommand{\K}{\mathbb{K}}
\newcommand{\N}{\mathbb{N}}
\newcommand{\Q}{\mathbb{Q}}
\newcommand{\R}{\mathbb{R}}
\newcommand{\Z}{\mathbb{Z}}

% Categories
\newcommand{\AlE}{\mathsf{AlE}}
\newcommand{\Ab}{\mathsf{Ab}}
\newcommand{\Alg}{\mathsf{Alg}}
\newcommand{\Cat}{\mathsf{Cat}}
\newcommand{\Com}{\mathsf{Com}}
\newcommand{\Crng}{\mathsf{Crng}}
\newcommand{\Grp}{\mathsf{Grp}}
\newcommand{\Mag}{\mathsf{Mag}}
\newcommand{\Man}{\mathsf{Man}}
\newcommand{\Met}{\mathsf{Met}}
\newcommand{\Pre}{\mathsf{Pre}}
\newcommand{\Rg}{\mathsf{Rg}}
\newcommand{\Rig}{\mathsf{Rig}}
\newcommand{\Ring}{\mathsf{Ring}}
\newcommand{\Rng}{\mathsf{Rng}}
\newcommand{\Semi}{\mathsf{Semi}}
\newcommand{\Set}{\mathsf{Set}}
\newcommand{\Simp}{\mathsf{Simp}}
\newcommand{\SN}{\mathsf{SN}}
\newcommand{\Top}{\mathsf{Top}}
\newcommand{\Uni}{\mathsf{Uni}}
\newcommand{\Vect}{\mathsf{Vect}}

% Custom environments
\NewDocumentEnvironment{textequation}{o}{
    \IfNoValueTF{#1}{%
        \ntextequation%
    }%
    {%
        \ytextequation{#1}%
    }
}
{
    \IfNoValueTF{#1}{%
        \endntextequation%
    }%
    {%
        \endytextequation%
    }%
}
\NewEnviron{ytextequation}[1]{%
    \begin{equation}\label{#1}%
        \pbox{.875\textwidth}{%
            \BODY%
        }%
    \end{equation}
}
\NewEnviron{ntextequation}{%
    \begin{equation}%
        \pbox{.875\textwidth}{%
            \BODY%
        }%
    \end{equation}
}
\NewEnviron{textequation*}{%
    \begin{equation*}%
    \pbox{.9\textwidth}{%
    \BODY%
    }%
    \end{equation*}
}

\newenvironment{solution}
    {\proof[Solution]}
    {\endproof}
    
% Custom commands
\newcommand{\blankline}{\vspace{\baselineskip}}

\newcounter{step}
\crefname{step}{Step}{Steps}
\newcommand{\Step}[1]{%
    \refstepcounter{step}%
    %
    \ifthenelse{\value{step}=1}{}{%
        \par%
        \blankline
    }
    \noindent
    \textsc{Step  \thestep :  #1}
    
    \noindent
}

% Re-defines proof environment to reset step counter
\makeatletter
\let\oldproof\proof
\def\proof{\setcounter{step}{0}\@ifnextchar[\proof@i \proof@ii}
\def\proof@i[#1]{\proof[#1]}
\def\proof@ii{\oldproof}
\makeatother
\section{The definition of a topological space}

We have mentioned topological spaces several times throughout the notes already, and finally we turn to studying general topological spaces themselves.  I said before that I think it is fair to say that the purpose of topological spaces is to introduce the most general context as possible in which one can talk about continuity.  The motivating result in this regard is \cref{thm3.4.16}, which says that a function is continuous iff the preimage of every open set is continuous.  The idea is then to axiomatize the notion of open set:  a topological space will be a set $X$ equipped with a collection of subsets $\mathcal{U}\subseteq 2^X$, called the \emph{open sets}.  Of course, if we want to be able to say anything of interest, we can't just take any old collection of subsets---we must require the collection to satisfy some conditions.  The conditions we require are those that come from \cref{thm3.4.34,thm3.4.36}, namely, that an arbitrary union and finite intersection of open sets is open.  So, without further ado, I present to you, the definition of a topological space.
\begin{dfn}[Topological space]\label{TopologicalSpace}
A \emph{topological space}\index{Topological space} is a set $X$ equipped with a collection of subsets $\mathcal{U}\subseteq 2^X$, the \emph{topology}\index{Topology} on $X$, such that
\begin{enumerate}
\item \label{enmTopologicalSpace.i}$\emptyset ,X\in \mathcal{U}$;
\item \label{enmTopologicalSpace.ii}$\bigcup _{U\in \mathcal{V}}U\in \mathcal{U}$ if $\mathcal{V}\subseteq \mathcal{U}$; and
\item \label{enmTopologicalSpace.iii}$\bigcap _{k=1}^mU_k\in \mathcal{U}$ if $U_k\in \mathcal{U}$ for $1\leq k\leq m$.
\end{enumerate}
\begin{rmk}
The elements of $\mathcal{U}$ are \emph{open sets}\index{Open (in a topological space)}.
\end{rmk}
\begin{rmk}
A subset $C\subseteq X$ is \emph{closed}\index{Closed (in a topological space)} iff $C^{\comp}$ is open.  (Recall that this is precisely the definition we gave in $\R$---see \cref{dfn3.4.17}.)
\end{rmk}
\begin{rmk}
In order to exclude stupid things like the empty topology, we require that the empty-set and the entire set are open.  (Recall that these were open in $\R$---see \cref{exr3.4.13,exr3.4.14}.)
\end{rmk}
\end{dfn}
Of course, you can specify a topology just as well by saying what the closed sets are (the open sets are then just the complements of these sets).
\begin{exr}\label{exr4.1.2}
Let $X$ be a set and let $\mathcal{C}\subseteq 2^X$ be a collection of subsets of $X$ such that
\begin{enumerate}
\item $\emptyset ,X\in \mathcal{C}$;
\item $\bigcap _{C\in \mathcal{D}}C$ if $\mathcal{D}\subseteq \mathcal{C}$; and
\item $\bigcup _{k=1}^mC_k\in \mathcal{C}$ if $C_k\in \mathcal{C}$ for $1\leq k\leq m$.
\end{enumerate}
Show that there is a unique topology on $X$ whose collection of closed sets is precisely $\mathcal{C}$.
\begin{rmk}
In other words, your collection of closed sets must be nontrivial, closed under arbitrary intersection, and closed under finite union, just as was the case in $\R$---see \cref{exr3.4.38x,exr3.4.40}.
\end{rmk}
\end{exr}
\begin{dfn}[$G_\delta$ and $F_\sigma$ sets]\label{GDeltaFSigma}
Let $S\subseteq X$ be a subset of a topological space.  Then, $S$ is a \emph{$G_\delta$ set}\index{$G_\delta$ set} iff $S$ is the countable intersection of open sets.  $S$ is an \emph{$F_\sigma$ set}\index{$F_\sigma$ set} iff $S$ i the countable union of closed sets.
\begin{rmk}
For us, these concepts will not come-up very often, so it is not imperative that you remember these terms.  They do come-up, however, and so it would be incomplete to not include them.  And of course, others use this terminology so you should at least know where to refer back to if you ever need to know.
\end{rmk}
\end{dfn}

\subsection{Bases, neighborhood bases, and generating collections}

It is often not necessary to define every single open set explicitly, but rather, only a special class of open sets that determine all the others.  For example, in the real numbers, to determine whether an arbitrary set was open, we made use of the `special' open sets $B_\varepsilon (x)$.  The idea that generalizes this notion is that of a \emph{base} for a topology.
\begin{dfn}[Base]\label{Base}
Let $X$ be a topological space and let $\mathcal{B}$ be a collection of open sets of $X$.  Then, $\mathcal{B}$ is a \emph{base}\index{Base} for the topology of $X$ iff the statement that a subset $U$ of $X$ is open is equivalent to the statement that, for every $x\in U$, there is some $B\in \mathcal{B}$ such that $x\in B\subseteq U$.
\end{dfn}
\begin{exm}
The collection $\left\{ B_\varepsilon (x):x\in \R ,\ \varepsilon >0\right\}$ is a base for the topology of $\R$ (by definition).
\end{exm}
The real reason bases are important is because they allow us to \emph{define} topologies, and so it is important to know when a collection of subsets of a set form a base for some topology.
\begin{prp}\label{prp4.1.5}
Let $X$ be a set and let $\mathcal{B}$ be a collection of subsets of $X$.  Then, there exists a unique topology for which $\mathcal{B}$ is a base iff
\begin{enumerate}
\item \label{enm4.1.4.i}$\mathcal{B}$ covers $X$; and
\item \label{enm4.1.4.ii}for every $x\in X$ and $B_1,B_2\in \mathcal{B}$ with $x\in B_1,B_2$, there is some $B_3\in \mathcal{B}$ such that $x\in B_3\subseteq B_1\cap B_2$.
\end{enumerate}
\begin{rmk}
If $X$ does not a priori come with a topology, we will still refer to any collection of sets that satisfy \ref{enm4.1.4.i}--\ref{enm4.1.4.ii} as a \emph{base}.
\end{rmk}
\begin{proof}
$(\Rightarrow )$ Suppose that there is a unique topology $\mathcal{U}$ for which $\mathcal{B}$ is a base.  We first show that $\mathcal{B}$ covers $X$.  We proceed by contradiction:  suppose there is some $x\in X$ which is not contained in any $B\in \mathcal{B}$.  Then, as $\mathcal{B}$ is a base for the topology, $X$ would not be open:  a contradiction.  Therefore, $\mathcal{B}$ covers $X$.

Now for the second property.  By the definition of a base, we have that $\mathcal{B}\subseteq \mathcal{U}$.  In particular, $B_1\cap B_2\in \mathcal{U}$, and so because $\mathcal{B}$ is a base for $\mathcal{U}$, there must be some $B_3\in \mathcal{B}$ such that $B_3\subseteq B_1\cap B_2$.

\blankline
\noindent
$(\Leftarrow )$ Suppose that (i) $\mathcal{B}$ covers $X$, and (ii) for every for every $B_1,B_2\in \mathcal{B}$ intersecting, there is some $B_3\in \mathcal{B}$ such that $B_3\subseteq B_1\cap B_2$.  We declare $U\subseteq X$ to be open iff for every $x\in U$ there is some $B\in \mathcal{B}$ with $x\in B\subseteq U$.  By the definition of bases, this was the only possibility.  We need only check that this is in fact a topology.  The empty-set is vacuously open.  $X$ is open because $\mathcal{B}$ covers $X$.  Let $\mathcal{V}$ be a collection of open sets and let $x\in \bigcup _{U\in \mathcal{V}}U$.  Then, $x\in U$ for some $U\in \mathcal{V}$, and so there is some $B\in \mathcal{B}$ such that $x\in B\subseteq U\subseteq \bigcup _{U\in \mathcal{V}}U$.  Thus, $\bigcup _{U\in \mathcal{V}}U$ is open.  Let $U_1,\ldots ,U_m$ be open and let $x\in \bigcap _{k=1}^mU_k$.  Then, there is some $B_k\in \mathcal{B}$ such that $x\in B_k\subseteq U_k$.  By \ref{enm4.1.4.ii}, there is some $B\in \mathcal{B}$ with $x\in B\subseteq B_1\cap \cdots \cap B_m\subseteq \bigcap _{k=1}^mU_k$, and so $\bigcap _{k=1}^mU_k$ is open.
\end{proof}
\end{prp}
\begin{exr}\label{exr4.1.7}
Let $X$ be a topological space and let $\mathcal{B}$ be a base for the topology on $X$.  Show that every open set is a union of elements of $\mathcal{B}$.
\end{exr}
There is a similar way of defining a topology.  Instead of specifying a base for all open sets, you specify a base for all the open sets at a point (see the following definitions) for every point.
\begin{dfn}[Neighborhood]\label{Neighborhood}
Let $X$ be a topological space and let $S\subseteq N\subseteq X$.  Then, $N$ is a \emph{neighborhood}\index{Neighborhood} of $S$ iff there is some open set $U\subseteq X$ such that $S\subseteq U\subseteq N$.  An \emph{open neighborhood}\index{Open neighborhood} of $S$ is just an open set which contains $S$.  A(n open) neighborhood of a point $x$ is a(n open) neighborhood of $\{ x\}$.
\end{dfn}
\begin{dfn}[Neighborhood base]\label{NeighborhoodBase}
Let $X$ be a topological space and for each $x\in X$ let $\mathcal{B}_x$ be a collection of neighborhoods\footnote{Not necessarily open!} of $x$.  Then, $\{ \mathcal{B}_x:x\in X\}$ is a \emph{neighborhood base}\index{Neighborhood base} for the topology of $X$ (and $\mathcal{B}_x$ is a \emph{neighborhood base} at $x$) iff the statement that a subset $U$ of $X$ is open is equivalent to the statement that, for every $x\in U$, there is some $B_x\in \mathcal{B}_x$ such that $B_x\subseteq U$.  In the case that every element of each $\mathcal{B}_x$ is open, we say that the neighborhood base is a \emph{local base}\index{Local base}.
\end{dfn}
\begin{exm}[A neighborhood base with no open sets]
Take $X\coloneqq \R$, and for $x\in X$ define $\mathcal{B}_x\coloneqq \{ D_{\varepsilon}(x):\varepsilon >0\}$.\footnote{Recall that $D_{\varepsilon}(x)\coloneqq \{ y\in \R :\abs{y-x}\leq \varepsilon \}$.}  While we have not actually defined a topology on $\R$ yet, once we do very shortly (\cref{exm3.1.21}), you can verify that this is in fact a neighborhood base, but no $D_{\varepsilon}(x)$ is open.
\end{exm}
Once again, neighborhood bases allow us to \emph{define} topologies.
\begin{prp}\label{prp4.1.8}
Let $X$ be a set and for each $x\in X$ let $\mathcal{B}_x$ be a nonempty collection of subsets of $X$ which contain $x$.  Then, there exists a unique topology for which $\mathcal{B}_x$ is a neighborhood base of $x$ iff for every $x\in X$ and $B_1,B_2\in \mathcal{B}_x$, there is some $B_3\in \mathcal{B}_x$ such that $B_3\subseteq B_1\cap B_2$.
\begin{proof}
$(\Rightarrow )$ Suppose that there exists a unique topology for which $\mathcal{B}_x$ is a neighborhood base at $x$.  Let $B_1,B_2\in \mathcal{B}_x$.  Then, $B_1$ and $B_2$ are neighborhoods of $x$, and so there are open sets $U_1\subseteq B_1$ and $U_2\subseteq B_2$ containing $x$.  Then, $U_1\cap U_2$ is open and contains $x$, and therefore, because $\mathcal{B}_x$ is a neighborhood base at $x$, there is some $B_3\in \mathcal{B}_x$ with $B_3\subseteq U_1\cap U_2\subseteq B_1\cap B_2$.

\blankline
\noindent
$(\Leftarrow )$ Suppose that for every $x\in X$ and $B_1,B_2\in \mathcal{B}_x$, there is some $B_3\in \mathcal{B}_x$ such that $B_3\subseteq B_1\cap B_2$.  We declare $U\subseteq X$ to be open iff for every $x\in U$ there is some $B\in \mathcal{B}_x$ with $x\in B\subseteq U$.  By definition of neighborhood bases, this was the only possibility.  We need only check that this is in fact a topology.  The empty-set is vacuously open.  $X$ is open because each $\mathcal{B}_x$ is nonempty.  Let $\mathcal{V}$ be a collection of open sets and let $x\in \bigcup _{U\in \mathcal{V}}U$.  Then, $x\in U$ for some $U\in \mathcal{V}$, and so there is some $B\in \mathcal{B}_x$ such that $x\in B\subseteq U\subseteq \bigcup _{U\in \mathcal{V}}U$.  Thus, $\bigcup _{U\in \mathcal{V}}U$ is open.  Let $U_1,\ldots ,U_m$ be open and let $x\in \bigcap _{k=1}^mU_k$.  Then, $x\in U_k$ for each $k$, and so there is some $B_k\in \mathcal{B}_x$ such that $x\in B_k\subseteq U_k$.  By hypothesis, there is then some $B\in \mathcal{B}_x$ with $x\in B\subseteq B_1\cap \cdots \cap B_m\subseteq \bigcap _{k=1}^mU_k$, and so $\bigcap _{k=1}^mU_k$ is open.
\end{proof}
\end{prp}
Sometimes we have a collection of sets that we would like to be open, but they do not necessarily form a base (or a neighborhood base), and so in this case we cannot just invoke \cref{prp4.1.5}.  However, what we can do is take the `smallest' topology which contains these sets.
\begin{prp}[Generating collection (of a topology)]\label{GeneratingCollection}
Let $X$ be a set and let $\mathcal{S}\subseteq 2^X$.  Then, there exists a unique topology $\mathcal{U}$ on $X$, the topology \emph{generated}\index{Generate (a topology)} by $\mathcal{S}$, such that
\begin{enumerate}
\item $\mathcal{S}\subseteq \mathcal{U}$; and
\item if $\mathcal{U}'$ is any other topology on $X$ containing $\mathcal{S}$, it follows that $\mathcal{U}\subseteq \mathcal{U}'$.
\end{enumerate}
Furthermore, the collection of all finite intersections of elements of $\mathcal{S}\cup \{ X\}$ is a base for this topology.\footnote{We throw $X$ in in case $\mathcal{S}$ did not cover $X$ (recall that bases (\cref{Base}) need to cover the space).}.  $\mathcal{S}$ is a \emph{generating collection}\index{Generating collection (of a topology)}.
\begin{rmk}
You should compare this with the definitions of the integers (\cref{Integers}), rationals (\cref{RationalNumbers}), closure (\cref{Closure}), and interior (\cref{Interior}).
\end{rmk}
\begin{proof}
Let $\mathcal{B}$ be the collection of all finite intersections of elements of $\mathcal{S}\cup \{ X\}$.  This definitely covers $X$ as $X\in \mathcal{B}$.  Furthermore, the intersection of any two elements of $\mathcal{B}$ is also an element of $\mathcal{B}$, by definition.  Therefore, there is a unique topology $\mathcal{U}$ on $X$ for which $\mathcal{B}$ is a base (\cref{prp4.1.5}).

By construction, $\mathcal{S}\subseteq \mathcal{B}\subseteq \mathcal{U}$.  On the other hand, if $\mathcal{U}'$ is any other topology for which every element of $\mathcal{S}$ is open, then, because topologies are closed under finite intersection, $\mathcal{U}'$ must contain $\mathcal{B}$, and hence it must contain $\mathcal{U}$ (because $\mathcal{U}'$ is closed under arbitrary union and every element of $\mathcal{U}$ is a union of elements of $\mathcal{B}$ (\cref{exr4.1.7})).
\end{proof}
\end{prp}
Generating collections are actually quite nice because several things can be checked just by looking at generating collections, which are generally significantly `smaller'\footnote{Though not literally in the sense of cardinality.} than the entire topology---see \cref{exr4.1.27,exr4.2.41}, and the \nameref{AlexanderSubbaseTheorem} (\cref{AlexanderSubbaseTheorem}).

\subsection{Some basic examples}

We can use the notion of a base to define the \emph{order topology}.
\begin{dfn}[Order topology]\label{OrderTopology}
Let $X$ be a totally-ordered set, and let $\mathcal{B}$ be the collection of sets of the form $(a,b)\coloneqq \left\{ x\in X:a<x<b\right\}$ for $a,b\in X$ with $a<b$.  (We also allow $a=-\infty$ and $b=+\infty$, in which case $(a,+\infty )\coloneqq \left\{ x\in X:x>a\right\}$ and similarly for $(-\infty ,b)$ and $(-\infty, +\infty)$.)
\begin{exr}
Use \cref{Base} to show that $\mathcal{B}$ is a base for a topology.
\end{exr}
The topology defined by $\mathcal{B}$ is the \emph{order topology}\index{Order topology} on $X$.
\end{dfn}
\begin{exm}[A partially-ordered set whose open intervals do not form a base]
Define
\begin{equation}
X\coloneqq \{ x,y_1,y_2,y_3,z_1,z_2\} ,
\end{equation}
and declare that
\begin{equation}
x\leq y_k\text{ and }z_1\geq y_1,y_2\text{ and }z_2\geq y_2,y_3
\end{equation}
for all $k$ ((and then take the `reflexive and transitive closure', so that, for example, we also have $x\leq z_l$, $y_k\leq y_k$, etc.)  Then,
\begin{equation}
(x,z_1)=\{ y_1,y_2\} \text{ and }(x,z_2)=\{ y_2,y_3\} ,
\end{equation}
so that
\begin{equation}
(x,z_1)\cap (x,z_2)=\{ y_2\} .
\end{equation}
However, $\{ y_2\}$ is not an interval because if we had $\{ y_2\} =(a,b)$ for $a<y_2$ and $b>y_2$, we would necessarily have $a=x$ or $a=-\infty$, and $b=z_1,z_2,+\infty$.  In all of these $6$ cases, $(a,b)$ must contain at least either $y_1$ or $y_3$ as well, and so in particular, it would not be the case that $\{ y_2\} =(a,b0$.
\begin{rmk}
This is why we only define the order topology for \emph{totally}-ordered sets.
\end{rmk}
\end{exm}
\begin{exm}[$\N$, $\Z$, $\Q$, and $\R$]\label{exm3.1.21}
$\N$, $\Z$, $\Q$, and $\R$ are all totally-ordered and so we may (and do) equip them with the order topology.
\end{exm}
The order topology on $\N$ and $\Z$ are examples of a very special topology, the `finest' one possible, namely the \emph{discrete topology}.
\begin{dfn}[Discrete topology]
Let $X$ be any set.  The \emph{discrete topology}\index{Discrete topology} is the topology in which \emph{every} subset is open.
\begin{rmk}
In other words, the topology is just the power-set of $X$.  This is a topology for tautological reasons.
\end{rmk}
\end{dfn}
\begin{prp}\label{prp4.1.11}
The order topologies on $\N$ and $\Z$ are both discrete.
\begin{proof}
We prove that the order topology on $\Z$ is discrete and leave the case of $\N$ as an exercise (a very similar argument will work).  We wish to show that every subset of $\Z$ is open.  To show this, it suffices to show that each singleton set is open because an arbitrary set is going to be a union of singletons.  So, let $m\in \Z$.  To show that $\{ m\}$ is open, it suffices to find $a,b\in \Z$ such that $(a,b)=\{ m\}$ (because every element in the base of a topology is automatically open).  Take $a\coloneqq m-1$ and $b\coloneqq m+1$.  Recall (\cref{exr1.2.14}) that there is no integer between $0$ and $1$.  It follows that there is no integer between $k$ and $k+1$ for all $k\in \Z$, and so indeed, $(m-1,m+1)=\{ m\}$.
\begin{exr}
Show that the order topology on $\N$ is discrete.
\end{exr}
\end{proof}
\end{prp}
\begin{exr}
Why is the topology on $\Q$ not discrete?
\end{exr}
The discrete topology is the `finest' topology you can have (finer means more open sets).  The `coarsest' topology you can have is the \emph{indiscrete topology}.
\begin{dfn}[Indiscrete topology]\label{dfnIndiscreteTopology}
Let $X$ be any set.  The \emph{indiscrete topology}\index{Indiscrete topology} is just $\{ \emptyset ,X\}$.
\begin{rmk}
The definition of a topology requires that at least the empty-set and the entire set are open.  The indiscrete topology is when nothing else is open.  It is not particularly useful, but it can be an easy source of some counter-examples (cf.~\cref{exm4.1.20}).
\end{rmk}
\end{dfn}

\subsection{Continuity}

We now finally vastly generalize our definition of continuity.
\begin{dfn}[Continuous function]
Let $f:X\rightarrow Y$ be a function between topological spaces and let $x\in X$.  Then, $f$ is \emph{continuous}\index{Continuous (at a point)} iff the preimage of every open neighborhood of $f(x)$ is an open neighborhood of $x$.  $f$ is \emph{continuous}\index{Continuous} iff $f$ is continuous at $x$ for all $x\in X$.
\end{dfn}
\begin{exr}
Show that $f:X\rightarrow Y$ is continuous iff the preimage of every open set is open.
\end{exr}
In fact, we can do a bit better than this.
\begin{exr}\label{exr4.1.27}
Let $X$ and $Y$ be topological spaces and let $\mathcal{S}$ be generate the topology of $Y$.  Show that $f$ is continuous iff $f^{-1}(S)$ is open for all $S\in \mathcal{S}$.
\begin{rmk}
In particular, this is true for $\mathcal{S}$ a \emph{base} for the topology of $Y$
\end{rmk}
\end{exr}
\begin{exr}
Let $f:X\rightarrow Y$ be any function between two topological spaces.  Show that, if $X$ has the discrete topology, then $f$ is continuous.
\begin{rmk}
In other words, every function on a discrete space is continuous.
\end{rmk}
\end{exr}

\begin{exm}[The category of topological spaces]
The category of topological spaces is the category $\Top$\index[notation]{$\Top$} whose collection of objects $\Top _0$ is the collection of all topological spaces, for every topological space $X$ and topological space $Y$ the collection of morphisms from $X$ to $Y$, $\Mor _{\Top}(X,Y)$, is precisely the set of all continuous functions from $X$ to $Y$, composition is given by ordinary function composition, and the identities of the category are the identity functions.
\begin{exr}
Show that the composition of two continuous functions is continuous.
\begin{rmk}
Note that this is something you need to check in order for $\Top$ to actually form a category ($\Mor _{\Top}(X,Y)$ needs to be closed under composition).  You also need to verify that the identity function is continuous, but this is trivial (the preimage of a set is itself, so\textellipsis ).
\end{rmk}
\end{exr}
\end{exm}
\begin{dfn}[Homeomorphism]\label{Homeomorphism}
Let $f:X\rightarrow Y$ be a function between topological spaces.  Then, $f$ is a \emph{homeomorphism}\index{Homeomorphism} iff $f$ is an isomorphism in $\Top$.
\end{dfn}
\begin{exr}
Show that a function is a homeomorphism iff (i) it is bijective, (ii) it is continuous, and (iii) its inverse is continuous.
\end{exr}
\begin{exr}\label{exr3.1.34}
Find an example of a function that is bijective and continuous, but not a homeomorphism.
\begin{rmk}
Contrast this with, for example, isomorphisms in $\Grp$.  It follows immediately from the definition that a function between groups is an isomorphism in $\Grp$ iff (i) it is bijective, (ii) it is a homomorphism, and (iii) its inverse is a homomorphism.  However, by \cref{exrA.2.11x}, if the original function is a homomorphism, then we get that its inverse is a homomorphism for free, so we only need to actually check (i) and (ii).
\end{rmk}
\end{exr}
\begin{dfn}[Embedding]\label{Embedding}
An \emph{(topological) embedding}\index{Embedding} is a function that is a homeomorphism onto its image.
\begin{rmk}
In other words, an embedding is like a homeomorphism with the exception that it might not be surjective.
\end{rmk}
\begin{rmk}
The word ``topological'' is in parentheses because the word ``embedding'' is used in other contexts (for analogous, but different, things---see, for example, the remark in \cref{prpB.10}).  This is not unlike how we have homomorphisms of rings and homomorphisms of groups.
\end{rmk}
\end{dfn}
\begin{exr}
Show that $f:X\rightarrow Y$ is a homeomorphism iff it is (i) bijective, (ii) it is continuous, and (iii) its inverse is continuous.
\end{exr}
\begin{exr}
What is an example of a function that is (i) bijective, (ii) continuous, but (iii) does not have continuous inverse.
\begin{rmk}
Note the contrast with isomorphisms in the category $\Ring$, for example.  If a function between rings is a bijective homomorphism, then its inverse is automatically a homomorphism.
\end{rmk}
\end{exr}

There is a useful result that is applicable in general whenever you want to check that a ``piece-wise'' function is continuous.
\begin{prp}[Pasting Lemma]\index{Pasting Lemma}\label{PastingLemma}
Let $X$ and $Y$ be topological spaces, let $C_1,C_2\subseteq X$ be closed, and let $f_1:C_1\rightarrow Y$ and $f_2:C_2\rightarrow Y$ be continuous.  Then, if $\restr{f_1}{C_1\cap C_2}=\restr{f_2}{C_1\cap C_2}$, then the function $f:C_1\cup C_2\rightarrow Y$ defined by
\begin{equation}
f(x)\coloneqq \begin{cases}f_1(x) & \text{if }x\in C_1 \\ f_2(x) & \text{if }x\in C_2\end{cases}
\end{equation}
is well-defined and continuous.
\begin{proof}
We leave this as an exercise.
\begin{exr}
Prove this yourself.
\end{exr}
\end{proof}
\end{prp}

\subsection{Some motivation}

At this point, it is reasonable for one to ask ``Why do we care about such generality in an introductory \emph{real} analysis course?  Shouldn't we only be concerned with the real numbers for now?''.  I would argue that, even in the case where you really are only interested in the real numbers, abstraction can shed light onto such a specific example.  The real numbers are many things:  a field, a totally-ordered set, a totally-ordered field, a metric space, a uniform space, a topological space, a manifold, etc..  In mathematics, we are not just concerned about \emph{what} is true, but \emph{why} things are true.  Because the real numbers are so special, having so much structure, there are many things true about them.  But by the same token, because there is so much structure, unless you go through the proofs yourself in detail, it can be difficult to recall \emph{why} things are true---does property XYZ hold for the real numbers because they are a totally-ordered field or because they are a metric space or because they are a topological vector space or because\textellipsis ?  Instead, however, if we step back and only study $\R$ as a topological space, it becomes clearer why certain things are true.  This allows us to tell easily what is true about the real numbers because they are a topological space, as opposed to what is true about the real numbers because they are a field.    For example, consider the following.
\begin{exm}
Though we have not technically defined it yet, hopefully you are familiar with the function $\arctan$ from calculus.  $\arctan :\R \rightarrow (-\frac{\uppi}{2},\frac{\uppi}{2})$ is a homeomorphism.
\end{exm}
Thus, from the point of view of general topology, there is no distinction between $\R$ and $(0,1)$ (see the exercise below).  This is completely analogous to the sense in which there is no difference between $\R$ and $2^{\N}$ at the level of sets.  ($\R$ and $(0,1)$ are isomorphic in $\Top$, and $\R$ and $2^{\N}$ are isomorphic in $\Set$.)  Recall from the very end of \cref{chp1} \nameref{chp1}---morphisms matter.
\begin{exr}
Find a homeomorphism from $(-\frac{\uppi}{2},\frac{\uppi}{2})$ to $(0,1)$.
\end{exr}

Perhaps a good example of the more general context making it clearer \emph{why} things are true is the Intermediate Value Theorem.  Here is the `classical' statement you are probably familiar with from calculus.\footnote{See \cref{ClassicalIntermediateValueTheorem} for the proof.}
\begin{textequation}
Let $f:[a,b]\rightarrow \R$ be continuous.  Then, for all $y$ between $f(a)$ and $f(b)$ (inclusive) there is some $x\in [a,b]$ such that $f(x)=y$.
\end{textequation}
Compare that with the more general statement.\footnote{See \cref{IntermediateValueTheorem} for the proof.}
\begin{textequation}
The continuous image of a connected set is connected.
\end{textequation}
I think it's fair to say that the latter is much more elegant, and perhaps even easier to understand.\footnote{Admittedly, the proof to go from the general statement to the `classical' statement is not completely trivial (a subset of $\R$ is connected iff it is an interval---see \cref{thm4.5.14}), but the two results are still `morally' the same.}

\section{A review}

Quite a many things that we did with the real numbers in the last chapter carry over to general topological spaces no problem.  For convenience, we present here all definitions and theorems which carry over nearly verbatim to topological spaces.  We will try to point out when things do \emph{not} carry over identically (for example, limits no longer need be unique (\cref{exm4.1.20})).

One important thing to keep in mind is:  \emph{Open neighborhoods of a point in a general topological space play the same role that $\varepsilon$-balls did in $\R$}.  Not only do we use this to determine appropriate generalizations, but if you are feeling a bit uncomfortable with topological spaces, this should help your intuition.

For the entirety of this section, $X$ and $Y$ will be general topological spaces.  We recommend that this section be used mainly as a reference---the pedagogy of the concepts is contained in the last chapter.
\begin{dfn}[Net]
A \emph{net}\index{Net} in $X$ is a function from a nonempty directed set $(\Lambda ,\leq )$ into $X$.
\end{dfn}
\begin{dfn}[Sequence]
A \emph{sequence}\index{Sequence} is a net whose directed set is order-isomorphic (i.e.~isomorphic in $\Pre$) to $(\N ,\leq )$.
\end{dfn}
\begin{dfn}[Limit (of a net)]
Let $\lambda \mapsto x_\lambda$ be a net and let $x_\infty \in X$.  Then, $x_\infty$ is a \emph{limit}\index{Limit (of a net)} of $\lambda \mapsto x_\lambda$ iff for every open neighborhood $U$ of $x_\infty$ there is some $\lambda _0$ such that, whenever $\lambda \geq \lambda _0$, it follows that $x_\lambda \in U$.  If a net has a limit, then we say that it \emph{converges}\index{Convergence}.
\end{dfn}
\begin{exm}[Limits need not be unique]\label{exm4.1.20}
Define $X\coloneqq \{ 0,1\}$ and equip $X$ with the indiscrete topology (\cref{dfnIndiscreteTopology}).  Then, the constant net $\lambda \mapsto x_\lambda \coloneqq 0$ converges to both $0$ and $1$ (the only open neighborhood of both of these points is $X$ itself, and of course the net is eventually contained in $X$).
\end{exm}
\begin{dfn}[Subnet]\label{Subnet}
Let $x:\Lambda \rightarrow X$ be a be a net.  Then, a \emph{subnet}\index{Subnet} of $x$ is a net $y:\Lambda '\rightarrow X$ such that
\begin{enumerate}
\item for all $\mu \in \Lambda '$, $y_\mu =x_{\lambda _\mu}$ for some $\lambda _\mu \in \Lambda$; and
\item whenever $U\subseteq X$ eventually contains $x$, it eventually contains $y$.
\end{enumerate}
A \emph{strict subnet}\index{Strict subnet} of $x$ is a net of the form $\restr{x}{\Lambda'}$ for $\Lambda '\subseteq \Lambda$ cofinal.  A \emph{subsequence}\index{Subsequence} is a subnet that is a sequence.
\end{dfn}
\begin{thm}[Kelley's Convergence Axioms]\index{Kelley's Convergence Axioms}\label{KelleysConvergenceAxioms}
\begin{enumerate}
\item \label{enmKelleysConvergenceAxioms.i}Constant nets converge to that constant.
\item \label{enmKelleysConvergenceAxioms.ii}A net converges to $x_\infty \in X$ iff every subnet has in turn a subnet which converges to $x_\infty \in X$.
\item \label{enmKelleysConvergenceAxioms.iii}Let $I$ be a directed set and for each $i\in I$ let $x^i:\Lambda ^i\rightarrow X$ be a convergent net.  Then, if $(x^\infty )_\infty \coloneqq \lim _i\lim _\lambda (x^i)_\lambda$ exists, then $I\times \prod _{i\in I}\Lambda ^i\ni (i,\lambda )\mapsto (x^i)_{\lambda ^i}$ converges to $(x^\infty )_\infty$.
\end{enumerate}
\begin{rmk}
We will see later (\cref{KelleysConvergenceTheorem}) that these three properties can be used to define a topology (hence, the reason we refer to them as \emph{axioms}).
\end{rmk}
\end{thm}
\begin{dfn}[Limit (of a function)]\label{dfnLimitOfAFunction}
Let $f:X \rightarrow Y$ be a function between topological spaces, and let $x\in X$ and $y\in Y$.  Then, $y$ is the \emph{limit}\index{Limit (of a function)} of $f$ at $x$ iff for every net $\lambda \mapsto x_\lambda$ such that (i) $x_\lambda \neq x$ and (ii) $\lim _\lambda x_\lambda =x$ we have $\lim _\lambda f(x_\lambda )=y$.
\begin{rmk}
In the real numbers, this is true if you replace the word ``net'' with the word ``sequence'', but this fails in general topological spaces---see \cref{exm4.2.8}.
\end{rmk}
\end{dfn}
\begin{prp}
Let $f:X\rightarrow Y$ be a function and let $x_0\in X$.  Then, $f$ is continuous at $x_0$ iff $\lim _{x\to x_0}f(x)=f(x_0)$.  $f$ is continuous iff it is continuous at $x_0$ for all $x_0\in X$.
\end{prp}
\begin{prp}
Let $f:X\rightarrow Y$ be a function.  Then, $f$ is continuous iff the preimage of every closed set is closed.
\end{prp}
\begin{dfn}[Accumulation point]
Let $S\subseteq X$ and let $x\in X$.  Then, $x$ is an \emph{accumulation point}\index{Accumulation point} of $S$ iff every open neighborhood of $x_0$ intersects $S$ at a point distinct from $x_0$.
\begin{rmk}
If you remove the requirement that the point be distinct from $x_0$, we obtain what is usually called an \emph{adherent point}\index{Adherent point}.
\end{rmk}
\end{dfn}
\begin{dfn}[Limit point]
Let $S\subseteq X$ and let $x\in X$.  Then, $x$ is a \emph{limit point}\index{Limit point} of $S$ iff there exists a net $\lambda \mapsto x_\lambda \in S$ with $x_\lambda \neq x$ such that $\lim _\lambda x_\lambda =x$.
\end{dfn}
\begin{prp}
Let $S\subseteq X$ and let $x\in X$.  Then, $x$ is an accumulation point of $S$ iff it is a limit point of $S$.
\begin{rmk}
If you replace ``net'' with ``sequence'' in the definition of a limit point, then this result will be \emph{false} in general; see \cref{exm4.2.8x}
\end{rmk}
\end{prp}
\begin{prp}
Let $C\subseteq X$.  Then, $C$ is closed iff it contains all its accumulation points.
\end{prp}
\begin{crl}
Let $C\subseteq X$.  Then, $C$ is closed iff it contains all its limit points.
\end{crl}
We proved that (\cref{prp3.4.27}) in $\R$ that $x\in \R$ is an accumulation point of a sequence iff there was a subsequence that converged to $x$.  We also gave an example (\cref{exm3.4.29}) of how this fails (even in $\R$) for general nets.  Not only this, but this also fails to hold (for sequences) in general topological spaces.
\begin{exm}[An accumulation point of a sequence to which no subsequence converges]\label{exm4.2.15}
Define $X\coloneqq \{ x_1,x_2,x_3\}$ and
\begin{equation}
\mathcal{U}\coloneqq \left\{ \emptyset ,X,\{ x_1,x_2\} \right\} .
\end{equation}
Consider the sequence $m\mapsto a_m$ defined by
\begin{equation}
a_m\coloneqq \begin{cases}x_2 & \text{if }m=0 \\ x_3 & \text{otherwise}\end{cases}.
\end{equation}
Then, it is true that every open neighborhood of $x_1\in X$ contains an element of the set $\{ a_m:m\in \N \}$, namely $a_0\coloneqq x_2$ distinct from $x_1$.  On the other hand, no subsequence of $m\mapsto a_m$ converges to $x_1$ as it is eventually outside an open neighborhood of $x_1$ (namely the open neighborhood $\{ x_1,x_2\}$).
\end{exm}
\begin{dfn}[Interior point]
Let $S\subseteq X$ and let $x_0\in X$.  Then, $x_0$ is an \emph{interior point}\index{Interior point} of $S$ iff there is some open neighborhood $U$ of $x_0$ such that $U\subseteq S$.
\end{dfn}
\begin{prp}[Closure]\label{Closure}
Let $S\subseteq X$.  Then, there exists a unique set $\Cls (S)\subseteq X$\index[notation]{$\Cls (S)$}, the \emph{closure}\index{Closure} of $S$, that satisfies
\begin{enumerate}
\item $\Cls (S)$ is closed;
\item $S\subseteq \Cls (S)$; and
\item if $C$ is any other closed set which contains $S$, then $\Cls (S)\subseteq C$.
\end{enumerate}
\end{prp}
\begin{prp}
Let $S\subseteq X$.  Then, $\Cls (S)$ is the union of $S$ and its set of accumulation points.
\end{prp}
\begin{thm}[Kuratowski Closure Axioms]\index{Kuratowski Closure Axioms}
Let $S,T\subseteq X$.  Then,
\begin{enumerate}
\item $\Cls (\emptyset) =\emptyset$;
\item $S\subseteq \Cls (S)$;
\item $\Cls (S)=\Cls \left( \Cls (S)\right)$; and
\item $\Cls (S\cup T)=\Cls (S)\cup \Cls (T)$.
\end{enumerate}
\end{thm}
\begin{prp}[Interior]\label{Interior}
Let $S\subseteq X$.  Then, there exists a unique set $\Int (S)\subseteq \R$\index[notation]{$\Int (S)$}, the \emph{interior}\index{Interior} of $S$, that satisfies
\begin{enumerate}
\item $\Int (S)$ is open;
\item $\Int (S)\subseteq S$; and
\item if $U$ is any other open set which is contained in $S$, then $U\subseteq \Int (U)$.
\end{enumerate}
\end{prp}
\begin{prp}
Let $S\subseteq X$.  Then, $\Int (S)$ is the set of interior points of $S$.
\end{prp}
\begin{thm}[Kuratowski Interior Axioms]\index{Kuratowski Interior Axioms}
Let $S,T\subseteq X$.  Then,
\begin{enumerate}
\item $\Int (X)=X$;
\item $\Int (S)\subseteq S$;
\item $\Int (S)=\Int \left( \Int (S)\right)$; and
\item $\Int (S\cap T)=\Int (S)\cap \Int (T)$.
\end{enumerate}
\end{thm}
\begin{prp}\label{prp4.2.21}
Let $S\subseteq X$.  Then, $S$ is closed iff $S=\Cls (S)$.
\end{prp}
\begin{prp}
Let $S\subseteq X$.  Then, $S$ is open iff $S=\Int (S)$.
\end{prp}

We postponed the following counter-examples because we technically had not introduced the notion of closure in a general topological space.
\begin{exm}[A limit point that is not a sequential limit point]\label{exm4.2.8x}
Define $X\coloneqq \R$.  We equip $\R$ with a nonstandard topology, the so-called \emph{cocountable topology}\index{Cocountable topology}.\footnote{Just as cofinite means that the complement is finite, \emph{Cocountable}\index{Cocountable} means that the complement is countable.  The name of the topology here derives from the fact that the \emph{open} sets have countable complement (except the empty-set of course).}  Let $C\subseteq X$ and declare that
\begin{textequation}
$C$ is closed iff either (i) $C=X$ or (ii) $C$ is countable.
\end{textequation}
Because the finite union of countable sets is countable and an arbitrary intersection of countable sets is countable (obviously), it follows that this defines a topology on $X$.

We first show that every convergent sequence is eventually constant.  So, let $m\mapsto x_m\in X$ be sequence that converges to $x_\infty \in X$.  We proceed by contradiction:  suppose that it is not eventually constant.  Then, the set
\begin{equation}
\{ m\in \N :x_m\neq x_\infty \}
\end{equation}
is cofinal, and hence defines a subsequence $n\mapsto x_{m_n}$ that (i) converges to $x_\infty$ but (ii) is never equal to $x_\infty$.  Hence, the set $C\coloneq \{ x_{m_n}:n\in \N \}$ does not contain $x_\infty$, and so $C^{\comp}$ is an open neighborhood of $x_\infty$.  But of course $n\mapsto x_{m_n}$ is not eventually contained in $C^{\comp}$---no term of this subsequence is contained in $C^{\comp}$.  Therefore, $n\mapsto x_{m_n}$ cannot converge to $x_\infty$:  a contradiction.  Therefore, $m\mapsto x_m$ must be eventually constant.

Define $U\coloneqq \Q ^{\comp}$.  As $U^{\comp}=\Q$ is countable, $U$ is open.  We note that the closure of $U$ is all of $\R$:  no countable set can contain $U$, and so the only closed set which contains $U$ is $\R$ itself.

It thus follows that, in particular, $0$ is a limit point of $U$.  On the other hand, as every convergent sequence is eventually constant, no sequence in $U\coloneqq \Q ^{\comp}$ can converge to $0$.
\end{exm}
\begin{exm}[Two distinct topologies with the same notion of sequential convergence]\label{exm4.2.25}
Let $X$ be as in \cref{exm4.2.8x} and denote the topology on $X$ given there (the cocountable topology) by $\mathcal{U}$.  Denote by $\mathcal{D}$ the discrete topology on $X$.  We saw in \cref{exm4.2.8x} that sequences converge iff they are eventually constant.  However, we also have the following result.
\begin{exr}
Let $X$ be a discrete space and let $\lambda \mapsto x_\lambda \in X$ be net.  Show that $\lambda \mapsto x_\lambda$ iff it is eventually constant.
\end{exr}
Thus, a given sequence $m\mapsto x_m\in X$ converges with respect to $\mathcal{U}$ iff it converges with respect to $\mathcal{D}$, and in this case, they converge to the same limit.  On the other hand, the set $\{ 0\}$ is \emph{not} open with respect to $\mathcal{U}$\footnote{This uses the fact that the real numbers are uncountable!} but it is open with respect to $\mathcal{D}$.

Even though the topologies are distinct, perhaps it is the case that they are \emph{homeomorphic}?  We show that this cannot happen.  In fact, we show that no bijective function from $\phi :(X,\mathcal{U})\rightarrow (X,\mathcal{D})$ is continuous.  If $\phi$ were such a function, then $\phi ^{-1}(\Q ^{\comp})$ would be uncountable and proper, and hence would not be closed, a contradiction of the fact that $\phi$ is continuous and $\Q ^{\comp}$ is closed with respect to $\mathcal{D}$.
\end{exm}
\begin{exm}[A sequentially-continuous function that is not continuous]\label{exm4.2.8}
Let $X$ be as in \cref{exm4.2.8x} ($\R$ with the cocountable topology) and define $f:X\rightarrow X$ by
\begin{equation}
f(x)\coloneqq \begin{cases}1 & \text{if }x\in \Q \\ -1 & \text{if }x\in \Q ^{\comp}\end{cases}.
\end{equation}
(Note that this is just the Dirichlet Function (\cref{DirichletFunction})!).

This function is certainly not continuous because, for example, the preimage of $\{ -1\}$ is not closed.  On the other hand, because every convergent sequence is eventually constant, the condition that $x_\lambda \neq x$ in the definition of a limit (\cref{dfnLimitOfAFunction}), $f$ vacuously satisfies the continuity condition for sequences.
\end{exm}
\begin{dfn}[Cover]
Let $S\subseteq X$ and let $\mathcal{U}\subseteq 2^{X}$.  Then, $\mathcal{U}$ is a \emph{cover}\index{Cover} of $S$ iff $S\subseteq \bigcup _{U\in \mathcal{U}}U$.  $\mathcal{U}$ is an \emph{open cover}\index{Open cover} iff every $U\in \mathcal{U}$ is open.  A \emph{subcover} of $\mathcal{U}$ is a subset $\mathcal{V}\subseteq \mathcal{U}$ that is still a cover of $S$.
\end{dfn}
\begin{dfn}[Quasicompact]\label{Quasicompact}
Let $S\subseteq X$.  Then, $S$ is \emph{quasicompact}\index{Quasicompact} iff every open cover of $S$ has a finite subcover.
\end{dfn}
\begin{exr}\label{exr4.2.33x}
Show that finite spaces are quaiscompact.
\end{exr}
\begin{exr}\label{exr4.2.33}
Show that closed subsets of quasicompact spaces are quasicompact.
\end{exr}
\begin{prp}\label{prp4.2.32}
Let $K\subseteq X$ and let $\mathcal{C}$ be a collection of closed subsets of $X$.  Then, $K$ is quasicompact iff whenever every finite intersection of elements of $\mathcal{C}$ intersects $K$, the entire intersection $\bigcap _{C\in \mathcal{C}}C$ also intersects $K$.
\end{prp}
\begin{prp}\label{prp4.2.31}
Let $K\subseteq X$.  Then, $K$ is quasicompact iff every net $\lambda \mapsto a_\lambda \in K$ has a subnet that converges to a limit in $K$.
\end{prp}
Note that the Heine-Borel and Bolzano-Weierstrass Theorems (\cref{HeineBorelTheorem,BolzanoWeierstrassTheorem}) do not hold in general, but we will wait until after having studied integration before writing down counter-examples.

\subsection{A couple new things}

We present in this subsection a couple of facts that, while not technically review per se, make more sense to place here once we begin study of new topological concepts in earnest.

\begin{dfn}[Dense]\label{Dense}
Let $X$ be a topological space and let $S\subseteq X$.  Then, $S$ is \emph{dense}\index{Dense} in $X$ iff $\Cls (S)=X$.
\end{dfn}
\begin{exr}\label{exr4.2.38}
Show that $\Q$ and $\Q ^{\comp}$ are both dense in $\R$.
\begin{rmk}
We mentioned way back when we discussed `density' of $\Q$ and $\Q ^{\comp}$ in $\R$ (\cref{thm3.2.14,thm3.3.76}) that these results aren't literally the statements that $\Q$ and $\Q ^{\comp}$ are dense in $\R$.  When we say that $\Q$ is dense in $\R$, what we mean of course is that $\Cls (\Q )=\R$ (and similarly for $\Q ^{\comp}$).  People `abuse' language and refer to the properties of \cref{thm3.2.14,thm3.3.76} as ``density'' because density in this sense (that is, in the sense of the definition above) follows as an easy corollary of these theorems.
\end{rmk}
\end{exr}

The next couple of facts have to do with generating collections and their relations to concepts reviewed in the previous subsection.
\begin{exr}\label{exr4.2.41}
Let $X$ be a topological space, let $\mathcal{S}$ generate the topology of $X$, let $\lambda \mapsto x_\lambda \in S$ be a net, and let $x_\infty \in X$.  Show that $\lambda \mapsto x_\lambda$ converges to $x_\infty$ iff $\lambda \mapsto x_\lambda$ is eventually contained in every element of $\mathcal{S}$ which contains $x_\infty$.
\begin{rmk}
In other words, for the purposes of convergence, it suffices to look only at a generating collection of the topology.
\end{rmk}
\end{exr}

There is another characterization of quasicompactness that we did not introduce in the previous chapter because it involves the concept of generating collections, something that we hadn't defined in that context.
\begin{thm}[Alexander Subbase Theorem]\label{AlexanderSubbaseTheorem}
Let $X$ be a topological space and let $\mathcal{S}$ be a generating collection for the topology on $X$.  Then, $X$ is quasicompact iff every cover by elements of $\mathcal{S}$ has a finite subcover.
\begin{rmk}
This is of course just the defining property of quasicompactness, the only change being that we need only check covers whose elements come from the generating collection $\mathcal{S}$.
\end{rmk}
\begin{rmk}
In particular, if $\mathcal{S}$ does not cover $X$, then $X$ is quasicompact.
\end{rmk}
\begin{rmk}
The term \emph{subbase}\index{Subbase} is sometimes used for generating collections which cover the space.  In fact, people usually make the requirement that the generating collection cover the space, but there is no need.
\end{rmk}
\begin{proof}\footnote{Proof adapted from \cite[pg.~139]{Kelley}.}
$(\Rightarrow )$ Of course, if $X$ is quasicompact, then \emph{every} open cover has a finite subcover, and so certainly open covers that come from $\mathcal{S}$ will have finite subcovers.

\blankline
\noindent
$(\Leftarrow )$ 
\Step{Make hypotheses}
Suppose that every cover by elements of $\mathcal{S}$ has a finite subcover.  To show that $X$ is quasicompact, we prove the contrapositive of the defining condition of quasicompactness.  That is, we show that every collection of open sets that has the property that no finite subset covers $X$, also does not cover $X$.  So, let $\mathcal{U}$ be a collection of open sets that has the property that no finite subset covers $X$.  We show that $\mathcal{U}$ itself does not cover $X$.

\Step{Enlarge $\mathcal{U}$ to a maximal collection}
Let $\tilde{\mathcal{U}}$ be the collection of all collections of open sets which (i) contain $\mathcal{U}$ (ii) also have the property that no finite subset covers $X$.  This is a set that is partially-ordered by inclusion, and so we intend to apply Zorn's Lemma (\cref{ZornsLemma}) to extract a maximal such element.  So, let $\tilde{\mathcal{W}}$ be a well-ordered subset of $\tilde{\mathcal{U}}$ and define
\begin{equation}
\mathcal{W}_0\coloneqq \bigcup _{\mathcal{W}\in \tilde{\mathcal{W}}}\mathcal{W}.
\end{equation}
Certainly $\mathcal{W}_0$ is a collection of open sets, a collection which contains $\mathcal{U}$.  In order to be an upper-bound for $\tilde{\mathcal{W}}$, however, we need to check that no finite subset covers $X$.  So, let $W_1,\ldots ,W_m\in \mathcal{W}_0$.  Then, each $W_k\in \mathcal{W}_k$ for some $\mathcal{W}_k\in \tilde{\mathcal{W}}$.  Because $\tilde{\mathcal{W}}$ is in particular totally-ordered, one of $\mathcal{W}_1,\ldots ,\mathcal{W}_m$ must contain all the others, and in particular, all the $W_k$s are contained in a single $\mathcal{W}_k$.  It follows that $\{ W_1,\ldots ,W_m\}$ cannot cover $X$.

Therefore, by Zorn's Lemma, there is a maximal collection of open sets $\mathcal{U}_0$ that (i) contains $\mathcal{U}$ and (ii) has the property that no finite subset covers $X$.  To show that $\mathcal{U}$ does not
cover $X$, it suffices to show that $\mathcal{U}_0$ does not cover $X$.

\Step{Show that if an element of $\mathcal{U}_0$ contains an intersection of open sets, it must contain one of those sets}
Let $U\in \mathcal{U}_0$ and suppose that $U_1\cap \cdots \cap U_m\subseteq U$ for $U_1,\ldots ,U_m$ open.  We proceed by contradiction:  suppose that $U_k\notin \mathcal{U}_0$ for all $k$.  This means that, by maximality, for each $U_k$, there are finitely many $U_k^1,\ldots ,U_k^{n_k}$ such that $X=U_k\cup U_k^1\cup \cdots \cup U_k^{n_k}$.  But then
\begin{equation}
U\cup \bigcup _{k=1}^m\bigcup _{l=1}^{n_k}U_k^l=\left( \bigcap _{k=1}^mU_k\right) \cup \left( \bigcup _{k=1}^m\bigcup _{l=1}^{n_k}U_k^l\right) \supseteq \bigcap _{k=1}^m\left( U_k\cup U_k^1\cup \cdots \cup U_k^{n_k}\right) =X,
\end{equation}
so that
\begin{equation}
U\cup \bigcup _{k=1}^m\bigcup _{l=1}^{n_k}U_k^l,
\end{equation}
a contradiction.  Therefore, some $U_k\in X$.

\Step{Deduce that $\mathcal{U}_0$ does not cover $X$}
Define
\begin{equation}
\mathcal{V}_0\coloneqq \mathcal{U}_0\cap \mathcal{S}.
\end{equation}
First of all, as no finite subset of $\mathcal{U}_0$ covers $X$, in particular, $X\notin \mathcal{U}_0$, so that
\begin{equation}
\mathcal{V}_0=\mathcal{U}_0\cap (\mathcal{S}\cup \{ X\} ).
\end{equation}
As every element of $\mathcal{V}_0$ comes from $\mathcal{U}_0$, it follows that no finite subset of $\mathcal{V}_0$ covers $X$.  On the other hand, every element of $\mathcal{V}_0$ comes from $\mathcal{S}$, so that, by hypothesis, it in turn follows that $\mathcal{V}_0$ does not cover $X$.  Thus, we will be done if we can show that
\begin{equation}
\bigcup _{V\in \mathcal{V}_0}V=\bigcup _{U\in \mathcal{U}_0}U.
\end{equation}
The $\subseteq$ inclusion is obvious because $\mathcal{V}_0\subseteq \mathcal{U}_0$.  For the other inclusion, let $x\in \bigcup _{U\in \mathcal{U}_0}U$.  Then, $x\in U$ for some $U\in \mathcal{U}_0$.  Because $\mathcal{S}$ is a generating collection, the collection of all finite intersections of elements of $\mathcal{S}\cup \{ X\}$ is a base (see \cref{GeneratingCollection}), and so there are $U_1,\ldots ,U_m\in \mathcal{S}$ such that $x\in U_1\cap \cdots \cap U_m\subseteq U$.  But then, by the previous step, $U_k\in \mathcal{U}_0$ for some $U_k$.  Of course, $U_k$ came from $\mathcal{S}\cup \{ X\}$, and so in fact $U_k\in \mathcal{U}_0\cap (\mathcal{S}\cup \{ X\} )=\mathcal{V}_0$, so that indeed $x\in \bigcup _{V\in \mathcal{V}_0}V$.
\end{proof}
\end{thm}

\section{Filter bases}\label{sct4.4}

This section is a bit of an aside and can probably be skipped without too much trouble.  Our motivation for covering filter bases is (i) it will help us demonstrate by our definition of subnet is the `correct' one, as opposed to, for example, the one currently given\footnote{5 July 2015} on Wikipedia, and (ii) it is something that is important enough that you should probably at least be aware of and will almost certainly eventually encounter if you decide to become a mathematician.

Filter bases are actually an alternative to nets.  In principle, one could do the entirety of topology never speaking of nets and instead using only filter bases.  This would be one motivation for introducing them (though we have decided to primarily stick to nets).
\begin{dfn}[Filter base]\label{FilterBase}
Let $(X,\leq )$ be a partially-ordered set and let $F\subset X$ be nonempty and not containing the empty-set.  Then, $F$ is a \emph{filter base}\index{Filter base} of $X$ iff for $x_1,x_2\in F$, there is some $x_3\in F$ such that $x_3\leq x_1,x_2$.\footnote{This property is called being \emph{downward-directed}.}
\begin{rmk}
This is the definition of an abstract filter base in any partially-ordered set.  For us, we will essentially only be interested in filter bases of the partially-ordered set $(2^X,\subseteq )$ for $X$ a topological space:  if $F$ is a filter base of $(2^X,\subseteq )$, $X$ a topological space, then we shall say that $F$ is a \emph{filter base in $X$}.  (Note that the elements of filter base do not have to be open sets.)
\end{rmk}
\begin{rmk}
This condition is exactly analogous to the condition for directed sets $\Lambda$, that for $x_1,x_2\in \Lambda$, there is some $x_3\in \Lambda$ with $x_3\geq x_1,x_2$.  In fact, you might even say that the definition is what it is because \emph{a filter base ordered by reverse-inclusion is a directed set}.
\end{rmk}
\end{dfn}
The relation between nets and filter bases is given by the following.
\begin{dfn}[Derived filter base]\label{DerivedFilterBase}
Let $X$ be a topological space, let $\lambda \mapsto x_\lambda \in X$ be a net, and define
\begin{equation}
\mathcal{F}_{\lambda \mapsto x_\lambda}\coloneqq \left\{ F\subseteq X:F\text{ eventually contains }\lambda \mapsto x_\lambda \right\} .
\end{equation}\index[notation]{$\mathcal{F}_{\lambda \mapsto x_\lambda}$}
\begin{exr}
Show that $\mathcal{F}$ is a filter base.
\end{exr}
$\mathcal{F}_{\lambda \mapsto x_\lambda}$ is the \emph{derived filter}\index{Derived filter} of $\lambda \mapsto x_\lambda$.
\begin{rmk}
Given a net, we just defined a canonically associated filter.  Of course, there will be many nets which give us this filter.  For example, I can change a single term of a net without affecting its derived filter.\footnote{In general at least.  In stupid cases, of course, e.g.~if the domain of the net is a single point, then this will change the derived filter.}
\end{rmk}
\begin{rmk}
Because of this, one might argue that filter bases are more fundamental than nets.  Nets somehow contain extra information that is completely irrelevant to topology.  An example of this is how one can always throw away finitely many terms of a sequence without affecting anything of importance.  Because the derived filter base of a net only contains information about what \emph{eventually} happens with the net, this extraneous information is lost when passing from the net to its derived filter.
\end{rmk}
\begin{rmk}
I personally find nets much more intuitive than filter bases (probably because they are a much more straightforward generalization of sequences than filter bases are), and thinking of how filter bases come from nets helps me understand some of the intuition of filter bases themselves
\end{rmk}
\end{dfn}

At the bare minimum, in order for filter bases and nets to be effectively equivalent for the purposes of topology, we must at least (i) define convergence of filter bases and (ii) show that convergence of a net agrees with convergence of its derived filter.
\begin{dfn}[Limit (of a filter base)]
Let $X$ be a topological space, let $\mathcal{F}$ be a filter base in $X$, and let $x_\infty \in X$.  Then, $x_\infty$ is a \emph{limit}\index{Limit (of a filter)} of $\mathcal{F}$ iff for every open neighborhood $U$ of $x_\infty$ there is some $F\in \mathcal{F}$ such that $F\subseteq U$.  If a filter base has a limit, then we say that it \emph{converges}\index{Convergence}.
\end{dfn}
\begin{prp}\label{prp4.3.6}
Let $X$ be a topological space, let $\lambda \mapsto x_\lambda \in X$ be a net, and let $x_\infty \in X$.  Then, $\lambda \mapsto x_\lambda$ converges to $x_\infty$ iff $\mathcal{F}_{\lambda \mapsto x_\lambda}$ converges to $x_\infty$.
\begin{proof}
$(\Rightarrow )$ Suppose that $\lambda \mapsto x_\lambda$ converges to $x_\infty$.  Let $U$ be an open neighborhood of $x_\infty$.  Then, $\lambda \mapsto x_\lambda$ is eventually contained in $U$.  Therefore, $U\in \mathcal{F}_{\lambda \mapsto x_\lambda}$, and so of course $\mathcal{F}_{\lambda \mapsto x_\lambda}$ converges to $x_\infty$.

\blankline
\noindent
$(\Leftarrow )$ Suppose that $\mathcal{F}_{\lambda \mapsto x_\lambda}$ converges to $x_\infty$.  Let $U$ be an open neighborhood of $x_\infty$.  Then, there is some $F\in \mathcal{F}_{\lambda \mapsto x_\lambda}$ such that $F\subseteq U$.  By definition of derived filter bases, it follows that $\lambda \mapsto x_\lambda$ is eventually contained in $F$, and hence eventually contained in $U$.  Therefore, $\lambda \mapsto x_\lambda$ converges to $x_\infty$.
\end{proof}
\end{prp}

Now we turn to filterings and subnets.  A filtering is to a filter base as a subnet is to a net.  Recall that one of our motivations for talking about filter bases at all was to argue that our definition of subnet was the `correct' one.  One nice thing about filter bases is that there is no confusion about what the definition of a filtering should be.\footnote{Though evidently there is some confusion as to what they should be called---see the remark in the definition below.}  We will then show that our definition of subnet coincides with the notion of a filtering.
\begin{dfn}[Filtering]\label{Filtering}
Let $\mathcal{F}$ be a filter base on $X$.  Then, a \emph{filtering}\index{Filtering} of $\mathcal{F}$ is a filter base $\mathcal{G}$ that has the property that, for every $F\in \mathcal{F}$, there is some $G\in \mathcal{G}$ such that $G\subseteq F$.
\begin{rmk}
Note that in some place\footnote{\emph{*cough*}---Wikipedia---\emph{*cough*}} this is called a \emph{refinement}.  This is poor terminology because it disagrees with the usual definition of refinements of covers---see \cref{dfnC.1}.  For comparison, we reproduce that definition here.
\begin{textequation}
$\mathcal{G}$ is a \emph{refinement} of $\mathcal{F}$ iff for every $G\in \mathcal{G}$ there is some $F\in \mathcal{F}$ such that $G\subseteq F$.
\end{textequation}
\end{rmk}
\end{dfn}
\begin{exr}\label{exr4.4.8}
Let $\mathcal{F}$ and $\mathcal{G}$ be filter bases.  Show that if $\mathcal{F}\subseteq \mathcal{G}$, then $\mathcal{G}$ is a filtering of $\mathcal{F}$.
\end{exr}
In fact, for derived filter bases, every filtering is of this form.
\begin{prp}\label{prp4.4.8}
Let $X$ be a topological space, let $\Lambda \ni \lambda \mapsto x_\lambda \in X$ be a net, and let $\Lambda '\ni \mu \mapsto \lambda _\mu \in \Lambda$ be a function such that $\mu \mapsto x_{\lambda _\mu}$ is a net.  Then the following are equivalent.
\begin{enumerate}
\item \label{enm4.4.8.i}$\mu \mapsto x_{\lambda _\mu}$ is a subnet of $\lambda \mapsto x_\lambda$.
\item \label{enm4.4.8.ii}$\mathcal{F}_{\lambda \mapsto x_\lambda}\subseteq \mathcal{F}_{\mu \mapsto x_{\lambda _\mu}}$.
\item \label{enm4.4.8.iii}$\mathcal{F}_{\mu \mapsto x_{\lambda _\mu}}$ is a filtering of $\mathcal{F}_{\lambda \mapsto x_\lambda}$.
\end{enumerate}
\begin{rmk}
In particular, as the definitions of subnet given in \cite{Kelley} (\cref{prp3.3.92}) and on Wikipedia\footnote{As of 16 July 2015} (\cref{prp3.3.93}) are \emph{not} equivalent (\cref{exm3.3.93,exr3.3.94}) to our definition of subnet (\cref{Subnet}), they are in turn not equivalent to the filtering of filter bases!
\end{rmk}
\begin{rmk}
Of course, in general there are filterings not of this form, but for \emph{derived} filter bases, filtering is equivalent to containing (as sets).
\end{rmk}
\begin{proof}
$(\ref{enm4.4.8.i}\Rightarrow \ref{enm4.4.8.ii})$ Suppose that $\mu \mapsto x_{\lambda _\mu}$ is a subnet of $\lambda \mapsto x_\lambda$.  Let $F\in \mathcal{F}_{\lambda \mapsto x_\lambda}$.  Then, $\lambda \mapsto x_\lambda$ is eventually in $F$ (see the definition of subnet, \cref{Subnet}), and so $\mu \mapsto x_{\lambda _\mu}$ is eventually in $F$, and so $F\in \mathcal{F}_{\mu \mapsto x_{\lambda _\mu}}$.

$(\ref{enm4.4.8.ii}\Rightarrow \ref{enm4.4.8.iii})$ \cref{exr4.4.8}

$(\ref{enm4.4.8.iii}\Rightarrow \ref{enm4.4.8.i})$
Suppose that $\mathcal{F}_{\mu \mapsto x_{\lambda _\mu}}$ is a filtering of $\mathcal{F}_{\lambda \mapsto x_\lambda}$.  Let $F\subseteq X$ be such that $F$ eventually contains $\lambda \mapsto x_\lambda$.  Then, $F\in \mathcal{F}_{\lambda \mapsto x_\lambda}$, and so there is some $F'\in \mathcal{F}_{\mu \mapsto x_{\lambda _\mu}}$ such that $F'\subseteq F$.  As $\mu \mapsto x_{\lambda _\mu}$ is eventually contained in $F'$, it is eventually contained in $F$, and so $\mu \mapsto x_{\lambda _\mu}$ is a subnet of $\lambda \mapsto x_\lambda$ (once again, by the definition of subnets, \cref{Subnet}).
\end{proof}
\end{prp}

In the next section, we present several new ways of defining topologies.  One of these ways will be by defining what it means for filters to converge, and so present here as theorems the results that will be used as axioms.
\begin{prp}[Kelley's Filter Convergence Axioms]\index{Kelley's Filter Convergence Axioms}\label{KelleysFilterConvergenceAxioms}
Let $X$ be a topological space.  Then,
\begin{enumerate}
\item \label{enmKelleysFilterConvergenceAxioms.i}$\mathcal{P}_x$ converges to $x$, where $\mathcal{P}_x\coloneqq \left\{ U\subseteq X:x\in U\right\}$;\footnote{``P'' is for \emph{principal}, the etymology being from the use of the word ``principal'' in the context of ideals in ring theory (to the best of my knowledge anyways).}
\item \label{enmKelleysFilterConvergenceAxioms.ii}$\mathcal{F}$ converges to $x$ iff for every filtering $\mathcal{G}\supseteq \mathcal{F}$, there is some filtering $\mathcal{H}\supseteq \mathcal{G}$ such that $\mathcal{H}$ converges to $x$; and
\item \label{enmKelleysFilterConvergenceAxioms.iii}for all directed sets $I$ and filters $\mathcal{F}^i$ converging to $x^i\in X$, for $i\in I$, if $\mathcal{F}_{i\mapsto x^i}$ converges to $x^\infty$, then
\begin{equation}
\begin{split}
\mathcal{F}_\infty & \coloneqq \left\{ U\subseteq X:\text{there exists }i_U\in I\text{ such that,}\right. \\
& \qquad \left. \text{whenever }i\geq i_U\text{, }U\supseteq F^i\text{ for some }F^i\in \mathcal{F}^i.\right\}
\end{split}
\end{equation}
also converges to $x^\infty$.
\end{enumerate}
\begin{rmk}
In other words, $\mathcal{F}_{\infty}$ consists of those sets that eventually contain some element of $\mathcal{F}^i$.
\end{rmk}
\begin{rmk}
Note that these are completely analogous to Kelley's (Net) Convergence Axioms (\cref{KelleysConvergenceAxioms}).
\end{rmk}
\begin{rmk}
Perhaps this could be made into an argument that somehow nets are more fundamental---to define a topology by convergence of filters, you have to make use of nets (in this case, $i\mapsto x^i$).\footnote{Or rather, I am not aware of a way to state the axioms without at least implicitly using nets.}
\end{rmk}
\begin{proof}
We ask you to prove \ref{enmKelleysFilterConvergenceAxioms.i} and \ref{enmKelleysFilterConvergenceAxioms.ii}.
\begin{exr}
Show \ref{enmKelleysFilterConvergenceAxioms.i}.
\end{exr}
\begin{exr}
Show \ref{enmKelleysFilterConvergenceAxioms.ii}.
\begin{rmk}
Hint:  Check out \cref{prp3.3.95}.
\end{rmk}
\end{exr}
Let $I$ be a directed set, for each $i\in I$ let $\mathcal{F}^i$ be a filter converging to $x^i\in X$, and suppose that $\mathcal{F}_{i\mapsto x^i}$ converges to $x^\infty$.  By \cref{prp4.3.6}, $i\mapsto x^i$ converges to $x^\infty$.  To show that $\mathcal{F}_\infty$ converges to $x^\infty$, it suffices to show that every open neighborhood of $x^\infty$ is an element of $\mathcal{F}_\infty$.  So, let $U$ be an open neighborhood of $x^\infty$.  Then, as $\mathcal{F}_{i\mapsto x^i}$ converges to $x^\infty$, it follows that there is some $V\in \mathcal{F}_{i\mapsto x^i}$ such that $V\subseteq U$.  As $i\mapsto x^i$ is eventually contained in $V$, there is some $i_0\in I$ such that, whenever $i\geq i_0$, it follows that $x^i\in V\subseteq U$.  Thus, for $i$ sufficiently large, $U$ is an open neighborhood of $x^i$.  But then, because $\mathcal{F}^i$ converges to $x^i$, it follows that there is some $F^i\in \mathcal{F}^i$ such that $F^i\subseteq U$, and so $U\in \mathcal{F}_{\infty}$ as desired.
\end{proof}
\end{prp}

\section{Equivalent definitions of topological spaces}

There is more than one way to define a topology.  By definition, the specification of a topology is just the specification of what sets are open.  Sometimes, however, what the open sets should be is not nearly as obvious as, for example, what convergence of nets should mean.  In this section, we present a couple of other ways you may define a topological space.  Which one is most useful, of course, will depend on the particular problem at hand.

To summarize, we have already shown that we may define a topology in the following ways.  We can define a topology
\begin{enumerate}
\item by specifying the open sets (\cref{TopologicalSpace});
\item by specifying the closed sets (\cref{exr4.1.2});
\item by specifying a base for the topology (\cref{prp4.1.5});
\item by specifying a neighborhood base for the topology (\cref{prp4.1.8}); or
\item by specifying a generating collection for the topology (\cref{GeneratingCollection}).
\end{enumerate}
Of course, there are many other ways to specify a topology as well, and it is the purpose of this section to list several others ways.

\subsection{Definition by specification of closures or interiors}

We have mentioned the Kuratowski Closure (Interior) Axioms as well as Kelley's (Filter) Convergence Axioms.  These play a role analogous to \ref{enmTopologicalSpace.i}--\ref{enmTopologicalSpace.iii} in the definition of a topological space, \cref{TopologicalSpace}.  For example, we have the following.
\begin{thm}[Kuratowski's Closure Theorem]\label{KuratowskisClosureTheorem}\index{Kuratowski's Closure Theorem}
Let $X$ be a set and let $\mathrm{C}:2^X\rightarrow 2^X$ be a function on the power-set of $X$.  Then, if
\begin{enumerate}
\item \label{enmKuratowskiClosureTheorem.i}$\mathrm{C}(\emptyset )=\emptyset$;
\item \label{enmKuratowskiClosureTheorem.ii}$S\subseteq \mathrm{C}(S)$;
\item \label{enmKuratowskiClosureTheorem.iii}$\mathrm{C}(S)=\mathrm{C}\left( \mathrm{C}(S)\right)$; and
\item \label{enmKuratowskiClosureTheorem.iv}$\mathrm{C}(S\cup T)=\mathrm{C}(S)\cup \mathrm{C}(T)$,
\end{enumerate}
then there exists a unique topology on $X$ such that $\Cls (S)=\mathrm{C}(S)$.
\begin{proof}
\Step{Make hypotheses}
Suppose that (i) $\mathrm{C}(\emptyset )=\emptyset$, (ii) $S\subseteq \mathrm{C}(S)$, (iii) $\mathrm{C}(S)=\mathrm{C}\left( \mathrm{C}(S)\right)$, and (iv) $\mathrm{C}(S\cup T)=\mathrm{C}(S)\cup \mathrm{C}(T)$.

\Step{Show that $S\subseteq T$ implies $\mathrm{C}(S)\subseteq \mathrm{C}(T)$}\label{stpKuratowskiClosureTheorem.2}
Suppose that $S\subseteq T$, so that $T=S\cup (T\setminus S)$, and hence $\mathrm{C}(T)=\mathrm{C}(S)\cup \mathrm{C}(T\setminus S)$, and  so $\mathrm{C}(S)\subseteq \mathrm{C}(T)$.

\Step{Define what should be the closed sets}
The idea of the proof is that the closed sets should be precisely the sets that are equal to their closure (\cref{prp4.2.21}).  We thus make the definition
\begin{equation}
\mathcal{C}\coloneqq \left\{ C\in 2^X:C=\mathrm{C}(C)\right\} .
\end{equation}

\Step{Verify that this defines a topology}
By \ref{enmKuratowskiClosureTheorem.i}, the empty-set is closed (by which we mean it is an element of $\mathcal{C}$).  By \ref{enmKuratowskiClosureTheorem.ii}, we have
\begin{equation}
X\subseteq \mathrm{C}(X)\subseteq X,
\end{equation}
so that $X$ is closed as well.  Let $\mathcal{D}\subseteq \mathcal{C}$ and define
\begin{equation}
B\coloneqq \bigcap _{C\in \mathcal{D}}C
\end{equation}
Then, of course, $B\subseteq C$ for all $C\in \mathcal{D}$, and so by \cref{stpKuratowskiClosureTheorem.2}, we have that $\mathrm{C}(B)\subseteq \mathrm{C}(C)$ for all $C\in \mathcal{D}$, and so
\begin{equation}
\mathrm{C}(B)\subseteq \bigcap _{C\in \mathcal{D}}\mathrm{C}(C)=\bigcap _{C\in \mathcal{D}}C\eqqcolon B.
\end{equation}
As $B\subseteq \mathrm{C}(B)$ by \ref{enmKuratowskiClosureTheorem.ii}, we have that $B=\mathrm{C}(B)$, so that $\mathcal{C}$ is closed under arbitrary intersections.  We now check that it is closed under finite unions, so let $C,D\in \mathcal{C}$.  Then,
\begin{equation}
\mathrm{C}(C\cup D)=\mathrm{C}(C)\cup \mathrm{C}(D)=C\cup  D,
\end{equation}
and so $C\cup D\in \mathcal{C}$.  Thus, $\mathcal{C}$ is closed under finite unions, and hence defines a topology.

\Step{Show that $\Cls (S)=\mathrm{C}(S)$}
Let $S\subseteq  X$.  By \ref{enmKuratowskiClosureTheorem.iii}, $\mathrm{C}(S)$ is closed, and by \ref{enmKuratowskiClosureTheorem.ii}, it contains $S$.  Therefore, $\Cls (S)\subseteq \mathrm{C}(S)$.  It follows that $\mathrm{C}\left( \Cls (S)\right) \subseteq \mathrm{C}(S)$.  On the other hand, because $S\subseteq \Cls (S)$, it follows that $\mathrm{C}(S)\subseteq \mathrm{C}\left( \Cls (S)\right) =\Cls (S)$, and so indeed $\Cls (S)=\mathrm{C}(S)$.
\end{proof}

\Step{Demonstrate uniqueness}
If another topology $\mathcal{V}$ satisfies this property, that is, $\Cls _{\mathcal{V}}(S)=C(S)$, then we have that $\Cls _{\mathcal{V}}(S)=\Cls (S)$ (no subscript indicates the closure in the topology defi8ned by $\mathcal{C}$).  It follows that a set $S$ is closed with respect to $\mathcal{V}$ iff it is equal to $\Cls _{\mathcal{V}}(S)$ iff it is equal to $\Cls (S)$ iff it is closed in the topology defined by $\mathcal{C}$.  Thus, the two topologies have the same closed sets, and hence are the same.
\end{thm}
Similarly, we have a `dual' interior theorem.
\begin{thm}[Kuratowski's Interior Theorem]\label{KuratowskisInteriorTheorem}\index{Kuratowski's Interior Theorem}
Let $X$ be a set and let $\mathrm{I}:2^X\rightarrow 2^X$ be a function on the power-set of $X$.  Then, if
\begin{enumerate}
\item $\mathrm{I}(X)=X$;
\item $\mathrm{I}(S)\subseteq S$;
\item $\mathrm{I}(S)=\mathrm{I}\left( \mathrm{I}(S)\right)$; and
\item $\mathrm{I}(S\cap T)=\mathrm{I}(S)\cap \mathrm{I}(T)$,
\end{enumerate}
then there exists a unique topology on $X$ such that $\Int (S)=\mathrm{I}(S)$.
\begin{rmk}
We omit the proof as it is completely `dual' to the corresponding closure proof.
\end{rmk}
\end{thm}

\subsection{Definition by specification of convergence}

The previous two results showed that we can define a topology by defining what the closure or interior of every set should be.  The next result says that we can define a topology by defining what it means for nets to converge.
\begin{thm}[Kelley's Convergence Theorem]\label{KelleysConvergenceTheorem}\index{Kelley's Convergence Theorem}
Let $X$ be a set, denote by $\mathcal{N}$ the collection of all nets in $X$, and let $\to$ be a relation on $\mathcal{N}\times X$.  Then, if
\begin{enumerate}
\item \label{enmKelleysConvergenceTheorem.i}$(\lambda \mapsto x_\infty)\to x_\infty$;
\item \label{enmKelleysConvergenceTheorem.ii}$(\lambda \mapsto x_\lambda )\to x_\infty$ iff every subnet $\mu \mapsto x_{\lambda _\mu}$ has in turn a subnet $\nu \mapsto x_{\lambda _{\mu _\nu}}$ such that $(\nu \mapsto x_{\lambda _{\mu _\nu}})\to x_\infty$; and
\item \label{enmKelleysConvergenceTheorem.iii} for all directed sets $I$ and nets $x^i:\Lambda ^i\rightarrow X$, if $x^i\to (x^i)_\infty$ and $(i\mapsto (x^i)_\infty )\to (x^\infty )_\infty$, then $\left( I\times \prod _{i\in I}\Lambda ^i\ni (i,\lambda )\mapsto (x^i)_{\lambda ^i}\right) \to (x^\infty )_\infty$,
\end{enumerate}
then there is a unique topology on $X$ such that $\lambda \mapsto x_\lambda$ converges to $x_\infty$ iff $(\lambda \mapsto x_\lambda )\to x_\infty$.
\begin{rmk}
People sometimes attempt to define a topology by defining what it means to \emph{sequences} to converge.  This is nonsensical.  For example, \ref{enmKelleysConvergenceTheorem.iii} doesn't even make sense in this context.  Moreover, because of examples like \cref{exm4.2.25} (the cocountable topology and discrete topology on $\R$ have the same notion of sequential convergence), trying to rectify this in general is hopeless.  My best guess is that this is because people tend to shy away from the usage of nets for some reason, and so they tend to define topologies using convergence of sequences.  In any case, I don't recall ever seeing a case where such a definition was given that \emph{didn't} make sense after you replaced the word ``sequence'' with the word ``net'.  That is to say, even though they are technically wrong, there is usually a very easy fix.
\end{rmk}
\begin{proof}
\Step{Make hypotheses}
Suppose that (i) $(\lambda \mapsto x_\infty)\to x_\infty$; (ii) $(\lambda \mapsto x_\lambda )\to x_\infty$ iff every subnet $\mu \mapsto x_{\lambda _\mu}$ has in turn a subnet $\nu \mapsto x_{\lambda _{\mu _\nu}}$ such that $(\nu \mapsto x_{\lambda _{\mu _\nu}})\to x_\infty$; and (iii) for all directed sets $I$ and nets $x^i:\Lambda ^i\rightarrow X$, if $x^i\to (x^i)_\infty$ and $(i\mapsto (x^i)_\infty )\to (x^\infty )_\infty$, then $\left( I\times \prod _{i\in I}\Lambda ^i\ni (i,\lambda )\mapsto (x^i)_{\lambda ^i}\right) \to (x^\infty )_\infty$.

\Step{Define the notion of a \emph{adherent point}}
For a subset $S\subseteq X$ and $x\in X$, we say that $x$ is an \emph{adherent} of $S$ iff there is some net $\lambda \mapsto x_\lambda \in S$ such that $(\lambda \mapsto x_\lambda )\to x$.

\Step{Define what should be the closure}
For a subset $S\subseteq X$, we define $\mathrm{C}(S)$ to be the set of adherent points of $S$.

\Step{Show that $\mathrm{C}$ satisfies Kuratowski's Closure Axioms}
The empty-set has no adherent points points, and so trivially $\mathrm{C}(\emptyset )=\emptyset$.

It follows from \ref{enmKelleysConvergenceTheorem.i} that $S\subseteq \mathrm{C}(S)$.

We wish to show that $\mathrm{C}\left( \mathrm{C}(S)\right) \subseteq \mathrm{C}(S)$, so let $(x^\infty )_\infty \in \mathrm{C}\left( \mathrm{C}(S)\right)$.  Then, there is a net $I\ni i\mapsto x^i\in \mathrm{C}(S)$ such that $(i\mapsto x^i )\to (x^\infty )_\infty$.  As each $x^i\in \mathrm{C}(S)$, there is a net $\Lambda ^i\ni \lambda ^i\mapsto (x^i)_{\lambda ^i}\in S$ such that $\left( \lambda ^i\mapsto (x^i)_{\lambda ^i}\right) \to (x^i)_\infty$.  By \ref{enmKelleysConvergenceTheorem.iii}, it then follows that $\left( I\times \prod _{i\in I}\Lambda ^i\ni (i,\lambda )\mapsto (x^i)_{\lambda ^i}\right) \to (x^\infty )_\infty$.  As each $(x^i)_{\lambda ^i}\in S$, it follows that $(x^\infty )_\infty$ is an adherent point of $S$, which completes the proof that $\mathrm{C}\left( \mathrm{C}(S)\right) =\mathrm{C}(S)$ (the $\mathrm{C}(S)\subseteq \mathrm{C}\left( \mathrm{C}(S)\right)$ follows from the fact that $S\subseteq \mathrm{C}(S)$).

We now show that $\mathrm{C}(S\cup T)=\mathrm{C}(S)\cup \mathrm{C}(T)$.  We have that $S\subseteq \mathrm{C}(S\cup T)$, and so $\mathrm{C}(S)\subseteq \mathrm{C}(S\cup T)$.  Similarly for $T$, and so we have $\mathrm{C}(S)\cup \mathrm{C}(T)\subseteq \mathrm{C}(S\cup T)$.

We now show that $\mathrm{C}(S\cup T)\subseteq \mathrm{C}(S)\cup \mathrm{C}(T)$.  So, let $x_\infty \in \mathrm{C}(S\cup T)$, so that there is a net $\Lambda \ni \lambda \mapsto x_\lambda \in S\cup T$ such that $(\lambda \mapsto x_\lambda )\to x_\infty$.  Define
\begin{equation}
\Lambda _S\coloneqq \left\{ \lambda \in \Lambda :x_\lambda \in S\right\} \text{ and }\Lambda _T\coloneqq \left\{ \lambda \in \Lambda :x_\lambda \in T\right\} .
\end{equation}
As $\Lambda =\Lambda _S\cup \Lambda _T$, either $\Lambda _S$ or $\Lambda _T$ is cofinal in $\Lambda$.  Without loss of generality, suppose that $\Lambda _S$ is cofinal in $\Lambda$, so that $\restr{x}{\Lambda _S}$ is a strict subnet of $\lambda \mapsto x_\lambda$.  By hypothesis, this in turn must have a subnet $\mu \mapsto x_{\lambda _\mu}$, $\lambda _\mu \in \Lambda _S$, such that $(\mu \mapsto x_{\lambda _\mu})\to x_\infty$.  Thus, $x_\infty \in \mathrm{C}(S)$, and so $\mathrm{C}(S\cup T)\subseteq \mathrm{C}(S)\cup \mathrm{C}(T)$.

It follows by Kuratowski's Closure Theorem (\cref{KuratowskisClosureTheorem}) that there is a unique topology on $X$ such that $\Cls (S)=\mathrm{C}(S)$.

\Step{Show that $\lambda \mapsto x_\lambda$ converges to $x_\infty$ iff $(\lambda \mapsto x_\lambda)\to x_\infty$}
Suppose that $\Lambda \ni \lambda \mapsto x_\lambda$ converges to $x_\infty$.  We proceed by contradiction:  suppose that it is not the case that $(\lambda \mapsto x_\lambda )\to x_\infty$.  Then, there must be some subnet $I\ni \mu \mapsto x_{\lambda _\mu}$ that has no subnet $\nu \mapsto x_{\lambda _{\mu _\nu}}$ such that $(\nu \mapsto x_{\lambda _{\mu _\nu}})\to x_\infty$.  To obtain a contradiction, we construct such a subnet.  For each $\mu _0$, define $S_{\mu _0}\coloneqq \{ x_{\lambda _\mu}:\mu \geq \mu _0\}$.  As $\lambda \mapsto x_\lambda$ converges to $x_\infty$, so does $\mu \mapsto x_{\lambda _\mu}$, and so $x_\infty \in \Cls (S_{\mu _0})$ for all $\mu _0$.  Hence, by the definition of our closure, for each $\mu _0\in I$ there is a net $\Lambda ^{\mu _0}\ni \nu ^{\mu _0}\mapsto x_{\lambda _{\mu _{\nu ^{\mu _0}}}}\in S_{\mu _0}$, so that $\mu _{\nu ^{\mu _0}}\geq \mu _0$, with $\left( \nu ^{\mu _0}\mapsto x_{\lambda _{\mu _{\nu ^{\mu _0}}}}\right) \to x_\infty$.\footnote{The superscript $\mu _0$ in $\nu ^{\mu _0}$ is just to help us keep track of which directed set $\nu ^{\mu _0}$ is contained in.}  From \ref{enmKelleysConvergenceTheorem.iii}, it follows that $\left( I\times \prod _{\mu _0\in I}\Lambda ^{\mu _0}\ni (\mu _0,\nu )\mapsto x_{\lambda _{\mu _{\nu ^{\mu _0}}}})\right) \to x_\infty$.  Thus, if this is in fact a subnet of $\mu \mapsto x_{\lambda _\mu}$, we will have our contradiction.  To show that this is indeed a subnet, let $\mu _0$ be arbitrary.  To show this, we apply \cref{prp3.3.92}.  Let $\nu _0\in \prod _{\mu _0\in I}\Lambda ^{\mu _0}$ be arbitrary, and suppose that $(\mu ,\nu )\geq (\mu _0,\nu _0)$.  Recall that we have that $\mu _{\nu ^{\mu _0}}\geq \mu _0$ always, and so certainly we will have that $\mu _{\nu ^{\mu _0}}\geq \mu _0$. 

Now suppose that $(\Lambda \ni \lambda \mapsto x_\lambda )\to x_\infty$.  We proceed by contradiction:  suppose that $\lambda \mapsto x_\lambda$ does not converge to $x_\infty$.  Then, there is a subnet $\mu \mapsto x_{\lambda _\mu}$ which has no subnet converging to $x_\infty$.  In particular, $\mu \mapsto x_{\lambda _\mu}$ itself does not converge to $x_\infty$, and so there is an open neighborhood $U$ of $x_\infty$ which does not eventually contain $\mu \mapsto x_{\lambda _\mu}$.  It follows that $I\coloneqq \{ \lambda _\mu :x_{\lambda _\mu}\notin U\}$ is cofinal, so that $I\ni \mu \mapsto x_{\lambda _\mu}$ is a subnet contained in $U^{\comp}$.  On the other hand, we know that $(I\ni \mu \mapsto x_{\lambda _\mu})\to x_\infty$, so that $x_\infty \in \Cls (\{ x_{\lambda _\mu}:\mu \in I\})$, but this is a contradiction of the fact that it has an open neighborhood which contains no point of this set.

\Step{Demonstrate uniqueness}
Recall that sets are closed iff they contain all their limit points.  If we have two topologies with the same notion of convergence, then the set of limit points of a given set in each topology are the same, and consequently, a set is closed in one iff it is closed in the other.
\end{proof}
\end{thm}
One thing to note is that, in practice, it is often easier to check two conditions which are equivalent to the second axiom:   (i) subnets of convergent nets converge to the same thing and (ii) that $(\Leftarrow )$ direction of the second axiom.
\begin{prp}\label{prp3.4.22}
Let $X$ be a set, denote by $\mathcal{N}$ the collection of all nets in $X$, and let $\to$ be a relation on $\mathcal{N}\times X$.  Then, the following are equivalent.
\begin{enumerate}
\item \label{enm3.4.22.i}$(\lambda \mapsto x_\lambda )\to x_\infty$ iff every subnet $\mu \mapsto x_{\lambda _\mu}$ has in turn a subnet $\nu \mapsto x_{\lambda _{\mu _\nu}}$ such that $(\nu \mapsto x_{\lambda _{\mu _\nu}})\to x_\infty$.
\item \label{enm3.4.22.ii}\begin{enumerate}
\item \label{enm3.4.22.ii.a}If $(\lambda \mapsto x_\lambda )\to x_\infty$ and $\mu \mapsto x_{\lambda _\mu}$ is a subnet of $\lambda \mapsto x_\lambda$, then $(\mu \mapsto x_{\lambda _\mu})\to x_\infty$; and
\item \label{enm3.4.22.ii.b}If every subnet $\mu \mapsto x_{\lambda _\mu}$ has in turn a subnet $\nu \mapsto x_{\lambda _{\mu _\nu}}$ such that $(\nu \mapsto x_{\lambda _{\mu _\nu}})\to x_\infty$, then $(\lambda \mapsto x_\lambda )\to x_\infty$.
\end{enumerate}
\end{enumerate}
\begin{proof}
$(\Rightarrow )$ Suppose that $(\lambda \mapsto x_\lambda )\to x_\infty$ iff every subnet $\mu \mapsto x_{\lambda _\mu}$ has in turn a subnet $\nu \mapsto x_{\lambda _{\mu _\nu}}$ such that $(\nu \mapsto x_{\lambda _{\mu _\nu}})\to x_\infty$.  \ref{enm3.4.22.ii.b} holds by hypothesis.  We thus check \ref{enm3.4.22.ii.a}.  So, let $\lambda \mapsto x_\lambda$ be a net such that $(\lambda \mapsto x_\lambda )\to x_\infty$ and let $\mu \mapsto x_{\lambda _\mu}$ be a subnet.  To show that $(\mu \mapsto x_{\lambda _\mu})\to x_\infty$, we show that every subnet $\nu \mapsto x_{\lambda _{\mu _\nu}}$ has in turn a subnet $\xi \mapsto x_{\lambda _{\mu _{\nu _\xi}}}$ such that $(\xi \mapsto x_{\lambda _{\mu _{\nu _\xi}}})\to x_\infty$.  So, let $\nu \mapsto x_{\lambda _{\mu _\nu}}$ be a subnet of $\mu \mapsto x_{\lambda _\mu}$.  This itself is also a subnet of $\lambda \mapsto x_\lambda$, and so by hypothesis, it has a subnet $\xi \mapsto x_{\lambda _{\mu _{\nu _\xi}}}$ such that $(\xi \mapsto x_{\lambda _{\mu _{\nu _\xi}}})\to x_\infty$.

$(\Leftarrow )$ Suppose that (a) if $(\lambda \mapsto x_\lambda )\to x_\infty$ and $\mu \mapsto x_{\lambda _\mu}$ is a subnet of $\lambda \mapsto x_\lambda$, then $(\mu \mapsto x_{\lambda _\mu})\to x_\infty$; and (b) if every subnet $\mu \mapsto x_{\lambda _\mu}$ has in turn a subnet $\nu \mapsto x_{\lambda _{\mu _\nu}}$ such that $(\nu \mapsto x_{\lambda _{\mu _\nu}})\to x_\infty$, then $(\lambda \mapsto x_\lambda )\to x_\infty$.  The $(\Leftarrow )$ direction of \ref{enm3.4.22.i} is true by hypothesis, so it suffices to show the $(\Rightarrow )$ direction of \ref{enm3.4.22.i}.  So, suppose that $(\lambda \mapsto x_\lambda )\to x_\infty$ and let $\mu \mapsto x_{\lambda _\mu}$ be a subnet.  Then, by hypothesis, we also have that $(\mu \mapsto x_{\lambda _\mu})\to x_\infty$, and so this subnet $\mu \mapsto x_\mu$ has in turn a subnet (namely itself) that is related to $x_\infty$ by $\to$.
\end{proof}
\end{prp}

And of course, there is an analogous result for filters.
\begin{thm}[Kelley's Filter Convergence Theorem]\index{Kelley's Filter Convergence Theorem}\label{KelleysFilterConvergenceTheorem}
Let $X$ be a set, denoted by $\tilde{\mathcal{F}}$ the collection of all filter bases in $X$, and let $\to$ be a relation on $\tilde{\mathcal{F}}\times X$.  Then, if
\begin{enumerate}
\item \label{enmKelleysFilterConvergenceTheorem.i}$\mathcal{P}_x\to x$, where $\mathcal{P}_x\coloneqq \left\{ U\subseteq X:x\in U\right\}$;\footnote{``P'' is for \emph{principal}, the etymology being from the use of the word ``principal'' in the context of ideals in ring theory (to the best of my knowledge anyways).}
\item \label{enmKelleysFilterConvergenceTheorem.ii}$\mathcal{F}\to x$ iff for every filtering $\tilde{\mathcal{F}}\ni \mathcal{G}\supseteq \mathcal{F}$, there is some filtering $\tilde{\mathcal{F}}\ni \mathcal{H}\supseteq \mathcal{G}$ such that $\mathcal{H}\to x$; and
\item \label{enmKelleysFilterConvergenceTheorem.iii}for all directed sets $I$ and convergent filters $\mathcal{F}^i\to x^i\in X$, for $i\in I$, if $\mathcal{F}_{i\mapsto x^i}\to x^\infty$, then
\begin{equation}
\begin{split}
\mathcal{F}_\infty & \coloneqq \left\{ U\subseteq X:\text{there exists }i_U\in I\text{ such that,}\right. \\
& \qquad \left. \text{whenever }i\geq i_U\text{, }U\supseteq F^i\text{ for some }F^i\in \mathcal{F}^i\text{.}\right\} \to x^\infty ,
\end{split}
\end{equation}
\end{enumerate}
then there is a unique topology on $X$ such that $\mathcal{F}$ converges to $x_\infty \in X$ iff $\mathcal{F}\to x_\infty$.
\begin{proof}
\Step{Make hypotheses}
Suppose that \ref{enmKelleysFilterConvergenceTheorem.i}$\mathcal{P}_x\to x$, where $\mathcal{P}_x\coloneqq \left\{ U\subseteq X:x\in U\right\}$; \ref{enmKelleysFilterConvergenceTheorem.ii} $\mathcal{F}\to x$ iff for every filtering $\tilde{\mathcal{F}}\ni \mathcal{G}\supseteq \mathcal{F}$, there is some filtering $\tilde{\mathcal{F}}\ni \mathcal{H}\supseteq \mathcal{G}$ such that $\mathcal{H}\to x$; and \ref{enmKelleysFilterConvergenceTheorem.iii} for all directed sets $I$ and convergent filters $\mathcal{F}^i\to x^i\in X$, for $i\in I$, if $\mathcal{F}_{i\mapsto x^i}\to x^\infty$, then $\mathcal{F}_\infty \to x^\infty$.

\Step{Define a relation on $\mathcal{N}\times X$}
Let $\mathcal{N}$ denote the collection of all nets in $X$, and for $\lambda \mapsto x_\lambda \in \mathcal{N}$ and $x_\infty \in X$, define
\begin{equation}
(\lambda \mapsto x_\lambda )\to x_\infty \text{ iff }\mathcal{F}_{\lambda \mapsto x_\lambda}\to x_\infty .\footnote{Even though there is no topology around (yet), the notion of ``eventually containing'' still makes sense, and so of course $\mathcal{F}_{\lambda \mapsto x_\lambda}$ still makes sense.}
\end{equation}

\Step{Verify \ref{enmKelleysConvergenceTheorem.i} of Kelley's Convergence Theorem}
As the derived filter base of the constant net $\lambda \mapsto x_\infty$ is just the principal filter base $\mathcal{P}_{x_\infty}$, it follows that \ref{enmKelleysConvergenceTheorem.i} of Kelley's Convergence Theorem, \cref{KelleysConvergenceTheorem} holds.

\Step{Verify $(\Rightarrow)$ of \ref{enmKelleysConvergenceTheorem.ii} of Kelley's Convergence Theorem}
Let $\lambda \mapsto x_\lambda$ be a net such that $(\lambda \mapsto x_\lambda )\to x_\infty$, and let $\mu \mapsto x_{\lambda _\mu}$ be a subnet of $\lambda \mapsto x_\lambda$.  Then, $\mathcal{F}_{\lambda \mapsto x_{\lambda}}\to x_\infty$.  As $\mathcal{F}_{\mu \mapsto x_{\lambda _\mu}}\supseteq \mathcal{F}_{\lambda \mapsto x_\lambda}$, by \ref{enmKelleysFilterConvergenceTheorem.ii}, there is a filter base $\mathcal{G}\supseteq \mathcal{F}_{\mu \mapsto x_{\lambda _\mu}}$ with $\mathcal{G}\to x_\infty$.  Without loss of generality, we may assume that each $G\in \mathcal{G}$ contains some $x_{\lambda _\mu}$ (we may throw away all such sets which do not contain some $x_{\lambda _\mu}$, and as $(\lambda \mapsto x_\lambda )\to x_\infty$< this will not affect the fact that $\mathcal{G}\to x_\infty$).  For $G_1,G_2\in \mathcal{G}$, define $G_1\leq G_2$ iff $G_1\supseteq G_2$.  By the definition of a filter base (\cref{FilterBase}), it follows that $(\mathcal{G},\leq )$ is a directed set.  For each $G\in \mathcal{G}$, let $x_{\lambda _{\mu _G}}$ denote any element contained of the net $\mu \mapsto x_{\lambda _\mu}$ contained in $G$.

We first check that $G\mapsto x_{\lambda _{\mu _G}}$ is a subnet of $\mu \mapsto x_{\lambda _\mu}$.  So, let $U\subseteq X$ eventually contain $\mu \mapsto x_{\lambda _\mu}$.  Then, $U\in \mathcal{F}_{\mu \mapsto x_{\lambda _\mu}}$, and so $U\in \mathcal{G}$ because $\mathcal{G}\supseteq \mathcal{F}_{\mu \mapsto x_{\lambda _\mu}}$.  Now suppose that $G\geq U$, so that, by definition of $\leq$ on $\mathcal{G}$ as a directed set, $G\subseteq U$.  Hence, for $G\geq U$, $x_{\lambda _{\mu _G}}\in G\subseteq U$, so that $G\mapsto x_{\lambda _{\mu _G}}$ is eventually contained in $U$, and hence is a subnet of $\mu \mapsto x_{\lambda _\mu}$.

Thus, as $G\mapsto x_{\lambda _{\mu _G}}$ is a subnet of $\mu \mapsto x_{\lambda _\mu}$ and $(\mu \mapsto x_{\lambda _\mu})\to x_\infty$, it follows that $(G\mapsto x_{\lambda _{\mu _G}})\to x_\infty$.  Thus, the arbitrary subnet of $\mu \mapsto x_{\lambda _\mu}$ of $\lambda \mapsto x_\lambda$ does indeed have a subnet that `converges to' $x_\infty$,\footnote{``Converges to' is in quotes because what we really mean is that it is related to $x_\infty$ by the relation $\to$.} and so $(\Rightarrow)$ of \ref{enmKelleysConvergenceTheorem.ii} of Kelley's Convergence Theorem is indeed satisfied.

\Step{Verify $(\Leftarrow )$ of \ref{enmKelleysConvergenceTheorem.ii} of Kelley's Convergence Theorem}
Let $\lambda \mapsto x_\lambda$ be a net such that for every subnet $\mu \mapsto x_{\lambda _\mu}$, there is some subnet of that subnet $\nu \mapsto x_{\mu _\nu}$ such that $(\nu \mapsto x_{\mu _\nu})\to x_\infty$.  We wish to show that $(\lambda \mapsto x_\lambda )\to x_\infty$.  In other words, we want to show that $\mathcal{F}_{\lambda \mapsto x_\lambda}\to x_\infty$.  To show this, we apply \ref{enmKelleysFilterConvergenceTheorem.ii}.  So, let $\mathcal{G}\supseteq \mathcal{F}_{\lambda \mapsto x_\lambda}$ be a filter base.  Then, using the same technique as in the previous step,\footnote{Turn $\mathcal{G}$ into a directed set by reverse-inclusion, show that each $G\in \mathcal{G}$ must without loss of generality contain some $x_\lambda$, and pick any one such lambda to form a subnet $G\mapsto x_{\lambda _G}$.} there is a subnet $\mathcal{G}\ni G\mapsto x_{\lambda _G}$ with $\mathcal{G}$ ordered by reverse-inclusion.  By hypothesis, there is then in turn a subnet of this $\mu \mapsto x_{\lambda _{G_\mu}}$ such that $(\mu \mapsto x_{\lambda _{G_\mu}})\to x_\infty$, and hence $\mathcal{F}_{\mu \mapsto x_{\lambda _{G_\mu}}}\to x_\infty$.  If $\mathcal{F}_{\mu \mapsto x_{\lambda _{G_\mu}}}\supseteq \mathcal{G}$, this completes this step by \ref{enmKelleysFilterConvergenceTheorem.ii}.  To show this, let $G_0\in \mathcal{G}$ and suppose that $G\geq G_0$.  Then, $G\subseteq G_0$, and so $x_{\lambda _G}\in G\subseteq G_0$, so that $G\mapsto x_{\lambda _G}$ is eventually contained in $G_0$.  Therefore, by the definition of a subnet (\cref{Subnet}), $\mu \mapsto x_{\lambda _{G_\mu}}$ is eventually contained in $G$, and so $G\in \mathcal{F}_{\mu \mapsto x_{\lambda _{G_\mu}}}$, and so $\mathcal{F}_{\mu \mapsto x_{\lambda _{G_\mu}}}\supseteq \mathcal{G}$.

\Step{Show \ref{enmKelleysConvergenceTheorem.iii} of Kelley's Convergence Theorem}
Let $I$ be a directed set, for each $i\in I$ let $x^i:\Lambda ^i\rightarrow X$ be a net such that $x^i\to (x^i)_\infty$, and suppose that $(i\mapsto (x^i)_\infty )\to (x^\infty )_\infty \in X$.  Note that, for $\mathcal{F}^i\coloneqq \mathcal{F}_{\lambda ^i\mapsto (x^i)_{\lambda ^i}}$ as in \ref{enmKelleysFilterConvergenceTheorem.iii},
\begin{equation}
\begin{split}
\mathcal{F}_\infty & \coloneqq \left\{ U\subseteq X:\text{there exists }i_U\in I\text{ such that, whenever }i\geq i_U\text{,}\right. \\
& \qquad \left. \text{it follows that }U\text{ eventually contains }\lambda ^i\mapsto (x^i)_{\lambda ^i}\right\} .
\end{split}
\end{equation}
By \ref{enmKelleysFilterConvergenceTheorem.iii}, $\mathcal{F}^\infty \to x^\infty$.

We wish to show that
\begin{equation}
\left( I\times \prod _{i\in I}\Lambda ^i\ni (i,\lambda )\mapsto (x^i)_{\lambda ^i}\right) \to (x^\infty )_\infty .
\end{equation}
In other words, we would like to show that
\begin{equation}
\mathcal{F}_{(i,\lambda )\mapsto (x^i)_{\lambda ^i}}\to (x^\infty )_\infty .
\end{equation}
To show this, by \ref{enmKelleysFilterConvergenceTheorem.ii}, it suffices to show that
\begin{equation}
\mathcal{F}_\infty \subseteq \mathcal{F}_{(i,\lambda )\mapsto (x^i)_{\lambda ^i}}.
\end{equation}
So, let $U\in \mathcal{F}_\infty$.  Then, there is some $i_U\in I$ such that, whenever $i\geq i_U$, it follows that $U$ eventually contains $\lambda ^i\mapsto (x^i)_{\lambda ^i}$.  It follows from this that, for all such $i$, there is a $(\lambda ^i)_U$ such that, whenever $\lambda ^i\geq (\lambda ^i)_U$, it follows that $(x^i)_{\lambda ^i}\in U$.  Define $\lambda _U\in \prod _{i\in I}\Lambda ^i$ such that $(\lambda _U)^i\coloneqq (\lambda ^i)_U$ for $i\geq i_U$ and anything for $i\not \geq i_u$.  Then, whenever $(i,\lambda )\geq (i_U,\lambda _U)$, it follows that $i\geq i_U$ and each $\lambda ^i\geq (\lambda _U)^i\coloneqq (\lambda ^i)_U$, so that
\begin{equation}
(x^i)_{\lambda ^i}\in U.
\end{equation}
Thus, $U$ eventually contains $(i,\lambda )\mapsto (x^i)_{\lambda ^i}$, and we are done with this step.

\Step{Show that the topology satisfies $\mathcal{F}\to x_\infty$ iff $\mathcal{F}$ converges to $x_\infty$}
After verifying \ref{enmKelleysConvergenceTheorem.i}--\ref{enmKelleysConvergenceTheorem.iii} of Kelley's Convergence Theorem, it follows that there is a unique topology on $X$ that has the property that $(\lambda \mapsto x )\to x_\infty$ iff $\lambda \mapsto x_\lambda$ converges to $x_\infty$.  We now check that the analogous property in terms of filters holds for this topology.

For the remainder of this step, we will make use of Kelley's Filter Convergence Axioms (\cref{KelleysFilterConvergenceAxioms}), which we already know to be true about actual convergence of filter bases.

Suppose that $\mathcal{F}\to x_\infty$.  To show that $\mathcal{F}$ converges to $x_\infty$, we prove that any filtering $\mathcal{G}\supseteq \mathcal{F}$ has in turn a filtering which converges to $x_\infty$.  Order $\mathcal{G}$ by reverse-inclusion and pick $x_G\in G\in \mathcal{G}$ so that $G\mapsto x_G$ is a net.  It follows that $\mathcal{F}_{G\mapsto x_G}\supseteq \mathcal{G}$, and so by \ref{enmKelleysFilterConvergenceTheorem.ii}, $\mathcal{F}_{G\mapsto x_G}\to x_\infty$, and so $(G\mapsto x_G)\to x_\infty$, and so $G\mapsto x_G$ converges to $x_\infty$.  To show that $\mathcal{F}_{G\mapsto x_G}$ converges to $x_\infty$, let $U\subseteq X$ be an open neighborhood of $x_\infty$.  Then, $G\mapsto x_G$ is eventually in $U$, and so $U\subseteq U\in \mathcal{F}_{G\mapsto x_G}$.  Thus, every filtering of $\mathcal{F}$ has in turn a filtering which converges to $x_\infty$, and so $\mathcal{F}$ converges to $x_\infty$.

The converse follows because this argument is ``$\to$''$\leftrightarrow$``converges to'' symmetric.
\end{proof}
\end{thm}
And of course, there is an analogous result to \cref{prp3.4.22} which gives an equivalent characterization of the second axiom which can sometimes be easier to check.
\begin{prp}
Let $X$ be a set, denoted by $\tilde{\mathcal{F}}$ the collection of all filter bases in $X$, and let $\to$ be a relation on $\tilde{\mathcal{F}}\times X$.  Then, the following are equivalent.
\begin{enumerate}
\item $\mathcal{F}\to x$ iff for every filtering $\tilde{\mathcal{F}}\ni \mathcal{G}\supseteq \mathcal{F}$, there is some filtering $\tilde{\mathcal{F}}\ni \mathcal{H}\supseteq \mathcal{G}$ such that $\mathcal{H}\to x$.
\item \begin{enumerate}
\item If $\mathcal{F}\to x_\infty$ and $\mathcal{G}\supseteq \mathcal{F}$, then $\mathcal{G}\to x_\infty$; and
\item If every filtering $\tilde{\mathcal{F}}\ni \mathcal{G}\supseteq \mathcal{F}$, there is some filtering $\tilde{\mathcal{F}}\ni \mathcal{H}\supseteq \mathcal{G}$ such that $\mathcal{H}\to x$.
\end{enumerate}
\end{enumerate}
\begin{proof}
We leave this as an exercise.
\begin{exr}
Complete the proof yourself, using the proof of \cref{prp3.4.22} as a guide.
\end{exr}
\end{proof}
\end{prp}

\subsection{Definition by specification of continuous functions}

The following two results define a topology on a set by simply `declaring' that a collection of functions be continuous.  This is similar in nature to how we defined a topology with a generating collection (\cref{GeneratingCollection})---in this case, we started with a collection of subsets, and simply `declared' them to be open.
\begin{prp}[Initial topology]\label{InitialTopology}
Let $X$ be a set, let $\mathcal{Y}$ be an indexed collection of topological spaces, and for each $Y\in \mathcal{Y}$ let $f_Y:X\rightarrow Y$ be a function.  Then, there exists a unique topology  $\mathcal{U}$ on $X$, the \emph{initial topology}\index{Initial topology} with respect to $\{ f_Y:Y\in \mathcal{Y}\}$, such that
\begin{enumerate}
\item $f_Y:X\rightarrow Y$ is continuous with respect to $\mathcal{U}$ for all $Y\in \mathcal{Y}$; and
\item if $\mathcal{U}'$ is another topology for which each $f_Y:X\rightarrow Y$ is continuous, then $\mathcal{U}\subseteq \mathcal{U}'$.
\end{enumerate}
Furthermore, if $\mathcal{S}_Y$ generates the topology on $Y$, then the collection
\begin{equation}
\{ f_Y^{-1}(U):Y\in \mathcal{Y},\ U\in \mathcal{S}_Y\}
\end{equation}
generates $\mathcal{U}$.
\begin{rmk}
In other words, the initial topology is the smallest topology for which each $f_Y$ is continuous.
\end{rmk}
\begin{rmk}
But what about the largest such topology?  Well, the largest such topology is always going to be the discrete topology, which is not very interesting.  This is how you remember whether the initial topology is the smallest or largest---it can't be the largest because the discrete topology always works.
\end{rmk}
\begin{rmk}
In particular,
\begin{equation}
\{ f_Y^{-1}(U):Y\in \mathcal{Y},\ U\subseteq Y\text{ open}\}
\end{equation}
generates the initial topology.
\end{rmk}
\begin{rmk}
Compare this with the definition of the integers, rationals, closure, interior, and generating collections (\cref{Integers,RationalNumbers,Closure,Interior,GeneratingCollection}).
\end{rmk}
\begin{proof}
For $Y\in \mathcal{Y}$, let $\mathcal{S}_Y$ be any generating collection, define
\begin{equation}
\mathcal{S}\coloneqq \{ f_Y^{-1}(U):Y\in \mathcal{Y},\ U\in \mathcal{S}_Y\} ,
\end{equation}
and let $\mathcal{U}$ be the topology generated by $\mathcal{S}$ (\cref{GeneratingCollection}).

As every element of $\mathcal{S}$ is open, it is certainly the case that each $f_Y$ is continuous by \cref{exr4.1.27} (it suffices to check continuity on a generating collection).  On the other hand, if $\mathcal{U}'$ is another topology for each each $f_Y$ is continuous, then it must certainly contain $\mathcal{S}$, in which case $\mathcal{U}'$ contains $\mathcal{U}$ by the definition of a generating collection.

\begin{exr}
Show that the initial topology is unique.
\end{exr}
\end{proof}
\end{prp}
There is a nice characterization of convergence in initial topologies.
\begin{prp}\label{prp4.4.5}
Let $X$ have the initial topology with respect to the collection $\{ f_Y:Y\in \mathcal{Y}\}$, let $\lambda \mapsto x_\lambda \in X$ be a net, and let $x_\infty \in X$.  Then, $\lambda \mapsto x_\lambda$ converges to $x_\infty$ in $X$ iff $\lambda \mapsto f_Y(x_\lambda )$ converges to $f_Y(x_\infty )$ in $Y$ for all $Y\in \mathcal{Y}$.  Furthermore, the initial topology is the unique topology that has this property.
\begin{proof}
$(\Rightarrow )$ Suppose that $\lambda \mapsto x_\lambda$ converges to $x_\infty$ in $X$.  Then, because each $f_Y$ is continuous, it follows that $\lambda \mapsto f_Y(x_\lambda )$ converges to $f_Y(x_\infty )$ in $Y$ for all $Y\in \mathcal{Y}$.

\blankline
\noindent
$(\Leftarrow )$ Suppose that $\lambda \mapsto f_Y(x_\lambda )$ converges to $f_Y(x_\infty )$ in $Y$ for all $Y\in \mathcal{Y}$.  To show that $\lambda \mapsto x_\lambda$ converges to $x_\infty$ in $X$, we apply \cref{exr4.2.41} (it suffices to check convergence on a generating collections).  We know that
\begin{equation}
\{ f_Y^{-1}(U):Y\in \mathcal{Y},\ U\subseteq Y\text{ open}\}
\end{equation}
generates the initial topology, and so we apply \cref{exr4.2.41} to this.  So, let $f_Y^{-1}(U)$ be an open neighborhood of $x_\infty$.  We must show that $\lambda \mapsto x_\lambda$ is eventually contained in $f_Y^{-1}(U)$.  However, if $f_Y^{-1}(U)$ is an open neighborhood of $x_\infty$, then $U\supseteq f_Y(f_Y^{-1}(U))$ is an open neighborhood of $f_Y(x_\infty )$, and so $\lambda \mapsto f_Y(x_\lambda )$ is eventually contained in $U$ because $\lambda \mapsto f_Y(x_\lambda )$ converges to $f_Y(x_\infty )$.  But then $\lambda \mapsto x_\lambda$ is eventually contained in $U$, and we are done.

\blankline
\noindent
Uniqueness follows from \nameref{KelleysConvergenceTheorem}.
\end{proof}
\end{prp}
Initial topologies are great in that you can determine whether functions into $X$ are continuous or not by looking at their composition with each $f_Y$.
\begin{prp}\label{prp3.4.6}
Let $X$ have the initial topology with respect to the collection $\{ f_Y:Y\in \mathcal{Y}\}$, let $Z$ be a topological space, and let $f:Z\rightarrow X$ be a function.  Then, $f$ is continuous iff $f_Y\circ f$ is continuous for all $Y\in \mathcal{Y}$.  Furthermore, the initial topology is the unique topology with this property.
\begin{proof}
$(\Rightarrow )$ Suppose that $f$ is continuous.  Then, because each $f_Y$ is itself continuous and compositions of continuous functions are continuous, it follows that $f_Y\circ f$ is continuous for all $Y\in \mathcal{Y}$.

\blankline
\noindent
$(\Leftarrow )$ Suppose that $f_Y\circ f$ is continuous for all $Y\in \mathcal{Y}$.  Let $\lambda \mapsto x_\lambda$ converge to $x_\infty \in X$.  To show that $f$ is continuous, it suffices to show that $\lambda \mapsto f(x_\lambda)$ converges to $f(x_\infty )$.  However, because each $f_Y\circ f$ is continuous, $\lambda \mapsto f_Y(f(x_\lambda ))$ converges to $f_Y(f(x_\infty ) )$.  Therefore, by the previous result, we do indeed have that $\lambda \mapsto f(x_\lambda)$ converges to $f(x_\infty )$.

\blankline
\noindent
Uniqueness follows from the uniqueness in the previous proposition.
\end{proof}
\end{prp}

There is a `dual' version of the initial topology, in which the functions map \emph{into} the set on which we would like to define a topology.
\begin{prp}[Final topology]\label{FinalTopology}
Let $X$ be a set, let $\mathcal{Y}$ be an indexed collection of topological spaces, and for each $Y\in \mathcal{Y}$ let $f_Y:Y\rightarrow X$ be a function.  Then, there exists a unique topology $\mathcal{U}$ on $X$, the \emph{final topology}\index{Final topology} with respect to $\{ f_Y:Y\in \mathcal{Y}\}$, such that
\begin{enumerate}
\item $f_Y:Y\rightarrow X$ is continuous with respect to $\mathcal{U}$ for all $Y\in \mathcal{Y}$; and
\item if $\mathcal{U}'$ is another topology for which each $f_Y$ is continuous, then $\mathcal{U}\supseteq \mathcal{U}'$.
\end{enumerate}
Furthermore,
\begin{equation}
\mathcal{U}=\{ U\in 2^X:f_Y^{-1}(U)\text{ is open for all }Y\in \mathcal{Y}\text{.}\} .
\end{equation}
\begin{rmk}
In other words, the final topology is the largest topology for which each $f_Y$ is continuous.
\end{rmk}
\begin{rmk}
But what about the smallest such topology?  Well, the smallest such topology is always going to be the indiscrete topology, which is not very interesting.  This is how you remember whether the final topology is the smallest or largest---it can't be the smallest because the indiscrete topology always works.
\end{rmk}
\begin{proof}
Define
\begin{equation}\label{3.4.9}
\mathcal{U}\coloneqq \{ U\in 2^X:f_Y^{-1}(U)\text{ is open for all }Y\in \mathcal{Y}\text{.}\} .
\end{equation}
\begin{exr}
Check that $\mathcal{U}$ is actually a topology.
\begin{rmk}
We didn't need to do any such checking in the construction of the initial topology because there we just took the topology \emph{generated} by the collection.
\end{rmk}
\begin{rmk}
I am not aware of `dual' to \cref{prp4.4.5} that characterizes convergence in final topologies.
\end{rmk}
\end{exr}
Of course every $f_Y$ is continuous with respect to $\mathcal{U}$ as, by definition, the preimage of every element of $\mathcal{U}$ is open.  Furthermore, anything larger than $\mathcal{U}$ would necessarily have to contain some set for which the preimage under some $f_Y$ would not be open, and so that $f_Y$ would not be continuous.
\begin{exr}
Show that the final topology is unique.
\end{exr}
\end{proof}
\end{prp}
There is likewise a `dual' result to \cref{prp3.4.6} which tells us when continuous functions \emph{on} a space equipped with a final topology are continuous.
\begin{prp}\label{prp3.4.34x}
Let $X$ have the final topology with respect to the collection $\{ f_Y:Y\in \mathcal{Y}\}$, let $Z$ be a topological space, and let $f:X\rightarrow Z$ be a function.  Then, $f$ is continuous iff $f\circ f_Y$ is continuous for all $Y\in \mathcal{Y}$.  Furthermore, the final topology is the unique topology with this property.
\begin{proof}
$(\Rightarrow )$ Suppose that $f$ is continuous.  Then, because each $f_Y$ is itself continuous and compositions of continuous functions are continuous, it follows that $f\circ f_Y$ is continuous for all $Y\in \mathcal{Y}$.

\blankline
\noindent
$(\Leftarrow )$ Suppose that $f\circ f_Y$ is continuous for all $Y\in \mathcal{Y}$.  Let $U\subseteq Z$ be open.  We must show that $f^{-1}(U)$ is open.  However, from \eqref{3.4.9}, we know that $f^{-1}(U)$ will be open iff $f_Y^{-1}\left( f^{-1}(U)\right)$ will be open for all $Y\in \mathcal{Y}$.  However, $f_Y^{-1}\left( f^{-1}(U)\right) =[f\circ f_Y]^{-1}(U)$ is open because $f\circ f_Y$ is continuous.

\blankline
\noindent
Let $\mathcal{U}$ be another topology that has the property that $f:X\rightarrow Z$ is continuous iff $f\circ f_Y$ is continuous for all $Y\in \mathcal{Y}$.  To show that $\mathcal{U}$ is the final topology, by the definition (\cref{FinalTopology}), it suffices to show that each $f_Y:Y\rightarrow X$ is continuous with respect to $\mathcal{U}$ and that any other such topology is contained in $\mathcal{U}$.  As $\id _X:\coord{X,\mathcal{U}}\rightarrow \coord{X,\mathcal{U}}$ is continuous, by hypothesis, it follows that $\id _X\circ f_Y=f_Y$ is continuous with respect to $\mathcal{U}$.  Let $\mathcal{U}'$ be another topology such that $f_Y$ is continuous with respect to $\mathcal{U}'$ for all $Y\in \mathcal{Y}$.  We must show that $\mathcal{U}'\subseteq \mathcal{U}$.  To show this, it suffices to show that $\id _X:\coord{X,\mathcal{U}}\rightarrow \coord{X,\mathcal{U}'}$ is continuous.  By the defining property of $\mathcal{U}$, to show this, it suffices to show that the composition of this with each $f_Y$ is continuous.  In other words, it suffices to show that each $f_Y$ is continuous with respect to $\mathcal{U}'$, but this is true by hypothesis.
\end{proof}
\end{prp}

\subsection{Summary}

We now quickly recap all the ways in which we know how to specify a topology on a set.
\begin{enumerate}
\item We can specify the open sets (\cref{TopologicalSpace})..
\item We can specify the closed sets (\cref{exr4.1.2}).
\item We can specify a base (\cref{Base}).
\item We can specify a neighborhood base (\cref{NeighborhoodBase}).
\item We can generate a topology (\cref{GeneratingCollection}).
\item We can define the closure of each set (\cref{KuratowskisClosureTheorem}).
\item We can define the interior of each set (\cref{KuratowskisInteriorTheorem}).
\item We can define convergence of nets (\cref{KelleysConvergenceTheorem}).
\item We can define convergence of filters (\cref{KelleysFilterConvergenceTheorem}).
\item We can declare that functions on the space are continuous (the initial topology---see \cref{InitialTopology}).
\item We can declare that functions into the space are continuous (the final topology---see \cref{FinalTopology}.
\end{enumerate}

\section{The subspace, quotient, product, and disjoint-union topologies}

The purpose of this section is to present several ways of constructing new topologies spaces from old.  In brief, the subspace topology will be the topology we put on subsets of topological spaces, the quotient topology will be the topology we put on quotients of topological spaces,\footnote{Quotients in the sense of \cref{dfnA.1.42}.}, the product topology is the topology we put on cartesian-products, the disjoint-union topology (surprise, surprise) is the topology we put on disjoint-unions of topological spaces.  They key to defining all of these topologies are the initial (for the subspace and product topologies) and final topologies (for the quotient and disjoint-union topologies) (\cref{InitialTopology,FinalTopology}).

\subsection{The subspace topology}

All \emph{subsets} of topological spaces have a canonically associated topology, called the \emph{subspace topology}.  Note that this is not completely immediate---for example, it is not the case that every subset of a ring is a ring.  There is definitely something to define and something to check (that the subspace topology is in fact a topology).
\begin{prp}[Subspace topology]\label{SubspaceTopology}
Let $X$ be a topological space and let $S\subseteq X$.  Then, there exists a unique topology $\mathcal{U}$ on $S$, the \emph{subspace topology}\index{Subspace topology}, that has the property that a function into $S$ is continuous iff it is continuous regarded as a function into $X$.  Furthermore, the subspace topology is the initial topology with respect to the inclusion $\iota :S\hookrightarrow X$.  In particular,
\begin{equation}
\mathcal{U}=\{ U\cap S:U\subseteq X\text{ is open.}\} .
\end{equation}
\begin{rmk}
Unless otherwise stated, subsets of topological spaces are always equipped with the subspace topology.
\end{rmk}
\begin{proof}
All of this follows from the definition of the initial topology and its characterization of continuity of continuity of functions into initial topologies (\cref{InitialTopology,prp3.4.6}), with exception of the fact that $\mathcal{U}=\{ U\cap S:U\subseteq X\text{ is open.}\}$.  \cref{InitialTopology} tells us that this generates the initial topology, but it does not tell us that it is a topology itself.  Of course, however, if a generating collection is itself a topology, then the topology it generates is just itself.  Therefore, it suffices just to check that $\{ U\cap S:U\subseteq X\text{ is open.}\}$ is in fact a topology.
\begin{exr}
Check that $\{ U\cap S:U\subseteq X\text{ is open.}\}$ is in fact a topology.
\end{exr}
\end{proof}
\end{prp}
\begin{exm}\label{exm4.1.14}
Consider the subspace topology on $[0,1]\subseteq \R$.  Note that $(\frac{1}{2},1]=(\frac{1}{2},\infty )\cap [0,1]$ is \emph{open in $[0,1]$}, despite the fact that it obviously not open in $\R$.
\end{exm}
\begin{exm}
Note that the order topology on $\Q$ is the space as the subspace topology inherited from $\R$.  This is of course because they are both equipped with the order topology of the same order.  Likewise, for $\N \subseteq \Z$ and $\Z \subseteq \Q$.
\end{exm}

\subsection{The quotient topology}

Whenever we have a surjective function from a topological space $X$ onto a set $Y$, we can use this function and the topology on $X$ to place a topology on $Y$.  Recall (\cref{exrA.1.81}) that every surjective function can be viewed as a quotient function---this of course is the etymology of the term ``quotient topology''.
\begin{prp}
Let $X$ be a topological space, let $Y$ be a set, and let $\q :X\rightarrow Y$ be surjective.  Then, there exists a unique topology $\mathcal{U}$ on $Y$, the \emph{quotient topology}\index{Quotient topology}, that has the property that a function on $Y$ is continuous iff its composition with $\q$ is continuous.  Furthermore, the subspace topology is the final topology with respect to $\q :X\rightarrow Y$.  In particular,
\begin{equation}
\mathcal{U}=\{ U\subseteq Y:\q ^{-1}(U)\text{ is open.}\} .
\end{equation}
\begin{rmk}
Unless otherwise stated, quotients of topological spaces are always equipped with the quotient topology.
\end{rmk}
\begin{proof}
All of this follows from the definition of the final topology and its characterization in terms of continuity of functions defined on final topologies (\cref{FinalTopology,prp3.4.34x}).
\end{proof}
\end{prp}

\subsection{The product topology}

The product topology is the canonical topology we put on a cartesian-product of topological spaces $X\times Y$.  While we do technically use this in places, we use it in ways where we could have gotten-away with not speaking of the product topology per se (for example, the product topology on $\R \times \R$ is the same as the topology defined by $\varepsilon$-balls).  The real reason we go to the trouble of talking about the product topology explicitly is for the proof of producing a \emph{counter-example} to the statement
\begin{textequation}
A space is quasicompact iff every net has a convergent \emph{strict} subnet.\footnote{In case you're skimming and didn't read the context, this statement is \emph{false}.}
\end{textequation}
We mentioned when we defined subnets (\cref{Subnet}) that the notion of a strict subnet was the more obvious ``naive'' notion (that is, you just take terms from the original net, subject to the only condition that your indices get arbitrarily large), but that this ``naive'' notion was insufficient because it didn't allow us to prove certain theorems.  They key result, that spaces are quasicompact iff every net has a convergent subnet (\cref{prp4.2.31}), was precisely the example of a theorem we had in mind.

In brief, the counter-example will be
\begin{equation}
X\coloneqq \prod _{2^{\N}}\{ 0,1\} ,
\end{equation}
that is, an uncountable product (precisely, a product over the power set of $\N$) of the two-element set $\{ 0,1\}$.  Obviously $\{ 0,1\}$ is quasicompact (all finite spaces are)---Tychnoff's Theorem is the statement that \emph{arbitrary} products of quasicompact spaces are quasicompact, which tells us that $X$ is quasicompact.

So before we do anything else then, we must first define the product topology.
\begin{prp}[Product topology]\label{ProductTopology}
Let $\mathcal{X}$ be an indexed collection of topological spaces.  Then, there exists a unique topology, the \emph{product topology}\index{Product topology}, on $\prod _{X\in \mathcal{X}}X$, that has the property that a function into $\prod _{X\in \mathcal{X}}X$ is continuous iff each component of the function is continuous.  Furthermore, the product topology is the initial topology with respect to $\{ \pi _X:X\in \mathcal{X}\}$.\footnote{Recall (\cref{CartesianProductCollection}) that $\pi _X:\prod _{X\in \mathcal{X}}X\rightarrow X$ is just the projection.}  In particular, if $\mathcal{S}_X$ is a generating collection for the topology on $X\in \mathcal{X}$, then the collection
\begin{equation}\label{4.5.4x}
\{ \pi _X^{-1}(U):U\in \mathcal{S}_X\}
\end{equation}
generates the product topology, so that
\begin{equation}\label{4.5.4}
\mathcal{B}\coloneqq \left\{ \prod _{X\in \mathcal{X}}S_X:S_X\in \mathcal{S}_X\text{ and all but finitely many }S_X=X\right\} \footnote{We can always without loss of generality assume that $X\in \mathcal{B}_X$ because, if it wasn't there before, throw it in.}
\end{equation}
is a base for the product topology.
\begin{rmk}
The ``all but finitely many'' phrase is \emph{crucial}.  For example, in the space
\begin{equation}
\prod _\N \R ,
\end{equation}
that is, a countably-infinite product of $\R$, the set
\begin{equation}
(0,1)\times (0,1)\times (0,1)\times \cdots 
\end{equation}
is \emph{not} even open in the product topology on $\prod _\N \R$ (much less an element of any base).  This topology (called the \emph{box topology}\index{Box topology}) might be your more naive guess, but it is `wrong' in the sense that, if we allow things like this, then we lose the property that the continuity of a function is determined by the continuity of the components of the function---see \cref{exm4.5.20} below.
\end{rmk}
\begin{rmk}
Unless otherwise stated, products of topological spaces are always equipped with the product topology.
\end{rmk}
\begin{proof}
All of this follows from the definition of the initial topology, its characterization in terms of continuity of functions into initial topologies, and the defining result of generating collections (\cref{InitialTopology,prp3.4.6,GeneratingCollection}).
\end{proof}
\end{prp}
We mentioned in the remarks of this theorem that if you take as a base sets of the form $\prod _{X\in \mathcal{X}}S_X$ (with $S_X\neq X$ \emph{for all $X\in \mathcal{X}$} permissible), then you will lose the property that a function into the product is continuous iff each of its components is.  We now present a counter-example.
\begin{exm}[A discontinuous function into the box topology with each component continuous]\label{exm4.5.20}
Define \emph{as a set}
\begin{equation}
X\coloneqq \prod _{2^\N}\R 
\end{equation}
and consider the function
\begin{equation}
\id _X:\coord{X,\text{product topology}}\rightarrow \coord{X,\text{box topology}}.
\end{equation}
Then, each component of this function is continuous because the component of the identity is just the projection from $X$ onto the corresponding copy of $\R$ (the preimage of $U\subseteq \R$ under this projection is open in the product topology because in fact, according to the theorem, such an element is in a generating collection of the product topology).  On the other hand,
\begin{equation}
\prod _{2^\N}(0,1)
\end{equation}
is open in the box topology (by definition), but not open in the product topology by the theorem above.
\end{exm}
There is a relatively useful corollary of the definition of the product topology that characterizes convergence.
\begin{crl}\label{crl4.5.15}
Let $\mathcal{X}$ be an indexed collection of topological spaces, let $\lambda \mapsto x_\lambda \in \prod _{X\in \mathcal{X}}X$ be a net, and let $x_\infty \in \prod _{X\in \mathcal{X}}X$.  Then, $\lambda \mapsto x_\lambda$ converges to $x_\infty$ iff $\lambda \mapsto (x_\lambda )_X$ converges to $(x_\infty )_X$ in $X$ for all $X\in \mathcal{X}$.
\begin{rmk}
In other words, nets converge to an element in a product iff every component converges to the corresponding component of that element.
\end{rmk}
\begin{proof}
As the product topology on $\prod _{X\in \mathcal{X}}X$ is the initial topology with respect to the projections, $\{ \pi _X:X\in \mathcal{X}\}$, this result follows from \cref{prp4.4.5}.
\end{proof}
\end{crl}
\begin{exr}[Projections are open]\label{ProjectionsAreOpen}
Let $\mathcal{X}$ be an indexed collection of topological spaces, let $X\in \mathcal{X}$, and let $U\subseteq \prod _{X\in \mathcal{X}}X$ be open.  Show that $\pi _X(U)\subseteq X$ is open.
\begin{rmk}
Functions that have the property that the image of open sets are open are \emph{open functions}\index{Open function}.
\end{rmk}
\end{exr}

Now that we have defined the product topology, we prove a relatively deep result concerning quasicompactness of products which will allow us to produce the desired counter-example.
\begin{thm}[Tychnoff's Theorem]\index{Tychnoff's Theorem}\label{TychnoffsTheorem}
Let $\mathcal{X}$ be a collection of quasicompact spaces.  Then, $\prod _{X\in \mathcal{X}}X$ is quasicompact.
\begin{proof}\footnote{Proof adapted from \cite[pg.~143]{Kelley}.}
Recall that the product topology on $\prod _{X\in \mathcal{X}}X$ has a generating collection of the form $\pi _X^{-1}(U_X)$ for $U_X\subseteq X$ open, where $\pi _X:\prod _{X\in \mathcal{X}}X\rightarrow X$ is the projection.  We apply Alexander's Subbase Theorem (\cref{AlexanderSubbaseTheorem}) to this generating collection.

So, let $\mathcal{U}$ be an open cover of $\prod _{X\in \mathcal{X}}X$ of subsets of the form $\pi _X^{-1}(U_X)$.  It suffices to show that if no finite subset of $\mathcal{U}$ covers $\prod _{X\in \mathcal{X}}X$, then $\mathcal{U}$ itself does not cover $X$.  Define
\begin{equation}
\mathcal{U}_X\coloneqq \left\{ U\subseteq X\text{ open}:\pi _X^{-1}(U)\in \mathcal{U}\right\} .
\end{equation}
If a finite subset of $\mathcal{U}_X$ covered $X$, then its preimage would cover $\prod _{X\in \mathcal{X}}X$, and so by quasicompactness of $X$, it follows that $\mathcal{U}_X$ does not cover $X$, so choose $x_X\in X$ not contained in $\bigcup _{U\in \mathcal{U}_X}U$.  Then, the element $x\in \prod _{X\in \mathcal{X}}X$ whose coordinate at $X\in \mathcal{X}$ is $x_X$ is not contained in $\bigcup _{U\in \mathcal{U}}U$.
\end{proof}
\end{thm}

And finally we are able to present our counter-example.
\begin{exm}[A quasicompact space with a net that has no convergent \emph{strict} subnet]\footnote{This example was shown to me by Eric Wofsey on \href{http://mathoverflow.net/questions/210947/a-quasicompact-space-with-a-net-that-contains-no-convergent-strict-subnet}{mathoverflow.net}.}
Define
\begin{equation}
X\coloneqq \prod _{S\subseteq \N}\{ 0,1\} =\{ 0,1\} ^{2^\N}
\end{equation}
that is, a product of $2^{\aleph _0}$ copies of the two element set $\{ 0,1\}$.  In other words, it is the set of all functions from $2^{\N}$ into $\{ 0,1\}$---in fact, for the most of this example, we shall think of this space as the collection of functions.   This is quasicompact by Tychnoff's Theorem (and because finite sets are quasicompact---see \cref{exr4.2.33x}).  On the other hand, we may define a sequence $m\mapsto x_m\in X$ as follows.
\begin{equation}
x_m(S)\coloneqq \begin{cases}1 & \text{if }m\in S \\ 0 & \text{if }m\notin S\end{cases},
\end{equation}
where $S\subseteq \N$ ($x_m$ is a function from $2^\N$ into $\{ 0,1\}$, and so $x_m(S)$ is the value of this function at the element $S\in 2^\N$).

We now show that this sequence has no convergent strict subnet (necessarily also a sequence).  We proceed by contradiction:  suppose that there were a cofinal subset $\Lambda '\subseteq \N$, $\Lambda '=\{ m_0,m_1,m_2,\ldots \}$ (with $m_n\leq m_{n+1}$), such that $n\mapsto x_{m_n}$ converges.  Then, by \cref{crl4.5.15} (nets in products converge iff each component does), for each $S\subseteq \N$, the sequence $n\mapsto x_{m_n}(S)\in \{ 0,1\}$ would have to converge.  Thus, for each $S\subseteq \N$, the sequence $n\mapsto x_{m_n}(S)$ would have to be either eventually $0$ or eventually $1$.  In other words, either (i) for all but finitely many $m_n\in \Lambda '$, $m_n\notin S$;  or (ii) for all but finitely many $m_n\in \Lambda '$, $m_n\in S$.  In other words, for all $S\subseteq \N$, either (i) there is a cofinite\footnote{\emph{Cofinite}\index{Cofinite} means that the complement is finite.} subset of $\Lambda '$ that is contained in $S^{\comp}$ or (ii) there is a cofinite subset of $\Lambda '$ that is contained in $S$.\footnote{Cofinite in $\Lambda '$, that is.}

So, take $S\coloneqq \{ m_0,m_2,m_4,\ldots \} \subset \Lambda '$.  Then, there is some cofinite subset $\Lambda ''\subseteq \Lambda '$ such that either $\Lambda ''\subseteq S$ or $\Lambda ''\subseteq S^{\comp}$.  In the former case, we have that
\begin{equation}
\text{finite set }=\Lambda '\setminus \Lambda '' \supseteq \Lambda '\setminus S=\{ m_1,m_3,m_5,\ldots \} :
\end{equation}
a contradiction.  Thus, we must have that $\Lambda ''\subseteq S^{\comp}$, and so
\begin{equation}
\text{finite set }=\Lambda '\setminus \Lambda ''\supseteq \Lambda '\setminus S^{\comp}=\{ m_0,m_2,m_4,\ldots \} :
\end{equation}
a contradiction.  As both possibilities resulted in a contradiction, this itself is a contradiction, and so our assumption that there was a convergent strict subnet must have been incorrect.  Therefore, $m\mapsto x_m$ contains no convergent strict subnet, despite the fact that $X$ is quasicompact.
\end{exm}

\subsection{The disjoint-union topology}

Our inclusion of the disjoint-union topology is mostly because of its obvious duality to the product topology---it feels incomplete not to include it.  On the other hand, while in principle, being completely dual to the product topology, it should be no more or less difficult work with, in practice it seems that it is \emph{much} easier to get a handle on, and so probably doesn't deserve as in-depth a treatment.
\begin{prp}[Disjoint-union topology]\label{DisjointUnionTopology}
Let $\mathcal{X}$ be an indexed collection of topological spaces.  Then, there exists a unique topology, the \emph{disjoint-union topology}\index{disjoint-union topology}, on $\coprod _{X\in \mathcal{X}}X$, that has the property that a function defined on $\coprod _{X\in \mathcal{X}}X$ is continuous iff its restriction to each component is continuous.  Furthermore, the disjoint-union topology is the final topology with respect to $\{ \iota _X:X\in \mathcal{X}\}$.\footnote{Recall (\cref{DisjointUnionCollection}) that $\iota _X:X\rightarrow \coprod _{X\in \mathcal{X}}X$ is just the inclusion.}  In particular, a set $U\subseteq \coprod _{X\in \mathcal{X}}X$ is open iff $U\cap \iota _X(X)$ is open for all $X\in \mathcal{X}$.
is a base for the product topology.
\begin{rmk}
Unless otherwise stated, disjoint-unions of topological spaces are always equipped with the disjoint-union topology.
\end{rmk}
\begin{proof}
All of this follows from the definition of the final topology and its characterization in terms of continuity of functions defined on final topologies (\cref{FinalTopology,prp3.4.34x}).
\end{proof}
\end{prp}

\section{Separation Axioms}\label{sct4.5}

We have mentioned the term ``$T_2$'' a couple of times now (see, for example, the definition of quasicompactness (\cref{Quasicompact}) and the \nameref{HeineBorelTheorem} (\cref{HeineBorelTheorem})).  One of the purposes of this section is to explain what we meant by this.  The term ``$T_2$'' is a separation axiom, and you should know the other separation axioms as well if you plan to become a mathematician, but this is admittedly not a priority for this course (a study of them in detail is better suited for a course on general topology itself).

\subsection{Separation of subsets}

In this subsection, we will define various levels of ``sepration'' of subsets of a topological space.  In the next section, we will then define corresponding levels of separation for spaces, which is roughly the condition that any two points have the corresponding level of separation.\footnote{If that doesn't make sense now, don't worry---I can't be completely clear just yet.  Worry if it doesn't make sense after the next subsection ;-).}

Throughout this subsection, let $S_1,S_2\subseteq X$ be \emph{disjoint} subsets of a topological space $X$.  In the various definitions will follow, we will say things like ``$S_1$ and $S_2$ are XYZ.''.  If $S_1=\{ x_1\}$ and $S_2=\{ x_2\}$ are singletons, then instead we will say that ``$x_1$ and $x_2$ are XYZ.''.  In fact, this is probably the case of most interest (though certainly not the only case).
\begin{dfn}[Topologically-distinguishable]\label{TopologicallyDistinguishable}
$S_1$ and $S_2$ are \emph{topologically-distinguish\-able}\index{Topologically-distinguishable} iff there is an open set containing $S_1$ but not $S_2$ or vice-versa.
\begin{rmk}
In other words, they do not have precisely the same open neighborhoods.
\end{rmk}
\begin{rmk}
For example, in an indiscrete space, no two points are topologically-distinguishable.  In $\R$ on the other hand (and almost every space you work with that is not expressly cooked-up for the sole purpose of being a counter-example to something), any two points are topologically-distinguishable.
\end{rmk}
\end{dfn}
\begin{exm}[Two distinct points which are not topologically-distinguishable]\label{exm4.5.2}
We just mentioned this in the remark above, but decided to place it in an example of its own to make it easier to spot if skimming.  Take $X\coloneqq \{ x_1,x_2\}$ and equip $X$ with the indiscrete topology, that is,
\begin{equation}
\mathcal{U}\coloneqq \{ \emptyset ,X\} .
\end{equation}
Then, $x_1\neq x_2$ but $x_1$ and $x_2$ are contained in precisely the same open sets.
\end{exm}
\begin{dfn}[Separated]\label{Separated}
$S_1$ and $S_2$ are \emph{separated}\index{Separated} iff there is an open neighborhood $U_1$ of $S_1$ not intersecting $S_2$ \emph{and} an open neighborhood $U_2$ of $S_2$ not intersecting $S_1$.
\begin{rmk}
The difference between this and topological-distinguishability is that this has to happen to \emph{both} $S_1$ and $S_2$, whereas, in the case of topological-distinguishability, we only require that at least one of them has an open neighborhood that does not intersect the other.
\end{rmk}
\end{dfn}
\begin{exm}[Two points which are topologically-distinguishable but not separated]\label{exm4.5.3}
Define $X\coloneqq \{ x_1,x_2\}$ and
\begin{equation}
\mathcal{U}\coloneqq \left\{ \emptyset ,X,\{ x_1\} \right\} .
\end{equation}
Then, $x_1$ and $x_2$ are topologically-distinguishable as $\{ x_1\}$ is an open neighborhood of $x_1$ that does not contain $x_2$.  On the other hand, every neighborhood of $x_2$ contains $x_1$.
\end{exm}
\begin{dfn}[Separated by neighborhoods]\label{SeparatedByNeighborhoods}\index{Separated by neighborhoods}
$S_1$ and $S_2$ are \emph{separated by neighborhoods}\index{Separated by neighborhoods} iff there is a neighborhood $U_1$ of $S_1$ and a neighborhood $U_2$ of $S_2$ with $U_1$ and $U_2$ disjoint.
\begin{rmk}
Equivalently, we may replace $U_1$ and $U_2$ with open neighborhoods.
\end{rmk}
\begin{rmk}
This is just like being separated, except that we may put $U_1$ around $S_1$ and $U_2$ around $S_2$ \emph{`simultaneously'} and have no intersection, whereas in the separated case, the $U_1$ and $U_2$ that `work' will in general intersect.
\end{rmk}
\end{dfn}
\begin{exm}[Two points which are separated but not separated by neighborhoods]\label{exm4.5.8}
Define $X\coloneqq \{ x_1,x_2,x_3\}$ and
\begin{equation}
\mathcal{U}\coloneqq \left\{ \emptyset ,X,\{ x_1,x_3\} ,\{ x_2,x_3\} ,\{ x_3\} \right\} .
\end{equation}
Then, $\{ x_1,x_3\}$ is an open neighborhood of $x_1$ which does not contain $x_2$ and $\{ x_2,x_3\}$ is an open neighborhood of $x_2$ which does not contain $x_1$.  On the other hand, every open neighborhood of $x_1$ intersects every open neighborhood of $x_2$ at $x_3$.
\end{exm}
\begin{dfn}[Separated by closed neighborhoods]\label{SeparatedByClosedNeighborhoods}
$S_1$ and $S_2$ are \emph{separated by closed neighborhoods}\index{Separated by closed neighborhoods} iff there is a closed neighborhood $C_1$ of $S_1$ and a closed neighborhood $C_2$ of $S_2$ with $C_1$ and $C_2$ disjoint.
\begin{rmk}
This is the same as being separated by neighborhoods, except that we can further require that the neighborhoods are closed.\footnote{Recall that neighborhoods do not have to be open---see \cref{Neighborhood}.}
\end{rmk}
\end{dfn}
\begin{exm}[Two points which are separated by neighborhoods but not by closed neighborhoods]\label{exm4.5.11}
Define $X\coloneqq \{ x_1,x_2,x_3\}$ and
\begin{equation}
\mathcal{U}\coloneqq \left\{ \emptyset ,X,\{ x_1\} ,\{ x_3\} ,\{ x_1,x_3\} \right\} .
\end{equation}
Then, $\{ x_1\}$ is a neighborhood of $x_1$, $\{ x_3\}$ is a neighborhood of $x_3$, and these two neighborhoods are disjoint, so that $x_1$ and $x_3$ are separated by neighborhoods.  On the other hand, $x_1$ only has three neighborhoods:  $\{ x_1\}$, $\{ x_1,x_3\}$, and all of $X$.  The latter must intersect every neighborhood of $x_3$ and the former is not closed because $\{ x_1\} ^{\comp}=\{ x_2,x_3\}$ is not open.  Therefore, we cannot separate $x_1$ and $x_3$ with \emph{closed} neighborhoods.
\end{exm}
There is an equivalent, alternative way to think about being separated by closed neighborhoods that you may find useful.
\begin{prp}\label{prp4.5.13}
$S_1$ and $S_2$ are separated by closed neighborhoods iff they are separated by open neighoborhoods with disjoint closure.
\begin{proof}
$(\Rightarrow )$ Suppose that $S_1$ and $S_2$ are separated by closed neighborhoods $C_1$ and $C_2$ respectively.  By the definition of a neighborhood (\cref{Neighborhood}), there are then open neighborhoods $U_1$ and $U_2$ with $S_1\subseteq U_1\subseteq C_1$ and $S_2\subseteq U_2\subseteq C_2$.  Then, $\Cls (U_1)\subseteq C_1$ and $\Cls (U_2)\subseteq C_2$, and so $U_1$ and $U_2$ constitute open neighborhoods of $S_1$ and $S_2$ with disjoint closures.

\blankline
\noindent
$(\Leftarrow )$ Suppose that $S_1$ and $S_2$ are separated by open neighborhoods with disjoint closure.  Then, these closure constitute closed neighborhoods which separate $S_1$ and $S_2$.
\end{proof}
\end{prp}
\begin{dfn}[Completely-separated]\label{CompletelySeparated}
$S_1$ and $S_2$ are \emph{completely-separated}\index{Completely-separated} or \emph{separated by (continuous) functions}\index{Separated by continuous functions} iff there is a continuous function $f:X\rightarrow [0,1]$ such that $\restr{f}{S_1}=0$ and $\restr{f}{S_2}=1$.
\begin{rmk}
Why does being completely-separated imply being separated by closed neighborhoods?\footnote{All the other implications are true too (i.e.~separated implies topologically-distinguishable, separated by neighborhoods implies separated, etc.), this is just the first one that is not completely obvious, which is why it is the only one we have asked about.}
\end{rmk}
\end{dfn}
\begin{exm}[Two points which are separated by closed neighborhoods but not completely-separated]\label{ArensSquare}\footnote{This is significantly more nontrivial than the preceding counter-examples and comes from \cite[pg.~98]{Steen}.}
Define
\begin{equation}
S\coloneqq \left( (0,1)\times (0,1)\right) \cap (\Q \times \Q )=\left\{ \coord{x,y}\in (0,1)\times (0,1):x,y\in \Q \right\} ,
\end{equation}
\begin{equation}
T\coloneqq \{ \tfrac{1}{2}\} \times \{ r\sqrt{2}:r\in \Q \} =\{ \coord{\tfrac{1}{2},r\sqrt{2}}:r\in \Q \} 
\end{equation}
and
\begin{equation}
X\coloneqq S\cup T\cup \{ \coord{0,0})\} \cup \{ \coord{1,0})\}.
\end{equation}
We define a topology on $X$ by defining a neighborhood base at each point (see \cref{prp4.1.8}).  For $\coord{x,y}\in X$, there are four cases:  (i) $\coord{x,y}=\coord{0,0}$, (ii) $\coord{x,y}=\coord{1,0}$, (iii) $\coord{x,y}\in T$, and (iv) $\coord{x,y}\in S$.  We define
\begin{equation}
\mathcal{B}_{\coord{x,y}}\coloneqq \begin{cases}\left\{ U\subseteq S:U\text{ is open in }S\text{.}\footnote{Open in the usual topology (the subspace topology inherited from $(0,1)\times (0,1)$).}\right\} & \text{if }\coord{x,y}\in S \\ \left\{ U_{\coord{x,y}}^m:m\in \Z ^+\right\} & \text{if }\coord{x,y}\in T\cup \{ \coord{0,0}\} \cup \{ \coord{1,0}\} \end{cases},
\end{equation}
where
\begin{equation}\label{4.5.20}
\begin{split}
U_{\coord{0,0})}^m & \coloneqq \{ \coord{0,0}\} \cup \left\{ \coord{x,y}\in (0,\tfrac{1}{4})\times (0,\tfrac{1}{m}):x,y\in \Q \right\} \\
U_{\coord{1,0}}^m & \coloneqq \{ \coord{1,0}\} \cup \left\{ (x,y)\in (\tfrac{3}{4},1)\times (0,\tfrac{1}{m}):x,y\in \Q \right\} \\
U_{\coord{\tfrac{1}{2},r\sqrt{2}}}^m & \coloneqq \left\{ \coord{x,y}\in (\tfrac{1}{4},\tfrac{3}{4})\times (r\sqrt{2}-\tfrac{1}{m},r\sqrt{2}+\tfrac{1}{m}):x,y\in \Q \right\} .
\end{split}
\end{equation}
By \cref{prp4.1.8}, there is a unique topology for which $\mathcal{B}_{\coord{x,y}}$ is a neighborhood base of $\coord{x,y}\in X$.

The closures of $\{ \coord{0,0}\} \cup (0,\frac{1}{4})\times (0,\frac{1}{n})$ and $\{ \coord{1,0}\} \cup (\frac{3}{4},1)\times (0,\frac{1}{n})$ in $X$ must be disjoint as, in particular, any point in the former cannot have $x$-coordinate exceeding $\frac{1}{4}$ and any point in the latter cannot have $x$-coordinate strictly less than $\frac{3}{4}$.

On the other hand, $\coord{0,0}$ and $\coord{1,0}$ cannot be separated by a function.  To see this, suppose that $f:X\rightarrow \R$ were a continuous function such that $f(\coord{0,0})=0$ and $f(\coord{1,0})=1$.  Then, $f^{-1}([0,\frac{1}{4}))$ would be an open neighborhood of $\coord{0,0}$, and so must contain $U_{\coord{0,0}}^m$ for some $m\in \Z ^+$.  Similarly, $f^{-1}((\frac{3}{4},1])$ must contain $U_{\coord{1,0}}^n$ for some $n\in \Z ^+$.  Let $r\in \Q$ be such that $r\sqrt{2}<\min \{ \frac{1}{m},\frac{1}{n}\}$.  Obviously, $f(\coord{\frac{1}{2},r\sqrt{2}})$ cannot be in both $[0,\frac{1}{4})$ and $(\frac{3}{4},1]$ as these sets are disjoint, so without loss of generality assume that it is not contained in $[0,\frac{1}{4})$, so let $U \subseteq [0,1]$ be an open neighborhood of $f(\coord{\frac{1}{2},r\sqrt{2}})$ with $\Cls (S)$ disjoint from $\Cls \left( [0,\frac{1}{4})\right) $, so that the preimages of $\Cls (U)$ and $\Cls \left( [0,\frac{1}{4}\right)$ are disjoint closed neighborhoods of $\coord{\frac{1}{2},r\sqrt{2}}$ and $\coord{0,0}$ respectively.  On the other hand, a disjoint closed neighborhood of $\coord{\frac{1}{2},r\sqrt{2}}$ must contain $U_{\coord{\frac{1}{2},r\sqrt{2}}}^o$ for $o\in \Z ^+$ with $r\sqrt{2}-\frac{1}{o}>0$.  As $r<\frac{1}{m}$, we have that $\frac{1}{o}<\frac{\sqrt{2}}{m}$.  But then, $\Cls \left( U_{\coord{0,0}}^m\right)$ and $\Cls (U_{\coord{\frac{1}{2},r\sqrt{2}}}^o)$ must intersect at $\coord{\frac{1}{4},r\sqrt{2}-\frac{1}{o}}$ because $r\sqrt{2}-\frac{1}{o}<\frac{1}{m}-\frac{1}{o}<\frac{1}{m}$.
\begin{rmk}
This is the \emph{Arens Square}\index{Arens Square}.
\end{rmk}
\end{exm}
\begin{dfn}[Perfectly-separated]
$S_1$ and $S_2$ are \emph{perfectly-separated}\index{Perfectly-separated} or \emph{precisely-separated}\index{Precisely-separated} iff there is a continuous function $f:X\rightarrow [0,1]$ such that $S_1=f^{-1}(0)$ and $S_2=f^{-1}(1)$.
\begin{rmk}
Being completely-separated means that $f$ is $0$ on $S_1$ and $1$ on $S_2$.  Being perfectly-separated means that, furthermore, $f$ is $0$ \emph{nowhere else} except on $S_1$ and $f$ is $1$ \emph{nowhere else} except on $S_2$.
\end{rmk}
\end{dfn}
\begin{exm}[Two points which are completely-separated but not perfectly-separated]\footnote{This comes from \cite[pg.~52]{Steen}.}\label{UncountableFortSpace}
This example is fairly similar to the cocountable topology example---see \cref{exm4.2.8x}.

Define $X\coloneqq \R$.  Let $C\subseteq X$ and declare that
\begin{textequation}
$X$ is closed iff either (i) $C$ contains $0$ or (ii) $C$ is finite.
\end{textequation}
You can check for yourself that this satisfies the defining conditions for a topology in terms of closed sets (\cref{exr4.1.2}).

We wish to show that $0,1\in X$ are completely-separated, but not perfectly-separated.  We first check that they are completely-separated by producing a continuous function $f:X\rightarrow [0,1]$ such that $f(0)=0$ and $f(1)=1$.  Define
\begin{equation}
f(x)\coloneqq \begin{cases}1 & \text{if }x=1 \\ 0 & \text{otherwise}\end{cases}.
\end{equation}
We first check that $f$ is continuous.  Let $C\subseteq [0,1]$ be closed.  If $C$ contains $0\in [0,1]$, then $f^{-1}(C)$ contains $0\in X$, and so is closed.  If it does not contain $0$, then $f^{-1}(C)$ is finite---either $C$ contained $1$ in which case $f^{-1}(C)=\{ 1\}$ or it did not in which case $f^{-1}(C)=\emptyset$.

Now we show that $0,1\in X$ are not \emph{perfectly} separated.  Suppose that there exists a continuous function $f:X\rightarrow [0,1]$ such that $\{ 0\} =f^{-1}(0)$ and $\{ 1\} =f^{-1}(1)$.  Then,
\begin{equation}\label{4.5.23}
\{ 0\} =f^{-1}(0)=f^{-1}\left( \bigcap _{m\in \Z ^+}[0,\tfrac{1}{m})\right) =\footnote{\cref{exrA.1.30}\ref{enmA.1.30.ii}}\bigcap _{m\in \Z ^+}f^{-1}\left( [0,\tfrac{1}{m})\right) .
\end{equation}
That is, $\{ 0\}$ is a $G_\delta$ set.\footnote{Recall that this is just a fancy-shmancy term for a set which is a countable intersection of open sets---see \cref{GDeltaFSigma}.}  However, by definition, a set is open iff it does not contain $0\in X$ or its complement is finite.  Of course, all the sets appearing in \eqref{4.5.23} must be of the latter kind.  However, taking the complement of this equation, we find
\begin{equation}
\R \setminus \{ 0\} ^{-1}\footnote{De Morgan's Laws---see \cref{DeMorgansLaws}}=\bigcup _{m\in \Z ^+}f^{-1}(0,\tfrac{1}{m})^\comp ,
\end{equation}
so that $\R \setminus \{ 0\}$ is a countably-infinite union of finite sets---a contradiction.
\begin{rmk}
This is the \emph{Uncountable Fort Space}\index{Uncountable Fort Space}.
\end{rmk}
\end{exm}

Note that we obviously have the implications
\begin{textequation}
perfectly-separated $\Rightarrow$ completely-separated $\Rightarrow $ separated by closed neighborhoods $\Rightarrow $ separated by neighborhoods $\Rightarrow$ separated $\Rightarrow$ topologically-distinguishable $\Rightarrow$ distinct.
\end{textequation}
Here, ``distinct'' literally means that $S_1$ and $S_2$ are not the same thing, i.e. $S_1\neq S_2$.  Furthermore, we have presented counter-examples after each definition to show that each implication is strict.

\subsection{Separation axioms of spaces}

In the previous subsection, we defined several levels of ``separation'' between two disjoints subsets of a topological space.  We now use these definitions to put conditions on topological spaces themselves.

Throughout this section, let $X$ be a topological space.
\begin{dfn}[$T_0$]\label{T0}
$X$ is \emph{$T_0$}\index{$T_0$} iff any two distinct points are topologically-distinguishable.
\begin{rmk}
Sometimes this condition is called \emph{kolmogorov}\index{Kolmogorov (topological space)}.  You will find that a lot (if not all) of the separation axioms of spaces have other names.  We have chosen the names we have because (i) other terminology is less consistent and (ii) it carries less information (of course that the subscript $0$ in $T_0$ has some significance).
\end{rmk}
\begin{rmk}
This is an insanely reasonable condition.  I might even argue that if you have a space you're trying to study that is not $T_0$, you're doing something wrong.  If the topology cannot distinguish between two points, either (i) you may as well identify those two points (see \cref{T0Quotient}) or (ii) you should probably consider adding more structure to your space that does distinguish between them.
\end{rmk}
\end{dfn}
\begin{exm}[A space which is not $T_0$]
Any indiscrete space with at least two points.  The example above in \cref{exm4.5.2} worked.
\end{exm}
\begin{prp}[$T_0$ quotient]\label{T0Quotient}
Let $X$ be a topological space.  Then, there exists a unique topological space $\TZero (X)$, the \emph{$T_0$ quotient}\index{$T_0$ quotient} of $X$, and a surjective map $\q :X\rightarrow \TZero (X)$ such that
\begin{enumerate}
\item \label{T0Quotient.i}$\TZero (X)$ is $T_0$; and
\item \label{T0Quotient.ii}if $Y$ is another $T_0$ space with a continuous map $\phi :X\rightarrow Y$, then there is a unique continuous map $\phi ':\TZero (X)\rightarrow Y$ such that $\phi =\phi '\circ \q$.
\end{enumerate}
\begin{rmk}
As $T_0$ is sometimes called kolmogorov, so to this is sometimes called the \emph{kolmogorov quotient}\index{Kolmogorov quotient}.
\end{rmk}
\begin{rmk}
Compare this with the definitions of the integers, rationals, closure, interior, generating collections, initial topology, and final topology (\cref{Integers,RationalNumbers,Closure,Interior,InitialTopology,FinalTopology}).  Note how this is a bit different---the key difference here is that the map from $X$ to $\TZero (X)$ is now \emph{surjective} (i.e.~a quotient map) instead of in all the previous cases where it was \emph{injective} (i.e.~an inclusion).
\end{rmk}
\begin{proof}
Define $x_1\sim x_2$ iff the open sets which contain $x_1$ are precisely the same as the open sets which contain $x_2$.
\begin{exr}
Show that $\sim$ is an equivalence relation.
\end{exr}
Define $\TZero (X)\coloneqq X/\sim$ and let $\q :X\rightarrow \TZero (X)$.
\begin{exr}
Show that $\TZero (X)$ satisfies \ref{T0Quotient.i} and \ref{T0Quotient.ii}.
\end{exr}
\begin{exr}
Show that $\TZero (X)$ is unique.
\end{exr}
\end{proof}
\end{prp}
\begin{dfn}[$T_1$]\label{T1}
$X$ is \emph{$T_1$}\index{$T_1$} iff any two distinct points are separated.
\begin{rmk}
Sometimes this condition is called \emph{accessible}\index{Accessible (topological space)} of \emph{fr\'{e}chet}\index{Fr\'{e}chet (topological space)}.  I also prefer $T_1$ over ``accessible'' because, not only does it carry slightly more information, but it's also a lot more common.  I would \emph{definitely} recommend not to use the term ``Fr\'{e}chet'' to describe this, as the term ``fr\'{e}ceht space'' is usually meant to describe something else entirely.
\end{rmk}
\begin{rmk}
Warning:  This is not the same as ``any two topologically-distinguishable points are separated''.  That condition is called $R_0$ and is rarely, if ever, used.\footnote{There is no difference between $R_0$ and $T_1$ for spaces which are $T_0$.}
\end{rmk}
\begin{rmk}
In contrast to the $T_0$ condition, there are \emph{incredibly} important examples of spaces that are not $T_1$.
\end{rmk}
\end{dfn}
\begin{exm}[A space that is $T_0$ but not $T_1$]
The example above in \cref{exm4.5.3} worked.
\end{exm}
While the above is probably the best way to state $T_1$ as a definition because it makes its similarity with other separation conditions more apparent, it is often best to think of $T_1$ spaces are spaces in which points are closed.
\begin{prp}\label{prp4.5.32}
Let $X$ be a topological space.  Then, $X$ is $T_1$ iff $\{ x\}$ is closed for all $x\in X$.
\begin{proof}
$(\Rightarrow )$ Suppose that $X$ is $T_1$.  Let $x\in X$.  For all $y\in X$ distinct from $x$, let $U_y$ be an open neighborhood of $y$ that does not contain $x$.  Then,
\begin{equation}
\bigcup _{y\neq x}U_y=X\setminus \{ x\}
\end{equation}
is open, and so
\begin{equation}
\bigcap _{y\neq x}U_y^{\comp}=\{ x\}
\end{equation}
is closed.

\blankline
\noindent
$(\Leftarrow )$ Suppose that $\{ x\}$ is closed for all $x\in X$.  Let $x_1,x_2\in X$.  Then, $\{ x_1\} ^{\comp}$ is an open neighborhood of $x_2$ that does not contain $x_1$ and $\{ x_2\} ^{\comp}$ is an open neighborhood of $x_1$ that does not contain $x_2$.
\end{proof}
\end{prp}
\begin{crl}
Every finite $T_1$ topological space is discrete.
\begin{rmk}
In particular, it is a waste of time to look for counter-examples in finite spaces if your space is $T_1$.
\end{rmk}
\begin{proof}
If the space is $T_1$, then every point is closed.  If the space is finite, then every subset is a union of finitely many points, that is, a finite union of closed sets, and hence closed.
\end{proof}
\end{crl}
\begin{dfn}[$T_2$]\label{T2}
$X$ is \emph{$T_2$}\index{$T_2$} iff any two distinct points are separated by neighborhoods.
\begin{rmk}
This is quite often referred to as \emph{hausdorff}\index{Hausdorff (topological space)}.  In constrast to most of the alternative terminologies, this actually might be more common than the term ``$T_2$''.
\end{rmk}
\end{dfn}
\begin{dfn}[Compact]\label{Compact}
$X$ is \emph{compact}\index{Compact} iff it is quasicompact and $T_2$.
\end{dfn}
\begin{prp}\label{prp4.5.37}
Let $X$ be a topological space.  Then, $X$ is $T_2$ iff limits are unique.
\begin{rmk}
Because of lack of uniqueness in general, we have been hesitant to write $\lim _\lambda x_\lambda$---if limits are not unique, then what limit does this symbol refer to?  Hereafter, however, in $T_2$ spaces, we will not hesitate to use this notation.
\end{rmk}
\begin{proof}
$(\Rightarrow )$ Suppose that $X$ is $T_2$.  Let $\lambda \mapsto x_\lambda \in X$ be a net and let $x_\infty ,x_\infty '\in X$ be limits of $X$.  We proceed by contradiction:  suppose that $x_\infty \neq x_\infty '$.  Then, there exist disjoint open neighborhoods $U$ of $x_\infty$ and $U'$ of $x_\infty '$.  As $\lambda \mapsto x_\lambda$ converges to $x_\infty$, it is eventually contained in $U$.  But then, as $U$ and $U'$ are disjoint, $\lambda \mapsto x_\lambda$ is not eventually contained in $U'$---a contradiction.

\blankline
\noindent
$(\Leftarrow )$ Suppose that limits are unique.  Let $x_\infty ,x_\infty '\in X$ be distinct.  We wish to show that there exist disjoint open neighborhoods of $x_\infty$ and $x_\infty '$.  We proceed by contradiction:  suppose that every open neighborhood of $x_\infty$ intersects every open neighborhood of $x_\infty '$.  Let $\Lambda$ be the collection of these intersections, that is, the collection of sets of the form $U\cap U'$ for $U$ and $U'$ open neighborhoods of $x_\infty$ and $x_\infty '$ respectively.  Order $\Lambda$ by reverse-inclusion so as to form a directed set, and for each $U\cap U'\in \Lambda$, choose $x_{U\cap U'}\in U\cap U'$.  Then, $\Lambda \ni U\cap U'\mapsto x_{U\cap U'}$ converges to both $x_\infty$ and $x_\infty'$---a contradiction.
\end{proof}
\end{prp}
\begin{exr}\label{exr4.6.37}
Show that a subspace of a $T_2$ space is $T_2$.
\end{exr}
\begin{exr}\label{exr4.6.38}
Show that an arbitrary product of $T_2$ spaces is $T_2$.
\begin{rmk}
Thus, by \nameref{TychnoffsTheorem} (\cref{TychnoffsTheorem}), an arbitrary product of compact spaces is compact.
\end{rmk}
\end{exr}
\begin{exr}\label{exr4.6.39}
Show that quasicompact subsets (which are in fact compact by \cref{exr4.6.37}) of a $T_2$ space can be separated by neighborhoods.
\end{exr}
Hopefully you found an example in \cref{exr3.1.34} of a continuous bijective function that was not a homeomorphism.  In certain special cases, however, you can immediately make this deduction.
\begin{exr}\label{exr3.6.46}
Show that continuous injective function from a quasicompact space into a $T_2$ space is a homeomorphism onto its image.
\end{exr}
\begin{exm}[A space that is $T_1$ but not $T_2$]
Let $X$ be as in \cref{exm4.2.8x}, that is the real numbers with the cocountable topology.  By definition, countable sets are closed (along with all of $X$ of course), and so certainly points are closed, and so $X$ is $T_1$.  On the other hand, any open neighborhood of $0$, must intersect any open neighborhood of $1$,  because both of these neighborhoods have countable complements and so, by the uncountability of $\R$, must intersect.
\end{exm}

Here is where the consistency of the terminology breaks-down.  If you guessed that the term $T_3$ refers to spaces in which any two distinct points are separated by closed neighborhoods, you'd be wrong.  Unfortunately, it seems that the term $T_3$ was already taken (we'll see what it means in a bit) when someone went to write this definition down, and so it is called $T_{2\frac{1}{2}}$.
\begin{dfn}[$T_{2\frac{1}{2}}$]\label{T212}
$X$ is \emph{$T_{2\frac{1}{2}}$}\index{$T_{2\frac{1}{2}}$} iff any two distinct points are separated by closed neighborhoods.
\begin{rmk}
The alternate terminology for this seems to be \emph{urysohn}\index{Urysohn}.
\end{rmk}
\end{dfn}
\begin{exm}[A space that is $T_2$ but not $T_{2\frac{1}{2}}$]\footnote{This comes from \cite[pg.~100]{Steen}.}\label{SimplifiedArensSquare}
This example is very similar to the Arens Square---see \cref{ArensSquare}.

Define
\begin{equation}
S\coloneqq ((0,1)\times (0,1))\cap (\Q \times \Q )=\left\{ \coord{x,y}\in (0,1)\times (0,1):x,y\in \Q \right\} 
\end{equation}
and
\begin{equation}
X\coloneqq S\cup \{ \coord{0,0}\} \cup \{ \coord{1,0}\} .
\end{equation}
(This is the rationals in the open unit square together with the bottom-left and bottom-right corners.)  We define a topology on $X$ by defining a neighborhood base at each point (see \cref{prp4.1.8}).  For $(x,y)\in X$, there are three cases:  (i) $\coord{x,y}=\coord{0,0}$, (ii) $\coord{x,y}=\coord{1,0}$, and (iii) $\coord{x,y}\in S$.  We define
\begin{equation}
\mathcal{B}_{\coord{x,y}}\coloneqq \begin{cases}\left\{ U\subseteq S:U\text{ is open in }S\text{.}\footnote{Open in the usual topology (the subspace topology inherited from $(0,1)\times (0,1)$).}\right\} & \text{if }\coord{x,y}\in S \\ \left\{ U_{\coord{x,y}}^m:m\in \Z ^+\right\} & \text{otherwise}\end{cases},
\end{equation}
where
\begin{equation}
\begin{split}
U_{\coord{0,0}}^m & \coloneqq \{ \coord{0,0}\} \cup \left\{ \coord{x,y}\in (0,\tfrac{1}{2})\times (0,\tfrac{1}{m}):x,y\in \Q \right\} \\
U_{\coord{1,0}}^m & \coloneqq \{ \coord{1,0}\} \cup \left\{ \coord{x,y}\in (\tfrac{1}{2},1)\times (0,\tfrac{1}{m}):x,y \in \Q \right\} .
\end{split}
\end{equation}
By \cref{prp4.1.8}, there is a unique topology for which $\mathcal{B}_{\coord{x,y}}$ is a neighborhood base of $\coord{x,y}\in X$.

The closure of any open neighborhood of $\coord{0,0}$ contains points with $x$-coordinate $\frac{1}{2}$ and $y$-coordinate arbitrarily small, and the same goes from any open neighborhood of $(1,0)$.  Thus, these two points are not separated by closed neighborhoods.\footnote{Recall that two points are separated by closed neighborhoods iff they are separated by open neighborhoods with disjoint closure---see \cref{prp4.5.13}.}

On the other hand, the $x$-coordinate of every point in every open neighborhood of $\coord{0,0}$ is strictly less than $\frac{1}{2}$, and similarly the $x$-coordinate of every point in every open neighborhood of $\coord{1,0}$ is strictly greater than $\frac{1}{2}$.  In particular, these two points are separated by neighborhoods.

For any point $\coord{x,y}$ distinct from $\coord{0,0}$ and $\coord{1,0}$, we can separate $\coord{x,y}$ from $\coord{0,0}$ with neighborhoods by taking $U_{\coord{0,0}}^m\ni \coord{0,0}$ for $m\in \Z ^+$ with $\frac{1}{m}<y$ and any $\varepsilon$-ball about $\coord{x,y}$ of with radius less than $y-\frac{1}{m}$.  Similarly for $\coord{x,y}$ and $\coord{1,0}$.  Of course, any two points distinct from both $\coord{0,0}$ and $\coord{1,0}$ are separated by neighborhoods because they can be separated by neighborhoods in $S$.  Thus, $X$ is indeed $T_2$.
\begin{rmk}
This is the \emph{Simplified Arens Square}\index{Simplified Arens Square}.
\end{rmk}
\end{exm}
You might be thinking to yourself ``Well, if $T_3$ wasn't separated by closed neighborhoods, then it must be completely-separated, right?''.  Sorry.  Wrong again.
\begin{dfn}[Completely-$T_2$]\label{CompletelyT2}
$X$ is \emph{completely-$T_2$}\index{Completely-$T_2$} iff any two distinct points are completely-separated.
\begin{rmk}
Naturally, this is sometimes called \emph{completely-hausdorff}\index{Completely-hausdorff}.
\end{rmk}
\begin{rmk}
Sometimes people will also say that \emph{continuous functions separate points}.
\end{rmk}
\end{dfn}
\begin{exm}[A space that is $T_{2\frac{1}{2}}$ but not completely-$T_2$]
The Arens Square from \cref{ArensSquare} will do the trick.  There, we provided an example of two points which are separated by closed neighborhoods, but not completely-separated.  This is of course already enough to show that the Arens Square is not completely-$T_2$, but we still need to check that \emph{any} two points can be separated by closed neighborhoods.

Denote the Arens Square by $X$ and let $\coord{\frac{1}{2},r\sqrt{2}}\in X$ for $r\in \Q$.  Take $m\in \Z ^+$ such that $r\sqrt{2}-\frac{1}{m}>0$ and take $n\in \Z ^+$ such that $\frac{1}{n}<r\sqrt{2}-\frac{1}{m}$.  Then, using the definition \eqref{4.5.20}, we see that the closures of $U_{(1,0)}^m$ and $U_{\coord{\frac{1}{2},r\sqrt{2}}}$ are disjoint.  Similarly, a point of this form can be separated from $(1,0)$ by closed neighborhoods.

For $\coord{x,y}\in S$, just take $m\in \Z ^+$ with $\frac{1}{m}<y$.  Then, the closure of $U_{\coord{0,0}}^m$ and any $\varepsilon$-ball around $\coord{x,y}$ with radius less than $y-\frac{1}{m}$ will be disjoint.  A similar trick works for $\coord{1,0}$ of course.

For $r,q\in \Q$ with $r<q$, choose $m,n\in \Z ^+$ so that $r\sqrt{2}+\frac{1}{m}<q\sqrt{2}-\frac{1}{n}$ (and so that $r\sqrt{2}-\frac{1}{m}>0$ and $q\sqrt{2}+\frac{1}{n}<1$ of course so that the open neighborhoods are actually contained in $X$).  Then, the closures of $U_{\coord{\frac{1}{2},r\sqrt{2}}}^m$ and $U_{\coord{\frac{1}{2},q\sqrt{2}}}^n$ are disjoint.

For $\coord{x,y}\in S$ and $\coord{\frac{1}{2},r\sqrt{2}}\in T$, we must have that $y\neq r\sqrt{2}$, and so, without loss of generality, that $y<r\sqrt{2}$.  Then, choose $m\in \Z ^+$ large enough so that $r\sqrt{2}-\frac{1}{m}>y$ (and so that $U_{\coord{\frac{1}{2},\sqrt{2}}}^m$ is contained in $X$), and take a $\varepsilon$-ball about $\coord{x,y}$ with radius less than $(r\sqrt{2}-\frac{1}{m})-y$.  Then, the closure of these two neighborhoods will be disjoint.

Finally, can separate two points in $S$ by closed neighborhoods because we could do so in $(0,1)\times (0,1)$ (recall that $S$ just as the subspace topology).
\end{exm}
And of course, as now you probably saw coming, $T_3$ does not mean distinct points can be perfectly-separated.
\begin{dfn}[Perfectly-$T_2$]\label{PerfectlyT2}
$X$ is \emph{perfectly-$T_2$}\index{Perfectly-$T_2$} iff any two distinct points can be perfectly-separated.
\begin{rmk}
Disclaimer:  I have never seen this term before.  Then again, I've never seen \emph{any} term to describe such spaces.  But honestly, if completely-$T_2$ means you can completely-separate points, then a space in which you can perfectly-separate points is going to be called\textellipsis
\end{rmk}
\end{dfn}
\begin{exm}[A space that is completely-$T_2$ but not perfectly-$T_2$]\label{exm4.5.48}
The Uncountable Fort Space from \cref{UncountableFortSpace} will do the trick.  There, we provided an example of two points which are completely-separated, but not perfectly-separated.  This is of course already enough to show that the Uncountable Fort Space is not perfectly-$T_2$, but we still need to check that \emph{any} two points can be completely-separated.

Recall that the Uncountable Fort Space was defined to be $X\coloneqq \R$ with the closed sets being precisely the finite sets and also the sets which contained $0$.  We showed above in \cref{UncountableFortSpace} that $1\in X$ can be completely-separated from $0\in X$.  Of course, there was nothing special about $1$, and so all we need to do is to show that we can completely-separate any two nonzero points in $X$.

So, let $x_1,x_2\in X$ be nonzero and distinct, and define $f:X\rightarrow [0,1]$ by
\begin{equation}
f(x)\coloneqq \begin{cases}0 & \text{if }x=x_1 \\ 1 & \text{if }x=x_2 \\ \tfrac{1}{2} & \text{otherwise}\end{cases}.
\end{equation}
Then, if $C\subseteq [0,1]$ closed contains $\frac{1}{2}$, $f^{-1}(C)$ contains $0\in X$, and is hence closed.  Otherwise, it is finite, as $f^{-1}\left( [0,1]\setminus \{ \frac{1}{2}\} \right) =\{ x_1,x_2\}$, and hence closed.  Thus, $f$ is continuous, and so completely-separates $x_1$ and $x_2$.
\end{exm}

But now we've gone through all the separation properties, right?  How could $T_3$ be a thing?  Well, now we enter a collection of separation axioms of a different nature---now we will focus on separating \emph{closed sets}.  One unfortunate fact, however, is that, if points are not closed, then these new separation axioms are \emph{not} strictly stronger than the separation axioms we just presented.  There are thus two versions of the following separation axioms:  ones in which the points are closed and ones in which they aren't.
\begin{dfn}[Regular]\label{Regular}
$X$ is \emph{regular}\index{Regular (topological space)} iff any closed set and a point not contained in it can be separated by neighborhoods.
\begin{rmk}
You'll note that we skipped right over being topologically-distinguishable and just being separated.  This is because a point is automatically separated from a closed set---the complement of the closed set is an open neighborhood of the point which does not intersect the closed set.
\end{rmk}
\begin{rmk}
Unfortunately, the terminology of separation axioms is so fucked that there's really no way of being consistent with all of the literature.  There are sources which reverse my conventions of regular and $T_3$.  We explain the motivation of our choice of convention in the definition of $T_3$ (\cref{T3}).
\end{rmk}
\end{dfn}
\begin{exm}[A space that is $T_0$ regular but not $T_1$]
The indiscrete topology on any set with at least two points is vacuously regular but not $T_0$.
\end{exm}
Stupid examples like this is why we almost always care about the case when we in addition impose the condition of being $T_1$.  This finally brings us to the definition of $T_3$.
\begin{dfn}[$T_3$]\label{T3}
$X$ is $T_3$ iff it is $T_1$ and regular.
\begin{rmk}
The $T_1$ condition (points are closed) is added so that this is a strict specifization of being $T_2$.  In fact, by the following proposition, we would have equally well said $T_0$.
\end{rmk}
\begin{rmk}
We choose to call this $T_3$ instead of regular because then we have $T_3\rightarrow T_2$.  If the terms were reversed, then there would be $T_3$ spaces (like indiscrete spaces) that were not $T_2$---ew.
\end{rmk}
\begin{rmk}
Warning:  Up until now, all of the $T_k$ axioms (including things like $T_{2\frac{1}{2}}$, completely-$T_2$, and perfectly-$T_2$) have been strictly comparable, that is, $T_1$ strictly implies $T_0$, $T_2$ strictly implies $T_1$, etc..  With the introduction of $T_3$, this is no longer the case.  It turns out that $T_3$ implies $T_{2\frac{1}{2}}$, but that $T_3$ is not comparable with either completely-$T_2$ or perfectly-$T_2$.
\end{rmk}
\end{dfn}
\begin{prp}\label{prp4.6.53}
If $X$ is $T_0$ and regular, then it is $T_2$.
\begin{proof}
Suppose that $X$ is $T_0$ and regular.  Let $x_1,x_2\in X$.  Because $X$ is $T_0$, without loss of generality there is some open neighborhood $U$ of $x_1$ which does not contain $x_2$.  Then, $U^{\comp}$ is closed and does not contain $x_1$, so because $X$ is regular, there are disjoint open neighborhoods $V_1$ and $V_2$ of $x_1$ and $U^{\comp}$ respectively.  Then, as $x_2\in U^{\comp}$, this implies that $x_1$ and $x_2$ are separated by neighborhoods.
\end{proof}
\end{prp}
\begin{prp}
If $X$ is $T_3$, then $X$ is $T_{2\frac{1}{2}}$.
\begin{proof}
Suppose that $X$ is $T_3$.  Let $x_1,x_2\in X$ be distinct.  As $X$ is $T_3$, it is definitely $T_2$, and so we can separate $x_1$ and $x_2$ with open neighborhoods $U_1$ and $U_2$ containing $x_1$ and $x_2$ respectively.  Then, $U_1^{\comp}$ is a closed set not containing $x_1$, and so because $X$ is $T_3$, there is an open neighborhood $V_1$ of $x_1$ and an open neighborhood $V_2$ of $U_1^{\comp}$ which are disjoint.  Then,
\begin{equation}
\Cls (U_2)\subseteq U_1^{\comp}\subseteq V_2\text{ and }\Cls (V_1)\subseteq V_2^{\comp},
\end{equation}
so that $\Cls (U_2)$ and $\Cls (V_1)$ are disjoint, so that $X$ is $T_{2\frac{1}{2}}$ by \cref{prp4.5.13}.
\end{proof}
\end{prp}
\begin{exm}[A space that is perfectly-$T_2$ but not $T_3$]\label{CocountableExtensionTopology}
Define $X\coloneqq \R$.  We equip $\R$ with the so-called \emph{cocountable extension topology}\index{Cocountable extension topology}.  Let $C\subseteq X$ and declare that
\begin{textequation}
$C$ is closed iff it is the union of a countable set and a set closed in the usual topology on $\R$.
\end{textequation}

Now, let $x_1,x_2\in X$ be distinct, and let $\phi :\R \rightarrow (0,1)$ be any homeomorphism (in the usual topology).  Now define $f:X\rightarrow [0,1]$ by
\begin{equation}
f(x)\coloneqq \begin{cases}0 & \text{if }x=x_1 \\ 1 & \text{if }x=x_2 \\ \phi (x) & \text{otherwise}\end{cases}.
\end{equation}
The preimage of any set in $[0,1]$ which contains neither $0$ nor $1$ will be closed because $\phi$ is a homeomorphism.  If it does contain $0$ or $1$, then the preimage will be the union of a closed set in the usual topology on $\R$ and a finite set, and hence will be closed.  Thus, $f$ is continuous, and so $X$ is perfectly-$T_2$.  We know check that $X$ is not $T_3$.

$\Q \subseteq X$ is closed because it is countable.  Any open neighborhood of $\Q$ must be of the form of a usual open neighborhood of $\Q$ with countably many points removed.  Of course, the only usual open neighborhood of $\Q$ is all of $\R$, and so every open neighborhood of $\Q$ is just $\R$ with countably many points removed.  On the other hand, $\sqrt{2}\notin \Q$ and any open neighborhood of $\sqrt{2}$ is a usual open neighborhood with countably many points removed.  Thus, by uncountability of $\R$, these two must intersection, and so you cannot separate $\Q$ and $\sqrt{2}$ with neighborhoods.  Thus, $X$ is not $T_3$.
\end{exm}
\begin{exm}[A space that is $T_3$ but not completely-$T_2$]\footnote{Evidently, this example originally comes from \cite{Thomas}, though I first heard about it from Brian M.~Scott on \href{http://math.stackexchange.com/questions/386742/making-tychonoff-corkscrew-in-counterexamples-in-topology-rigorous}{math.stackexchange}.}\label{ThomasTentSpace}
For $m\in 2\Z$, define
\begin{equation}
L_m\coloneqq \{ m\} \times [0,\tfrac{1}{2}).
\end{equation}
For $n\in 1+2\Z$ and $k\in \Z$ with $k\geq 2$, let $T_{n,k}$ be the \emph{legs} of an isosceles triangle in $\R ^2\times \R ^2$ with apex at $p_{n,k}\coloneqq \coord{n,1-\frac{1}{k}}$ and base $[n-(1-\frac{1}{k}),n+(1+\frac{1}{k})]\times \{ 0\}$ (including the end-points of the legs on the base but not the apex $p_{n,k}$).  Then, define
\begin{equation}
\begin{split}
Y_0 & \coloneqq \bigcup _{m\in 2\Z}L_m \\
Y_1 & \coloneqq \bigcup _{n\in 1+2\Z ;\ k\in \Z ,k\geq 2}\{ p_{n,k}\} \\
Y_2 & \coloneqq \bigcup _{n\in 1+2\Z ;\ k\in \Z ,k\geq 2}T_{n,k} \\
Y & \coloneqq Y_0\cup Y_1\cup Y_2.
\end{split}
\end{equation}
We define a topology on $Y$ by defining a neighborhood base at each point---see \cref{prp4.1.8}.  For $\coord{x,y}\in Y_2$, we declare a neighborhood base to be simply $\{ \coord{x,y}\}$.  For $\coord{x,y}=p_{n,k}$, we declare a neighborhood base of $\coord{x,y}$ to consist of subsets of the form $\{ p_{n,k}\} \cup S$ for $S\subseteq T_{n,k}$ cofinite in $T_{n,k}$.  For $\coord{x,y}\in L_m$ for $m\in 2\Z$, we declare a neighborhood base of $\coord{x,y}$ to consist of subsets of the form $\{ \coord{m,y}\}\cup S$ for $S\subseteq Y\cap \left( (m-1,m+1)\times \{ y\}\right)$ cofinite in $Y\cap \left( (m-1,m+1)\times \{ y\}\right)$.  It follows from \cref{prp4.1.8} that there is a unique topology for which $\mathcal{B}_{\coord{x,y}}$ is a neighborhood base of $\coord{x,y}\in Y$. 

Finally define
\begin{equation}
X\coloneqq Y\sqcup \{ p_1,p_2\}
\end{equation}
for distinct new points $p_1,p_2$.  To define a topology on $X$, we once again use \cref{prp4.1.8}.  A neighborhood base at $\coord{x,y}\in Y$ consists of precisely the same sets as it did before.  We furthermore declare that
\begin{equation}
\begin{split}
\mathcal{B}_{p_1} & \coloneqq \left\{ U_{p_1}^\alpha :\alpha \in \R \right\} \\
\mathcal{B}_{p_2} & \coloneqq \left\{ U_{p_2}^\alpha :\alpha \in \R \right\}
\end{split}
\end{equation}
to be neighborhood bases of $p_1$ and $p_2$ respectively, where
\begin{equation}
\begin{split}
U_{p_1}^\alpha & \coloneqq \left\{ \coord{x,y}\in Y:x<\alpha \right\} \\
U_{p_2}^\alpha & \coloneqq \left\{ \coord{x,y}\in Y:x>\alpha \right\} .
\end{split}
\end{equation}

We first check that $X$ is not completely-$T_2$ by showing that no continuous function can separate $p_1$ from $p_2$.

First suppose that $f(p_{n,k})=\alpha$.  Then,
\begin{equation}
f^{-1}(\alpha )\cap T_{n,k}=f^{-1}\left( \bigcap _{m\in \Z ^+}(\alpha -\tfrac{1}{m},\alpha +\tfrac{1}{m})\right) \cap T_{n,k}=\bigcap _{m\in \Z ^+}S_m,
\end{equation}
where
\begin{equation}
S_m\coloneqq f^{-1}((\alpha -\tfrac{1}{m},\alpha +\tfrac{1}{m}))\cap T_{n,k}
\end{equation}
is cofinite in $T_{n,k}$.  Hence,
\begin{equation}
T_{n,k}\setminus \left( f^{-1}(\alpha )\cap T_{n,k}\right) =\bigcup _{m\in \Z ^+}T_{n,k}\setminus S_m.
\end{equation}
In other words, the number of points in $T_{n,k}$ that map to a different value than $f(p_{n,k})$ is at most countable.  Less precisely, $f$ is constant on $T_{n,k}$ modulo a countable set.

Let $C_{n,k}$ denote the set of $y$-values in $[0,\tfrac{1}{k})$ for which there is some point in $T_{n,k}$ with that $y$-coordinate and that maps to $f(p_{n,k})$.  We just showed that this set is cocountable in $[0,1-\frac{1}{k})$, and hence it is certainly cocountable in $[0,\frac{1}{2})$.  Then, the intersection of them $\bigcap _{k\in \Z ,k\geq 2}C_{n,k}$ must in turn then be cocountable in $[0,\frac{1}{2})$, and in particular, be nonempty.  So, let $y_m\in L_m$ be such that there is some $q_{m-1,k}\in T_{m-1,k}$ and some $q_{m+1,k}\in T_{m+1,k}$ with $f(q_{m-1,k})=f(p_{m-1,k})$ and $f(q_{m+1,k})=f(p_{m+1,k})$.  Because the open neighborhoods of $y_m$ are cofinite in $((m-1,m+1)\times \{ y\} )\cap Y$, the sequence $k\mapsto q_{m-1,k}$ must eventually be in every neighborhood of $y_m$, and so we have $\lim _kq_{m-1,k}=y_m=\lim _kq_{m+1,k}$.  It follows that
\begin{equation}
f(y_m)=\lim _kf(q_{m+1,k})=\lim _kf(p_{m+1,k})=\lim _kf(q_{m+2,k})=f(y_{m+2}).
\end{equation}
The crux:  for each $m\in 2\Z$, there exists points $y_m\in L_m$ and $y_{m+2}\in L_{m+2}$ with $f(y_m)=f(y_{m+2})$.  By the definition of the open neighborhoods of $p_1$ and $p_2$, it follows that $f(p_1)=\lim _{m\to -\infty}f(y_m)=\lim _{m\to +\infty}f(y_m)=f(p_2)$, so that $p_1$ and $p_2$ are not completely-separated, and so $X$ is not completely-$T_2$.

We now check that $X$ is $T_3$.  We first check that it is $T_1$.  $p_1$ is closed because
\begin{equation}
\{ p_1\} =\bigcap _{\alpha \in \R}(U_{p_2}^\alpha )^{\comp}.
\end{equation}
Similarly for $p_2$.  Each point $\coord{m,y}\in L_m$ is closed because the union of all open sets not contained in $\mathcal{B}_{\coord{m,y}}$, $\mathcal{B}_{p_1}$, or $\mathcal{B}_{p_2}$ is precisely the complement of $\coord{m,y}$.  Similarly for the $p_{n,k}$s.  For $\coord{x,y}\in T_{n,k}$, the intersection over $\{ \coord{x',y'}\} ^{\comp}$ for $\coord{x',y'}\neq \coord{x,y}$ in $T_{n,k}$ is $\{ \coord{x,y}\}$ and so will be closed if $T_{n,k}$ is.  Each $T_{n,k}$ is open, however, (being the union of its points, which are open), and so $T_{n,k}$ is the complement of the union over $T_{n',k'}$ with $n'\neq n$ and $k'\neq k$ with $\{ p_1,p_2\}$ removed.

Now we just need to show that we can separate closed sets from points with open neighborhoods.  One thing to note is that we need only separate closed sets that are complements of some element of some $\mathcal{B}_p$ from points, because every closed set is an intersection of sets of this form.  So, let $p\in X$ and let $C\subseteq X$ be closed not containing $p$.  There is nothing to do but break it down into cases.

First suppose that $p=p_1$ and $C\subseteq Y\cup \{ p_2\}$.  If $C$ contained points of arbitrarily small $x$-coordinate, then because it is closed, it would have to contain $p_1$.  Thus, there is some $x_0\in \R$ such that the $x$-coordinate of every point (besides $p_2$ of course) is at least $x_0$.  Then we can take $U_{p_1}^{\frac{x_0}{8}}$ as our open neighborhood of $p_1$ and $U_{p_2}^{\frac{x_0}{4}}$ as our open neighborhood of $C$.  Similarly if our point is $p_2$.

We can easily separate points of $T_{n,k}$ from closed sets as these points themselves are open.

Now take $p=\coord{m,y}\in L_m$ and $C\subseteq X$ not containing $p$.  If $C$ intersects $((m-1,m+1)\times \{ y\} )\cap Y$ at more than finitely many points, then $y$ would be an accumulation point of $C$, and hence would be contained in $C$.  Thus, it must intersection $((m-1,m+1)\times \{ y\})\cap Y$ at at most finitely many points, in which case we can simply remove them to obtain an open neighborhood of $\coord{m,y}$.  These finitely many points are elements of $T_{m-1,k}$ or $T_{m+1,k}$ for some $k$, and so are in particular open.  Thus, the union of these finitely many points together with $U_{p_1}^{(m-1)-(1-\frac{1}{k})}$ and $U_{p_2}^{(m+1)+(1+\frac{1}{k})}$ then must contain $C$ (because these two neighborhoods of $p_1$ and $p_2$ contain all $T_{n,l}$ for $n\neq m-1,m+1$).

Finally if one of the points is some $p_{n,k}$, then using very similar logic as in the previous case, $C$ must intersect $T_{n,k}$ at at most finitely many points.  Removing these points from $T_{n,k}$ gives us an open neighborhood which is disjoint from the neighborhood formed from the union of these finitely many points (which are open) and the complement of $T_{n,k}$.
\begin{rmk}
I do not know of a name for this space, but in order to have something to refer to, I shall call it the \emph{Thomas Tent Space}\index{Thomas Tent Space}.\footnote{The $T_{n,k}$s are like ``tents'', and indeed, that is the word Scott used to describe them.}
\end{rmk}
\end{exm}
Thus, once you hit $T_{2\frac{1}{2}}$, you can increase your separation in one of two noncomparable ways:  you can make your space completely-$T_2$ or you can make your space $T_3$.  At the end of the section, we will summarize how all the separation axioms related to each other.
\begin{dfn}[$T_{3\frac{1}{2}}$]\label{T312}
$X$ is \emph{$T_{3\frac{1}{2}}$}\index{$T_{3\frac{1}{2}}$} iff it is $T_1$ and any closed set can be separated by closed neighborhoods from a point it does not contain.
\begin{rmk}
This is not universally accepted terminology.  Often people use the term $T_{3\frac{1}{2}}$ to refer to what I call completely-$T_3$.  I imagine this is the case because it turns out that my $T_{3\frac{1}{2}}$ is equivalent to $T_3$ (see the next proposition).
\end{rmk}
\begin{rmk}
You will notice a pattern with the terminology.  If $T_k$ means you can separate XYZ from ABC with neighborhoods, then $T_{k\frac{1}{2}}$ means you can separate XYZ from ABC with \emph{closed} neighborhoods; completely-$T_k$ means you can completely-separate XYZ from ABC; perfectly-$T_k$ means you can perfectly-separate XYZ from ABC.  This admittedly creates conflict with some people's terminology, but the terminology of separation axioms is so varied from source to source that it was impossible to come up with a naming system that didn't conflict with something.  I chose what I did because it is the most systematic that does not completely depart from the established nomenclature.
\end{rmk}
\end{dfn}
\begin{prp}\label{prp4.5.70}
If $X$ is $T_3$, then $X$ is $T_{3\frac{1}{2}}$
\begin{proof}
Suppose that $X$ is $T_3$.  Let $C\subseteq X$ be closed and let $x\in C^{\comp}$.  As $X$ is $T_3$, there is an open neighborhood $U$ of $C$ and an open neighborhood $V$ of $X$ which are disjoint.  Then, $V^{\comp}$ is a closed set not containing $X$, and so there is an open neighborhood $W_1$ of $V^{\comp}$ disjoint and an open neighborhood $W_2$ of $x$ which are disjoint.  Then,
\begin{equation}
C\subseteq U\subseteq \Cls (U)\subseteq V^{\comp}\subseteq W_1\text{ and }x\in W_2\subseteq \Cls (W_2)\subseteq W_1^{\comp},
\end{equation}
and so $U$ and $W_2$ are open neighborhoods of $C$ and $x$ respectively with disjoint closures, so that $X$ is $T_{3\frac{1}{2}}$ by \cref{prp4.5.13}.
\end{proof}
\end{prp}
\begin{dfn}[Completely-$T_3$]\label{CompletelyT3}
$X$ is \emph{completely-$T_3$}\index{Completely-$T_3$} iff it is $T_1$ and any closed set can be completely-separated from a point it does not contain.
\begin{rmk}
As noted above, sometimes people use the term $T_{3\frac{1}{2}}$ for this property.  This is also sometimes called \emph{tychonoff}\index{tychonoff}.
\end{rmk}
\end{dfn}
\begin{exm}[A space that is $T_{3\frac{1}{2}}$ but not completely-$T_3$]
The Thomas Tent Space $X$ from \cref{ThomasTentSpace} will do the trick.  We showed there that it is $T_3$, but not completely-$T_2$.  From \cref{prp4.5.70}, it follows that $X$ is $T_{3\frac{1}{2}}$.  However, it if were completely-$T_3$, then it would also be completely-$T_2$---a contradiction.  Therefore, $X$ is likewise not completely-$T_3$.
\end{exm}
\begin{dfn}[Perfectly-$T_3$]\label{PerfectlyT3}
$X$ is \emph{perfectly-$T_3$} iff it is $T_1$ and any closed set can be perfectly-separated from a point it does not contain.
\end{dfn}
\begin{exm}[A space that is completely-$T_3$ but not perfectly-$T_3$]\label{exm4.6.80}
The Uncountable Fort Space from \cref{UncountableFortSpace} will do the trick.  In \cref{exm4.5.48}, we showed that this was completely-$T_2$ but not perfectly-$T_2$.  As it is not perfectly-$T_2$, it is certainly not perfectly-$T_3$, though we still need to check that it is completely-$T_3$.

Recall that the Uncountable Fort Space was defined to be $X\coloneqq \R$ with the closed sets being precisely the finite sets and also the sets which contained $0$.

So, let $C\subseteq X$ be closed and let $x_0\in C^{\comp}$.  First let us do the case where $x_0=0$.  Then, we may define $f:X\rightarrow [0,1]$ by
\begin{equation}
f(x)\coloneqq \begin{cases}1 & \text{if }x\in C \\ 0 & \text{otherwise}\end{cases}.
\end{equation}
Then, $f^{-1}(1)=C$ is closed and $f^{-1}(0)$ contains $0\in X$, and so is closed.  Thus, $f$ is continuous.  Now consider the case where $0\in C$.  Then, we may define $f:X\rightarrow [0,1]$ by
\begin{equation}
f(x)\coloneqq \begin{cases}1 & \text{if }x=x_0 \\ 0 & \text{otherwise}\end{cases}.
\end{equation}
$f^{-1}(1)$ is just a point and so is closed and $f^{-1}(0)$ contains $0\in X$ and so is closed.  Finally, let us consider the case where $x_0\neq 0$ and $0\notin C$.  Then, we may define $f:X\rightarrow [0,1]$ by
\begin{equation}
f(x)\coloneqq \begin{cases}1 & \text{if }x\in C \\ 0 & \text{if }x=x_0 \\ \tfrac{1}{2} & \text{otherwise}\end{cases}.
\end{equation}
Then, $f^{-1}(1)=C$ is closed, $f^{-1}(0)=\{ x_0\}$ is finite and hence closed, and $f^{-1}(\frac{1}{2})$ contains $0\in X$ and is hence closed.  Thus, indeed, $X$ is completely-$T_3$.
\end{exm}

The separation axioms $T_0$ through perfectly-$T_2$ all had to do with separation of points.  All the $T_3$ separation axioms had to do with separating closed sets from points.  Finally, the $T_4$ axioms have to do with separating closed sets from closed sets.
\begin{dfn}[Normal]\label{Normal}
$X$ is \emph{normal}\index{Normal (topological space)} iff any two disjoint closed subsets can be separated by neighborhoods.
\begin{rmk}
Similarly as with the definition of regular (\cref{Regular}), we do not need to consider topological-distinguishability and mere separatedness.
\end{rmk}
\begin{rmk}
Similarly as with the term ``regular'', some authors reverse my conventions of normal and $T_4$.  The motivation of our choice of convention is the same as it was for $T_3$.
\end{rmk}
\end{dfn}
\begin{exm}[A space that is normal but not $T_0$]
The indiscrete topology on any set with at least two points is vacuously normal but not $T_0$.
\end{exm}
Stupid examples like this is why we almost always care about the case when we in addition impose the condition of being $T_1$.
\begin{dfn}[$T_4$]\label{T4}
$X$ is \emph{$T_4$}\index{$T_4$} iff it is $T_1$ and normal.
\begin{rmk}
Just as before, the condition of $T_1$ is imposed so that this is a strict specifization of being $T_3$.
\end{rmk}
\end{dfn}
There are at least two relatively large families of spaces that are $T_4$---metric spaces (which are in fact perfectly-$T-4$---see \cref{prp5.4.13}) and \emph{compact} spaces.
\begin{prp}\label{prp4.6.83}
If $X$ is compact, then it is $T_4$.
\begin{proof}
Suppose that $X$ is compact.  Then, closed subsets are quasicompact (\cref{exr4.2.33}), and hence can be separated by neighborhoods by \cref{exr4.6.39}.
\end{proof}
\end{prp}
\begin{prp}
If $X$ is $T_4$, then it is $T_{3\frac{1}{2}}$.
\begin{proof}
Suppose that $X$ is $T_4$.  Let $C\subseteq X$ be closed and let $x\in C^{\comp}$.  As $X$ is $T_4$, it is definitely $T_3$, and so we can separate $C$ and $x$ with open neighborhoods $U_1$ and $U_2$ containing $C$ and $x$ respectively.  Then, $U_1^{\comp}$ is a closed set disjoint from $C$, and so because $X$ is $T_4$, there is an open neighborhood $V_1$ of $C$ and an open neighborhood $V_2$ of $U_1^{\comp}$ which are disjoint.  Then,
\begin{equation}
\Cls (U_2)\subseteq U_1^{\comp}\subseteq V_2\text{ and }\Cls (V_1)\subseteq V_2,
\end{equation}
so that $\Cls (U_2)$ and $\Cls (V_1)$ are disjoint, so that $X$ is $T_{3\frac{1}{2}}$ by \cref{prp4.5.13}.
\end{proof}
\end{prp}
There are spaces that are perfectly-$T_3$ but not $T_4$; however, unfortunately we haven't yet the firepower to construct such a space yet (or at least prove it has the desired properties).  The counter-example we have in mind is \emph{Niemytzki's Tange Disk Topology}, and you can find it in \cref{NiemytzkisTangentDiskTopology}.

Recall that (\cref{ThomasTentSpace}) there are $T_3$ spaces that are \emph{not} completely-$T_2$.  This does not happen with $T_4$ and completely-$T_3$!  Fortunately, every $T_4$ space is completely-$T_3$, as we shall see in a moment (see \cref{UrysohnsLemma}).
\begin{dfn}[$T_{4\frac{1}{2}}$]\label{T412}
$X$ is \emph{$T_{4\frac{1}{2}}$}\index{$T_{4\frac{1}{2}}$} iff it is $T_1$ and any two disjoint closed subsets can be separated by closed neighborhoods.
\begin{rmk}
Just as with $T_{3\frac{1}{2}}$ (\cref{T312}), this terminology is not standard, presumably because it is actually just equivalent to $T_4$.
\end{rmk}
\end{dfn}
\begin{prp}\label{prp4.5.88}
If $X$ is $T_4$, then $X$ is $T_{4\frac{1}{2}}$.
\begin{proof}
Suppose that $X$ is $T_4$.  Let $C_1,C_2\subseteq X$ be disjoint closed subsets.  As $X$ is $T_4$, there are disjoint open neighborhoods $U_1$ and $U_2$ of $C_1$ and $C_2$ respectively.  Then, $U_1^{\comp}$ is a closed set not containing $C_1$, and so there are disjoint open neighborhoods $V$ and $W$ of $C_1$ and $U_1^{\comp}$ respectively.  Then,
\begin{equation}
C_2\subseteq U_2\subseteq \Cls (U_2)\subseteq U_1^{\comp}\subseteq W\text{ and }C_1\subseteq V\subseteq \Cls (V)\subseteq W^{\comp},
\end{equation}
and so $V$ and $U_2$ are open neighborhoods of $C_1$ and $C_2$ respectively with disjoint closures, so that $X$ is $T_{4\frac{1}{2}}$ by \cref{prp4.5.13}.
\end{proof}
\end{prp}
\begin{dfn}[Completely-$T_4$]\label{CompletelyT4}
$X$ is \emph{completely-$T_4$} iff it is $T_1$ and any two disjoint closed subsets can be completely-separated.
\end{dfn}
This is in fact \emph{equivalent} to being $T_4$, though this is relatively nontrivial, and the statement even has a name associated to it.  Before we prove that, however, we first present a useful `'lemma'.
\begin{prp}\label{prp4.5.91}
Let $X$ be a $T_1$ topological space.  Then,
\begin{enumerate}
\item \label{enm4.5.91.i}$X$ is $T_3$ iff whenever $U$ is an open neighborhood of $x\in X$, there is an open neighborhood $V$ of $x$ such that $x\in V\subseteq \Cls (V)\subseteq U$; and
\item \label{enm4.5.91.ii}$X$ is $T_4$ iff whenever $U$ is an open neighborhood of a closed subset $C\subseteq X$, there is an open neighborhood $V$ of $C$ such that $C\subseteq V\subseteq \Cls (V)\subseteq U$.
\end{enumerate}
\begin{proof}
We first prove \ref{enm4.5.91.i}.

$(\Rightarrow )$ Suppose that $X$ is $T_3$.  Let $U$ be an open neighborhood of $x\in X$.  Then, $U^{\comp}$ is a closed set that does not contain $x$, and so there are disjoint open neighborhoods $V$ and $W$ of $x$ and $U^{\comp}$ respectively.  Then,
\begin{equation}
x\in V\subseteq \Cls (V)\subseteq W^{\comp}\subseteq U.
\end{equation}

\blankline
\noindent
$(\Leftarrow )$ Suppose that whenever $U$ is an open neighborhood of $x\in X$, there is an open neighborhood $V$ of $x$ such that $x\in V\subseteq \Cls (V)\subseteq U$.  Let $C\subseteq X$ be closed and let $x\in C^{\comp}$.  Then, $C^{\comp}$ is an open neighborhood of $x$, and so there is an open neighborhood $V$ of $x$ such that
\begin{equation}
x\in V\subseteq \Cls (V)\subseteq C^{\comp},
\end{equation}
and so $V$ and $\Cls (V)^{\comp}$ are disjoint open neighborhoods of $x$ and $C$ respectively, so that $X$ is $T_3$.
\begin{exr}
Prove \ref{enm4.5.91.i}.
\end{exr}
\end{proof}
\end{prp}
\begin{thm}[Urysohn's Lemma]\index{Urysohn's Lemma}\footnote{Proof adapted from \cite[pg.~207]{Munkres}.}\label{UrysohnsLemma}
If $X$ is $T_4$, then $X$ is completely-$T_4$.
\begin{rmk}
This is often stated as ``If $C_1,C_2\subseteq X$ are disjoint closed subsets of a $T_4$ space $X$, then there is a continuous function $f:X\rightarrow [0,1]$ that is $0$ on $C_1$ and $1$ on $C_2$.''.
\end{rmk}
\begin{proof}
Suppose that $X$ is $T_4$.  Let $C_1,C_2\subseteq X$ be closed.

Let $\{ r_m:m\in \N \}$ be an enumeration of the rationals in $[0,1]$ with $r_0=1$ and $r_1=0$.  We define a collection of open sets $\{ U_{r_m}:m\in \N \}$ inductively.  During this process, we will apply \cref{prp4.5.91} repeatedly.

First of all, define $U_1\coloneqq C_2^{\comp}$.  Then, $U_1$ is an open neighborhood of $C_1$, and so there is some other open neighborhood $U_0$ of $C_1$ such that $C_1\subseteq U_0\subseteq \Cls (U_0)\subseteq U_1$.

Suppose that we have defined $U_{r_0},\ldots U_{r_m}$ such that $\Cls (U_p)\subseteq U_q$ if $r<q$ for $r,q\in \{ r_0,\ldots ,r_m\}$.  We wish to define $U_{m+1}$ so that this property still remains to be true.  As there are only finitely many rational numbers in $\{ r_0,\ldots ,r_m\}$ there is a largest $p_0\in \{ r_0,\ldots ,r_m\}$ with $p_0<r_{m+1}$ and similarly there is a smallest $q_0\in \{ r_0,\ldots ,r_m\}$ with $r_{m+1}<q_0$.  Take $U_{r_{m+1}}$ so that
\begin{equation}
\Cls (U_{p_0})\subseteq U_{r_{m+1}}\subseteq U_{q_0}.
\end{equation}
Proceeding inductively, this allows us to define $U_{r_m}$ for all $m\in \N$, and hence, we have defined $U_r$ for all $r\in \Q \cap [0,1]$.  By construction, it follows that
\begin{equation}
\Cls (U_p)\subseteq U_q
\end{equation}
for $p,q\in \Q \cap [0,1]$ for $p<q$.

For $x\in X$, define
\begin{equation}
Q_x\coloneqq \{ r\in \Q \cap [0,1]:x\in U_r\} .
\end{equation}
Note that $Q_x$ is empty iff $x\in C_2$.  We then in turn define
\begin{equation}
f(x)\coloneqq \begin{cases}1 & \text{if }x\in C_2 \\ \inf \left( Q_x\right) & \text{otherwise}\end{cases}.
\end{equation}
Of course $f(C_2)=\{ 1\}$.  Furthermore, if $x\in C_1$, then $x\in U_0$, and so indeed $f(x)=0$.  Thus, we need only check that $f$ is continuous.

So, let $x_0\in X$.  First suppose that $f(x_0)\neq 0,1$.  The other two cases are similar.  Let $\varepsilon >0$ be such that $B_\varepsilon (f(x_0))\subseteq [0,1]$ (if $f(x_0)=0$, for example, then you will instead use $[0,\varepsilon )$ in place of $B_\varepsilon (f(x_0))$).  Let $p,q\in \Q$ be such that
\begin{equation}
f(x_0)-\varepsilon <p<f(x_0)<q<f(x_0)+\varepsilon .
\end{equation}
Then, $U\coloneqq U_q\setminus \Cls (U_p)$ is open in $X$ and we claim that $f(U)\subseteq B_\varepsilon (x_0)$.  This will show that $f$ is continuous at $x_0$, and hence continuous as $x_0$ was arbitrary.

So, let $x\in U_q\setminus \Cls (U_p)$.  Then, $q\in Q_x$, and so $f(x)=\inf (Q_x)\leq q$.  On the other hand, as $x\notin U_p$, it cannot be in $U_r$ for any $r\leq p$ as $U_r\subseteq U_p$ for $r\leq p$.  Therefore, $p$ is a lower-bound for $Q_x$, and so $f(x)=\inf (Q_x)\geq p$.  Hence,
\begin{equation}
f(x_0)-\varepsilon <p\leq f(x)\leq p<f(x_0)+\varepsilon ,
\end{equation}
and we are done.
\end{proof}
\end{thm}
\begin{dfn}[Perfectly-$T_4$]\label{PerfectlyT4}
$X$ is \emph{perfectly-$T_4$} iff it is $T_1$ and any two disjoint closed can be perfectly-separated.
\begin{rmk}
For some reason, this is sometimes called $T_6$.  What is $T_5$ you ask?  Evidently $T_5$ means $T_1$ and every subspace is $T_4$.
\end{rmk}
\end{dfn}
\begin{exm}[A space that is completely-$T_4$ but not perfectly-$T_4$]
The Uncountable Fort Space from \cref{UncountableFortSpace} will once again do the trick.  In \cref{exm4.5.48}, we show that it is not perfectly-$T_2$, and so it is certainly not going to be perfectly-$T_4$.  We still need to check that it is completely-$T_4$.

Recall that the Uncountable Fort Space was defined to be $X\coloneqq \R$ with the closed sets being precisely the finite sets and also the sets which contained $0$.

So, let $C_1,C_2\subseteq X$ be disjoint closed sets.  Let us first do the case where neither $C_1$ nor $C_2$ contains $0\in X$.  Then, we may define $f:X\rightarrow [0,1]$ by
\begin{equation}
f(x)\coloneqq \begin{cases}0 & \text{if }x\in C_1 \\ 1 & \text{if }x\in C_2 \\ \tfrac{1}{2} & \text{otherwise}\end{cases}.
\end{equation}
The preimage of $0$ is $C_1$ is closed, the preimage of $1$ is $C_2$ is closed, and the preimage of $\frac{1}{2}$ contains $0\in X$ and so is closed.  Thus, this function is continuous.  Now suppose that $0\in C_1$.  Then, we may define $f:X\rightarrow [0,1]$ by
\begin{equation}
f(x)\coloneqq \begin{cases}0 & \text{if }x\in C_1 \\ 1 & \text{if }x\in C_2\end{cases}.
\end{equation}
The preimage of $0$ is $C_1$ is closed and the preimage of $1$ is $C_2$ is closed.  Thus, this function is continuous.
\end{exm}

\subsection{Summary}

We summarize what we have covered so far in this section.

First of all, there are several levels of separation between two different objects in a space
\begin{equation}
\begin{split}
\text{Distinct} & \Leftarrow \footnote{A two-point space with the indiscrete topology---see \cref{exm4.5.2}.}\text{Topologically-distinguishable}\Leftarrow \footnote{A two-point space with precisely one of the points open---see \cref{exm4.5.3}.}\text{Separated} \\
& \qquad \Leftarrow \footnote{A certain three-point space---see \cref{exm4.5.8}.}\text{Separated by neighborhoods}\Leftarrow \footnote{Another three-point space---see \cref{exm4.5.11}.}\text{Separated by closed neighborhoods} \\
& \qquad \qquad \Leftarrow \footnote{The Arens Square---see \cref{ArensSquare}.}\text{Completely-separated}\Leftarrow \footnote{The Uncountable Fort Space---see \cref{UncountableFortSpace}.}\text{Perfectly-separated}
\end{split}
\end{equation}
The arrows indicated implication of course, and all these implications are strict, as indicated by the examples referenced in the footnotes.

There are three `families' of separation axioms of spaces:  (i) separation of pairs of points, (ii) separation of closed sets from points, and (iii) and separation of disjoint closed sets.

In the first family:
\begin{enumerate}
\item Points being topologically-distinguishable is $T_0$ (\cref{T0}).
\item Points being separated is $T_1$ (\cref{T1}).
\item Points being separated by neighborhoods is $T_2$ (\cref{T2}).
\item Points being separated by closed neighborhoods is $T_{2\frac{1}{2}}$ (\cref{T212}).
\item Points being completely-separated is completely-$T_2$ (\cref{CompletelyT2}).
\item Points being perfectly-separated is perfectly-$T_2$ (\cref{PerfectlyT2}).
\end{enumerate}
In general, appending ``$\frac{1}{2}2$'', ``completely'', or ``perfectly'' to a separation axiom in this way changes whatever the separation axiom was now to ``separated by closed neighborhoods'', ``completely-separated'', and ``perfectly-separated'' respectively.

Closed sets and points, as well as closed sets and closed sets, are automatically separated, and so there are no separation axioms analogous to $T_0$ and $T_1$ for families (ii) and (iii).

For (ii) and (iii), in order for them to be directly comparable with (i), we require that points be closed (that is, we explicitly require spaces in families (ii) and (iii) to be $T_1$---see \cref{prp4.5.32}).  Without this extra assumption, the separation axioms are called ``regular'' and ``normal'' respectively.

Thus, for the second family:\footnote{Remember that we require the spaces to be $T_1$ in addition to these properties.}
\begin{enumerate}
\item Closed sets and points being separated by neighborhoods is $T_3$ (\cref{T3}).
\item Closed sets and points being separated by closed neighborhoods is $T_{3\frac{1}{2}}$ (\cref{T312}).
\item Closed sets and points being completely-separated is completely-$T_3$ (\cref{CompletelyT3}).
\item Closed sets and points being perfectly-separated is perfectly-$T_3$ (\cref{PerfectlyT3}).
\end{enumerate}

And similarly for the third family:
\footnote{Remember that we require the spaces to be $T_1$ in addition to these properties.}
\begin{enumerate}
\item Closed sets being separated by neighborhoods is $T_4$ (\cref{T3}).
\item Closed sets being separated by closed neighborhoods is $T_{4\frac{1}{2}}$ (\cref{T412}).
\item Closed sets being completely-separated is completely-$T_4$ (\cref{CompletelyT4}).
\item Closed set being perfectly-separated is perfectly-$T_4$ (\cref{PerfectlyT4}).
\end{enumerate}

We now illustrate how all these axioms are related to each other.  All implications are strict (unless otherwise indicated by a $\Leftrightarrow$), in which case the offending counter-example is given in the indicated footnote.  Perhaps the only real surprise\footnote{That's not to say that the counter-examples are easy---they're not (in fact, some of them are \emph{really fucking hard}, like, among-the-most-difficult-things-in-the-notes hard)---but rather that you probably expect that some counter-example exists, even if incredibly exotic.} is the equivalence of $T_{4\frac{1}{2}}$ and completely-$T_4$:  \nameref{UrysohnsLemma} (\cref{UrysohnsLemma}).\footnote{Though perhaps it's worth nothing that $T_{k\frac{1}{2}}$ is equivalent to completely-$T_k$ for $k=3,4$ but \emph{not} $k=2$.}
\begin{equation}\label{4.6.105}
\begin{tikzcd}
T_0 & & & & \\
T_1 \ar[u,Rightarrow,"\footnote{A two-point space in which precisely one point is open---see \cref{exm4.5.3}.}"] & & & & \\
T_2 \ar[u,Rightarrow,"\footnote{$\R$ with the cocountable topology---see \cref{exm4.2.8x}.}"] & & & & \\
T_{2\tfrac{1}{2}} \ar[u,Rightarrow,"\footnote{The Simplified Arens Square---see \cref{SimplifiedArensSquare}.}"] & T_3 \ar[l,Rightarrow,"\footnote{$\R$ with the cocountable extension topology---see \cref{CocountableExtensionTopology}.}"] & T_{3\tfrac{1}{2}} \ar[l,Leftrightarrow,"\footnote{See \cref{prp4.5.70}.}"] & T_4 \ar[l,Rightarrow,"\footnote{Niemytzki's Tangent Disk Topology---see \cref{NiemytzkisTangentDiskTopology}.}"] & T_{4\tfrac{1}{2}} \ar[l,Leftrightarrow,"\footnote{See \cref{prp4.5.88}.}"] \\
\text{Completely-}T_2 \ar[u,Rightarrow,"\footnote{The Arens Square---see \cref{ArensSquare}.}"] & & \text{Completely-}T_3 \ar[ll,Rightarrow,"\footnote{$\R$ with the cocountable extension topology---see \cref{CocountableExtensionTopology}.}"] \ar[u,Rightarrow,"\footnote{The Thomas Tent Space---see \cref{ThomasTentSpace}.}"] & & \text{Completely-}T_4 \ar[ll,Rightarrow,"\footnote{Niemytzki's Tangent Disk Topology---see \cref{NiemytzkisTangentDiskTopology}.}"] \ar[u,Leftrightarrow,"\footnote{Urysohn's Lemma, \cref{UrysohnsLemma}.}"] \\
\text{Perfectly-}T_2 \ar[u,Rightarrow,"\footnote{The Uncountable Fort Space--see \cref{UncountableFortSpace}.}"] & & \text{Perfectly-}T_3 \ar[ll,Rightarrow,"\footnote{$\R$ wit the cocountable extension topology---see \cref{CocountableExtensionTopology}.}"] \ar[u,Rightarrow,"\footnote{The Uncountable Fort Space---see \cref{UncountableFortSpace}.}"] & & \text{Perfectly-}T_4 \ar[ll,Rightarrow,"\footnote{Niemytzki's Tangent Disk Topology---see \cref{NiemytzkisTangentDiskTopology}.}"] \ar[u,Rightarrow,"\footnote{The Uncountable Fort Space---see \cref{UncountableFortSpace}.}"]
\end{tikzcd}.
\end{equation}

\section{The Intermediate and Extreme Value Theorems}

\subsection{Connectedness and the Intermediate Value Theorem}

You'll recall from calculus that the classical statement of the Intermediate Value Theorem is
\begin{textequation}
Let $f:[a,b]\rightarrow \R$ be continuous.  Then, for all $y$ between $f(a)$ and $f(b)$ (inclusive), there exists $x\in [a,b]$ such that $f(x)=y$.
\end{textequation}
We will see that the proper way to interpret this statement is that the image of $f$ is connected, so that if the image contains $f(a)$ and $f(b)$ (which it does by definition), then it must contain everything in-between as well.  Of course, in order to make this precise, we have to first define what it means to be connected.
\begin{dfn}[Connected and disconnected]\label{Connected}
Let $X$ be a topological space.  Then, $X$ is \emph{disconnected}\index{Disconnected} iff there exist disjoint nonempty open sets $U,V\subset X$ such that $X=U\cup V$.  $X$ is \emph{connected}\index{Connected} iff it is not disconnected.  A subset $S$ of $X$ is connected iff it is connected in its subspace topology.
\begin{rmk}
The intuition for the definition of disconnected of course is that we break-up the space into two separate pieces which have no overlap.
\end{rmk}
\begin{rmk}
Another way to say this is that $X$ is disconnected iff it has a partition with two open sets.
\end{rmk}
\begin{rmk}
Note that it is \emph{not} the case that $S$ is disconnected iff $S=U\cup V$ for nonempty disjoint subsets $U,V\subseteq X$.  The reason for the difference is that \emph{subsets of $S$ which are open in $S$ need not be open in $X$} (see \cref{exm4.1.14}).
\end{rmk}
\end{dfn}
\begin{exm}
Define $S\coloneqq [0,1]\cup [2,3]\subseteq \R \eqqcolon X$.  Then, $X$ is certainly disconnected because $U\coloneqq [0,1]$ and $V\coloneqq [2,3]$ are nonempty disjoint open subsets of $S$ such that $S=[0,1]\cup [2,3]$.  Note, however, that if we required $U$ and $V$ to be open \emph{in $X\coloneqq \R$}, then this would not show that $S$ is disconnected.  This is why we require that $U$ and $V$ are \emph{open in the subspace topology of $S$}, instead of open in the ambient space.
\end{exm}
\begin{prp}
Let $X$ be a topological space and let $\mathcal{U}$ be a collection of connected subsets of $X$ with nonempty intersection.  Then, $\bigcup _{U\in \mathcal{U}}U$ is connected.
\begin{proof}
To simplify notation, let us write $X'\coloneqq \bigcup _{U\in \mathcal{U}}U$.  We proceed by contradiction:  suppose that $\bigcup _{U\in \mathcal{U}}U$ is disconnected, so that we may write
\begin{equation}
X'=V\cup W
\end{equation}
for $V,W\subseteq X'$ open, nonempty, and disjoint.  Let $x_0\in \bigcap _{U\in \mathcal{U}}$ and without loss of generality assume that $x_0\in V$.  For each $U\in \mathcal{U}$, let us write
\begin{equation}
U_V\coloneqq U\cap V\text{ and }U_W\coloneqq U\cap W,
\end{equation}
so that
\begin{equation}
U=U_V\cup U_W
\end{equation}
for all $U\in \mathcal{U}$.  As $U$ is connected, it follows that, for each $U$, either $U_V$ or $U_W$ is empty.  However, we know that $x_0\in U_V$, and so in fact, we must have that $U_W=\emptyset$ for all $U\in \mathcal{U}$, which in turn implies that $W=\emptyset$:  a contradiction.
\end{proof}
\end{prp}
\begin{dfn}[Connected component]
Let $X$ be a topological space and let $x_1,x_2$.  Then, $x_1$ and $x_2$ are \emph{connected} (to each other) iff there exists a connected set $U\subseteq X$ with $x_1,x_2\in U$.
\begin{prp}
The relation of being connected to is an equivalence relation on $X$.
\begin{proof}
$x$ is connected to itself because $\{ x\}$ is connected.  The relation is symmetric because the definition of the relation is symmetric.  If $x_1$ is connected to $x_2$ and $x_2$ is connected to $x_3$, then there is some connected set $U$ which contains $x_1$ and $x_2$, and there is some connected set $V$ which contains $x_2$ and $x_3$.  As $U$ and $V$ both contain $x_2$, it follows from the previous proposition that $U\cup V$ is connected, and hence $x_1$ is connected to $x_3$.
\end{proof}
\end{prp}
A \emph{connected component}\index{Connected component} of $X$ is an equivalence class of some point with respect to the relation of being connected to.
\end{dfn}
\begin{dfn}[Totally disconnected]
Let $X$ be a topological space.  Then, $X$ is \emph{totally-disconnected} iff every nonempty connected component of $X$ is a point.
\end{dfn}
\begin{prp}
Let $X$ be a discrete topological space.  Then, $X$ is totally-disconnected.
\begin{proof}
Let $U\subseteq X$ have at least two distinct points $x_1$ and $x_2$.  Then,
\begin{equation}
U=\{ x_1\} \cup (U\setminus \{ x_1\} ),
\end{equation}
and as every subset in a discrete space is open, it follows that $U$ is disconnected.
\end{proof}
\end{prp}
\begin{exm}[$\N$, $\Z$, and $\Q$ are totally-disconnected]
That $\N$ and $\Z$ are totally-disconnected follows from the fact that they are discrete.

$\Q$ is also totally-disconnected,\footnote{In particular, there are totally-disconnected spaces which are not discrete.} but this is more difficult to see.  Let $U\subseteq \Q$ have at least two distinct points $q_1$ and $q_2$.  Without loss of generality, suppose that $q_1<q_2$.  Then, by `density' of $\Q ^{\comp}$ in $\R$ (\cref{thm3.3.76}), there is some $x\in \Q ^{\comp}$ with $q_1<x<q_2$.  Define
\begin{equation}
V\coloneqq (-\infty ,x)\cap U \text{ and }W\coloneqq (x,\infty )\cap U .
\end{equation}
Both $V$ and $W$ are open by the definition of the subspace topology (\cref{SubspaceTopology}) of $U$, and both are nonempty because $q_1<x$ and $q_2>x$.  Thus, as $U=V\cup W$, $U$ is disconnected, and hence $\Q$ is totally-disconnected.
\end{exm}

We mentioned at the beginning of this section that the proper way to interpret the Intermediate Value Theorem is the statement that the image of connected sets are connected.  There is one other thing we need to check though---we need to check that intervals in $\R$ are in fact connected.
\begin{thm}\label{thm4.5.14}
Let $I\subseteq \R$.  Then, $I$ is connected iff it is an interval.
\begin{rmk}
In particular, $\R$ itself is connected, in contrast with $\N$, $\Z$, and $\Q$.
\end{rmk}
\begin{proof}
$(\Rightarrow )$ Suppose that $I$ is connected.  Let $a,b\in I$ with $a\leq b$ and let $x\in \R$ with $a\leq x\leq b$.  We must show that $x\in I$.  If either $x=a$ or $x=b$, we are done, so we may as well suppose that $a<x<b$.  We proceed by contradiction:  suppose that $x\notin I$.  Then,
\begin{equation}
I=(I\cap (-\infty ,x))\cup (I\cap (x,\infty )),
\end{equation}
and so as $I$ is connected, we must have that either $I\cap (-\infty ,x)$ is empty or $I\cap (x,\infty )$ is empty.  But $a$ is in the former and $b$ is in the latter:  a contradiction.

\blankline
\noindent
$(\Leftarrow )$ Suppose that $I$ is an interval.  As $I$ is an interval, by \cref{prp3.3.70}, we have that $I=[(a,b)]$ for $a=\inf (I)$ and $b=\sup (I)$.\footnote{Recall that this notation just means that the end-points can be either open or closed; see the remark in \cref{prp3.3.70}.}  We wish to show that $I$ is connected.  If $a=b$, then either $I=\emptyset$ or $I=\{ a\}$, in which case $I$ is trivially connected.  Therefore, we may assume without loss of generality that $a<b$.

We proceed by contradiction:  suppose that $I$ is disconnected.  The, we have that $I=U\cup V$ for $U,V\subseteq I$ open in $I$ disjoint and nonempty.  As $U$ and $V$ are open in $I$ and cover $I$, we must have that at least one of them contains an open neighborhood of $a$, so without loss of generality, suppose that $[(a,x)\subseteq U$ for some $x\in \R$ with $a<x\leq b$.  Now define
\begin{equation}
S\coloneqq \left\{ x\in I :[(a,x)\subseteq U\right\} .
\end{equation}
We just showed that this set is nonempty.  It is also bounded above by $b$ as $b$ is in particular a bound of $I$.  Therefore, it has a supremum.  We wish to show that $\sup (S)=b$.

Note that $a<\sup (S)\leq b$, and so, because $I$ is an interval, either $\sup (S)\in I$ or $\sup (S)=b$.  In the latter case, we are done, so we may as well assume that $\sup (S)\in I$.

In the case we are considering in which $\sup (S)\in I$, we show that in fact $\sup (S)\in U$.  We proceed by contradiction:  suppose that $\sup (S)\in V$ (here is where we use the fact that $\sup (S)\in I$.  As $V$ is open, there is a neighborhood of $\sup (S)$ completely contained in $V$.  On the other hand, by \cref{prp1.4.11}, this neighborhood has to contain some element of $S$, which in turn would imply that it would have to contain some element of $U$.  But then $U$ intersects $V$:  a contradiction.  Therefore, $\sup (S)\in U$.

We now finish the proof that $\sup (S)=b$.  We certainly have that $\sup (S)\leq b$ as $b$ is an upper-bound of $S$.  Thus, it suffices to show that $\sup (S)\geq b$.  To show this, we proceed by contradiction:  suppose that $\sup (S)<b$.  Then, because $U$ is open, there is some $\varepsilon >0$ such that $(\sup (S)-\varepsilon ,\sup (S)+\varepsilon )\subseteq U$.  By \cref{prp1.4.11}, there must be some $x\in S$ with $\sup (S)-\varepsilon <x\leq \sup (S)$, so that $[(a,x)\subseteq U$.  But then,
\begin{equation}
[(a,x))\cup (\sup (S)-\varepsilon ,\sup (S)+\varepsilon )=[(a,\sup (S)+\varepsilon )\subseteq U,
\end{equation}
and so, in particular, there is some $x'>\sup (S)$ such that $[(a,x')\subseteq U$:  a contradiction.  Therefore, we must have that $\sup (S)=b$.

Now that we have finally succeeded in showing that $\sup (S)=b$, we finish the proof by coming to a contradiction of the assumption of disconnectedness.  As $\sup (S)=b$, this means that for every $x\in I$ with $a<x<b$, we have that $[(a,x)\subseteq U$, and hence
\begin{equation}
\bigcup _{a<x<b}[(a,x)=[(a,b)\subseteq U.
\end{equation}
Thus, either $V=\{ b\}$ or $V=\emptyset$:  a contradiction of being open in $I$ or being nonempty respectively.
\end{proof}
\end{thm}
And now we finally get to the statement of the `true' Intermediate Value Theorem.
\begin{thm}[Intermediate Value Theorem]\label{IntermediateValueTheorem}\index{Intermediate Value Theorem}
Let $f:X\rightarrow Y$ be a continuous function and let $S\subseteq X$ be connected.  Then, $f(S)$ is connected.
\begin{proof}
We proceed by contradiction:  suppose that $f(S)$ is disconnected.  Then, $f(S)=U\cup V$ for $U,V\subseteq f(S)$ open in $f(S)$ disjoint and nonempty.  By continuity, we have that\footnote{Also recall that $f^{-1}(f(S))\supseteq S$---see \cref{exrA.1.47}\ref{enmA.1.47.ii}.}
\begin{equation}
S\supseteq \footnote{This follows from the fact that we are applying the preimage of \emph{the restriction} of $f$ to $S$.}\restr{f}{S}^{-1}(U)\cup \restr{f}{S}^{-1}(V)\supseteq S,
\end{equation}
and so
\begin{equation}
S=\restr{f}{S}^{-1}(U)\cup \restr{f}{S}^{-1}(V).
\end{equation}
As $U$ and $V$ are open in $S$ and $f$ is continuous, $\restr{f}{S}^{-1}(U)$ and $\restr{f}{S}^{-1}(V)$ are open in $S$.  They also must be disjoint, for a point which lied in their intersection would be mapped into $U\cap S$ via $f$.  Therefore, because $S$ is connected, we have that either $\restr{f}{S}^{-1}(U)$ or $\restr{f}{S}^{-1}(V)$ is empty, which implies respectively that either $U$ or $V$ is empty:  a contradiction.
\end{proof}
\end{thm}
As a corollary of this (and the fact that a subnet of $\R$ is connected iff it is an interval---see \cref{thm4.5.14}), we have the classical statement of the Intermediate Value Theorem.
\begin{crl}[Classical Intermediate Value Theorem]\label{ClassicalIntermediateValueTheorem}\index{Classical Intermediate Value Theorem}
Let $f:[a,b]\rightarrow \R$ be continuous.  Then, $f([a,b])$ is an interval.  In particular, any element between $f(a)$ and $f(b)$ is in the image of $f$.
\begin{proof}
The ``in particular'' part follows from the definition of an interval, \cref{Interval}.

$[a,b]$ is connected by \cref{thm4.5.14}, and so, by the Intermediate Value Theorem, $f([a,b])$ is connected, and so by \cref{thm4.5.14} again, is an interval.
\end{proof}
\end{crl}

\subsection{Quasicompactness and the Extreme Value Theorem}

You'll recall from calculus that the classical statement of the Extreme Value Theorem is
\begin{textequation}
Let $f:[a,b]\rightarrow \R$ be continuous.  Then, there exists $x_1,x_2\in [a,b]$ such that $f(x_1)=\inf _{x\in [a,b]}\left\{ f(x)\right\}$ and $f(x_2)=\sup _{x\in [a,b]}\left\{ f(x)\right\}$.  In other words, continuous functions \emph{attain} their maximum and minimum on closed intervals.
\end{textequation}
This is actually a special case of (or follows easily from) a \emph{much} more general, elegant statement.
\begin{thm}[Extreme Value Theorem]\label{ExtremeValueTheorem}\index{Extreme Value Theorem}
Let $f:X\rightarrow Y$ be continuous and let $K\subseteq X$.  Then, if $K$ is quasicompact, then $f(K)$.
\begin{rmk}
In other words, the continuous image of a quasicompact set is quasicompact.
\end{rmk}
\begin{proof}
Suppose that $K$ is quasicompact.  Let $\mathcal{U}$ be an open cover of $f(K)$.  Then, $f^{-1}(\mathcal{U})\coloneqq \left\{ f^{-1}(U):U\in \mathcal{U}\right\}$ is an open cover of $K$, and therefore it has a finite subcover $\{ f^{-1}(U_1),\ldots f^{-1}(U_m)\}$.  In other words,
\begin{equation}
K\subseteq f^{-1}(U_1)\cup \cdots \cup f^{-1}(U_m),
\end{equation}
and hence
\begin{equation}
\begin{split}
f(K) & \subseteq f\left( f^{-1}(U_1)\cup \cdots \cup f^{-1}(U_m)\right) =\footnote{By \cref{exrA.1.30}\ref{enmA.1.30.iii}.}f\left( f^{-1}(U_1)\right) \cup \cdots \cup f\left( f^{-1}(U_m)\right) \\
& =\footnote{By \cref{exrA.1.47}\ref{enmA.1.47.i}.}\subseteq U_1\cup \cdots \cup U_m,
\end{split}
\end{equation}
so that $\{ U_1,\ldots ,U_m\}$ is a finite subcover of $\mathcal{U}$, and hence $f(K)$ is quasicompact.
\end{proof}
\end{thm}
And now we can present the classical version of the theorem.
\begin{crl}[Classical Extreme Value Theorem]\label{ClassicalExtremeValueTheorem}\index{Classical Extreme Value Theorem}
Let $f:[a,b]\rightarrow \R$ be continuous.  Then, $f([a,b])$ is closed and bounded.  In particular, it attains a maximum and minimum on $[a,b]$.
\begin{rmk}
I suppose that usually in calculus people don't make any statement regarding the image being an interval---this is separately stated as the Intermediate Value Theorem.
\end{rmk}
\begin{proof}
The ``in particular'' is a result of \cref{exr3.4.27}, the statement that closed bounded sets contain their supreumum and infimum.

By the \nameref{HeineBorelTheorem} (\cref{HeineBorelTheorem}), $[a,b]$ is quasicompact.  Therefore, by the Extreme Value Theorem, $f([a,b])$ is quasicompact, and therefore, closed and bounded, again by the \nameref{HeineBorelTheorem}.
\end{proof}
\end{crl}

In fact, we may as well just combine the classical statements into one.
\begin{crl}[Classical Intermediate-Extreme Value Theorem]\label{ClassicalIntermediateExtremeValueTheorem}\index{Classical Intermediate-Extreme Value Theorem}
Let $f:[a,b]\rightarrow \R$ be continuous.  Then, $f([a,b])$ is a closed, bounded, interval.
\end{crl}

\section{Local properties}

For most topological properties, there is a ``local'' version.
\begin{mdf}[Locally XYZ]\label{LocallyXYZ}
A topological space is \emph{locally XYZ}\index{Locally XYZ} iff each point has a neighborhood base consisting of sets that are XYZ.
\end{mdf}
Of particular importance are the notions of local connectedness, local quasicompactness, and locally (completely/perfectly)-$T_a$.
\begin{exr}
Show that if a space is $T_2$ then it is locally $T_2$.  Find a counter-example to show that converse is false.
\end{exr}
\begin{exr}
Show that if a space is locally quasicompact and $T_2$, then it is locally compact.  Find a counter-example to show the converse is false.
\end{exr}
The following is the result related to locally quasicompactness that will be used in the proof of \nameref{HaarHowesTheorem}.
\begin{prp}\label{prp5.2.4}
Let $X$ be a locally compact space, let $K\subseteq X$ be quasicompact, and let $U\subseteq X$ contain $X$.  Then, there is an open set $V\subseteq X$ with compact closure such that
\begin{equation}
K\subseteq V\subseteq \Cls (V)\subseteq U.
\end{equation}
\begin{proof}
We leave this as an exercise.
\begin{exr}
Complete the proof yourself.
\end{exr}
\end{proof}
\end{prp}
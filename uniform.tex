\section{Motivation}

A uniform space is the most general context in which one can talk about concepts such as uniform continuity, uniform convergence, cauchyness, completeness, etc..  To formalize this notion, we will equip a set with a distinguished set of covers, called \emph{uniform covers}.  The example you should always keep in the back of your mind is the collection of all $\varepsilon$-balls for a \emph{fixed} $\varepsilon$:  $\mathcal{U}_\varepsilon \coloneqq \left\{ B_\varepsilon (x):x\in \R \right\}$.  The idea is that, somehow, all of the sets in the same uniform cover are of the `same size'.

Having specified the uniform covers, we will then be able to say things like a net $\lambda \mapsto x_\lambda$ is cauchy iff for every uniform cover $\mathcal{U}$, there is some $U\in \mathcal{U}$ such that $\lambda \mapsto x_\lambda$ is eventually contained in $U$.  In the case that the collection of uniform covers is $\left\{ \mathcal{U}_\varepsilon :\varepsilon >0\right\}$,\footnote{Disclaimer:  The collection of all the $\mathcal{U}_{\varepsilon}$ is not actually a uniformity but rather a \emph{uniform base}---see \cref{UniformBase}.} you can check that this is precisely the definition of cauchyness we had in $\R$ (\cref{dfn3.3.26}).

Moreover, the generalization from $\R$ to uniform spaces is not a needless abstraction.  Indeed, I am \emph{required} to cover metric spaces (\cref{MetricSpace}) in this course, and this is just a very special type of uniform space.  Indeed, essentially every topological space we look at in these notes---besides ones cooked up for the express purpose of producing a counter-example---has a canonical uniformity.  On the other hand, it is certainly not the case that every topological space we encounter will be a metric space.  For example, something as simple as all continuous functions from $\R$ to $\R$ has no canonical metric,\footnote{For what it's worth, I believe the topology is metrizable (homeomorphic to a metric space), but certainly not with any metric you would like to work with, much less a canonical one.} but is trivially a uniform space (because it is a topological group---see \cref{dfnB.7}).

\section{Basic definitions and facts}

A uniform space will wind-up being a set equipped with a special set of covers, the \emph{uniform covers}.  Of course, however, as you should expect, we cannot just take \emph{any} collection of covers and declare them to be the uniform covers---the collection of uniform covers has to satisfy certain reasonable properties, analogous to the properties satisfied by the collections of all $\varepsilon$-balls.  They key requirement is that the collection of uniform covers has to be \emph{downward-directed} with respect to a relation called \emph{star-refinement}.\footnote{You are not expected to know what either of these terms mean yet.}  Thus, before getting to the definition of a uniform space itself, we must say what we mean by ``star-refinement'' (we will say what we mean by ``downward-directed'' in the definition of a uniform space itself).

\subsection{Star-refinements}

\begin{dfn}[Star]\label{Star}
Let $X$ be a set, let $S\subseteq X$, and let $\mathcal{U}$ be a cover of $X$.  Then, the \emph{star}\index{Star} of $S$ with respect to $\mathcal{U}$, $\Star _{\mathcal{U}}(S)$,\index[notation]{$\Star _{\mathcal{U}}(S)$} is defined by
\begin{equation}
\Star _{\mathcal{U}}(S)\coloneqq \bigcup _{U\in \mathcal{U}\st U\cap S\neq \emptyset}U.
\end{equation}
The star of a point is the star of its singleton and denoted $\Star _{\mathcal{U}}(x)$\index[notation]{$\Star _{\mathcal{U}}(x)$}.
\begin{rmk}
In other words, the star of a set with respect to a cover is the union of all elements of the cover which intersect the set.
\end{rmk}
\end{dfn}
\begin{exm}[The star of the preimage is not the preimage of the star]
One might hope for the preimage of the star to be the star of the preimage, that is, for $f:X\rightarrow Y$, $V\subseteq Y$, and $\mathcal{V}$ a cover of $Y$, that
\begin{equation}
f^{-1}\left( \Star _{\mathcal{V}}(V)\right) =\Star _{f^{-1}(\mathcal{V})}(f^{-1}(V)).
\end{equation}
Unfortunately, this is not necessarily the case.  For example, take $f$ to be the inclusion $\R \hookrightarrow \R ^2$ (with image the $x$-asix), take $V$ to be any subset of $\R ^2$ which does not intersect the $x$-axis (e.g.~$V=\{ \coord{x,y}\in \R ^2:y=1\}$), and take $\mathcal{V}\coloneqq \{ \R ^2\}$, namely the cover of $\R ^2$ consisting of only $\R ^2$ itself.  Then,
\begin{equation}
\Star _{\mathcal{V}}(V)=\R ^2,
\end{equation}
and so
\begin{equation}
f^{-1}\left( \Star _{\mathcal{V}}(V)\right) =\R .
\end{equation}
On the other hand,
\begin{equation}
\Star _{f^{-1}(\mathcal{V})}(f^{-1}(V))=\emptyset 
\end{equation}
simply because $f^{-1}(V)=\emptyset$.
\end{exm}
Despite this, we always have one inclusion.
\begin{prp}\label{prpC.2.3}
Let $f:X\rightarrow Y$ be a function, let $V\subseteq Y$, and let $\mathcal{V}$ be a cover of $Y$.  Then,
\begin{equation}
\Star _{f^{-1}(\mathcal{V})}(f^{-1}(V))\subseteq f^{-1}\left( \Star _{\mathcal{V}}(V)\right) .
\end{equation}
Furthermore, if $f$ satisfies $f(f^{-1}(S))=S$ for all $S\subseteq X$, then we have equality.
\begin{rmk}
A sufficient condition for $f(f^{-1}(S))=S$ is for $f$ to be surjective because then $f$ has a right-inverse---see \cref{exrA.1.9}\ref{enmA.1.9.ii}
\end{rmk}
\begin{proof}
Note that we have that (\cref{exrA.1.30}\ref{enmA.1.30.ii} and \cref{exrA.1.47}\ref{enmA.1.47.i}) $V\cap V_0\supseteq f(f^{-1}(V)\cap f^{-1}(V_0))$.  Therefore, if $f^{-1}(V)$ intersects $f^{-1}(V_0)$, it must be the case that $V$ intersects $V_0$.  Furthermore, if we have that $f(f^{-1}(S))=S$, then we would in fact that that $V\cap V_0=f(f^{-1}(V)\cap f^{-1}(V_0))$, so that in this case $V$ intersects $V_0$ iff $f^{-1}(V)$ intersects $f^{-1}(V_0)$.
\begin{equation}
\begin{split}
\Star _{f^{-1}(\mathcal{V})}(f^{-1}(V_0)) & =\bigcup _{V\in \mathcal{V}\st f^{-1}(V)\cap f^{-1}(V_0)\neq \emptyset}f^{-1}(V) \\
& =\footnote{\cref{exrA.1.30}\ref{enmA.1.30.i}}f^{-1}\left( \bigcup _{V\in \mathcal{V}\st f^{-1}(V)\cap f^{-1}(V_0)\neq \emptyset}V\right) \\
& \subseteq \footnote{Here we are using the fact that $f^{-1}(V)$ intersects $f^{-1}(V_0)$ implies that $V$ intersects $V_0$.  Also note that we have equality here if $f(f^{-1}(S))=S$.}f^{-1}\left( \bigcup _{V\in \mathcal{V}\st V\cap V_0\neq \emptyset}V\right) =f^{-1}\left( \Star _{\mathcal{V}}(V_0)\right) .
\end{split}
\end{equation}
\end{proof}
\end{prp}
We also have the `dual' counter-example and result for the image.
\begin{exm}[The star of the image is not the image of the star]
Take $f:\R ^2\rightarrow \R$ to be the projection onto the $x$-axis, define
\begin{equation}
\mathcal{U}\coloneqq \{ [m+y,m+1+y]\times \{ y\} :m\in \Z ,\ y\in \R \} ,
\end{equation}
and $U_0\coloneqq (0,1)\times \{ 0\}$.  Then,
\begin{equation}
\Star_{\mathcal{U}}(U_0)=[0,1]\times \{ 0\} ,
\end{equation}
and so
\begin{equation}
f\left( \Star _{\mathcal{U}}(U_0)\right) =[0,1].
\end{equation}
On the other hand,
\begin{equation}
f(\mathcal{U})=\{ [m+y,m+1+y]:m\in \Z ,\ y\in \R \} 
\end{equation}
and $f(U_0)=(0,1)$, and so $f(U_0)$ intersects $[m+y,m+1+y]$ for $m=0$ and $-1<y<1$, and so
\begin{equation}
\Star _{f(\mathcal{U})}f(U_0)\supseteq \bigcup _{-1<y<1}[y,y+1]=(-1,2),
\end{equation}
which is strictly larger than $f\left( \Star _{\mathcal{U}}(U_0)\right)$.
\end{exm}
\begin{prp}\label{prp4.2.17}
Let $f:X\rightarrow Y$ be a function, let $U\subseteq X$, and let $\mathcal{U}$ be a cover of $X$.  Then,
\begin{equation}
\Star _{f(\mathcal{U})}(f(U))\supseteq f(\Star _{\mathcal{U}}(U)).
\end{equation}
Furthermore, if $f$ is surjective and satisfies $f^{-1}(f(S))=S$ for all $S\subseteq X$, then we have equality.
\begin{rmk}
Requiring that $f$ be surjective is not really a big deal---we need $f$ to be surjective for the image of a cover to be a cover anyways.
\end{rmk}
\begin{rmk}
A sufficient condition for $f^{-1}(f(S))=S$ is for $f$ to be injective because then $f$ has a left-inverse---see \cref{exrA.1.9}\ref{enmA.1.9.i}.
\end{rmk}
\begin{proof}
Note that we have that (\cref{exrA.1.30}\ref{enmA.1.30.iv} and \cref{exrA.1.47}\ref{enmA.1.47.ii}) $U\cap U_0\subseteq f^{-1}(f(U)\cap f(U_0))$.  Therefore, if $U$ intersects $U_0$, it must be the case that $f(U)$ intersects $f(U_0)$.  Furthermore, if we have that $f$ is surjective and $f^{-1}(f(S))=S$ for all $S\subseteq X$, then we have that
\begin{equation}
U\cap U_0=f^{-1}(f(U))\cap f^{-1}(f(U_0))=f^{-1}(f(U)\cap f(U_0)),
\end{equation}
and hence
\begin{equation}
f(U\cap U_0)=f(U)\cap f(U_0),
\end{equation}
so in this case $U$ intersects $U_0$ iff $f(U)$ intersects $f(U_0)$.  Hence,
\begin{equation}
\begin{split}
f(\Star _{\mathcal{U}}(U)) & =f\left( \bigcup _{U'\in \mathcal{U}\st U'\cap U\neq \emptyset}U'\right) =\footnote{\cref{exrA.1.30}\ref{enmA.1.30.iii}}\bigcup _{U'\in \mathcal{U}\st U'\cap U\neq \emptyset}f(U') \\
& \subseteq \footnote{Here we are using the fact that $U$ intersects $U_0$ implies that $f(U)$ intersects $f(U_0)$.  Also note that we have equality here if $f$ is surjective and $f^{-1}(f(S))=S$.}\bigcup _{U'\in \mathcal{U}\st f(U')\cap f(U)\neq \emptyset}f(U')=\Star _{f(\mathcal{U})}(U).
\end{split}
\end{equation}
\end{proof}
\end{prp}

\begin{dfn}[Refinement and star-refinement]\label{dfnC.1}
Let $X$ be a set, and let $\mathcal{U}$ and $\mathcal{V}$ be covers on $X$.
\begin{enumerate}
\item $\mathcal{U}$ is a \emph{refinement}\index{Refinement} of $\mathcal{V}$, written $\mathcal{U}\preceq \mathcal{V}$\index[notation]{$\mathcal{U}\preceq \mathcal{V}$} iff for every $U\in \mathcal{U}$ there is some $V\in \mathcal{V}$ such that $U\subseteq V$.
\item $\mathcal{U}$ is a \emph{star-refinement}\index{star-refinement} of $\mathcal{V}$, written $\mathcal{U}\llcurly \mathcal{V}$\index[notation]{$\mathcal{U}\llcurly \mathcal{V}$} iff for every $U\in \mathcal{U}$ there is a $V\in \mathcal{V}$ such that $\Star _{\mathcal{U}}(U)\subseteq V$.
\end{enumerate}
\begin{rmk}
The intuition is that every element of $\mathcal{U}$ is small enough to be contained in some element of $\mathcal{V}$.
\end{rmk}
\begin{rmk}
In other words, $\mathcal{U}$ is a star-refinement of $\mathcal{V}$ iff for all $U\in \mathcal{U}$, there is some $V\in \mathcal{V}$ such that, whenever $U'\in \mathcal{U}$ intersects $U$, it follows that $U'\subseteq V$.  The intuition for star-refinements is the same as for refinements, except that a star-refinement is \emph{much} finer than a mere refinement.
\end{rmk}
\end{dfn}
\begin{exr}
Show that $\preceq$ is a preorder, but not a partial-order.
\end{exr}
\begin{exr}\label{exr4.2.22}
Show that $\llcurly$ is transitive, but not even reflexive.
\end{exr}
Any two covers always have a common refinement.  In fact, they have a canonical (but not unique!) largest one.
\begin{dfn}[Meet of covers]
Let $X$ be a set, and let $\mathcal{U}$ and $\mathcal{V}$ be covers of $X$.  Then, the \emph{meet}\index{Meet (of covers)} of $\mathcal{U}$ and $\mathcal{V}$, $\mathcal{U}\wedge \mathcal{V}$\index[notation]{$\mathcal{U}\wedge \mathcal{V}$}, is defined by
\begin{equation}
\mathcal{U}\wedge \mathcal{V}\coloneqq \{ U\cap V:U\in \mathcal{U}\text{ and }V\in \mathcal{V}\} .
\end{equation}
\begin{rmk}
The term ``meet'' and notation ``$\mathcal{U}\wedge \mathcal{V}$'' is notation taken from the theory of partially-ordered sets where $x\wedge y$ (the \emph{meet}) of $x$ and $y$ is defined to be $\inf \{ x,y\}$.  In our case, however, this is abuse of notation and terminology as $\preceq$ is not a partial-order (and so infinma need not be unique---see \cref{exr1.4.4}).
\end{rmk}
\end{dfn}
\begin{exr}
Show that (i) $\mathcal{U}\wedge \mathcal{V}\preceq \mathcal{U},\mathcal{V}$; and (ii) if $\mathcal{W}$ refines both $\mathcal{U}$ and $\mathcal{V}$, then it refines $\mathcal{U}\wedge \mathcal{V}$.  Find an example to show that it is \emph{not} the unique such cover with these two properties.
\end{exr}
\begin{exr}
Does every cover have \emph{any} maximal star-refinement?
\end{exr}

Later it will be useful to know that taking the preimage preserves (star-)refinement.  But first, we need to be a bit careful about what we mean by the image and preimage of a cover.
\begin{dfn}[Image and preimage of a cover]
Let $f:X\rightarrow Y$ be a function, let $\mathcal{U}$ be a cover of $X$, and $\mathcal{V}$ be a cover of $Y$.  Then, the \emph{image} of $\mathcal{U}$, $f(\mathcal{U})$\index[notation]{$f(\mathcal{U})$}, is defined by
\begin{equation}
f(\mathcal{U})\coloneqq \{ f(U):U\in \mathcal{U}\} .
\end{equation}
The \emph{preimage} of $\mathcal{V}$, $f^{-1}(\mathcal{V}$\index[notation]{$f^{-1}(\mathcal{V})$}, is defined by
\begin{equation}
f^{-1}(\mathcal{V})\coloneqq \{ f^{-1}(V):V\in V\in \mathcal{V}\} .
\end{equation}
\begin{rmk}
We needed to make these definitions because, technically speaking, we only defined the image an preimage of \emph{subsets} of $X$ and $Y$ respectively.  As $\mathcal{U}$ and $\mathcal{V}$ are subsets of $2^X$ and $2^Y$ respectively, not $X$ and $Y$, to talk about their `usual' image and preimage, we would need to have a function from $2^X$ to $2^Y$.\footnote{Actually, given a function $f:X\rightarrow Y$, we obtain a function from $2^X$ to $2^Y$ \emph{and} a function from $2^Y$ to $2^X$---$f:2^X\rightarrow 2^Y$ (the function that sends a set to its image) and $f^{-1}:2^Y\rightarrow 2^X$ (the function that sends a set to its preimage) respectively.  The preimage of a cover is the image of the cover with respect to the preimage function $f^{-1}:2^Y\rightarrow 2^X$.  Likewise, the image of a cover is the image of the cover with respect to the image function $f:2^X\rightarrow 2^Y$.}  In particular, note that the definition of the preimage of a cover is \emph{not} $\{ U\in 2^X:f(U)\in \mathcal{V}\}$.
\end{rmk}
\end{dfn}
\begin{prp}\label{prpB.2.12}
Let $f:X\rightarrow Y$ be a function and let $\mathcal{U}$ and $\mathcal{V}$ be covers of $Y$ such that $\mathcal{U}\preceq \mathcal{V}$ ($\mathcal{U}\llcurly \mathcal{V}$).  Then, $f^{-1}(\mathcal{U})\preceq f^{-1}(\mathcal{V})$ ($f^{-1}(\mathcal{U})\llcurly f^{-1}(\mathcal{V})$).
\begin{proof}
We first do the case with $\mathcal{U}\preceq \mathcal{V}$.  Let $f^{-1}(U)\in f^{-1}(\mathcal{U})$ with of course $U\in \mathcal{U}$.  Then, as $\mathcal{U}\preceq \mathcal{V}$, there is some $V\in \mathcal{V}$ such that $U\subseteq V$.  Then, $f^{-1}(U)\subseteq f^{-1}(V)$, and so $f^{-1}(\mathcal{U})\preceq f^{-1}(\mathcal{V})$.

Now we do the case $\mathcal{U}\llcurly \mathcal{V}$.  Let $f^{-1}(U)\in f^{-1}(\mathcal{U})$ with of course $U\in \mathcal{U}$.  Then, as $\mathcal{U}\llcurly \mathcal{V}$, there is some $V\in \mathcal{V}$ such that $\Star _{\mathcal{U}}(U)\subseteq V$.  Hence,
\begin{equation}
\Star _{f^{-1}(\mathcal{U})}(f^{-1}(U))\subseteq f^{-1}\left( \Star _{\mathcal{U}}(U)\right) \subseteq f^{-1}(V),
\end{equation}
where we have applied \cref{prpC.2.3}, and so $\mathcal{U}\llcurly \mathcal{V}$.
\end{proof}
\end{prp}
\begin{exr}
Show that if $\mathcal{U}\preceq \mathcal{V}$, then $f(\mathcal{U})\preceq f(\mathcal{V})$.
\end{exr}
Unfortunately, however, in general, it will not be the case that the image preserves star-refinements.
\begin{exr}
Find an example of covers $\mathcal{U}$ and $\mathcal{V}$ with $\mathcal{U}\llcurly \mathcal{V}$, but $f(\mathcal{U})$ not a star-refinment of $f(\mathcal{V})$.
\end{exr}
However, in special case, it will.
\begin{prp}\label{prp4.2.17x}
Let $f:X\rightarrow Y$ be a function and let $\mathcal{U}$ and $\mathcal{V}$ be covers of $X$ such that $\mathcal{U}\llcurly \mathcal{V}$.  Then, if $f$ is surjective and $f^{-1}(f(U))=U$ for all $U\in \mathcal{U}$, then $f(\mathcal{U})\llcurly f(\mathcal{V})$.
\begin{proof}
Suppose that $f$ is surjective and $f^{-1}(f(S))=S$ for all $S\subseteq X$.
\begin{exr}
Using the fact that $f$ preserves stars by \cref{prp4.2.17} to show that $f(\mathcal{U})\llcurly f(\mathcal{V})$.
\end{exr}
\end{proof}
\end{prp}

\subsection{Uniform spaces}

\begin{dfn}[Uniform space]\label{UniformSpace}
A \emph{uniform space}\index{Uniform space} is a set $X$ equipped with a nonempty collection $\widetilde{\mathcal{U}}$ of covers, the \emph{uniformity}\index{Uniformity}, such that
\begin{enumerate}
\item \label{UniformSpace.UpwardClosed}(Upward-closed)\index{Upward-closed} if $\mathcal{U}\in \widetilde{\mathcal{U}}$ and $\mathcal{U}\llcurly \mathcal{V}$, then $\mathcal{V}\in \widetilde{\mathcal{U}}$; and
\item \label{UniformSpace.DownwardDirected}(Downward-directed)\index{Downward-directed} if $\mathcal{U},\mathcal{V}\in \widetilde{\mathcal{U}}$, then there is some $\mathcal{W}\in \widetilde{\mathcal{U}}$ such that $\mathcal{W}\llcurly \mathcal{U}$ and $\mathcal{W}\llcurly \mathcal{V}$.
\end{enumerate}
\begin{rmk}
The elements of $\widetilde{\mathcal{U}}$ are \emph{uniform covers}\index{Uniform covers}.
\end{rmk}
\begin{rmk}
The intuition is that, in a given uniform cover $\mathcal{U}$, every element of $\mathcal{U}$ is `of the same size' (think $\mathcal{U}\coloneqq \{ B_{\varepsilon}(x):x\in \R \}$ for a \emph{fixed} $\varepsilon >0$).
\end{rmk}
\begin{rmk}
Note that, by taking $\mathcal{U}=\mathcal{V}$ in \ref{UniformSpace.DownwardDirected}, we see that, in particular, every uniform cover is star-refined by some other uniform cover.
\end{rmk}
\begin{rmk}
Note that the cover $\{ X\}$ is an element of every uniformity.  This follows from the fact that any collection of uniform covers is required to be nonempty and the fact that collections of uniform covers are upward-closed with respect to star-refinement.  (We mention this because sometimes that $\{ X\}$ is a uniform cover is taken as an axiom, in place of the requirement that the collection of uniform covers simply be nonempty.)
\end{rmk}
\end{dfn}
Of incredible importance is that uniformities \emph{define} a canonical topology.  Thus, we can think of uniform spaces as topological spaces with \emph{extra structure}.
\begin{prp}[Uniform topology]\label{UniformTopology}
Let $\coord{X,\widetilde{\mathcal{X}}}$ be a uniform space.  Then, for $x\in X$,
\begin{equation}
\mathcal{B}_x\coloneqq \left\{ \Star _{\mathcal{U}}(x):\mathcal{U}\in \widetilde{\mathcal{U}}\right\}
\end{equation}
is a neighborhood base at $x$.  The topology defined by this neighborhood base is the \emph{uniform topology}\index{Uniform topology} on $X$ with respect to $\widetilde{\mathcal{U}}$.
\begin{rmk}
Unless otherwise stated, uniformities are \emph{always} equipped with the uniform topology.
\end{rmk}
\begin{proof}
Let $\Star _{\mathcal{U}_1}(x),\Star _{\mathcal{U}_2}(x)\in \mathcal{B}_x$.  Let $\mathcal{U}_3$ be a common star-refinement of both $\mathcal{U}_1$ and $\mathcal{U}_2$.  We wish to show that
\begin{equation}
\Star _{\mathcal{U}_3}(x)\subseteq \Star _{\mathcal{U}_1}(x),\Star _{\mathcal{U}_2}(x).
\end{equation}
By $1\leftrightarrow 2$ symmetry, it suffices to just prove one of these inclusions.  By definition, we have
\begin{equation}
\Star _{\mathcal{U}_3}(x)\coloneqq \bigcup _{U\in \mathcal{U}_3\st x\in U}U.
\end{equation}
So, let $U\in \mathcal{U}_3$ contain $x$.  Because $\mathcal{U}_3$ star-refines $\mathcal{U}_1$, there is some $V\in \mathcal{U}_1$ such that
\begin{equation}
\Star _{\mathcal{U}_3}(U)\subseteq V.
\end{equation}
In particular, $U\subseteq V$.  Then, $V$ contains $x$, and so $V\subseteq \Star _{\mathcal{U}_1}(x)$, and so $U\subseteq \Star _{\mathcal{U}_1}(x)$.  It follows that
\begin{equation}
\Star _{\mathcal{U}_3}(x)\subseteq \Star _{\mathcal{U}_1}(x),
\end{equation}
and we are done.
\begin{rmk}
Note that this proof did not make use of the upward-closed axiom.  Thus, in fact, uniform bases (see below in \cref{UniformBase}) suffice to define the uniform topology as well.
\end{rmk}
\end{proof}
\end{prp}
\begin{exm}[Discrete and indiscrete uniform spaces]
Just as with topological spaces, we can always put the largest and the smallest uniformity on a set $X$.  The former case, in which every cover of $X$ is a uniform cover, is the \emph{discrete uniformity}\index{Discrete uniformity}, and the latter, in which the only uniform cover is $\{ X\}$, is the \emph{indiscrete uniformity}\index{Indiscrete uniformity}.
\begin{exr}
Show that the uniform topology with respect to the discrete uniformity is the discrete topology and that the uniform topology with respect to the indiscrete uniformity is the indiscrete topology.
\end{exr}
\end{exm}

Just as we have continuous maps between topological spaces, we have \emph{uniformly}-continuous maps between uniform spaces.
\begin{dfn}[Uniformly-continuous function]
Let $f:X\rightarrow Y$ be a function between uniform spaces.  Then, $f$ is \emph{uniformly-continuous} iff the preimage of every uniform cover is a uniform cover.
\end{dfn}
\begin{exm}[The category of uniform spaces]
The category of uniform spaces is the category $\Uni$\index[notation]{$\Uni$} whose collection of objects $\Uni _0$ is the collection of all uniform spaces, for every uniform space $X$ and uniform space $Y$ the collection of morphisms from $X$ to $Y$, $\Mor _{\Uni}(X,Y)$, is precisely the set of all uniformly-continuous functions from $X$ to $Y$, composition is given by ordinary function composition, and the identities of the category are the identity functions.
\begin{exr}
Show that the composition of two uniformly-continuous functions is uniformly-continuous.
\begin{rmk}
Note that this is something you need to check in order for $\Uni$ to actually form a category $(\Mor _{\Uni}(X,Y)$ needs to be closed under composition).  You also need to verify the identity function is uniformly-continuous, but this is trivial (the preimage of a cover is itself, so\textellipsis ).
\end{rmk}
\end{exr}
\end{exm}
\begin{dfn}[Uniform-homeomorphism]\label{UniformHomeomorphism}
Let $f:X\rightarrow Y$ be a function between uniform spaces.  Then, $f$ is a \emph{uniform-homeomorphism}\index{Uniform-homeomorphism} iff it is an isomorphism in $\Uni$.
\end{dfn}
\begin{exr}
Show that a function is a uniform-homeomorphism iff (i) it is bijective, (ii) it is uniformly-continuous, and (iii) its inverse is uniformly-continuous.
\end{exr}
\begin{exr}
Show that if a function is uniformly-continuous, then it is continuous.
\end{exr}
\begin{exr}
Find an example of a function that is bijective and uniformly-continuous, but not a uniform-homeomorphism.
\end{exr}

\subsection{Uniform bases, generating collections, and the initial and final uniformities}

It is usually convenient to not specify every uniform cover explicitly, but rather, to specify a certain collection of uniform covers and then take the uniformity `generated' by this collection.  This is analogous to how it is often convenient to only specify a base for a topology.
\begin{dfn}[Uniform base]\label{UniformBase}
Let $X$ be a uniform space and let $\widetilde{\mathcal{B}}$ be a collection of uniform covers of $X$.  Then, $\widetilde{\mathcal{B}}$ is a \emph{uniform base}\index{Uniform base} for the uniformity on $X$ iff the statement that a cover $\mathcal{U}$ is a uniform cover is equivalent to the statement that there is some $\mathcal{B}\in \widetilde{\mathcal{B}}$ such that $\mathcal{B}\llcurly \mathcal{U}$.
\begin{rmk}
You should compare this to the definition of a base for a topology (\cref{Base}).
\end{rmk}
\end{dfn}
And just like with bases, the real reason uniform bases are important is because they allow us to \emph{define} uniformities.  Thus, same as before, it is important to know when a collection of covers of a set form a uniform base for some uniformity.
\begin{prp}\label{prp4.3.2}
Let $X$ be a set and let $\widetilde{\mathcal{B}}$ be a nonempty collection of covers of $X$.  Then, there exists a unique uniformity for which $\widetilde{\mathcal{B}}$ is a uniform base iff $\widetilde{\mathcal{B}}$ is downward-directed with respect to $\llcurly$.
\begin{rmk}
Just as we did for bases, if a set $X$ does not a priori come with a uniformity, we will still refer to any collection of covers that is downward-directed with respect to $\llcurly$ as a \emph{uniform base}.
\end{rmk}
\begin{proof}
$(\Rightarrow )$ Suppose that there exists a uniformity for which $\widetilde{\mathcal{B}}$ is a uniform base.  Let $\mathcal{B},\mathcal{C}\in \widetilde{\mathcal{B}}$.  Then, there is certainly some uniform cover $\mathcal{U}$ which star-refines both $\mathcal{B}$ and $\mathcal{C}$ (recall that covers in $\widetilde{\mathcal{B}}$ are a priori taken to be uniform covers).  Thus, we will be done if we can show that $\mathcal{U}\in \widetilde{\mathcal{B}}$.  However, because $\mathcal{U}$ is a uniform cover and $\widetilde{\mathcal{B}}$ is a uniform base, there is some $\mathcal{D}\in \widetilde{\mathcal{B}}$ such that $\mathcal{D}\llcurly \mathcal{U}$.  As $\mathcal{U}$ star-refines both $\mathcal{B}$ and $\mathcal{C}$, it follows that $\mathcal{D}$ does as well.

\blankline
\noindent
$(\Leftarrow )$ Suppose that $\widetilde{\mathcal{B}}$ is downward-directed with respect to $\llcurly$.  Define $\widetilde{\mathcal{U}}$ to be the collection of covers that are star-refined by some element of $\widetilde{\mathcal{B}}$.  As $\widetilde{\mathcal{B}}$ is nonempty, so to is $\widetilde{\mathcal{U}}$ (it must contain $\{ X\}$).  By the definition of uniform bases, this was the only possibility.  As $\widetilde{\mathcal{B}}$ is nonempty and downward-directed with respect to $\llcurly$, it follows that $\widetilde{\mathcal{U}}$ contains $\widetilde{\mathcal{B}}$, and in particular is nonempty.  $\widetilde{\mathcal{U}}$ is upward-closed with respect to $\llcurly$ because if $\mathcal{U}$ is a uniform cover and star-refines $\mathcal{V}$, then there is some $\mathcal{B}\in \widetilde{\mathcal{B}}$ that star-refines $\mathcal{U}$, and hence in turn star-refines $\mathcal{V}$.  We now check that it is downward-directed with respect to $\llcurly$.  If $\mathcal{U}$ and $\mathcal{V}$ are covers, then there are $\mathcal{B},\mathcal{C}\in \widetilde{\mathcal{B}}$ that star-refine $\mathcal{U}$ and $\mathcal{V}$ respectively.  Because $\widetilde{\mathcal{B}}$ is downward-directed, there is then some $\mathcal{D}\in \widetilde{\mathcal{B}}$ which star-refines both $\mathcal{B}$ and $\mathcal{C}$, and hence both $\mathcal{U}$ and $\mathcal{V}$.
\end{proof}
\end{prp}
As we mentioned above at the end of the proof of \cref{UniformTopology}, uniform bases define the uniform topology just as well as the entire uniformity.
\begin{crl}
Let $\widetilde{\mathcal{B}}$ be a uniform base for the uniform space $X$.  Then, for $x\in X$,
\begin{equation}
\mathcal{B}_x\coloneqq \left\{ \Star _{\mathcal{U}}(x):\mathcal{U}\in \widetilde{\mathcal{U}}\right\}
\end{equation}
is a neighborhood base at $x$ for the uniform topology.
\end{crl}
\begin{exr}
Show that $\widetilde{\mathcal{U}}\coloneqq \left\{ \{ x\} :x\in X\right\}$ is a uniform base for the discrete uniformity.
\begin{rmk}
That is, the collection consisting of just a \emph{single} open cover, which itself is just the collection of all singletons, forms a uniform base.  In other words, you need to check that $\mathcal{U}$ star-refines itself.\footnote{Of course, while $\llcurly$ in general is not reflexive, that doesn't mean we can't at least have $\mathcal{U}\llcurly \mathcal{U}$ \emph{some} of the time.}
\end{rmk}
\begin{rmk}
The `dual' result for the indiscrete uniformity is trivial---the indiscrete uniformity by definition only has a single cover to begin with (namely $\{ X\}$), and that single cover certainly forms a uniform base for itself.
\end{rmk}
\end{exr}
We will want to check that two collections of uniform covers are the same by just looking at uniform bases.  We did not present it because we did not need to make use of it, but of course there is an analogous result for bases of topological spaces.
\begin{prp}\label{prp1.6}
Let $X$ be a set, and let $\widetilde{\mathcal{B}}$ and $\widetilde{\mathcal{C}}$ be uniform bases on $X$.  Then, $\widetilde{\mathcal{B}}$ and $\widetilde{\mathcal{C}}$ determine the same uniformity iff for every $\mathcal{B}\in \widetilde{\mathcal{B}}$, there is some $\mathcal{C}\in \widetilde{\mathcal{C}}$ with $\mathcal{C}\llcurly \mathcal{B}$; and for every $\mathcal{C}\in \widetilde{\mathcal{C}}$, there is some $\mathcal{B}\in \widetilde{\mathcal{B}}$ with $\mathcal{B}\llcurly \mathcal{C}$.
\begin{proof}
$(\Rightarrow )$ Suppose that $\widetilde{\mathcal{B}}$ and $\widetilde{\mathcal{C}}$ determine the same uniformity.  Let $\mathcal{B}\in \widetilde{\mathcal{B}}$.  Then, $\mathcal{B}$ is in particular in the uniformity generated by $\widetilde{\mathcal{B}}$, and hence in the uniformity generated by $\widetilde{\mathcal{C}}$.  Thus, there is some $\mathcal{C}\in \widetilde{\mathcal{C}}$ such that $\mathcal{C}\llcurly \mathcal{B}$.  By symmetry $\widetilde{\mathcal{B}}\leftrightarrow \widetilde{\mathcal{C}}$, the other result is true as well.

\blankline
\noindent
$(\Leftarrow )$ Suppose that for every $\mathcal{B}\in \widetilde{\mathcal{B}}$, there is some $\mathcal{C}\in \widetilde{\mathcal{C}}$ with $\mathcal{C}\llcurly \mathcal{B}$; and for every $\mathcal{C}\in \widetilde{\mathcal{C}}$, there is some $\mathcal{B}\in \widetilde{\mathcal{B}}$ with $\mathcal{B}\llcurly \mathcal{C}$.  Let $\mathcal{U}$ be a uniform cover in the uniformity determined by $\widetilde{\mathcal{B}}$.  Then, there is some $\mathcal{B}\in \widetilde{\mathcal{B}}$ such that $\mathcal{B}\llcurly \mathcal{U}$.  By the hypothesis, then, there is some $\mathcal{C}\in \widetilde{\mathcal{C}}$ with $\mathcal{C}\llcurly \mathcal{B}\llcurly \mathcal{U}$.  Thus, $\mathcal{U}$ is in the uniformity determined by $\widetilde{\mathcal{C}}$.  By $\widetilde{\mathcal{B}}\leftrightarrow \widetilde{\mathcal{C}}$ symmetry, the reverse inclusion is also true.
\end{proof}
\end{prp}
We will also want to check whether a function is uniformly-continuous by simply looking at a uniform base.
\begin{prp}\label{prpB.3.4}
Let $f:(X,\widetilde{\mathcal{U}})\rightarrow (Y,\widetilde{\mathcal{V}})$ be a function between uniform spaces and let $\widetilde{\mathcal{C}}$ be a uniform base for $\widetilde{\mathcal{V}}$.  Then, $f$ is uniformly-continuous iff $f^{-1}(\mathcal{C})\in \mathcal{U}$ for each $\mathcal{C}\in \widetilde{\mathcal{C}}$.
\begin{proof}
$(\Rightarrow )$ There is nothing to check (because every open cover in a uniform base is itself a uniform cover).

\blankline
\noindent
$(\Leftarrow )$ Suppose that $f^{-1}(\mathcal{C})\in \widetilde{\mathcal{U}}$ for each $\mathcal{C}\in \widetilde{\mathcal{C}}$.  We need to show that the preimave of \emph{every} uniform cover is a uniform cover.  So, let $\mathcal{V}\in \widetilde{\mathcal{V}}$.  Then, there is some $\mathcal{C}\in \widetilde{\mathcal{C}}$ such that $\mathcal{C}\llcurly \mathcal{V}$.  Then, by \cref{prpB.2.12}, it follows that $f^{-1}(\mathcal{C})\llcurly f^{-1}(\mathcal{V})$.  As $f^{-1}(\mathcal{C})\in \widetilde{\mathcal{U}}$ and $\widetilde{\mathcal{U}}$ is upward-closed with respect to $\llcurly$, it follows that $f^{-1}(\mathcal{V})\in \widetilde{\mathcal{U}}$, so that $f$ is uniformly-continuous.
\end{proof}
\end{prp}

A uniform space does not start its life as a topological space---instead, it obtains a canonical topology from its uniformity.  In particular, it does not make sense a priori to just restrict to open covers.  On the other hand, once we specify the uniform covers, a topology is determined, and then, it turns out (as the following proposition shows), that it suffices to just look at \emph{open} uniform covers, more precisely, those uniform covers obtained by taking the interior of the covers in your uniform base.
\begin{prp}\label{lma5.1.16}
Let $\widetilde{\mathcal{B}}$ be a uniform base on a set $X$.  Then, (i) $\Int (\mathcal{B})\coloneqq \{ \Int (B):B\in \mathcal{B}\}$ is (still) a cover of $X$ and (ii) $\Int (\widetilde{\mathcal{B}})\coloneqq \left\{ \Int (\mathcal{B}):\mathcal{B}\in \widetilde{\mathcal{B}}\right\}$ is (still) a uniform base on $X$ that generates the same uniform base as $\widetilde{\mathcal{B}}$.
\begin{proof}
Let $\mathcal{B}\in \widetilde{\mathcal{B}}$ and let $x\in X$.  Let $\mathcal{C}$ be a star-refinement of $\mathcal{B}$.  Let $C\in \mathcal{C}$ contain $x$ and let $B\in \mathcal{B}$ be such that $\Star _{\mathcal{C}}(C)\subseteq B$.  Then,
\begin{equation}
x\in \Star _{\mathcal{C}}(x)\subseteq \Star _{\mathcal{C}}(C)\subseteq B,
\end{equation}
and so $x\in \Int (B)$, and so indeed $\Int (\mathcal{B})$ is a cover of $X$.

Let $\mathcal{B},\mathcal{C}\in \widetilde{\mathcal{B}}$ and let $\mathcal{D}$ be a common star-refinement of $\mathcal{B}$ and $\mathcal{C}$.  We show that $\Int (\mathcal{D})$ is a common star-refinement of $\Int (\mathcal{B})$ and $\Int (\mathcal{C})$.  Because of $\mathcal{B}\leftrightarrow \mathcal{C}$ symmetry, it suffices to show that it is a star-refinement of $\Int (\mathcal{C})$.  So, let $\Int (D)\in \Int (\mathcal{D})$.  Then, there is some $C\in \mathcal{C}$ such that $\Star _{\mathcal{D}}(D)\subseteq C$, and so because union of the interiors is contained in the interior of the union (\cref{exr3.4.53}\ref{enm3.4.53.ii}), we have
\begin{equation}
\Star _{\Int (\mathcal{D})}(\Int (D))\subseteq \Star _{\Int (\mathcal{D})}(D)\subseteq \Int (C),
\end{equation}
and so indeed $\Int (\mathcal{D})$ star-refines $\Int (\mathcal{D})$.

It remains to show that $\widetilde{\mathcal{B}}$ and $\Int (\widetilde{\mathcal{B}})$ induce the same uniform structure.  To show this, we apply \cref{prp1.6}.  Let $\mathcal{B}\in \widetilde{\mathcal{B}}$ and let $\mathcal{C}$ be a star-refinement of $\mathcal{B}$.  Then, $\Int (\mathcal{C})$ certainly star-refines $\mathcal{B}$.  For the other direction, let $\mathcal{D}$ be a star-refinement of $\mathcal{C}$.  We show that $\mathcal{D}$ star-refinement $\Int (\mathcal{B})$.  Let $D\in \mathcal{D}$ and let $C\in \mathcal{C}$ be such that $\Star _{\mathcal{D}}(D)\subseteq C$.  Let $B\in \mathcal{B}$ be such that $\Star _{\mathcal{C}}(C)\subseteq B$.  Let $x\in \Star _{\mathcal{D}}(D)$.  Then,
\begin{equation}
x\in C\subseteq \Star _{\mathcal{C}}(x)\subseteq \Star _{\mathcal{C}}(C)\subseteq B,
\end{equation}
and so indeed $x\in \Int (B)$.
\end{proof}
\end{prp}

As the idea of a uniformity is inherently `global' in nature, there really isn't a way to define a uniformity that is analogous to the method of defining a topology by specifying neighborhood bases.\footnote{Of course one cannot hope to make a statement like this precise (What does it mean for a method of defining a uniformity to be ``analogous to'' a method of defining a topology?), but hopefully the intuition is clear.  All elements in a uniform cover, no matter where they are in the space, are supposed to be thought of as the same size.  How could one hope to encode the idea of two sets living `far away' are of the same size by the specification of local information alone?}  There is, of course, a way to specifying a uniformity by merely declaring a given collection of covers to be uniform covers.  This is of course analogous to the specifying of a generating collection of a topology.
\begin{prp}[Generating collection of covers]
Let $X$ be a set and let $\widetilde{\mathcal{S}}$ be a nonempty collection of covers of $X$.  Then, there exists a unique uniformity $\widetilde{\mathcal{U}}$ on $X$, the uniformity \emph{generated by}\index{Generate (a uniformity)} by $\widetilde{\mathcal{S}}$, such that
\begin{enumerate}
\item $\widetilde{\mathcal{S}}\subseteq \widetilde{\mathcal{U}}$; and
\item if $\widetilde{\mathcal{U}}'$ is any other uniformity containing $\widetilde{\mathcal{S}}$, it follows that $\widetilde{\mathcal{U}}\subseteq \widetilde{\mathcal{U}}'$.
\end{enumerate}
Furthermore, the collection of all finite meets form a uniform base for this uniformity.  $\widetilde{\mathcal{S}}$ is the \emph{generating collection}\index{Generating collection (of covers)}.
\begin{proof}
We leave the proof as an exercise.
\begin{exr}
Prove this result, using the proof of \cref{GeneratingCollection} (the analogous result for topological spaces) as guidance.
\end{exr}
\end{proof}
\end{prp}
With topological spaces, we could define a topology by specifying the closure or interior, or defining a notion of convergence.  To the best of my knowledge, there is no analogous method for any of these definitions of defining uniformities.  As for the initial and final topologies, however, there are most certainly analogous constructions.
\begin{prp}[Initial uniformity]\label{InitialUniformity}
Let $X$ be a set, let $\mathcal{Y}$ be an indexed collection of uniform spaces, and for each $Y\in \mathcal{Y}$ let $f_Y:X\rightarrow Y$ be a function.  Then, there exists a unique uniformity $\widetilde{\mathcal{U}}$ on $X$, the \emph{initial uniformity} with respect to $\{ f_Y:Y\in \mathcal{Y}\}$, such that
\begin{enumerate}
\item $f_Y:X\rightarrow Y$ is uniformly-continuous with repsect to $\widetilde{\mathcal{U}}$ for all $Y\in \mathcal{Y}$; and
\item if $\widetilde{\mathcal{U}}'$ is another uniformity for which each $f_Y$ is uniformly-continuous, then $\widetilde{\mathcal{U}}\subseteq \widetilde{\mathcal{U}}'$.  Furthermore, if $\widetilde{\mathcal{S}}_Y$ generates the uniformity on $Y$, then the collection
\begin{equation}
\{ f_Y^{-1}(\mathcal{U}):Y\in \mathcal{Y},\ \mathcal{U}\in \widetilde{\mathcal{S}}\}
\end{equation}
generates $\widetilde{\mathcal{U}}$.
\end{enumerate}
\begin{rmk}
In other words, the initial uniformity is the smallest uniformity for which each $f_Y$ is uniformly-continuous.
\end{rmk}
\begin{rmk}
But what about the largest such uniformity?  Well, the largest such uniformity is always going to be the discrete uniformity, which is not very uninteresting.  This is how you remember whether the initial uniformity is the smallest or largest---it can't be the largest because the discrete uniformity always works.
\end{rmk}
\begin{rmk}
In particular,
\begin{equation}
\{ f_Y^{-1}(\mathcal{U}):Y\in \mathcal{Y},\ \mathcal{U}\text{ a uniform cover of }Y\text{.}\}
\end{equation}
generates this with the initial uniformity.
\end{rmk}
\begin{proof}
We leave the proof as an exercise.
\begin{exr}
Prove this result, using the proof of \cref{InitialTopology} (the defining result of the initial topology) as guidance.
\end{exr}
\end{proof}
\end{prp}
Of course, we have a result that is perfectly analogous to \cref{prp3.4.6} (a function is continuous iff its composition with each $f_Y$ is continuous).
\begin{prp}\label{prp4.2.54}
Let $X$ have the initial uniformity with respect to the collection $\{ f_Y:Y\in \mathcal{Y}\}$, let $Z$ be a uniform space, and let $f:Z\rightarrow X$ be a function.  Then, $f$ is uniformly-continuous iff $f_Y\circ f$ is uniformly-continuous for all $Y\in \mathcal{Y}$.  Furthermore, the initial uniformity is the unique uniformity with this property.
\begin{proof}
We leave the proof as an exercise.
\begin{exr}
Prove this result, using the proof of \cref{prp3.4.6} (the analogous result for the initial topology) as guidance.
\end{exr}
\end{proof}
\end{prp}
And just as we had with topological spaces, there is a `dual' version of the initial uniformity.
\begin{prp}[Final uniformity]\label{FinalUniformity}
Let $X$ be a set, let $\mathcal{Y}$ be an indexed collection of uniform spaces, and for each $Y\in Y$ let $f_Y:Y\rightarrow X$ be a function.  Then, there exists a unique uniformity $\widetilde{\mathcal{U}}$ on $X$, the \emph{final uniformity}\index{Final uniformity} with respect to $\{ f_Y:Y\in \mathcal{Y}\}$, such that
\begin{enumerate}
\item $f_Y:Y\rightarrow X$ is uniformly-continuous with respect to $\widetilde{\mathcal{U}}$; and
\item if $\widetilde{\mathcal{U}}'$ is another uniformity for which each $f_Y$ is uniformly-continuous, then $\widetilde{\mathcal{U}}\supseteq \widetilde{\mathcal{U}}'$.
\end{enumerate}
Furthermore,
\begin{equation}
\widetilde{\mathcal{U}}=\{ \mathcal{U}\text{ a cover of }X:f_Y^{-1}(\mathcal{U})\text{ is a uniform cover for all }Y\in \mathcal{U}\text{.}\} .
\end{equation}
\begin{rmk}
In other words, the final uniformity is the largest uniformity for which each $f_Y$ is uniformly-continuous.
\end{rmk}
\begin{rmk}
But what about the smallest such uniformity?  Well, the smallest such uniformity is always going to be the indiscrete uniformity, which is not very interest.  This is how you remember whether the final uniformity is the smallest or largest---it can't be the smallest because the indiscrete uniformity always works.
\end{rmk}
\begin{proof}
We leave the proof as an exercise.
\begin{exr}
Prove this result, using the proof of \cref{FinalTopology} (the defining result of the final topology) as guidance.
\end{exr}
\end{proof}
\end{prp}
And the result `dual' to \cref{prp4.2.54}:
\begin{prp}
Let $X$ have the final uniformity with respect to the collection $\{ f_Y:Y\in \mathcal{Y}\}$, let $Z$ be a uniform space, and let $f:X\rightarrow Z$.  Then, $f$ is uniformly-continuous iff $f_Y\circ f$ is uniformly-continuous for all $Y\in \mathcal{Y}$.  Furthermore, the final uniformity is the unique uniformity with this property.
\begin{proof}
We leave the proof as an exercise.
\begin{exr}
Prove this result, using the proof of \cref{prp3.4.34x} (the analogous result for the final topology) as guidance 
\end{exr}
\end{proof}
\end{prp}
Of course, just as with topological spaces, a key application of the initial and final uniformities is that they provide canonical uniformities on subsets, quotients, products, and disjoint-unions.  The definitions and results are completely analogous to the case of topological spaces, and so we omit stating them explicitly.

After having discussed the real numbers themselves as a uniform space, we show below (\cref{exm4.2.85}) that functions even as nice as polynomials are not uniformly continuous.  On the other hand, when restricted to \emph{quasicompact} sets, all continuous functions are uniformly-continuous.
\begin{prp}\label{prp4.2.73}
Let $f:X\rightarrow Y$ be a continuous function between uniform spaces and let $K\subseteq X$ be quasicompact.  Then, $\restr{f}{K}:K\rightarrow Y$ is uniformly-continuous.
\begin{rmk}
Ideally we would have presented this result shortly after giving the definition of uniformly-continuous functions, however, we do technically need the notion of the subspace uniformity to state this result.
\end{rmk}
\begin{proof}
Let $\mathcal{V}$ be a uniform cover of $Y$.  We would like to show that $f^{-1}(\mathcal{V})\wedge \{ K\}$ is a uniform-cover of $K$.  To do this, by upward-closedness, it suffices to find a uniform-cover of $K$ which star-refines $f^{-1}(\mathcal{V})\wedge \{ K\}$.

$f^{-1}(\mathcal{V})$, while not necessarily a uniform cover of $X$, will certainly be an open cover, and in particular will be an open cover of $K$.  So, for $x\in K$, let $V_x\in \mathcal{V}$ be such that $x\in f^{-1}(V_x)$.  Then, choose a uniform cover $\mathcal{U}_x$ of $X$ such that
\begin{equation}
\Star _{\mathcal{U}_x\wedge \{ K\}}(x)\subseteq f^{-1}(V_x)\cap K.
\end{equation}
As
\begin{equation}
\left\{ \Star _{\mathcal{U}_x\wedge \{ K\}}(x):x\in K\right\}
\end{equation}
is an open cover of $K$, there is a finite subcover.  So, let $x_1,\ldots ,x_m\in K$ be such that
\begin{equation}
\left\{ \Star _{\mathcal{U}_{x_k}\wedge \{ K\}}(x_k):1\leq k\leq m\right\}
\end{equation}
is an open cover of $K$.  Let $\mathcal{U}$ be a common star-refinement of each $\mathcal{U}_k$\footnote{It is here that the finiteness given to us by quasicompactness is key.}.  Then,
\begin{equation}
\Star _{\mathcal{U}_0\wedge \{ K\}}(x)\subseteq f^{-1}(V_x)\cap K
\end{equation}
for all $x\in K$.

Let $\mathcal{U}$ be in turn a star-refinement of $\mathcal{U}_0$.  We show that $\mathcal{U}\wedge \{ K\}$ is a star-refinement of $f^{-1}(\mathcal{V})\wedge \{ K\}$.  So, let $U\in \mathcal{U}$.  Let $U_0\in \mathcal{U}_0$ be such that $\Star _{\mathcal{U}}(U)\subseteq U_0$.  Let $x\in U$.  Then,
\begin{equation}
\Star _{\mathcal{U}\wedge \{ K\}}(U\cap K)\subseteq U_0\cap K\subseteq \Star _{\mathcal{U}_0\wedge \{ K\}}(x)\subseteq f^{-1}(V_x)\cap K.
\end{equation}
\end{proof}
\end{prp}

\horizontalrule

\begin{exm}[The real numbers]
The real numbers have a canonical uniformity (and in fact, we will see below that this is just a special base of a more general construction):  let $\varepsilon >0$ and define
\begin{equation}
\mathcal{U}_{\varepsilon}\coloneqq \left\{ B_{\varepsilon}(x):x\in \R \right\}
\end{equation}
\index[notation]{$\mathcal{U}_{\varepsilon}$} and
\begin{equation}
\widetilde{\mathcal{U}}\coloneqq \left\{ \mathcal{U}_{\varepsilon}:\varepsilon >0\right\} .
\end{equation}
\begin{exr}
Show that $\widetilde{\mathcal{U}}$ is a uniform base on $\R$.
\end{exr}
\begin{exr}
Show that $f:\R \rightarrow \R$ is uniformly-continuous iff for every $\varepsilon >0$ there is some $\delta >0$ such that $f(B_{\delta}(x))\subseteq B_{\varepsilon}(f(x))$ for every $x\in \R$.
\begin{rmk}
Compare this with the condition for $f:\R \rightarrow \R$ being \emph{continuous at $a\in \R$} given in \cref{exr3.4.5}\ref{enm3.4.5.iii}.  We will spell-it-out here for convenience:
\begin{textequation}
$f:\R \rightarrow \R$ is continuous iff for every $x\in \R$ and for every $\varepsilon >0$ there is some $\delta >0$ such that $f(B_{\delta}(x))\subseteq B_{\varepsilon}(f(x))$.
\end{textequation}
The key difference between continuity and uniform-continuity is \emph{the location in which the quantification ``for every $x\in \R$'' appears}.  In the former (just continuous case), your choice of $\delta$ is \emph{allowed to depend on $x$}, whereas to be uniformly-continuous, \emph{a single $\delta$ has to `work' for every $x\in \R$}.
\end{rmk}
\begin{rmk}
The result you just proved characterizing uniform-continuity in $\R$ is often taken as the definition of uniform-continuity.  Had we studied uniform-continuity in the context of just the real numbers first (as opposed to in the context of uniform spaces), we would have done the same.  My personal feeling, however, is that uniform continuity is not that incredibly important, at least not to the point where it is worth going out of our way to discuss it just in the context of $\R$.  The real reason we discuss uniform spaces is for the purpose of discussing cauchyness and completeness, not uniform continuity per se (and also of course because a huge collection of examples of topological spaces are canonically uniform spaces).
\end{rmk}
\end{exr}
\end{exm}
\begin{exm}[A uniformly-continuous function]
By \cref{prp4.2.73}, any continuous function restricted to a quasicompact set will be uniformly-continuous, so, for example the function $x\mapsto x^2$ is uniformly-continuous on $[0,1]$.  However, be careful:  it is not uniformly-continuous on all of $\R$.
\end{exm}
\begin{exm}[A continuous function that is not uniformly-continuous]\label{exm4.2.85}
Define $f:\R \rightarrow \R$ by $f(x)\coloneqq x^2$.  Of course $f$ is continuous (because it is the product of continuous functions---see \cref{exr3.4.12}).

On the other hand, we show that $f$ does not satisfy the condition given in the previous exercise.  Take $\varepsilon \coloneqq 1$.  Then, if $f$ were uniformly-continuous, there should be some $\delta >0$ such that
\begin{equation}
\left\{ x^2:\abs{x-x_0}<\delta \right\} \subseteq \left\{ x\in \R :\abs{x-x_0^2}<1\right\} 
\end{equation}
for all $x_0\in \R$.  However, $(x_0+\frac{1}{2}\delta )^2$ is an element of the left-hand side, but
\begin{equation}
\left( x_0+\tfrac{1}{2}\delta \right) ^2-x_0^2=\delta x_0+\tfrac{1}{4}\delta ^2
\end{equation}
is not less than $1$ in general (for example, for $x_0=\frac{1}{\delta}(1-\tfrac{1}{4}\delta ^2)$.

On the other hand, by \cref{prp4.2.73}, $f$ restricted to any closed interval is uniformly-continuous.
\end{exm}

\section{Semimetric spaces and topological groups}

As was previously mentioned, one big motivation for studying uniform spaces is that a huge collection of very important examples of topological spaces admit a canonical uniformity.  Two such families of spaces that we will study are \emph{semimetric spaces} and \emph{topological groups}.

\subsection{Semimetric spaces}

Before we talk about any sort of uniformity, we had better first say what we mean by \emph{semimetric space}.
\begin{dfn}[Semimetric and metric]\label{Semimetric}
Let $X$ be a set.  Then, a \emph{semimetric}\index{Semimetric} on $X$ is a function $\metric:X\times X\rightarrow \R _0^+$ such that
\begin{enumerate}
\item (Symmetry) $\metric[x][y]=\metric[y][x]$; and
\item (Triangle Inequality) $\metric[x][z]\leq \metric[x][y]+\metric[y][z]$.
\end{enumerate}
$\metric$ is a \emph{metric}\index{Metric} if furthermore (Definiteness) $\metric[x][y]=0$ implies $x=y$.
\begin{rmk}
Semimetrics are also sometimes called \emph{pseudometric}\index{Pseudometrics}.  However, the term seminorm (something we haven't discussed yet---see \cref{Seminorm}) is actually much more common than either of these terms, and as metrics are to norms (also something we haven't discussed yet---see \cref{Seminorm} again) as semimetrics/pseudometrics are to seminorms, we feel as if the terminology ``semimetric'' is more appropriate.
\end{rmk}
\begin{rmk}
It is much more common to denote (semi)metrics by ``$d(-,-)$'', however, this conflicts with our conventions of reserving the letter ``$d$'' for dimension and differentials.
\end{rmk}
\end{dfn}
\begin{exm}
Let $X\coloneqq \R$ and define $\metric[x][y]\coloneqq \abs{x-y}$.  Then, $\metric$ is in fact a metric.
\begin{rmk}
Of course, this is where the notation $\metric$ in general comes from.
\end{rmk}
\end{exm}
\begin{exm}[A semimetric that is not a metric]\label{exm4.4.3}
Let $X$ be a topological space and let $K\subseteq X$ be quasicompact.  For $f,g\in \Mor _{\Top}(X,\R )$, we define
\begin{equation}\label{4.4.4}
\metric[f][g]_K\coloneqq \sup _{x\in K}\{ \abs{f(x)-g(x)}\} .\footnote{We require that $K$ be quasicompact so that $f-g$ is bounded on $K$ (by the \nameref{ExtremeValueTheorem} (\cref{ExtremeValueTheorem}))}.
\end{equation}
In general, this will not be a metric.  For example, take $X\coloneqq \R$ and $K\coloneqq [0,1]$.  Then,  $\abs{f,0}_K=0$ iff $\restr{f}{[0,1]}=0$, but of course, there are many nonzero real-valued continuous functions on $\R$ that vanish on $[0,1]$ (by Urysohn's Lemma (\cref{UrysohnsLemma}), for example, if you want to use a sledgehammer (or maybe just a hammer?) to swat a fly).
\end{exm}
\begin{dfn}[Semimetric space]\label{Semimetric space}
A \emph{semimetric space}\index{Semimetric space} is a set $X$ equipped with a collection $\mathcal{D}$ of semimetrics such that if $\metric[x][y]=0$ for all $\metric \in \mathcal{D}$, then $x=y$.
\begin{rmk}
For some reason, it seems that semimetric spaces are also referred to as \emph{gauge spaces}\index{Gauge spaces}.  Off the top of my head, I can think of at least two other distinct ways in which the term ``gauge'' is used in mathematics, and so I would recommend not using this terminology
\end{rmk}
\end{dfn}
\begin{dfn}[Metric space]\label{MetricSpace}
A \emph{metric space}\index{Metric space} $(X,\abs{-,-})$ is a semimetric space $(X,\mathcal{D})$ in which $\mathcal{D}$ is a singleton, $\mathcal{D}=\{ \metric \}$.
\begin{rmk}
Of course, the definiteness condition in the definition of a semimetric space forces $\abs{-,-}$ to be a metric.
\end{rmk}
\begin{rmk}
The reason we take $\mathcal{D}$ to be a singleton instead of just an arbitrary collection of \emph{metrics} is to agree with standard terminology (metric spaces are almost always taken to be sets equipped with a (\emph{single}) metric).
\end{rmk}
\end{dfn}
\begin{exm}[A semimetric space that is not a metric space]
Let $X$ be a topological space, and for $K\subseteq X$ quasicompact nonempty, let $\metric _K$ be the semimetric on $\Mor _{\Top}(X,\R )$ in \eqref{4.4.4}, that is
\begin{equation}
\metric[f][g]\coloneqq \sup _{x\in K}\{ \abs{f(x)-g(x)}\} .
\end{equation}
We already know from \cref{exm4.4.3} that each $\metric _K$ is a semimetric on $\Mor _{\Top}(X,\R )$.  What we need to show that $\metric[f][g]_K=0$ for all $K\subseteq X$ quasicompact implies $f=g$.  This however follows from the fact that $K=\{ x\}$ for $x\in X$ is quasicompact (do you see why?).

As $\mathcal{D}$ clearly contains more than one element (at least so long as $X$ contains more than one point), you might think that this shows that this cannot be a metric space.  However, the real question is \emph{is it uniformly-homeomorphic} to a metric space?\footnote{Of course, this doesn't quite make sense yet as we have not put a uniformity on semimetric spaces.}  While not a proof, it is clear that it should not be as, unless $X$ is quasicompact itself, no element of $\mathcal{D}$ will actually be a metric.
\end{exm}

It's worth nothing that, in a metric space, for every closed subset, the distance (as defined below in \eqref{4.8.50} from a point to the closed subset is a continuous function.
\begin{prp}\label{prp4.8.49}
Let $\coord{X,\metric}$ be a metric space and let $C\subseteq X$.  Then, the function $\dist _C:X\rightarrow \R$ defined by
\begin{equation}\label{4.8.50}
\dist _C(x)\coloneqq \inf _{c\in C}\{ \metric[x][c]\} 
\end{equation}\index[notation]{$\dist _C(x)$}
is uniformly-continuous and furthermore $\dist _C^{-1}(0)=C$.
\begin{proof}
Let $\varepsilon >0$.  Let $x_1,x_2\in X$ lie in some $\varepsilon$ ball.  Choose some $c\in C$ such that $\metric[x_1][c]-\dist_C(x_1)<\varepsilon$..  Then,
\begin{equation}
\dist _C(x_2)\leq \metric[x_2][c]\leq \metric[x_2][x_1]+\metric[x_1][c]<2\varepsilon +\dist _C(x_1),
\end{equation}
and so
\begin{equation}
\dist _C(x_2)-\dist _C(x_1)<2\varepsilon .
\end{equation}
By $1\leftrightarrow 2$ symmetry, we also have that
\begin{equation}
\dist _C(x_1)-\dist _C(x_2)<2\varepsilon ,
\end{equation}
and hence
\begin{equation}
\abs{\dist _C(x_1)-\dist _C(x_2)}<2\varepsilon .
\end{equation}
This shows that $\dist _C$ is uniformly-continuous.

Of course $C\subseteq \dist _C^{-1}(0)$.  On the other hand, if $x\in \dist _C^{-1}(0)$, then $x$ is an accumulation point of $C$, and hence contained in $C$.  Thus, $C=\dist _C^{-1}(0)$.
\end{proof}
\end{prp}
\begin{prp}\label{prp5.4.13}
Metric spaces are uniformly-perfectly-$T_4$.
\begin{proof}
Let $X$ be a metric space and let $C_1,C_2\subseteq X$ be closed and disjoint.
\begin{exr}
Show that there is some $f_1:X\rightarrow [0,\frac{1}{2}]$ uniformly-continuous, equal to $0$ precisely on $C_1$, and equal to $\frac{1}{2}$ on $C_2$.  Similarly, show that there is some $f_2:X\rightarrow [0,\frac{1}{2}]$ uniformly-continuous, equal to $\frac{1}{2}$ precisely on $C_2$, and equal to $0$ on $C_1$.
\end{exr}
Define $f\coloneqq f_1+f_2$.  Then, this is certainly $0$ on $C_1$ and $1$ on $C_2$.  Conversely, suppose that $f(x)=0$.  Then, in particular, $f_1(x)=0=f_2(x)=0$, and so in particular $x\in C_1$.  On the other hand, suppose that $f(x)=1$.  Then, we must have in particular that $f_2(x)=\frac{1}{2}$, which implies that $x\in C_2$.  Thus, $f^{-1}(0)=C_1$ and $f^{-1}(1)=C_2$, and hence $X$ is perfectly-$T_4$.
\end{proof}
\end{prp}

Now that we've gotten that out of the way, we are ready to equip semimetric spaces with a topology and uniformity.
\begin{dfn}[Uniformity on a semimetric space]\label{dfnB.10}
\begin{savenotes}
Let $(X,\mathcal{D})$ be a semimetric space, and equip $X$ with the uniformity generated by the uniform base defined by
\begin{equation}\label{B.11}
\widetilde{\mathcal{B}}_{\mathcal{D}}\coloneqq \left\{ U_{\varepsilon _1,\ldots ,\varepsilon _n}^{\metric _1,\ldots ,\metric _m}:m\in \Z ^+;\ \metric _1,\ldots ,\metric _m\in \mathcal{D};\ \varepsilon _1,\ldots ,\varepsilon _m>0\right\} ,\index[notation]{$\widetilde{\mathcal{U}}_{\mathcal{D}}$}
\end{equation}
where
\begin{equation}\label{1.12}
\mathcal{B}_{\varepsilon _1,\ldots ,\varepsilon _n}^{\metric _1,\ldots ,\metric _m}\coloneqq \left\{ B_{\varepsilon _1,\ldots ,\varepsilon _n}^{\metric _1,\ldots ,\metric _m}(x):x\in X\right\}
\end{equation}
and
\begin{equation}\label{1.13}
B_{\varepsilon _1,\ldots ,\varepsilon _m}^{\metric _1,\ldots ,\metric _m}\coloneqq \left\{ y\in X:\metric[y][x]_1<\varepsilon _1,\ldots ,\metric[y][x]_m<\varepsilon _m\right\} .
\end{equation}
\begin{exr}
Show that $\widetilde{\mathcal{B}}_{\mathcal{D}}$ is indeed a uniform base.
\end{exr}
\end{savenotes}
\end{dfn}
\begin{exm}[Discrete metric]
Not only does the discrete topology come from a uniformity, but so to does the discrete uniformity in turn come from a metric.

Let $X$ be a set and for $x,y\in X$, define
\begin{equation}
\metric[x][y]\coloneqq \begin{cases}0 & \text{if }x=y \\ 1 & \text{otherwise}\end{cases}.
\end{equation}
\begin{exr}
Show that $\metric$ is indeed a metric on $X$.
\end{exr}
\begin{exr}
Show that the uniformity defined by $\metric$ is the discrete uniformity.
\end{exr}
\end{exm}
\begin{exr}
Why does the indiscrete uniformity (on a set with at least two elements) not come from a metric?
\end{exr}
The following immediately follows from our result \cref{prpB.3.4} characterizing uniform-continuity in terms of uniform bases.
\begin{prp}\label{prp4.8.54}
Let $\coord{X,\widetilde{\mathcal{U}}}$ be a uniform space, let $\coord{Y,\metric}$ be a metric space, and let $f:X\rightarrow Y$.  Then, $f$ is uniformly-continuous iff for every $\varepsilon >0$ there is some $\mathcal{U}\in \widetilde{\mathcal{U}}$ such that for every $U\in \mathcal{U}$, whenever $x_1,x_2\in U$, it follows that $\metric[f(x_1)][f(x_2)]<\varepsilon$.
\end{prp}

But before we head onto topological groups, what about the morphisms in the category of semimetric spaces, you ask?  Good question.
\begin{dfn}[Bounded map (of semimetric spaces)]\label{BoundedMap}
Let $f:\coord{X,\mathcal{D}}\rightarrow \coord{Y,\mathcal{E}}$ be a function between semimetric spaces.  Then, $f$ is \emph{bounded}\index{Bounded map (of semimetric spaces)} iff for every $\metric _0\in \mathcal{E}$, there are \emph{finitely-many} $\metric _1,\ldots ,\metric _m\in \mathcal{D}$ and constants $K_1,\ldots ,K_m\geq 0$ such that
\begin{equation}
\metric[f(x_1)][f(x_2)]_0\leq K_1\metric[x_1][x_2]_1+\cdots +K_m\metric[x_1][x_2]_m
\end{equation}
for all $x_1,x_2\in X$.
\begin{rmk}
If $X$ and $Y$ are metric spaces with metric $\metric _X$ and $\metric _Y$ respectively, this condition reads just
\begin{equation}
\metric[f(x_1)][f(x_2)]_Y\leq K\metric[x_1][x_2]_X.
\end{equation}
In this case, $f$ is called \emph{lipschitz-continuous}\index{Lipschitz-continuous}\footnote{Dear lord.  His name is a juxtaposition of the word ``lip'' and the word ``shits''.  You have my sympathies, sir\textellipsis .}
\end{rmk}
\begin{rmk}
Of all the categories we've come across, that the bounded maps are the `right' notion of morphism between semimetric spaces is probably the least obvious.\footnote{Of course, we can declare any collection of morphisms we like.  It's just that, taking the morphisms to be \emph{all} functions when the objects are groups (for example) is not particularly useful---the category won't be able to tell that the groups are groups!}  The motivation for the definition is that, for seminormed vector spaces, this definition is equivalent to continuity---see \cref{exr4.3.57}.  Perhaps a simpler explanation is that it guarantees uniform-continuity.
\end{rmk}
\end{dfn}
\begin{exr}
Show that bounded maps between semimetric spaces are uniformly-continuous.
\end{exr}
\begin{exm}[The category of semimetric spaces]
The category of semimetric spaces is the category $\Semi \Met$ whose collection of objects $\Semi \Met _0$ is the collection of all semimetric spaces, for every semimetric space $X$ and semimetric space $Y$ the collection of all morphisms from $X$ to $Y$, $\Mor _{\Semi \Met}(X,Y)$, is precisely the set of all bounded maps from $X$ to $Y$, composition is given by ordinary function composition, and the identities of the category are the identity functions.
\begin{rmk}
Every semimetric space is canonically a uniform space---see \cref{dfnB.10}.  By the previous exercise, every bounded map is likewise uniformly-continuous.  Therefore, in fact, the category $\Semi \Met$ \emph{embeds} in $\Uni$.\footnote{The thing to take note of is that \emph{both} the objects \emph{and} the morphisms have to be contained in $\Uni$.}
\end{rmk}
\end{exm}
\begin{exm}[A lipschitz-continuous function]
The function $x\mapsto x$ from $\R$ to $\R$.
\end{exm}
\begin{exm}[A uniformly-continuous function that is not lipschitz-continuous]\label{exm4.3.34}
Define $f:[0,1]\rightarrow \R$ by $f(x)\coloneqq \sqrt{x}$.  This function is continuous on $[0,1]$, and hence uniformly-continuous because $[0,1]$ is quasicompact by the \nameref{HeineBorelTheorem} (and by \cref{prp4.2.73}).  On the other hand, to show that it is \emph{not} lipschitz-continuous, we need to show that
\begin{equation}
\frac{\sqrt{x}-\sqrt{y}}{x-y}
\end{equation}
is \emph{not} bounded for $x,y\in [0,1]$ distinct.  However, simply take $x=0$.  Then, we need to show that
\begin{equation}
\frac{\sqrt{y}}{y}=\frac{1}{\sqrt{y}}
\end{equation}
is not bounded on $[0,1]$.  Equivalently, you can show that $\lim _{y\to 0^+}\sqrt{y}=0$.\footnote{We have technically not defined one-sided limits.  If this bothers you, it's not a bad exercise to try to come-up with the definition yourself.}
\end{exm}

\subsection{Topological groups}

Before we talk about any sort of uniformity, we had better first define what we mean by a topological group.
\begin{dfn}[Topological group]\label{TopologicalGroup}
A \emph{topological group} is a group $\coord{G,\cdot ,1,\blank ^{-1}}$ (\cref{Group}) equipped with a topology such that
\begin{enumerate}
\item $\cdot :G\times G\rightarrow G$ is continuous; and
\item $\blank ^{-1}:G\rightarrow G$ is continuous.
\end{enumerate}
\begin{rmk}
That is to say, a topological group is a thing that is both a group and a topological space, subject to a couple of `compatibility' axioms that demand that the two structures `work together'.  This is very analogous to our definition of preordered rgs (\cref{dfn1.1.38})---a preordered rg is both a preordered set and a rg subject to a couple of ``compatibility conditions''.  This idea is not uncommon throughout all of mathematics, and, as you might have expected by this point, can be unified with the use of categories.
\end{rmk}
\begin{rmk}
Recall that in a remark of the definition of a group (\cref{Group}), we made a slight deal about ``having inverses'' not being stated as an \emph{extra property} but rather as \emph{extra structure}.  This is one reason why.  When we go to define a topological group, the the operation of taking inverses should be thought of as just that---an operation, on the same footing as the product.  If we think of the operation of taking inverses as on the same footing as the product, then we almost have to also assume that the inverse operation is likewise continuous, whereas if it were thought of just as an existence property, it would not make as much sense to do this.
\end{rmk}
\end{dfn}
\begin{exm}[The category of topological groups]
The category of topological groups is the category $\Top \Grp$\index[notation]{$\Top \Grp$} whose collections of objects $\Top \Grp _0$ is the collection of all topological groups, for every topological group $G$ and topological group $H$ the collection of morphisms from $G$ to $Y$, $\Mor _{\Top \Grp}(G,H)$, is precisely the set of all continuous group homomorphisms from $G$ to $H$, composition is given by ordinary function composition, and the identities of the category are the identity functions.
\end{exm}
\begin{exm}[A topological group that is not $T_0$]
\begin{savenotes}
Define $G\coloneqq \R /\Q$.\footnote{This is the quotient rng construction---see \cref{IdealsAndQuotientGroups}.}  Let $\q :\R \rightarrow G$ be the quotient map (i.e.~the map that sends an element to its equivalence class) and equip $\R /\Q$ with the quotient topology.
\begin{exr}
Show that $+:G\times G\rightarrow G$ and $\blank ^{-1}:G\rightarrow G$ are continuous.
\end{exr}
We now check that the quotient topology on $G$ is not $T_0$.  So, let $x\in \R$ be irrationals.  We wish to show every open neighborhood of $x+\Q$ contains $0+\Q$ and conversely.  So, let $U\subseteq x+\Q$ be open.  Then, by definition, $\q ^{-1}(U)$ is an open neighborhood of $x$, and so by density, must contain some rational number $r\in \q ^{-1}(U)$.  But then, $0+\Q =r+\Q \in \q \left( \q ^{-1}(U)\right) \subseteq U$.  On the other hand, if $U$ is an open neighborhood of $0+\Q$, $\q ^{-1}(U)$ is an open neighborhood of $0$, and so by density again, must contain some $\varepsilon >0$ so that $x-\varepsilon$ is rational.  But if $x-\varepsilon \in \Q$, then $x+\Q =\varepsilon +\Q \in \q \left( \q ^{-1}(U)\right) \subseteq U$.
\begin{rmk}
We show in the next section that every $T_0$ uniform space is in fact completely-$T_3$.  Thus, this serves as an example of a uniform space which is not completely-$T_3$.
\end{rmk}
\end{savenotes}
\end{exm}
A large number of examples of topological groups arise from totally-ordered rngs (or more generally, totally-ordered commutative groups).
\begin{exr}\label{exr4.8.58}
Let $G$ be a totally-ordered commutative group.  Show that $G$ is a topological group with respect to the order topology.
\end{exr}

Before we put a uniformity on topological groups, it will be useful to know at least one basic fact about them.
\begin{prp}\label{prp4.8.59}
\begin{savenotes}
Let $G$ be a topological group and let $U$ be a neighborhood of the identity.  Then, there exists an open neighborhood $V$ of the identity such that (i) $VV\subseteq U$ and (ii) $V^{-1}=V$.
\begin{rmk}
The notation means what you think it means:  $VV\coloneqq \{ v_1v_2:v_1\in V,\ v_2\in V\}$ (note how this is not the same as $V^2$) and $V^{-1}\coloneqq \{ v^{-1}:v\in V\}$.
\end{rmk}
\begin{proof}
Regarding the group operation $\cdot$ as a function from $G\times G$ to $G$, we know that $+^{-1}(U)$ is an open neighborhood of $\coord{1,1}$ in $G\times G$, and therefore we have that $V\times W\subseteq \cdot ^{-1}(U)$ for some $V,W\subseteq G$ open neighborhoods of the identity (by the definition of the product topology \cref{ProductTopology}).   Replace $V$ with $V\cap W$, another open neighborhood of the identity, so that $V\times V\subseteq \cdot ^{-1}(U)$.  In other words, $VV\subseteq U$.  Now do this exact same construction again and find another open neighborhood of the identity $W$ (replacing our `old' $W$) with $WW\subseteq V$.  Now define
\begin{equation}
W'\coloneqq W\cap [\blank ^{-1}](W^{-1}),
\end{equation}
that is, the intersection of $W$ with the preimage of $W^{-1}$ under the inverse function $\blank ^{-1}:G\rightarrow G$.\footnote{Yes, I am aware that this notation is ridiculously obtuse.}  This will be yet another open neighborhood of the identity, with both $W',(W')^{-1}\subseteq W$.  Finally, define
\begin{equation}
W''\coloneqq (W')(W')^{-1}.
\end{equation}
This certainly satisfies $(W'')^{-1}=W''$, and furthermore,
\begin{equation}
W''W''\coloneqq (W')(W')^{-1}(W')(W')^{-1}\subseteq WWWW\subseteq VV\subseteq U.
\end{equation}
\end{proof}
\end{savenotes}
\end{prp}

\begin{dfn}[Uniformity on a topological group]\label{dfnB.7}
Let $G$ be a topological group, and equip $G$ with the uniformityd generated by the uniform base defined by
\begin{equation}\label{B.8}
\widetilde{\mathcal{B}}_G\coloneqq \left\{ \mathcal{B}_U:U\ni 1\text{ is open.}\right\} ,\index[notation]{$\widetilde{\mathcal{B}}_G$}
\end{equation}
where
\begin{equation}\label{1.9}
\mathcal{B}_U\coloneqq \left\{ gU:g\in G\right\} .
\end{equation}
\index[notation]{$\mathcal{B}_U$}
\begin{exr}
Show $\widetilde{\mathcal{B}}_G$ is indeed a uniform base.
\end{exr}
\begin{rmk}
Note that we could have equally well taken the covers $U_G$ for only $U\in \mathcal{N}$, $\mathcal{N}$ a fixed neighborhood base of the identity (by \cref{prp1.6}).
\end{rmk}
\end{dfn}
We have a potential problem here---$G$ started its life as a topological group, and in particular, as a topological space.  We then equipped it with a uniform structure, from which it obtains the uniform topology.  The question arises:  ``Are these topologies the same, and if not, which one should we use?''.  Fortunately, it turns out that they are the same.
\begin{exr}
Show that the topology on a topological group agrees with the uniform topology induced by the uniform base in \eqref{B.8}.
\end{exr}

We have yet another potential problem here---$\R$ is both a metric space and a topological group (with respect to $+$), so which uniformity should we use?  Fortunately, we needn't worry about this, because the two uniformities are the same.
\begin{exr}
Show that $\widetilde{\mathcal{B}}_{\coord{\R ,\metric}}$ and $\widetilde{\mathcal{B}}_{\coord{\R ,+}}$ define the same uniformity on $\R$.
\end{exr}

You might say that functional analysis is the study of topological vector spaces (the term ``functional'' a result of the fact that many `spaces' of functions are topological vector spaces).  As vector spaces are in particular a group (just forget about the scalars), everything we say regarding the uniformities of topological groups also applies to uniformities of topological vector spaces.  Thus, a knowledge of uniform spaces is very useful when studying functional analysis.  In particular, the following result is used ubiquitously (to the point where it is so common that it is not really even mentioned)
\begin{prp}\label{prpB.10}
\begin{savenotes}
Let $f:G\rightarrow H$ be a group homomorphism between topological groups and let $x_0\in G$.  Then, if $f$ is continuous at $x_0$, then $f$ is uniformly-continuous.
\begin{rmk}
This shows that the category $\Top \Grp$ \emph{embeds}\footnote{We have not defined what precisely this means for categories, but with a little mathematical maturity, you can probably figure it out.  In any case, it's okay if you don't know the precise definition} into the category $\Uni$.  We already knew that the \emph{objects} `embedded' (from the canonical uniformity on topological groups given in \cref{dfnB.7})---this result tells us furthermore that the morphisms `embed' as well.
\end{rmk}
\begin{proof}
\Step{Make hypotheses}
Suppose that $f$ is continuous at $x_0$.

\Step{Show that $f$ is continuous.}
We first show that $f$ is continuous (as opposed to just continuous at $x_0$).  To show that, we show that $f$ is continuous at $x\in G$ for arbitrary $x$.  Let $V$ be a neighborhood of $f(x)\in H$.  Then, $f(x_0)f(x)^{-1}V$ is a neighborhood of $f(x_0)\in H$.  As $f$ is continuous at $x_0$, it follows that $f^{-1}\left( f(x_0)f(x)^{-1}V\right)$ is a neighborhood of $x_0$, and so $xx_0^{-1}f^{-1}\left( f(x_0)f(x)^{-1}V\right)$ is a neighborhood of $x$.\footnote{The juxtaposition here is being used to denote multiplication in the group.  Be careful not to confuse preimages with inverse elements (even though the same symbol is used, the context makes the notation unambiguous).}  However,
\begin{equation}
\begin{split}
f\left( xx_0^{-1}f^{-1}\left( f(x_0)f(x)^{-1}V\right) \right) & =f(x)f(x_0)^{-1}f\left( f^{-1}\left( f(x_0)f(x)^{-1}V\right) \right) \\
& \subseteq f(x)f(x_0)^{-1}f(x_0)f(x)^{-1}V=V,
\end{split}
\end{equation}
so that
\begin{equation}
xx_0^{-1}f^{-1}\left( f(x_0)f(x)^{-1}V\right) \subseteq f^{-1}(V),
\end{equation}
so that $f^{-1}(V)$ is a neighborhood of $x$, so that $f$ is continuous at $x$.

\Step{Show that $f$ is uniformly-continuous.}
To show that $f$ is uniformly-continuous, we apply \cref{prpB.3.4}.  So, let $V\subseteq H$ be an open neighborhood of the identity and consider the cover $\mathcal{U}_V\coloneqq \left\{ hV:h\in H\right\}$.  To show that $f^{-1}(\mathcal{U}_V)$ is a uniform cover, it suffices to find an open neighborhood $U\subseteq G$ of the identity such that $\mathcal{U}_U\llcurly f^{-1}(\mathcal{U}_V)$.  Take $U'$ to be an open neighborhood of the identity such that $U'U'\subseteq f^{-1}(V)$, and then in turn take $U$ to be an open neighborhood of the identity such that (i) $UU\subseteq U'$ and (ii) $U=U^{-1}$ (which we may do by \cref{prp4.8.59}).  We wish to show that
\begin{equation}\label{B.4.7}
\Star _{\mathcal{U}_U}(U)=\bigcup _{x\in G\st xU\cap U\neq \emptyset}xU\subseteq f^{-1}(V).
\end{equation}
It will follow from this (see \eqref{4.8.72}) that $\mathcal{U}_U\llcurly f^{-1}(\mathcal{U}_V)$.  So, let $x\in G$ be such that $xU\cap U\neq \emptyset$.  Then, there are $u_1,u_2\in U$ such that $xu_1=u_2$, so that $x=u_2u_1^{-1}\in UU^{-1}=UU\subseteq U'$.  Thus, $xU\subseteq U'U'\subseteq f^{-1}(V)$.  \eqref{B.4.7} follows from this.  From this, we have
\begin{equation}\label{4.8.72}
\begin{split}
\Star _{\mathcal{U}_U}(x_0U) & =\bigcup _{x\in G\st xU\cap x_0U\neq \emptyset}xU=\bigcup _{x\in G\st xU\cap U\neq \emptyset}x_0xU=x_0\Star _{\mathcal{U}_U}(U) \\
& \subseteq x_0f^{-1}(V)\subseteq f^{-1}(f(x_0)V)\in f^{-1}(\mathcal{U}_V),
\end{split}
\end{equation}
so that $\mathcal{U}_U\llcurly f^{-1}(\mathcal{U}_V)$.
\end{proof}
\end{savenotes}
\end{prp}

\subsection{Topological vector spaces and algebras}

An \emph{incredibly} family of examples of topological groups are the topological vector spaces.
\begin{dfn}[Topological vector space]
A \emph{topological vector space}\index{Topological vector space} is real vector space $\coord{V,+,0,\R ,\cdot}$ such that
\begin{enumerate}
\item $\coord{V,+,0}$ is a topological group; and
\item $\cdot :\R \times V\rightarrow V$ is continuous.
\end{enumerate}
\begin{rmk}
Of course, this definition makes sense if we were to replace $\R$ with any topological field\footnote{What do you think the definition of a topological field should be?}, but for our purposes, restricting ourselves to working over the reals will be sufficient.  The other case of most interest is over $\C$, but we have not even defined the complex numbers.  Most of the time, our topology comes from a collection of seminorms (see \cref{Seminorm} below), in which case it takes a fair amount of effort to work over topological fields whose topology does not come from a norm (at least in the traditional sense of the word).
\end{rmk}
\end{dfn}
One big reason why we are interested in topological vector spaces is because almost all of the examples of semimetrics we counter actually come \emph{seminorms}.
\begin{dfn}[Seminorm and norm]\label{Seminorm}
Let $V$ be a real vector space.  Then, a \emph{seminorm}\index{Seminorm} on $V$ is a function $\norm :V\rightarrow \R _0^+$ such that
\begin{enumerate}
\item (Homogeneity) $\norm[\alpha v]=\norm[\alpha ]\norm[v]$ for $\alpha \in \R$ and $v\in V$;
\item (Triangle Inequality) $\norm[v_1+v_2]\leq \norm[v_1]+\norm[v_2]$.
\end{enumerate}
$\norm$ is a \emph{norm} if furthermore (Definiteness) $\norm[x]=0$ implies $x=0$.
\end{dfn}
\begin{dfn}[Semimetric induced by a seminorm]
Let $V$ be a real vector space, let $\norm$ be a seminorm on $V$, and let $v_1,v_2\in V$.  Then, the \emph{semimetric induced by $\norm$}, $\metric$, is defined by
\begin{equation}
\metric[v_1][v_2]\coloneqq \norm[v_1-v_2].
\end{equation}
\begin{exr}
Check that $\metric$ is indeed a semimetric.  Show that if $\norm$ is a norm then $\metric$ is a metric.
\end{exr}
\begin{rmk}
Intuitively, the seminorm of something is like its `size' and semimetric is like `distance'---the `distance' between two vectors is the `size' of their difference.
\end{rmk}
\end{dfn}
\begin{dfn}[Seminormed vector space and normed vector space]\label{SeminormedVectorSpace}
A \emph{seminormed vector space}\index{Seminormed vector space} is a real vector space $V$ together with a collection of seminorms $\mathcal{D}$ such that $\coord{V,\mathcal{D}}$ is a semimetric space.  $\coord{V,\mathcal{D}}$ is a \emph{normed vector space}\index{Normed vector space} iff it is furthermore a metric space.
\end{dfn}
As by now you should have expected, the morphisms of seminormed vector spaces are the bounded (\cref{BoundedMap}) linear maps.
\begin{exm}[The category of seminormed vector spaces]
The category of seminormed vector spaces is the category $\Semi \Vect$ whose collection of objects $\Semi \Vect _0$ is the collection of all seminormed vector spaces and for every seminormed vector space $V$ and seminormed vector space $W$ the collection of all morphisms from $V$ to $W$, $\Mor _{\Semi \Vect}(V,W)$ is preicsely the set of all bounded linear maps from $V$ to $W$, composition is givne by ordinary function composition, and the identities of the category are the identity.
\end{exm}
\begin{exr}
Let $f:\coord{V,\mathcal{D}}\rightarrow \coord{W,\mathcal{E}}$ be a linear map between seminormed vector spaces.  Show that $f$ is bounded iff for every $\norm _0\in \mathcal{E}$ there are \emph{finitely-many} $\norm _1,\ldots ,\norm _m\in \mathcal{D}$ and constants $K_1,\ldots ,K_m\geq 0$ such that
\begin{equation}
\norm[f(v)]_0\leq K_1\norm[v]_1+\cdots +K_m\norm[v]_m
\end{equation}
for all $v\in V$.
\begin{rmk}
Compare this with the definition of bounded maps of semimetric spaces (\cref{BoundedMap}).  Essentially this boils down to the statement that, to check that $f$ is bounded, it suffices to check for $x_2=0$ and $x_1\eqqcolon v$ arbitrary (in the notation of \cref{BoundedMap}).
\end{rmk}
\end{exr}
Note that a priori a seminormed vector space is not a topological vector space.  However, being in metric a semimetric space, it is in fact a uniform space, and so in turn is equipped with its uniform topology.  The question is then whether it is a topological vector space with respect to the uniform topology.  Of course, the answer is in the affirmative.  Once we know that the seminormed vector space $V$ is likewise a topological vector space, we know in turn that its underlying topological group $\coord{V,+,0,-}$ induces in turn yet another uniform structure, and so a new question arises as to whether or not this uniform structures agrees with the one induced from the semimetric space structure.  Of course, the answer to this is likewise in the affirmative.
\begin{exr}
Let $V$ be a seminormed vector space.  Show that $V$ is a topological vector space with respect to the uniform topology induced by the semimetric uniformity.
\end{exr}
\begin{exr}
Let $V$ be a seminormed vector space.  Show that the uniformity induced by the topological group structure $\coord{V,+,0,-}$ is the same as the semimetric uniformity.
\end{exr}
As a mater of fact, the morphisms don't care whether you're thinking of things as a semimetric space or as a topological group either.
\begin{exr}\label{exr4.3.57}
Let $f:\coord{V,\mathcal{D}}\rightarrow \coord{W,\mathcal{E}}$ be a \emph{linear} map between two seminormed vector spaces.  Show that it is continuous iff it is bounded.
\end{exr}
\begin{textequation}
Unless otherwise stated, seminormed vector spaces are always equipped with the uniformity induced by the semimetric space structure (or, equivalently, the uniformity induced by the topological group structure).
\end{textequation}

In fact, a lot of examples of seminormed vector spaces have \emph{even more} structure, namely, the structure of an algebra.
\begin{dfn}[Algebra]\label{Algebra}
An \emph{algebra}\index{Algebra} is a set $A$ equipped with the structure of a vector space over a field $F$ and the structure of a ring such that
\begin{enumerate}
\item $(\alpha _1\alpha _2)\cdot a=\alpha _1\cdot (\alpha _2\cdot a)$ for $\alpha _1,\alpha _2\in F$ and $a\in A$; and
\item $\alpha \cdot (a_1a_2)=(\alpha \cdot a_1)a_2$ for $\alpha \in F$ and $a_1,a_2\in A$.
\end{enumerate}
\begin{rmk}
That is, an algebra is both a vector space and a ring subject to a couple of compatibility axioms.
\end{rmk}
\end{dfn}
\begin{dfn}[Homomorphism (of algebras)]\label{HomomorphismOfAlgebras}
Let $A$ and $B$ be algebras and let $f:A\rightarrow B$ be a function.  Then, $f$ is a \emph{homomorphism}\index{Homomorphism (of algebras)} iff $f$ is both a linear map of the underlying vectors spaces and a ring homomorphism of the underlying vector spaces.
\end{dfn}
\begin{exm}[The category of algebras over a field $F$]
The category of algebras over a field $F$ is the category $\Alg _F$ whose collection of objects is the collection of all algebras over $F$, for every algebra $A$ and algebra $B$ over $F$ the collection of all morphisms from $X$ to $Y$, $\Mor _{\Alg _F}(A,B)$, is precisely the set of all homomorphisms from $A$ to $B$s, composition is given by ordinary function composition, and the identities of the categories are the identity functions.
\end{exm}
\begin{dfn}[Seminormed algebra]\label{SeminormedAlgebra}
A \emph{seminormed algebra}\index{Seminormed algebra} is an algebra whose underlying vector space is a seminormed vector space such that $\norm[a_1a_2]\leq \norm[a_1]\norm[a_2]$ for $a_1,a_2\in A$.
\end{dfn}
\begin{exm}[The category of seminormed algebras]
The category of seminormed algebras is the category $\Semi \Alg$ whose collection of objects $\Semi \Alg _0$ is the collection of all seminormed algebras, for every seminormed algebras $A$ and seminormed algebra $B$ the collection of all morphisms from $A$ to $B$, $\Mor _{\Semi \Alg}(A,B)$, is precisely the set of all bounded homomorphisms, composition is given by ordinary function composition, and the identities of the category are the identity functions.
\end{exm}
With these new definitions in hand, we now present an incredibly important example of a seminormed algebra, an example that was a large part of the motivation for introducing seminormed algebras at all.
\begin{exm}[Uniform convergence on quasicompact subsets]\label{exm4.3.60}
Let $X$ be a topological space and define
\begin{equation}
A\coloneqq \Mor _{\Top}(X,\R ).
\end{equation}
Pointwise addition and pointwise scalar multiplication gives $A$ the structure of a real vector space.  Pointwise multiplication gives $A$ in turn the structure of a real algebra.  The collection $\{ \norm _K:K\subseteq X\text{ quasicompact}\}$, where
\begin{equation}
\norm[f]_K\coloneqq \sup _{x\in K}\{ \abs{f(x)}\} 
\end{equation}
then gives $A$ the structure of a seminormed algebra.  It is thus canonically a uniform space (and in turn a topological space).  If $X$ itself is quasicompact, convergence in $\Mor _{\Top}(X,\R )$ is called \emph{uniform convergence}\index{Uniform convergence}.\footnote{Despite the name and the context in which we're presenting it, uniform convergence actually has nothing to do with uniform spaces per se (in contrast to uniform continuity, for example).  $\Mor _{\Top}(X,\R )$ has a topology, and hence a notion of convergence, which we happen to call ``uniform convergence''.  In particular, we only needed to equip $\Mor _{\Top}(X,\R )$ with a topology to define uniform convergence.  The reason we waited until the chapter on uniform spaces, of course, is because $\Mor _{\Top}(X,\R )$ obtains is topology from a family of semimetrics (or in the case $X$ is quasicompact, just a single metric), not because of any direct connection with uniform convergence and uniform spaces.}  Thus, in the general case, people refer to convergence in the topological space $\Mor _{\Top}(X,\R )$ as \emph{uniform convergence on quasicompact subsets}.

The reason the case $X$ quasicompact is special is because, in this case, it is actually isomorphic (in the category of seminormed algebras) to a normed algebra.
\begin{exr}
Let $X$ be quasicompact.  Show that
\begin{equation}
\id _X:\coord{X,\{ \norm _X\}}\rightarrow \coord{X,\{ \norm _K:K\subseteq X\text{ quasicompact}\}} .
\end{equation}
is an isomorphism in the category of seminormed algebras.
\begin{rmk}
The point is that, if all we care about is the seminormed algebra structure, we may always assume without loss of generality that $\Mor _{\Top}(X,\R )$ is in fact a \emph{normed} algebra with the single norm being given by $\norm[f]_X\coloneqq \sup _{X\in X}\{ \abs{f(x)}\}$.
\end{rmk}
\end{exr}
\begin{textequation}
Unless otherwise stated, $\Mor _{\Top}(X,\R )$ is a always given the structure of a seminormed algebra, the algebra structure defined pointwise and the seminorms being $\norm _K$ for $K\subseteq X$ quasicompact.
\end{textequation}

Of \emph{incredible} importance is that this space is in fact complete (at least for so-called \emph{quasicompactly-generated} spaces).  Of course, we need to first actually define what we mean by complete, and so we postpone this result---see \cref{thm4.5.6}.
\end{exm}

\section{$T_0$ uniform spaces are uniformly-completely-$T_3$}

Our goal in this subsection is to show that all $T_0$ uniform spaces are uniformly-$T_3$.  Of course, to prove this, we had better say what uniformly-completely-$T_3$ means.

\subsection{Separation axioms in uniform spaces}

Throughout this subsection, let $S_1,S_2\subseteq X$ be \emph{disjoint} subsets of a uniform space $X$.
\begin{dfn}[Uniformly-distinguishable]\label{UniformlyDistinguishable}
$S_1$ and $S_2$ are \emph{uniformly-distinguishable}\index{Uniformly-distinguishable} iff there is some uniform cover $\mathcal{U}$ for which $\Star _{\mathcal{U}}(S_1)\neq \Star _{\mathcal{U}}(S_2)$.
\end{dfn}
\begin{dfn}[Uniformly-separated]\label{UniformlySeparated}
$S_1$ and $S_2$ are \emph{uniformly-separated}\index{Uniformly-separated} iff there is some uniform cover $\mathcal{U}$ for which $\star _{\mathcal{U}}(S_1)$ is disjoint from $\Star _{\mathcal{U}}(S_2)$.
\begin{rmk}
Note that this is analogous to the topological condition of being ``separated by neighborhoods'' (\cref{SeparatedByNeighborhoods})---there is not really any uniform analgoe of just being plain separated (\cref{Separated}).
\end{rmk}
\end{dfn}
\begin{dfn}[Uniformly-completely-separated]\label{UniformlyCompletelySeparated}
$S_1$ and $S_2$ are \emph{uniformly-completely-separated}\index{Uniformly-completely-separated} iff there is a uniformly-continuous function $f:X\rightarrow [0,1]$ such that $\restr{f}{S_1}=0$ and $\restr{f}{S_2}=1$.
\end{dfn}
\begin{exr}
Show that if $S_1$ and $S_2$ are uniformly-completely-separated, then they are uniformly-separated.
\end{exr}
\begin{dfn}[Uniformly-perfectly-separated]\label{UniformlyPerfectlySeparated}
$S_1$ and $S_2$ are \emph{uniformly-perfectly-separated}\index{Uniformly-perfectly-separated} iff there is a uniformly-continuous function $f:X\rightarrow [0,1]$ such that $S_1=f^{-1}(0)$ and $S_2=f^{-1}(1)$.
\end{dfn}
\begin{dfn}[Uniformly-$T_0$]\label{UniformlyT0}
$X$ is \emph{uniformly-$T_0$}\index{Uniformly-$T_0$} iff any two distinct points are uniformly-distinguishable.
\end{dfn}
\begin{dfn}[Uniformly-$T_2$]\label{UniformlyT2}
$X$ is \emph{uniformly-$T_2$}\index{Uniformly-$T_2$} iff any two distinct points can be uniformly-separated.
\end{dfn}
\begin{dfn}[Uniformly-completely-$T_2$]\label{UniformlyCompletelyT2}
$X$ is \emph{uniformly-completely-$T_2$}\index{uniformly-completely-$T_2$} iff any two distinct points can be uniformly-completely-separated.
\end{dfn}
\begin{dfn}[Uniformly-perfectly-$T_2$]\label{UniformlyPerfectlyT2}
$X$ is \emph{uniformly-perfectly-$T_2$}\index{uniformly-perfectly-$T_2$} iff any two distinct points can be uniformly-perfectly-separated.
\end{dfn}
\begin{dfn}[Uniformly-$T_3$]\label{UniformlyT3}
$X$ is \emph{uniformly-$T_3$}\index{Uniformly-$T_3$} iff it is $T_1$ and any closed set and a point not contained in it can be uniformly-separated.
\end{dfn}
\begin{dfn}[Uniformly-completely-$T_3$]\label{UniformlyCompletelyT3}
$X$ is \emph{uniformly-completely-$T_3$}\index{Uniformly-completely-$T_3$} iff it is $T_1$ and any closed set and a point not contained in it can be uniformly-completely-separated.
\end{dfn}
\begin{dfn}[Uniformly-perfectly-$T_3$]\label{UniformlyPerfectlyT3}
$X$ is \emph{uniformly-perfectly-$T_3$}\index{Uniformly-perfectly-$T_3$} iff it is $T_1$ and any closed set and a point not contained in it can be uniformly-perfectly-separated.
\end{dfn}
\begin{dfn}[Uniformly-$T_4$]\label{UniformlyT4}
$X$ is \emph{uniformly-$T_4$}\index{Uniformly-$T_4$} iff it is $T_1$ and any two disjoint closed subsets can be uniformly-separated.
\end{dfn}
\begin{dfn}[Uniformly-completely-$T_4$]\label{UniformlyCompletelyT4}
$X$ is \emph{uniformly-completely-$T_4$}\index{Uniformly-completely-$T_4$} iff it is $T_1$ and any two disjoint closed subsets can be uniformly-completely-separated.
\end{dfn}
\begin{dfn}[Uniformly-perfectly-$T_4$]\label{UniformlyPerfectlyT4}
$X$ is \emph{uniformly-perfectly-$T_4$}\index{Uniformly-perfectly-$T_4$} iff it is $T_1$ and any closed set and any two disjoint closed subsets can be can be uniformly-perfectly-separated.
\end{dfn}
The goal of this section is to prove that all of the from uniformly-$T_0$ to uniformly-completely-$T-3$ (that is, $T_0$ implies uniformly-completely-$T_3$---see \cref{crl4.4.16}).  We also present counter-examples to show that all these equivalent axioms are strictly weaker than both uniformly-$T_4$ (see \cref{}) and uniformly-perfectly-$T_4$ (see \cref{}

\subsection{The key result}

We actually prove a stronger result which requires the notion of the \emph{diameter} of a set (in a metric space).
\begin{dfn}[Diameter]\label{Diameter}
Let $\coord{X,\metric}$ be a metric space and let $S\subseteq X$.  Then, the \emph{diameter}\index{Diameter} of $S$, $\diam (S)$, is defined by
\begin{equation}
\diam (S)\coloneqq \sup _{x,y\in S}\{ \metric[x][y]\} .
\end{equation}
\begin{rmk}
Of course, it may be the case that $\diam (S)=\infty$.
\end{rmk}
\end{dfn}
And now we are ready to state our key result.
\begin{thm}
\begin{savenotes}
Let $\mathcal{U}$ be a uniform cover of a $T_0$ uniform space $X$.  Then, there exists a metric space $Y$ and a uniformly-continuous surjective function $\q :X\rightarrow Y$ such that, if $\diam (S)<1$ for $S\subseteq Y$, then $\q ^{-1}(S)$ will be contained in some element of $\mathcal{U}$.
\begin{proof}\footnote{Proof adapted from \cite[pg.~8]{Isbell}.}
To construct $Y$, we shall put a semimetric on $X$ and then take the quotient set with respect to the equivalence relation of `being infinitely close to each other'.

\Step{Construct a sequence of star-refinements of $\mathcal{U}$}
Let us write $\mathcal{U}_0\coloneqq \mathcal{U}$.  Then, we take a star-refinement $\mathcal{U}_1$ of $\mathcal{U}_0$, in turn another star-refinement $\mathcal{U}_2$ of $\mathcal{U}_1$, and so on.

\Step{Define $\ell (x_1,x_2)$ for $x_1,x_2\in X$}\label{stp4.8.76.2}
Define
\begin{equation}
\ell (x_1,x_2)\coloneqq \begin{cases}2 & \text{if }x_2\notin \Star _{\mathcal{U}_0}(x_1) \\ 2^{1-\max \{ m\in \N :x_2\notin \Star _{\mathcal{U}_m}(x_1)\}} & \text{otherwise}\end{cases}.
\end{equation}
Note that this in particular implies that $\ell (x_1,x_2)=0$ if $x_2\in \Star _{\mathcal{U}_m}(x_1)$ for all $m\in \N$.  (We do need to make the other extreme case explicit as the maximum of the empty-set is $-\infty$.)  Thus, the statement that $X$ is $T_0$ (together with the fact that stars for a base for the topology (\cref{UniformTopology})) implies that either $\ell (x_1,x_2)>0$ or $\ell (x_2,x_1)>0$.

\Step{Define $\ell (\mathcal{P})$ for paths $\mathcal{P}$}
For the purposes of this proof, a \emph{path} from $x_1$ to $x_2$ will be a finite sequence of points $(x^\infty,x^1,\ldots ,x^m)$ with $x^\infty=x_1$ and $x^m=x_2$.\footnote{The superscripts (as opposed to subscripts) if obviously for the purpose of not conflicting with the subscripts on $x_1$ and $x_2$.}  If $\mathcal{P}=(x^\infty,\ldots ,x^m)$, then we define $\ell (\mathcal{P})\coloneqq \ell (x^\infty,x^1)+\ell (x^1,x^2)+\cdots +\ell (x^{m-1},x^m)$.   We shall call this the \emph{length} of the path.

\Step{Define the semimetric}
Finally, we define
\begin{equation}
\metric[x_1][x_2]\coloneqq \min \{ \inf \left( \left\{ \ell (\mathcal{P}):\mathcal{P}\text{ is a path from }x_1\text{ to }x_2\text{ or a path from }x_2\text{ to }x_1\text{.}\right\} \right) ,1\} .
\end{equation}

\Step{Show that this is in fact a semimetric}
From the definition, we have that $\metric$ is symmetric (this is the reason for putting the ``or'' in the definition---note that the definition of $\ell (x_1,x_2)$ is not manifestly symmetric).  The triangle inequality follows from the fact that a path from $x_1$ to $x_3$ and a path from $x_3$ to $x_2$ gives us a path from $x_1$ to $x_2$, with the length of this new path being the sum of the lengths of the other two.  Thus, $\metric$ is in fact a semimetric.

\Step{Construct $Y$}
Define $x_1\sim x_2$ iff $\metric[x_1][x_2]=0$.  That this is an equivalence relation follows from the fact that $\metric$ is a semimetric.  Thus, we may define
\begin{equation}
Y\coloneqq X/\sim .
\end{equation}

\Step{Construct the metric on $Y$}
We abuse notation and write the induced metric on $Y$ with the same symbol $\metric$ as the semimetric on $X$:
\begin{equation}
\metric[[x_1]_{\sim}][[x_2]_{\sim}]\coloneqq \metric[x_1][x_2].
\end{equation}
\begin{exr}
Check that $\metric$ on $Y$ is well-defined.
\end{exr}

\Step{Show that this in fact a metric}
$\metric$ on $Y$ is automatically symmetric and satisfies the triangle inequality because $\metric$ on $X$ does.  Furthermore, if $\metric[[x_1]_{\sim}][[x_2]_{\sim}]=0$, then $\metric[x_1][x_2]=0$, and so $x_1\sim x_2$ by the definition of $\sim$.  Thus, $\metric$ is indeed a metric on $Y$.

\Step{Define $\q :X\rightarrow Y$}
We take $\q :X\rightarrow Y$ to be the quotient map:  $\q (x)\coloneqq [x]_{\sim}$.  Of course $\q$ is surjective (all quotient maps are).

\Step{Show that $\q$ is uniformly-continuous}
We apply \cref{prp4.8.54} which characterizes uniform-continuity for functions whose codomain is a metric space.  So, let $\varepsilon >0$.  We must find a uniform cover $\mathcal{U}$ of $X$ such that for every $U\in \mathcal{U}$, whenever $x_1,x_2\in U$, it follows that $\metric[\q (x_1)][\q (x_2)]<\varepsilon$.  It suffices to show this for $\varepsilon \coloneqq 2^{1-m}$.  We show that $\mathcal{U}_m$ is a uniform cover that `works'.  So, let $U\in \mathcal{U}_m$ and let $x_1,x_2\in U$.   Then, in particular, $x_2\in \Star _{\mathcal{U}_m}(x_1)$, and so
\begin{equation}
\metric[\q (x_1)][\q (x_2)]\coloneqq \metric[x_1][x_2]<2^{1-m}\eqqcolon \varepsilon .
\end{equation}

\Step{Finish the proof by proving the desired property of $\q$}
Let $S\subseteq Y$ and suppose that $\diam (S)<1$.  We wish to show that there is some $U\in \mathcal{U}_0$ such that $S\subseteq U$.  It suffices to show that for $x_1,x_2\in S$, there is some $U_{x_1,x_2}\in \mathcal{U}_1$ such that $x_1,x_2\in U_{x_1,x_2}$.  This is because, if this is true, then $S\subseteq \Star _{\mathcal{U}_1}(x_1)$, which in turn is contained in some element of $\mathcal{U}_0$ because $\mathcal{U}_1$ star-refines $\mathcal{U}_0$.

To show this, it suffices to show that if $\metric[x_1][x_2]<2^{1-m}$ for $m\in \Z ^+$, then there is some $U\in \mathcal{U}_m$ such that $x_1,x_2\in U$.  So, let $x_1,x_2\in X$ be such that $\metric[x_1][x_2]<2^{1-m}$.  Then, without loss of generality, there is some path $(x^\infty,\ldots ,x^n)$ from $x_1$ to $x_2$ with
\begin{equation}\label{4.8.53}
\ell (x^\infty,x^1)+\cdots +\ell (x^{n-1},x^n)<2^{1-m}.
\end{equation}
It thus suffices to show that, whenever \eqref{4.8.53} holds, there is some $U\in \mathcal{U}_m$ with $x^\infty,x^n\in U$.  We prove this by induction on $n$.  For $n=1$, \eqref{4.8.53} implies that $\ell (x_1,x_2),\ell (x_2,x_1)<2^{1-m}$, as was mentioned above in \cref{stp4.8.76.2}, because $X$ is $T_0$, at least one of these is strictly positive---without loss of generality suppose that $\ell (x_1,x_2)>0$.  Then, the fact that $\ell (x_1,x_2)<2^{1-m}$ implies that
\begin{equation}
2^{1-\max \{ o\in \N :x_2\notin \Star _{\mathcal{U}_o}(x_1)\}}<2^{1-m},
\end{equation}
so that
\begin{equation}
m\leq \max \{ o\in \N :x_2\notin \Star _{\mathcal{U}_o}(x_1)\} =\footnote{This equality implicitly uses the fact that $\ell (x_1,x_2)>0$, so that $\{ o\in \N :x_2\in \Star _{\mathcal{U}_o}\}$ is bounded above.}\min \{ o\in \N :x_2\in \Star _{\mathcal{U}_o}(x_1)\} ,
\end{equation}
which implies that $x_2\in \Star _{\mathcal{U}_{m+1}}(x_1)$, which implies that there is some $U\in \mathcal{U}_{m+1}$ such that $x_1,x_2\in U$.  Thus, this does the case for $n=1$.  (Note that in fact we can take $U\in \mathcal{U}_{m+1}$---this will be important later.)

Now assume the result is true for all $k\leq n$.  We wish to prove the result for $n+1$.

We must have that $\ell (x^\infty,x^1)<2^{1-n}$, because otherwise we would have to have that $\ell (x^k,x^{k+1})$ for $k\geq 1$, in which case $x^k$ and $x^{k+1}$ lie in some $U\in \mathcal{U}_o$ for $o$ arbitrarily large.  We can then guarantee that $x^k$ for $k\geq 1$ are obtained in some element of $\mathcal{U}_{m+1}$, and hence, as $x^\infty$ and $x^1$ are obtained in some element of $\mathcal{U}_{m+1}$, everything is contained in some element of $\mathcal{U}_m$.

Thus, without loss of generality assume that $\ell (x^\infty,x^1)<2^{1-n}$.  Then, the inequality \eqref{4.8.53} implies that there is some $k_0$ such that
\begin{equation}
\ell (x^\infty,x^1)+\cdots \ell (x^{k_0-1},x^{k_0})\leq 2^{-m}.
\end{equation}
(This is the same inequality with $m$ one larger).  Similarly, there is some $l_0$ such that
\begin{equation}
\ell (x^{l_0},x^{l_0+1})+\cdots +\ell (x^{n-1},x^n)\leq 2^{-m}.
\end{equation}
It then follows that we must also have that
\begin{equation}
\ell (x^{k_0},x^{k_0+1})+\cdots +\ell (x^{l_0-1},x^{l_0}).
\end{equation}
By the induction hypotheses, we then must have in particular that there are $U_1,U_2,U_3\in \mathcal{U}_{m+1}$ such that $x^\infty,x^{k_0}\in U_1$, $x^{l_0},x^n\in U_2$, and $x^{k_0},x^{l_0}\in U_3$.  Then, there is some $U\in \mathcal{U}_m$ such that $\Star _{\mathcal{U}_{m+1}}(U_3)\subseteq U$.  However, as $U_1,U_2\subseteq \Star _{\mathcal{U}_{m+1}}(U_3)$, this completes the proof.
\end{proof}
\end{savenotes}
\end{thm}
From this, that every uniform space is uniformly-completely-$T_3$ follows relatively easily.
\begin{crl}\label{crl4.4.16}
Let $X$ be a $T_0$ uniform space.  Then, $X$ is uniformly-completely-$T_3$.
\begin{proof}\footnote{Proof adapted from \cite[pg.~8]{Isbell}.}
Let $X$ be a uniform space.  We first show that $X$ is $T_1$.  We know that $X$ is $T_0$, and so by \cref{prp4.6.53} (regular $T_0$ spaces are $T_2$), $X$ is $T_2$, hence $T_1$.

We now show that uniformly-continuous functions on $X$ can separate closed sets from points.  So, let $C\subseteq X$ be closed, and let $x_0\in C^{\comp}$.  As $C$ is closed, there must be some neighborhood of $x_0$ that does not intersect $C$ (otherwise, $x_0$ would be an accumulation point of $C$).  Then, because stars for a basis for the topology (\cref{UniformTopology}), there is a uniform cover $\mathcal{U}$ such that
\begin{equation}\label{4.8.90}
\Star _{\mathcal{U}}(x_0)\subseteq C^{\comp}.
\end{equation}

Now apply the previous theorem for the uniform cover $\mathcal{U}$, so that there is a metric space $\coord{Y,\metric}$ and a uniformly-continuous map $\q :X\rightarrow Y$ such that, if $\diam (S)<1$ for $S\subseteq Y$, it follows that $\q ^{-1}(S)$ is contained in some element of $\mathcal{U}$.  From \eqref{4.8.90}, it follows that $C\cup \{ x_0\}$ is not contained in any element of $\mathcal{U}$, and so
\begin{equation}\label{4.8.97}
\diam (\q (C)\cup \{ \q (x_0)\})\geq 1.
\end{equation}
Define $f:X\rightarrow [0,1]$ by
\begin{equation}
f(x)\coloneqq \footnote{See \eqref{4.8.50} for the definition of $\dist _C$.}\max \{ \dist _C(x) ,1\} .
\end{equation}
This is uniformly-continuous because $\dist _C$ is.  \eqref{4.8.97} implies that $\dist _C(x_0)\geq 1$, and so $f(x_0)=1$.  We showed in \cref{prp4.8.49} that $\dist _C(C)=0$.
\end{proof}
\end{crl}

\subsection{The counter-examples}

We know from the diagram \eqref{4.6.105}, that if we are to `do any better' in terms of separation axioms, we would be able to prove that every uniform space is either perfectly-$T_3$ or completely-$T_4$ (which is equivalent to $T_4$ by Urysohn's Lemma, \cref{UrysohnsLemma}).  Unfortunately, however, there exist counter-examples to both these separation axioms.
\begin{exm}[A uniform space that is not perfectly-$T_3$]\label{exm4.4.20}
The Uncountable Fort Space $X$ of \cref{UncountableFortSpace} will do just fine yet again.  We already know that this space is not perfectly-$T_3$ from \cref{exm4.6.80}.  Thus, all that remains to be done is to equip $X$ with a uniformity that generates the Uncountable Fort Space Topology.

Define
\begin{equation}
\widetilde{\mathcal{B}}\coloneqq \left\{ f^{-1}(\mathcal{B}_{\varepsilon}):f\in \Mor _{\Top}(X,\R ),\ \varepsilon >0\right\} .
\end{equation}
\begin{exr}
Show that $\widetilde{\mathcal{B}}$ is a uniform base for the Uncountable Fort Space Topology.
\end{exr}
\end{exm}
\begin{exm}[A uniform space that is not $T_4$]\label{exm4.4.23}
\begin{savenotes}
We define a topology on $\Mor _{\Set}(\R ,\R )$,\footnote{$\Mor _{\Set}(\R ,\R )$ is our fancy-schmancy notation for the set of all functions from $\R$ to $\R$.} the \emph{topology of pointwise convergence}.  We use \nameref{KelleysConvergenceTheorem}, \cref{KelleysConvergenceTheorem}, to do it.

For $f_\infty \in \Mor _{\Set}(\R ,\R )$ and a let $\lambda \mapsto f_\lambda \in \Mor _{\Set}(\R ,\R )$.  Then,
\begin{textequation}
we declare that the net $\lambda \mapsto f_\lambda$ converges to $f_\infty$ iff the net $\lambda \mapsto f_\lambda (x)$ converges to $f_\infty (x)$ for every $x\in \R$.
\end{textequation}
\begin{exr}
Show that this definition satisfies the axioms of \nameref{KelleysConvergenceTheorem}, and so define a topology on $\Mor _{\Set}(\R ,\R )$.
\end{exr}
\begin{exr}
Show that ${\Mor _{\Set}(\R ,\R ),+}$ is a topological group, where $+$ is defined pointwise:
\begin{equation}
[f_1+f_2](x)\coloneqq f_1(x)+f_2(x).
\end{equation}
\end{exr}
Thus, $\Mor _{\Set}(\R ,\R )$ is canonically a uniform space (and hence completely-$T_3$).

We now shows that $\Mor _{\Set}(\R ,\R )$ is not completely-$T_4$ with respect to this topology.  To do this, we first show that $\Mor _{\Set}(\R ,\Z )\subseteq \Mor _{\Set}(\R ,\R )$ is not completely-$T_4$.\footnote{The proof of this is adapted from \cite[pg.~206]{Munkres}.}

For $m\in \Z$, define
\begin{equation}
P_m\coloneqq \left\{ f\in \Mor _{\Set}(\R ,\Z ):\restr{f}{[f^{-1}(m)]^{\comp}}\text{ is injective.}\right\} ,
\end{equation}
that is, the set of all functions that are injective `modulo sending more than one point to $m$'.  We show that $P_0$ and $P_1$ are closed and disjoint, but cannot be separated by open neighborhoods.

We first check that $P_0$ is closed (the proof that $P_1$ is closed is nearly identical).  So, let $\lambda \mapsto f_\lambda \in P_0$ converge to $f_\infty$.  Suppose that $f_\infty (x_1)=f_\infty (x_2)$ is distinct from $0$.  Then, by our definition of convergence and the fact that our functions are taking values in the integers, it must be the case that $\lambda \mapsto f_\lambda (x_i)$ is eventually equal to $f_\infty (x_i)$ for $i=1,2$.  Then, in particular, we will have that $f_{\lambda _0}(x_1)=f_{\lambda _0}(x_2)$ for $\lambda _0$ sufficiently large, and hence $x_1=x_2$.  Thus, $f_\infty \in P_0$.

We now check that $P_0$ and $P_1$ are disjoint.  If $f$ is injective on the complement of $f^{-1}(0)$ (i.e.~if $f\in P_0$), then this complement must be countable (because the image of $f$ lies in $\Z$).  In particular, there must be at least two elements in $f^{-1}(0)$, and so $f(x_1)=0=f(x_2)$ for $x_1\neq x_2$.  But then $f^{-1}$ cannot be injective on the complement of $f^{-1}(1)$, and so $f\notin P_1$.  Thus, $P_0$ is disjoint from $P_1$.

Let $A\coloneqq \{ \alpha _0,\alpha _1,\alpha _2,\ldots \}$ be a countably-infinite subset of $\R$, and for $S\subseteq A$ a finite, let us define
\begin{equation}
U_{S,f}\coloneqq \{ g\in \Mor _{\Set}(\R ,\Z ):\restr{g}{S}=\restr{f}{S}\} .
\end{equation}
The complement of this is the collection of all functions which differ from $f$ at at least one point of $S$.  Because the functions take their value in $\Z$, however, if you take a net of such functions, the limit (if it has one) must still disagree with $f$ at least one point.  Therefore, $U_{S,f}^{\comp}$ is closed, and hence $U_{S,f}^{\comp}$ is open.  Moreover, as
\begin{equation}
U_{S,f}\cap U_{T,f}=U_{S\cap T,f},
\end{equation}
it follows from \cref{prp4.1.8} that this is a neighborhood base for $f$.  If fact, if we restrict ourselves to only taking $S$ from a given infinite subset of $A$, we still get a neighborhood base.

Now, let $U$ and $V$ be open neighborhoods of $P_0$ and $P_1$ respectively.  Of course, we seek to show that $U$ and $V$ must intersect.  We seek to construct a sequence of functions $f_k\in U$ and a countably-infinite collection of finite subsets $B_k=\{ \alpha _0,\ldots ,\alpha _{m_k}\}$ of $A$ such that $U_{B_k,f_k}\subseteq U$ and
\begin{equation}\label{5.5.29}
f_k(x)\coloneqq \begin{cases} k & \text{if }x=\alpha _k\in B_{k-1} \\ 0 & \text{otherwise}\end{cases}.
\end{equation}
We do so inductively.  (We take $B_0\coloneqq \emptyset$.)

Take $f_1\coloneqq 0$, so that of course $f_1\in P_0$.  Thus, because $\left\{ U_{S,f_1}:S\subseteq A\text{ finite.}\right\}$ is a neighborhood base at $f_1$, there must be some finite subset $B_1\subseteq A$ such that $U_{B_1,f_1}\subseteq U$.  In fact, we can enlarge $B_1$ so that it is of the form $B_1=\{ \alpha _0,\ldots ,\alpha _{m_1}\}$.  Now define $f_2$ according to \eqref{5.5.29}, that is
\begin{equation}
f_2(x)\coloneqq \begin{cases}k & \text{if }x=\alpha _k\in \in B_1 \\ 0 & \text{otherwise}\end{cases}.
\end{equation}
Then, $f_2\in U$, and so there must be some finite set $B_2\subseteq A$ such that $U_{B_2,f_2}\subseteq U$.  By enlarging $B_2$ if necessary, we can guarantee it is of the form $B_2=\{ \alpha _0,\ldots ,\alpha _{m_2}\}$ with $m_2>m_1$.  Then, we may define
\begin{equation}
f_3(x)\coloneqq \begin{cases}2 & \text{if }x\in B_2\setminus B_1 \\ 1 & \text{if }x\in B_1 \\ 0 & \text{otherwise}\end{cases}.
\end{equation}
Then, for the same reason as before, $f_3\in U$, and so there is some finite set $B_3\subseteq A$ (that is without loss of generality of the form $B_3=\{ \alpha _0,\ldots ,\alpha _{m_3}\}$ with $m_3>m_2$) and $U_{B_3,f_3}\subseteq U$.  Continue this process inductively.

Now define $g\in \Mor _{\Set}(\R ,\Z )$ by
\begin{equation}
g(x)\coloneqq \begin{cases}k & \text{if }x=\alpha _k\in A \\ 1 & \text{otherwise}\end{cases}.
\end{equation}
Then, $g\in P_1$, and so there is some finite set $B\subseteq A$ such that $U_{B,g}\subseteq V$.  Let $m$ be sufficiently large so that $B\subseteq B_m$.  Then, $f_m\in U_{B,g}\subseteq V$ and $f_m\in U_{B_m,f_m}\subseteq U$, and so, in particular, lies in $U\cap V$.

This shows that $\Mor _{\Set}(\R ,\Z )$ is not $T_4$, but we still must show that $\Mor _{\Set}(\R ,\Z )$ itself is not $T_4$.  This will follow from the following lemma.
\begin{lma}
Let $X$ be $T_4$ and let $C\subseteq X$ be closed.  Then, $X$ is $T_4$.
\begin{rmk}
Note that this doesn't hold in general.  For example, this example shows that $\prod _\R \R \subseteq \prod _\R [-\infty ,\infty ]$ is not normal even though $\prod _\R [-\infty ,\infty ]$ is (it is compact by \cref{exr4.6.38} and \nameref{TychnoffsTheorem}, \cref{TychnoffsTheorem}, and hence $T_4$ by \cref{prp4.6.83}.
\end{rmk}
\begin{proof}
Let $C_1,C_2\subseteq C$ be disjoint and closed.  Then, by definition of the subspace topology (\cref{SubspaceTopology}), $C_1=C_1'\cap C$ and $C_2=C_2'\cap C$ for $C_1',C_2'\subseteq X$ closed, and so $C_1$ and $C_2$ are themselves closed in $X$ because $C$ is closed.  Thus, because $X$ is $T_4$, $C_1$ and $C_2$ can be separated by neighborhoods in $X$, and hence can be separated by neighborhoods in $C$.
\end{proof}
\end{lma}
\end{savenotes}
\end{exm}

Finally, we are ready to begin discussing cauchyness and completeness in the general context of uniform spaces.

\section{Cauchyness and completeness}

\begin{dfn}[Cauchyness]\label{Cauchyness}
Let $\coord{X,\widetilde{\mathcal{U}}}$ be a uniform space and let $\lambda \mapsto x_\lambda$ be a net.  Then, we say that $\lambda \mapsto x_\lambda$ is \emph{cauchy}\index{Cauchy (in uniform spaces)} iff for every $\mathcal{U}\in \widetilde{\mathcal{U}}$, there is some $U\in \mathcal{U}$ such that $\lambda \mapsto x_\lambda$ is eventually contained in $U$.
\begin{rmk}
You should compare this with our definition of cauchyness in $\R$, \cref{dfn3.3.26}.  As the collection of all $\varepsilon$-balls for all $\varepsilon >0$ forms a uniform base, if we show that we can replace the entire uniformity in the test of cauchyness, then the definition we gave before in \cref{dfn3.3.26} will be a word-for-word special case of this definition.  Indeed, part of the motivation for phrasing the definition in \cref{dfn3.3.26} the way we did was to make the transition to this higher level of generality as transparent as possible.  See the paragraphs that follow \cref{dfn3.3.26} for a few more comments regarding this.
\end{rmk}
\end{dfn}
To check whether a net is cauchy, it suffices to check on just a uniform base for the collection of uniform covers.
\begin{prp}\label{prpB.16}
Let $X$ be a set, let $\widetilde{\mathcal{B}}$ be a uniform base on $X$, and let $\lambda \mapsto x_\lambda$ be a net.  Then, $\lambda \mapsto x_\lambda$ is cauchy iff for every $\mathcal{B}\in \widetilde{\mathcal{B}}$ there is some $B\in \mathcal{B}$ such that $\lambda \mapsto x_\lambda$ is eventually contained in $B$.
\begin{rmk}
With this equivalence, the definition we gave before for cauchyness in $\R$ in \cref{dfn3.3.26} is literally verbatim equivalent to this definition upon replacement of $\widetilde{\mathcal{B}}$ with $\{ \mathcal{B}_\varepsilon :\varepsilon >0\}$ and of $\mathcal{B}$ with $\mathcal{B}_\varepsilon \coloneqq \{ B_\varepsilon (x):x\in \R \}$.
\end{rmk}
\begin{proof}
$(\Rightarrow )$ There is nothing to check.

\blankline
\noindent
$(\Leftarrow )$ Suppose that for every $\mathcal{B}\in \widetilde{\mathcal{B}}$ there is some $B\in \mathcal{B}$ such that $\lambda \mapsto x_\lambda$ is eventually contained in $B$.  Denote the uniformity on $X$ by $\widetilde{\mathcal{U}}$.  Let $\mathcal{U}\in \widetilde{\mathcal{U}}$.  Then, there is some $\mathcal{B}\in \widetilde{\mathcal{B}}$ such that $\mathcal{B}\llcurly \mathcal{U}$.  Thus, there is some $B\in \mathcal{B}$ such that $\lambda \mapsto x_\lambda$ is eventually contained in $B$.  As $\mathcal{B}\llcurly \mathcal{U}$, there is some $U\in \mathcal{U}$ such that $\Star _{\mathcal{B}}(B)\subseteq U$.   In particular, $B\subseteq U$, and so if $\lambda \mapsto x_\lambda$ is eventually contained in $B$, it is certainly eventually contained in $U$.
\end{proof}
\end{prp}
From this, we obtain relatively nice description of what it means to be cauchy in our two large families of examples, namely semimetric spaces and topological groups.
\begin{exr}\label{exr4.5.3x}
Let $\coord{X,\mathcal{D}}$ be a semimetric space and let $\lambda \mapsto x_\lambda \in X$ be a net.  Show that $\lambda \mapsto x_\lambda$ is cauchy iff for every $\metric \in \mathcal{D}$ and for every $\varepsilon >0$, $\lambda \mapsto x_\lambda$ is eventually contained in $B_{\varepsilon}^{\metric}(x)$ for \emph{some} $x\in X$.
\begin{rmk}
In other words, a net in a semimetric space is cauchy iff it is cauchy with respect to each semimetric.
\end{rmk}
\end{exr}
\begin{exr}
Let $G$ be a topological group and let $\lambda \mapsto g_\lambda \in G$ be a net.  Show that $\lambda \mapsto g_\lambda$ is cauchy iff for every open neighborhood $U$ of the identity there is \emph{some} $g\in G$ such that $\lambda \mapsto g_\lambda$ is eventually contained in $gU$.
\end{exr}

Just as continuous functions preserve convergence, so to do uniformly-continuous functions preserve cauchyness.
\begin{exr}\label{exr4.5.3}
Let $f:X\rightarrow Y$ be uniformly-continuous and let $\lambda \mapsto x_\lambda \in X$ be cauchy.  Show that $\lambda \mapsto f(x_\lambda)$ is cauchy.
\end{exr}
Warning:  Continuous functions do \emph{not} necessarily preserve cauchyness.
\begin{exm}[A continuous image of a cauchy net need not be cauchy]
Let $X \coloneqq (-\frac{\uppi}{2},\frac{\uppi}{2})$ and define $m\mapsto \arctan (m)\in X$.  This is certainly cauchy as, for every $\varepsilon >0$, it is eventually contained in $(\frac{\uppi}{2}-\varepsilon ,\frac{\uppi}{2})$.  On the other hand, its image under the continuous function $\tan :X\rightarrow \R$ is not even bounded, much less cauchy.
\end{exm}

Of course, if we know what it means for nets to be cauchy, then we likewise have a notion of what it means for uniform spaces to be \emph{complete}.
\begin{dfn}[Completeness]\label{Completeness}
A uniform space is \emph{complete}\index{Complete (uniform space)} iff every cauchy net converges.
\begin{rmk}
In case there might be some confusion (e.g.~if the topology of the underlying uniform space comes from a totally-ordered set), then we shall say \emph{cauchy-complete} in contrast to \emph{dedekind complete}.
\end{rmk}
\end{dfn}
\begin{exr}
Show that every compact metric space is complete.
\end{exr}

\subsection{Completeness of $\Mor _{\Top}(X,\R )$}

We mentioned back above in \cref{exm4.3.60} that $\Mor _{\Top}(X,\R )$ is complete.  We now prove this.
\begin{thm}\label{thm4.5.6}
Let $X$ be a topological space that has the property that a subset is open iff its intersection with each quasicompact subset $K$ is open in $K$.  $\Mor _{\Top}(X,\R )$ is complete.
\begin{rmk}
In particular, a the limit of a uniformly convergent net of continuous functions is continuous.  This need not be the case if the convergence is just pointwise---see \cref{exm4.5.12x}.
\end{rmk}
\begin{rmk}
This condition on $X$ is called \emph{quasicompactly-generated}\index{Quasicompactly-generated}.  For example, the cocountable topology on $\R$ is \emph{not} quasicompactly-generated---see \cref{exm4.5.12}.
\end{rmk}
\begin{proof}
\Step{Prove the result for $X$ quasicompact}
We first take $X$ to be quasicompact.  In this case, the uniform structure on $\Mor _{\Top}(X,\R )$ is the same as that generated by the single norm $\norm _X$.  Therefore, by \cref{exr4.5.3x} (cauchyness in semimetric spaces), to show that $\Mor _{\Top}(X,\R )$ is complete, it suffices to show that every net that is cauchy with respect to $\norm _X$ converges.  So, suppose that $\lambda \mapsto f_\lambda \in \Mor _{\Top}(X,\R )$ is cauchy.  As
\begin{equation}
\abs{f_{\lambda _1}(x)-f_{\lambda _2}(x)}\leq \norm[f_{\lambda _1}-f_{\lambda _2}],
\end{equation}
it follows that, for each $x\in X$, the net $\lambda \mapsto f_\lambda (x)\in \R$ is cauchy.  As $\R$ is complete, this net has a limit.  Call this limit $f_\infty (x)$.  We need to check two things:  (i) that $x\mapsto f_\infty (x)$ is continuous (so that indeed $f_\infty \in \Mor _{\Top}(X,\R )$, and (ii) that $\lambda \mapsto f_\lambda$ converges to $f_\infty$ in $\Mor _{\Top}(X,\R )$.

We first show that $f_\infty$ is continuous.  Let $\varepsilon >0$.  Let $\lambda _0$ be such that, whenever $\lambda _1,\lambda _2\geq \lambda _0$, it follows that $\norm[f_{\lambda _1}-f_{\lambda _2}]<\varepsilon$.  Let $U$ be an open neighborhood of $x_\infty$ such that $f_{\lambda _0}(U)\subseteq B_{\varepsilon}(f_{\lambda _0}(x_\infty ))$.  Let $x\in U$.  Let $\lambda _1,\lambda _2\geq \lambda _0$ be such that $\abs{f_{\lambda _1}(x)-f_\infty (x)},\abs{f_{\lambda _2}(x_\infty )-f_\infty (x_\infty )}<\varepsilon$.  Then,
\begin{equation}
\begin{split}
\abs{f_\infty (x)-f_\infty (x_\infty )} & \leq \abs{f_\infty (x)-f_{\lambda _1}(x)}+\abs{f_{\lambda _1}(x)-f_{\lambda _0}(x)}+\abs{f_{\lambda _0}(x)-f_{\lambda _0}(x_\infty )} \\
& \qquad +\abs{f_{\lambda _0}(x_\infty )-f_{\lambda _2}(x_\infty )}+\abs{f_{\lambda _2}(x_\infty )-f_\infty (x_\infty )}<5\varepsilon .
\end{split}
\end{equation}
Thus, $f_\infty$ is continuous.

We now check that $\lambda \mapsto f_\lambda$ converges to $f_\infty$ in $\Mor _{\Top}(X,\R )$.  Let $\varepsilon >0$.  Now that we know that $f_\infty$ is continuous, for each $x\in X$, there is some open neighborhood $U_x$ of $x$ such that $f_\infty U_x)\subseteq B_{\varepsilon}(f_\infty (x))$.  Then, 
\begin{equation}
\left\{ U_x:x\in X\right\} 
\end{equation}
is an open cover of $X$.  Therefore, there is a finite subcover, $U_{x_1},\ldots ,U_{x_m}$.  Thus, we may choose $\lambda _0$ such that, whenever $\lambda \geq \lambda _0$, it follows that $\abs{f_\lambda (x_k)-f_\infty (x_k)}<\varepsilon$ for all $1\leq k\leq m$.  Now let $x\in X$ be arbitrary.  Without loss of generality, assume that $x\in U_1$.  Then, whenever $\lambda \geq \lambda _0$, it follows that
\begin{equation}
\abs{f_\lambda (x)-f_\infty (x)}\leq \abs{f_\lambda (x)-f_\lambda (x_1)}+\abs{f_\lambda (x_1)+f_\infty (x_1)}<2\varepsilon .
\end{equation}
Taking the supremum over $x$, we find that
\begin{equation}
\norm[f_\lambda -f_\infty ]<\varepsilon ,
\end{equation}
so that indeed $\lambda \mapsto f_\lambda$ converges to $f_\infty$.

\Step{Prove the result in general}
We now do the general case, in which case $X$ is not necessarily quasicompact.  So, let $\lambda \mapsto f_\lambda \in \Mor _{\Top}(X,\R )$ be cauchy.  Then, by \cref{exr4.5.3} again, we must have that $\lambda \mapsto \restr{f_\lambda}{K}\Mor _{\Top}(K,\R )$ is cauchy for each quasicompact subset $K\subseteq X$.  Therefore, by quasicompact case, $\lambda \mapsto f_\lambda$ converges to its pointwise limit $f_\infty$ on each quasicompact subset.  As $f_\infty$ is continuous on each quasicompact subset, the intersection of the preimage of every open set with every quasicompact subset of $X$ is open, and hence, by hypothesis, is open.  Therefore, $f_\infty$ is continuous.  Furthermore, $\lambda \mapsto f_\lambda$ converges to $f_\infty$ in $\Mor _{\Top}(X,\R )$ because it converges to $f_\infty$ on each quasicompact subset.
\end{proof}
\end{thm}
Now that we've just proven what goes right, we turn to the more interesting side of things---what can go wrong.
\begin{exm}[A pointwise limit of continuous functions that is not continuous]\label{exm4.5.12x}
Let $f_m:[0,1]\rightarrow \R$ be defined by $f_m(x)\coloneqq x^m$.  Then, of course each $f_m$ is continuous.  On the other hand,
\begin{equation}
\lim _mf(x)=\begin{cases}0 & \text{if }x\in [0,1) \\ 1 & \text{if }x=1\end{cases},
\end{equation}
which is clearly not continuous.
\end{exm}
\begin{exm}[A discontinuous function that is continuous on every quasicompact subset]\label{exm4.5.12}
Take $X\coloneqq \R$ and equip it with our good old-buddy, the cocountable topology.  We claim that a subset of $X$ is quasicompact iff it is finite.  Finite subsets are always quasicompact.  On the other hand, take $K\subseteq X$ infinite.  Then, there is in particular a countably-infinite subset $\{ x_0,x_1,x_2,\ldots \} \subseteq K$.  Define $C_m\coloneqq \{ x_k:k\geq m\}$ and $\mathcal{C}\coloneqq \{ C_m:m\in \N \}$.  Then, $\mathcal{C}$ is a collection of closed subsets of $X$.  Furthermore, the intersection of any finitely many of them intersects $K$.  On the other hand, the intersection of all of them is empty.  Therefore, $K$ is not quasicompact.  In particular, $X$ is \emph{not} quasicompactly-generated.

Now let $f:X\rightarrow \R$ be any discontinuous function.  For example, the Dirichlet Function is discontinuous with respect to the cocountable topology (because $\Q ^{\comp}$ is not closed).  On the other hand, finite subsets of $X$ are discrete, and so $f$ restricted to finite subsets must be continuous.
\end{exm}
We can use this trick to find an example of a space for which $\Mor _{\Top}(X,\R )$ is not complete.
\begin{exm}[A topological space for which $\Mor _{\Top}(X,\R )$ is not complete]
Take $X\coloneqq \R$ equipped with the cocountable extension topology.  Recall that this means that the only closed sets are (i) $X$ itself, (ii) countable subsets, and (iii) subsets which are closed in the usual topology of $\R$.

The same proof in the previous example shows that the only quasicompact subsets of $X$ are the finite sets.  Let $f:X\rightarrow \R$ be the Dirichlet function.  As $f^{-1}(-1)=\Q ^{\comp}$ is not closed, $f$ is not continuous.  We construct a net of functions in $\Mor _{\Top}(X,\R )$ converging to $f$ uniformly on each quasicompact (i.e.~each finite) subset of $X$.

Our directed set $\Lambda$ is the collection of all finite subsets of $X$ ordered by inclusion.  For $S\subseteq \Lambda$, let us write $S=\{ x_1,\ldots ,x_m\}$ with $x_k<x_{k+1}$ and define
\begin{equation}
f_S(x)\coloneqq \begin{cases}f(x) & \text{if }x\in S \\ f(x_1) & \text{if }x<x_1 \\ \frac{f(x_{k+1}-f(x_k)}{x_{k+1}-x_k}(x-x_k) & \text{for }x_k<x<x_{k+1} \\ f(x_m) & \text{if }x_m<x\end{cases},
\end{equation}
That it, is is a constant $f(x_1)$ for all $x\leq x_1$, and similarly for $x\geq x_m$.  For $x$ between $x_k$ and $x_{k+1}$, is it a just the line segment going from $f(x_k)$ at $x=x_k$ to $f(x_{k+1})$ at $x=x_{k+1}$.  By construction, this is continuous with respect to the usual topology, and hence continuous with respect to the cocountable extension topology.
\end{exm}
The key result of this subsection is that $\Mor _{\Top}(X,\R )$ is complete for $X$ quasicompactly-generated.  If turns out that the converse of this is \emph{false}.
\begin{exm}[A space for which $\Mor _{\Top}(X,\R )$ is complete yet is not quasicompactly-generated]
Take $X\coloneqq \R$ equipped with the cocountable topology.  We show that every element of $\Mor _{\Top}(X,\R )$ is constant.

Let $f:X\rightarrow \R$ be continuous.  If $f$ is not constant, then $f^{-1}(a)$ is proper for every $a\in \R$, and hence countable.  The image cannot be countable then, because if it were, $\R$ would be a countable union of countable sets, and hence countable.  Furthermore, if the image were contained in a \emph{proper} closed subset of $\R$, then by taking the preimage, we would have a countable set that is the uncountable union of nonempty disjoint sets.  Thus, the image must be uncountable with closure all of $\R$.  Thus, every open interval of $\R$ intersects the image.  In particular, as $\R =\bigcup _{m\in \Z}[m,m+1]$, some interval $[m,m+1]$ must intersect the image at uncountable many points.  But then,
\begin{equation}
\text{countable set}=f^{-1}([m,m+1])=\bigcup _{a\in f(X)\cap [m,m+1]}f^{-1}(a)
\end{equation}
is an uncountable union of disjoint sets--- a contradiction.  Therefore, $f$ is constant.

A cauchy net of constant functions amounts to a cauchy net of real numbers, which converges, and so the original cauchy net of constant functions converges to this constant function.  Therefore, $\Mor _{\Top}(X,\R )$ is complete.
\end{exm}

\subsection{The completion}

And now we show that every uniform space can be \emph{completed}.\footnote{Of course, we already know that $\Q$ is not complete (see \cref{prp3.3.59,prp3.3.68}), and so in general there will most certainly be some `completing' to be done.}
\begin{thm}[Completion]\label{Completion}
\begin{savenotes}
Let $X$ be a uniform space.  Then, there exists a complete uniform space $\Cmp (X)$, the \emph{completion}\index{Cauchy completion}\index{Completion (of a uniform space)} of $X$, such that
\begin{enumerate}
\item $\Cmp (X)$ contains $X$; and
\item if $Y$ is any other complete uniform space which contains $X$, then $Y$ contains $\Cmp (X)$.
\end{enumerate}
Furthermore,
\begin{enumerate}
\item $\Cmp (X)$ is unique up to uniform homeomorphism;\footnote{This of course justifies our use of the notation $\Cmp (X)$.} and
\item $X$ is dense in $\Cmp (X)$.
\end{enumerate}
\begin{rmk}
You will see in the proof that this is the ``cauchy sequence construction'' we mentioned in a footnote right before our proof of existence of the real numbers (\cref{RealNumbers}).  It turns out that, in the case of $\R$, this will gives us the right answer, but that the passage from $\Q$ to $\R$ should really be thought of as the \emph{dedekind completion}, not the \emph{cauchy completion}.
\end{rmk}
\begin{proof}
By replacing $X$ with $\TZero (X)$ if necessary, we may without loss of generality assume that $X$ is $T_0$.

\Step{Define an equivalence relation $\sim$ on cauchy nets}\label{stpB.5.6.1}
Define first as a set
\begin{equation}
X'\coloneqq \left\{ \lambda \mapsto x_\lambda \in X:\lambda \mapsto x_\lambda \text{ is cauchy.}\right\} 
\end{equation}
For $\lambda \mapsto x_\lambda$ and $\mu \mapsto y_\mu$ cauchy nets, define $\lambda \mapsto x_\lambda \sim \mu \mapsto y_\mu$ iff open subsets of $X$ eventually contain $\lambda \mapsto x_\lambda$ iff they eventually contain $\mu \mapsto y_\mu$.

\Step{Show that $\sim$ is an equivalence relation}
That $\lambda \mapsto x_\lambda \sim \lambda \mapsto x_\lambda $ is tautological.  The definition of $\sim$ is $\lambda \mapsto x_\lambda \leftrightarrow \mu \mapsto y_\mu$ symmetric, and so of course $\lambda \mapsto x_\lambda \sim \mu \mapsto y_\mu $ implies $\mu \mapsto y_\mu \sim \lambda \mapsto x_\lambda$.

As for transitivity, suppose that $\lambda \mapsto x_\lambda \sim \mu \mapsto y_\mu $ and $\mu \mapsto y_\mu \sim \nu \mapsto z_\nu$.  We wish to show that $\lambda \mapsto x_\lambda \sim \nu \mapsto z_\nu $.  Let $U\subseteq X$ be open.  We must show that $U$ eventually contains $\lambda \mapsto x_\lambda$ iff it eventualy contains $\nu \mapsto z_\nu$.  By $\lambda \mapsto x_\lambda \leftrightarrow \nu \mapsto z_\nu$, it suffices to show only one of these directions.  So, suppose that $U$ eventually contains $\lambda \mapsto x_\lambda$.  Then, $U$ eventually contains $\mu \mapsto y_\mu$ (because $\lambda \mapsto x_\lambda \sim \mu \mapsto y_\mu$), and so $U$ eventually contains $\nu \mapsto z_\nu$ (because $\mu \mapsto y_\mu \sim \nu \mapsto z_\nu$).

\Step{Define $\Cmp (X)$ as a set}
We define
\begin{equation}
\Cmp (X)\coloneqq X'/\sim .
\end{equation}

\Step{Define a uniformity on $X'$.}
Denote the uniformity on $X$ by $\widetilde{\mathcal{U}}$.  For every uniform cover $\mathcal{U}\in \widetilde{\mathcal{U}}$, we define a corresponding cover $\mathcal{U}'$ of $X'$.  For $\mathcal{U}\in \widetilde{\mathcal{U}}$ and $U\in \mathcal{U}$, define
\begin{equation}\label{B.26}
\begin{split}
U' & \coloneqq \left\{ x\in X':x\text{ is eventually contained in }U\text{.}\right\} \\
\mathcal{U}' & \coloneqq \{ U':U\in \mathcal{U}\} \\
\widetilde{\mathcal{U}}' & \coloneqq \{ \mathcal{U}':\mathcal{U}\in \widetilde{\mathcal{U}}\} .
\end{split}
\end{equation}
We wish to show that $\widetilde{\mathcal{U}}'$ is a uniform base on $X'$.\footnote{The ``B'' is of course to remind us that we don't know this to be a uniformity per se, only a uniform base.}  This will then define a topology and compatible uniformity for each each $\mathcal{U}'$ is a uniform cover by \cref{UniformTopology} and \cref{prp4.3.2} respectively.  To show this, we must show (i) that each element of $\widetilde{\mathcal{U}}'$ is in fact a cover of $X'$ and (ii) that $\widetilde{\mathcal{U}}'$ is downward-directed with respect to star-refinement.

We first check that each $\mathcal{U}'$ is in fact a cover of $X'$.  So, let $\mathcal{U}'\in \widetilde{\mathcal{U}}'$ be arbitrary and let $x\in X'$.  As $x$ is cauchy, there is some $U\in \mathcal{U}$ such that $x$ is eventually contained in $U$, so that $x\in U'$, and hence $\mathcal{U}'$ covers $X'$.

We now check that $\widetilde{\mathcal{U}}'$ is downward-directed with respect to star-refinement.  So, let $\mathcal{U}',\mathcal{V}'\in \widetilde{\mathcal{U}}'$.  Let $\mathcal{W}$ be a common star-refinement of $\mathcal{U}$ and $\mathcal{V}$.  We wish to show that $\mathcal{W}'$ is a common star-refinement of $\mathcal{U}'$ and $V'$.  By $\mathcal{U}\leftrightarrow \mathcal{V}$ symmetry, it suffices to prove that $\mathcal{W}'$ is a star-refinement of $\mathcal{U}'$.  So, let $W_0'\in \mathcal{W}'$.  As $\mathcal{W}$ is a star-refinement of $\mathcal{U}$, it follows that there is some $U_0\in \mathcal{U}$ such that $\Star _{\mathcal{W}}(W_0)\subseteq U_0$.  We wish to show that $\Star _{\mathcal{W}'}(W_0')\subseteq U_0'$.  So, let $W'\in \mathcal{W}'$ intersect $W_0'$.  Then, there is a net that is eventually contained in both $W$ and $W_0$, so that, in particular, $W$ and $W_0$ intersect.  It follows that $W\subseteq U_0$, and hence in turn, that any net eventually contained in $W$ is eventually contained in $U_0$, so that $W'\subseteq U_0'$, and hence $\Star _{\mathcal{W}'}(W_0')\subseteq U_0'$ as desired.

This completes the proof that $\widetilde{\mathcal{U}}'$ is a uniform base on $X'$.

\Step{Show that the quotient map $\q :X'\rightarrow \Cmp (X)$ satisfies $\q ^{-1}(\q (U'))=U'$}\label{Completion.5}
As it is always the case that $U'\subseteq \q ^{-1}(\q (U'))$, it suffices to show that $\q ^{-1}(\q (U'))\subseteq S$.  So, let $x\in \q ^{-1}(\q (U'))$.  Then, $x$ is a cauchy net and there is another cauchy net $y\in U'$ such that $x\sim y$.  That is, open sets eventually contain $x$ iff they eventually contain $y$.  However, as $y\in U'$, by definition of $U'$ (\eqref{B.26}), $y$ is eventually contained in $U$, and so $x$ is eventually contained in $U$, and so $x\in U'$.  Hence, $\q ^{-1}(\q (U'))\subseteq U'$, and we are done.

\Step{Define a uniform base on $\Cmp (X)$}
For every uniform cover $\mathcal{U}\in \widetilde{\mathcal{U}}$, we define a corresponding cover $\Cmp (\mathcal{U})$ of $\Cmp (X)$.  For $\mathcal{U}\in \widetilde{\mathcal{U}}$ and $U\in \mathcal{U}$, define\footnote{As you might have guessed, this is just the quotient uniformity induced from the one on $X'$ written out exlicitly.}
\begin{equation}
\begin{split}
\Cmp (U) & \coloneqq \q (U') \\
\Cmp (\mathcal{U}) & \coloneqq \q (\mathcal{U}')\coloneqq \{ \Cmp (U):U\in \mathcal{U}\} \\
\widetilde{\Cmp (\mathcal{U})} & \coloneqq \q (\widetilde{\mathcal{U}}') \coloneqq \{ \Cmp (\mathcal{U}):\mathcal{U}\in \widetilde{\mathcal{U}}\} .
\end{split}
\end{equation}
We claim that $\widetilde{\Cmp (\mathcal{U})}$ is a uniform base on $\Cmp (X)$.  Certainly each $\Cmp (\mathcal{U})$ is a cover of $\Cmp (X)$ because $\mathcal{U}'$ is a cover of $X'$ and $\q$ is surjective.

It follows from \cref{prp4.2.17x} that $\q$ preserves star-refinement (the purpose of the previous step was to verify the requisite hypotheses of \cref{prp4.2.17x}), and so that $\Cmp (\widetilde{\mathcal{U}})$ is a uniform base follows from the fact that that $\widetilde{\mathcal{U}}'$ was a uniform base on $X'$.

\Step{Show that $\Cmp (X)$ is complete.}
Let $\lambda \mapsto \q (x^\lambda )$ be cauchy in $\Cmp (X)$.  By definition of cauchyness and our uniformity on $\Cmp (X)$, this means that, for every uniform cover $\Cmp (\mathcal{U})\in \Cmp (\widetilde{\mathcal{U}})$, there is some $\Cmp (U)\in \Cmp (\mathcal{U})$ such that $\lambda \mapsto \q (x^\lambda )$ is eventually contained in $\Cmp (U)\coloneqq \q (U')$.  Thus, for each $x^\lambda \in X'$ for $\lambda$ sufficiently large, there is some $y^\lambda \in U'$ with $x^\lambda \sim y^\lambda$.  However, by definition of $U'$, $U$ eventually contains $y^\lambda$, and hence, because $x^\lambda \sim y^\lambda$, eventually contains $x^\lambda$, and so in fact $x^\lambda \in U'$.  Thus, we have a net $\lambda \mapsto x^\lambda \in X'$ that has the property that, for every uniform cover $\mathcal{U}\in \widetilde{\mathcal{U}}$, there is some $U\in \mathcal{U}$ such that $\lambda \mapsto x^\lambda$ is eventually contained in $U'$.

Let us denote the domain of $\lambda \mapsto x^\lambda$ by $\Lambda$.  Now, each $x^\lambda$ is itself a net, and so let us denote its domain by $M^\lambda$.  Define $x^\infty \in X'$ to be the net
\begin{equation}
\Lambda \times \prod _{\lambda \in \Lambda}M^\lambda \ni (\lambda ,\mu )\mapsto (x^\lambda )_{\mu ^\lambda}\in X.
\end{equation}
We first must check that this is cauchy, so that indeed $x^\infty \in X'$.

So, let $\mathcal{U}\in \widetilde{\mathcal{U}}$ be a uniform cover.  Then, there is some $U\in \mathcal{U}$ such that $\lambda \mapsto x^\lambda$ is eventually contained in $U'$.  So, let $\lambda _0$ be such that, whenever $\lambda \geq \lambda _0$, it follows that $x^\lambda \in U'$.  For all such $\lambda$, the net $x^\lambda$ itself must be eventually contained in $U$, so let $\mu _0^\lambda$ be such that, whenever $\mu ^\lambda \geq \mu _0^\lambda$, it follows that $(x^\lambda )_{\mu ^\lambda}$.  (For $\lambda$ not at least $\lambda _0$, $\mu _0^\lambda$ may be anything).  Now, suppose that $(\lambda ,\mu )\geq (\lambda _0,\mu _0)$.  Then,
\begin{equation}
(x^\lambda )_{\mu ^\lambda}\in U
\end{equation}
Thus, $(\lambda ,\mu )\mapsto (x^\lambda )_{\mu ^\lambda}$ is cauchy.

We now show that $\lambda \mapsto \q (x^\lambda )$ converges to $\q (x^\infty )$.  To show this, as stars form neighborhood bases, it suffices to show that $\lambda \mapsto \q (x^\lambda )$ is eventually contained in $\Star _{\Cmp (\mathcal{U})}(\q (x^\infty ))$ for all $\mathcal{U}\in \widetilde{\mathcal{U}}$.  As $\q$ is surjective, the preimage of a star is equal to the star of the preimage, and so it suffices to show that $\lambda \mapsto x^\lambda$ is eventually contained in $\Star _{\mathcal{U}'}(x^\infty )$ for all $\mathcal{U}\in \widetilde{\mathcal{U}}$.  So, let $\mathcal{U}\in \widetilde{\mathcal{U}}$ be a uniform cover.  Then, there is some $U\in \mathcal{U}$ such that $\lambda \mapsto x^\lambda$ is eventually contained in $U'$.  Thus, we will be done if we can show that $x^\infty \in U'$ (so that then $U'\subseteq \Star _{\mathcal{U}'}(x^\infty )$.  To show this, we must show that $x^\infty$ is eventually contained in $U$.  However, the exact same argument that was used above to show that $x^\infty$ was cauchy shows precisely this.  Therefore, $\lambda \mapsto x^\lambda$ converges to $x^\infty$.

\Step{Show that $\Cmp (X)$ contains $X$}
Of course, when we say that $\Cmp (X)$ ``contains'' $X$, what we really means is that there is a subset of $\Cmp (X)$ which is uniformly-homeomorphic to $X$.  So, we define $\iota :X\rightarrow \Cmp (X)$, and show that it is a uniform-homeomorphism onto its image.\footnote{Its image will be equipped with the subspace uniformity, that is, the initial uniformity (see \cref{InitialUniformity}) with respect to the inclusion into $\Cmp (X)$.}

Define $\c :X\rightarrow X'$ by $\c (x)\coloneqq (\lambda \mapsto x_\lambda \coloneqq x)$, that is, $\c$ sends $x$ to the constant net with value $x$.  (Constant nets converge, and hence are in particular cauchy.)  Then we define $\iota \coloneqq \q \circ \c$.  We first show that this is injective.  Suppose that $\iota (x_1)=\iota (x_2)$.  Then, every neighborhood that eventually contains the constant net $x_1$ eventually contains the constant net $x_2$.  In other words, open sets in $X$ contain $x_1$ iff they contain $x_2$, which implies that $x_1=x_2$ because $X$ is $T_0$.

We now check that $\iota$ is uniformly-continuous.  We claim that $\iota ^{-1}(\Cmp (\mathcal{U})=\mathcal{U}$.  However, using result that $\q ^{-1}(\q (U'))=U'$ from \cref{Completion.5}, we have that
\begin{equation}
\iota ^{-1}(\Cmp (\mathcal{U}))=\c ^{-1}\left( \q ^{-1}\left( \q (\mathcal{U}')\right) \right) =\c ^{-1}(\mathcal{U}')=\{ \c ^{-1}(U'):U\in \mathcal{U}\} .
\end{equation}
However, the only constant nets which are eventually contained in $U$ are the elements of $U$ themselves, and so $\c ^{-1}(U')=U$, and so indeed
\begin{equation}
\iota ^{-1}(\Cmp (\mathcal{U}))=\mathcal{U}.
\end{equation}
The subspace uniformity induced on $\iota (X)$ is simply
\begin{equation}\label{4.5.13}
\left\{ \Cmp (\mathcal{U})\wedge \{ \iota (X)\} :\mathcal{U}\in \widetilde{\mathcal{U}}\right\} ,
\end{equation}
that is, a cover of $\iota (X)$ is uniform iff it is a cover that is obtained from a uniform cover of $\Cmp (X)$ by simply restricting that cover to $\iota (X)$.  Because the preimage of a cover with respect its inverse is the same as the image of that uniform cover, to show that the inverse of $\iota :X\rightarrow \iota (X)$ is uniformly-continuous, it suffices to show that $\iota (\mathcal{U})$ is a uniform cover on $\iota (X)$.  By \eqref{4.5.13}, it thus suffices to show that
\begin{equation}\label{4.5.14}
\iota (\mathcal{U})=\Cmp (\mathcal{U})\wedge \{ \iota (X)\} .
\end{equation}
However,
\begin{equation}
\begin{split}
\Cmp (\mathcal{U}) \wedge \{ \iota (X)\} & =\left\{ \q (U')\cap \q (\c (X)):U\in \mathcal{U}\right\}  =\footnote{Careful:  $f(X)\cap f(Y)=f(X\cap Y)$ does not hold in general.  It does, however, hold for $\q$ because $\q$ is surjective and satisfies $\q ^{-1}(\q (U))=U$.}\left\{ \q (U'\cap \c (X)):U\in \mathcal{U}\right\} \\
& =\left\{ \iota (U):U\in \mathcal{U}\right\} ,
\end{split}
\end{equation}
which demonstrates the truth of \eqref{4.5.14}.

\Step{Show that any other complete uniform space that contains $X$ contains $\Cmp (X)$}
Let $Y$ be some other complete uniform space that contains $X$.  Let $x\in X'$ be a cauchy net in $X$ and let $x_\infty$ be its (unique) limit in $Y$.  Define $\kappa :\Cmp (X)\rightarrow Y$ by $\kappa (\q (x)) \coloneqq x_\infty$.
\begin{exr}
Show that $\kappa$ is well-defined and is a uniform-homeomorphism onto its image.
\end{exr}

\Step{Show that $\Cmp (X)$ is unique}
Let $Y$ be another complete uniform space which (i) contains $X$ and (ii) is contained in every other complete uniform space that contains $X$.  From this, we know that $Y$ is contained in $\Cmp (X)$.  On the other hand, we already knew that $\Cmp (X)$ was contained in $Y$.  Therefore, $\Cmp (X)=Y$.

\Step{Show that $X$ is dense in $\Cmp (X)$}
Let $\q (x)\in \Cmp (X)$.
\begin{exr}
Show that $\lambda \mapsto \iota (x_\lambda )$ converges to $\q (x)$ in $\Cmp (X)$.
\end{exr}
It follows from this exercise that $\Cls (\iota (X))=\Cmp (X)$, and so that $X$ is dense in $\Cmp (X)$.
\end{proof}
\end{savenotes}
\end{thm}
One thing that we will frequently want to do is extend a given function to its completion.  If the codomain is complete as well, we can do this, and in fact, we can do it whenever we have a continuous function defined on a dense subspace.
\begin{prp}
\begin{savenotes}
Let $S\subseteq X$ be a dense subset of a topological space, let $Y$ be a complete $T_0$ uniform space, and let $f:S\rightarrow Y$ be uniformly-continuous.  Then, there exists a unique uniformly-continuous map $g:X\rightarrow Y$ such that $\restr{g}{S}=f$.
\begin{rmk}
Mere continuity does not suffice, even in the nicest of cases.  See the counter-example below (\cref{exm4.5.20x}).
\end{rmk}
\begin{proof}
Let $x\in X$.  As $\Cls (S)=X$, there is a net $\lambda \mapsto x_\lambda \in X$ converging to $x$.  Pick any such net\footnote{Our proof of uniqueness will show that this choice ultimately does not matter.}.  By \cref{exr4.5.3},  $\lambda \mapsto f(x_\lambda )$ is cauchy in $Y$, so that we may simply take its limit (because $Y$ is complete and is $T_0$, and hence $T_2$, so that limits are unique---see \cref{prp4.5.37}).  So, let us define $g:X\rightarrow Y$ by
\begin{equation}
g(x)\coloneqq \lim _\lambda f(x_\lambda ).
\end{equation}
\begin{exr}
Show that if $\mu \mapsto y_\mu$ also converges to $x$, then $\lim _\lambda f(x_\lambda )=\lim _\mu f(x_\mu )$.
\end{exr}
This exercises established uniqueness.  Thus, the choice of net does not matter, and so for $x\in S$, we may simply take the constant net $\lambda \mapsto x_\lambda \coloneqq x$, so that $g$ is indeed an extension of $f$.
\begin{exr}
Show that $g$ is uniformly-continuous.
\end{exr}
\end{proof}
\end{savenotes}
\end{prp}
\begin{exm}[A real-valued continuous function on a dense subspace that does \emph{not} extend]\label{exm4.5.20x}
In the notation of the previous proposition, take $S\coloneqq \R$, $X\coloneqq [-\infty ,+\infty]$, $Y\coloneqq \R$, and $f\coloneqq \id _{\R}$.  If this had an extension to all of $X$, then in particular the sequence $m\mapsto x_m\coloneqq m$ would have to converge in $\R$.
\begin{rmk}
In fact, if perhaps you thought you could make use of the fact that continuous functions restricted to quasicompact subsets are uniformly-continuous (\cref{prp4.2.73}) to prove the result in special cases, this even provides a counter-example in which every point of $X$ has a compact neighborhood.
\end{rmk}
\end{exm}
Among other things, the significance of this result is that group operations of topological groups extend uniquely to their completions.

Having shown that uniform spaces always have completions, finally, we may return to an unresolved issue all the way back from \cref{chp1}.
\begin{exm}[A nonzero totally-ordered cauchy-complete field distinct from $\R$]
Recall the field of rational functions with coefficients in the reals, $\R (x)$, from \cref{exm2.3.12}.  Being a totally-ordered field, it is in particular a topological group (\cref{exr4.8.58}) with respect to its underlying commutative group $\coord{\R (x),+,0,-}$, and so has a canonical uniform structure.  Thus, we may complete to form the complete topological field $\Cmp (\R (x))$.
\begin{exr}
Extend the order on $\R (x)$ to $\Cmp (\R (x))$ so that $\Cmp (\R (x))$ is a totally-ordered field containing $\R (x)$.
\end{exr}
By construction then, $\Cmp (\R (x))$ is a nonzero totally-ordered cauchy-complete field.  Not only is it distinct from $\R$ it cannot even embed in $\R$ as, if it did, then so to $\R (x)$ would embed in $\R$ (as it embeds in $\Cmp (\R (x))$), and hence $\R (x)$ would be archimedean---a contradiction of \cref{exm2.3.12}.
\end{exm}

In conclusion:
\begin{footnoteequation}
\begin{textequation}
If a space is quasicompactly-generated, then $\Mor _{\Top}(X,\R )$ is complete.  Furthermore, if $\Mor _{\Top}(X,\R )$ is complete, then $X$ is a topological space.\footnote{Uhm, duh.  The content here is not in the implication, but rather in the counter-example.}  Both of these implications are strict:  the reals with the cocountable topology show that the first implication is strict, and the reals with the cocountable extension topology show that the second implication is strict.
\end{textequation}
\end{footnoteequation}

\subsection{Complete metric spaces}

We present here two important results that are specific to complete metric spaces.

\subsubsection{The Baire Category Theorem}

The Baire Category Theorem is an important result concerning complete \emph{metric} spaces.  It has many important applications, most of which we haven't the time to present.  One important application for us, however, is that it will allow us to finally wrap up the separation axiom counter-examples (see \cref{NiemytzkisTangentDiskTopology}) below.
\begin{thm}[Baire Category Theorem]\label{BaireCategoryTheorem}
Let $X$ be a complete metric space.  Then,
\begin{enumerate}
\item the countable intersection of open dense subsets of $X$ is dense; and
\item the countable union of closed sets with empty interior has empty interior.
\end{enumerate}
\begin{rmk}
These conclusions are equivalent, the equivalence being obtained by taking the complement of the conclusion.  The former is arguably a bit more intuitive to prove, while the latter the form that is probably more frequently used in concrete situations.
\end{rmk}
\begin{rmk}
Warning:  This is \emph{false} in general for complete uniform spaces---see \cref{exm4.5.2x}.
\end{rmk}
\begin{rmk}
The word ``category'' in the name has nothing to do with categories---the terminology it refers to is archaic.
\end{rmk}
\begin{proof}
For $m\in \N$, let $U_m\subseteq X$ be an open dense subset.  We wish to show that
\begin{equation}
U\coloneqq \bigcap _{m\in \N}U_m
\end{equation}
is dense.  The definition of density is that the closure is equal to all of $X$, in other words, that every point of $X$ is an accumulation point, or in other words, that every open subset intersects the dense set.  So, let $V\subseteq X$ be open.  We wish to show that $V$ intersects $U$.

As $U_0$ is dense, $V$ intersects $U_0$, say at $x_0\in U_0\cap V$.  As $U_0\cap V$ is open.  We can fit an $\varepsilon$-ball around $x_0$ inside $U_0\cap V$.  In fact, as $X$ is perfectly-$T_4$, and in particular $T_3$, we can find some $\varepsilon _0>0$ such that
\begin{equation}
\Cls \left( B_{\varepsilon _0}(x_0)\right) \subseteq U_0\cap V.
\end{equation}
In fact, by making $\varepsilon _0$ smaller if necessary, we may without loss of generality assume that $\varepsilon _0<2^{-0}=1$.

Now, because $U_1$ is dense, there is some $x_1\in U_1\cap B_{\varepsilon _0}(x_0)$, and so, just the same as before, there is some $0<\varepsilon <1<2^{-1}$ such that
\begin{equation}
\Cls \left( B_{\varepsilon _1}(x_1)\right) \subseteq U_1\cap B_{\varepsilon _0}(x_0).
\end{equation}
Proceeding inductively, we can find $x_m\in X$ and $0<\varepsilon _m<2^{-m}$ such that
\begin{equation}
\Cls \left( B_{\varepsilon _m}(x_m)\right) \subseteq U_{m-1}\cap B_{\varepsilon _{m-1}}(x_{m-1}).
\end{equation}

We now check that $m\mapsto x_m$ is cauchy.  Let $\varepsilon >0$.  Choose $m\in \N$ such that $\varepsilon _m<\varepsilon$.  Suppose that $n\geq m$.  Then, $x_n\in B_{\varepsilon _n}(x_n)\subseteq B_{\varepsilon _m}(x_m)\subseteq B_{\varepsilon}(x_m)$.  Thus, $m\mapsto x_m$ is eventually contained in some $\varepsilon$ ball, and is hence cauchy.  As $X$ is complete, it converges (to a unique limit) and so we may define
\begin{equation}
x_\infty \coloneqq \lim _mx_m.
\end{equation}

We claim that $x\in U\cap V$.  As explained above, this will complete the proof.  $m\mapsto x_m$ is eventually contained in $\Cls (B_{\varepsilon _0}(x_0))\subseteq V$, and so $x_\infty \in \Cls (B_{\varepsilon _0}(x_0))\subseteq V$.  Similarly, $m\mapsto x_m$ is eventually contained in $\Cls \left( B_{\varepsilon _m}(x_m)\right) \subseteq U_{m_1}$, and so, same as before, $x_\infty \in U_{m_1}$.  Hence, $x_\infty \in U$, and we are done.
\end{proof}
\end{thm}

\begin{exm}[A complete uniform space which is not a baire space]\begin{savenotes}\footnote{A \emph{baire space}\index{Baire space} is a topological space in which the conclusion of the \nameref{BaireCategoryTheorem} holds.  This example was inspired by \href{http://mathoverflow.net/questions/212308/baire-category-theorem-for-complete-uniform-spaces}{priel's answer} on mathoverflow.net.  Thanks to Nate Eldredge for nudging me towards the correct proof.}\label{exm4.5.2x}
Define
\begin{equation}
X\coloneqq \{ f:\Z ^+ \rightarrow [0,1] :f(m)=0\text{ for all but finitely many }m\in \N \text{.}\} 
\end{equation}
We define a uniformity on $X$ as follows.  First of all, for $m\in \Z ^+$, define
\begin{equation}
X_m\coloneqq \underbrace{[0,1] \times \cdots \times [0,1]}_{m}
\end{equation}
equipped with the product uniformity.  Note that $X_m$ embeds in $X$ via $\iota _m:X_m\rightarrow X$ defined by
\begin{equation}
\iota _m(\coord{x_1,\ldots ,x_m})\coloneqq k\mapsto \begin{cases}x_k & \text{if }k\leq m \\ 0 & \text{otherwise}\end{cases},
\end{equation}
that is, $\coord{x_1,\ldots ,x_m}$ is sent to the function from $\Z ^+$ into $[0,1]$ which sends $k$ to $x_k$ for $k\leq m$ and $0$ otherwise.  We then equip $X$ with the final uniformity with respect to the collection $\{ \iota _m:m\in \Z ^+\}$.

We first wish to show that $\Lambda \ni \lambda \mapsto x_\lambda$ is eventually contained in $X_{m_0}$ for some $m_0\in \Z ^+$.  To show this, we proceed by contradiction:  suppose that for every $m\in \N$ and every $\lambda$ there is some $\lambda _{m,\lambda}\geq \lambda$ such that $x_{m,\lambda}\notin X_m$.  Define $\Lambda '\coloneqq \Z ^+\times \Lambda$.  Then, $\Lambda '\ni \coord{m,\lambda}\mapsto x_{\lambda _{m,\lambda}}$ is a subnet of $\lambda \mapsto x_\lambda$, and hence is in turn cauchy.  Thus, for every $\varepsilon _1,\varepsilon _2,\varepsilon _3,\ldots >0$, there is some $\coord{m_0,\lambda _0}$ such that, whenever $\coord{m,\lambda},\coord{n,\mu}\geq \coord{m_0,\lambda _0}$, it follows that
\begin{equation}
\abs{x_{\lambda _{m,\lambda}}(k)-x_{\lambda _{n,\mu}}(k)}<\varepsilon _k
\end{equation}
for all $k\in \Z ^+$.  As $x_{\lambda _{m_0,\lambda _0}}(k)=0$ for all $k$ sufficiently large, we find that
\begin{equation}
x_{\lambda _{n,\mu}}(k)<\varepsilon _k
\end{equation}
for all $k$ sufficiently large, say for $k\geq k_0$, and all $\coord{n,\mu}\geq \coord{m_0,\lambda _0}$.  Take $n\coloneqq \max \{ k_0,m_0\}$.  As $x_{\lambda _{n,\mu}}\notin X_n$, there is some $l>n\geq k_0$ such that $x_{\lambda _{n,\mu}}(l)\neq 0$.  Then,
\begin{equation}
0<x_{\lambda _{n,\mu}}(l)<\varepsilon _l.
\end{equation}
As $\varepsilon _l$ is arbitrary, this is a contradiction.  Thus, there is some $m_0\in \Z ^+$ such that $\lambda \mapsto x_\lambda$ is eventually contained in $X_{m_0}$.

However, $X_{m_0}$, being a finite product of compact metric spaces, is a compact metric space, and hence complete, so that $\lambda \mapsto x_\lambda$ converges in $X_{m_0}$, and hence in $X$.  Therefore, $X$ is complete.

We now wish to check that $X$ is not a baire space.  From the definition, we have that
\begin{equation}
X=\bigcup _{m\in \Z ^+}X_m.
\end{equation}
\begin{exr}
Show that $X_m$ is closed in $X$.
\end{exr}
Thus, to show that $X$ is not a baire space, it suffices to show that each $X_m$ has empty interior.  So, let $x\in X_m$.  We show that every open neighborhood around $x$ contains an element of $X_{m+1}$ that is not contained in $X_m$.  Let us write
\begin{equation}
x=\coord{x_1,x_2,\ldots ,x_m,0,0,0,\ldots}
\end{equation}
Then, for every neighborhood $U$ of $x$, there will be some $\varepsilon _0>0$ sufficiently small so that
\begin{equation}
x=\coord{x_1,x_2,\ldots ,x_m,\varepsilon _0,0,0,\ldots}\in U,
\end{equation}
so that $x$ is not in the interior of $X_m$, so that each $X_m$ has empty interior.
\end{savenotes}
\end{exm}

Finally, we are able to tie-up our one loose end with the separation axioms.
\begin{exm}[A space that is perfectly-$T_3$ but not $T_4$]\footnote{This example comes from \cite[pg.~100]{Steen}.}\label{NiemytzkisTangentDiskTopology}
Define $X\coloneqq \R \times \R _0^+$, the upper-half plane, and define a base for a topology on $X$ by
\begin{equation}
\begin{split}
\mathcal{B} & \coloneqq \left\{ B_\varepsilon (\coord{x,y})\subseteq \R \times \R ^+:\varepsilon >0,\ \coord{x,y}\in \R \times \R ^+\right\} \\
& \qquad \cup \left\{ B_\varepsilon (\coord{x,\varepsilon})\cup \{ \coord{x,0}\} :x\in \R ,\ \varepsilon >0\right\} .
\end{split}
\end{equation}
That is, we take all $\varepsilon$-balls contained in the (strict) upper half-plane together with all $\varepsilon$-balls in the upper half-plane which are `tangent' to the $x$-axis (together with the point of tangency).

We first check that $X$ is perfectly-$T_3$.  So, let $C\subseteq X$ be closed and let $\coord{x_0,y_0}\in C^{\comp}$.  The subspace topology of $\R ^+\times \R ^+\subseteq X$ is just the usual topology, which, being a metric space, is perfectly-$T_4$.  Thus, we only need to check the case where $y_0=0$.  Let $\varepsilon >0$ be such that $B_{\varepsilon}(\coord{x_0,\varepsilon})$ is disjoint from $C$.  Then define $f:X\rightarrow [0,1]$ by
\begin{equation}
f(\coord{x,y})\coloneqq \begin{cases}0 & \text{if }\coord{x,y}=\coord{x_0,0} \\ 1 & \text{if }\coord{x,y}\notin B_{\varepsilon}(\coord{x_0,\varepsilon})\cup \{ \coord{x_0,0}\} \\ \frac{(x-x_0)^2+y^2}{2\varepsilon y}\end{cases}.
\end{equation}
For $0<a<1$, $f^{-1}([0,a)]$ is $B_{\varepsilon a}(\coord{x,\varepsilon a})\cup \{ \coord{x,0}\}$, hence open; $f^{-1}((a,1])$ is $X\setminus \Cls (B_{\varepsilon a}(\coord{x,\varepsilon a}))$, hence open; and so $f^{-1}(a,b)=f^{-1}((a,1])\cap f^{-1}([0,b))$ is open.

We now check that $X$ is not $T_4$.  To do this, we show that $\Q ,\R \setminus \Q \subseteq \R \times \R _0^+$ (as subsets of the $x$-axis) are closed and cannot be separated by neighborhoods.\footnote{We write $\Q$ and $\R \setminus \Q$ instead of $\Q \times \{ 0\}$ and $(\R \setminus \Q )\times \{ 0\}$.}  In fact, we check that every subset $S\subseteq \R \times \{ 0\}$ is closed.  To show that, we show that $S^{\comp}$ is open.  For $\coord{x,y}\in S^{\comp}$, if $y>0$, then certainly we can put an $\varepsilon$-ball around it that does not intersect the $x$-axis, and hence does not intersection $S$.  On the other hand, for $\coord{x,0}\in S^{\comp}$, $B_{\varepsilon}(\coord{x,\varepsilon})\cup \{ \coord{x,0}\}$ intersects the $x$-axis only at $\coord{x,0}$, and so does not intersect $S$.  Therefore, $S^{\comp}$ is open, and so $S$ is closed.

We now check that $\Q ^{\comp}$ and $(\R \setminus \Q )^{\comp}$ cannot be separated by neighborhoods.\footnote{Note that we cannot write $\Q ^{\comp}$ to denote the irrationals as usual because, in this context, $\Q ^{\comp}$ means $(\R \times \R _0^+)\setminus \Q$.}  Let $U$ be an open neighborhood o $\R \setminus \Q$.  We show that there is some point $q_0\in \Q$ every neighborhood of which intersects $U$.

For $x\in \R \setminus \Q$, let $\varepsilon _x>0$ be such that
\begin{equation}
\R \setminus \Q \ni x\in B_{\varepsilon _x}(\coord{x,\varepsilon _x})\cup \{ x\} \subseteq U.
\end{equation}
For $m\in \Z ^+$, define
\begin{equation}
S_m\coloneqq \{ x\in \R \setminus \Q :\varepsilon _x>\tfrac{1}{m}\} .
\end{equation}
Then,
\begin{equation}
\R =\bigcup _{m\in \Z ^+}S_m\cup \bigcup _{x\in \Q}\{ x\},
\end{equation}
and so
\begin{equation}
\R =\bigcup _{m\in \Z ^+}\Cls _{\R}(S_m)\cup \bigcup _{x\in \Q}\{ x\} .\footnote{The subscript $\R$ here is to indicate that we are taking the closure with respect to the usual topology on $\R$.}
\end{equation}
As $\R$ is a complete metric space, by the \nameref{BaireCategoryTheorem}, there must be some $m_0\in \Z ^+$ such that $\Cls _{\R}(S_m)$ does \emph{not} have empty interior.  So, let $(a,b)\subseteq \Cls _{\R}(S_m)$ and let $\Q \ni q_0\in (a,b)$.  Then, for every $\varepsilon >0$, $(q_0-\varepsilon ,q_0+\varepsilon )$ intersects $S_m$, say at $x_\varepsilon$.  But then, for all $\varepsilon$ sufficiently small,
\begin{equation}
\coord{x_\varepsilon ,\varepsilon}\in B_{\varepsilon}(\coord{q_0,\varepsilon})\cap B_{\tfrac{1}{m}}(\coord{x_\varepsilon ,\tfrac{1}{m}})\subseteq B_{\varepsilon}(\coord{q_0,\varepsilon})\cap U.
\end{equation}
Thus, every neighborhood of $\coord{q_0,0}\in \Q$ intersects $U$.
\begin{rmk}
This is \emph{Niemytzki's Tangent Disk Topology}\index{Niemytzki's Tangent Disk Topology}.
\end{rmk}
\end{exm}

\subsubsection{Banach Fixed-Point Theorem}

We finish the chapter with an application of the \nameref{BaireCategoryTheorem} that will prove useful to us later when we study differentiation.
\begin{thm}[Banach Fixed-Point Theorem]\index{Banach Fixed-Point Theorem}\label{BanachFixedPointTheorem}
Let $\coord{X,\metric}$ be a metric space and let $f:X\rightarrow X$ be such that
\begin{equation}\label{4.5.70}
\metric[f(x_1)][f(x_2)]\leq M\metric[x_1][x_2]
\end{equation}
for some $0\leq 1<M$.  Then, there is \emph{at most one} point $x_0\in X$ such that $f(x_0)=x_0$.  If $X$ is nonempty and complete, then there is \emph{exactly one} $x_0\in X$ such that $f(x_0)=x_0$.
\begin{rmk}
Such an $x_0$ is called a \emph{fixed-point}, hence the name of the theorem.  Thus, the theorem tells us that (i) fixed-points, if they exist, have to be unique; and (ii) in the (nonempty) complete case, there has to be some fixed-point (and hence exactly one fixed-point).
\end{rmk}
\begin{rmk}
\eqref{4.5.70} is just the the statement that $f$ is lipschitz-continuous \emph{for a constant} $M<1$.  Such maps are called \emph{contraction-mapppings}, hence the alternative name for this theorem, the \emph{Contraction-Mapping Theorem}\index{Contraction-Mapping Theorem}.
\end{rmk}
\begin{proof}
Let $x_1,x_2\in X$ be two fixed points of $f$.  Then,
\begin{equation}
\metric[x_1][x_2]=\metric[f(x_1)][f(x_2)]\leq M\metric[x_1][x_2],
\end{equation}
and hence, if $\metric[x_1][x_2]\neq 0$, we would have (because $M<1$)
\begin{equation}
\metric[x_1][x_2]<\metric[x_1][x_2]:
\end{equation}
a contradiction.  Therefore, $\metric[x_1][x_2]=0$, and hence $x_1=x_2$.

Now take $X$ to be nonempty and complete.  We construct an actual fixed point of $f$.  Let $x_0\in X$ (there is such a point because $X$ is nonempty).  For $m\in \Z ^+$, define
\begin{equation}
x_m\coloneqq f(x_{m-1}).
\end{equation}
We wish to show that the sequence $m\mapsto x_m$ is cauchy.  If we can do so, then its limit $x_\infty$ must exist, and so by taking the limit of the previous equation, we would find that $x_\infty =f(x_\infty )$ ($f$ is lipschitz-continuous, hence uniformly-continuous, hence continuous).  Thus, it suffices to show that $m\mapsto x_m$ is cauchy

To see this, we first notice that\footnote{We're using $y$s instead of $x$s because those symbols are already used-up.}
\begin{equation}
\metric[y_1][y_2]\leq \metric[y_1][f(y_1)]+\metric[f(y_1)][f(y_2)]+\metric[f(y_2)][y_2]\leq \metric[y_1][f(y_1)]+M\metric[y_1][y_2]+\metric[f(y_2)][y_2],
\end{equation}
and so
\begin{equation}\label{4.5.75}
\metric[y_1][y_2]\leq \frac{1}{1-M}\left( \metric[y_1][f(y_1)]+\metric[f(y_2)][y_2]\right) 
\end{equation}
Also note that
\begin{equation}
\metric[f^m(y_1)][f^m(y_2)]\leq M^m\metric[y_1][y_2],
\end{equation}
which follows of course from just applying \eqref{4.5.70} inductively.  Taking $y_1\coloneqq f^m(x_0)$ and $y_2\coloneqq f^n(x_0)$ in \eqref{4.5.75}, we find
\begin{equation}
\begin{split}
\metric[x_m][x_n] & \leq \frac{1}{1-M}\left( \metric[x_m][x_{m+1}]+\metric[x_{n+1}][x_n]\right) \leq \frac{1}{1-M}\left( M^m\metric[x_0][f(x_0)]+M^n\metric[f(x_0)][x_0]\right) \\
& =\frac{M^m+M^n}{1-M}\metric[x_0][f(x_0)].
\end{split}
\end{equation}
Because $M<1$, we can make $\frac{M^m+M^n}{1-M}$ arbitrarily small by taking $m$ and $n$ sufficiently large.\footnote{If you are not comfortable with this amount of detail, I suggest you fill in the gaps.  You will want to get to the point where you feel comfortable just asserting cauchyness after obtaining an inequality like this.}  Hence, this sequence is cauchy, and we are done.
\end{proof}
\end{thm}
\begin{exm}[A contraction mapping with no fixed-point]
Take $X\coloneqq \R ^+$ and define $f(x)\coloneqq \frac{1}{2}x$.  The fixed-point `should' be $0$, but $0\notin \R ^+$.  You can turn this intuition into a proof that this map indeed has no fixed point, and we recommend you try to do so.
\end{exm}